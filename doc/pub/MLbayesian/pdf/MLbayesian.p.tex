%%
%% Automatically generated file from DocOnce source
%% (https://github.com/hplgit/doconce/)
%%
%%
% #ifdef PTEX2TEX_EXPLANATION
%%
%% The file follows the ptex2tex extended LaTeX format, see
%% ptex2tex: http://code.google.com/p/ptex2tex/
%%
%% Run
%%      ptex2tex myfile
%% or
%%      doconce ptex2tex myfile
%%
%% to turn myfile.p.tex into an ordinary LaTeX file myfile.tex.
%% (The ptex2tex program: http://code.google.com/p/ptex2tex)
%% Many preprocess options can be added to ptex2tex or doconce ptex2tex
%%
%%      ptex2tex -DMINTED myfile
%%      doconce ptex2tex myfile envir=minted
%%
%% ptex2tex will typeset code environments according to a global or local
%% .ptex2tex.cfg configure file. doconce ptex2tex will typeset code
%% according to options on the command line (just type doconce ptex2tex to
%% see examples). If doconce ptex2tex has envir=minted, it enables the
%% minted style without needing -DMINTED.
% #endif

% #define PREAMBLE

% #ifdef PREAMBLE
%-------------------- begin preamble ----------------------

\documentclass[%
oneside,                 % oneside: electronic viewing, twoside: printing
final,                   % draft: marks overfull hboxes, figures with paths
10pt]{article}

\listfiles               %  print all files needed to compile this document

\usepackage{relsize,makeidx,color,setspace,amsmath,amsfonts,amssymb}
\usepackage[table]{xcolor}
\usepackage{bm,ltablex,microtype}

\usepackage[pdftex]{graphicx}

\usepackage[T1]{fontenc}
%\usepackage[latin1]{inputenc}
\usepackage{ucs}
\usepackage[utf8x]{inputenc}

\usepackage{lmodern}         % Latin Modern fonts derived from Computer Modern

% Hyperlinks in PDF:
\definecolor{linkcolor}{rgb}{0,0,0.4}
\usepackage{hyperref}
\hypersetup{
    breaklinks=true,
    colorlinks=true,
    linkcolor=linkcolor,
    urlcolor=linkcolor,
    citecolor=black,
    filecolor=black,
    %filecolor=blue,
    pdfmenubar=true,
    pdftoolbar=true,
    bookmarksdepth=3   % Uncomment (and tweak) for PDF bookmarks with more levels than the TOC
    }
%\hyperbaseurl{}   % hyperlinks are relative to this root

\setcounter{tocdepth}{2}  % levels in table of contents

% prevent orhpans and widows
\clubpenalty = 10000
\widowpenalty = 10000

% --- end of standard preamble for documents ---


% insert custom LaTeX commands...

\raggedbottom
\makeindex
\usepackage[totoc]{idxlayout}   % for index in the toc
\usepackage[nottoc]{tocbibind}  % for references/bibliography in the toc

%-------------------- end preamble ----------------------

\begin{document}

% matching end for #ifdef PREAMBLE
% #endif

\newcommand{\exercisesection}[1]{\subsection*{#1}}


% ------------------- main content ----------------------



% ----------------- title -------------------------

\thispagestyle{empty}

\begin{center}
{\LARGE\bf
\begin{spacing}{1.25}
Machine Learning: Bayesian Machine Learning
\end{spacing}
}
\end{center}

% ----------------- author(s) -------------------------

\begin{center}
{\bf Master of Science thesis project${}^{}$} \\ [0mm]
\end{center}

\begin{center}
% List of all institutions:
\end{center}
    
% ----------------- end author(s) -------------------------

% --- begin date ---
\begin{center}
Nov 28, 2019
\end{center}
% --- end date ---

\vspace{1cm}


\subsection{Machine Learning and the Quantum Many-body Problem}

Solving quantum mechanical problems for atoms,  molecules, materials, and
interfaces is of fundamental importance to a large number of
disciplines including physics, chemistry, and materials science. Since
the early development of quantum mechanics, it has been noted, by
Dirac among others, that \emph{...approximate, practical methods of applying quantum mechanics should be developed, which can lead to an explanation of the main features of complex atomic systems without too much computation}. 

Historically, this has meant invoking
approximate forms of the underlying interactions (mean field, tight
binding, etc.) or relying on phenomenological fits to a limited number
of either experimental observations or theoretical results (e.g., force fields). 

The development of feature-based models is not
new in the scientific literature. Indeed, prior even to the acceptance
of the atomic hypothesis, van der Waals argued for an equation of
state based on two physical features. Machine learning (i.e.,
fitting parameters within a model) has been used in physics and
chemistry since the dawn of the computer age. The term machine
learning is new; the approach is not.

More recently, high-level ab initio calculations have been used to
train artificial neural networks to \href{{http://www.sciencedirect.com/science/article/pii/S0927025615007806?via%3Dihub}}{fit high-dimensional interaction
models}  and to make informed predictions about \href{{https://www.nature.com/articles/srep40827}}{material properties}. 

Machine learning can also be used to accelerate or bypass some of the
heavy machinery of the ab initio method itself. In the work of \href{{https://journals.aps.org/prl/abstract/10.1103/PhysRevLett.108.253002}}{Snyder et al},  the authors
replaced the kinetic energy functional within density-functional
theory with a machine-learned one, 
\emph{learned} the mappings from potential to electron density and from
charge density to kinetic energy, respectively.

\subsection{Thesis Projects}



\paragraph{Specific tasks and milestones.}
The specific task here is to implement and study the recently developed
deep learning algorithms for solving quantum mechanical many-particle
problems. The results can  be easily compared with exisiting standard
many-particle codes developed by former students at the Computational
Physics group. These codes will serve as useful comparisons in order
to gauge the appropriateness of recent Machine Learning approaches to
quantum mechanical problems.   The aim here is to use recurrent neural networks to study quantum mechanical many-body methods like the family of similarity renormalization group methods.
This method is a rewrite of many-body equations in terms of coupled ordinary differential equations, see chapter 10 of \href{{https://www.springer.com/gp/book/9783319533353}}{Lecture Notes in Physics vol. 936}. 



The projects can easily be split into several parts and form the basis for collaborations among several students. The milestones are as follows
\begin{enumerate}
\item Spring 2020: 

\item Fall 2020: 

\item Spring 2021: 
\end{enumerate}

\noindent
The thesis is expected to be handed in May/June  2021.

\paragraph{References.}
Highly relevant articles for possible thesis projects are:


% ------------------- end of main content ---------------

% #ifdef PREAMBLE
\end{document}
% #endif

