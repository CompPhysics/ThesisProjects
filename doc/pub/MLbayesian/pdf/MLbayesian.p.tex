%%
%% Automatically generated file from DocOnce source
%% (https://github.com/hplgit/doconce/)
%%
%%
% #ifdef PTEX2TEX_EXPLANATION
%%
%% The file follows the ptex2tex extended LaTeX format, see
%% ptex2tex: http://code.google.com/p/ptex2tex/
%%
%% Run
%%      ptex2tex myfile
%% or
%%      doconce ptex2tex myfile
%%
%% to turn myfile.p.tex into an ordinary LaTeX file myfile.tex.
%% (The ptex2tex program: http://code.google.com/p/ptex2tex)
%% Many preprocess options can be added to ptex2tex or doconce ptex2tex
%%
%%      ptex2tex -DMINTED myfile
%%      doconce ptex2tex myfile envir=minted
%%
%% ptex2tex will typeset code environments according to a global or local
%% .ptex2tex.cfg configure file. doconce ptex2tex will typeset code
%% according to options on the command line (just type doconce ptex2tex to
%% see examples). If doconce ptex2tex has envir=minted, it enables the
%% minted style without needing -DMINTED.
% #endif

% #define PREAMBLE

% #ifdef PREAMBLE
%-------------------- begin preamble ----------------------

\documentclass[%
oneside,                 % oneside: electronic viewing, twoside: printing
final,                   % draft: marks overfull hboxes, figures with paths
10pt]{article}

\listfiles               %  print all files needed to compile this document

\usepackage{relsize,makeidx,color,setspace,amsmath,amsfonts,amssymb}
\usepackage[table]{xcolor}
\usepackage{bm,ltablex,microtype}

\usepackage[pdftex]{graphicx}

\usepackage[T1]{fontenc}
%\usepackage[latin1]{inputenc}
\usepackage{ucs}
\usepackage[utf8x]{inputenc}

\usepackage{lmodern}         % Latin Modern fonts derived from Computer Modern

% Hyperlinks in PDF:
\definecolor{linkcolor}{rgb}{0,0,0.4}
\usepackage{hyperref}
\hypersetup{
    breaklinks=true,
    colorlinks=true,
    linkcolor=linkcolor,
    urlcolor=linkcolor,
    citecolor=black,
    filecolor=black,
    %filecolor=blue,
    pdfmenubar=true,
    pdftoolbar=true,
    bookmarksdepth=3   % Uncomment (and tweak) for PDF bookmarks with more levels than the TOC
    }
%\hyperbaseurl{}   % hyperlinks are relative to this root

\setcounter{tocdepth}{2}  % levels in table of contents

% prevent orhpans and widows
\clubpenalty = 10000
\widowpenalty = 10000

% --- end of standard preamble for documents ---


% insert custom LaTeX commands...

\raggedbottom
\makeindex
\usepackage[totoc]{idxlayout}   % for index in the toc
\usepackage[nottoc]{tocbibind}  % for references/bibliography in the toc

%-------------------- end preamble ----------------------

\begin{document}

% matching end for #ifdef PREAMBLE
% #endif

\newcommand{\exercisesection}[1]{\subsection*{#1}}


% ------------------- main content ----------------------



% ----------------- title -------------------------

\thispagestyle{empty}

\begin{center}
{\LARGE\bf
\begin{spacing}{1.25}
Bayesian Machine Learning
\end{spacing}
}
\end{center}

% ----------------- author(s) -------------------------

\begin{center}
{\bf Master of Science thesis project${}^{}$} \\ [0mm]
\end{center}

\begin{center}
% List of all institutions:
\end{center}
    
% ----------------- end author(s) -------------------------

% --- begin date ---
\begin{center}
Nov 30, 2019
\end{center}
% --- end date ---

\vspace{1cm}


\subsection{Bayesian Machine Learning, Level Densities and Probability}

The level density $\rho(E)$ as function of energy $E$ plays a central in many
physics applications, ranging from the modeling of nuclear
astrophysics reactions central to the synthesis of the elements to the
classification and understanding of phases and phase transition in for example condensed matter
physics.

In statistical physics it defines the thermodynamical potential in the micro-canonical ensemble and thereby the entropy.
For a finite isolated many-body system (for example an atomic nucleus), the correct
thermodynamical ensemble is the microcanonical one. In this ensemble,
the nuclear level density, the density of eigenstates of a nucleus
at a given excitation energy, is the important quantity that may be
used to describe thermodynamic properties of nuclei, such as the 
nuclear entropy, specific heat, and temperature. 
Bethe first described the level density using a 
non-interacting fermi gas model
for the nucleons.
Modifications to this picture, such as the 
back-shifted fermi gas which includes pair and
shell effects not present in Bethe's original formulation, are in 
wide use.  
The level density
$\rho$ defines the partition function for 
the microcanonical ensemble and the entropy through the well-known
relation $S(E)=k_Bln(\rho(E))$.
Here $k_B$ is Boltzmann's constant and $E$ is the energy.  In 
the microcanonical ensemble, we could then extract expectation values 
for thermodynamical quantities like temperature $T$, or the 
heat capacity $C$. The temperature in the microcanonical ensemble 
is defined as 
\begin{equation}
      \langle T\rangle=\left(\frac{dS(E)}{dE}\right)^{-1}.
      \label{eq:temp}
\end{equation}
It is a function of the excitation energy, which is 
the relevant variable of interest in the microcanonical ensemble. 
However, since the extracted level density is given only at discrete 
energies, the calculation of expectation values 
like $T$, involving derivatives of the partition function, is not
reliable unless a strong smoothing over energies is 
performed. Another possibility
is to perform a transformation to the canonical ensemble.
The partition function for the 
canonical ensemble is related to that of the 
microcanonical ensemble through a Laplace transform
\begin{equation}
     Z(\beta)=\int_0^{\infty}dE\rho(E)\exp{(-\beta E)}.
     \label{eq:zcan}
\end{equation}
Here we have defined $\beta=1/k_BT$.
Since we will  deal with discrete energies, the 
Laplace transform of Eq.\ (\ref{eq:zcan}) takes the form
\begin{equation}
         Z(\beta)=\sum_E \Delta E\rho(E)\exp{(-\beta E)},
         \label{eq:zactual}
\end{equation}
where $\Delta E$ is the energy bin used.
With $Z$ we can evaluate the 
entropy in the canonical ensemble using the definition of the 
free energy 
\begin{equation}
     F(T)= -k_B T \ln Z(T)=\langle E(T)\rangle - TS(T).
\end{equation}
Note that the temperature $T$ is now the variable of 
interest and the energy $E$ is given by the expectation 
value $\langle E\rangle$ as a function of $T$. Similarly, 
the entropy $S$ is also a function of $T$.
For finite systems, fluctuations in various 
expectation values can be large.
In nuclear and solid state physics, thermal properties have 
mainly been studied in the canonical and grand-canonical ensemble. 
In order to obtain the level density, the inverse transformation 
\begin{equation}
      \rho(E) =\frac{1}{2\pi i}\int_{-i\infty}^{i\infty}
 d\beta Z(\beta) \exp{(\beta E)},
      \label{eq:zbigcan}
\end{equation}
is normally used. Compared with Eq.\ (\ref{eq:zcan}), this 
transformation is rather difficult to perform since 
the integrand $\exp{\left(\beta E+ \ln Z(\beta)\right)}$ is a 
rapidly varying function of the integration parameter. In order to obtain 
the density of states, approximations like the saddle-point 
method, viz., an expansion of the exponent in the integrand 
to second order around the equilibrium point and subsequent integration, 
have been used widely.

For the ideal Fermi gas, this gives the following density of states
\begin{equation}
      \rho_{\rm ideal}(E)=\frac{\exp{(2\sqrt{aE})}}{E\sqrt{48}},
      \label{eq:omegaideal}
\end{equation}
where $a$ in nuclear physics is a factor 
typically of the order $a=A/8$ with dimension 
MeV$^{-1}$, $A$ being the mass number of a given nucleus. 



To obtain an experimental level density is a rather hard task.
Ideally, we would like an experiment to provide us with the level 
density as a function of excitation energy and thereby 
the \emph{full} partition function for the microcanonical ensemble. 
It is only rather recently that 
experimentalists have been able to develop methods for extracting level densities at low spin from
measured $\gamma$-spectra. 
These measurements were performed at the Oslo 
Cyclotron Laboratory.

The Oslo cyclotron group has developed a method to 
extract nuclear level densities at low spin from 
measured $\gamma$-ray spectra.
The main advantage of utilizing $\gamma$-rays as a probe for 
level density is that the nuclear system is likely thermalized prior 
to the $\gamma$-emission. In addition, the method 
allows for the simultaneous extraction of level density 
and $\gamma$-strength function over a wide energy region. 




With the level density we can in turn define a probability
distribution function (PDF) in say for example the canonical
ensemble. Alternatively, if we have the PDF we can find the level density.
Having a PDF allows us also to quantify in a rigorous way statistical
confidence intervals, statistical errors and other statistical quantities with far reaching
consequences for our understanding of a specific physics problem. 
In experiments we do however normally not have the above
quantities. This means that we need to translate experimental results
via some theoretical modeling into suitable quantities that can be
used to define either a PDF or the density of states.

A typical situation which occurs in for example nuclear reaction
experiments performed at the cyclotron of the University of Oslo, is
that one can extract the number of counts as function of the
excitation energy $E_x$ of a given nucleus and the resulting gmamma
energy $E_{\gamma}$ from Compton scattering. This quantity, labelled
$N(E_x,E_{\gamma})$ can in turn be used to define either a PDF or the density of state.


In this project we will use Bayesian statistics and Bayesian machine
learning to extract first the PDF based on the above experimental data
in order to define a posterior distribution $P(E_x\vert E_{\gamma})$,
that is the likelihood of the state of energy $E_x$ given a certain
$\gamma$-energy.  This quantity will in turn be used to identify a
density of states. A short note on Bayes' rule and some other elements of statistics are included at the end here.


\subsection{Thesis Projects}

The aim of this thesis project is employ Bayesian machine learning to
define a PDF, either from experiment or from theoretical simulations.
Eventually, based on the PDF, can attempt to define the level density
$\rho(E)$, or the other way around. The first step is to use an
already available model for extracting the level density from exact
diagonalization. This model, a so-called simplified pairing model is
described in detail by Dean and Hjorth-Jensen in references below, as
well as in chapter 8 of Lecture Notes in Physics, volume \textbf{936}.  These
data will then be used to define a posterior distribution based on a
Bayesian machine learning approach.






\paragraph{Specific tasks and milestones.}
The projects can easily be split into several parts and form the basis for collaborations among several students. The milestones are as follows
\begin{enumerate}
\item Spring 2020: Use the simple pairing model to generate training data on the density of states from numerical diagonalization (existing code) and develop a Bayesian Neural Network code and algorithm to extract a PDF. This PDF expresses the likelihood for finding the system at a given energy.

\item Fall 2020: Based on the experience with the theoretical model, the next step is to use experimental data from the Oslo cyclotron (see discussions above) in order to extract $P(E_x\vert E_{\gamma})$ using Bayesian Machine Learning.

\item Spring 2021: Analysis of results and determination of level density. Finalize thesis. 
\end{enumerate}

\noindent
The thesis is expected to be handed in May/June  2021.


\paragraph{References.}
\begin{enumerate}
\item \textbf{Pairing in nuclear systems: from neutron stars to finite nuclei}, DJ Dean, M Hjorth-Jensen, \href{{http://journals.aps.org/rmp/abstract/10.1103/RevModPhys.75.607}}{Reviews of Modern Physics 75, 607  (2003)}.

\item \href{{http://www.springer.com/us/book/9783319533353}}{Morten Hjorth-Jensen, M.P. Lombardo and U. van Kolck}, Volume \textbf{936}, (2017), see chapter 8
\end{enumerate}

\noindent
\paragraph{Appendix: Brief note on Bayesian Statistics.}
The aim is  to assess hypotheses by calculating their probabilities $p(H_i | \ldots)$ conditional on known and/or presumed information using the rules of probability theory.
Bayes' theorem is based on the standard  Probability Theory Axioms:

\begin{enumerate}
\item Product (AND) rule : $p(A, B | I) = p(A|I) p(B|A, I) = p(B|I)p(A|B,I)$. Should read $p(A,B|I)$ as the probability for propositions $A$ AND $B$ being true given that $I$ is true.

\item Sum (OR) rule: $p(A + B | I) = p(A | I) + p(B | I) - p(A, B | I)$. $p(A+B|I)$ is the probability that proposition $A$ OR $B$ is true given that $I$ is true.

\item Normalization: $p(A|I) + p(\bar{A}|I) = 1$. $\bar{A}$ denotes the proposition that $A$ is false.
\end{enumerate}

\noindent
Bayes' theorem follows directly from the product rule
$$
p(A|B,I) = \frac{p(B|A,I) p(A|I)}{p(B|I)}.
$$
The importance of this property to data analysis becomes apparent if we replace $A$ and $B$ by hypothesis($H$) and data($D$):
\begin{align}
p(H|D,I) &= \frac{p(D|H,I) p(H|I)}{p(D|I)}.
\label{eq:bayes}
\end{align}
The power of Bayes’ theorem lies in the fact that it relates the quantity of interest, the probability that the hypothesis is true given the data, to the term we have a better chance of being able to assign, the probability that we would have observed the measured data if the hypothesis was true.

The various terms in Bayes’ theorem have formal names. 
\begin{itemize}
\item The quantity on the far right, $p(H|I)$, is called the \emph{prior} probability; it represents our state of knowledge (or ignorance) about the truth of the hypothesis before we have analysed the current data. 

\item This is modified by the experimental measurements through $p(D|H,I)$, the \emph{likelihood} function, 

\item The denominator $p(D|I)$ is called the \emph{evidence}. It does not depend on the hypothesis and can be regarded as a normalization constant.

\item Together, these yield the \emph{posterior} probability, $p(H|D, I )$, representing our state of knowledge about the truth of the hypothesis in the light of the data. 
\end{itemize}

\noindent

% ------------------- end of main content ---------------

% #ifdef PREAMBLE
\end{document}
% #endif

