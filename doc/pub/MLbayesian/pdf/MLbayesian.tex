%%
%% Automatically generated file from DocOnce source
%% (https://github.com/hplgit/doconce/)
%%
%%


%-------------------- begin preamble ----------------------

\documentclass[%
oneside,                 % oneside: electronic viewing, twoside: printing
final,                   % draft: marks overfull hboxes, figures with paths
10pt]{article}

\listfiles               %  print all files needed to compile this document

\usepackage{relsize,makeidx,color,setspace,amsmath,amsfonts,amssymb}
\usepackage[table]{xcolor}
\usepackage{bm,ltablex,microtype}

\usepackage[pdftex]{graphicx}

\usepackage[T1]{fontenc}
%\usepackage[latin1]{inputenc}
\usepackage{ucs}
\usepackage[utf8x]{inputenc}

\usepackage{lmodern}         % Latin Modern fonts derived from Computer Modern

% Hyperlinks in PDF:
\definecolor{linkcolor}{rgb}{0,0,0.4}
\usepackage{hyperref}
\hypersetup{
    breaklinks=true,
    colorlinks=true,
    linkcolor=linkcolor,
    urlcolor=linkcolor,
    citecolor=black,
    filecolor=black,
    %filecolor=blue,
    pdfmenubar=true,
    pdftoolbar=true,
    bookmarksdepth=3   % Uncomment (and tweak) for PDF bookmarks with more levels than the TOC
    }
%\hyperbaseurl{}   % hyperlinks are relative to this root

\setcounter{tocdepth}{2}  % levels in table of contents

% prevent orhpans and widows
\clubpenalty = 10000
\widowpenalty = 10000

% --- end of standard preamble for documents ---


% insert custom LaTeX commands...

\raggedbottom
\makeindex
\usepackage[totoc]{idxlayout}   % for index in the toc
\usepackage[nottoc]{tocbibind}  % for references/bibliography in the toc

%-------------------- end preamble ----------------------

\begin{document}

% matching end for #ifdef PREAMBLE

\newcommand{\exercisesection}[1]{\subsection*{#1}}


% ------------------- main content ----------------------



% ----------------- title -------------------------

\thispagestyle{empty}

\begin{center}
{\LARGE\bf
\begin{spacing}{1.25}
Machine Learning: Bayesian Machine Learning
\end{spacing}
}
\end{center}

% ----------------- author(s) -------------------------

\begin{center}
{\bf Master of Science thesis project${}^{}$} \\ [0mm]
\end{center}

\begin{center}
% List of all institutions:
\end{center}
    
% ----------------- end author(s) -------------------------

% --- begin date ---
\begin{center}
Nov 28, 2019
\end{center}
% --- end date ---

\vspace{1cm}


\subsection*{Bayesian Machine Learning, Level Densities and Probability}

The level density $\rho(E)$ as function of energy $E$ plays a central in many
physics applications, ranging from the modeling of nuclear
astrophysics reactions central to the synthesis of the elements to the
classification and understanding of phases in condensed matter
physics.

In statistical physics it defines the thermodynamical potential in the micro-canonical ensembler and thereby the entropy as
\[
S(E) = -k_B \ln{(\rho(E))},
\]
and the partition function $Z(\beta)$ (with $\beta = 1/k_BT$)  as
\[
Z(\beta) = \int dE \exp{(-\beta E)}\rho(E),
\]
and the expectation values of various moments of the energy
as
\[
\mathbb{E}^n(\beta) = \frac{\int dE E^n\exp{(-\beta E)}\rho(E)}{Z(\beta)}. 
\]
We can rewrite this equation as
\[
\mathbb{E}^n(\beta) = \int dE E^n P(E\vert\beta), 
\]

where $P(E\vert\beta)$ is the likelihood of being in a state with
energy $E$ with temperature $\beta$. The probability is defined as

\[
P(E\vert\beta) = \frac{\exp{(-\beta E)}\rho(E)}{Z(\beta)}. 
\]

With the density of states we can in turn define a probability distribution function (PDF) in say for example the canonical ensemble. Alternatively, if we have the PDF we can find the denisty of states.




\subsection*{Thesis Projects}

The aim of this thesis project is employ Bayesian machine learning to
define a PDF, either from experiment or from theoretical simulations.
Eventually, based on the PDF, can attempt to define a a level density
$\rho(E)$, or the other way around. The first step is to use an
already available model for extracting the level density from exact
diagonalization. These data will then be used to define a posterior
distribution based on a Bayesian machine learning approach.






\paragraph{Specific tasks and milestones.}
The projects can easily be split into several parts and form the basis for collaborations among several students. The milestones are as follows
\begin{enumerate}
\item Spring 2020: 

\item Fall 2020: 

\item Spring 2021: 
\end{enumerate}

\noindent
The thesis is expected to be handed in May/June  2021.

\paragraph{References.}
Highly relevant articles for possible thesis projects are:


% ------------------- end of main content ---------------

\end{document}

