%%
%% Automatically generated file from DocOnce source
%% (https://github.com/hplgit/doconce/)
%%
%%


%-------------------- begin preamble ----------------------

\documentclass[%
oneside,                 % oneside: electronic viewing, twoside: printing
final,                   % draft: marks overfull hboxes, figures with paths
10pt]{article}

\listfiles               %  print all files needed to compile this document

\usepackage{relsize,makeidx,color,setspace,amsmath,amsfonts,amssymb}
\usepackage[table]{xcolor}
\usepackage{bm,ltablex,microtype}

\usepackage[pdftex]{graphicx}

\usepackage[T1]{fontenc}
%\usepackage[latin1]{inputenc}
\usepackage{ucs}
\usepackage[utf8x]{inputenc}

\usepackage{lmodern}         % Latin Modern fonts derived from Computer Modern

% Hyperlinks in PDF:
\definecolor{linkcolor}{rgb}{0,0,0.4}
\usepackage{hyperref}
\hypersetup{
    breaklinks=true,
    colorlinks=true,
    linkcolor=linkcolor,
    urlcolor=linkcolor,
    citecolor=black,
    filecolor=black,
    %filecolor=blue,
    pdfmenubar=true,
    pdftoolbar=true,
    bookmarksdepth=3   % Uncomment (and tweak) for PDF bookmarks with more levels than the TOC
    }
%\hyperbaseurl{}   % hyperlinks are relative to this root

\setcounter{tocdepth}{2}  % levels in table of contents

% prevent orhpans and widows
\clubpenalty = 10000
\widowpenalty = 10000

% --- end of standard preamble for documents ---


% insert custom LaTeX commands...

\raggedbottom
\makeindex
\usepackage[totoc]{idxlayout}   % for index in the toc
\usepackage[nottoc]{tocbibind}  % for references/bibliography in the toc

%-------------------- end preamble ----------------------

\begin{document}

% matching end for #ifdef PREAMBLE

\newcommand{\exercisesection}[1]{\subsection*{#1}}


% ------------------- main content ----------------------



% ----------------- title -------------------------

\thispagestyle{empty}

\begin{center}
{\LARGE\bf
\begin{spacing}{1.25}
High-Performance Computing 
\end{spacing}
}
\end{center}

% ----------------- author(s) -------------------------

\begin{center}
{\bf Master of Science thesis project${}^{}$} \\ [0mm]
\end{center}

\begin{center}
% List of all institutions:
\end{center}
    
% ----------------- end author(s) -------------------------

% --- begin date ---
\begin{center}
Jan 22, 2018
\end{center}
% --- end date ---

\vspace{1cm}


\subsection*{Quantum Monte Carlo using CPUs and GPUs}

The aim of this thesis is to develop  diffusion Monte Carlo (DMC) and a Variational
Monte Carlo (VMC) programs written in OpenCL and/or CUDA, in order to test the potential
of Graphical processing units (GPUs) in Monte Carlo studies of both bosonic and fermionic systems.

The thesis entails the development of  VMC and DMC codes for bosons and fermions,
tailored to run on large supercomputing cluster with thousands of cores.
These codes will then be rewritten in OpenCL and/or CUDA and tested on different GPUs,
in order to see if a considerable speedup can be gained.
Such a speedup, if we at least an order of magnitude is gained compared 
with existing in C++/Fortran codes that run on present  supercomputers, will have trememdous consenquences for ab initio studies of quantum mechanical systems.  The prices of the GPUs are of the order of few thousands of NOK, compared
with large supercomputers which can cost billions of NOK. 

The physics cases will deal with Bose-Einstein condensation and studies
of the structure of atoms like Neon or Argon or electrons confined to move in harmonic oscilaltor like traps. 
In this way, both bosonic and fermionic systems will be tested.


The aim is to use these methods and evaluate 
the ground state properties of
a trapped, hard sphere Bose gas over a wide range of densities
using VMC and DMC  methods with several 
trial wave functions, as well as testing fermionic systems. These wave functions are used 
to study the sensitivity of condensate and 
non-condensate properties to the hard sphere radius and the number 
of particles.
The traps we will use are both a spherical symmetric (S) harmonic and 
an elliptical (E) harmonic trap in three dimensions given by 
 \begin{equation}
V_{ext}({\bf r}) = 
\Bigg\{
\begin{array}{ll}
        \frac{1}{2}m\omega_{ho}^2r^2 & (S)\\
\strut
	\frac{1}{2}m[\omega_{ho}^2(x^2+y^2) + \omega_z^2z^2] & (E)
\label{trap_eqn}
\end{array}
\end{equation}
with 
\begin{equation}
    H = \sum_i^N \left(
        \frac{-\hbar^2}{2m}
        { \bigtriangledown }_{i}^2 +
        V_{ext}({\bf{r}}_i)\right)  +
        \sum_{i<j}^{N} V_{int}({\bf{r}}_i,{\bf{r}}_j),
\end{equation}
as the two-body Hamiltonian of the system.
Here $\omega_{ho}^2$ defines the trap potential strength.  In the case of the
elliptical trap, $V_{ext}(x,y,z)$, $\omega_{ho}=\omega_{\perp}$ is the trap frequency
in the perpendicular or $xy$ plane and $\omega_z$ the frequency in the $z$
direction.
The mean square vibrational amplitude of a single boson at $T=0K$ in the 
trap (\ref{trap_eqn}) is $<x^2>=(\hbar/2m\omega_{ho})$ so that 
$a_{ho} \equiv (\hbar/m\omega_{ho})^{\frac{1}{2}}$ defines the 
characteristic length
of the trap.  The ratio of the frequencies is denoted 
$\lambda=\omega_z/\omega_{\perp}$ leading to a ratio of the
trap lengths
$(a_{\perp}/a_z)=(\omega_z/\omega_{\perp})^{\frac{1}{2}} = \sqrt{\lambda}$.

We represent the inter-boson interaction by a pairwise, hard core potential
\begin{equation}
V_{int}(r) =  \Bigg\{
\begin{array}{ll}
        \infty & r \leq a\\
        0 & {r} > a
\end{array}
\end{equation}
where $a$ is the hard core diameter of the bosons.  Clearly, $V_{int}(r)$
is zero if the bosons are separated by a distance $r$ greater than $a$ but
infinite if they attempt to come within a distance $r \leq a$.

Similarly, for the fermionic systems we will study systems of electrons confined to move in two- and three-dimensional harmonic oscillator traps, so-called quantum dot systems.


\paragraph{Progress plan and milestones.}
The aims and progress plan of this thesis are as follows

\begin{itemize}
\item Develop a program that implements serially both variational and diffusion Monte Carlo methods for bosons and fermions.

\item Implement interacting and non-interacting systems

\item Implement various potentials

\item Parallelise the program with MPI and OpenCL to use all the available cores

\item Implement the blocking and the bootstrap methods  for error calculation
\end{itemize}

\noindent
The more explicit timeline is 

\begin{enumerate}
 \item Fall 2017: Develop a VMC and DMC code for boson in OpenCL and CUDA

 \item Fall 2017: Develop a VMC and DMC  code for fermions in OpenCL.

 \item Spring 2018: Extensive benchmarks of codes and writeup of master thesis.
\end{enumerate}

\noindent
The thesis is expected to be finalized May/June 2018.
















% ------------------- end of main content ---------------

\end{document}

