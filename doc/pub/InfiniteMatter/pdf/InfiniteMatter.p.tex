%%
%% Automatically generated file from DocOnce source
%% (https://github.com/hplgit/doconce/)
%%
%%
% #ifdef PTEX2TEX_EXPLANATION
%%
%% The file follows the ptex2tex extended LaTeX format, see
%% ptex2tex: http://code.google.com/p/ptex2tex/
%%
%% Run
%%      ptex2tex myfile
%% or
%%      doconce ptex2tex myfile
%%
%% to turn myfile.p.tex into an ordinary LaTeX file myfile.tex.
%% (The ptex2tex program: http://code.google.com/p/ptex2tex)
%% Many preprocess options can be added to ptex2tex or doconce ptex2tex
%%
%%      ptex2tex -DMINTED myfile
%%      doconce ptex2tex myfile envir=minted
%%
%% ptex2tex will typeset code environments according to a global or local
%% .ptex2tex.cfg configure file. doconce ptex2tex will typeset code
%% according to options on the command line (just type doconce ptex2tex to
%% see examples). If doconce ptex2tex has envir=minted, it enables the
%% minted style without needing -DMINTED.
% #endif

% #define PREAMBLE

% #ifdef PREAMBLE
%-------------------- begin preamble ----------------------

\documentclass[%
oneside,                 % oneside: electronic viewing, twoside: printing
final,                   % draft: marks overfull hboxes, figures with paths
10pt]{article}

\listfiles               %  print all files needed to compile this document

\usepackage{relsize,makeidx,color,setspace,amsmath,amsfonts,amssymb}
\usepackage[table]{xcolor}
\usepackage{bm,ltablex,microtype}

\usepackage[pdftex]{graphicx}

\usepackage[T1]{fontenc}
%\usepackage[latin1]{inputenc}
\usepackage{ucs}
\usepackage[utf8x]{inputenc}

\usepackage{lmodern}         % Latin Modern fonts derived from Computer Modern

% Hyperlinks in PDF:
\definecolor{linkcolor}{rgb}{0,0,0.4}
\usepackage{hyperref}
\hypersetup{
    breaklinks=true,
    colorlinks=true,
    linkcolor=linkcolor,
    urlcolor=linkcolor,
    citecolor=black,
    filecolor=black,
    %filecolor=blue,
    pdfmenubar=true,
    pdftoolbar=true,
    bookmarksdepth=3   % Uncomment (and tweak) for PDF bookmarks with more levels than the TOC
    }
%\hyperbaseurl{}   % hyperlinks are relative to this root

\setcounter{tocdepth}{2}  % levels in table of contents

% prevent orhpans and widows
\clubpenalty = 10000
\widowpenalty = 10000

% --- end of standard preamble for documents ---


% insert custom LaTeX commands...

\raggedbottom
\makeindex
\usepackage[totoc]{idxlayout}   % for index in the toc
\usepackage[nottoc]{tocbibind}  % for references/bibliography in the toc

%-------------------- end preamble ----------------------

\begin{document}

% matching end for #ifdef PREAMBLE
% #endif

\newcommand{\exercisesection}[1]{\subsection*{#1}}


% ------------------- main content ----------------------



% ----------------- title -------------------------

\thispagestyle{empty}

\begin{center}
{\LARGE\bf
\begin{spacing}{1.25}
Studies of Infinite Matter Systems, from the Homogeneous Electron Gas to Dense Matter 
\end{spacing}
}
\end{center}

% ----------------- author(s) -------------------------

\begin{center}
{\bf Master of Science thesis project${}^{}$} \\ [0mm]
\end{center}

\begin{center}
% List of all institutions:
\end{center}
    
% ----------------- end author(s) -------------------------

% --- begin date ---
\begin{center}
Jan 22, 2018
\end{center}
% --- end date ---

\vspace{1cm}


\subsection{Infinite Matter studies}

Bulk nucleonic matter is interesting for several reasons. The equation of state (EoS) of
neutron matter determines properties of supernova
explosions, and of neutron
stars and it links the latter to neutron radii in atomic
nuclei  and the symmetry
energy . Similarly, the compressibility of
nuclear matter is probed in giant dipole
excitations, and the symmetry energy of nuclear
matter is related to the difference between proton and neutron radii
in atomic nuclei. The
saturation point of nuclear matter determines bulk properties of
atomic nuclei, and is therefore an important constraint for nuclear
energy-density functionals and mass models.



\paragraph{Many-body approaches to infinite matter.}
For an infinite homogeneous system  like nuclear matter or the electron gas, 
the one-particle wave functions are given by plane wave functions
normalized to a volume $\Omega$ for a quadratic box with length
$L$. The limit $L\rightarrow \infty$ is to be taken after we
  have computed various expectation values. In our case we will
  however always deal with a fixed number of particles and finite size
  effects become important. 
\[
\psi_{\mathbf{k}m_s}(\mathbf{r})= \frac{1}{\sqrt{\Omega}}\exp{(i\mathbf{kr})}\xi_{m_s}
\]
where $\mathbf{k}$ is the wave number and $\xi_{m_s}$ is a spin function
for either spin up or down
\[ 
\xi_{m_{s}=+1/2}=\left(\begin{array}{c} 1
  \\ 0 \end{array}\right) \hspace{0.5cm}
\xi_{m_{s}=-1/2}=\left(\begin{array}{c} 0 \\ 1 \end{array}\right).\]
The single-particle energies for the three-dimensional electron gas
are
\[    
\varepsilon_{n_{x}, n_{y}, n_{z}} = \frac{\hbar^{2}}{2m}\left( \frac{2\pi }{L}\right)^{2}(n_{x}^{2}+n_{y}^{2}+n_{z}^{2}),
\]
resulting in the magic numbers $2$, $14$, $38$, $54$, etc.  

In general terms, our Hamiltonian contains at most a two-body
interactions. In second quantization, we can write our Hamiltonian as
\begin{equation}
\hat{H}= \sum_{pq}\langle p | \hat{h}_0 | q \rangle a_p^{\dagger} a_q + \frac{1}{4}\sum_{pqrs}\langle pq |v| r s \rangle a_p^{\dagger} a_q^{\dagger} a_s a_r,
\label{eq:ourHamiltonian}
\end{equation} 
where the operator $\hat{h}_0$ denotes the single-particle
Hamiltonian, and the elements $\langle pq|v|rs\rangle$ are the
anti-symmetrized Coulomb interaction matrix elements.  Normal-ordering
with respect to a reference state $|\Phi_0\rangle$ yields
\begin{equation}
\hat{H}=E_0 + \sum_{pq}f_{pq}\lbrace a_p^{\dagger} a_q\rbrace + \frac{1}{4}\sum_{pqrs}\langle pq |v| r s \rangle \lbrace a_p^{\dagger} a_q^{\dagger} a_s a_r \rbrace,
\label{eq:normalorder}
\end{equation}
where $E_0=\langle\Phi_0| \hat{H}| \Phi_0\rangle$ is the reference energy
and we have introduced the so-called  Fock matrix element defined as
\begin{equation}
f_{pq} = \langle p|\hat{h}_0| q \rangle + \sum\limits_{i} \langle pi |v| qi\rangle.
\label{eq:fockelement}
\end{equation}
The curly brackets in Eq.~(\ref{eq:normalorder}) indicate that the
creation and annihilation operators are normal ordered.

The unperturbed part
of the Hamiltonian is defined as the sum over all the single-particle
operators $\hat{h}_0$, resulting in
\[
\hat{H}_{0}=\sum_i\langle i|\hat{h}_0|i \rangle= {\displaystyle\sum_{\mathbf{k}_{i}m_{s_i}}}
\frac{\hbar^{2}k_i^{2}}{2m}a_{\mathbf{k}_{i}m_{s_i}}^{\dagger}
a_{\mathbf{k}_{i}m_{s_i}}.
\]
We will throughout suppress, unless explicitely needed, all references
to the explicit quantum numbers $\mathbf{k}_{i}m_{s_i}$. The summation
index $i$ runs over all single-hole states up to the Fermi level.

The general anti-symmetrized two-body interaction matrix element 
\[
\langle pq |v| r s \rangle = \langle
\mathbf{k}_{p}m_{s_{p}}\mathbf{k}_{q}m_{s_{q}}|v|\mathbf{k}_{r}m_{s_{r}}\mathbf{k}_{s}m_{s_{s}}\rangle,
\]
is given by the following expression
  \begin{align}
    & \langle
    \mathbf{k}_{p}m_{s_{p}}\mathbf{k}_{q}m_{s_{q}}|v|\mathbf{k}_{r}m_{s_{r}}\mathbf{k}_{s}m_{s_{s}}\rangle
    \nonumber \\ =&
    \frac{e^{2}}{\Omega}\delta_{\mathbf{k}_{p}+\mathbf{k}_{q},
      \mathbf{k}_{r}+\mathbf{k}_{s}}\left\{
    \delta_{m_{s_{p}}m_{s_{r}}}\delta_{m_{s_{q}}m_{s_{s}}}(1-\delta_{\mathbf{k}_{p}\mathbf{k}_{r}})\frac{4\pi
    }{\mu^{2} + (\mathbf{k}_{r}-\mathbf{k}_{p})^{2}} \right. \nonumber
    \\ & \left. -
    \delta_{m_{s_{p}}m_{s_{s}}}\delta_{m_{s_{q}}m_{s_{r}}}(1-\delta_{\mathbf{k}_{p}\mathbf{k}_{s}})\frac{4\pi
    }{\mu^{2} + (\mathbf{k}_{s}-\mathbf{k}_{p})^{2}} \right\},
    \nonumber
  \end{align}
for the three-dimensional electron gas.  The energy per electron computed with
the reference Slater determinant can then be written as 
(using hereafter only atomic units, meaning that $\hbar = m = e = 1$)
\[
E_0/N=\frac{1}{2}\left[\frac{2.21}{r_s^2}-\frac{0.916}{r_s}\right],
\]
for the three-dimensional electron gas.  This will serve (with the
addition of a Yukawa term) as the first benchmark in setting up a
program for doing Coupled-cluster theories. 
The electron gas provides a very useful benchmark at the Hartree-Fock level since it provides
an analytical solution for the Hartree-Fock energy single-particle energy and the total  energy per particle. 

In addition to the above studies of the electron gas, it
is important to study properly the boundary conditions as well. 

The next
step is to perform a coupled-cluster calculations at the
coupled-cluster with doubles excitations, the so-called CCD approach,
for the electron gas.  The next step
is to include triples correlations and perform full doubles and
triples calculations (CCDT) for neutron matter using a simplified model for the nuclear force, the so-called Minnesota potential. 
This allows for a benchmark of codes to infinite nuclear matter. The next step is to include realistic models for the nuclear forces and 
study the EoS for pure neutron matter, asymmetric
nuclear matter (for different proton fractions) and symmetric nuclear
matter. With this one can study $\beta$-stable neutron star matter and
extract important information about the symmetry energy in infinite
matter and the composition of a neutron star. 
The resulting effective interactions at the two-body level can in turn, if time allows, be included in the study of neutrino emissivities in dense matter. The processes of most interest are the so-called 
modified  URCA prosesses
\begin{equation}
    n+n\rightarrow p+n +e +\overline{\nu}_e,
    \hspace{0.5cm} p+n+e \rightarrow
    n+n+\nu_e .
    \label{eq:ind_neutr}
\end{equation}
These reactions correspond to the processes for 
$\beta$-decay and electron capture with a bystanding neutron.
The calculation of neutrino spectra has important consequences for our basic understanding on how neutron stars cool, the synthesis of the elements and neutrino oscillations in dense matter. 





% ------------------- end of main content ---------------

% #ifdef PREAMBLE
\end{document}
% #endif

