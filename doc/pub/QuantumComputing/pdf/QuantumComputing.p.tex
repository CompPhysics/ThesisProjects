%%
%% Automatically generated file from DocOnce source
%% (https://github.com/hplgit/doconce/)
%%
%%
% #ifdef PTEX2TEX_EXPLANATION
%%
%% The file follows the ptex2tex extended LaTeX format, see
%% ptex2tex: http://code.google.com/p/ptex2tex/
%%
%% Run
%%      ptex2tex myfile
%% or
%%      doconce ptex2tex myfile
%%
%% to turn myfile.p.tex into an ordinary LaTeX file myfile.tex.
%% (The ptex2tex program: http://code.google.com/p/ptex2tex)
%% Many preprocess options can be added to ptex2tex or doconce ptex2tex
%%
%%      ptex2tex -DMINTED myfile
%%      doconce ptex2tex myfile envir=minted
%%
%% ptex2tex will typeset code environments according to a global or local
%% .ptex2tex.cfg configure file. doconce ptex2tex will typeset code
%% according to options on the command line (just type doconce ptex2tex to
%% see examples). If doconce ptex2tex has envir=minted, it enables the
%% minted style without needing -DMINTED.
% #endif

% #define PREAMBLE

% #ifdef PREAMBLE
%-------------------- begin preamble ----------------------

\documentclass[%
oneside,                 % oneside: electronic viewing, twoside: printing
final,                   % draft: marks overfull hboxes, figures with paths
10pt]{article}

\listfiles               %  print all files needed to compile this document

\usepackage{relsize,makeidx,color,setspace,amsmath,amsfonts,amssymb}
\usepackage[table]{xcolor}
\usepackage{bm,ltablex,microtype}

\usepackage[pdftex]{graphicx}

\usepackage[T1]{fontenc}
%\usepackage[latin1]{inputenc}
\usepackage{ucs}
\usepackage[utf8x]{inputenc}

\usepackage{lmodern}         % Latin Modern fonts derived from Computer Modern

% Hyperlinks in PDF:
\definecolor{linkcolor}{rgb}{0,0,0.4}
\usepackage{hyperref}
\hypersetup{
    breaklinks=true,
    colorlinks=true,
    linkcolor=linkcolor,
    urlcolor=linkcolor,
    citecolor=black,
    filecolor=black,
    %filecolor=blue,
    pdfmenubar=true,
    pdftoolbar=true,
    bookmarksdepth=3   % Uncomment (and tweak) for PDF bookmarks with more levels than the TOC
    }
%\hyperbaseurl{}   % hyperlinks are relative to this root

\setcounter{tocdepth}{2}  % levels in table of contents

% prevent orhpans and widows
\clubpenalty = 10000
\widowpenalty = 10000

% --- end of standard preamble for documents ---


% insert custom LaTeX commands...

\raggedbottom
\makeindex
\usepackage[totoc]{idxlayout}   % for index in the toc
\usepackage[nottoc]{tocbibind}  % for references/bibliography in the toc

%-------------------- end preamble ----------------------

\begin{document}

% matching end for #ifdef PREAMBLE
% #endif

\newcommand{\exercisesection}[1]{\subsection*{#1}}


% ------------------- main content ----------------------



% ----------------- title -------------------------

\thispagestyle{empty}

\begin{center}
{\LARGE\bf
\begin{spacing}{1.25}
Quantum Computing and Many-Particle Problems
\end{spacing}
}
\end{center}

% ----------------- author(s) -------------------------

\begin{center}
{\bf Master of Science thesis project${}^{}$} \\ [0mm]
\end{center}

\begin{center}
% List of all institutions:
\end{center}
    
% ----------------- end author(s) -------------------------

% --- begin date ---
\begin{center}
Nov 28, 2019
\end{center}
% --- end date ---

\vspace{1cm}


\subsection{Quantum Computing and Machine Learning}



\textbf{Quantum Computing and Machine Learning} are two of the most promising
approaches for studying complex physical systems where several length
and energy scales are involved.  Traditional many-particle methods,
either quantum mechanical or classical ones, face huge dimensionality
problems when applied to studies of systems with many interacting
particles. To be able to define properly effective potentials for
realistic molecular dynamics simulations of billions or more particles,
requires both precise quantum mechanical studies as well as algorithms
that allow for parametrizations and simplifications of quantum
mechanical results. Quantum Computing offers now an interesting
avenue, together with traditional algorithms, for studying complex
quantum mechanical systems. Machine Learning on the other hand allows us to parametrize
these results in terms of classical interactions. These interactions
are in turn suitable for large scale molecular dynamics simulations of
complicated systems spanning from subatomic physics to materials
science and life science.

\subsection{Thesis Projects}

Here we present possible theses paths based on Quantum Computing  and
studies of quantum mechanical systems.  Possible systems are fermion
or boson systems where the quantum mechanical particles are confined
to move in various types of traps. A typical example which one could
start with is to study a system of one and two electrons in two or three
dimensions whose motion is confined by a harmonic  oscillator potential. This
system has, for one and two electrons only in two or three dimensions,
analytical solutions for the energy and the state
functions. 


Strongly confined electrons offer a wide variety of complex and subtle
phenomena which pose severe challenges to existing many-body methods.
Quantum dots in particular, that is, electrons confined in
semiconducting heterostructures, exhibit, due to their small size,
discrete quantum levels.  The ground states of, for example, circular
dots show similar shell structures and magic numbers as seen for atoms
and nuclei. These structures are particularly evident in measurements
of the change in electrochemical potential due to the addition of one
extra electron, $\Delta_N=\mu(N+1)-\mu(N)$. Here $N$ is the number of
electrons in the quantum dot, and $\mu(N)=E(N)-E(N-1)$ is the
electrochemical potential of the system.  Theoretical predictions of
$\Delta_N$ and the excitation energy spectrum require accurate
calculations of ground-state and of excited-state energies.  Small
confined systems, such as quantum dots (QD), have become very popular
for experimental study. 

Beyond their possible relevance for
nanotechnology, they are highly tunable in experiments and introduce
level quantization and quantum interference in a controlled way. 

A proper theoretical understanding of such systems
requires the development of appropriate and reliable theoretical
few- and many-body methods.  Furthermore, for quantum dots with more
than two electrons and/or specific values of the external fields, this
implies the development of few- and many-body methods where
uncertainty quantifications are provided.  For most methods, this
means providing an estimate of the error due to the truncation made in
the single-particle basis and the truncation made in limiting the
number of possible excitations.  For systems with more than three or
four electrons, \textbf{ab initio} methods that have been employed in
studies of quantum dots are variational and diffusion Monte Carlo, path integral approaches, large-scale diagonalization (full configuration
interaction and to a more
limited extent coupled-cluster theory.
Exact diagonalization studies are accurate for a very small number of
electrons, but the number of basis functions needed to obtain a given
accuracy and the computational cost grow very rapidly with electron
number.  In practice they have been used for up to eight
electrons, but the accuracy is very
limited for all except $N\le 3$ .  Monte Carlo methods have been
applied up to $N\sim 100$ electrons. Diffusion Monte Carlo, with
statistical and systematic errors, provide, in principle, exact
benchmark solutions to various properties of quantum dots. However,
the computations start becoming rather time-consuming for larger
systems.  Mean field methods like various Hartree-Fock approaches and/or 
current density functional
methods give results that are
satisfactory for a qualitative understanding of some systematic
properties. However, comparisons with exact results show discrepancies
in the energies that are substantial on the scale of energy
differences. The above-mentioned many-body methods all experience what is the loosely called the \emph{curse of dimensionality}. This means that the increased number of degrees freedom hinders the application of most first principle methods. As an example, for direct diagonalization methods, Hamiltonian matrices of dimensionalities larger than ten billion basis states, are simply computationally intractable. Such a dimensionality translates into few interacting particles only. For larger systems one is limited to much more approximative methods. 
Recent approaches in Quantum Computing (and Machine Learning as well) hold promise however to circumvent partly the above problems with increasing degrees of freedom. 
The aim of these thesis topics aim thus at exploring Quantum Computing algorithms for solving quantal many-particle problems. 

\paragraph{Specific tasks and milestones.}
The specific task here is to implenent and study Quantum Computing algorithms 
for solving quantum mechanical many-particle
problems. 
Recent scientific articles have shown the reliability of these methods on existing and real quantum computers, see for example 
\href{{https://arxiv.org/abs/1801.03897}}{Dumitrescu et al}.

Here the focus is first on tailoring a Hamiltonian like the pairing Hamiltonian and/or Anderson Hamiltonian in terms of quantum gates, as done by \href{{https://arxiv.org/abs/0705.1928}}{Ovrum and Hjorth-Jensen}.

Reproducing these results will be the first step of this thesis project. The next step includes adding more complicated terms to the Hamiltonian, like a particle-hole interaction as done in the work of \href{{http://iopscience.iop.org/article/10.1088/0954-3899/37/6/064035/meta}}{Hjorth-Jensen et al}.

The final step is to implement the action of these Hamiltonians on existing quantum computers like \href{{https://www.rigetti.com/}}{Rigetti's Quantum Computer}. 

The projects can easily be split into several parts and form the basis for collaborations among several students. The milestones are as follows
\begin{enumerate}
\item Spring 2020: Study and write a program to reproduce the pairing model results of \href{{https://arxiv.org/abs/0705.1928}}{Ovrum and Hjorth-Jensen} using the the \textbf{quantum phase estimation algotrithm}.

\item Fall 2020: Add more complicated terms to the Hamiltonian and rewrite these in terms of quantum gates. Write program to compute expectation values and compare with other many-body methods like exact diagonalization methods. The addition of variational quantum eigensolver (VQE) should also be implemented and compared with the phase estimation algorithm.

\item Spring 2021: Finalize thesis project and study other quantum mechanical methods and systems. 
\end{enumerate}

\noindent
The thesis is expected to be handed in May/June 2021.

\paragraph{Additional Literature.}
\begin{enumerate}
\item The :Github repository":"https://github.com/mhjensen/QuantumComputing" contains additional information, codes, articles and textbooks on quantum computing and quantum information theory.

\item The \href{{http://www.int.washington.edu/PROGRAMS/17-66W/}}{Whitebook on Quantum Computing for the Nuclear Many-body problem}, see the \textbf{whitepaper} link contains many interesting articles and links. 
\end{enumerate}

\noindent

% ------------------- end of main content ---------------

% #ifdef PREAMBLE
\end{document}
% #endif

