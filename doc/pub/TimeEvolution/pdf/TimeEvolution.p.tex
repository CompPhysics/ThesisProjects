%%
%% Automatically generated file from DocOnce source
%% (https://github.com/hplgit/doconce/)
%%
%%
% #ifdef PTEX2TEX_EXPLANATION
%%
%% The file follows the ptex2tex extended LaTeX format, see
%% ptex2tex: http://code.google.com/p/ptex2tex/
%%
%% Run
%%      ptex2tex myfile
%% or
%%      doconce ptex2tex myfile
%%
%% to turn myfile.p.tex into an ordinary LaTeX file myfile.tex.
%% (The ptex2tex program: http://code.google.com/p/ptex2tex)
%% Many preprocess options can be added to ptex2tex or doconce ptex2tex
%%
%%      ptex2tex -DMINTED myfile
%%      doconce ptex2tex myfile envir=minted
%%
%% ptex2tex will typeset code environments according to a global or local
%% .ptex2tex.cfg configure file. doconce ptex2tex will typeset code
%% according to options on the command line (just type doconce ptex2tex to
%% see examples). If doconce ptex2tex has envir=minted, it enables the
%% minted style without needing -DMINTED.
% #endif

% #define PREAMBLE

% #ifdef PREAMBLE
%-------------------- begin preamble ----------------------

\documentclass[%
oneside,                 % oneside: electronic viewing, twoside: printing
final,                   % draft: marks overfull hboxes, figures with paths
10pt]{article}

\listfiles               %  print all files needed to compile this document

\usepackage{relsize,makeidx,color,setspace,amsmath,amsfonts,amssymb}
\usepackage[table]{xcolor}
\usepackage{bm,ltablex,microtype}

\usepackage[pdftex]{graphicx}

\usepackage[T1]{fontenc}
%\usepackage[latin1]{inputenc}
\usepackage{ucs}
\usepackage[utf8x]{inputenc}

\usepackage{lmodern}         % Latin Modern fonts derived from Computer Modern

% Hyperlinks in PDF:
\definecolor{linkcolor}{rgb}{0,0,0.4}
\usepackage{hyperref}
\hypersetup{
    breaklinks=true,
    colorlinks=true,
    linkcolor=linkcolor,
    urlcolor=linkcolor,
    citecolor=black,
    filecolor=black,
    %filecolor=blue,
    pdfmenubar=true,
    pdftoolbar=true,
    bookmarksdepth=3   % Uncomment (and tweak) for PDF bookmarks with more levels than the TOC
    }
%\hyperbaseurl{}   % hyperlinks are relative to this root

\setcounter{tocdepth}{2}  % levels in table of contents

% prevent orhpans and widows
\clubpenalty = 10000
\widowpenalty = 10000

% --- end of standard preamble for documents ---


% insert custom LaTeX commands...

\raggedbottom
\makeindex
\usepackage[totoc]{idxlayout}   % for index in the toc
\usepackage[nottoc]{tocbibind}  % for references/bibliography in the toc

%-------------------- end preamble ----------------------

\begin{document}

% matching end for #ifdef PREAMBLE
% #endif

\newcommand{\exercisesection}[1]{\subsection*{#1}}


% ------------------- main content ----------------------



% ----------------- title -------------------------

\thispagestyle{empty}

\begin{center}
{\LARGE\bf
\begin{spacing}{1.25}
Time Evolution of Quantum Mechanical systems
\end{spacing}
}
\end{center}

% ----------------- author(s) -------------------------

\begin{center}
{\bf Master of Science thesis project${}^{}$} \\ [0mm]
\end{center}

\begin{center}
% List of all institutions:
\end{center}
    
% ----------------- end author(s) -------------------------

% --- begin date ---
\begin{center}
Dec 27, 2018
\end{center}
% --- end date ---

\vspace{1cm}


\subsection{Aim}

The aim of this thesis is to study the time evolution of a system of
quantum dots (electrons confined in two- or three-dimensional traps) using the
\href{{https://github.com/haakoek/PythonVersionMaster/tree/master/Thesis/Chapters}}{time-dependent Coupled Cluster method
method}. The
first step is to study a system of two electrons in two or three
dimensions whose motion is confined to an oscillator potential. This
system has, for two electrons only in two or three dimensions,
analytical solutions for the energy and the state
functions. Similarly, the matrix elements of the Coulomb interaction
have analytical expressions when one employs a harmonic oscillator
basis in two or three dimensions. The time-evolution of such a system
can thus be given in terms of analytical expressions. The first step
consists in studying numerically such systems using the coupled
cluster method with singles and double excitations.  

For two electrons
only the singles and doubles excitations provide an exact answer to
the quantum mechanical The numerical results will be compared with
analytical expressions. The next step is to add time-dependent
external potentials and study how the system evolves with time under
the influence of externally applied time-dependent fields and more confined electrons.  The final steps consists of extending the system to a so-called double potential well and study the time evolution of this system using the time-dependent Coupled Cluster method.




\paragraph{General introduction to possible physical systems.}
What follows here is a general introduction to systems of confined
electrons in two or three dimensions.  However, although the thesis
will focus on such systems, the codes will be written so that other
systems of trapped fermions or eventually bosons can be
handled. Examples could be neutrons in a \href{{https://journals.aps.org/prc/abstract/10.1103/PhysRevC.84.044306}}{harmonic oscillator trap}
or \href{{https://journals.aps.org/rmp/abstract/10.1103/RevModPhys.72.895}}{ions in various traps}.


Strongly confined electrons offer a wide variety of complex and subtle
phenomena which pose severe challenges to existing many-body methods.
Quantum dots in particular, that is, electrons confined in
semiconducting heterostructures, exhibit, due to their small size,
discrete quantum levels.  The ground states of, for example, circular
dots show similar shell structures and magic numbers as seen for atoms
and nuclei. These structures are particularly evident in measurements
of the change in electrochemical potential due to the addition of one
extra electron, $\Delta_N=\mu(N+1)-\mu(N)$. Here $N$ is the number of
electrons in the quantum dot, and $\mu(N)=E(N)-E(N-1)$ is the
electrochemical potential of the system.  Theoretical predictions of
$\Delta_N$ and the excitation energy spectrum require accurate
calculations of ground-state and of excited-state energies.  Small
confined systems, such as quantum dots (QD), have become very popular
for experimental study. 

Beyond their possible relevance for
nanotechnology, they are highly tunable in experiments and introduce
level quantization and quantum interference in a controlled way. In a
finite system, there cannot, of course, be a true phase transition,
but a cross-over between weakly and strongly correlated regimes is
still expected. There are several other fundamental differences
between quantum dots and bulk systems: 
\begin{enumerate}
\item Broken translational symmetry in a QD reduces the ability of the electrons to delocalize. As a result, a Wigner-type cross-over is expected for a smaller value of $r_s$, that is the so-called gas parameter $r_s=(c_d/a_B)(1/n)^d$, where $n$ is the electron density, $d$ is the spatial dimension, $a_B$ the effective Bohr radius and $c_d$ a dimension dependent constant.

\item Mesoscopic fluctuations, inherent in any confined system, lead to a rich interplay with the correlation effects. These two added features make strong correlation physics particularly interesting in a QD. As clean 2D bulk samples with large $r_s$ are regularly fabricated these days in semiconductor heterostructures, it seems to be just a matter of time before these systems are patterned into a QD, thus providing an excellent probe of correlation effects.
\end{enumerate}

\noindent
The above-mentioned quantum mechanical levels can, in turn, be tuned
by means of, for example, the application of various external fields.
The spins of the electrons in quantum dots provide a natural basis for
representing \href{{https://journals.aps.org/pra/abstract/10.1103/PhysRevA.57.120}}{so-called qubits}. The capability to
manipulate and study such states is evidenced by several recent
experiments.  Coupled quantum dots are particularly
interesting since so-called two-qubit quantum gates can be realized by
manipulating the exchange coupling which originates from the repulsive
Coulomb interaction and the underlying Pauli principle.  For such
states, the exchange coupling splits singlet and triplet states, and
depending on the shape of the confining potential and the applied
magnetic field, one can allow for electrical or magnetic control of
the exchange coupling. In particular, several recent experiments and
theoretical investigations have analyzed the role of effective
spin-orbit interactions in quantum dots and their influence on the
exchange coupling.

A proper theoretical understanding of the exchange coupling,
correlation energies, ground state energies of quantum dots, the role
of spin-orbit interactions and other properties of quantum dots as
well, requires the development of appropriate and reliable theoretical
few- and many-body methods.  Furthermore, for quantum dots with more
than two electrons and/or specific values of the external fields, this
implies the development of few- and many-body methods where
uncertainty quantifications are provided.  For most methods, this
means providing an estimate of the error due to the truncation made in
the single-particle basis and the truncation made in limiting the
number of possible excitations.  For systems with more than three or
four electrons, \textbf{ab initio} methods that have been employed in
studies of quantum dots are variational and diffusion Monte Carlo, path integral approaches, large-scale diagonalization (full configuration
interaction and to a more
limited extent coupled-cluster theory.
Exact diagonalization studies are accurate for a very small number of
electrons, but the number of basis functions needed to obtain a given
accuracy and the computational cost grow very rapidly with electron
number.  In practice they have been used for up to eight
electrons, but the accuracy is very
limited for all except $N\le 3$ .  Monte Carlo methods have been
applied up to $N\sim 100$ electrons. Diffusion Monte Carlo, with
statistical and systematic errors, provide, in principle, exact
benchmark solutions to various properties of quantum dots. However,
the computations start becoming rather time-consuming for larger
systems.  Mean field methods like various Hartree-Fock approaches and/or 
current density functional
methods give results that are
satisfactory for a qualitative understanding of some systematic
properties. However, comparisons with exact results show discrepancies
in the energies that are substantial on the scale of energy
differences.


\paragraph{Specific tasks and milestones.}
The specific task here is to study the time evolution of quantum mechanical systems using the Coupled Cluster (CC) method, in order to be able to study
the time evolution of an interacting quantum mechanical system, in
particular for electrons confined to move in two or three
dimensions. In this case, the system we will start with is that of
electrons confined in two- and three-dimensional regions, so-called quantum dots.
If properly object-oriented, the codes could also be used to study
atoms or molecules confined to three dimensions. The algorithmic
details behind the time-dependent coupled.  The final aim is to extend the
formalism and algorithms developed in the \href{{https://github.com/haakoek/PythonVersionMaster/tree/master/Thesis/Chapters}}{thesis of Haakon Emil Kristiansen}  to systems
with two or more electrons trapped in more than one oscillator
well. Such systems have been used as prototype systems for testing
quantum algorithms and building quantum circuits.  This method has
never before been applied to systems of strongly confined electrons
and opens up several interesting avenues for further research programs
as well as eventual publications.

The thesis project can easily be split into several parts and form the basis for the collaborations among several students. The milestones are as follows
\begin{enumerate}
\item Spring 2xxx: Start writing a Coupled Cluster code with doubles excitations only that solves a system of two electrons in two or three dimensions in a single Harmonic oscillator well. Finalize eventual remaining courses.

\item Fall 2xxx: Extend the project to include singles excitations, time evolution and a double potential well as discussed by \href{{https://journals.aps.org/prb/abstract/10.1103/PhysRevB.85.035319}}{Nielsen et al}.

\item Spring 2xxx: Include orbital dependencies as discussed in the \href{{http://aip.scitation.org/doi/abs/10.1063/1.4718427}}{article by Kvaal}. Finalize thesis. 
\end{enumerate}

\noindent
The thesis is expected to be handed in May/June  2xxx.

\paragraph{References.}
Highly relevant articles for possible thesis projects are:

\begin{enumerate}
\item \href{{http://aip.scitation.org/doi/abs/10.1063/1.4718427}}{\nolinkurl{http://aip.scitation.org/doi/abs/10.1063/1.4718427}}

\item \href{{https://journals.aps.org/prb/abstract/10.1103/PhysRevB.85.035319}}{\nolinkurl{https://journals.aps.org/prb/abstract/10.1103/PhysRevB.85.035319}}

\item \href{{https://journals.aps.org/prb/abstract/10.1103/PhysRevB.82.075319}}{\nolinkurl{https://journals.aps.org/prb/abstract/10.1103/PhysRevB.82.075319}}

\item \href{{https://arxiv.org/pdf/1006.2735.pdf}}{\nolinkurl{https://arxiv.org/pdf/1006.2735.pdf}}

\item \href{{https://juser.fz-juelich.de/record/187784/files/PhysRevB.91.075301.pdf}}{\nolinkurl{https://juser.fz-juelich.de/record/187784/files/PhysRevB.91.075301.pdf}}
\end{enumerate}

\noindent

% ------------------- end of main content ---------------

% #ifdef PREAMBLE
\end{document}
% #endif

