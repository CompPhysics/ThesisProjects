%%
%% Automatically generated file from DocOnce source
%% (https://github.com/hplgit/doconce/)
%%
%%
% #ifdef PTEX2TEX_EXPLANATION
%%
%% The file follows the ptex2tex extended LaTeX format, see
%% ptex2tex: http://code.google.com/p/ptex2tex/
%%
%% Run
%%      ptex2tex myfile
%% or
%%      doconce ptex2tex myfile
%%
%% to turn myfile.p.tex into an ordinary LaTeX file myfile.tex.
%% (The ptex2tex program: http://code.google.com/p/ptex2tex)
%% Many preprocess options can be added to ptex2tex or doconce ptex2tex
%%
%%      ptex2tex -DMINTED myfile
%%      doconce ptex2tex myfile envir=minted
%%
%% ptex2tex will typeset code environments according to a global or local
%% .ptex2tex.cfg configure file. doconce ptex2tex will typeset code
%% according to options on the command line (just type doconce ptex2tex to
%% see examples). If doconce ptex2tex has envir=minted, it enables the
%% minted style without needing -DMINTED.
% #endif

% #define PREAMBLE

% #ifdef PREAMBLE
%-------------------- begin preamble ----------------------

\documentclass[%
oneside,                 % oneside: electronic viewing, twoside: printing
final,                   % draft: marks overfull hboxes, figures with paths
10pt]{article}

\listfiles               %  print all files needed to compile this document

\usepackage{relsize,makeidx,color,setspace,amsmath,amsfonts,amssymb}
\usepackage[table]{xcolor}
\usepackage{bm,ltablex,microtype}

\usepackage[pdftex]{graphicx}

\usepackage[T1]{fontenc}
%\usepackage[latin1]{inputenc}
\usepackage{ucs}
\usepackage[utf8x]{inputenc}

\usepackage{lmodern}         % Latin Modern fonts derived from Computer Modern

% Hyperlinks in PDF:
\definecolor{linkcolor}{rgb}{0,0,0.4}
\usepackage{hyperref}
\hypersetup{
    breaklinks=true,
    colorlinks=true,
    linkcolor=linkcolor,
    urlcolor=linkcolor,
    citecolor=black,
    filecolor=black,
    %filecolor=blue,
    pdfmenubar=true,
    pdftoolbar=true,
    bookmarksdepth=3   % Uncomment (and tweak) for PDF bookmarks with more levels than the TOC
    }
%\hyperbaseurl{}   % hyperlinks are relative to this root

\setcounter{tocdepth}{2}  % levels in table of contents

% prevent orhpans and widows
\clubpenalty = 10000
\widowpenalty = 10000

% --- end of standard preamble for documents ---


% insert custom LaTeX commands...

\raggedbottom
\makeindex
\usepackage[totoc]{idxlayout}   % for index in the toc
\usepackage[nottoc]{tocbibind}  % for references/bibliography in the toc

%-------------------- end preamble ----------------------

\begin{document}

% matching end for #ifdef PREAMBLE
% #endif

\newcommand{\exercisesection}[1]{\subsection*{#1}}


% ------------------- main content ----------------------



% ----------------- title -------------------------

\thispagestyle{empty}

\begin{center}
{\LARGE\bf
\begin{spacing}{1.25}
Multiscale Physics and Machine Learning
\end{spacing}
}
\end{center}

% ----------------- author(s) -------------------------

\begin{center}
{\bf Master of Science thesis project${}^{}$} \\ [0mm]
\end{center}

\begin{center}
% List of all institutions:
\end{center}
    
% ----------------- end author(s) -------------------------

% --- begin date ---
\begin{center}
Dec 21, 2018
\end{center}
% --- end date ---

\vspace{1cm}


\subsection{Multiscale physics, from quantum mechanical simulations of atoms and molecules to molecular dynamics}

What is presented here is a set of several possible  projects which can lay the ground for 
Master theses projects.  The various projects are listed below,
opening up for several possibilities.

The aim of a program on multiscale physics is to develop a first
principle approach to systems of relevance for a variety of fields,
from materials science to nano-technology and biological systems and
even atomic nuclei and stars.  Common to all these systems is that
they entail a truly multiscale physics program that involves a proper
understanding of the links between the various scales, starting from
quantum-mechanical first principle studies of atoms, molecules and
eventually other spatially confined systems to Density functional
theories and finally microscopically derived potentials to be used in
molecular dynamics calculations. Many of the quantum-mechanical codes have been developed by previous 
students at the Computational Physics group. The pertinent codes can easily be used to address many of the projects listed below.

The computations required for accurate modeling and simulation of
large-scale systems with atomistic resolution involve a hierarchy of
levels of theory: quantum mechanics (QM) to determine the electronic
states; force fields to average the electronics states and to obtain
atom based forces (FF), molecular dynamics (MD) based on such an FF;
mesoscale or coarse grain descriptions that average or homogenize
atomic motions; and finally continuum level descriptions.  By basing
computations on first principles QM it is possible to overcome the
lack of experimental data to carry out accurate predictions with
atomistic resolu- tion, which would otherwise be
impossible. Furthermore, QM provides the funda- mental information
required to describe quantum effects, electronically excited states,
as well as reaction paths and barrier heights involved in chemical
reactions processes. However, the practical scale for accurate QM
today is <1,000 atoms per molecule or periodic cell (a length scale of
a few nanometers) whereas the length scale for modeling supramolecular
systems in biology may be in the tens of nano- meters, while
elucidating the interfacial effects between grains in composite
materials may require hundreds of nanometers, and modeling turbulent
fluid flows or shock- induced instabilities in multilayered materials
may require micrometers. Thus, simulations of engineered materials and
systems may require millions to billions of atoms, rendering QM
methods impractical.  Nonetheless, QM methods are essential for
accurately describing atomic-level composition, structure and energy
states of materials, considering the influence of electronic degrees
of freedom. By incorporating time-dependent information, the dynamics
of a system under varying conditions may be explored from QM-derived
forces, albeit within a limited timescale ($t <1$ ps). The prominent
challenge for theory and computation involves efficiently bridging,
from QM first-principles, into larger length scales with predominantly
heterogeneous spatial and density distributions, and longer timescales
of simulation – enough to connect into engineering-level design
variables – while retaining the appropriate accuracy and
certainty. Equally challenging remains the inverse top-down
engineering design problem, by which macroscopic material/process
properties would be tunable from optimizing its atomic-level
composition and structure. The aim of this large project is to to
develop breakthrough methods to staple and extend hierarchically over
existing to develop the necessary tools to enable continuous lateral
(multi-paradigm) and hierarchical (multiscale) couplings, between the
different theories and models as a function of their length- and
timescale range – a strategy often referred to as
First-Principles-Based Multiscale-Multiparadigm Simulation.  


To achieve these goals, several theses projects are shortly defined
below. These projects open also up for several fruitful collaborations
between the involved MSc students. The group in Computational Physics
has long-standing experience in defining projects where several
students may find a common a ground, either from a formalism point of
view or (and possibly and as well) phenomenological point of view.

There are several possible projects. The systems we have in mind are
mainly molecules like SiO$_2$, H$_2$O, CaCO$_3$ and other more
complicated molecules. To model the interaction between such molecules
and eventually derive microscopic interactions can be done via several
MSc projects. The following projects can be defined and extended upon:

\begin{enumerate}
\item Start with exisiting codes and thesis work by previous Master of Science students at the computational physics group. Here we think in particular of using existing Hartree-Fock codes for atoms and molecules and use these to develop potential surfaces and thereby extract parametrized potentils for molecular dynamics calculations. One possible path is to start with a gas of hydrogen and compare the parametrized potentials from Hartree-Fock theory with exisiting models.

\item Quantum mechanical correlations are however treated only approximatively in Hartree-Fock theory. Monte Carlo methods offer a path to cpmpute exactly ground state properties of correlated quantum mechanical systems. Here one can use existing (developed at the Computational Physics group) or develop a variational and diffusion Monte Carlo code that computes properties such as binding energies, root mean square radii, charge distributions and local potentials.  This topic can easily form the basis for one or two MSc thesis projects. These first principle approaches can be used to develop potential surfaces, which in turn can be used to parametrize potentials for molecular dynamics studies.  

\item If more complex atoms and molecules are involved, one may consider freezing the inner degrees of freedom using for example many-body perturbation theory to derive an effective interaction for electrons outside a chosen core. Alternatively one can use existing coupled cluster codes (also part of the computational quantum mechanics codes developed at the computational physics group) to derive potential surfaces and thereby parametrize potentials to use in molecular dynamics calculations.  

\item The quantum mechanical calculations can in turn be used to define better density functionals for the above atoms. Density functional theory can in turn be used to study for example lattices of atoms and molecules with say periodic boundary conditions.  

\item These effective potentials can in turn be compared with existing models such as the so-called ReaxFF potential. The implementation and studies of this potential for studies of say SiO$_2$ compounds could form the basis for one or two MSc thesis projects. 

\item Test several Machine Learning algorithms such as Multiperceptron models and the newly develop \href{{https://www.oreilly.com/ideas/neuroevolution-a-different-kind-of-deep-learning}}{neuroevolution} approach. 

\item For those interested in a more high-performance computing project, a lively research area is to develop and test the applicability of neural network algorithms on high-performance computing facilities using both GPUs and CPUs. 
\end{enumerate}

\noindent
The milestones are as follows
\begin{enumerate}
\item Spring 2019: Finalize relevant courses and start looking at Machine Learning approaches

\item Fall 2019: Work on thesis project

\item Spring 2020: Finalize thesis project.
\end{enumerate}

\noindent
The thesis is expected to be handed in May/June  2020.



\paragraph{Recent thesis and relevant literature.}
The recent theses of Morten Ledum, John-Anders Stende and Haakon Treider are excellent starting points for further explorations of Machine Learning approaches to Molecular dynamics studies. The recent text on \href{{https://www.oreilly.com/ideas/neuroevolution-a-different-kind-of-deep-learning}}{neuroevelolution algorithms}  may also be a good read.









% ------------------- end of main content ---------------

% #ifdef PREAMBLE
\end{document}
% #endif

