%%
%% Automatically generated file from DocOnce source
%% (https://github.com/hplgit/doconce/)
%%
%%
% #ifdef PTEX2TEX_EXPLANATION
%%
%% The file follows the ptex2tex extended LaTeX format, see
%% ptex2tex: http://code.google.com/p/ptex2tex/
%%
%% Run
%%      ptex2tex myfile
%% or
%%      doconce ptex2tex myfile
%%
%% to turn myfile.p.tex into an ordinary LaTeX file myfile.tex.
%% (The ptex2tex program: http://code.google.com/p/ptex2tex)
%% Many preprocess options can be added to ptex2tex or doconce ptex2tex
%%
%%      ptex2tex -DMINTED myfile
%%      doconce ptex2tex myfile envir=minted
%%
%% ptex2tex will typeset code environments according to a global or local
%% .ptex2tex.cfg configure file. doconce ptex2tex will typeset code
%% according to options on the command line (just type doconce ptex2tex to
%% see examples). If doconce ptex2tex has envir=minted, it enables the
%% minted style without needing -DMINTED.
% #endif

% #define PREAMBLE

% #ifdef PREAMBLE
%-------------------- begin preamble ----------------------

\documentclass[%
oneside,                 % oneside: electronic viewing, twoside: printing
final,                   % draft: marks overfull hboxes, figures with paths
10pt]{article}

\listfiles               %  print all files needed to compile this document

\usepackage{relsize,makeidx,color,setspace,amsmath,amsfonts,amssymb}
\usepackage[table]{xcolor}
\usepackage{bm,ltablex,microtype}

\usepackage[pdftex]{graphicx}

\usepackage[T1]{fontenc}
%\usepackage[latin1]{inputenc}
\usepackage{ucs}
\usepackage[utf8x]{inputenc}

\usepackage{lmodern}         % Latin Modern fonts derived from Computer Modern

% Hyperlinks in PDF:
\definecolor{linkcolor}{rgb}{0,0,0.4}
\usepackage{hyperref}
\hypersetup{
    breaklinks=true,
    colorlinks=true,
    linkcolor=linkcolor,
    urlcolor=linkcolor,
    citecolor=black,
    filecolor=black,
    %filecolor=blue,
    pdfmenubar=true,
    pdftoolbar=true,
    bookmarksdepth=3   % Uncomment (and tweak) for PDF bookmarks with more levels than the TOC
    }
%\hyperbaseurl{}   % hyperlinks are relative to this root

\setcounter{tocdepth}{2}  % levels in table of contents

% prevent orhpans and widows
\clubpenalty = 10000
\widowpenalty = 10000

% --- end of standard preamble for documents ---


% insert custom LaTeX commands...

\raggedbottom
\makeindex
\usepackage[totoc]{idxlayout}   % for index in the toc
\usepackage[nottoc]{tocbibind}  % for references/bibliography in the toc

%-------------------- end preamble ----------------------

\begin{document}

% matching end for #ifdef PREAMBLE
% #endif

\newcommand{\exercisesection}[1]{\subsection*{#1}}


% ------------------- main content ----------------------



% ----------------- title -------------------------

\thispagestyle{empty}

\begin{center}
{\LARGE\bf
\begin{spacing}{1.25}
Machine Learning and Neuroscience
\end{spacing}
}
\end{center}

% ----------------- author(s) -------------------------

\begin{center}
{\bf Master of Science thesis project${}^{}$} \\ [0mm]
\end{center}

\begin{center}
% List of all institutions:
\end{center}
    
% ----------------- end author(s) -------------------------

% --- begin date ---
\begin{center}
Dec 22, 2018
\end{center}
% --- end date ---

\vspace{1cm}


\subsection{Machine learning in neuroscience}

In recent years, machine learning and artificial intelligence
algorithms have been utilized in solving many fascinating problems in
different fields of science, including neuroscience.  Machine learning
has seen a rapid development in recent years and has now become a
mainstream tool in both research and industry. In particular, so
called "deep networks" with free and easy-to-use tools such as
TensorFlow has become very popular for learning and inference from
observed data.  The field of neuroscience has many applications for
machine learning as data, in particular at the level of networks of
biological nerve cells, are very difficult to interpret in terms of
properties of underlying nerve cells. One reason is that data sets
from electrical recordings in the cortex are huge (gigabytes of data
is not unusual from a single recording session).

Another reason is that the electrical signals themselves do not react
to the activity of the nerve cells in a simple and intuitive way.  In
this project, machine learning techniques will be used to analyse and
interpret dat from neuroscience simulatins and experiments.

The first step is to apply convolutional neural networks on data
generated by simulations of neural network using tools developed at
CINPLA at the University of Olso. In this model world the outcome is
known and the accuracy of machine learning in making predictions can
be tested. The next is to apply such validated machine learning tools
on experimental data recorded at CINPLA.  Here we will use techniques
from unsupervised learning with autoencoders and reinforcement
learning applied to data with noise as well.

\paragraph{Milestones and plans.}
The milestones are as follows
\begin{enumerate}
\item Spring 2019: Analyze simulated data with Convolutional Networks and reproduce results from simulations

\item Fall 2019: Include reinforcement learning and autoencoders and analyse data from experiments at CINPLA on for example visual cortex in mammals

\item Spring 2020: Finalize thesis project
\end{enumerate}

\noindent
The thesis is expected to be handed in May/June  2020.


% ------------------- end of main content ---------------

% #ifdef PREAMBLE
\end{document}
% #endif

