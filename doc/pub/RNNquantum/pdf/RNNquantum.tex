%%
%% Automatically generated file from DocOnce source
%% (https://github.com/hplgit/doconce/)
%%
%%


%-------------------- begin preamble ----------------------

\documentclass[%
oneside,                 % oneside: electronic viewing, twoside: printing
final,                   % draft: marks overfull hboxes, figures with paths
10pt]{article}

\listfiles               %  print all files needed to compile this document

\usepackage{relsize,makeidx,color,setspace,amsmath,amsfonts,amssymb}
\usepackage[table]{xcolor}
\usepackage{bm,ltablex,microtype}

\usepackage[pdftex]{graphicx}

\usepackage[T1]{fontenc}
%\usepackage[latin1]{inputenc}
\usepackage{ucs}
\usepackage[utf8x]{inputenc}

\usepackage{lmodern}         % Latin Modern fonts derived from Computer Modern

% Hyperlinks in PDF:
\definecolor{linkcolor}{rgb}{0,0,0.4}
\usepackage{hyperref}
\hypersetup{
    breaklinks=true,
    colorlinks=true,
    linkcolor=linkcolor,
    urlcolor=linkcolor,
    citecolor=black,
    filecolor=black,
    %filecolor=blue,
    pdfmenubar=true,
    pdftoolbar=true,
    bookmarksdepth=3   % Uncomment (and tweak) for PDF bookmarks with more levels than the TOC
    }
%\hyperbaseurl{}   % hyperlinks are relative to this root

\setcounter{tocdepth}{2}  % levels in table of contents

% prevent orhpans and widows
\clubpenalty = 10000
\widowpenalty = 10000

% --- end of standard preamble for documents ---


% insert custom LaTeX commands...

\raggedbottom
\makeindex
\usepackage[totoc]{idxlayout}   % for index in the toc
\usepackage[nottoc]{tocbibind}  % for references/bibliography in the toc

%-------------------- end preamble ----------------------

\begin{document}

% matching end for #ifdef PREAMBLE

\newcommand{\exercisesection}[1]{\subsection*{#1}}


% ------------------- main content ----------------------



% ----------------- title -------------------------

\thispagestyle{empty}

\begin{center}
{\LARGE\bf
\begin{spacing}{1.25}
Machine Learning, Deep learning and Quantum Mechanics, a Focus on Recurrent Neural Networks
\end{spacing}
}
\end{center}

% ----------------- author(s) -------------------------

\begin{center}
{\bf Master of Science thesis project${}^{}$} \\ [0mm]
\end{center}

\begin{center}
% List of all institutions:
\end{center}
    
% ----------------- end author(s) -------------------------

% --- begin date ---
\begin{center}
Dec 1, 2019
\end{center}
% --- end date ---

\vspace{1cm}


\subsection*{Machine Learning and the Quantum Many-body Problem}


The aim of this thesis project is to employ and develop Recurrent neural
networks and similar deep learning algorithms  for studies of many
interacting particles.  The thesis project can be combined with the
inclusion of more traditional many-body methods like coupled cluster
theory, large-scale eigenvalue methods and in-medium similarity
renormalization group theory.


Typical systems which can be studied are strongly confined
electrons. These systems offer a wide variety of complex and subtle
phenomena which pose severe challenges to existing many-body methods.
Quantum dots in particular, that is, electrons confined in
semiconducting heterostructures, exhibit, due to their small size,
discrete quantum levels.  The ground states of, for example, circular
dots show similar shell structures and magic numbers as seen for atoms
and nuclei.  Beyond their possible relevance for nanotechnology, they
are highly tunable in experiments and introduce level quantization and
quantum interference in a controlled way. Other systems of great
interest (and which are similar in terms of interaction models) is the
infinite homogeneous elctron gas in two and three dimensions.

A proper theoretical understanding of such systems requires the
development of appropriate and reliable theoretical few- and many-body
methods.  For systems with more than three or four electrons, \textbf{ab
initio} methods that have been employed in studies of quantum dots are
variational and diffusion Monte Carlo, path integral approaches,
large-scale diagonalization (full configuration interaction and to a
more limited extent coupled-cluster theory.  Exact diagonalization
studies are accurate for a very small number of electrons, but the
number of basis functions needed to obtain a given accuracy and the
computational cost grow very rapidly with electron number.  In
practice they have been used for up to eight electrons, but the
accuracy is very limited for all except $N\le 3$.  Monte Carlo methods
have been applied up to $N\sim 100$ electrons. Diffusion Monte Carlo,
with pertinent statistical errors, provide, in principle, exact
benchmark solutions to various properties of quantum dots. However,
the computations start becoming rather time-consuming for larger
systems.  Mean field methods like various Hartree-Fock approaches
and/or current density functional methods give results that are
satisfactory for a qualitative understanding of some systematic properties.

Other systems of interest are studies of infinite systems such as the
homogeneous electron gas and/or infinite nuclear matter.  The latter
is a widely popular many-body system, with far ranging consequences
and interests, from the structure of neutron stars to a deeper
understanding of neutrino oscillations.

The above-mentioned many-body methods all experience what
is the loosely called the \emph{curse of dimensionality}. This means that
the increased number of degrees freedom hinders the application of
most first principle methods. As an example, for direct
diagonalization methods, Hamiltonian matrices of dimensionalities
larger than ten billion basis states, are simply computationally
intractable. Such a dimensionality translates into few interacting
particles only. For larger systems one is limited to much more
approximative methods.  Reecent approaches in Machine Learning as well
as in quantum computing, hold promise however to circumvent partly the
above problems with increasing degrees of freedom.  The aim of this thesis project is
thus to explore various Machine Learning
approaches.

\paragraph{Specific tasks and milestones.}
The specific task here is to implement and study recently developed
deep learning algorithms based on neural networks and in particular on
recurrent neural networks for solving quantum mechanical many-particle
problems. The results can be easily compared with exisiting standard
many-particle codes developed by former students at the Computational
Physics group. These codes will serve as useful comparisons in order
to gauge the appropriateness of recent Machine Learning approaches to
quantum mechanical problems.  The aim here is to use recurrent neural
networks (RNNs) to study quantum mechanical many-body methods like the family
of similarity renormalization group methods.  This method is a rewrite
of many-body equations in terms of coupled ordinary differential
equations, see chapter 10 of \href{{https://www.springer.com/gp/book/9783319533353}}{Lecture Notes in Physics
vol. 936}.



The projects can easily be split into several parts and form the basis
for collaborations among several students. The milestones are as
follows:


\begin{enumerate}
\item Spring 2020: Develop a code for solving the Schroedinger equation for one and two particles in 1, 2 and 3 dimensions using recurrent neural networks and the Similarity Renormalization Group method. Compare the results to exact numerical diagonalization of the same systems.

\item Fall 2020: Extend the project to include the in-medium similarity renormalization group method and develop an RNN based code for handling many-particle systems.

\item Spring 2021: The choice of systems here is optional. Examples could be the quantum dot systems mentioned above or the homogenoeus electron gas. Such calculations have never been performed before and can lay the foundaton for several scientific articles.
\end{enumerate}

\noindent
The thesis is expected to be handed in May/June  2021.

\paragraph{References.}
Highly relevant articles for possible thesis projects are:

\begin{enumerate}
\item \href{{https://www.springer.com/gp/book/9783319533353}}{Hergert et al}, chapter 10 in particular

\item \href{{https://journals.aps.org/pra/abstract/10.1103/PhysRevA.96.042113}}{Mills et al} 

\item \href{{https://arxiv.org/abs/1909.02487}}{Pfau et al, Ab-Initio Solution of the Many-Electron Schrödinger Equation with Deep Neural Networks}

\item See also \href{{https://www.nature.com/articles/s41524-019-0221-0}}{Recent advances and applications of machine learning in solid-state materials science, by  Jonathan Schmidt et al}
\end{enumerate}

\noindent

% ------------------- end of main content ---------------

\end{document}

