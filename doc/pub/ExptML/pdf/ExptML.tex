%%
%% Automatically generated file from DocOnce source
%% (https://github.com/hplgit/doconce/)
%%
%%


%-------------------- begin preamble ----------------------

\documentclass[%
oneside,                 % oneside: electronic viewing, twoside: printing
final,                   % draft: marks overfull hboxes, figures with paths
10pt]{article}

\listfiles               %  print all files needed to compile this document

\usepackage{relsize,makeidx,color,setspace,amsmath,amsfonts,amssymb}
\usepackage[table]{xcolor}
\usepackage{bm,ltablex,microtype}

\usepackage[pdftex]{graphicx}

\usepackage[T1]{fontenc}
%\usepackage[latin1]{inputenc}
\usepackage{ucs}
\usepackage[utf8x]{inputenc}

\usepackage{lmodern}         % Latin Modern fonts derived from Computer Modern

% Hyperlinks in PDF:
\definecolor{linkcolor}{rgb}{0,0,0.4}
\usepackage{hyperref}
\hypersetup{
    breaklinks=true,
    colorlinks=true,
    linkcolor=linkcolor,
    urlcolor=linkcolor,
    citecolor=black,
    filecolor=black,
    %filecolor=blue,
    pdfmenubar=true,
    pdftoolbar=true,
    bookmarksdepth=3   % Uncomment (and tweak) for PDF bookmarks with more levels than the TOC
    }
%\hyperbaseurl{}   % hyperlinks are relative to this root

\setcounter{tocdepth}{2}  % levels in table of contents

% prevent orhpans and widows
\clubpenalty = 10000
\widowpenalty = 10000

% --- end of standard preamble for documents ---


% insert custom LaTeX commands...

\raggedbottom
\makeindex
\usepackage[totoc]{idxlayout}   % for index in the toc
\usepackage[nottoc]{tocbibind}  % for references/bibliography in the toc

%-------------------- end preamble ----------------------

\begin{document}

% matching end for #ifdef PREAMBLE

\newcommand{\exercisesection}[1]{\subsection*{#1}}


% ------------------- main content ----------------------



% ----------------- title -------------------------

\thispagestyle{empty}

\begin{center}
{\LARGE\bf
\begin{spacing}{1.25}
Nuclear Physics Experiments and  Machine Learning
\end{spacing}
}
\end{center}

% ----------------- author(s) -------------------------

\begin{center}
{\bf Master of Science thesis project${}^{}$} \\ [0mm]
\end{center}

\begin{center}
% List of all institutions:
\end{center}
    
% ----------------- end author(s) -------------------------

% --- begin date ---
\begin{center}
Dec 21, 2018
\end{center}
% --- end date ---

\vspace{1cm}


\subsection*{Machine Learning for interpreting Nuclear Physics experiments}



The classical picture of spherical nuclei is far from the reality of the
true nuclear structure. Shape coexistence is a nuclear phenomenon, where
the nucleus exists in two stable shapes at \href{{https://www.europhysicsnews.org/articles/epn/pdf/2001/01/epn01101.pdf}}{the same excitation energy}.
Nuclear properties provide unique information on the impetuses that
foster changes to the nuclear structure of rare isotopes. In some
neutron-rich nuclei, $0^{+}$ states are predicted to exhibit shape
coexistence. Therefore they are compelling to study, but \href{{http://iopscience.iop.org/article/10.1088/0954-3899/43/2/024001}}{experimentally
challenging}.
At low energies, where the only energetically allowed decay mode is
$0^{+} \rightarrow 0^{+}$, conversion electron spectroscopy is the
only viable technique to probe their properties.

At the National Superconducting Cyclotron Laboratory at Michigan State University Sean Liddick's group employs conversion electron spectroscopy to study
these transition rates. When a neutron-rich nucleus beta decays, a
neutron transforms into a proton and emits an electron $\beta$. The
excited nucleus can then interact electromagnetically with the
surrounding orbital electrons. This can result in the ejection of an
internal conversion electron $e^{-}$ from the
\href{{https://www.sciencedirect.com/science/article/pii/S0065253908608884}}{atom}.
Because this process is essentially simultaneous in time, it is pivotal
to differentiate between the electron $\beta$ emitted from the
nucleus and the internal conversion electron $e^{-}$ emitted from
the atom.

This project attempts to use supervised machine learning algorithms as a
means to distinguish between one and two electron events and predict the
electron(s) corresponding initial position(s) in a scintillator.


We chose to use convolutional neural networks to combat our problem.
Convolutional neural networks (CNN) are a class of deep neural networks
optimized for analyzing images. CNNs provide the computer with the
ability to see. This will allow us to treat each scintillator event as a
visual image, so the computer can see where the electron by looking at
all of the non-zero pixels. 
For more information see the \href{{https://github.com/harrisonlabollita/Harrison-LaBollita/tree/master/Machine\%20Learning}}{data analysis here}

The milestones are as follows
\begin{enumerate}
\item Spring 2019: Analyze simulated data with Convolutional Networks and reproduce results from simulations

\item Fall 2019: Include reinforcement learning and autoencoders and analyse data from experiments at Michigan State University

\item Spring 2020: Finalize thesis project.
\end{enumerate}

\noindent
The thesis is expected to be handed in May/June  2020.


% ------------------- end of main content ---------------

\end{document}

