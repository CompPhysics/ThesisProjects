%%
%% Automatically generated file from DocOnce source
%% (https://github.com/hplgit/doconce/)
%%
%%
% #ifdef PTEX2TEX_EXPLANATION
%%
%% The file follows the ptex2tex extended LaTeX format, see
%% ptex2tex: http://code.google.com/p/ptex2tex/
%%
%% Run
%%      ptex2tex myfile
%% or
%%      doconce ptex2tex myfile
%%
%% to turn myfile.p.tex into an ordinary LaTeX file myfile.tex.
%% (The ptex2tex program: http://code.google.com/p/ptex2tex)
%% Many preprocess options can be added to ptex2tex or doconce ptex2tex
%%
%%      ptex2tex -DMINTED myfile
%%      doconce ptex2tex myfile envir=minted
%%
%% ptex2tex will typeset code environments according to a global or local
%% .ptex2tex.cfg configure file. doconce ptex2tex will typeset code
%% according to options on the command line (just type doconce ptex2tex to
%% see examples). If doconce ptex2tex has envir=minted, it enables the
%% minted style without needing -DMINTED.
% #endif

% #define PREAMBLE

% #ifdef PREAMBLE
%-------------------- begin preamble ----------------------

\documentclass[%
oneside,                 % oneside: electronic viewing, twoside: printing
final,                   % draft: marks overfull hboxes, figures with paths
10pt]{article}

\listfiles               %  print all files needed to compile this document

\usepackage{relsize,makeidx,color,setspace,amsmath,amsfonts,amssymb}
\usepackage[table]{xcolor}
\usepackage{bm,ltablex,microtype}

\usepackage[pdftex]{graphicx}

\usepackage[T1]{fontenc}
%\usepackage[latin1]{inputenc}
\usepackage{ucs}
\usepackage[utf8x]{inputenc}

\usepackage{lmodern}         % Latin Modern fonts derived from Computer Modern

% Hyperlinks in PDF:
\definecolor{linkcolor}{rgb}{0,0,0.4}
\usepackage{hyperref}
\hypersetup{
    breaklinks=true,
    colorlinks=true,
    linkcolor=linkcolor,
    urlcolor=linkcolor,
    citecolor=black,
    filecolor=black,
    %filecolor=blue,
    pdfmenubar=true,
    pdftoolbar=true,
    bookmarksdepth=3   % Uncomment (and tweak) for PDF bookmarks with more levels than the TOC
    }
%\hyperbaseurl{}   % hyperlinks are relative to this root

\setcounter{tocdepth}{2}  % levels in table of contents

% prevent orhpans and widows
\clubpenalty = 10000
\widowpenalty = 10000

% --- end of standard preamble for documents ---


% insert custom LaTeX commands...

\raggedbottom
\makeindex
\usepackage[totoc]{idxlayout}   % for index in the toc
\usepackage[nottoc]{tocbibind}  % for references/bibliography in the toc

%-------------------- end preamble ----------------------

\begin{document}

% matching end for #ifdef PREAMBLE
% #endif

\newcommand{\exercisesection}[1]{\subsection*{#1}}


% ------------------- main content ----------------------



% ----------------- title -------------------------

\thispagestyle{empty}

\begin{center}
{\LARGE\bf
\begin{spacing}{1.25}
Machine Learning, Deep learning and Quantum Mechanics
\end{spacing}
}
\end{center}

% ----------------- author(s) -------------------------

\begin{center}
{\bf Master of Science thesis project${}^{}$} \\ [0mm]
\end{center}

\begin{center}
% List of all institutions:
\end{center}
    
% ----------------- end author(s) -------------------------

% --- begin date ---
\begin{center}
Dec 1, 2020
\end{center}
% --- end date ---

\vspace{1cm}


\subsection{Machine Learning and the Quantum Many-body Problem}

Solving quantum mechanical problems for atoms,  molecules, materials, and
interfaces is of fundamental importance to a large number of
disciplines including physics, chemistry, and materials science. Since
the early development of quantum mechanics, it has been noted, by
Dirac among others, that \emph{...approximate, practical methods of applying quantum mechanics should be developed, which can lead to an explanation of the main features of complex atomic systems without too much computation}. 

Historically, this has meant invoking
approximate forms of the underlying interactions (mean field, tight
binding, etc.) or relying on phenomenological fits to a limited number
of either experimental observations or theoretical results (e.g., force fields). 
The development of feature-based models is not
new in the scientific literature. Indeed, prior even to the acceptance
of the atomic hypothesis, van der Waals argued for an equation of
state based on two physical features. Machine learning (i.e.,
fitting parameters within a model) has been used in physics and
chemistry since the dawn of the computer age. The term machine
learning is new; the approach is not.

More recently, high-level ab initio calculations have been used to
train artificial neural networks to \href{{http://www.sciencedirect.com/science/article/pii/S0927025615007806?via%3Dihub}}{fit high-dimensional interaction
models}  and to make informed predictions about \href{{https://www.nature.com/articles/srep40827}}{material properties}. 

Machine learning can also be used to accelerate or bypass some of the
heavy machinery of the ab initio method itself. In the work of \href{{https://journals.aps.org/prl/abstract/10.1103/PhysRevLett.108.253002}}{Snyder et al},  the authors
replaced the kinetic energy functional within density-functional
theory with a machine-learned one, 
\emph{learned} the mappings from potential to electron density and from
charge density to kinetic energy, respectively.

\subsection{Thesis Projects}

Here we present possible theses paths based on Machine Learning and
studies of quantum mechanical systems.  Possible systems are fermion
or boson systems where the quantum mechanical particles are confined
to move in various types of traps. A typical example which one could
start with is to study a system of one and two electrons in two or three
dimensions whose motion is confined by a harmonic  oscillator potential. This
system has, for one and two electrons only in two or three dimensions,
analytical solutions for the energy and the state
functions. 


Strongly confined electrons offer a wide variety of complex and subtle
phenomena which pose severe challenges to existing many-body methods.
Quantum dots in particular, that is, electrons confined in
semiconducting heterostructures, exhibit, due to their small size,
discrete quantum levels.  The ground states of, for example, circular
dots show similar shell structures and magic numbers as seen for atoms
and nuclei. These structures are particularly evident in measurements
of the change in electrochemical potential due to the addition of one
extra electron.

Small confined systems, such as quantum dots (QD), have become very
popular for experimental study.  Beyond their possible relevance for
nanotechnology, they are highly tunable in experiments and introduce
level quantization and quantum interference in a controlled way.

Similarly, other fermionic systems like atoms, molecules, nuclei, the
infinite homogeneous electron gas and infinite nuclear matter, are all
systems which can be studied with the same many-body methods.
Thus, a proper theoretical understanding of such systems requires the
development of appropriate and reliable theoretical few- and many-body
methods.  Furthermore, for say quantum dots with more than two electrons
and/or specific values of the external fields, this implies the
development of few- and many-body methods where uncertainty
quantifications are provided.  For most methods, this means providing
an estimate of the error due to the truncation made in the
single-particle basis and the truncation made in limiting the number
of possible excitations.  For systems with more than three or four
electrons, \textbf{ab initio} methods that have been employed in studies of
quantum dots are variational and diffusion Monte Carlo, path integral
approaches, large-scale diagonalization (full configuration
interaction and to a more limited extent coupled-cluster theory.
Exact diagonalization studies are accurate for a very small number of
electrons, but the number of basis functions needed to obtain a given
accuracy and the computational cost grow very rapidly with electron
number.  In practice they have been used for up to eight electrons,
but the accuracy is very limited for all except $N\le 3$ .  Monte
Carlo methods have been applied up to $N\sim 100$ electrons. Diffusion
Monte Carlo, with statistical and systematic errors, provide, in
principle, exact benchmark solutions to various properties of quantum
dots. However, the computations start becoming rather time-consuming
for larger systems.  Mean field methods like various Hartree-Fock
approaches and/or current density functional methods give results that
are satisfactory for a qualitative understanding of some systematic
properties. However, comparisons with exact results show discrepancies
in the energies that are substantial on the scale of energy
differences. The above-mentioned many-body methods all experience what
is the loosely called the \emph{curse of dimensionality}. This means that
the increased number of degrees freedom hinders the application of
most first principle methods. As an example, for direct
diagonalization methods, Hamiltonian matrices of dimensionalities
larger than ten billion basis states, are simply computationally
intractable. Such a dimensionality translates into few interacting
particles only. For larger systems one is limited to much more
approximative methods.  Reecent approaches in Machine Learning as well
as in quantum computing, hold promise however to circumvent partly the
above problems with increasing degrees of freedom.


The specific aim of this
thesis topic is to, based on quantum Monte Carlo methods, to
explore deep learning approaches to many-body systems based on neural networks, see the recent work of for example
\href{{https://arxiv.org/abs/2007.14282}}{Adams et al.}. 

\paragraph{Specific tasks and milestones.}
The specific task here is to implement and study deeep learning methods 
for solving quantum mechanical many-particle
problems of fermions. The results can  be easily compared with exisiting standard
many-particle codes developed by former students at the Computational
Physics group. These codes will serve as useful comparisons in order
to gauge the appropriateness of recent Machine Learning approaches to
quantum mechanical problems.

Four recent articles (see below) have shown the reliability of these methods and
the aim is to study and implement some of these algorithms to first a system
of one electron moving in a confining potential. Thereafter we switch
to the interacting two-electron and many-electron problems and apply
these algorithms.  We will focus first on systems of quantum dots, but the codes can easily be extended to systems of atoms and molecules.

The basic framework to be developed contains:
\begin{enumerate}
\item The determination of a self-consistent single-particle basis using the Hartree-Fock method.

\item The development of a Variational Monte Carlo code for fermions. If time allows, excursions into diffusion Monte Carlo are also possible.

\item The development of a trial wave function for the correlated part based on deep neural networks.
\end{enumerate}

\noindent
The milestones are

\begin{enumerate}
\item Spring semester 2021: Develop a Variational Monte Carlo framework for fermions without a self-consistent single-particle basis. Parts of this will be done in the course FYS4411, Computational Quantum Mechanics.

\item Fall semester 2021: Develop a code for a self-consistent single-particle basis using the Hartree-Fock method. The code should be flexible enough to handle different types of interacting  fermionic systems.

\item Fall semester 2021: Introduce neural networks for the correlated part of the trial wave function used in Monte Carlo studies.

\item Spring semester 2022: Perform numerical studies of different quantum mechanical systems. Study weakly correlated and strongly correlated fermionic systems, ground state properties and correlation functions like two-body densities.

\item Spring semester 2022: Final thesis
\end{enumerate}

\noindent
The thesis is expected to be handed in May/June  2022.

\paragraph{References.}
Highly relevant articles for possible thesis projects are:

\begin{enumerate}
\item \href{{http://science.sciencemag.org/content/355/6325/602}}{Carleo and Troyer} 

\item \href{{https://journals.aps.org/pra/abstract/10.1103/PhysRevA.96.042113}}{Mills et al} 

\item \href{{https://arxiv.org/abs/1909.02487}}{Pfau et al, Ab-Initio Solution of the Many-Electron Schrödinger Equation with Deep Neural Networks}

\item \href{{https://arxiv.org/abs/2007.14282}}{Adams et al., Variational Monte Carlo calculations of $A\le 4$  nuclei with an artificial neural-network correlator ansatz}.

\item See also \href{{https://www.nature.com/articles/s41524-019-0221-0}}{Recent advances and applications of machine learning in solid-state materials science, by  Jonathan Schmidt et al}
\end{enumerate}

\noindent

% ------------------- end of main content ---------------

% #ifdef PREAMBLE
\end{document}
% #endif

