\documentclass{report}

\begin{document}
\subsection*{Abstract}

The thesis concerns a numerical implementation of the Parquet summation of diagrams within Green's functions theory applied to calculations of nuclear systems. The main motivation has been to investigate whether it is possible to develop this approach to a level of accuracy and reliability comparable to other {\it ab initio} nuclear structure methods. 

The Green's functions approach is theoretically well-established in many-body theory, but to our knowledge, no actual application to nuclear systems has been previously published. It has a number of desirable properties, foremost the gentle scaling with system size compared to direct diagonalization and the closeness to experimentally accessible quantities. The main drawback is numerical instabilities due to the pole structure of the one-particle propagator which leads to convergence difficulties. This issue is one of the main focal points of the work presented in this thesis. Several strategies to improve the convergence properties are described and investigated.

We have applied the method both to a simple model which can be solved by exact diagonalization and to the more realistic ${}^4$He system. The results shows that our implementation is close to the exact solution in the simple model as long as the interaction strengths are small. As the number of particles increases, convergence is increasingly hard to obtain. We obtain results reasonably close to results from comparable approaches in the ${}^4$He case. The numerical instabilities in the current implementation still prevents the desired accuracy and stability necessary to achieve the current benchmark standards. 


\end{document}