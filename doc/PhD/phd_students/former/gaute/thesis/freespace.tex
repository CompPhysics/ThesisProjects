\chapter{Two-body scattering}

\section{Momentum space representation}
The Schr\"odinger equation in abstract vector representation is
\begin{equation}
  \left( T + V \right) \vert \psi_\alpha \rangle = E_\alpha \vert\psi_\alpha \rangle 
  \label{eq:eq1}
\end{equation}
The most commonly used representations of Eq.~\ref{eq:eq1} are the coordinate and
the momentum space representations, which defines the completeness relations
\begin{eqnarray}
 {\bf 1}&  = &  \int d{\bf r} \:\vert  {\bf r} \rangle \langle {\bf r}\vert, \:\: 
 \langle  {\bf r}\vert  {\bf r'} \rangle = \delta ( {\bf r}-{\bf r'}) \\
{\bf 1} & = & \int d{\bf k} \:\vert  {\bf k} \rangle \langle {\bf k}\vert, \:\: 
  \langle  {\bf k}\vert  {\bf k'} \rangle = \delta ( {\bf k}-{\bf k'}) 
\end{eqnarray}
from which the normalized plane-wave states are given
\footnote{Some authors define the plane wave by  $ \langle  {\bf r}\vert  {\bf k} \rangle = 
  \exp \left( i {\bf k\cdot r} \right) $, in which case 
  $ \langle  {\bf k}\vert  {\bf k'} \rangle = \left( 2\pi \right)^3 \delta ( {\bf k}-{\bf k'}) $
  and each $\int d{\bf k}$ is replaced by $ (1/2\pi)^{3/2}\int d{\bf k} $.}�
\begin{equation}
  \langle  {\bf r}\vert  {\bf k} \rangle = \left( 1\over2\pi \right)^{3/2}\exp \left( i {\bf k\cdot r} \right)
  \label{eq:planewave1}
\end{equation}
which is a transformation function defining the mapping from the abstract 
$ \vert {\bf k} \rangle $ to the abstract $\vert {\bf r}\rangle $ space.
Eq.~\ref{eq:eq1} becomes in the momentum space representation 
\begin{equation}
  {\hbar^2 \over 2\mu} k^2 \psi_\alpha({\bf k})  + 
  \int d{\bf k'}\: \langle {\bf k}\vert V\vert {\bf k'}\rangle \psi_\alpha({\bf k'}) = 
  E_\alpha \psi_\alpha({\bf k})
  \label{eq:eq2}
\end{equation}
Here $\psi_\alpha({\bf k}) = \langle {\bf k} \vert \psi_\alpha \rangle $. 
Expanding $ \psi_\alpha({\bf k}) $ in a complete set of spherical harmonics 
\begin{equation}
  \psi_\alpha({\bf k}) = \sum _{lm}�\psi_{\alpha lm}(k)Y_{lm} (\hat{k}), \:\:
  \psi_{\alpha lm}(k) = \int d\hat{k}�Y_{lm}^*(\hat{k})\psi_\alpha({\bf k})   
  \label{eq:eq3}
\end{equation}
and inserting Eq.~\ref{eq:eq3} in Eq.~\ref{eq:eq2}, and projecting Eq.~\ref{eq:eq2}
on $Y_{lm}(\hat{k})$ gives the angular momentum coupled equation, 
\begin{equation}
  \left( {\hbar^2 \over 2\mu} k^2 - E_{\alpha lm}\right) \psi_{\alpha lm}(k) =  
  -\sum_{l'm'} \int_{0}^\infty dk' {k'}^2 V_{lm, l'm'}(k,k') \psi_{\alpha l'm'}(k') 
  \label{eq:eq4}
\end{equation}
where
\begin{equation}
  V_{lm, l'm'}(k,k') = \int d\hat{k}\: \int d\hat{k'} Y_{lm}^*(\hat{k})
  \langle {\bf k}\vert V \vert {\bf k'}\rangle Y_{l'm'}(\hat{k}')
  \label{eq:eq5}
\end{equation}
The momentum space representation of $V$ is related to the 
coordinate space representation by a double Fourier-Bessel transformation 
\begin{equation} 
  V_{lm, l'm'}(k,k') = {2\over \pi} i^{l'-l} \int dr\: r^2\int dr'\: {r'}^2 
  j_l(kr) j_{l'}(k'r') V_{lm,l'm'}(r,r') 
  \label{eq:eq6}
\end{equation}
here 
\begin{equation}
  V_{lm,l'm'}(r,r')  = \int d\hat{r} \:\int d\hat{r}' \:Y_{lm}^*(\hat{r})
  \langle {\bf r}\vert V \vert {\bf r}'\rangle Y_{l'm'}(\hat{r'})
  \label{eq:eq7}
\end{equation}
and the partial wave expansion of the plane-wave ~\ref{eq:planewave1} has been
utilized, e.g. 
\begin{equation}
  \langle  {\bf r}\vert  {\bf k} \rangle = \left( 1\over2\pi \right)^{3/2}\exp \left( i {\bf k\cdot r} \right) = 
  4\pi \sum_{lm} i^l j_l(kr) Y_{lm}^*(\hat{k})Y_{lm}(\hat{r}) 
  \label{eq:planewave2}
\end{equation} 
In the case of local interactions, e.g. 
\begin{equation}
  \langle {\bf r} \vert V \vert {\bf r'}\rangle = V({\bf r}) \delta\left( {\bf r} - {\bf r'} \right) 
  \label{eq:local}
\end{equation}
utilizing the partial wave expansion of the dirac delta function 
\begin{eqnarray}
  \delta\left( {\bf r} - {\bf r'} \right) & = & { \delta\left( r-r' \right)\over r^2 }\delta( \hat{r},\hat{r}' ) \\ 
  \delta( \hat{r},\hat{r}' ) & = & \sum_{l}{2l +1\over 4\pi} P_l(\hat{r}\cdot\hat{r}') = 
  \sum_l \left( Y_l(\hat{r}) \cdot Y_l(\hat{r}')\right) 
  \label{eq:dirac_delta}
\end{eqnarray}
the interaction in $\vert klm \rangle $ representation takes the form  
\begin{equation} 
  V_{lm, l'm'}(k,k') = {2\over \pi} i^{l'-l} \int dr\: r^2\: 
  j_l(kr) j_{l'}(k'r) V_{lm,l'm'}(r) 
  \label{eq:local_eq}
\end{equation}
where 
\begin{equation}
  V_{lm,l'm'}(r)  = \int d\hat{r} \: \:Y_{lm}^*(\hat{r})
  \vert V( {\bf r} ) Y_{l'm'}(\hat{r})
\end{equation}
further if the interaction is in addition central, $V({\bf r}) = V(r) $,
\begin{equation}
  V_{lm.l'm'}(r) = v(r)\int d\hat{r}\: Y_{lm}^*(\hat{r})Y_{l'm'}(\hat{r}) 
  = V(r)\delta_{l,l'}\delta_{m,m'}
\end{equation}
and 
\begin{equation}
  V_{lm.l'm'}(k,k') = V_{l}(k,k') \delta_{l,l'}\delta_{m,m'}
\end{equation}
where
\begin{equation}
  V_{l}(k,k') = {2\over \pi}\int dr\: r^2j_l(kr)j_l(k'r)V(r)
\end{equation}
Assuming a local and central interaction, Eq.~\ref{eq:eq4} 
decouples in angular momentum, and gives the momentum space representation of the 
Schr\"odinger equation
\begin{equation}
  \left( {\hbar^2 \over 2\mu} k^2 - E_{\alpha l}\right) \psi_{\alpha l}(k) =  
  - \int_{0}^\infty dk' {k'}^2 V_{l}(k,k') \psi_{\alpha l }(k') 
  \label{eq:momentum_space}
\end{equation}
The wave functions $\psi_{\alpha l }(k) $ defines a complete orthogonal set 
of functions, which spans the space of functions with a positive finite Euclidean norm 
( also called $l^2$-norm), $ \sqrt{ \langle \psi_\alpha \vert \psi_\alpha \rangle} $, which 
is a Hilbert space. The corresponding normalized wave function in coordinate space
is given by the Fourier-Bessel transform 
\begin{equation}
  \phi_{\alpha l}(r)  = \sqrt{ 2\over \pi}\int dk\: k^2 j_l(kr) \psi_{\alpha l}(k)
\end{equation}  
the orthogonlaity of the coordinate wave functions is easily proved using the orthogonality 
of the spherical Bessel functions,
\begin{equation}
  \int dr\: r^2 j_l(kr) j_l(k'r) = {\pi \over 2}�{ \delta (k-k')\over kk'} 
\end{equation}

  
