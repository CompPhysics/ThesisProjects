\chapter{Introduction}
\section{Overall motivation and aims}

\section{Exploring exotic structures by complex scaling techniques}

In nuclear physics, like in atomic physics, the expansion of many-body wavefunctions on
single particle bases, generated by a suitable potential has been 
common practice. 
For a given potential the single particle eigenstates form a complete set of states, 
\begin{equation}
{\bf 1} = \sum _{b}\vert\psi_{nl}\rangle\langle\psi_{nl}\vert + 
{1\over 2}\int_{-\infty}^{\infty} dk k^2\vert\psi_{l}(k)\rangle\langle\psi_{l}(k)\vert,   
\label{eq:unity1}
\end{equation}
A proof of this completeness relation,
more precisely known as the \emph{resolution of unity}, 
is given by Newton in \cite{newton}. The relation also applies to the binary
interaction of say two nucleons and their relative motion.
 The sum is over the bound states in the
system, while the integral is over the positive energy continuum states. 
The infinite space spanned by this basis is given by all square integrable 
functions on the real energy axis, known as the $L^{2}$ space, which forms a Hilbert space.
In the case of a confining harmonic oscillator potential there is an infinite number
of bound states and no continuum integral.

During the last decade the exploration of nuclear driplines has pushed traditional 
single particle methods to their limits of applicability. 
The traditional shell-model with harmonic oscillator single particle wave functions
 works well in the regime 
of well bound nuclei. Moving towards the driplines however the nuclei cease to be well bound,  
and coupling to continuum structures plays an important role. A modification of the shell model 
where bound, resonant and continuum states are treated on equal footing
has been under development the last few years, and has 
become known as the \emph{Gamow shell model}, see 
\cite{liotta,betan,witek1,witek2,roberto}. The first attempt to also include antibound states 
in a realistic nuclear calculation is that of \cite{vertse}, where
the role of antibound states in the pole RPA description of the giant monopole resonance was
investigated. Recently the role of anti-bound states in the Gamow shell model 
description of halo nuclei has been discussed, \cite{betan2}.  
  
 
The study of two-body resonant structures has a long history in 
theoretical physics. There exists a variety of methods described in  
textbooks such as \cite{newton, kukulin, sitenko, zeldovich}. Among the more popular methods
are those of analytic continuation; 
the complex scaling method (CSM) and the method based on 
analytic continuation in the coupling constant (ACCC).

In this work we consider an approach formulated for integral equations 
in momentum space.  The method is based on 
deforming contour integrals in momentum space, and is known as  
the contour deformation or distortion 
method (CDM). It has been shown in \cite{afnan1} that a \emph{contour rotation} in 
momentum space is equivalent to a rotation of the corresponding differential equation in
coordinate space. The coordinate space analog is often referred to as the 
\emph{dilation group transformation}, or \emph{complex scaling}. The
\emph{dilation group transformation} was first formulated and discussed in 
\cite{abc,abc1}, and was developed to examine the spectrum of the Green's 
function on the second energy sheet. 

The contour deformation method (CDM) formulated \emph{in momentum space} is not new
in nuclear physics. It was studied and applied in the 1960`s  and 1970`s,  
see for example \cite{brayshaw,nuttal,stelbovics,glockle}, especially in the field of 
three-body systems. Most of these references
applied a \emph{contour rotation} in momentum space. By restricting 
oneself to a rotated contour certain limitations and restrictions however appear in the
equations, determined by the analytical structure of the integral kernels and potentials. 
In \cite{glockle} a more sophisticated choice of contour, 
based on rotation and translation, 
was applied to the three-nucleon momentum space Faddeev equation for a separable Yamaguchi interaction.
This choice of contour made it possible to avoid the logarithmic singularities of the Faddeev kernel
and hence allowed for a continuation in energy to the non-physical energy sheet. 

A revitalising of the contour deformation method in momentum space is in place,
given the new theoretical challenges of dripline physics.   
CDM is a method which allows for accurate and stable solutions of bound, anti-bound, capture, and 
decay states. We consider a generalized type of contour, allowing for an analytic continuation into the 
third quadrant of the complex $k$-plane. Antibound - and capture states near the scattering
threshold may then be calculated at a specified accuracy.  This choice 
of contour may be regarded as belonging to the \emph{Berggren class} of contours 
\cite{berggren}. Berggren \cite{berggren} and later Lind \cite{lind} studied various 
completeness relations derived by analytic continuation of the completeness 
relation, stated in equation~(\ref{eq:unity1}), to the complex plane. The Berggren completeness 
includes discrete summation 
over resonant as well as bound states. Our choice of contour differs from recently applications
of the Berggren formalism, see for example \cite{liotta,betan,witek1,witek2,roberto}, 
in that the contour approaches infinity along complex
rays in the complex $k$-plane as opposed to contours which approach 
infinity along the real $k$-axis.
We will point out the intimate relationships between complex coordinate scaling, the 
general Berggren basis and the method of continuation of the scattering equations to 
the second energy sheet by contour deformation.  

Complex scaling \emph{in coordinate space} has for a long time 
been used extensively in atomic and molecular physics, see
\cite{moise}. 
During the last decade it has also been applied in nuclear physics, 
as interest in loosely bound nuclear halo systems has grown, see for 
example \cite{csoto, garrido, imante}. Complex scaling 
in coordinate space is usually based on a variational method \cite{moise}, and an 
optimal variational basis and scaling parameters have to be searched for. One of the disadvantages
of the coordinate space approach is that the boundary conditions have to be 
built into the equations, and convergence may be slow if the basis does not 
mirror the physical outgoing boundary conditions well.  

There are several advantages in considering the contour deformation method 
in momentum space. First, most realistic potentials derived from field theoretical 
considerations are given explicitly in momentum space. Secondly, the boundary conditions 
are automatically built into the integral equations. Moreover, 
the Gamow states (physical resonances) \cite{kukulin} in momentum space
are non-oscillating and rapidly decreasing, even for Gamow 
states with large widths, far from the real 
energy axis,
as opposed to the complex scaled coordinate space counterpart. 
The latter states are represented by  
strongly oscillating and exponentially decaying functions. 
Finally, numerical procedures are
often easier to implement and check. Convergence is easily obtained by just increasing
the number of integration points in the numerical integration. 

If one restricts the deformation to a rotation of the contour,
as studied in \cite{brayshaw,nuttal,stelbovics,nuttal1,tikto}, 
one is not able to expose antibound states in the
energy spectrum, since the maximum allowed rotation angle does not allow rotation into the
third quadrant of the complex momentum plane. This limitation is sometimes used as an 
argument  for advocating other approaches, such as the ACCC method, see the recent work
of Aoyama  \cite{aoyama}. 
We will show that by distorting the contour by  
\emph{rotation and translation}  into the third quadrant of the complex $k$-plane, we 
are able to introduce a new feature to the complex scaling method, namely 
\emph{accurate calculation of antibound states as well as bound and resonant states}. 
CDM represents also an alternative to the so-called \emph{exterior complex scaling} method. 
The \emph{exterior complex scaling} method
was just formulated to avoid intrinsic non-analyticities of the potential, and in this way 
calculation of resonances in \emph{non-dilation} analytic potentials are made possible, see
\cite{moise} and references therein.  

The contour deformation method has also been applied to the solution of the
full off-shell scattering amplitude ($t$-matrix), see \cite{afnan1,nuttal, stelbovics,afnan}. 
By rotating the integration contour, an integral equation is obtained with a 
compact integral kernel. This has numerical advantages as the kernel is no longer
singular. As discussed in \cite{nuttal}, a rotation of the contour gives certain
restrictions on the rotation angle and maximum incoming/outgoing momentum in
the scattering amplitude. We will again show that our extended choice of contour in momentum 
space avoids all these limitations and that an accurate calculation of the
scattering amplitude can be obtained. 
Thus, the method we advocate allows us to give 
an accurate calculation of the full energy spectrum. Moreover, it yields  
a powerful method for calculating the full off-shell complex scattering amplitude ($t$-matrix). 
It is also rather straightforward to extend
this scheme to in-medium scattering in e.g., infinite nuclear matter.

\section{Challenges for the newly developed Gamow Shell Model}

Present and proposed nuclear structure research facilities
for radioactive beams 
will open new territory into regions of heavier nuclei.
Such systems pose significant challenges to existing 
nuclear structure models since many of
these nuclei will be unstable and short-lived. How to deal with weakly
bound systems and coupling to resonant states is an open and interesting problem in
nuclear spectroscopy. Weakly bound systems cannot be properly  described within a 
standard shell-model approach since even bound states exhibit a strong coupling to
the single-particle continuum.

It is therefore important to investigate theoretical methods that will allow
for a description of systems involved in such
element production.  Ideally, we would like to start from an ab initio 
approach  with the free nucleon-nucleon interaction and eventually also 
three-body interactions as the basic building
blocks for the derivation of an effective shell-model interaction. 
The newly developed Gamow shell model offers such a possibility, 
see for example 
Refs.~\cite{michel1,michel2,michel3,liotta,betan,witek1,witek2,roberto,betan2}.
Similarly, the recent work on the continuum shell-model by Volya and Zelevinsky
\cite{vz2003} conveys similar interesting perspectives. Here we focus on the 
Gamow shell model, which has proved 
to be a powerful tool in describing and understanding the formation of 
multi-particle resonances within a shell-model formulation. Representing
the shell-model equations using a  Berggren basis
\cite{berggren,berggren1,berggren2,berggren3,lind,hagen}, allows for a simple
interpretation of multi-particle resonances in terms of single-particle
resonances, as opposed to the traditional harmonic oscillator representation, where 
resonances never appear explicitly. 

Although the Gamow shell model approach is a powerful tool in this respect, there
are major computational and theoretical challenges that
need to be overcome if we aim at a realistic description 
of weakly bound and unbound nuclei. 
One of the challenges regarding the Gamow shell model 
discussed in Refs.~\cite{liotta, betan}, was
the problem of choosing a contour in the complex $k$-plane that 
in the many-particle case selects the physical interesting
states from the dense distribution of continuum states. 
In Refs.~\cite{liotta,betan}  the authors employ a ``square-well'' contour, which in the 
two-particle case separates the resonances from the complex-continuum 
states. In the case where more than two particles are present in the shell-model space, 
the resonant states mix  with the complex continuum states,
and an identification of the multi-particle resonances becomes difficult.

In this work we consider as a test case the light 
drip-line nuclei $^{5,6,7}$He, and the formation 
of resonances in these nuclei 
starting from a single-particle picture. These nuclei have also been 
studied with a number of other methods, see for example Ref.~\cite{jonson}, 
and references therein.
We  construct a single-particle basis using the contour deformation method 
in momentum space, discussed in detail in Ref.~\cite{hagen}, see also Ref.~\cite{nimrod}
for further references on complex scaling.
We show that choosing a rotated plus translated contour in the complex
plane, a large portion of the many-particle energy surface is free from
complex continuum states. This choice of contour isolates the physical resonances,
and allows for a clear distinction of many-particle resonances 
from the dense distribution 
of complex continuum states, also in the case when the number of particles 
exceeds two. 

The most severe problem and future challenge is that the 
shell-model dimension increases dramatically for $n > 2$
particles moving in a large valence space, 
this is what we henceforth refer to as the dimensionality problem.
Using a technique such as the traditional Lanczos iteration method \cite{lanczo}
fails in Gamow shell model calculations.
Dealing with large real symmetric matrices, the Lanczos scheme is a powerful 
method when one wishes to calculate the states lowest in energy.
In Gamow shell model calculations there may be  a large 
number of complex continuum states
lying below the physical resonances in real energy.
In addition it is difficult to
predict where the multi-particle resonances will appear after diagonalization. 
In Refs.~\cite{witek1,witek2} this problem was circumvented by 
choosing a small number of complex continuum states in the single-particle basis. 
It was also pointed out that the results obtained 
were not converged with respect to the number of single-particle continuum orbits.
In Ref.~\cite{michel2} another approach was considered, where at
most two particles where allowed to move in complex continuum states.
This was based on the assumption that these configurations play the dominant
role in the formation of many-particle resonances, and configurations where
more than two particles move in continuum states could be neglected.

Our aim in this work is to propose an effective interaction scheme 
which allows for a much larger number of complex continuum states in the calculations, 
and in addition takes into account the mixing of 
configurations where all particles may move in 
complex continuum states. We show that
if one aims at accurate calculations of the multi-particle resonances,
the effect of all particles moving in the continuum may not always be
neglected. Our choice of contour
allows for a perturbative treatment of the many-particle resonances, 
and we propose a perturbation theory based scheme which combines 
the Lee-Suzuki similarity transformation  
method \cite{suzuki1,suzuki2,suzuki3,suzuki4}
and the so-called  multi-reference perturbation method \cite{multi1,multi2,multi3}
to account for couplings
with configurations where all single-particles move in complex continuum states.

Presently, Gamow shell model calculations have been performed with phenomenological 
nucleon-nucleon interactions. A major challenge is to construct
effective nucleon-nucleon interactions for drip-line nuclei starting from 
a realistic nucleon-nucleon interaction. 
In this paper we focus on the choice of contour and the dimensionality
problem. The effective nucleon-nucleon interaction adopted is purely
phenomenological. 
However, the scheme we present, although implemented with a phenomenological 
nucleon-nucleon interaction, allows to define effective interactions computed with the 
complex scaled single-particle basis.
The problem of constructing an effective interaction based on present interaction models
for the nucleon-nucleon force will be considered in a forthcoming work.

The outline of this work is as follows. Sec.~\ref{sec:formalism} gives a brief
description of the contour deformation method in momentum space, 
and presents calculations
of the energy spectrum of the nuclei $^{5,6,7}$He. 
Sec.~\ref{sec:suzuki} presents first the 
Lee-Suzuki transformation method generalized to complex interactions. Thereafter we
apply the similarity transformation method to the unbound nucleus $^7$He and give a 
convergence study of the $J^\pi=3/2^-_{1}$ resonance, the ground state of $^He$. 
Sec.~\ref{sec:multi} gives a brief
outline of the multi-reference perturbation method, 
and its application to this state. In Sec.~\ref{sec:scheme} we present 
an effective interaction scheme, which combines 
the Lee-Suzuki similarity transformation and the multi-reference perturbation method, 
for calculation of 
multi-particle resonances in weakly bound nuclei. Sec.~\ref{sec:sec6} contains an
application of our truncation method to the $^7$He case.
Sec.~\ref{sec:conclusion} 
gives the conclusions of the present study and 
future perspectives and challenges for 
Gamow shell model calculations.


