\chapter{}
\section{Left and right eigenvectors and bi-orthogonal sets.}
\label{sec:biort}
Given the eigenvalue equation, 
\begin{equation}
  A^{\mathrm{T}} y = \lambda y, 
\end{equation}
where the $n\times n$ matrix $A^{\mathrm{T}}$ is of general form. The eigenvalues $\lambda $ are 
determined by the characteristic equation,
\begin{equation}
  \mathrm{det}(A^{\mathrm{T}}-\lambda I) = \mathrm{det}(A-\lambda I) = 0,
\end{equation}
which shows that the eigenvalues of $A^{\mathrm{T}}$ are the same
as those of $A$. Consider the eigenvalue equation for the $i$'th 
eigenvector, 
\begin{equation}
  A^{\mathrm{T}} y_i = \lambda_i y_i, 
\end{equation}
then the transpose of this equation gives,
\begin{equation}
  y_i^{\mathrm{T}}A  = \lambda_i y_i^{\mathrm{T}}, 
  \label{eq:left}
\end{equation}
here $y_i^{\mathrm{T}}$ is the left eigenvector of $A$ corresponding
to the eigenvalue $\lambda_i$. The eigenvalue equation for $A$ is 
\begin{equation}
  A x_j = \lambda_j x_j, 
  \label{eq:right}
\end{equation}
where it is seen that $x_j$ is the right eigenvector of the matrix $A$ 
corresponding to the eigenvalue $\lambda_j$.
Now multiply Eq.~(\ref{eq:left}) with $x_j$ from the right and
Eq.~(\ref{eq:right}) with $y_i^{\mathrm{T}}$ from the left, and subtract to obtain,
\begin{eqnarray}
  \lambda_j y_i^{\mathrm{T}}x_j & = & \lambda_i y_i^{\mathrm{T}}x_j, \\
  \Rightarrow \: (\lambda_j -\lambda_i) y_i^{\mathrm{T}}x_j & = & 0, \\
  \Rightarrow \: y_i^{\mathrm{T}}x_j & = & 0 \:\: \mathrm{if} \lambda_i \ne \lambda_j, 
\end{eqnarray}
where it is customary to say that the 
left $y_i^{\mathrm{T}}$  and right $x_j$ eigenvectors of a 
matrix $A$ 
are \emph{bi-orthogonal} to each other.
If all $n$  
eigenvalues of the matrix $A$�are distinct, then
$y_i^{\mathrm{T}}x_j = 0 $ for $i,j = 1,2,...,n, \: i \ne j$, but 
$y_i^{\mathrm{T}}x_i \ne 0 $.  This implies that the right and left 
eigenvectors can be scaled so they form a complete set of \emph{bi-orthogonal} 
vectors.
\begin{equation}
  Y^\mathrm{T} X = \sum_{i,j=1}^n y_i^\mathrm{T}x_i = \sum_{i,j=1}^n\delta_{i,j} = 1.
\end{equation}
Here $Y^\mathrm{T}$ is a matrix whose rows are $y_i^\mathrm{T}, i=1,...,n$�and
$X$�is a matrix whose columns are $x_i, i = 1,...,n$.
This shows explicitly that $Y^\mathrm{T}$ is the inverse of $X$. 
It is clear that this works for any matrix which has a complete 
set of linearly independent eigenvectors (a nondefective matrix) 
regardless of whether the eigenvalues are distinct. 
As we have seen before, we can write in this case
\begin{displaymath}X^{-1}AX=\, {\rm diag}\, [\lambda _i]. \end{displaymath}
Indeed, if we can find any matrix $Z$ such that $Z^{-1}AZ$ is diagonal, 
then the columns of $Z$ are the right eigenvectors of $A$, and the rows of $Z^{-1}$ 
are the left eigenvectors of $A$, 
while the diagonal entries of $Z^{-1}AZ$ are the eigenvalues of $A$.
Defective matrices have an incomplete set of eigenvectors, 
and the theory requires their reduction to Jordan normal form. 

In the special case of $A$ being a complex symmetric matrix, which is
often the case in physical applications, then the left eigenvectors
are just the transpose of the right eigenvectors. In this case it is sufficient to
solve the right eigenvalue equation, and a complete set of \emph{bi-orthogonal} vectors
are obtained directly. 


\section{Three-body matrix elements in $j-j$ coupling} 
\label{sec:3matel}
Throughout this section the anti-symmetric three-, two-body 
wave functions are written as $\bar{\Psi}�$ and $\bar{\Phi}$�
respectively, while the non-anti-symmetrized functions are without 
the bar. The single-particle wave functions are given by $\phi $. 
The anti-symmetric three-body wave function may be written 
\begin{equation}
  \bar{\Psi}_{(ab)c}^{JM}(123) = {1\over \sqrt{3}}\left\{ \Psi_{(ab)c}^{JM}(123) 
  - \Psi_{(ab)c}^{JM}(132) + \Psi_{(ab)c}^{JM}(231) \right\} 
  \label{eq:threebody}
\end{equation}
here $(ab)c$ labels all relevant single particle quantum numbers  
$a = {n_a, l_a,j_a}$, and the coupling rule $\left( j_a\otimes j_b\right) _{J_{ab}}\otimes j_c$  is 
indicated.
The wave function in Eq.~\ref{eq:threebody} is 
anti-symmetric only in case where at least two of the orbits $abc$ are 
different. In the case $a=b=c$ one has to make use of  coefficients of 
fractional parentage to make the three-body wave function anti-symmetric.
 The non anti-symmetrized wave functions 
$ \Psi_{(ab)c}^{JM}(123) $ are given by 
\begin{equation}
  \Psi_{(ab)c}^{JM}(123) = \Psi\left( \bar{\Phi}_{ab}^{J_{ab}}(12)\phi_c(3); JM \right)
\end{equation}
Here $ \bar{\Phi}_{ab}^{J_{ab}}(12) $ is an anti-symmetric two-particle wave function.
\begin{equation}
  \bar{\Phi}_{ab}^{J_{ab}}(12) = {1\over \sqrt{ 2 (1+\delta_{ab})}}\sum_{m_a,m_b} 
  \langle j_a m_a, j_b m_b \vert J_{ab} M_{ab}\rangle \left( \phi_a(1) \phi_b(2) - \phi_a(2)\phi_b(1)\right)
\end{equation}
In the following derivation the total spin and projection $JM$ will 
be suppressed for notational economy. 

Consider a matrix element of the three-body wave function in Eq.~\ref{eq:threebody}
with a general interaction consisting of only two-body terms $V= V_{12} +V_{13}+V_{23}$.
\begin{eqnarray}
  \langle \bar{\Psi}_{(ab)c}(123) \vert V \vert \bar{\Psi}_{(de)f}(123)\rangle = \\
	  {1\over \sqrt{3}}\langle \Psi_{(ab)c}(123) 
	  - \Psi_{(ab)c}(132) + \Psi_{(ab)c}(231) \vert V \vert \bar{\Psi}_{(de)f}(123)\rangle
\end{eqnarray}
from the anti-symmetry follows
\begin{eqnarray}
  \langle {\Psi}_{(ab)c}(123) \vert V  \vert \bar{\Psi}_{(de)f}(123)\rangle & = &  
  \langle -{\Psi}_{(ab)c}(131) \vert V\vert \bar{\Psi}_{(de)f}(123)\rangle  \\
 & = & \langle {\Psi}_{(ab)c}(231) \vert V\vert \bar{\Psi}_{(de)f}(123)\rangle 
\end{eqnarray} 
and henceforth
\begin{eqnarray}
  \langle \bar{\Psi}_{(ab)c}(123) \vert V \vert \bar{\Psi}_{(de)f}(123)\rangle  =
  \sqrt{3}\langle {\Psi}_{(ab)c}(123) \vert V \vert \bar{\Psi}_{(de)f}(123)\rangle = \\
  \langle {\Psi}_{(ab)c}(123) \vert V_{12}\vert \Psi_{(de)f}(123)-\Psi_{(de)f}(132)+\Psi_{(de)f}(231)\rangle + \\
  2\langle {\Psi}_{(ab)c}(123)\vert V_{23}\vert \Psi_{(de)f}(123)-\Psi_{(de)f}(132)+\Psi_{(de)f}(231)\rangle 
  \label{eq:matel1}
\end{eqnarray}
Starting with the matrix element of $V_{12}$, one has
\begin{equation}
  \langle {\Psi}_{(ab)c}(123) \vert V_{12} \vert \Psi_{(de)f}(123)-\Psi_{(de)f}(132)+\Psi_{(de)f}(231)\rangle =
  V_{12}^1 + V_{12}^2 + V_{12}^3
  \label{eq:v12}
\end{equation}
where
\begin{equation}
  V_{12}^1 = \langle ab \vert V_{12} \vert  de \rangle^{AS}_{J_{ab}} \:\delta_{c,f}\:\delta_{J_{ab},J_{de}}
\end{equation}
and
\begin{equation}
  V_{12}^2+V_{12}^3 = \langle {\Psi}_{(ab)c}(123) \vert V_{12} \vert 
  -\left( \Psi_{(de)f}(132) - \Psi_{(de)f}(231) \right) \rangle
\end{equation}
recoupling $1,3 \rightarrow 1,2 $ in  $ \Psi_{(de)f}(132) $ and 
$ 2,3 \rightarrow 2,1 $ in $ \Psi_{(de)f}(231)  $ one may show by 
angular momentum algebra that 
\begin{eqnarray}
 -  \left( \Psi_{(de)f}(132) - \Psi_{(de)f}(231) \right) = \\
 \left( {1+\delta_{d,f} \over 1 + \delta_{d,e}} \right)^{1/2} \sum_{J_{df}} 
  (-1)^{j_d + J_{df}-J_{de}-J}U(j_e\:j_d\:J\:j_f; J_{de}\:J_{df})\Psi_{(df)e}(123) \\
  -\left( {1+\delta_{e,f} \over 1 + \delta_{d,e}} \right)^{1/2} \sum_{J_{ef}} 
  (-1)^{j_d + J_{ef}-J}U(j_d\:j_e\:J\:j_f; J_{de}\:J_{ef})\Psi_{(ef)d}(123) 
\end{eqnarray}
here $U(j_a\:j_b\:J\:j_c; J_{ab}\:J_{bc}) $ are the normalized Racah coefficients. 
It follows that the terms $V_{12}^2 $ and $V_{12}^3$ are given by
\begin{eqnarray}
  \nonumber
  V_{12}^2 = \left( {1+\delta_{d,f} \over 1 + \delta_{d,e}} \right)^{1/2} \sum_{J_{df}} 
  (-1)^{j_d + J_{df}-J_{de}-J}U(j_e\:j_d\:J\:j_f; J_{de}\:J_{df}) \times\\
  \langle ab\vert V_{12}\vert df\rangle_{J_{ab}}^{AS}\:\delta_{c,e}\delta_{J_{ab},J_{df}} \\
  \nonumber
  V_{12}^3 = - \left( {1+\delta_{e,f} \over 1 + \delta_{d,e}} \right)^{1/2} \sum_{J_{ef}} 
  (-1)^{j_d + J_{ef}-J}U(j_d\:j_e\:J\:j_f; J_{de}\:J_{df}) \times \\
  \nonumber
  \langle ab\vert V_{12}\vert ef\rangle_{J_{ab}}^{AS}\:\delta_{c,d}\delta_{J_{ab},J_{ef}} \\
\end{eqnarray}
Calculating the matrix element of $V_{23}$ one first recouple $1,2 \rightarrow 2,3$ in
the $\langle \mathrm{bra} \vert$.
\begin{eqnarray}
  \Psi_{(ab)c}(123) = \left( { 1\over 2(1+\delta_{a,b})} \right) \times \\
  \left\{ 
  \sum_{J_{bc}}(-1)^{j_a+J_{bc}-J}U(j_aj_bJj_c; J_{ab}J_{bc}) 
  \Psi\left( \Phi_{bc}^{J_{bc}}(23)\phi_a(1) \right) + \right. \\
    \left. \sum_{J_{ac}}(-1)^{j_a-J_{ab}+J_{ac}-J} U(j_b j_a J j_c; J_{ab} J_{ac}) 
  \Psi\left( \Phi_{ac}^{J_{ac}}(23)\phi_b(1)\right) \right\}
\end{eqnarray}
here $ \Phi_{bc}^{J_{bc}} $ and $ \Phi_{ac}^{J_{ac}} $ are non anti-symmetric two-body wave functions.
Anti-symmetric two-body matrix elements may be expressed in terms non anti-symmetric 
matrix elements by
\begin{equation}
\langle\bar{\Phi}_{ab}(12) \vert V_{12} \vert \bar{\Phi}_{cd}(12) \rangle = 
\left( {2\over (1+\delta_{ab})} \right)^{1/2}
\langle\Phi_{ab}(12) \vert V_{12} \vert \bar{\Phi}_{cd}(12) \rangle 
\end{equation}
Evaluating the matrix element of $V_{23}$ one has to evaluate the following 
matrix elements
\begin{equation}
  \langle \Psi_{(bc)a}(231) \vert V_{23} \vert \Psi_{(de)f}(123)-\Psi_{(de)f}(132)+\Psi_{(de)f}(231)\rangle
\end{equation}
and  
\begin{equation}
  \langle \Psi_{(ac)b}(231) \vert V_{23} \vert \Psi_{(de)f}(123)-\Psi_{(de)f}(132)+\Psi_{(de)f}(231)\rangle
\end{equation}
which are evaluated in the same manner as the evaluation of $V_{12}$ in Eq.~\ref{eq:v12}. 
After some angular momentum recoupling algebra in the $\vert \mathrm{ket}\rangle $, one 
ends up with the final expressions
\begin{eqnarray}
  \nonumber
  \langle \bar{\Psi}_{(ab)c}(123) \vert V \vert \bar{\Psi}_{(de)f}(123)\rangle  =
  \nonumber
  \langle ab \vert V_{12} \vert  de \rangle^{AS}_{J_{ab}} \:\delta_{c,f}\:\delta_{J_{ab},J_{de}} + \\
  \nonumber 
  \left( {1+\delta_{d,f} \over 1 + \delta_{d,e}} \right)^{1/2} 
  (-1)^{j_d - J_{de}+J_{ab}-J}U(j_ej_d J j_f; J_{de} J_{ab}) 
  \langle ab\vert v_{}\vert df\rangle_{J_{ab}}^{AS}\:\delta_{c,e} - \\
  \nonumber
  \left( {1+\delta_{e,f} \over 1 + \delta_{d,e}} \right)^{1/2} 
  (-1)^{j_d + J_{ab}-J}U(j_d j_e J j_f; J_{de} J_{ab}) 
  \langle ab\vert v_{}\vert ef\rangle_{J_{ab}}^{AS}\:\delta_{c,d} + \\
  \nonumber
  \left( { 1+\delta_{b,c} \over 1+\delta_{a,b} } \right)^{1/2}(-1)^{j_a+J_{de}-J}
  U(j_a j_b J j_c; J_{ab} J_{de})   
  \langle  bc \vert v_{} \vert de\rangle_{J_{de}}^{AS}\delta_{a,f} +  \\
  \nonumber
  \left( {1+\delta_{b,c}\over 1+\delta_{a,b} } \right)^{1/2}\sum_{J_{bc}}(-1)^{j_a+j_d-2J} 
  U(j_a j_b J j_c; J_{ab} J_{bc})   \times \\ 
  \nonumber
  \left\{ \left( {1+\delta_{d,f}\over 1+\delta_{d,e} } \right)^{1/2}(-1)^{J_{de}}
  U(j_e j_d J j_f; J_{de} J_{bc}) \langle bc \vert v_{} \vert df \rangle_{J_{bc}}^{AS}\delta_{a,e} + \right. \\ 
  \nonumber
  \left.
  \left( {1+\delta_{e,f}\over 1+\delta_{d,e}} \right)^{1/2}
  U(j_d j_e J j_f; J_{de} J_{bc}) \langle bc \vert v_{} \vert ef \rangle_{J_{bc}}^{AS}\delta_{a,d} \right\}
  + \\ 
  \nonumber
  \left( {1+\delta_{a,c}\over 1+\delta_{a,b}} \right)^{1/2}(-1)^{j_a-J_{ab}+J_{de}-J}
  U(j_b j_a J j_c; J_{ab} J_{de})   
  \langle  ac \vert v_{} \vert de\rangle_{J_{de}}^{AS}\delta_{b,f} +  \\
  \nonumber
  \left( {1+\delta_{a,c}\over 1+\delta_{a,b}} \right)^{1/2}\sum_{J_{ac}}(-1)^{j_a+j_d-J_{ab}-2J} 
  U(j_b j_a J j_c; J_{ab} J_{ac})   \times \\ 
  \nonumber
  \left\{ \left( { 1+\delta_{d,f}\over 1+\delta_{d,e}} \right)^{1/2}(-1)^{J_{de}}
  U(j_e j_d J j_f; J_{de} J_{ac}) \langle ac \vert v_{} \vert df \rangle_{J_{ac}}^{AS}\delta_{b,e} + \right. \\ 
  \left.
  \left( {1+\delta_{e,f}\over 1+\delta_{d,e}} \right)^{1/2}
  U(j_d j_e J j_f; J_{de} J_{ac}) \langle ac \vert v_{} \vert ef \rangle_{J_{ac}}^{AS}\delta_{b,d} \right\}
  \label{eq:matel2}
\end{eqnarray} 
In the case $ a = b = d = e \neq c = f $ and $ J_{ab} = J_{aa}, J_{de}={J'}_{aa}$
are even, Eq.~\ref{eq:matel2} simplifies to 
\begin{eqnarray}
  \nonumber
  \langle \bar{\Psi}_{(aa)c}(123) \vert V \vert \bar{\Psi}_{(aa)c}(123)\rangle  =
  \nonumber
  \langle aa \vert v \vert  aa \rangle^{AS}_{J_{aa}} \:\delta_{J_{aa},{J'}_{aa}} + \\
  \nonumber
  2\sum_{J_{ac}}(2J_{ac}+1) \sqrt{(2J_{aa}+1)(2{J'}_{aa} +1)} 
  \left\{ \begin{array}{ccc}
    j_a & j_a & J_{aa} \\
    j_c & J & J_{ac}�
  \end{array}\right\}
  \left\{ \begin{array}{ccc}
     j_a & j_a & {J'}_{aa} \\
    j_c & J & J_{ac}�
  \end{array}\right\}
  \langle ac \vert v \vert ac\rangle_{J_{ac}} 
\end{eqnarray}
where the normalized Racah coefficients are expressed in terms of $6-j$ symbols.
Next consider the case where all the single particle orbits in the ket are equivalent, 
i.e. $ d = e = f$,  in this case one has to make a coefficients of fractional parentage
expansion to make the three-body wave function totally anti-symmetric in the $j-j$ coupling
scheme;
\begin{equation}
  \bar{\Psi}_{ddd}(123) = \sum_K  \left. \langle j_d^2 K, j_d\vert \right\} j_d^3 J\rangle 
  \Psi_{(dd)d}(123)
\end{equation}
In this way the wave function is expressed in terms of anti-symmetric two-particle 
wave functions, and one may proceed in the same manner as for the case considered above. 
After some angular momentum recouplings, one ends up with the final 
expression for the matrix element,
\begin{eqnarray}
  \nonumber
  \langle \bar{\Psi}_{(ab)c}(123) \vert V \vert \bar{\Psi}_{(dd)d}(123)\rangle  =
  \nonumber
  \sqrt{3} \left. \langle j_d^2 J_{ab}, j_d\vert \right\} j_d^3 J\rangle  
  \langle ab \vert v \vert  dd \rangle^{AS}_{J_{ab}} \delta_{c,d} + \\
  \nonumber
  \sqrt{3}(-1)^{j_a+j_d-2J}
  \sum_K \left. \langle j_d^2 K, j_d\vert \right\} j_d^3 J\rangle \times \\
  \nonumber
  \left\{ \left( {1+\delta_{b,c}\over 1+\delta_{a,b}} \right)^{1/2} 
  \sum_{J_{bc}} U(j_a j_b J j_c; J_{ab} J_{bc})U(j_d j_dJ j_d; K J_{bc}) 
  \langle bc\vert v \vert dd\rangle_{J_{bc}}\delta_{a,d} \right. + \\
  \left. \left( {1+\delta_{a,c}\over 1+\delta_{a,b}} \right)^{1/2} 
  \sum_{J_{ac}} (-1)^{J_{ab}} U(j_b j_a J j_c; J_{ab} J_{ac})U(j_d j_dJ j_d; K J_{ac}) 
  \langle ac\vert v \vert dd\rangle_{J_{bc}}\delta_{b,d} \right\}
\end{eqnarray}


