\chapter{Summary and perspectives}
\label{chap:conclusions}
The main purpose of this thesis has been to study and develop methods 
suitable for study of resonance phenomena in nuclear and subatomic physics. 
Our emphasis has been on the momentum space formulation of the Schr\"odinger
equation. It has been shown, starting with 
the integral formulation of the Schr\"odinger equation, that an efficient way 
of obtaining a complete set of states including bound-, anti-bound and
resonant states is through the Contour Deformation Method. The 
strength of the Contour Deformation Method has been 
illustrated by studying a wide range of different cases in 
subatomic physics where resonance phenomena appear. These
applications ranges from the case of a single particle moving 
in a spherically symmetric field to the case of strong deformations
of the field. Further, it has been studied how resonances may be solved for
in complex potentials which models absorbtive and emittive 
processes, using the Contour Deformation Method. 
The results obtained in these specific applications, strongly favour the 
Contour Deformation Method in comparison with other methods
such as complex coordinate scaling and analytic continuation in the coupling
strength. The most appealing feature of CDM is that, not only does  
it give accurate results for resonances and anti-bound states, but 
in addition it provides us with a complete set of states 
which may be used in many different eigenfunction expansions. 
The only limitation of CDM is that the analytic structure of the
potential has to be known, since the choice of contour has to 
be dictated by the singularity structure of the potential. The 
revival and study of CDM applied to nuclear physics, may 
be considered the main issue of the first part of this thesis, and 
is also the topic of Paper I. 

In the second part of this thesis, the focus is directed towards
the issue of how resonance phenomena may be understood in nuclei, 
where several valence particles are present. The newly developed
Gamow Shell Model is a promising approach in the study of loosely bound and 
unbound nuclei along the drip lines. The main ingredient in the Gamow Shell Model 
is the construction of a complete set of many-body Slater determinants 
built up from a single particle Berggren basis. It has been shown in this
work that a viable starting point in Gamow Shell Model studies is 
to obtain a single particle basis by the Contour Deformation 
Method in momentum space. The results displayed in Paper II, 
indicate rapid convergence for many-body resonances using 
a single particle basis in momentum space.

The challenge for present and future Gamow Shell Model calculations is
how to deal with the extreme growth of the number of Slater determinants
in the many-body expansion basis. This topic was the main issue of the second
part of the thesis. The basic idea was to modify standard effective interaction theory
and many-body perturbation theory, so that that their range of applicability 
encompass the complex interactions and matrices which follows from the 
generalization of the standard Shell Model to the complex energy plane.
Further, the extreme dimension of the Shell Model Hamiltonian matrix
requires development of large-scale matrix diagonalization 
routines which can handle both real and complex matrices. In this
thesis it was shown how the Lanczos iteration method may be generalized
to  complex energy matrices. It was further shown, that by choosing a 
reasonable initial Lanczos vector for the 0th order 
multi-particle resonance, the multi-particle resonance may be 
unambigously picked out from the set of states obtained from
diagonalization at each iteration, by identifying the state which 
has the largest overlap with the 0th order Lanczos vector. 

Another important result was the generalization of the Lee-Suzuki similarity 
transformation to include complex interactions.
The emphasis was on the derivation of effective interactions for 
for loosely bound or unbound nuclei which has a strong coupling with 
the continuum. We demonstrated by a numerical study in Paper II, 
that the construction of an  effective two-body interaction 
based on the Lee-Suzuki similarity transformation method, leads to a drastic
reduction of the Gamow shell-model dimensionality for more than two particles.
Furthermore it was shown in Paper II 
that the one-state-at-a-time Multi-Reference-Perturbation-Theory 
combined with the construction of an effective two-body interaction, 
reduces drastically the dimension of Shell Model space.
This result is very promising 
when extending the Gamow shell model to applications in structure 
calculations of heavier dripline nuclei, 
with a larger number of valence particles moving in a large valence space. 


With further progress in computational power   
one may hope that \emph{ab intio}  calculations of light and medium 
size nuclei within the Berggren representation may become possible in the near future. 
Coupled-Cluster techniques has proven to be a promising method for 
calculations of medium size nuclei. 
Very recently \cite{cc1,cc2,cc3}, converged Coupled-Cluster 
results for the ground- and first excited state of $^{16}$O where reported, using 
modern nucleon-nucleon interactions derived from effective field-theory.
A promising way of approach would be 
to generalize the Coupled-Cluster method to complex interactions, and
at the first stage see how resonant structures are formed in light nuclei
starting from an  \emph{ab initio} approach. 
As this thesis only deals with a phenomenological residual nucleon-nucleon 
interaction, the next step is to include a realistic and microscopically derived 
effective nucleon-nucleon interaction in Gamow Shell Model calculations. 
Constructing a single particle Berggren basis by solving the Hartree-Fock
equation self-consistently with an effective nucleon-nucleon interaction 
constructed from the G-matrix approach or with the recently
developed low-momentum nucleon-nucleon interaction ($V_{\mathrm{low-k}}$ ),
and then calculating matrix elements of the effective interaction in
this basis is a future challenge for the Gamow Shell Model.
How single particle resonances are formed from the underlying 
nucleon-nucleon interaction is a very interesting study in itself, and
work along these lines are in progress using a renormalized nucleon-nucleon  
interaction of the $V_{\mathrm{low-k}}$   type, generalized to the complex 
$k$-plane.


