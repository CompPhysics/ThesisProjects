\chapter{Paper2}
\section{Introduction to Paper 2} 
Paper II deals with the recently developed Gamow Shell Model. 
Constructing a single particle Berggren basis in momentum 
space, generated from the Sack-Biedenharn and Breit (SBB) potential 
for $^5$He, a complete anti-symmetric two- and three particle 
basis is constructed. Limiting the discussion to $p_{1/2}$ and 
$p_{3/2}$ single particle motion, the energy spectrum of $^6$He 
and $^7$He is solved for, using a phenomenological nucleon-nucleon
interaction of a Gaussian type. In using CDM in constructing the 
single particle basis, it is shown that convergence of the energy 
spectrum of $^6$He with increasing number of non-resonant continuum 
orbitals is rather fast. However the dimension of the many-body space
for nuclei with a larger number of valence particles increases 
extemely fast, and direct diagonalization methods are no longer possible. 
This paper deals primarily with this \emph{dimensionality} problem. 
It is shown how the Lee-Suzuki similarity transformation method 
may be generalized to complex interactions. Constructing a
two-body effective interaction in a reduced space, 
which exactly reproduce a limited set
of eigen values of the full Hamiltonian, is used in the calculation 
of the resonant energy spectrum of $^7$He. It is shown that
the convergence using the similarity transformed interaction is
appreciably faster than compared with the bare interaction as
the model space is increased. Further, we discuss how a 
Multi-Reference-Perturbation-Theory-Method (MRPTM), which 
differs from standard MRPTM in that it is a one-state-at-a-time 
perturbation theory, may be applied to Gamow-Shell-Model calculations. 
It is shown that to second order, MRPTM  gives
satisfactory converged results for $^7$He, and reducing the dimension of the
full problem to $8-10\%$. Finally an effective interaction scheme
for the Gamow-Shell-Model is discussed, which combines the 
Lee-Suzuki similarity transformation method with the one-state-at-a-time
MRPTM. And converged results for the resonant spectrum of $^7$He 
is shown using a model space consisting of both $p_{1/2}$ and 
$p_{3/2}$ single particle orbitals. The dimension is in the
most severe case, of the state $J^\pi = {3/2}^-�$, reduced from 
$\approx 40000$ to $ \approx 1600$.  This is a promising 
result, and which may allow for Gamow-Shell-Model studies of nuclei 
consisting of a larger number of valence particles moving in a large
valence space.


\newpage
\section{ \it Effective Interaction Techniques for the Gamow Shell Model}
\label{sec:paper2}
\vskip 2.cm
{\Large G.~Hagen, M.~Hjorth-Jensen and J.~S.~Vaagen  \\[0.5cm] 
Accepted in Phys. Rev. C} 
