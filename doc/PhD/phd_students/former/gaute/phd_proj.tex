\documentstyle[a4wide,12pt,norsk]{article}
\begin{document}
\pagestyle{plain}

\section*{DOKTORGRADSPROSJEKTER
I TEORETISK KJERNEFYSIKK FOR GAUTE HAGEN OG MAXIM KARTAMYSHEV }

\subsection*{Introduksjon og m\aa setting}

Studier av atomkjerners struktur er et viktig forskningsfelt for
forst\aa else av kjernemateriens egenskaper. Omfattende eksperimentelle
data er tilgjengelig. En teoretisk analyse og modell m\aa\  kunne
reprodusere denne eksperimentelle informasjon som et f\o rste steg til
en fysisk forst\aa else av atomkjernen og i neste omgang kjernemateriens mer
generelle egenskaper. Dette danner bakgrunnen for gruppens arbeid med
atomkjernens struktur.

Med utgangspunkt i metoder til beskrivelse av vekselvirkningen
mellom hadroner i en atomkjerne med mange partikler har vi utviklet
simuleringsmodeller basert p\aa\ omfattende datamaskinprogrammer til \aa\
beskrive atomkjerners egenskaper. Vi starter med
Schr\"{o}dingerlikningen for et system av Z protoner og N n\o ytroner
og finner egentilstandene for dette omfattende kvantemekaniske
mange-partikkel system. Dette gj\o r det mulig \aa\ analysere data fra
et verdensomspennende nett av kjernefysiske eksperimentelle laboratorier.

En viktig ingrediens i studier av atomkjerner er en effektiv
vekselvirkning basert p\aa\ mange-legeme teoretiske metoder.
Mange-legemeteori
danner grunnlaget for studier av systemer hvor
flere partikler vekselvirker. Med flere vil vi, siden
det er kjernefysiske systemer vi i all hovedsak vektlegger,
meine mere enn 7-8 partikler.
En slik effektiv vekselvirkning kan deretter brukes i f.eks.~stor-skala
skall modell programmer, slik som f.eks.~det vi har utviklet i Oslo.

Doktorgradsprosjektene til Gaute og Maxim  faller derfor innenfor 
temaet utvikling av mange-legeme teoretiske metoder med hovedvekt p\aa\
utvikling av mange-legeme formalisme innenfor det som kalles
for 
Coupled-Cluster metoden og Parquet-diagram summasjonsmetoden, 
se referanse listen nedenfor.
Anvendelsene er effektive vekselvirkninger for b\aa de tradisjonell 
skall-modellberegninger og for kjerner med svakt bundne tilstander 
med sterk kopling til kontinuum. Metodene kan ogs\aa\ anvendes for studier
av kjernematerie. 

I tillegg er innlemming av tre-partikkel vekselvirkninger basert
p\aa\ nylige parametriseringer av tre-legeme krefter et viktig
element avhandlingene.

M\aa lene for prosjektene, med milep\ae ler er listet opp
i neste avsnitt.

\subsection*{Prosjektbeskrivelse}

Prosjektene har f\o lgende faglige m\aa lsetting

\begin{itemize}

\item {\bf Summasjon av to-legeme diagrammer vha.\ av 
            Coupled Cluster metoden og summasjon av Parquet diagrammer}

Coupled cluster metoden (CCM) har i flere \aa r v\ae rt
en av de ledende metoder i mange-legemeteori med anvendelser 
i b\aa de fysikk 
og kvantekjemi. Den best\aa r av et hierarki av ikke-lin\ae re
likninger som tillatter en \aa\ summere opp
store klasser av mangelegeme diagrammer. 
M\aa lsettinga her \aa\ studere en reduksjon av dette
likningsettet til det som kalles to-legeme CCM, med
anvendelse for valenssystemer med henblikk p\aa\
\aa\ rekne ut effektive vekselvirkninger for 
atomkjerner. Denne effektive vekselvirkning \o nsker vi 
s\aa\ \aa\ anvende i skallmodellberegninger av egenskaper
til ulike atomkjerner.
Vi \o nsker ogs\aa\ \aa\ sammenlikne disse resultater
med de metoder vi har brukt hittil for beregning
av effektive vekselvirkninger og dermed foreta et teoretisk
studium av flere mange-legeme teoretiske metoder. Dette har aldri
blitt gjort f\o r for effektive vekselvirkninger mellom
valenspartikler og kan dermed danne grunnlaget 
for dypere innsikter om mange-legeme teoretiske metoder. 

Den andre metoden vi tenker oss er summasjon av Parquet diagrammer,
basert p\aa\ Green's funksjon formalismen. En kritisk sammenlikning
av disse to metoden har aldri blitt gjort f\o r, ei  heller en fullstendig
studie av disse metodene for kjernefysiske systemer.

Dette delprosjektet krever utvikling av to programmer, et for Coupled-Cluster
metoden og et for summasjon av Parquet diagrammene.

Som basis i dette f\o rste prosjektet tenker vi oss \aa\ bruke en
harmonisk oscillator basis og en s\aa kalt to-legeme vekselvirkning
definert for et stort, men redusert Hilbert rom (G-matrise). Denne 
G-matrisen og parallellisering har Maxim allerede jobbet med.

Som en kort oppsummering, her skal vi teste og utvikle programmer
for to metoder til \aa\ summere opp til uendelig orden en viss
klasse av to-legeme korrelasjoner i en harmonisk oscillator basis.

M\aa lsetting er at dette delprosjektet avsluttes v\aa ren 2002.

\item  {\bf Summasjon av to-legeme Parquet diagrammer med plane b\o lger}

Vi \o nsker deretter \aa\ forandre basis slik at eventuelt plane b\o lger
eller en blanding av plane b\o lger og bundne tilstander kan brukes for \aa\
beskrive et system hvor f.eks. svakt bundne tilstander forekommer.

Parquet-diagram metoden er i prinsippet kompleks, og tillater derfor
en naturlig innlemmelse av bidrag fra tilstander i et eventuelt kontinuum.

M\aa let med dette prosjektet er \aa\ inkludere slike effekter i v\aa r
effektive vekselvirkning for \aa\ kunne studer systemer som
$^{105}$Sb, som er en kjent proton-emitter, eller 
$^{18}$F med sterk kopling til kontinuum.

M\aa lsetting er at dette delprosjektet avsluttes innen utgangen av  2002.

\item {\bf Beregning av tre-legemekorrelasjoner vha.\ av 
            Coupled Cluster metoden}

I kjernefysikk er det generell enighet om at
tre-legemekrefter er viktige for \aa\ kunne reprodusere 
bindingsenergien for kjerner med mere enn 2 nukleoner, slik
som tritium.
Det som ikke er helt klart er hvordan en skal
ta hensyn til slike korreksjoner
i mange-legemeberegninger av effektive vekselvirkninger
for endelige kjerner eller uendelig materie. 

I dette prosjektet \o nsker vi \aa\ studere rollen en bestemt
type effektive tre-legemekorrelasjoner spiller 
i strukturberegninger for atomkjerner. Basert p\aa\ det foreg\aa ende
prosjektet, kan metoden med summasjon av to-legeme diagrammer
utvides til \aa\ inkludere effektive tre-legeme korrelasjoner.  
Basisvekselvirkningen vil v\ae re
en to-legeme nukleon-nukleonvekselvirkning.

Her tenker vi ogs\aa\ \aa\ bake inn tre-legeme krefter.

Som basis her bruker vi  en
harmonisk oscillator basis.
Prosjektet her er en naturlig videref\o ring av det f\o rste.


M\aa lsetting er at dette delprosjektet avsluttes innen utgangen av 2002.


\item {\bf Beregning av tre-legemekorrelasjoner vha.\ av 
            Parquet diagram  metoden}
Som foreg\aa ende prosjekt, men denne gang vha.~Parquet metoden.
M\aa lsetting er at dette delprosjektet avsluttes innen utgangen av 2003.


\item {\bf Studier av kjernematerie og andre fysiske systemer}

Med utvikling av programmer for b\aa de Parquet metoden og Coupled-Cluster
metoden, er en overf\o ring til studier til kjernematerie og n\o ytronstjerne
fysikk en naturlig anvendelse. 

Likes\aa\ er systemer som kvantepunkter, quantum dots, elektroner avgrensa
i sm\aa\ to-dimensjonale omr\aa der i lag mellom halvledere,
et interessant studieomr\aa de for effektive to og tre partikkel 
vekselvirkninger.

M\aa lsetting er at dette delprosjektet avsluttes innen utgangen av 2004.

\end{itemize}



\subsection*{Litteraturliste, Coupled Cluster metoden}

\begin{enumerate}
\item H. K\"ummel, K. H. L\"uhrmann, and J. G. Zabolitzky, Phys. Rep.
36, 1 (1978). {\bf Dette er hovedartikkelen}.
\item H.~M\"uther and A.~Polls, Prog.~Part.~Nucl.~Phys.~{\bf 45}, 243 (2000).
Bra og overskiktlig lesning, se spesielt avsnitt 2.2.
\item Jochen H. Heisenberg and Bogdan Mihaila, 
Phys. Rev. C 59, 1440 (1999)
\item Jochen H. Heisenberg and Bogdan Mihaila, 
Phys. Rev. C 60, 054303 (1999) 
\item R. F. Bishop, Theor. Chim. Acta 80, 95 (1991).
\end{enumerate}

\subsection*{Litteraturliste, Parquet metoden}

\begin{enumerate}
\item {\bf Viktig artikkel} : A.D.\ Jackson, A.\ Lande and R.A.\ Smith, 
Phys.\ Rep.\  {\bf 86} (1982) 55; A.\ Lande and R.A.\ Smith, Phys.\ Rev.\ 
{\bf A45}
(1992) 913 og referanser der.
\item En skisse av metoden : M. Hjorth-Jensen,  nucl-th/9811101
\item {\bf Pedagogisk intro} : 
J.P.\ Blaizot and G.\ Ripka, {\em Quantum theory of finite systems},
(MIT press, Cambridge, USA, 1986), chapter 15.
\item 
N.E.\ Bickers and D.J.\ Scalapino, Ann.\ Phys.\ 
{\bf 193} (1989) 206;
N.E.\ Bickers and S.R.\ White, Phys.\ Rev.\ 
{\bf B43} (1991) 8044; 
N.E.\ Bickers and D.J.\ Scalapino, Phys.\ Rev.\
{\bf B46} (1992) 8050.
\item J.\ Yeo and M.A. Moore, Phys.\ Rev.\ 
{\bf B54} (1996) 4218.

\end{enumerate}


\end{document}


















