\documentstyle[11pt]{article}

\textheight 43\baselineskip
\oddsidemargin -10 pt
\evensidemargin 10 pt
\marginparwidth 1 in
\oddsidemargin 0 in
\marginparwidth 0.75 in
\textwidth 6.375 true in
\raggedbottom

\newcommand{\pref}[1]{(\ref{#1})}

\begin{document}

\noindent
Project description:\\

\noindent
{\Large{\bf Quantum information, formal and physical aspects}}\\
{Dr. scient. program for Eirik Ovrum}\\



\noindent
{\large{\bf Introduction}}


\noindent
Quantum information and quantum computing are presently fields of intense
research activity. The interest is partly due to the expectations
that quantum physics opens new and qualitatively different
possiblities of technological development. But the interest
is also due to the understanding that in this field there lies a source of new
and better insight into basic questions in the interpretation of quantum
physics.

The aim of the present doctoral program is to study several aspects of quantum
information, both from the  formal side,
where algorithms for simulations of quantum
systems will be studied, and from the physical side where the application of
concepts from quantum information theory in the study of many-body systems
will be studied. The first part follows up the Master's thesis of Eirik
Ovrum on quantum computation. The second part will focus in particular on
physical aspects of {\em quantum entanglement}. Of particular interest is
to study the relative entanglement entropy in  many-body models.
\\

\noindent
{\large{\bf Research group, background for the program}}

\noindent
The planned research work will be performed at the Centre of
Mathematical Applications and the Department
of Physics with Prof.{\em Morten Hjorth-Jensen} and Prof. {\em Jon
Magne Leinaas} 
as supervisors. (They have both acted
as research
advisors for Eirik under his Master's thesis work.) Although the local
interest in the field of quantum information is fairly new, there has in recent
years been held lecture series and seminars on the subject with several  of the
faculty present in addition to a large number of students taking actively
part. It is of
great interest to follow up this activity and to involve students at
the doctoral
level. We will seek contact with others that are active in the
field, in particular Prof. {\em Carsten Lutken} and Prof.{\em Yuri
Galperin}. In
Trondheim the contact with Prof. {\em Jan Myrheim} will be of importance.\\

\noindent
{\large{\bf Research program}}\\
Eirik Ovrum has written a cand.~scient. (Master's degree)
thesis on {\em Quantum Computing and
Many-Body  Physics}. The thesis describes an algorithm for finding energy
eigenvalues of a quantum system on a quantum computer, and it shows
the result of
simulating the (quantum) calculation of the energy levels of a spin chain. The
work gives a good starting point for further research in the field.  Below we
list some main subjects that the research program will focus on.\\

\noindent
{\em Algorithms for quantum computing}\\
The aim of this part of the program is to extend the study of the algorithm
introduced in  \cite{eirik}, where the method of finding the energy
eigenvalues of a
1-dimensional Heisenberg spin model has been studied. It is of interest to
generalize the method to other models and to perform a simulation of the
algorithm along the lines done in the thesis. In this simulation the
reduction to
single qubit and two-qubit operations are performed and also a
simulation of the
quantum measurements.
Systems of interest range from simple pairing Hamiltonians with
e.g., doubly degenerate single-particle levels to more realistic shell-model
Hamiltonians in atomic or nuclear physics. The aim is to study alternative
algorithms based on ideas from quantum computing and compare them with
more traditional approaches, such as the shell-model or Monte-Carlo
methods

It is also of interest to compare the efficiency of the
performance of the quantum and the classical computer, where the computational
time on the quantum computer is found by counting the number of elementary
operations. To study the scaling behaviour it is of interest to perform the
simulation for a somewhat larger number of spins than is included in
the thesis.\\

\noindent
{\em Entanglement in spin systems}\\
The study of quantum entanglement in many-body systems is a
relatively unexplored
subject, although recently some interesting papers have appeared
\cite{Osborne, Vidal, Latorre}. The main idea is that entanglement is
important,
not only in the original context of quantum communication, but also
for the study
of general behaviour of many-particle systems, like scaling and phase
transitions.
The possible connection to concepts of entropy in quantum field
theories and black
hole physics have been pointed out. It is of great interest to examine these
questions further, even if a clear understanding of the role of entanglement is
still lacking. The approach will be, like in the papers sited, to do numerical
studies of different aspects of entanglement, like block spin entanglement and
entanglement correlations, in various spin models in one (and two)
dimensions.\\

\noindent
{\em Field theory and entropy}\\
There have been several studies of so-called geometrical entropy in field
theories, where special emphasis has been put on the scaling behavior
\cite{Srednicki, Holzhey}. Typically, the geometrical entropy scales with the
surface area rather than the volume. This result seems consistent
with that of the
one-dimensional spin systems referred to above and confirms the idea that
geometrical entropy and entropy of entanglement are closely related.
The entropy
introduced by renormalization, where high energy degrees of freedom are
"integrated out" is however less well understood. The purpose of this
part of the
program is to study different types of entanglement entropy in field
theories, and
in particular examine the behaviour of the entropy introduced by removing high
energy degrees of freedom. A model with coupled harmonic oscillators
will be used,
where the effect of removing high energy degrees of freedom is compared with
results from exact diagonalization.


\begin{thebibliography}{99}


\bibitem{eirik} E. Ovrum, {\em Quantum Computing and Many-Body
Physics}, Thesis
for the Cand. Scient degree, Department of Physics, University of Oslo 2003,
www.fys.uio.no/thesis.ps

\bibitem{Osborne}
T.J. Osborne and M.A. Nielsen, {\em Entanglement, quantum phase transition and
density matrix renormalization}, preprint quant-ph/0109024 (2001).

\bibitem{Vidal}
G. Vidal, J.I. Latorre, E. Rico and A. Kitaev, {\em Entanglement in quantum
critical phenomena}, preprint quant-ph/0211074 (2002).

\bibitem{Latorre}
J.I. Latorre, E. Rico and G. Vidal, {\em Ground state entanglement in quantum
spin chains}, preprint quant-ph/0304098 (2003).

\bibitem{Srednicki}
M. Srednicki, {\em }, Phys. Rev. Lett. {\bf 71} (1993) 666.

\bibitem{Holzhey}
C. Holzhey, F. Larsen and F. Wilczek, {\em }, Nucl. Phys.B {\bf 424}
(1994) 443.


\end{thebibliography}

\end{document}
