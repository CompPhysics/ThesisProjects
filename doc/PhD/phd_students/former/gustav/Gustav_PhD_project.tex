\documentclass[prl,preprint]{revtex4}
\usepackage{graphicx,amsmath,amssymb,bm}

\def\be{\begin{equation}}
\def\ee{\end{equation}}
\def\ba{\begin{eqnarray}}
\def\ea{\end{eqnarray}}
\def\bas{\begin{eqnarray*}}
\def\eas{\end{eqnarray*}}


\newcommand{\la}{\Lambda}
\newcommand{\vlowk}{V_{{\rm low}\,k}}
\newcommand{\fmi}{\, \text{fm}^{-1}}
\newcommand{\mev}{\, \text{MeV}}
\newcommand{\kev}{\, \text{keV}}
\newcommand{\hw}{\hbar \omega}

\begin{document}

\title{Thesis project for PhD student Gustav Jansen: Spherical Coupled-Cluster theory for open-shell systems}

\author{Gustav Ragnar Jansen, g.r.jansen@fys.uio.no}
\affiliation{Department of Physics and Center of Mathematics for  
Applications, University of Oslo, N-0316 Oslo, Norway}
\author{Advisor I: Morten Hjorth-Jensen, mhjensen@fys.uio.no}
\affiliation{Department of Physics and Center of Mathematics for  
Applications, University of Oslo, N-0316 Oslo, Norway}
\author{Advisor II: G.~Hagen, hageng@ornl.gov} 
\affiliation{Physics Division, Oak Ridge National
Laboratory, Oak Ridge, TN 37831, USA} 

\maketitle


{\bf Project summary:}
In the last few years Coupled-Cluster theory has seen a revival in 
the nuclear structure community. Up until recently Coupled-Cluster 
theory was implemented in an uncoupled scheme (m-scheme). 
Although simple in its form, the m-scheme representation puts
constraints on the size of the model space
and the number particles considered. Recently, Coupled-Cluster with Singles- and Doubles approximation (CCSD)
was derived and implemented in a J-coupled scheme. Taking the spherical 
symmetry of closed shell nuclei into account, and realizing that the
Coupled-Cluster Singles- and Doubles similarity transformed
Hamiltonian has at most two-body terms, an efficient spherical CCSD code was implemented. This 
representation reduces the number of non-linear equations and  the computational cost dramatically, 
allowing us to reach into the medium mass region of the nuclear
chart. 
%Due to the computational cost 
%of the m-scheme representation, the interactions had to be softened by renormalization procedures in order
%to achieve convergence within the model-spaces considered. 
Within the spherical scheme, it is now 
possible to reach convergence of medium mass nuclei starting from ``bare'' interactions. 
Single-reference Coupled-Cluster theory works well for nuclei with closed shells such as $^4$He, $^{16}$O
and $^{40}$Ca, see Ref.~[1]. Extending Coupled-Cluster theory to open-shell nuclei, with one or more particles outside
a closed shell is a challenging problem and has been a topic of research for several years. There
exist several, complementary, approaches to open-shell systems. The usual approaches are based on 
multi-reference Coupled-Cluster theory, where the reference state is
built from all possible configurations allowed by symmetry. In addition
to the growth of numerical and mathematical complexity, these approaches often suffer from intruder
states and in some cases the loss of size-extensivity. In extending
Coupled-Cluster theory to open-shell system, one would ideally like to keep
the simplicity of the single-reference Coupled-Cluster theory, and 
consistently build upon the single-reference
Coupled-Cluster theory for closed shell systems. Equation-of-Motion 
Coupled-Cluster (EOM-CC) theory
provides such a consistent framework. The basic idea behind EOM-CC is
to construct the similarity transformed Hamiltonian from the
Coupled-Cluster ground state solution, and then diagonalize it within
a given sub-space of particle-hole excited reference states. This
method can obtain both ground- and excited states of closed and
open-shell nuclei. One-particle attached/removed EOM-CC has been the 
most common method in order to solve for systems with A $\pm 1$
particles outside a closed shell. For a given approximation at the
Coupled-Cluster level, e.g. CCSD, there is a natural truncation of
n-particle-m-hole excitations. For CCSD, there are only 1p (1h) and
2p1h (2h1p) excited states, since any higher exciation can not connect
to the similarity transformed Hamiltonian. There is also a natural
extension to two-particles attached and removed EOM-CC, and in
principle any number of particles can be added or removed. 

In this PhD project we intend to derive and implement all diagrams for 
two-particle attached/removed EOM-CC. At leading order this will
include excitations of the type 2p (2h) and 3p-1h (3h-1p). It is clear 
that the number of configurations will be large, and eventually become
a memory issue 
when increasing the system size (number of particles and
model-space). At leading order we have $n_{o}n_{u}^3$ number
of configurations where $n_{o}$ is the number of occupied and 
$n_{u}$ the number of unoccupied orbitals. As a first step the
equations will be implemented in m-scheme. This approach, does however
slightly violate size-extensivity due to unlinked diagrams, it has
been shown that this violation affects mostly the high lying
spectrum. A method which restores size-extensivity and works for
open-shell systems is the Fock-Space Multireference Coupled-Cluster
theory by R. Bartlett and M. Musial [2]. We will study how well the 
two-particle attached/removed EOM-CC approach performs by comparing 
ground and excited states of $^6$He with exact
diagonalization. Having assessed the quality of this approach, we
intend to derive and implement a spherical particle attached/removed
EOM-CC code. In this way we can target specific states of interest,
and most importantly the dimensionality of the problem will decrease
dramatically. Each diagram will be checked against the m-scheme code
ensuring correct implementation and derivation. With this 
fully operational a lot of interesting physics problems can be addressed.
Ab-initio calculation of the whole oxygen chain ( $^{20-28}$O ) will
be possible. It will be very useful for obtaining a
microscopic understanding of pairing densities in nuclei. Further, it
will allow for microscopic calculation of neutrino-less double-beta decay. 
As a final application this approach will allow for the construction
of an effective two-body interaction which can be input in shell-model
approaches where particles move with respect to a closed core. 


These topics will form the main part of the PhD thesis, providing thereby an unprecendented 
level of many-body physics in the derivation of effective interactions for configuration interaction. In addition, the codes will be made flexible enough
to be able to include studies of hypernuclei, partly studied in Gustav Jansen's Master thesis.
Furthermore, this flexibility will allow for studies of atomic systems and quantum dots as well.

The main issue of the thesis will deal with two-body interactions and equations at the level of 
two-particle-two-hole correlations built from an $N$-particle Slater determinant. However, if time allows, the inclusion of triples and three-body interactions, is an important and actual topic.


\begin{enumerate}
\item G.~Hagen, T.~Papenbrock, D.~J.~Dean and M.~Hjorth-Jensen, Phys.~Rev.~Lett.~{\bf 101}, in press (2008).
\item R.~J.~Bartlett and M.~Musial, Rev.~Mod.~Phys.~{\bf 79}, 291 (2007) and references therein.
\end{enumerate}

\end{document}



