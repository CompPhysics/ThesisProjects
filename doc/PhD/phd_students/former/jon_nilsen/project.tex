\documentstyle[a4wide,11pt]{article}

\begin{document}

\pagestyle{plain}

\begin{center} \huge \bf PhD project in Computational Physics 
for Jon Kristian Nilsen \end{center}

\section*{Vortices in Bose-Einstein Condensates: A Quantum Monte Carlo Analysis}

The spectacular demonstration of Bose-Einstein condensation (BEC) in gases of
alkali atoms $^{87}$Rb, $^{23}$Na, $^7$Li confined in magnetic
traps\cite{anderson95,davis95,bradley95} has led to an explosion of interest in
confined Bose systems. Of interest is the fraction of condensed atoms, the
nature of the condensate, the excitations above the condensate, the atomic
density in the trap as a function of Temperature and the critical temperature of BEC,
$T_c$. The extensive progress made up to early 1999 is reviewed by Dalfovo et
al.\cite{dalfovo99}.

A key feature of the trapped alkali and atomic hydrogen systems is that they are
dilute. The characteristic dimensions of a typical trap for $^{87}$Rb is
$a_{h0}=\left( {\hbar}/{m\omega_\perp}\right)^\frac{1}{2}=1-2 \times 10^4$
\AA\ (Ref. 1). The interaction between $^{87}$Rb atoms can be well represented
by its s-wave scattering length, $a_{Rb}$. This scattering length lies in the
range $85 < a_{Rb} < 140 a_0$ where $a_0 = 0.5292$ \AA\ is the Bohr radius.
The definite value $a_{Rb} = 100 a_0$ is usually selected and
for calculations the definite ratio of atom size to trap size 
$a_{Rb}/a_{h0} = 4.33 \times 10^{-3}$ 
is usually chosen \cite{dalfovo99}. A typical $^{87}$Rb atom
density in the trap is $n \simeq 10^{12}- 10^{14}$ atoms/cm$^3$ giving an
inter-atom spacing $\ell \simeq 10^4$ \AA. Thus the effective atom size is small
compared to both the trap size and the inter-atom spacing, the condition
for diluteness (i.e., $na^3_{Rb} \simeq 10^{-6}$ where $n = N/V$ is the number
density). In this limit,
although the interaction is important, dilute gas approximations such as the
Bogoliubov theory\cite{bogoliubov47}, valid for small $na^3$ and large
condensate fraction $n_0 = N_0/N$, describe the system well. Also, since most
of the atoms are in the condensate (except near $T_c$), the Gross-Pitaevskii
equation\cite{gross61,pitaevskii61} for the condensate describes the whole gas
well. 

Recently however experiments have been performed which go beyond the dilute gas model.
These experiments challenge the foundation of the above mean field picture.
Ab initio tools like Diffusion Monte Carlo (DMC) and Green's function Monte Carlo 
(GFMC) methods are therefore to be preferred 
when one moves to larger densities. DMC and GFMC offer in principle exact solutions
of the many-body Schr\"odinger equation.
Furthermore, there has  recently been 
much experimental and theoretical interest in excited
states of Bose-Einstein condensates characterized 
by a quantized circulation, so-called vortices.
The existence of these
excited condensate states is crucial to studies of  the superfluid behavior of
trapped atomic condensates

The purpose of this Dr.~Scient (PhD) thesis is twofold:
\begin{itemize}
\item To develop Diffusion Monte Carlo and GFMC programs for studying systems 
beyond the dilute limit, both ground state features and excited states.
This will allow for in principle exact solutions of Schr\"odinger's equation
for densities beyond the dilute gas limit.
The aim is to use these methods and evaluate 
ground state and excited state properties of
a trapped, hard sphere Bose gas over a wide range of densities
using DMC and GFMC  methods with several 
trial wave functions. These wave functions are used 
to study the sensitivity of condensate and 
non-condensate properties to the hard sphere radius and the number 
of particles.

Jon Nilsen has already developed a large parallel Variational 
Monte Carlo (VMC) code for studying Bose Einstein condensates. In Ref.~\cite{jon2005},
properties of the ground state and excited states were studied using this program.
This paper is part of Jon Nilsen's PhD project. 

\item The second aim of this thesis project is to study parallel implementations
of the DMC and GFMC programs 
on various machine structures, from local PC clusters to large scale machines 
available through the NOTUR project, using both
distributed and shared memory. The reason for this part is due to the 
complexity represented
by the DMC and GFMC calculations, where tipycally many random walkers are needed in order
to get good statistics and to obtain the profiles of the wave functions of the
various states. Parallel algorithms are presently the only reasonable way of obtaining
statistics of good quality. 

The expertise gained in such studies can also be transferred to other many-body 
projects, such as the Large-Scale shell model code for nuclear physics studies developed
by Torgeir Engeland. 
\end{itemize}



\begin{thebibliography}{999}%\parskip=0pt\itemsep 0pt%
\footnotesize
\bibitem {anderson95}%1
M.H. Anderson, J.R. Ensher, M.R. Matthews, C.E. Wieman, and E.A. Cornell, {\em Science} {\bf
269}, 198 (1995).

\bibitem{davis95}%2
K.B. Davis, M.-O. Mewes, M.R. Andrews, N.J. van Druten, D.S. Durfee, D.M. Kurn, and W. Ketterle,
{\em Phys. Rev. Lett.} {\bf 75}, 3969 (1995).

\bibitem{bradley95}%3
C.C. Bradley, C.A. Sackett, J.J. Tolett, and R.G. Hulet, {\em Phys. Rev. Lett.} {\bf 75}, 1687
(1995); C.C. Bradley, C.A. Sackett, and R.G. Hulet, {\em ibid.} {\bf 78}, 985 (1997).

\bibitem{dalfovo99}%4
F. Dalfovo, S. Giorgini, L. Pitaevskii, and S. Stringari, {\em Rev. Mod. Phys.} {\bf 71}, 463
(1999).

\bibitem{bogoliubov47}%5
N.N. Bogoliubov, {\em J. Phys. (Moscow)} {\bf 11}, 23 (1947).

\bibitem{gross61}%6
E.P. Gross, {\em Nuovo Cimento} {\bf 20}, 454 (1961).

\bibitem{pitaevskii61}%7
L.P. Pitaevskii, {\em Ah. Eksp. Teor. Fiz.} {\bf 40}, 646 (1961) [{\em Sov. Phys. JETP} {\bf 13},
451 (1961)].

\bibitem{jon2005} J.~K.~Nilsen, J.~Mur-Petit,
  M.~Guilleumas, M.~Hjorth-Jensen, 
  and A.~Polls, submitted to Phys.~Rev.~A.

\end{thebibliography}


\end{document}







\bibitem{fabro99} A.~Fabrocini and A.~Polls, Phys.~Rev.~A {\bf 60},
  2319 (1999). 


\bibitem{glyde1} J.~L.~Dubois and H.~R.~Glyde, Phys.~Rev.~A {\bf 63},
  023602 (2001). 

\bibitem{glyde2} A.~R.~Sakhel, J.~L.~Dubois, and H.~R.~Glyde,
  Phys.~Rev.~A {\bf 66}, 063610 (2002). 

\bibitem{glyde3} J.~L.~Dubois and H.~R.~Glyde, Phys.~Rev.~A {\bf 68},
  033602 (2003).

\bibitem{blume1} D.~Blume and C.~H.~Greene, Phys.~Rev.~A {\bf 63},
  063601 (2001).

\bibitem{hpl} H.~P.~Langtangen, {\em Computational Partial Differential Equations},
(Springer, Berlin, 1999).

\bibitem{flo80} K.~T.~R.~Davies, H.~Flocard, S.~Krieger, and M.~S.~Weis,
 Nucl.~Phys.~A {\bf 342}, 111 (1980).

\end{thebibliography}


\end{document}


















