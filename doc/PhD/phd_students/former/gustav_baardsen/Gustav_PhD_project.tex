\documentclass[prl,preprint]{revtex4}
\usepackage{graphicx,amsmath,amssymb,bm}


\begin{document}

\title{Thesis project for PhD student Gustav B\aa rdsen: Coupled-Cluster theory and {\em ab initio} density functional theory for nuclear matter}

\author{Gustav B\aa rdsen, gba@cma.uio.no}
\affiliation{Center of Mathematics for  
Applications, University of Oslo, N-0316 Oslo, Norway}
\author{Advisor I: Morten Hjorth-Jensen, mhjensen@fys.uio.no}
\affiliation{Department of Physics and Center of Mathematics for  
Applications, University of Oslo, N-0316 Oslo, Norway}
\author{Advisor II: Trygve Ulf Helgaker, t.u.helgaker@kjemi.uio.no} 
\affiliation{Department of Chemistry and Center for Theoretical and Computational Chemistry, University of Oslo, N-0316 Oslo, Norway}


\maketitle


{\bf Project summary:}
In the last few years Coupled-Cluster theory has seen a revival in 
the nuclear structure community. Up until recently Coupled-Cluster 
theory was implemented in an uncoupled scheme (m-scheme). 
Although simple in its form, the m-scheme representation puts
constraints on the size of the model space
and the number particles considered. Recently, Coupled-Cluster with Singles- and Doubles approximation (CCSD)
was derived and implemented in a J-coupled scheme. Taking the spherical 
symmetry of closed shell nuclei into account, and realizing that the
Coupled-Cluster Singles- and Doubles similarity transformed
Hamiltonian has at most two-body terms, an efficient spherical CCSD code was implemented. This 
representation reduces the number of non-linear equations and  the computational cost dramatically, 
allowing us to reach into the medium mass region of the nuclear
chart. Within the next two years our plans are to perform {\em ab initio} calculations of all closed-shell nuclei, from $^4$He to $^{208}$Pb.
Furthermore, using
the spherical scheme, it is now 
possible to reach convergence of medium mass nuclei starting from ``bare'' interactions. 
Single-reference Coupled-Cluster theory works well for nuclei with closed shells such as $^4$He, $^{16}$O
and $^{40}$Ca, see Ref.~[1], see also 
the recent review of R. Bartlett and M. Musial [2] and the text of Helgaker 
{\em et al} [3] for more information on coupled cluster theory.

Extending Coupled-Cluster theory to infinite matter studies will be a huge step forward allowing for an understanding of nuclear correlations at the levels of
singles, doubles and partially triples from few nucleons to infinitely many. It will most likely represent the first {\em ab initio}  calculation of infinite matter. 

These calculations will also provide us with insights about missing three-body and more complicated many-body correlations in nuclear systems as function of the number of particles. The calculations
of Ref.~[1] show that coupled-cluster theory at the level of singles, doubles and and partial triples correlations is size extensive and that the missing
agreement with experiment is mainly due to missing three-body forces in the Hamiltonian.  These three-body contributions stay stable as function of increasing
mass number. If this pertains to infinite matter as well, one can infer 
that three-body correlations or more complicated many-body correlations do not
blow up as one adds more and more particles. This has huge consequences for many-body studies of nuclear systems. 
Calculations of heavier closed-shell nuclei will also be performed as a part of the path towards infinite nuclear matter.

The calculations will be done using our existing coupled-cluster codes written in an angular momentum coupled-scheme, but using interaction matrix elements 
with a plane wave basis, following the recipe outlined in Ref.~[4]. The investigation of the coupled-cluster formalism in a plane wave basis will form a large part of this PhD project. 

The next step of this thesis, is to use the coupled-cluster calculations
for infinite matter as a basis for constructing a density functional for 
infinite matter, something which has never been done before. These calculations
will thereby mimick the corresponding ones made by Kohn and Sham and many others 
on the infinite electron gas, see Ref.~[5]. These calculations paved the way for the development of the extremely succesful density functional theory. 
The approach which will be chosen for this thesis is the so-called adiabatic
approach, see Ref.~[6] for some recent applications.  The second advisor has worked on this method lately
and the Center of excellence at UiO 'Center of Theoretical and Computational Chemistry' has the necessary
expertise.  
Infinite matter is a translationally invariant system, a feature which allows for rather straightforward applications of the adiabatic approach. 
With these insights, the next aim is to use the results for several 
closed-shell nuclei to compute a corresponding density functional for 
finite nuclei. 

A proper density functional  for nuclei is crucial if one wants to study nuclear systems with many active particles. This is in particular relevant for medium-heavy and heavy nuclei between closed-shell systems, 
where the dimensionalities are well 
beyond the capabilities of {\em ab initio} methods like full configuration interaction approaches, coupled-cluster theories, various Monte Carlo methods or Green's function based approaches. Many of the coming experimental programs worldwide will address such nuclear systems in order to study the stability of matter. 
A density functional based on {\em ab initio} calculations of closed-shell nuclei has the possibility 
to incorporate many-body effects beyond what is normally done in nuclear density functional theory and related
mean-field approaches.  In order to achieve a higher predictivity in studies of nuclei, it is crucial to develop methods which can allievate the curse of the increased dimensionality.

The focus of  the thesis will be on Hamiltonians based on 
 two-body interactions and equations at the level of 
two-particle-two-hole and tree-particle-three-hole correlations built from an $N$-particle Slater determinant. However, if time allows, the inclusion of three-body interactions, is an important and highly actual topic in nuclear physics. 


\begin{enumerate}
\item G.~Hagen, T.~Papenbrock, D.~J.~Dean and M.~Hjorth-Jensen, Phys.~Rev.~Lett.~{\bf 101}, 092502 (2008).
\item R.~J.~Bartlett and M.~Musial, Rev.~Mod.~Phys.~{\bf 79}, 291 (2007) and references therein.
\item T.~Helgaker, P.~J{\o}rgensen, and J.~Olsen, {\em Molecular Electronic Structure Theory. Energy and Wave
  Functions}, (Wiley, Chichester, 2000).
\item G.~Hagen, M.~Hjorth-Jensen, and N.~Michel, Phys.~Rev.~C, {\bf 73}, 064307 (2006), and references therein.
\item W.~Kohn and L.~J.~Sham, Phys.~Rev.~{\bf 140}, 1133 (1965).
\item  A.~M.~Teale, S.~Coriani, and T.~U.~Helgaker, J.~Chem.~Phys.~{\bf 130}, 104111 (2009), and references therein.

\end{enumerate}

\end{document}





