\documentclass[a4paper,12pt,twoside]{article}
\usepackage[T1]{fontenc}
\usepackage{latexsym}
\usepackage[swedish]{babel}
\usepackage{psfig}
\usepackage{float}
\usepackage[dvips]{graphicx}

%\addtolength{\topmargin}{-1.5cm}
%\addtolength{\textheight}{5cm}
%\addtolength{\oddsidemargin}{-1cm}
%\addtolength{\evensidemargin}{-3cm}
%\addtolength{\textwidth}{3cm}

%\pagestyle{empty}
\bibliographystyle{plain}

\author{}
\title{A study of the physics and mathematics behind Bose-Einstein condensation.}
 
\begin{document}

\maketitle


The physics of  Bose-Einstein condensation of dilute atomic gases
represents an extremely lively experimental and theoretical branch of modern 
physics, with at least three Nobel prizes in Physics awarded during the last 10-15 years. 
A Bose-Einstein condensate
is a phase of matter formed by bosons (in this case atoms with integer angular momenta) 
cooled to temperatures very close to absolute zero (0 kelvins or -273.15 degrees Celsius). 
Under such supercooled conditions, a large fraction of the atoms collapse into the lowest
quantum state, at 
which point quantum effects become apparent on a macroscopic scale. The 
transition from a normal gas to a Bose-Einstein condensed systems results in macroscopic
phenomema like viscous-free fluid.
Quantized vortices can also be formed and 
recent experiments have 
verified the intriguing feature that Bose-condensed atoms are ``laser-like''. In other words, 
the matter waves of the atoms are coherent. In recent experiments at MIT one  
demonstrated a rudimentary ``atom laser'' that generates a beam of coherent atoms, 
in analogy to the emission of coherent photons by an optical laser.


The condensates are however extremely fragile. The slightest interaction with the outside world can be 
enough to warm them past the condensation threshold, forming a normal gas and loosing their
interesting properties. 
It is likely to be some time before any practical applications are developed.
Nevertheless, they have proved to be useful in exploring a wide range of questions in
fundamental physics, 
and in the years since the initial discoveries we have seen an explosion in experimental and
theoretical activity. 
Examples include experiments that have demonstrated interference between condensates due
to wave-particle duality,
the study of superfluidity and quantized vortices, and the slowing of light pulses to very low speeds using 
electromagnetically induced transparency. 


The dynamics of a Bose-Einstein condensate at zero temperature is assumed to be more or less 
properly described by the Gross-Pitaevskii equation or modified Gross-Pitaevkii equation for
denser condensates. 
The Gross-Pitaevskii equation is a Schr\"odinger equation with nonlinearities of third order.
It is a non-linear partial differential equation.
While a description is available when $T=0$ 
it is a fact that getting there one must start at some positive $T>0$.  From a practical point
of view it is of great importance that
this process goes as smoothly, controlled, and predictable as possible in order to open up for
possible technical applications. In achieving this a thorough, qualitative, and quantifiable understanding of the phenomena is needed.

 
It is therefore reasonable to assume any mathematical description of the evolution
of such a system, as the temperature drops, to be rich enough and have an intriguing mathematical strtucture well
worth of a serios mathematical undertaking. 


In a series of well cited papers \cite{G1}-\cite{G6} Gardiner et.al  have developed a quantum kinetic theory designed to describe 
the behaviour of a dilute quantum Bose gas not only at temperature $T=0$, but also at $T>0$. 
Any theory governing the process of laser cooling 
must include both quantum mechanics and the thermal effects arising from the finite temperature. 
The problem with the resulting theory is that the equations are 
difficult to handle exactly.

In Ref.~\cite{GP}, a combination of the mathematics of the BEC description with that of the quantum kinetic theorey is made, 
together with the idea of local energy and momentum conservation in order 
to overcome this situation. 
The result is an equation governing the coupling of the thermalized condensate band to the
Gross-Pitaevskii equation.
It is a stochastic Gross-Pitaevskii equation with both additive and multiplicative noise
(white in time and correlated in space).
Since building a model often involves many approximations it is not clear exactly which
properties have survived. Thus, the model
requires an analytical/computational investigation.
The equation is hitherto not seen in the mathematics community.  


The existence of an invariant measure has been theoretically obtained (see \cite{DO}) 
for a significantly simpler (but somewhat similar) equation. We will pursue investigations in
this direction.
Several functionals of the solution corresponding to physically interesting properties  are
also important to study.  
We hope to establish some basic Ito formula.
Of course, the natural starting point is a proof of existence and uniqueness of a solution.
A result in this direction for a simpler equation is \cite{BD2}.

The experience in the physics part of the team is on the case $T=0$ and the related numerical
methods. The methods are finite element approaches to  non-linear
partial differential equations and Monte Carlo methods (variational and diffusion Monte Carlo)
for quantum mechanical systems
at zero temperature. The diffusion Monte Carlo methods yield in principle an exact solution
to Schr\"odinger's equation
for many interactiing particles at any density, viz, given a certain Hamiltonian, one obtains the exact solution.
The Gross-Pitavaeskii equation is an approximation to the fully interacting cases that works well at the low
densities typical for Bose-Einstein condensates. 
In this density regime, one can approximate the Schr\"odinger equation for a many-particle system to effective
one-particle equations, of which the Gross-Pitavaeskii equation is one possibility. 
 
Comparisons with diffusion Monte Carlo and the Gross-Pitavaeskii equation
for Bose-Einstein condensates in many different density regimes 
show excellent agreement. Monte Carlo methods are, however, extremely time-consuming
from a numerical point of view and the Gross-Pitavaeskii equation is the preferred approach.
For large systems,
even a smallish number such as  100 particles, we may speak of at least two-three orders of
magnitude in CPU expenditure.
A Bose-Einstein condensate may count millions of atoms, a calculation beyond reach for diffusion
Monte Carlo methods. 

However, since these methods apply to zero temperature only, 
we expect that the physics
group within CMA will gain insight and experience in the $T>0$ case. The topic of Bose-Einstein
condensate
is a hot one and shows good promise of being both scientifically and technologically
highly interesting  in the foreseeable
future. Furthermore, we plan to develop a large code for Monte Carlo calculations at finite $T$
using the path-integral method. As with diffusion Monte carlo methods, this method provides also in principle  an exact answer to Schr\"odinger's equation. The answer serves as a benchmark 
to our new stochastic Gross-Pitavaeskii equation (SPDE). 
Path-integral Monte Carlo approaches are again extremely time-consuming. Our path-integral  approach
will however serve to justify the SPDE approach and provide a theoretical benchmark and justification of 
the use of the stochastic Gross-Pitavaeskii equation. 

The experience of the mathematics part of the team is based on stochastic partial differential equations. The highly 
non-trivial SPDE associated to Bose-Einstein condensate will ensure that

1) a new important SPDE comes out ``on the market'',

2) the development of tools suitable for this new class of SPDE will be initiated and hopefully, to some extent, completed, 

3) a development of numerical methods will give interesting numerical results where analytical tools are not yet available.

This will certainly provide original input for the stochastic team at CMA as well as open up for opportunities to delve further 
into this new application. It may also bring about new and original modeling in the finance application.

The article \cite{GP} will be the basis for a common investigation and
the stochastic Gross-Pitaevskii equation will be the starting point for both mathematical and 
numerical/computational work aimed at providing a clearer picture of the process of cooling down a Bose gas to condensation.

  
It is hoped that the theoretical studies will result in publication in journals like ``Journals of Mathematical Physics'' and
that the simulation studies be published in for example  ``Physical Review''. 

\begin{thebibliography}{99}
\bibitem{BD2} A. De Bouard, A. Debussche. A Stochastic Nonlinear Schr\"odinger Equation with Multiplicative Noise.
\emph{Communications in Mathematical Physics} {\bf 205}, 161-181 (1999). 
\bibitem{BD} A. De Bouard, A. Debussche. Blow-up for the Stochastic Nonlinear Schrodinger Equation with Multiplicative noise.
\emph{Annals of Probability} 2005, {\bf 33}, 3, 1078-1110.
\bibitem{DO}A. Debussche, C. Odasso. Ergodicity for the Weakly Damped Stochastic Non-linear Schrodinger Equation.
\emph{Journal of Evolution Equations}, {\bf 5}, 3, 2005.
\bibitem{GP} C.W. Gardiner, J.R. Angelin, T.I.A. Fudge. The stochastic Gross-Pitaevskii equation.
\emph{J. Phys. B: At. Mol. Opt. Phys.} {\bf 35}, 1555-1582 (2002).  
\bibitem{G1} C.W. Gardiner, P. Zoller.  Quantum Kinetic Theory I: 
A Quantum Kinetic Master Equation for Condensation of a 
weakly interacting Bose gas without a trapping potential. \emph{Phys. Rev.} A {\bf 55}, 2902 (1997). 
\bibitem{G2} D. Jaksch, C.W. Gardiner, P. Zoller.  Quantum Kinetic Theory II: 
Simulation of the Quantum Boltzmann master equation. \emph{Phys. Rev.} A {\bf 56}, 1 (1997).
\bibitem{G3} C.W. Gardiner, P. Zoller.  Quantum Kinetic Theory III: 
Quantum Kinetic Master Equation for strongly condensated condensated trapped systems. 
\emph{Phys. Rev.} A {\bf 58}, 1 (1998).
\bibitem{G4} D. Jaksch, C.W. Gardiner, K.M. Gheri, P. Zoller.  Quantum Kinetic Theory IV: 
Intensity and amplitude fluctuations of a Bose-Einstein condensate at finite temperature including trap loss.
 \emph{Phys. Rev.} A {\bf 58}, 1450 (1997).
\bibitem{G5} C.W. Gardiner, P. Zoller.  Quantum Kinetic Theory V: 
Quantum kinetic master equation for mutual interaction of condensate and noncondensate. \emph{Phys. Rev.} A {\bf 61}, 033601 (2000).  
\bibitem{G6} C.W. Gardiner, P. Zoller.  Quantum Kinetic Theory VI: 
The growth of a Bose-Einstein condensate. \emph{Phys. Rev.} A {\bf 62}, 033606 (2000).  
\bibitem{Gb} C.W. Gardiner. \emph{Quantum Noise}, Springer-Verlag, (1991).
\bibitem{GP} C.W. Gardiner, J.R. Anglin, T. I. A Fudge. The stochastic Gross-Pitaevskii equation. 
\emph{Journal of Physics B: Molecular and Optical Physics}, {\bf 35} (2002), 1555-1582.
\end{thebibliography}
\end{document}



