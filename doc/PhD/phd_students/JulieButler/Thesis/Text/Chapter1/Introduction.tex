%% INFINITE MATTER (ELECTRON GAS AND NUCLEAR MATTER)

%% MANY-BODY PHYSICS
Describing a system of many-interacting particles is difficult in the context of classical mechanics. It becomes even more complicated when the quantum mechanical properties of the particles are included. The only quantum mechanical many-body problems we can solve exactly are uninteresting toy models, and thus we must develop a framework to approximately solve the many-body problem for large systems of interest, such as atomic nuclei and infinite matter systems \cite{Ref13}. This is the basis of the field of many-body physics, which intersects with nuclear theory, condensed-matter theory, and quantum chemistry.

Specifically, we will look at \textit{ab initio}, or first principles, many-body methods where the calculations start from a given Hamiltonian and attempt to solve the many-body problem without allowing any uncontrolled approximation. These first principle calculations are essential in nuclear physics and quantum chemistry calculations. It should be noted that while a proper \textit{ab initio} calculation in nuclear physics would begin with the quark and gluon degrees of freedom, in this thesis we use \textit{ab initio} to refer to a group of many-body methods which can perform calculations on a system given the Hamiltonian of the systems and relevant laws of motion.


Though the first principle of many-body calculations can be expensive, both in terms of computational time and computational resources, they play an important role in many-body physics for a few reasons. First, performing these calculations at a fine-grained and controllable level allows us to probe new and interesting particle interactions and challenge existing computational methods. Second, these first principle interactions allow theoretical many-body physicists to guide new experiments and interpret the results from these experiments. Furthermore, since these calculations start from first principles and are not phenomenological, this \textit{ab initio} many-body methods can make predictions on systems where no experimental data exists \cite{Ref16}.

%% AB INTO IN NUCLEAR PHYSICS
In nuclear physics, the results from performing an \textit{ab initio} calculation on a light nucleus are expected to yield a binding energy that has an error of less than a few percent when compared to experimental results. While this is a good accuracy considering that these methods start only from the Hamiltonian of the system, phenomenological many-body methods such as the shell model achieve a higher level of accuracy since they are adjusted and fit with experimental data. However, \textit{ab initio} methods still have a place in nuclear physics as they allow us to test and expand our knowledge of the various nuclear interaction models. Additionally, they allow us to study nuclei from the nuclear chart in regions where very little data is available for phenomenological models \cite{Ref16}. They are many \textit{ab initio} many-body methods available for nuclear studies, and they differ in their computational costs and the regions of the nuclear chart where they can be applied. They also have differences in flexibility and ability to deal with various Hamiltonians.

%% COUPLED CLUSTER THEORY
The \textit{ab initio} many-body method which will be the focus of this thesis is coupled cluster theory, originally named exp S, which originated in the nuclear physics community over sixty years ago p. Coester and K\"{u}mmel proposed an exponential ansatz that could be used to describe the correlations within a nucleus. Since its inception in the late 1950s and early 1960s, coupled cluster theory has become an important many-body method, both in its home field of nuclear physics and quantum chemistry. 

There are a few reasons why coupled cluster theory is such a popular many-body method. First, coupled cluster theory is fully microscopic and is both size extensive and size consistent. Second, coupled cluster theory can exactly produce the fully correlated many-body wavefunction if all components of its cluster operator are considered. However, since this is usually not possible, coupled cluster theory provides a systematic truncation method that allows for hierarchical improvements. Third, though coupled cluster theory is not a variational method, its energy behaves in most cases as a variational quantity. Finally, the form of the equations in coupled cluster theory is very amenable to parallel implementation, including with MPI and utilizing graphical processing unit (GPU) computations \cite{Ref148}.

%% COUPLED CLUSTER THEORY IN QUANTUM CHEMISTRY
 In the 1960s and 1970s, coupled cluster theory was further developed and was applied to the field of quantum chemistry, where it continues to flourish to this day \cite{Ref151, Ref150, Ref149}. Though this thesis is in nuclear physics, we will take inspiration from quantum chemistry and use some of the advancements made in that field. Ref. \cite{Ref149} provides a good review of coupled cluster theory's importance to the field of quantum chemistry.

%% COUPLED CLUSTER THEORY IN NUCLEAR PHYSICS
There are several good reviews on coupled cluster theory's history in nuclear physics and developments at that time (see Refs. \cite{Ref147, Ref68, Ref16}). However, we will summarize the highlights and mention some recent coupled cluster results in nuclear physics that post-date these reviews. Though coupled cluster theory was initially developed in nuclear physics, it grew slowly in the field and experienced only slow changes and growth during the second half of the 20th century \cite{Ref68}. This was due to a combination of inaccurate models to describe the nuclear interaction and computational limitations, which made describing nuclear systems of interest challenging and inaccurate. Heisenberg and Mihaila were the first to use a high-precision nucleon-nucleon interaction and three-nucleon forces in a coupled cluster calculation of a nucleus, specifically oxygen-16 \cite{Ref154}. Coupled cluster theory experienced a revival in nuclear physics in the early 2000s (see, for example, Ref. \cite{Ref148}) and has been successfully and accurately applied to both medium mass nuclei and is now being extended to study high mass nuclei. See. Ref. \cite{Ref16} for a review of coupled cluster calculations on medium mass nuclei among other systems and Ref. \cite{Ref20} for a recent result concerning coupled cluster studies of high mass nuclei.  

Though coupled cluster theory in nuclear physics does differ from coupled cluster theory in quantum chemistry, an essential advancement in the revival of coupled cluster theory in nuclear physics in the early 2000s occurred when Dean and Hjorth-Jensen took the standard approach to coupled cluster theory from quantum chemistry and translated it to nuclear physics \cite{Ref148}. Additionally, within nuclear physics, several conceptual and theoretical developments involving nuclear interactions increased coupled cluster accuracy when applied to nuclear problems. These advances include the development of soft interactions thru the application of a renormalization group (RG) transformation and the inclusion of three-nucleon interactions. These changes, allowing with the development of more realistic nuclear interactions \cite{Ref42,Ref44,Ref45,Ref46,Ref59, Ref101, Ref200, Ref201}, made coupled cluster theory are more feasible method for application to nuclear studies. Finally, increased computational technology has allowed more computational cycles for faster calculations and the option for coupled cluster calculations to be further accelerated by leveraging modern computational methods such as parallel computing and utilizing graphical processing units (GPUs) \cite{Ref16,Ref82}. Recent advances in coupled cluster theory have allowed calculations on nuclei as large as lead-208 (\cite{Ref20}).

%% INFINITE MATTER AND CC
One important system which can be analyzed with coupled cluster theory is a system of infinite matter, meaning that the system contains an infinite number of particles and spans an infinite volume in space. The most important infinite matter system in nuclear physics is infinite nuclear matter, where all of the particles are protons and neutrons in various fractions. In this thesis, we will investigate two infinite nuclear matter systems: pure neutron matter, where all the particles are protons, and symmetric nuclear matter, where the particles are evenly divided between protons and neutrons.

%% IMPORTANCE OF IN
Infinite nuclear matter has several essential applications within nuclear physics and astrophysics. Within nuclear physics, calculations of infinite nuclear matter are performed alongside calculations of finite nuclei to support the results (see, for example, \cite{Ref20, Ref101, Ref103}). Additionally, calculations of infinite nuclear matter are essential for determining the nuclear equation of state, which can determine the behavior of both finite nuclei and the behavior of matter in astrophysical objects. Finally, some astrophysical objects, such as neutron stars, are challenging to study using traditional observational astronomy. However, we can still conclude their behaviors and properties by performing many-body studies on their particles. The matter that makes up objects, such as neutron stars, can be modeled by various types of infinite nuclear matter\cite{Ref34, Ref35,Ref36,Ref37,Ref38,Ref39,Ref41,Ref55,Ref56,Ref58, Ref84,Ref85}.  There have been several coupled cluster studies of infinite nuclear matter previously \cite{Ref4, Ref9, Ref8, Ref16, Ref93}.

%% ELECTRON GAS AS A TEST BED
Outside of nuclear systems, this thesis will also perform coupled cluster calculations of the homogeneous electron gas (HEG), an infinite matter system containing only electrons with a positive background charge, so is no net charge \cite{Ref66}. The electron gas is essential in many fields of chemistry, which as density functional theory, where it is used to benchmark calculations, and is used as a model for the electrons in superconducting metals. The HEG has been extensively studied with coupled cluster theory for decades \cite{Ref2, Ref71,Ref72,Ref73,Ref74, Ref95}. For our purposes, the homogeneous electron gas will be a test case for developing methods that can be applied to infinite nuclear matter. The homogeneous electron gas, made entirely of electrons, is governed by the Coulomb force, a much simpler force than the nuclear forces. Thus coupled cluster calculations of the homogeneous electron gas are less time-consuming than calculations of infinite nuclear matter, making it a great sandbox to build our methodologies.

%% TRUNCATION PROBLEMS IN CC STUDIES OF INFINITE MATTER
While it may be evident that modeling an infinite system in a finite computational space is not feasible, this is a significant hurdle to performing accurate coupled-cluster calculations on infinite matter systems. The number of single-particle states and the number of particles in the system must be truncated, introducing basis incompleteness and finite size errors, respectively. The coupled cluster calculation, specifically the coupled cluster correlation operator, must also be truncated due to computational time and resource limitations.

%% WHY MACHINE LEARNING?ADD MORE HERE
While traditional methods of extrapolation can be used to remove the basis incompleteness and finite size errors with some success (\cite{Ref99, Ref100, Ref96, Ref2, Ref74, Ref86, Ref87, Ref94, Ref98}), these methods are usually very specific to the exact system and many-body method being studied. It should also be noted that many of these studies focus on extrapolations to the thermodynamic limit and not the complete basis limit. Additionally, performing the calculations at large values of M and N to mitigate these errors is computationally prohibitive due to a significant amount of time and resources, machine learning offers a new and promising avenue for removing these errors. Machine learning is a field of programming where the algorithms "learn" to perform a task by being given examples of the task they are to perform. Machine learning is becoming a common tool in many-body physics, and there have been many recent studies analyzing its various uses in the field \cite{Ref6, Ref7, Ref17, Ref22, Ref23,Ref25,Ref28,Ref29,Ref30,Ref31,Ref32, Ref67,Ref81, Ref105, Ref33, Ref106, Ref107, Ref108, Ref117}. A review is provided in Ref. \cite{Ref109}. The goal of this thesis is to use a class of machine learning algorithms based on Bayesian statistics and a novel method of formatting the data to create an algorithm called sequential regression extrapolation (SRE), which is capable of removing the basis incompleteness and finite size errors from coupled cluster calculations of infinite matter systems, using only complete calculations performed at small values of M and N. Additionally, we wish to develop the SRE method is such a way that it can be applied to remove truncation errors from any system and using any many-body method. Machine learning has been used to perform extrapolations in many-body calculations (\cite{Ref67, Ref110, Ref111, Ref112, Ref114, Ref115,Ref6}), and some of these studies do attempt to use machine learning ro remove the basis incompleteness or the finite size errors, but the SRE method is unique among these studies. We will use the SRE predictions and results calculated at the complete basis limit and the thermodynamic limit to determine the accuracy of the extrapolations. We will also determine the time saved by generating a small training data set at low M and N and using the SRE algorithm to extrapolate versus performing a coupled cluster calculation near the absolute basis limit or the thermodynamic limit.

%% STRUCTURE OF THESIS
The remainder of this thesis contains seven content chapters followed by conclusions and future works chapter. Chapter 2 will develop the basics of many-body theory, including second quantization and diagrammatic methods. Chapter 3 develops Hartree-Fock theory, many-body perturbation theory, and coupled cluster theory as the many-body methods of interest in this thesis. Chapter 4 expands on the infinite matter systems studied in this work, including the three-dimensional homogeneous electron gas, pure neutron matter, and symmetric nuclear matter. Chapter 5 introduces machine learning, including neural networks, various regression algorithms, and Bayesian machine learning. Chapter 6 develops the machine learning-based extrapolation algorithm, sequential regression extrapolation (SRE), which will be used to remove the basis incompleteness and finite size errors from coupled cluster calculations of infinite matter. Finally, Chapters 7 and 8 are the results chapter, showing the success of the SRE algorithm in removing the basis incompleteness and finite size errors from infinite matter calculations, first of the electron gas (Chapter 7) and then from infinite nuclear matter (Chapter 8).
