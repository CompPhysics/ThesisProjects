% INTRODUCTION
Machine learning has recently become a popular tool in physics and is being applied to a wide range of problems. Machine learning has found applications across all areas of physics. Specifically looking at neural networks, they have been used in nuclear physics to perform extrapolations (for example Refs. \cite{Ref24}, \cite{Ref25}, \cite{Ref26},\cite{Ref27},\cite{Ref28},\cite{Ref29},\cite{Ref30},\cite{Ref31}) and to directly solve the many-body problem (for example Ref \cite{Ref32}).  Neural networks can be problematic for \textit{ab initio} data sets due to very small data sets, but will be explored in this thesis as a possible machine learning method.  Machine learning can also be used in many-body physics to simulate the wavefunction of a system to find the variationally lowest energy of the system {ADD REFERENCES}.  More generally, machine learning algorithms have been used to predict the coupled cluster energies of molecules using a variety of machine learning algorithms.  The work presented in this thesis, though unique, has been inspired by these previous applications of machine learning in the field of many-body physics. 