% What is machine learning?
Machine learning is the field that occurs at the intersection of data science and artificial intelligence; it is the science of programming a computer so that it can learn from a given data set \cite{Ref12}. Machine learning algorithms can solve these problems without being programmed with task-specific instructions; machine learning encompasses a set of generic algorithms that can be applied to various problems.

% When is machine practical learning?
Machine learning can be applied to a wide variety of problems, but it is most useful when traditional styles of programming cannot solve the problem or would take a very long time to solve a problem. This includes problems where the traditional solution would have a long list of rules; machine learning can find patterns in the data set without problem-specific programming. Machine learning also excels in problems where a large amount of complex data is difficult to sort and work with by hand or with traditional programming \cite{Ref12}.


% Briefly explain some classifications of machine learning
Machine learning algorithms can be classified into one of three categories: supervised, unsupervised, and reinforcement learning. In supervised learning, the training data given to the algorithm is labeled, meaning it has both an X and a y component. Therefore, supervised learning aims to map every X to its corresponding y correctly. There are two types of supervised learning: classification (when y contains a finite number of possible values) and regression (when y contains an infinite number of possible values). All machine learning algorithms developed in this chapter will be supervised learning. Unsupervised learning algorithms work with unlabelled data, meaning they only receive the X component of the data set. Common unsupervised learning tasks are clustering and dimensionality reduction. Finally, reinforcement learning describes a set of algorithms where an agent learns to solve a problem by maximizing its reward. Supervised and unsupervised learning are common in machine learning applications in physics, while reinforcement learning in physics applications is less common.  

% INTRODUCTION
Machine learning has recently become a popular tool in physics and is being applied to a wide range of problems. Machine learning has found applications across all areas of physics. Specifically looking at neural networks, they have been used in nuclear physics to perform extrapolations (for example Refs. \cite{Ref24}, \cite{Ref25}, \cite{Ref26},\cite{Ref27},\cite{Ref28},\cite{Ref29},\cite{Ref30},\cite{Ref31}) and to directly solve the many-body problem (for example Ref \cite{Ref32}).  Neural networks can be problematic for \textit{ab initio} data sets due to very small data sets, but will be explored in this thesis as a possible machine learning method.  Machine learning can also be used in many-body physics to simulate the wavefunction of a system to find the variationally lowest energy of the system {ADD REFERENCES}.  More generally, machine learning algorithms have been used to predict the coupled cluster energies of molecules using a variety of machine learning algorithms.  The work presented in this thesis, though unique, has been inspired by these previous applications of machine learning in the field of many-body physics. 

% Explain the structure of the chapter
The remainder of this chapter will be structured as follows. The following sections will develop four standard supervised learning algorithms: linear regression, its related algorithms ridge and kernel ridge regression, and finally, a brief discussion on neural networks and recurrent neural networks. Next, there will be a discussion of Bayesian statistics to prepare the reader for the development of the final two machine learning algorithms: Bayesian ridge regression and Gaussian processes.