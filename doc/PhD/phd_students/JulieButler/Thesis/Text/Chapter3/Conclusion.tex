In this chapter, we have developed three many-body methods we will use throughout this thesis. The most fundamental of these methods is the Hartree-Fock theory, which will recover a reasonable amount of the system's energy. It is an independent particle model, so it cannot account for the energy from interacting particles. The first post-Hartree-Fock method we developed, so-called because it improves the Hartree-Fock result, was many-body perturbation theory (MBPT) which provides a convenient truncation scheme that allows for systematic improvements to the Hartree-Fock results. The final many-body method, developed in great detail due to its importance later in this thesis, was coupled cluster theory, which is more accurate than MBPT but much more time-consuming. Coupled cluster theory also provides a convenient truncation scheme and, at the triple level, allows us to recover most of the system's energy even with the method being truncated. Now that we have developed our many-body methods, the next chapter will be devoted to developing the system to which these methods will be applied: infinite matter.