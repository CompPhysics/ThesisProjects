For almost all CC calculations performed computationally, the cluster operator must be truncated \cite{Ref7}. Systems where CC calculations can be performed with the full cluster operator tend to be uninteresting toy models \cite{Ref8}. However, the form of the cluster operator provides a physically motivated method for truncation where all $m$-particle $m$-hole excitation operators over a certain level are set to zero. This approximation scheme is referred to as the SUB$_n$ approximation scheme \cite{Ref8}. There is also a convenient naming scheme when the cluster operator is truncated in this way. For example, if $\hat{T} \approx \hat{T}_1$, the approximation is called the coupled cluster single approximation (CCS), and if $\hat{T} \approx \hat{T}_1 + \hat{T}_2$ then we call this the coupled cluster singles and doubles approximation (CCSD), and if $\hat{T} \approx \hat{T}_1 + \hat{T}_2 + \hat{T}_3$ this leads to the coupled cluster singles, doubles, and triples approximation (CCSDT) \cite{Ref6}. Due to computational limitations, approximations over the triples level are rare, but have been performed on a limited set of systems \cite{Ref95}. For the infinite matter systems that are defined in the next chapter, the $\hat{T}_1$ operator gives no contribution to the energy so that we can simplify the above truncations: CCSD becomes CCD (coupled-cluster doubles) where $\hat{T} \approx \hat{T}_2$ and CCSDT becomes CCDT (coupled-cluster doubles and triples) where $\hat{T} \approx \hat{T}_2 + \hat{T}_3$. As the primary goal of this thesis is to use machine learning to extend the range of pre-existing coupled cluster methods instead of developing new methods or improving on existing methods, the explanations below are kept short and generally free of derivations. There have been many great works deriving the various coupled cluster methods in great detail, and we would like to point the readers to Refs. \cite{Ref21, Ref155, Ref149}, among others, for complete derivations of the methods introduced in this section.

As is expected, the higher the level of approximation, the more accurate the calculations but, the more extended run times. However, if having all $N$ terms in the cluster operator gives 100$\%$ of the energy of an $N$-particle system, then the CCSD approximation gives approximately 90$\%$ of the total energy, and CCSDT gives almost 99$\%$ of the total energy \cite{Ref21}. Each additional $m$-particle $m$-hole excitation operator does increase the accuracy of the energy calculation, but the improvements progressively decrease in size. Additionally, as expected, the computational time and resource requirements increase drastically as the order of the CC calculation increase. For example, the expected run time for a CCSD calculation is $O(M^6)$, where $M$ is the number of single-particle states. The expected run time for a CCSDT calculation is $O(M^8)$, and considering $M$ is likely to be 1,000 or higher for accurate calculations, this extra factor of $M^2$ represents a significant increase in run time. However, since we would like to achieve the accuracy of CCSDT, recovering 99$\%$ of the total energy, we can instead look at some approximative methods to the CCSDT approximation. These methods will not be as accurate as the CCSDT method, but they will be more accurate than the CCSD method as they contain some components from the $\hat{T}_3$ operator. Additionally, while these approximative triples methods will have a longer run time than a CCSD calculation, they will have much shorter run times than a complete CCSDT calculation, making the triples approximations a good balance of accuracy and run time.


\subsection{CCSDT Approximations}
There are two types of approximative triples calculations: non-iterative perturbative triples approximations, CCSD(T), and iterative CCSDT-n approximations.  The methods differ in their complexity and formulations, but give simular results and thus will be compared below.
    
	\subsubsection{Non-Iterative Triples Approximations}
    The perturbative triples approximation, CCSD(T), is a non-iterative triples approximation that is the gold standard for quantum chemistry applications \cite{Ref16}. The CCSD(T) method was developed in the late 1980s in the field of quantum chemistry \cite{Ref158}. The computational considerations for CCSD(T) have two components: an iterative component with a cost of $O(M^6)$ (the CCSD equations) and a non-iterative component with a cost of $O(M^7)$, making it faster than complete CCSDT calculations by a factor of M or greater. Since M can be over 1,000, this can represent a significant decrease in the run time for a calculation.
    
    We will start developing the CCSD(T) equations by defining the following two permutation operators:
    \begin{equation}
        \hat{P}(a/bc) = 1 - \hat{P}_{ab} - \hat{P}_{ac},
    \end{equation}

    and:

    \begin{equation}
        \hat{P}(a/bc | k/ij) = \hat{P}(a/bc) \hat{P}(k/ij),
    \end{equation}

    where $\hat{P}_{pq}$ is the permutation between states $p$ and $q$.  Next we can define an equation through which the $\hat{T}_3$ amplitudes can be defined from the $\hat{T}_2$ amplitudes:

    \begin{equation}\label{triples_amps}
        \epsilon_{ijk}^{abc}t_{ijk}^{abc} = \hat{P}(a/bc | k/ij) \sum_d \langle bc | \hat{v}_2 | dk \rangle t_{ij}^{ad} = \hat{P}(c/ab|i/jk)\sum_l \rangle lc | \hat{v}_2 | jk \rangle t_{il}^{ab},
    \end{equation}

    where:

    \begin{equation} \label{t3_from_t2}
         \epsilon_{ijk}^{abc} = (\epsilon_i + \epsilon_j \epsilon_k) - (\epsilon_a + \epsilon_b + \epsilon_c).
    \end{equation}

    Eq. \ref{triples_amps} defines the leading order (i.e., the second-order) terms in the full CCSDT triples equations. This equation will be used here to define the CCSD(T) method and in the next section to define the CCSDT-1a method. Next we can define $E_T^{(4)}$, the triples-excitation contributions to the MBPT4 (MBPT to the fourth order):

    \begin{equation}
        E_T^{(4)} = \sum_{\substack{i > j > k \\ a > b > c}} {t_{ijk}^{abc}}^{(2)*}{t_{ijk}^{abc}}^{(2)}\epsilon^{abc}_{ijk}.
    \end{equation}

    Here we are defining the second order $\hat{T}_3$ amplitudes (see Sec. 10.5 of Ref. \cite{Ref21}). However, for CCSD(T), we are going to define these as ${t_{ijk}^{abc}}^{[2]}$ instead of ${t_{ijk}^{abc}}^{(2)}$ because we will generate the $\hat{T}_3$ amplitudes using Eq. \ref{t3_from_t2} and the converged $\hat{T}_2$ amplitudes (the square brackets represent that we are using converged $\hat{T}_2$ amplitudes and the number in the brackets is a generalized order). From here, we are in a place to define the energy that results from a CCSD(T) calculation using the following equation:

    \begin{equation}
        E_{CCSD(T)} = E_{CCSD} + E_T^{[4]} = E_{CCSD} + E_t^{[4]} + E_{st}^{[4]},
    \end{equation}

    where:
    
    \begin{equation}
        E_t^{[4]} = \sum_{\substack{i > j > k \\ a > b > c}} {t_{ijk}^{abc}}^{[2]*}{t_{ijk}^{abc}}^{[2]}\epsilon^{abc}_{ijk},
    \end{equation}

    and where:
    
    \begin{equation}
        E_{st}^{[4]} = (\sum_{ia}{t_i^a}^{[2]})\sum_{\substack{j>k \\ b>c}} \rangle bc | \hat{v}_2 | jk \rangle {t_{ijk}^{abc}}^{[2]}.
    \end{equation}

    Note that the only difference between CCSD(T) and another non-iterative CCSDT approximate method, CCSD[T] (see. Ref. \cite{Ref157}), is the inclusion of the $E_{st}^{[4]}$ term, which is usually relatively small. We should also mention an improvement to CCSD(T), called $\Lambda$-CCSD(T), which has been developed in quantum chemistry and is defined in Ref. \cite{Ref140}.

    \subsubsection{Iterative Triples Approximations}
    
    Another method of approximating the triple contribution is through the iterative CCSDT-$n$ methods, where $n$ = 1a, 1b, 2, 3, 4, or 5. Though computationally more expensive than their non-iterative counterparts, iterative triples approximations are more accurate \cite{Ref16}. All methods will be briefly described here, but we will focus on CCSDT-1a, which will be used for calculations later in this thesis. For more information on CCSDT-1b - CCSDT-4, please refer to Ref. \cite{Ref155} and Ref. \cite{Ref157}.

    The simplest of the CCSDT-n methods is the CCSDT-1a approximation and is created by setting $\hat{T}_1$ = $\hat{T}_3$ = 0 when being projected against the three-particle three-hole excited state. The CCSDT-1a covers some infinite order terms from the $\hat{T}_3$ operator. This contrasts CCSD, which covers no terms from the $\hat{T}_3$, and CCSDT, which covers all terms. As shown in Table \ref{tab:ccsdtn-projections}, the CCSDT-1a method uses the operator $e^{\hat{T}_1 + \hat{T}_2 + \hat{T}_3}$ when projected against the singly excited state, the operator $e^{\hat{T}_1 + \hat{T}_2} + \hat{T}_3$ when projected against the doubly excited state, and the operator $ 1 + \hat{T}_2$ when projected against the three-particle three-hole excited state. These projections are used when constructing the amplitude equations, which are used to derive the T-amplitudes and the energies and are in contrast to a complete CCSDT calculation where the operator $e^{\hat{T}_1 + \hat{T}_2 + \hat{T}_3}$ is used no matter the state \cite{Ref155, Ref157}. Some successive approximations are made in the CCSDT-1a approximation that will be discussed as the other methods are made.

    The CCSDT-1b model uses the complete $e^{\hat{T}_1 + \hat{T}_2 + \hat{T}_3}$ operator for the singly and doubly excited states but, like CCSDT-1a, uses the operator $ 1 + \hat{T}_2$ for the three-particle three-hole excited state. Because the full operator is projected against the doubly excited state, the disconnected $\hat{T}_1\hat{T}_3$ clusters are included. Additionally, CCSDT-1b has a similar computational time to CCSDT-1a \cite{Ref157}.

    The next approximation is the CCSDT-2 approximation, which like CCSDT-1b, applies the complete $e^{\hat{T}_1 + \hat{T}_2 + \hat{T}_3}$ operator to the single and doubly excited states and uses $\hat{T}_1 = \hat{T}_3 = 0$ when applied to the triply excited state. However, it changes the operator projected on the triply excited states to $e^{\hat{T}_2}$ \cite{Ref155, Ref157}. This is an important approximation since it adds the effects of the disconnected $\hat{T}_2\hat{T}_2$ clusters on the $T_3$ amplitudes. The CCSDT-3 approximation is conceptually the simplest, where only $\hat{T}_3$ is set to zero, such that the full $e^{\hat{T}_1 + \hat{T}_2 + \hat{T}_3}$ operator is projected onto the singly and doubly excited states but the operator $e^{\hat{T}_1 + \hat{T}_2}$ is projected onto the triply excited state \cite{Ref157}. The final approximation, CCSDT-4, projects the full $e^{\hat{T}_1 + \hat{T}_2 + \hat{T}_3}$ operator onto the singly and doubly excited states but uses the operator $e^{\hat{T}_1 + \hat{T}_2} + \hat{T}_3$ for the triply excited states. The CCSDT-5 approximation is simply the full CCSDT calculation.

    The reason to perform this approximation over the complete CCSDT calculation is computational run times. A total CCSDT calculation is expected to have a run time on the order of $O(M^8)$. The expected run times for the CCSDT-n approximations are in Table \ref{tab:ccsdtn_runtimes} \cite{Ref155}. CCSDT-1a, CCSDT-1b, CCSDT-2, and CCSDT-3 have computational run times that are shorter than a complete CCSDT calculation by a factor of M (where typically M is greater than 1,000) but are more extended than CCSD calculation by a factor of M. However, they include contributions from $\hat{T}_3$ and the triply excited states that CCSD does not. CCSDT-4 has the exact computational cost as a complete CCSDT calculation, so there are not many cases where it would be preferred over just using a complete CCSDT calculation. Additionally, about 75$\%$ of the extra energy from the triples contribution is captured by the CCSDT-1a approximation, which CCSDT-1b, CCSDT-2, and CCSDT-3 not adding any other  contributions to the energy from full summations, though they do add some contributions from partial summations \cite{Ref155}. Thus as a good compromise of accuracy and computational run time, we will use the CCSDT-1a approximation.

    \begin{table}[H]
        \centering
        \begin{tabular}{|c|c|c|c|}\hline
            Method & $|\Phi^a_i\rangle $ & $|\Phi^{ab}_{ij}\rangle$ & $|\Phi_{ijk}^{abc}\rangle$ \\ \hline
             CCSD & $e^{\hat{T}_1 + \hat{T}_2}$ & $e^{\hat{T}_1 + \hat{T}_2}$ & N/A \\ \hline
             CCSDT-1a & $e^{\hat{T}_1 + \hat{T}_2 + \hat{T}_3}$ & $e^{\hat{T}_1 + \hat{T}_2} + \hat{T}_3$ & $ 1 + \hat{T}_2$ \\ \hline
             CCSDT-1b & $e^{\hat{T}_1 + \hat{T}_2 + \hat{T}_3}$ & $e^{\hat{T}_1 + \hat{T}_2 + \hat{T}_3}$ & $ 1 + \hat{T}_2$ \\ \hline
             CCSDT-2 & $e^{\hat{T}_1 + \hat{T}_2 + \hat{T}_3}$ & $e^{\hat{T}_1 + \hat{T}_2 + \hat{T}_3}$ & $e^{\hat{T}_2}$ \\ \hline
             CCSDT-3 & $e^{\hat{T}_1 + \hat{T}_2 + \hat{T}_3}$ & $e^{\hat{T}_1 + \hat{T}_2 + \hat{T}_3}$ & $e^{\hat{T}_1 + \hat{T}_2}$ \\ \hline
             CCSDT-4 & $e^{\hat{T}_1 + \hat{T}_2 + \hat{T}_3}$ & $e^{\hat{T}_1 + \hat{T}_2 + \hat{T}_3}$ & $e^{\hat{T}_1 + \hat{T}_2} + \hat{T}_3$ \\ \hline
             CCSDT & $e^{\hat{T}_1 + \hat{T}_2 + \hat{T}_3}$ & $e^{\hat{T}_1 + \hat{T}_2 + \hat{T}_3}$ & $e^{\hat{T}_1 + \hat{T}_2 + \hat{T}_3}$ \\ \hline
        \end{tabular}
        \caption{This table summarizes the approximations for $e^{\hat{T}}$ the various CCSDT-$n$ methods use in the amplitude equations, along with the approximations used for CCSD and CCSDT.}
        \label{tab:ccsdtn-projections}
    \end{table}

    \begin{table}[H]
        \centering
        \begin{tabular}{|c|c|} \hline
             Method &  Computational Cost \\ \hline
             CCSD & $O(M^6)$ \\ \hline 
             CCSDT-1a & $O(M^7)$ \\ \hline 
             CCSDT-1b & $O(M^7)$ \\ \hline 
             CCSDT-2 & $O(M^7)$ \\ \hline
             CCSDT-3& $O(M^7)$ \\ \hline 
             CCSDT-4 & $O(M^8)$ \\ \hline 
             CCSDT & $O(M^8)$ \\ \hline 
        \end{tabular}
        \caption{The predicted computational run times of various coupled cluster calculations as a function of M, the number of single particle states in the calculation.}
        \label{tab:ccsdtn_runtimes}
    \end{table}


    If we consider the Hamiltonian with at most a two-body force written in normal product form, we get:
    \begin{equation}
        \bar{H} = \bar{H}_0 + \bar{V}_2 = \sum_{pq}\langle p | f | q \rangle \{a^\dagger_pa_q\} + \sum{pqrs} \langle pq | \hat{v}_2 | rs \rangle \{a^\dagger_pa^\dagger_qa_sa_r\}.
    \end{equation}

    We can then write the CCSD equations using this Hamiltonian as (following Ref. \cite{Ref155}):

    \begin{equation} \label{ccsd_e}
        \langle \Phi_0 | \bar{H}(\hat{T}_2 + \frac{1}{2}\hat{T}_1^2) | \Phi_0 \rangle = \Delta E_{CCSD},
    \end{equation}

    and:

    \begin{equation} \label{ccsd_amp_single}
        \langle \Psi_i^a | \bar{H}(\hat{T}_1 + \hat{T}_2 + \hat{T}_1\hat{T}_2 + \frac{1}{2}\hat{T}_1^2 + \frac{1}{3!}\hat{T}_1^3)|\Phi_0 \rangle = 0,
    \end{equation}

    and finally:

    \begin{equation} \label{ccsd_amp_double}
        \langle \Phi_{ij}^{ab} | \bar{H}(1 + \hat{T}_1 + \hat{T}_2 + \frac{1}{2}\hat{T}_2^2 + \hat{T}_1\hat{T}_2 + \frac{1}{2}\hat{T}_1^2\hat{T}_2 + \frac{1}{3!}\hat{T}_1^3 + \frac{1}{4}\hat{T}_1^4|\Phi_0\rangle = 0.
    \end{equation}

    Here we have the CCSD energy equation in Eq. \ref{ccsd_e}, the amplitude equation for the singly excited states in Eq. \ref{ccsd_amp_single}, and the amplitude for the doubly excited states in Eq. \ref{ccsd_amp_double}.
    
    Using the same system, the CCSDT-1a equations are as follows (also following Ref. \cite{Ref155}), with the correlation energy equation being:

    \begin{equation}
        \langle \Phi_0 | \bar{H} (\hat{T}_2 + \frac{1}{2}\hat{T}_1^2) | \Phi_0 \rangle = \Delta E_{CCSDT-1a},
    \end{equation}    

    the singly excited amplitude equation being:

    \begin{equation}
        \langle \Phi_i^a | \bar{H} (\hat{T}_1 + \hat{T}_2 | \hat{T}_3 + \frac{1}{2}\hat{T}_1^2 + \hat{T}_1\hat{T}_2 | \frac{1}{3!} \hat{T}_1^3 | \Phi_0 \rangle = 0,
    \end{equation}

    the doubly exicted amplitude equation being:

    \begin{equation}
        \langle \Phi^{ab}_{ij} | \bar{H} (1 + \hat{T}_1 + \hat{T}_2 | \hat{T}_3 | \frac{1}{2}\hat{T}_1^2 + \hat{T}_1\hat{T}_2 + \frac{1}{2}\hat{T}_2^2 | \frac{1}{3!}\hat{T}_1^3 | \frac{1}{2}\hat{T}_2\hat{T}_1^2 | \frac{1}{4!}\hat{T}_1^4 | \Phi_0 \rangle = 0,
    \end{equation}

    and finally the triply excited amplitude equation being:
    
    \begin{equation}
        \langle \Phi_{ijk}^{abc} | \bar{H}_0\hat{T}_3 + \bar{V}\hat{T}_2 | \Phi_0 \rangle = 0.
    \end{equation}



      We can write the T-amplitudes for the $\hat{T}_3$ operator that will result from a CCSDT-1a calculation as (which is the same as Eq. \ref{t3_from_t2}):

    \begin{equation} \label{ccdt1_eq}
        \epsilon_{ijk}^{abc}t_{ijk}^{abc} = \hat{P}(a/bc|k/ij)\sum_d \langle bc | \hat{v}_2 | dk \rangle t_{ij}^{ad} = \hat{P}(c/ab|i/jk).
    \end{equation}

    For the CCSDT-1a method, we use the unconverged $\hat{T}_2$ amplitudes to calculate the $\hat{T}_3$ amplitudes and converge both sets of amplitudes in an iterative method that requires 10 to 20 steps on average \cite{Ref21}. Another improvement of the CCSDT-1a model over a complete CCSDT calculation is that it avoids the storage of the $\hat{T}_3$ amplitudes. So not only are there reduced computational time requirements, but there is also a reduction in the amount of memory required \cite{Ref21,Ref155}.
.
    Briefly, we can compare CCSD(T) and CCSDT-1a since we will be using both approximations later in this thesis. Timing-wise, CCSD(T) should provide a better cost-performance ratio as its run time is an iterative $O(M^6)$ followed by a non-iterative $O(M^7)$ while CCSDT-1a is a fully iterative $O(M^7)$. Additionally, CCSDT-1a and CCSD(T) should produce similar results, but CCSD(T) has fewer $\hat{T}_3$ terms than CCSDT-1a. However, as both methods have the same number of $\hat{T}_3\hat{T}_3$ coupling terms, CCSD(T) should avoid the exaggeration of the $\hat{T}_3$ effects that can plague CCSDT-1a calculations. Finally, for systems where $\hat{T}_3$ have significant effects, both CCSDT-1a and CCSD(T) provides a reliable method, but for practical applications, the CCSD(T) calculation is preferable as a CCSDT approximation due to run time considerations and the reduction in exaggerating the effects of the $\hat{T}_3$ operator \cite{Ref155}.



