By this point, it has been well established that finding the energy of a many-body system is done by solving the following eigenvalue problem where $|\Psi \rangle$ is a many-body wavefunction:

\begin{equation}
     \hat{H} | \Psi \rangle = E | \Psi \rangle \longrightarrow \langle \hat{H} | \Psi \rangle = E.
\end{equation}

However, this problem can only be solved fully for a few uninteresting systems. Therefore, we have developed our first many-body method, HF theory, capable of finding approximately the energy of a many-body system. However, HF is an independent particle model, and we also wish to include the interactions between the particles in the energy. Thus, we will start by developing many-body perturbation theory (MBPT), a post-Hartree-Fock method, so called because it starts with the Hartree-Fock energy but then adds a correction to account for the interactions between the particles \cite{Ref21, Ref163, Ref164, Ref165}.

Many-body perturbation theory assumes that the Hamiltonian can be split into two pieces, a non-interacting component, $\hat{H}_0$, and an interacting component, $\hat{H}_I$. The interacting component is a small perturbation away from the non-interacting component, thus the method's name. Thus, we have:

\begin{equation}
    \hat{H} = \hat{H}_0 + \hat{H}_I.
\end{equation}

From this definition of the Hamiltonian, we can split the energy of the many-body system into two components: a reference energy, $E_0$, and a correlation energy, $\Delta E_{MBPT}$, leading to:

\begin{equation}
     E = E_{Ref} + \Delta E_{MBPT}.
\end{equation}

The reference energy is defined using the total Hamiltonian and the Fermi vacuum state, $|\Phi_0\rangle$ and the total Hamiltonian as:

\begin{equation}
    E_{Ref} = \langle \Phi_0 | \hat{H} | \Phi_0 \rangle.
\end{equation}

In practice, the reference energy is, in fact, the Hartree-Fock energy, assuming we are using a Fock basis, which we will be using for every calculation in this thesis. Thus the correlation energy is an additional term to the MBPT energy compared to the HF energy.

Now we can define the MBPT correlation energy as the matrix elements formed when the interacting component of the Hamiltonian is applied to a many-body wavefunction as the ket and the Fermi vacuum as the bra, giving:

\begin{equation}
    \Delta E_{MBPT} = \langle \Phi_0 | \hat{H}_I | \Psi \rangle.
\end{equation}

However, solving equation is no simpler than solving the original eigenvalue problem. Thus we will rephrase the MBPT correlation energy as:

\begin{equation} \label{correlation}
    \Delta E_{MBPT} = \sum_{i=1}^\infty \Delta E^{(i)},
\end{equation}

where $\Delta E^{(i)}$ is the i-th order correction to the MBPT correlation energy. We can define the form of the first two corrections as follows. The first order correction to the MBPT energy is:

\begin{equation}
    \Delta E^{(1)} = \langle \Phi_0 | \hat{H}_I | \Psi_0 \rangle.
\end{equation}

The second order correction to the MBPT energy is: 

\begin{equation}
    \Delta E^{(2)} = \frac{1}{4} \sum_{ijab} \frac{\langle ij | \hat{V}_2 | ab \rangle \langle ab | \hat{v}_2 | ij \rangle}{(\epsilon_i + \epsilon_j) - (\epsilon_a + \epsilon_b)}
\end{equation}

Theoretically, there are infinite corrections to the MBPT correlation energy. In practice, we cannot compute infinite corrections to the MBPT energy, and we must truncate Eq. \ref{correlation} to a finite number of terms. However, the form of the correlation energy provides a convenient truncation scheme. The MBPT truncation MBPT1 (many-body perturbation theory to the first order) uses the approximation $\Delta E_{MBPT} \approx \Delta E^{(1)}$, the MBPT truncation scheme MBPT2 (many-body perturbation theory to the second order) uses the approximation $\Delta E_{MBPT} \approx \Delta E^{(1)} + \Delta E^{(2)}$, and so on. This thesis will primarily focus on MBPT2 as the leading MBPT approximation used in these calculations. Any MBPT results presented here will only include one-body and two-body interactions.