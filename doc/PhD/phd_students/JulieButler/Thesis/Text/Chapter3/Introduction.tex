\textit{Ab initio} many-body methods aim to solve the many-body problem starting only from the Hamiltonian of the system and a set of known approximations \cite{Ref6}. Although a proper \textit{ab initio} approach in nuclear physics entails dealing with degrees of freedom that include quarks and gluons, we use \textit{ab initio} here to mean a method that can systematically improve upon the many-body framework devised in the last chapter starting from a given Hamiltonian and the respective laws of motion. Many \textit{ab initio} many-body methods are applied to nuclear physics and other fields, but for this thesis, we will limit the scope to just three of these methods. Hartree-Fock theory (HF) is one of the oldest and simplest many-body methods. However, despite its simplicity and small computational requirements, it is one of the least accurate methods. The other two methods investigated in this chapter are post-Hartree Fock methods, which improve the Hartree Fock result. These two methods are many-body perturbation theory (MBPT) and coupled cluster theory (CC).

This chapter will first explore Hartree-Fock theory as the simplest many-body method and the basis of MBPT and CC. Then, after a brief introduction to post-Hartree Fock methods, MBPT will be investigated, followed by a thorough explanation of coupled cluster theory. 