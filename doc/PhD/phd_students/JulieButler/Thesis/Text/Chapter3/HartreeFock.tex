The first many-body method we will explore in this chapter is the Hartree-Fock theory (HF) \cite{Ref161, Ref162}. HF, one of the earliest many-body methods, is an iterative algorithm that can be used to find the ground state of a system given its Hamiltonian \cite{Ref5}. As an independent-particle model, Hartree-Fock Theory begins with the Slater determinant, defined in Section 2.2 and rewritten here:

\begin{equation}
    \Phi = \frac{1}{N} \begin{vmatrix}
        \psi_1(\vec{x}_1) & \psi_2(\vec{x}_1) & . & . & . & \psi_N(\vec{x}_1) \\
        \psi_1(\vec{x}_2) & \psi_2(\vec{x}_2) & . & . & . & \psi_N(\vec{x}_2) \\
        . & . & . & . & . & . \\
        . & . & . & . & . & . \\
        . & . & . & . & . & . \\
        \psi_1(\vec{x}_\nu) & \psi_2(\vec{x}_\nu) & . & . & . & \psi_N(\vec{x}_\nu) \\
    \end{vmatrix},
\end{equation}

where $\Phi$ is the many-body wavefunction and $\psi_n(\vec{r}_m)$ is the single-particle wavefunction of the $m$-th particle with spatial and spin coordinates described by $\vec{r}_m$. Hartree-Fock Theory uses an iterative algorithm to vary the single-particle wavefunctions, $\psi$ such that the energy of the entire many-body wavefunction, represented by the Slater determinant, is minimized \cite{Ref21}. This minimization can be achieved by solving eigenvalue problems for all single-particle wavefunctions. All of the eigenvalue equations are coupled and have the form:

\begin{equation}
    \hat{f} \psi_i = \epsilon_i \psi_i,
\end{equation}

where $\epsilon_i$ is the single-particle energy and $\hat{f}$ is the single-particle Fock operator.  The single-particle Fock operator, which depends on all of the single-particle wavefunctions, has the following matrix elements:

\begin{equation}
    \langle p | \hat{f} | q \rangle = \langle p | \hat{t} | q \rangle + \sum_i \langle pi | \hat{v}_2 | qi \rangle_{A},
\end{equation}

where $\hat{t}$ is the single-particle kinetic energy operator and $\hat{v}_2$ is the interaction between two particles \cite{Ref4}.  It is important to note that for nuclear problems, interactions beyond the two-body level contribute significantly to the total interaction. We can also define the many-body Fock operator, $\hat{F}$, as:

\begin{equation}
    \hat{F} = \sum_{pq} \langle p | \hat{f} | q \rangle a^\dagger_p a_q.
\end{equation}

For many systems, the solution provided by HF provides an excellent initial approximation for the ground state energy and its corresponding many-body wavefunction. HF can recover approximately 99$\%$ of the ground state energy and approximately 95$\%$ of the corresponding wavefunction for electronic systems \cite{Ref4, Ref21}. However, since HF is an independent particle model, the missing energy comes from the interaction between particles, and therefore this energy must be recovered as well. The need to recover the energy from the interactions between particles, known as the correlation energy, has led to the development of more advanced many-body methods. This chapter will develop two of these, many-body perturbation theory and coupled cluster theory.