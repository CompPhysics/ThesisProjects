\subsection{Homogeneous Electron Gas Conclusion}
In this chapter, we have looked at several methods for removing the basis incompleteness error from CCD calculations of the HEG without performing calculations at a high number of single-particle states. Using the traditional methods of truncating the basis at some number of shells or using a 1/M power law fails to give reasonable predictions for the correlations energies in the complete basis limit, which increases the basis size of the complete calculations that need to be performed. Additionally, the two methods we looked at, which are based on the relationship of the convergence of the MBPT2 and CCD correlation energies, gave better predictions for the correlation energies in the complete basis limit but still failed to accurately reproduce the correlation energies in the complete basis limit for a high number of electrons.

As we began to develop the SRE method, we had a method that could accurately recreate the correlation energies in the complete basis limit for various numbers of electrons and values of $r_s$ with only minimal training data. However, when we developed the SRE method with ridge regression, we got an average percent error of 0.45$\%$ across 70 complete basis limit predictions if the best value of $\alpha$ was chosen. On the other hand, if the best value of $\alpha$ was not used, then the average percent error was infinity. This highlights the drawback of traditional machine learning methods: the abundance of hyperparameters and the need to perform hyperparameter tuning. Even though ridge regression only has one hyperparameter, $\alpha$, finding its best value still requires hyperparameter tuning, which means we must create a validation data set already in the complete basis limit, which takes a long time to generate and reduces the time savings of using the SRE method to predict the correlation energies in the complete basis limit.

Thus we were motivated to use Bayesian machine learning, as these algorithms find the optimized values of their hyperparameters, eliminating the need for hyperparameter tuning and a validation data set. The first Bayesian algorithm we tested was Bayesian ridge regression, which produced all 70 correlation energies in the complete basis limit with an average error of only 0.47$\%$, almost identical to the average ridge regression error. Additionally, since Bayesian ridge regression is a Bayesian algorithm, it produces uncertainties in its predictions, so we have error bars on our results. Since we only needed 16 training points to train the algorithm fully and no validation set, the total time saved using SRE to predict the total basis limit correlation energies instead of fully calculating them was over 3.5 node days, with an error of only 0.47$\%$. Next, we attempted to reduce the amount of training data needed, and in order to accomplish this, we had to use a new Bayesian machine learning algorithm, Gaussian processes (GP). We could predict the correlation energies in the complete basis limit using only ten training points with the GP algorithm. This resulted in an average percent error of 1.16$\%$ across all 70 predictions, leading to over one week of computational time savings.  

Finally, we looked at an initial analysis using the SRE method to predict the CCD correlation energy in the thermodynamic limit. However, this work is still in progress because, when this thesis was written, it was impossible to know the TDL energies for the specific system we are modeling. Thus we cannot calculate the error in our predictions.

To conclude, the SRE method with Bayesian machine learning algorithms can accurately produce the correlations energies of the HEG in the complete basis limit while saving many hours of computational time and producing uncertainties in its results. As the electron gas is a test bed for developing methods that will eventually be applied to more complicated infinite matter systems, the analysis performed in this chapter motivates using the SRE method to predict correlation energies in the complete basis limit for other infinite matter systems.