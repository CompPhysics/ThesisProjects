Before developing the different many-body methods, we must develop a  framework for representing specific quantities found in many-body theory. This framework work is called Dirac bra-ket notation, and it dramatically simplifies many quantum mechanical expressions. It is commonly used in quantum mechanics and many-body physics because it provides a simple way to represent complex quantum mechanical operations as linear algebra problems.

The base quantity of Dirac notation is a ket. Given a column vector, \textbf{x}, which contains n elements, we can denote the vector in ket notations as:

\begin{equation}
    | \textbf{x} \rangle = \begin{bmatrix}
                                x_0 \\
                                x_1 \\
                                x_2 \\
                                . \\
                                . \\
                                . \\
                                x_{n-1}
                        \end{bmatrix},
\end{equation}

where we will begin numbering the elements from 0, leading to the last element having an index of n-1. In general, x$_i$ $\in$ $\mathbb{C}$.

The corresponding bra vector to the ket is simply the Hermitian conjugate of $| \textbf{x} \rangle$:

\begin{equation}
    \langle \textbf{x} | = \begin{bmatrix}
        x_0^* & x_1^* & x_2^* & . & . & . & x_{n-1}^* \end{bmatrix}.
\end{equation}

Note that the bra vector is a row vector instead of a column vector.

%% QUANTUM MECHANICAL SIGNIFICANCE OF BRAS AND KETS
In quantum mechanics, different kets represent the system's different states. These kets differ from representing quantum mechanical states in wave mechanics (where the state is represented as a function) and in matrix mechanics (where the states are represented as basis expansions). However, kets encode the same information as these other representations. In addition, kets allow invariance in the state, meaning that the choice of coordinates and basis can be chosen at any point and quickly changed \cite{Ref5}. For now, the bras and kets will only describe a state with one particle; we will develop a formalism to describe many-particle states as bras and kets later in this section.

It is also important to note that since the ket $|\textbf{x} \rangle$ is a quantum mechanical state, then it belongs to an infinite dimensional Hilbert space, H ($| \textbf{x} \rangle \in H$). The corresponding bra, $\langle \textbf{x} |$, also belongs to the same Hilbert space \cite{Ref5}.


Hilbert spaces are closed under linear combinations, which gives rise to the superposition of quantum states \cite{Ref5}. Dirac notation also provides an easy way to represent states in a superposition, meaning that they are simultaneous in two or more states, and the same state is determined upon system measurement. Given two states $| \textbf{x} \rangle$ and $| \textbf{y} \rangle$, we can say that state $| \textbf{z} \rangle$ is in a superposition of $| \textbf{x} \rangle$ and $| \textbf{y} \rangle$ with the following:

\begin{equation}
    | \textbf{z} \rangle = \alpha| \textbf{x} \rangle + \beta | \textbf{y} \rangle,
\end{equation}

where $\alpha$ and $\beta$ are complex numbers.  The bra of the above state $\langle \textbf{z} |$ can be denoted as: 

\begin{equation}\label{superposition_bra}
    \langle \textbf{z} | = \alpha^* \langle \textbf{x} + \beta^* \langle \textbf{y} |.
\end{equation}

Note that in Eq. \ref{superposition_bra}, the coefficients of the kets are $\alpha^*$ and $\beta^*$ (i.e., complex conjugates) instead of just $\alpha$ and $\beta$.

The coefficients $\alpha$ and $\beta$ have a physical interpretation. When the state $| \textbf{z} \rangle$ is measured and collapses into either $| \textbf{x} \rangle$ or $| \textbf{y} \rangle$, there is an $|\alpha|^2$ probability that the state will be found in $| \textbf{x} \rangle$ and a $|\beta|^2$ chance that it will be found in $| \textbf{y} \rangle$.  Note that the notation $|\alpha|^2$ represents the complex modulus of $\alpha$. Since both $|\alpha|^2$ and $|\beta|^2$ represent probabilities that the two different outcomes of measuring state $| \textbf{z} \rangle$ will occur then $|\alpha|^2 + |\beta|^2 = 1$.


\subsection*{Inner Product}

For kets $| \textbf{x} \rangle$ and $| \textbf{y} \rangle$, the inner product between the two will be defined as:

\begin{equation}
    \langle \textbf{y} | \textbf{x} \rangle = y_0^*x_0 + y_1^*x_1 + y_2^*x_2 + ... + y_{n-1}^*x_{n-1}.
\end{equation}

It is important to note that$\langle \textbf{y} | \textbf{x} \rangle$ = $\langle \textbf{x} | \textbf{y} \rangle$.  Also note that $\langle \textbf{y} | \textbf{x} \rangle$ will sometimes be referred to as the overlap between the states \textbf{x} and \textbf{y}.

%% ExPLAIN ORTHONORMAL BASIS

Many quantum mechanics bases are orthogonal and normalized and therefore called orthonormal. However, two restrictions are placed on most bases in many-body theory. First, inner products are used to ensure that a basis is both orthogonal and normalized. In terms of inner products, this means that for each state, \textbf{i}, in the basis, the inner product of the state with itself must be 1 to be normalized:

\begin{equation}
    \langle \textbf{i} | \textbf{i} \rangle = 1,\ \forall\ \textbf{i}.
\end{equation}

The second restriction is that the basis is orthogonal. This means that the inner product of a state \textbf{i} with another state \textbf{j} must be zero:

\begin{equation}
    \langle \textbf{i} | \textbf{j} \rangle = 0,\ \forall\ \textbf{i} \neq \textbf{j}.
\end{equation}

For a basis that is both normalized and orthogonal, the basis is said to be orthonormal, and the conditions can be represented in compact delta notation as:

\begin{equation}
    \langle \textbf{i} | \textbf{j} \rangle = \delta_{ij},\ \forall\ \textbf{i}, \textbf{j}.
\end{equation}

\subsection*{Outer Product}

The outer product between $|\textbf{x} \rangle$ and $|\textbf{y} \rangle$ can be computing using the following formula:

\begin{equation}
    |\textbf{x} \rangle \langle \textbf{y} | = \begin{bmatrix}
        x_0y_0^* & x_0y_1^* & x_0y_2^* & . & . & . & x_0y_{n-1}^* \\
        x_1y_0^* & x_1y_1^* & x_1y_2^* & . & . & . & x_1y_{n-1}^* \\        
        x_2y_0^* & x_2y_1^* & x_2y_2^* & . & . & . & x_2y_{n-1}^* \\   
        . & . & . & . & . & . & . \\
        . & . & . & . & . & . & . \\
        . & . & . & . & . & . & . \\        
        x_{n-1}y_0^* & x_{n-1}y_1^* & x_{n-1}y_2^* & . & . & . & x_{n-1}y_{n-1}^* \\           
    \end{bmatrix}.
\end{equation}

Another restriction on a quantum mechanical basis is that it is complete. We can define the completeness relation for a general basis as: 

\begin{equation}\label{completeness}
    \sum_i |i\rangle \langle i | = \textbf{I},
\end{equation}

where \textbf{I} is the identity matrix in Hilbert space H.


\subsection*{Tensor Product}

Tensor products are essential when describing many-particle systems from one-particle state vectors (or wavefunctions). In wave mechanics, we would describe the many-particle wavefunction with A particles as

\begin{equation}\label{many_body_wavefunction}
    \Psi (\textbf{x}_1, \textbf{x}_2, .... \textbf{x}_A) = \psi_{p_1}(\textbf{x}_1)\psi_{p_2}(\textbf{x}_2)...\psi_{p_A}(\textbf{x}_A),
\end{equation}

where $\psi_{p_i}$ represent one-particle wave functions and $\textbf{x}_i$ are the single-particle degrees of freedom (usually position and spin) \cite{Ref5}. These are called product states, forming a complete A-body basis \cite{Ref5}. In Dirac notation, we can also represent a many-particle wavefunction similarly, but in this case, we will use tensor products \cite{Ref5}:

\begin{equation}\label{many_body_ket}
    |\Psi\rangle = |\psi_{p_1} \psi_{p_2} ... \psi_{p_A} \rangle = |\psi_{p_1}\rangle \otimes |\psi_{p_2}\rangle \otimes ... \otimes |\psi_{p_A}\rangle.
\end{equation}

In Dirac notation, we can represent the complete A-body basis as:

\begin{equation}
    |\Psi \rangle = \sum_{p_1,...p_A} d_{p_1...p_A}|\psi_{p_1}...\psi_{p_A}\rangle,
\end{equation}

where $d_{p_1...p_A} = \langle \psi_{p_1} ... \psi_{p_A} | \Psi \rangle$ is the overlap between the states.

Note that the order of the single particle wavefunctions matters in either Eq. \ref{many_body_wavefunction} or in Eq. \ref{many_body_ket}. It is possible to exchange the order of the single particle wavefunctions, which will be discussed in the next section.

\subsection*{Operators, Expectation Values, and Matrix Elements}

In Dirac notation, operators are matrices typically represented using hat notation: $\hat{o}$. To find the expectation value of this operator in Dirac notation, it is placed in between a bra and a ket:

\begin{equation}
    \langle \psi | \hat{o} | \phi \rangle ,
\end{equation}

where $\psi$ and $\phi$ can be the same or different wavefunction depending on the operator. This combination of bra, operator, and ket is also called a matrix element. Several operators will be defined in the next section once more framework exists to support them, but we can define one operator here. The exchange operator, $\hat{P}$, is defined such that:

\begin{equation}
    \hat{P}|\psi_1\psi_2\rangle = \pm |\psi_2\psi_1\rangle,
\end{equation}

where $\psi_i$ are single particle wavefunction \cite{Ref12}.  A plus sign on the right side of the equation indicates the particles are bosons, and a minus sign indicates the particles are fermions; in this work, we will only consider particles that are fermions. Now, we can define the following two matrix elements using the exchange operator and the orthonormal rules extended to a many-body basis:

\begin{equation}
    \langle \psi_1 \psi_2 | \hat{P} | \psi_1 \psi_2 \rangle = -\langle \psi_1 \psi_2 | \psi_2 \psi_1 \rangle = 0.
\end{equation}

and

\begin{equation}
    \langle \psi_1 \psi_2 | \hat{P} | \psi_2 \psi_1 \rangle = -\langle \psi_1 \psi_2 | \psi_1 \psi_2 \rangle = -1.
\end{equation}

Note that the exchange operator can work on more than two-particle states. If there are more than two particles in the many-body wavefunction, the particles the exchange operator is acting on are denoted as subscripts:

\begin{equation}
    \hat{P}_{12}|\psi_1 \psi_2 \psi_3 ... \rangle = -|\psi_2 \psi_1 \psi_3...\rangle .
\end{equation}


\subsection*{Slater Determinants}

The wavefunction for an A-particle system that solves Schr\"{o}dinger's equation precisely can be expressed as an expansion of Slater determinants which span a complete, usually infinite, basis \cite{Ref1}. However, usually, the single-particle space must be truncated only to contain M single-particle states due to computational limitations, for which an A-particle system leads to $\binom{M}{A}$ determinants \cite{Ref1}.

We can represent out many-body wavefunction, $\Phi$, as the following determinant:

\begin{equation}
    \Phi = \frac{1}{N} \begin{vmatrix}
        \psi_1(\vec{x}_1) & \psi_2(\vec{x}_1) & . & . & . & \psi_N(\vec{x}_1) \\
        \psi_1(\vec{x}_2) & \psi_2(\vec{x}_2) & . & . & . & \psi_N(\vec{x}_2) \\
        . & . & . & . & . & . \\
        . & . & . & . & . & . \\
        . & . & . & . & . & . \\
        \psi_1(\vec{x}_\nu) & \psi_2(\vec{x}_\nu) & . & . & . & \psi_N(\vec{x}_\nu) \\
    \end{vmatrix},
\end{equation}

where $\Phi$ is the many-body wavefunction and $\psi_n(\vec{r}_m)$ is the single-particle wavefunction of the n-th particle with spatial and spin coordinates described by $\vec{r}_m$. Each column represents a different single-particle wavefunction, and each row represents a different set of spatial and spin coordinates. Thus represented in the Slater determinant represents every particle being in every configuration.