We now see how many-body states and operators can be written in second quantization. Though second quantization simplifies many-body calculations, some can still become long and complicated with strings of creation and annihilation operators. Normal ordering simplifies these calculations by arranging chains of creation and annihilation operators such that annihilation operators are all on the right and creation operators are all on the left. We can thus define a typical ordered sequence of operators as:

\begin{equation}
    n[\hat{o}_1 ... \hat{o}_m ] = (-1)^R a^\dagger_{i}...a\dagger_ja_{j+1}...a_m,
\end{equation}

where $n[...]$ defines a normal ordered sequence of operators, $\hat{o}_i$ represents either a creation or annihilation operator, and $R$ handles the sign change for the permutations—each time two operators exchange location, $R$ increments by 1. Normal ordering is essential because it makes determining which expressions and matrix elements will yield a non-zero result easier. For example, consider the following matrix element.

\begin{equation}
    \langle 00101|a_4a^{\dagger^1}a_0|10001\rangle
\end{equation}

Written in its current form, it is difficult to tell if this matrix element will be non-zero without evaluating each operator. However, if we normal order the operators, it becomes more apparent.

\begin{equation}
    \langle 00101|n[a_3a^{\dagger^1}a_0]|10010\rangle = -\langle 00101|n[a^{\dagger^1}a_4a_0]|10001\rangle = \langle 00101|a^{\dagger^1}a_0a_4|10001\rangle
\end{equation}

Now it is obvious that the matrix element will yield zero as $a^{\dagger^1}$ acting on the bra will return zero.

All operators that have been derived so far and which will be derived in the coming chapters will be written in a normal ordered form as this will drastically simplify the extensive calculations we will be performing.  

A closely related concept to normal ordering is Wick's theorem, which allows for these complex many-body states and operators to be simplified further. However, Wick's theorem will not be used in future derivations, so it will not be derived here. For a thorough explanation and derivation of Wick's theorem, see Ref. \cite{Ref12} and \cite{Ref21}.