Modeling a system containing many-interacting particles can be a complex problem, even in classical mechanics. This problem becomes even more complicated when the quantum mechanical properties of the particles are taken into account, as this drastically increases the system's complexity. Furthermore, when the particles are protons and neutrons, this problem becomes exceedingly tricky due to the need to include complex nuclear interaction. However, it is still possible to make meaningful studies of nuclear many-body systems using many-body methods \cite{Ref13}.

We will start our discussion of many-particle systems by defining the many-body Schr\"{o}dinger equation used to find the system's energies. The time-dependent Schr\"{o}dinger equation, shown in Eq. \ref{time_dependent}, can be used to find all of the possible wavefunctions of the many-body system. Each wavefunction describes the system in a different configuration, so a many-body system has many wavefunctions that need to be found. This can be represented as:

\begin{equation} \label{time_dependent}
	i\hbar \frac{\partial \Phi(\vec{r}, t)}{\partial t} = \hat{H}\Phi(\vec{r}, t),
\end{equation}

where $\hbar$ is the reduced Planck's constant, $\Phi$ is a wavefunction of the system, and $\hat{H}$ is the many-body Hamiltonian, which is unique to the system being studied. In this thesis we will use the common notation where capital Greek letters ($\Phi$ and $\Psi$) represent many-body wavefunctions but lowercase Greek letters ($\phi$ and $\psi$) represent single-particle wavefunctions. In many cases, such as in this thesis, we remove the time dependence from the many-body problem to reduce its complexity. This reduces the Schr\"{o}dinger equation to its time-independent form:

\begin{equation} \label{time_independent}
	\hat{H}\Phi(\vec{r}, t) = E\Phi(\vec{r}, t),
\end{equation}

where $E$ is the system's energy if it is in the state represented by the wavefunction $\Phi$. Removing the time dependence reduces the Schr\"{o}dinger equation to an eigenvalue problem where the eigenvalues are the energies of the many-body system. The eigenvectors are the corresponding wavefunctions (or state vectors).

Next, we need to define the Hamiltonian of our system. The Hamiltonian for a many-particle system can be represented generally in the form shown in Eq. \ref{hamiltonian} \cite{Ref8}:

\begin{equation} \label{hamiltonian}
	\hat{H} = \hat{K} + \hat{V}_2 + \hat{V}_3 + ...\ .
\end{equation}


In the above equation, $\hat{K}$ is the kinetic energy of the system, $\hat{V}_{2}$ represents the two-body interaction and $\hat{V}_3$ represents the three-body interaction. For a system containing A particles, there are a maximum of A-body interactions, though, in practice, the number of interactions (which corresponds to the number of matrix elements in the Hamiltonian) is usually limited. It is common to limit the many-body Hamiltonian to the 2-body interaction level, sometimes the 3-body interaction level. It is uncommon to include interactions higher than the 3-body interaction level, especially for large systems or nuclear systems with complex interactions, since the complexity of the interactions increases drastically with the number of particles involved. Each sum also includes an increasing number of particles at higher interaction levels. For example, the kinetic energy only has A terms, but the two-body interaction sum already has $\frac{1}{2}$A(A-1) terms \cite{Ref13}.

Once the Hamiltonian has been defined, the next step in the many-body problem is to obtain the eigenvalues and the eigenfunctions from $\hat{H}$, which is a matrix. The eigenvalues of $\hat{H}$ are the energies of the system, and the eigenvectors are the wavefunctions of the system and are thus the solutions to the time-independent Schr\"{o}dinger equation shown in Eq. \ref{time_independent}. However, in practice, Eq. \ref{time_independent} can only be solved exactly for small, uninteresting systems. Thus we must turn to many-body methods to approximately solve for the energies of a many-body system \cite{Ref1, Ref3, Ref13}. In the following two sections, we will develop the notation (Dirac notation) and a way to represent the many-body basis that is more convenient (occupation representation and second quantization). Then we will revisit the many-body Hamiltonian once these tools have been developed before closing the chapter. Finally, we will develop other formalisms, including normal ordering and diagrammatic methods.