%%% THESIS CONCLUSIONS %%%

% SRE as a valid extrapolation method
The sequential regression extrapolation (SRE) method developed here based on Bayesian regression algorithms is an accurate and valuable extrapolator for removing basis incompleteness errors from coupled cluster calculations of infinite matter systems. Furthermore, when the infinite matter system needs to be taken to the thermodynamic limit, it is possible to use SRE to perform this task. Since the SRE algorithm uses training data taken from calculations at small numbers of single-particle states to predict the correlation energy at many single-particle states, the SRE algorithm can offer significant time savings over performing fully converged correlation energy calculations. As shown in this thesis, using the SRE algorithm to save over 100 node hours in the calculations of just one correlation energy is possible. Furthermore, this huge time savings does come with a loss in the accuracy of performing the whole savings. However, the average percent error between the SRE prediction and the fully calculated result was typically less than 1$\%$, making it quite a slight difference compared to the large amount of computational time that has been saved.

% From the CCD vs. CCDT side
Furthermore, besides developing the SRE method, we also compared different methods of performing coupled cluster calculations of systems of infinite nuclear matter. First, we compared the results from two different interactions: the Minnesota potential, a toy interaction, and chiral NNLO potentials, which are much more realistic. By comparing these two, we learned that they differ quite significantly around densities of infinite nuclear matter that are similar to nuclear densities and that calculations containing NNLO potentials do take much longer to compute compared to a Minnesota potential applied to the same system. Furthermore, we could also compare the difference between the coupled cluster approximations CCD, CCDT-1, and CCD(T) on calculations of infinite matter systems. We found that the triples approximations give significant results when compared to the CCD approximation, so they are worth performing even though they provide an increase in computational time.

%% SIZE OF MACHINE LEARNING SYSTEM
Though it has been mentioned throughout this thesis, it is essential to emphasize the size of the training data sets used in this work. This work's most extensive training data set used only 16 training points, and the smallest training set used only 3 points. Some areas of physics could be faster to adopt machine learning because of the vast amount of training data that some machine learning algorithms require. However, this work has shown that accurate machine learning predictions can be made with very few training points, thus encouraging using machine learning as a tool in fields with small data sets.

%%% FUTURE WORKS %%%

\subsection*{Possible Future Works}
% Full triples and different proton fractions
A few notable future works stem directly from the work presented here. First, while we showed results from both the CCD< CCDT-1, and CCD(T) approximations, we did not have the capabilities to produce coupled cluster correlation energies using a complete CCDT calculation. This is mainly due to the high computational costs of a complete triple calculation ($O(M^8)$), but advancements, such as those made in Refs. Furthermore, ADD REFERENCES HERE are making this a much more achievable goal for the near future.

Additionally, while we only looked at pure neutron matter and symmetric nuclear matter here, there are other proton fractions of interest. If we want to model the equation of the state of nuclear matter thoroughly, then we need to be able to accurately predict the properties of neutron matter at proton fractions beyond just 0.0 and 0.5.

% PARAGRAPH ABOUT NOT BEING LIMITED TO INFINITE SYSTEMS
While truncating the number of particles in a calculation is limited to infinite matter and other large systems, truncations occur in every \textit{ab initio} many-body calculation. Basis truncation is especially common and occurs in almost every calculation except some simple toy models. The last part of this thesis will be dedicated to exploring some possible future applications of the SRE methodology which has been developed.

% PARAGRAPH ABOUT CC CALCULATIONS OF THE NUCLEUS
An extension of the work presented in this thesis is to apply the SRE method to remove basis incompleteness errors from coupled cluster calculations of nuclei. Though nuclei are finite systems and the number of nucleons in the system generally does not need to be truncated, the number of single-particle states is still truncated, leading to a need to extrapolate to an infinite model space \cite{Ref6}. This is especially true for heavy nuclei and nuclei that are weakly bound \cite{Ref6}. There are methods to perform these extrapolations on nuclei calculations, but when using the harmonic oscillator basis, which mixes the ultraviolet and infrared cutoffs, these extrapolation methods can fail \cite{Ref6}. Machine learning has been used to perform similar extrapolations (see Ref. \cite{Ref6}, \cite{Ref22}, \cite{Ref23}, for example), but these were performed with neural networks and thus incurred all of the problems that were experienced with neural networks in this thesis.

%% MBPT CALCULATIONS AND HIGHER ORDERS
Additionally, the SRE method has no reason to be restricted to only predicting CC energies using MBPT2 energies. Extending the SRE method to other many-body methods should also be possible. 

% PARAGRAPH ABOUT FCI
One of the most accurate yet restricted \textit{ab initio} many-body methods is full configuration interaction theory (FCI), which uses a variationally optimized linear combination of the full set of Slater determinants. FCI is used in nuclear physics and electronic-structure theory, but its complexity limits it to only the smallest of systems. The ground state energy, which FCI finds, is the lowest (variationally) and most accurate that can be achieved. If an infinite single particle basis is used, FCI produces the solutions to the Schr\"{o}dinger equation. However, due to computational limitations, FCI calculations must be performed with a finite basis, meaning they will fail to retrieve the total energy \cite{Ref1}. However, it is possible that SRE could be applied to FCI calculations using, for example, Hartree Fock calculations as the "fast method" to quickly and accurately extrapolate FCI results to the infinite basis limit, thus recovering the Schr\"{o}dinger equation results.

 % APPLICATIONS OUTSIDE OF NUCLEAR PHYSICS
While most of this thesis, except for the sections on the electron gas, have been devoted to nuclear physics applications, \textit{ab initio} many-body methods occur in many other fields besides nuclear physics. For example, coupled cluster theory, and other many-body methods, are prevalent in other fields of physics and quantum chemistry. Therefore, the development of the SRE method should improve calculations outside of the realm in which it was developed.