As mentioned throughout this chapter, several truncations and approximations must be made to study these infinite matter systems in a computational framework. Though computational limitations require these truncations and approximations, they introduce errors into the calculations, making them undesirable \cite{Ref1, Ref2}. Though it is impossible to remove every error that comes from the various truncations and approximations, this thesis will focus on mitigating the effects of three truncation errors: the error that results from truncating the number of single-particle states in the basis, the error that results from truncating the number of particles and volume of the infinite matter system, and the error that results from truncating the coupled cluster correlation operator, $\hat{T}$.

% truncation of M basis incompleteness error
As discussed in Section 2 of this chapter, the number of single-particle states ($M$) in the calculation is truncated due to truncating the total number of energy shells allowed in the system. This introduces an error in the calculation called the basis incompleteness error. The basis incompleteness error can be mitigated by increasing the number of single-particle states (or energy shells) in the calculation. Unfortunately, this also drastically increases the computational time and resources needed. Coupled cluster theory has polynomial time scaling with respect to $M$, which, while being better than other many-body methods, still means that computational times can be prohibitive at large values of $M$.  Additionally since increasing the number of single-particle states in the system increases the number of matrix elements in the calculations, computational resources are also increased.

% truncation of N finite size effects
While infinite matter systems theoretically contain infinite particles over an infinite volume, these values must be truncated to finite values computationally \cite{Ref8}. This introduces an error in the calculations called finite size error. Since the number of particles in the system and the volume of the system are related by a fixed density ($\rho_0 = N/\Omega$), it is sufficient to increase $N$ to the infinite limit while keeping the density constant. However, the same problem occurs when attempting to increase $M$ since coupled cluster theory also has polynomial time scaling with respect to the number of particles in the system; however, the power tends to be higher with particles than with single-particle states. Additionally, matrix elements involving occupied states (or particles) are computationally more complex than those involving only unoccupied states. Therefore, adding more particles to the system increases the computational resources needed than additional single-particle states.

% reiterate the truncation of coupled cluster 
As discussed in the previous chapter, the coupled cluster correlation operator is:

\begin{equation} \label{T_repeat}
	\hat{T} = \sum_{m=1}^N \hat{T}_m,
\end{equation}

where $N$ is the total number of particles in the system and $\hat{T}_m$ represents the $m$-particle $m$-hole excitation operator. We can make this operator specific to infinite matter systems by making two changes. First, infinite matter systems contain particles, so $N = \infty$. Second, the 1-particle 1-hole excitation operator, $\hat{T}_1$, will always be zero for infinite matter systems due to symmetry. Therefore, we can write the cluster operator for infinite matter systems as:

\begin{equation} \label{T_repeat2}
	\hat{T} = \sum_{m=2}^\infty \hat{T}_m.
\end{equation}

Also, as discussed in the previous chapter, the correlation operator is truncated in practice by setting excitation operators over a certain level to zero. In this thesis, the coupled cluster results will either be calculated using coupled cluster doubles (CCD) where $\hat{T} \approx T_2$ or approximations to coupled cluster doubles triples (CCDT) where $\hat{T} \approx T_2 + T_3$. Since the full cluster operator is not used in the calculations, this does reduce the accuracy of the calculations. However, in this thesis, we will not be extrapolating to the complete coupled cluster calculation since this would require performing coupled cluster calculations at more truncation levels. However, we will attempt to improve the accuracy of infinite matter calculations by predicting the CCDT result from the CCD result and the CCD result from the MBPT2 result, thus improving the accuracy while keeping computational time and resources small.


However, in the infinite matter calculations, there are sources of error from approximations and truncations that are not addressed in this thesis but could be the subject of future work. Some of them will be discussed briefly here. First, the Coulomb interaction, the predominant interaction in the homogeneous electron gas, is a long-range interaction that acts over an infinite distance. Truncating the volume of the electron gas to a finite size also truncates the Coulomb interaction. It is important to note that this is a finite size error occurring in the HEG but not in infinite nuclear matter \cite{Ref4}. This means that studies of the HEG at finite sizes will face additional complications compared to studies of infinite nuclear matter \cite{Ref4}. In addition to truncating the coupled cluster correlation operator, not all n-body interactions are included in the calculation because the Hamiltonian is also truncated. In this work, calculations are limited to including, at most, the 2-body or 3-body matrix elements. However, the highest level of matrix elements contributing to the system is $N$-body, where $N$ is the total number of particles. Finally, while the Coulomb interaction, the primary interaction of the electron gas has a simple closed-form representation that can be calculated precisely, the primary interaction in the infinite nuclear matter is the nuclear interaction, dominated by the strong force. However, the nuclear interaction has no simple closed-form solution and cannot be represented precisely. Therefore all nuclear interactions are approximations with various levels of accuracy compared to real nuclear interaction.

%% PREVIOUS WORKS TO CORRECT THESE ERRORS WITH TRADITIONAL EXTRAPOLATION