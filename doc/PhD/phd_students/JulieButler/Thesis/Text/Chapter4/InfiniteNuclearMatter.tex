Infinite nuclear matter is defined as a system containing infinite nucleons (protons and neutrons) that only interact via the nuclear forces \cite{Ref8}. Studies of infinite nuclear matter are essential for understanding the matter within dense astronomical objects such as neutron stars \cite{Ref3}. Neutron stars are exciting because they offer insights into nuclear processes and astrophysical observables. However, neutrons stars also contain matter that spans several orders of magnitude and contain many different compositions of matter \cite{Ref3,Ref35,Ref36,Ref37,Ref38,Ref39,Ref41}. Neutron star matter occurs at densities of 0.1 fm$^{-3}$ or greater and consists of various fractions of neutrons, protons, electrons, and muons. These particles exist in beta equilibrium ($\beta$-equilibrium) governed by the weak force.

Studies of infinite nuclear matter are also focused on determining the equation of state (EoS) . When considering applications to neutron stars, the EoS can help determine the mass range, the relationship between the star's mass and its radius, the thickness of the star's crust, and the rate at which the star will cool down \cite{Ref3}. Determination of the EoS also links neutron stars to the neutron skin in atomic nuclei and the symmetry energy of nuclear matter . Symmetry energy is crucial because it relates to the difference between proton and neutron radii in nuclei \cite{Ref8}.

Solving the EoS depends on our ability to solve the many-body problem for infinite nuclear matter \cite{Ref3,Ref8}. The nuclear matter has been of interest in many-body studies since the early days of the field (see Ref. \cite{Ref51} for a review of these early studies). These early calculations were performed with Brueckner-Bethe-Goldstone theory (\cite{Ref52,Ref53}). However, modern many-body studies of nuclear matter are performed with a varied of methods, including coupled-cluster theory (CC) (\cite{Ref3,Ref4,Ref5,Ref8,Ref9,Ref16,Ref43,Ref47}), Monte Carlo methods (\cite{Ref33,Ref54,Ref55,Ref56,Ref57,Ref58,Ref59,Ref60,Ref61}), Green's function methods (\cite{Ref50,Ref62,Ref63,Ref64}) and methods from the renormalization group theory family (\cite{Ref48,Ref3,Ref65}). The coupled-cluster theory will be the many-body method of interest for the remainder of this thesis. Coupled-cluster calculations of nuclear matter date back to the 1970s and 1980s \cite{Ref8}.

One property of infinite nuclear matter we are interested in is the proton fraction, which is defined as 
\begin{equation}\label{proton_fraction}
    x_p = \frac{\rho_p}{\rho},
\end{equation}
where $\rho_p$ is the density of protons in the matter, $\rho_n$ is the denisty of neutrons in the matter, and $\rho$ = $\rho_p$ + $\rho_n$ is the total density of the infinite nuclear matter \cite{Ref3}.  Defining different proton fractions defines different types of infinite nuclear matter systems. For example, if $x_p$ = 0, the infinite nuclear matter system contains only neutrons. We will refer to this system as pure neutron matter (PNM). If $x_p$ = 1/2, then the system contains an equivalent number of protons and neutrons, and this system is called symmetric nuclear matter (SNM). 

From the proton fraction, we can define symmetry energy as the difference between the energy for pure neutron matter and symmetric nuclear matter at a set density. \cite{Ref3}

\begin{equation}
    E_{sym}(\rho) = E(\rho, x_p=0) - E(\rho, x_p=1/2) 
\end{equation}

All energies for infinite nuclear matter calculations will be, by convention, reported in units of MeV.