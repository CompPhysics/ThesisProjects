The properties of infinite nuclear matter and finite nuclear systems are determined by the nuclear forces that govern the interactions between the nucleons. These nuclear forces are given by two fundamental forces: the strong nuclear force, which binds the nucleons together, and the electromagnetic force, which causes repulsion between the protons. The strong force is stronger over short distances (which causes nuclei to hold together), but the electromagnetic force is a long-range force acting over large distances. The nature of the strong force needs to be better understood, and this can make it challenging to model the nuclear forces in theoretical calculations \cite{Ref200, Ref201}. Because of this, the interaction used to model the nuclear forces in this notebook is the "toy" interaction called the Minnesota potential, which is defined in Ref. \cite{Ref166} and used in, for example, Refs \cite{Ref3, Ref5, Ref167} to model the nuclear interaction. The Minnesota potential is a local, nucleon-nucleon-only, moderately soft potential reproducing the nucleon-nucleon effective range parameters. It provides a reasonable approximation for the binding energies of light nuclei, but it is a simple interaction that is computationally inexpensive compared to more realistic nuclear interactions. It should be noted that the Minnesota potential is a phenomenological interaction, meaning that it was constructed by fitting experimental data instead of being derived from first principles. The matrix elements for the Minnesota potential are given by:

$$V_{i,j} = \frac{1}{2}[V_R + \frac{1}{2}(1+P^\sigma_{ij})V_t + \frac{1}{2}(1-P^\sigma_{ij})V_s](1+P^r_{ij}).$$

In the above equation, $P^\sigma$ is the spin exchange operator, $P^r$ is the space exchange operator, and V$_R$, V$_t$, and V$_s$ are given by the following equations, where r$_{ij}$ is the distance between two nucleons and the constants are found by fitting to experimental data.

$$V_R = V_{0R}e^{-\kappa_Rr_{ij}^2}$$

$$V_t = -V_{0t}e^{-\kappa_tr_{ij}^2}$$

$$V_t = V_{0s}e^{-\kappa_sr_{ij}^2}.$$

Though the Minnesota potential is computationally inexpensive and captures important aspects of the nuclear interactions, modern advancements have improved our models of nuclear forces \cite{Ref20, Ref122, Ref43, Ref200, Ref201, Ref202, Ref203, Ref204, Ref123, Ref125,Ref129, Ref130, Ref131, Ref132, Ref133, Ref134, Ref138}. The second nuclear force used to model infinite nuclear matter in this thesis, which is a more model interaction, is derived from practical field theory (EFT) \cite{Ref3} \cite{Ref38, Ref16, Ref42, Ref43, Ref44, Ref45, Ref46, Ref47, Ref48, Ref49, Ref50, Ref200, Ref201}. The nuclear forces derived from EFT have an advantage over other nuclear forces in that the two-body and the many-body interactions can be derived in a mutually consistent matter \cite{Ref8}. Much recent progress has been made in deriving the nuclear forces based on chiral EFT and the nuclear Hamiltonian for many nucleonic matter calculations, not using forces from chiral EFT, including both nucleon-nucleon (NN) and three-nucleon (3N) forces \cite{Ref123, Ref129, Ref138, Ref139, Ref146, Ref122, Ref131}. For the results presented herein, we will use the parametrization NNLO$_{opt}$ for the NN and local 3N interactions.  

It should be noted that implementing the 3N forces in the single particle basis used for infinite nuclear matter calculations is much simpler than implementing them in the harmonic oscillator basis commonly used for nuclei calculations \cite{Ref9}. However, the large number of matrix elements needed to compute 3N forces is still the computationally limiting factor in these calculations \cite{Ref9}.

The specific chiral potentials used in this work are detailed in Refs. \cite{Ref200} and \cite{Ref201}.  They have optimized $\Delta$-full interactions and are calibrated with nuclear matter properties \cite{Ref200}. The $\Delta$-full interaction is detailed in Refs. \cite{Ref202, Ref203, Ref204, Ref205, Ref206, Ref207}.  The interaction employs a standard non-local regulator function of the form:

\begin{equation}
    f(p) = e^{(p/\Lambda)^{2n}},
\end{equation}

where $p$ is the relative momentum, $n$ is 4, and $\Lambda$ is 394 MeV. We are using these interactions at the next-to-next-to-leading order (NNLO). In this order, 17 low-energy coefficients (LECs) parameterize the interaction and whose values are given in Ref. \cite{Ref200}. This leads to a form of the Hamiltonian, which can be written as:

\begin{equation}
    H(\alpha) = h_0 + \sum_{i=1}^{N_{LECs}=17}\alpha_ih_i,
\end{equation}

where $h_0 = t_{kin} + v_0$,  $t_{kin}$ is the kinetic energy, $v_0$  is a constant potential that does not depend on the LECs, $\vec{\alpha}$ is a vector that denotes all of the LEC.  Of the two interactions used in this thesis to model nuclear forces, the optimized $\Delta$-full interaction is more accurate than the Minnesota potential, but has more complex matrix elements which leads to longer run times.