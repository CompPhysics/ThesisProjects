We can begin to define a single particle basis for infinite matter calculations in three dimensions by defining the single particle wavefunction as plane waves with the form:

\begin{equation} \label{im_wavefunction}
	\psi_{\vec{k},\sigma} = \frac{1}{\sqrt{\Omega}} e^{i\vec{k}\cdot\vec{r}}\xi_\sigma,
\end{equation}

where $\xi_\sigma$ has two possible values corresponding to spin-up particles:

\begin{equation} \label{spin_up}
	\xi_{\sigma=\frac{1}{2}} = \begin{bmatrix}
									1 \\
									0 \\
								\end{bmatrix},
\end{equation}

or spin-down particles: 

\begin{equation} \label{spin_down}
	\xi_{\sigma=-\frac{1}{2}} = \begin{bmatrix}
									0 \\
									1 \\
								\end{bmatrix}.
\end{equation}


Additionally, in Eq. \ref{im_wavefunction}, $\Omega = L^3$ corresponds to the volume of the infinite matter system. The limit $L \rightarrow \infty$ is taken to accurately simulate an infinite matter system after the various expectation values have been calculated \cite{Ref3, Ref13, Ref1}.

When using the plane waves as the single particle wavefunctions and assuming periodic boundary conditions, we can define the single particle momentums to be:

\begin{equation} \label{sp_momentum}
	k_i = \frac{2\pi}{L}n_i,
\end{equation}

where $i$ = $x$, $y$, or $z$ and n$_i$ = 0, $\pm$1, $\pm$2, ... \cite{Ref3, Ref1}.  From the momentum numbers, we can define the kinetic energy of an infinite matter system in second quantization to be:

\begin{equation} \label{im_kinetic}
	\hat{T} = \sum_{\vec{p}\sigma_p} \frac{\hbar^2k_p^2}{2m}a^\dagger_{\vec{p}\sigma_p}a_{\vec{p}\sigma_p}.
\end{equation}

Finally, we can define the energy of each single particle state in terms of its quantum numbers:

\begin{equation} \label{sp_energy_n}
	\epsilon_{n_x,n_y,n_z} = \frac{\hbar^2}{2m}(\frac{2\pi}{L})^2(n_x^2 + n_y^2 + n_z^2).
\end{equation}

We can rewrite Eq. \ref{sp_energy_n} to be in terms of the momentum using Eq. \ref{sp_momentum}, resulting in:

\begin{equation} \label{sp_energy_k}
	\epsilon_{n_x, n_y, n_z} = \frac{\hbar^2}{2m}(k_x^2 + k_y^2 + k_z^2).
\end{equation}

%% ADD A SECTION HERE ABOUT THE FERMI LEVEL AND THE FERMI MOMENTUM

Some of the single-particle states in a system will contain particles, which we will call occupied single-particle states. On the other hand, some will not contain particles, referred to as unoccupied single-particle states. When the system is arranged in its ground state formation (i.e., the lowest energy configuration), the occupied single-particle states fill the states with the lowest single-particle energies. Thus, the occupied single particle states will lie inside a so-called Fermi sphere, described by the Fermi momentum, k$_F$, and the Fermi energy, E$_F$. The Fermi level divides the occupied and unoccupied single-particle states in this ground state configuration.

The following set of equations can relate to the Fermi momentum and the Fermi energy:


\begin{equation} \label{fermi_energy}
	E_f = \frac{\hbar^2k_f^2}{2m} \longrightarrow k_f = \sqrt{\frac{2mE_f}{\hbar^2}}.
\end{equation}

 From the Fermi momentum, we can define the average density of the system, $\rho_0$, to be:

\begin{equation} \label{density_fermi_momentum}
	\rho_0 = \frac{k_f^3}{3\pi^2}.
\end{equation}

The average density, $\rho_0$ can also be defined as $\rho_0 = N/\Omega$, where N is the number of particles in the system and like L, the limit $N \rightarrow \infty$ is taken. However, the limits $N, L \rightarrow \infty$ are taken such that $\rho_0$ remains at a fixed, finite value. Finally, N and k$_F$ can be related with:

\begin{equation}
	N = \frac{k_f^3\Omega}{3\pi^2} \longrightarrow k_F = (\frac{3\pi^2N}{\Omega})^{1/3}.
\end{equation}

While in this theoretical model, there are an infinite number of single-particle states in the basis, in practice, due to computational limitations, the total number of single-particle states included in the calculation, M, must be truncated to a finite number. In this work, we will assume that the number of occupied and unoccupied single particle states correspond to closed-shell structure, and the total number of shells will be finite. We will place a spherical energy cut-off on the quantum numbers such that n$_x^2$ + n$_y^2$ + n$_z^2$ $\leq$ N$_{shells}$-1, where N$_{shells}$ is the total number of energy shells included in the calculation. It is important to note that N$_{shells}$ $\neq$ N$_{max}$.  Rather N$_{max}$ corresponds to the maximum value of n$_{x}$, n$_{y}$, or n$_{z}$ that is included in the calculation. This closed shell structure is also used for coupled cluster calculations in Ref. \cite{Ref3}, \cite{Ref5}, and \cite{Ref8} among others.

Enforcing a closed-shell structure for the occupied and unoccupied single-particle states and truncating the total number of shells allowed in the system restricts the number of allowed particles and total single-particle states to a finite set of values. We refer to these allowed numbers as "magic numbers," They correspond to the total number of single-particle states when s shells are included in a calculation. The first magic number is two, as there are two single-particle states in the first energy shell. The second magic number is 14, as there are 14 single-particle states when two energy shells are included (2 in the first shell and 12 in the second shell). Continuing as such gives the following magic numbers: 2, 14, 38, 54, 66, 114, ... .  Note that these are only the magic numbers for the homogeneous electron gas and pure neutron matter. For symmetric nuclear matter, all magic numbers are doubled (4, 28, 76, 108, 132, 228, ...). In the context of this thesis, the term shells will refer to the total number of energy shells used in the calculation, $M$ will refer to the total number of single-particle states in the system (both occupied and unoccupied), and open shells will refer to the number of shells above the Fermi level only.