The homogeneous electron gas (HEG) is an infinite matter system containing only electrons with a uniform positive background charge such that the overall net charge of the system is zero \cite{Ref4}. Though the HEG as a theoretical model exists in one, two, and three dimensions \cite{Ref4}, this thesis will only focus on the three-dimensional electron gas. The HEG in three dimensions is essential in density functional theory, where it is the cornerstone of the local density approximation \cite{Ref4}. It is also a reasonable model for several systems of interest in quantum chemistry and condensed matter physics, including the electrons in semiconductors and alkali metals \cite{Ref4}. Additionally, studies of the HEG can be used to build a framework that can be transferred to studies of infinite nuclear matter, which is essential for studying the equation of the state of nuclear matter and many-body studies of neutron stars. The HEG is used as a framework for developing analysis tools for other many-body systems because the HEG is a relatively simple infinite matter system, leading to many published properties (both analytical and numerical) for results to be compared to. Other studies have used the HEG as a system that allows for comparisons between the results of different many-body methods (see, for example, Ref. \cite{Ref4}).

The Hamiltonian for the HEG consisting of $N$ electrons can be defined as

\begin{equation}
    \hat{H} = \hat{K} + \hat{V}_{ee} + \hat{V}_{be} + \hat{V}_{bb},
\end{equation}

where $\hat{K}$ is the kinetic energy operator, $\hat{V}_{ee}$ is the interaction between all sets of two electrons, $\hat{V}_{be}$ describes the interaction of all electrons with the positive background charge, and $\hat{V}_{bb}$ describes the contribution of the background charge interacting with itself. Since the HEG comprises a positively charged background and negatively charged electrons, the Coulomb force is the primary force at play in the interactions. The Coulomb force is a long-range force that acts over an infinite space. However, instead of using the Coulomb interaction, we will instead use Ewald's interaction, which splits the electron interaction into a short-range term and a long-range term while simultaneously dealing with $\hat{V}_{be}$ and $\hat{V}_{bb}$. We can rewrite the HEG Hamiltonian using the Ewald interaction as:

\begin{equation}
    \hat{H} = -\frac{1}{2}\sum_\alpha \nabla^2_\alpha + \frac{1}{2}\sum_{\alpha \neq \beta}\hat{v}_{\alpha\beta} + \frac{1}{2}Nv_M,
\end{equation}

where $\alpha$ and $\beta$ are electron indices \cite{Ref1}.  The first term is the kinetic energy operator, and the final two terms comprise the Ewald interaction \cite{Ref1}. The term $v_M$ is called the Madelung term and $\hat{v}_{\alpha\beta}$, the two-electron operator, is defined as:

\begin{equation}\label{two_electron_operator}
    \hat{v}_{\alpha\beta} = \frac{1}{V}\sum_{\vec{q}} v_{\vec{q}}\ e^{i\vec{q}(\vec{r}_\alpha - \vec{r}_\beta)},
\end{equation}

where $V = L^3$ is the finite volume of the HEG and $v_{\vec{q}} = \frac{4\pi}{{\vec{q}}^2}$ if $\vec{q} \neq 0$ and $v_{\vec{q}} = 0$ if $\vec{q} = 0$ \cite{Ref1}. This choice of Hamiltonian leads to the plane wave basis for the single-particle states described in Section 2 of this chapter. While the Ewald interaction makes the HEG easier to work with by splitting up the long-range and short-range components of the electron-electron interaction, it cannot correctly describe the exchange-correlation energy \cite{Ref4}.

Note that the density of the homogeneous electron gas is typically described by a parameter called the Wigner-Seitz radius. The Wigner-Seitz radius can be defined as

\begin{equation}
    r_s = \frac{r_0}{r_B},
\end{equation}

where r$_B$ is the Bohr radius, and r$_0$ is related to the size of the box containing the electron gas by

\begin{equation}
    \frac{4}{3}\pi r_0^3 = \frac{N}{L^3},
\end{equation}

where the density of the HEG is $d = N/L^3$, measured in fm$^{-3}$ \cite{Ref5}. A larger value of r$_s$ means the modeled system is in a larger box and therefore is less dense than a smaller value of r$_s$ for the same number of particles. It is important to note that the HEG behaves differently at different densities. At low densities (or high r$_s$), the electrons in the HEG are for a lattice. This process is called Wigner crystallization and results from the long-range repulsive interactions between the electrons \cite{Ref4}. At higher densities (or lower values of r$_s$), the HEG can better be described as a liquid instead of as a gas \cite{Ref4}.


The convergence of the correlation energy for the electron gas using plane waves as a basis has yet to be thoroughly investigated, but some work has been done in Ref. \cite{Ref1}. One reason why the correlation energy of the electron gas is an interesting problem is that it has substantial contributions from the electron-electron cusp. However, this electron-electron cusp does lead to trouble when making computational truncations as a finite set of single-particle states cannot accurately describe the behavior in this region \cite{Ref1}.

It is important to note that the MBPT2 correlation energies are well-defined for a finite electron gas. However, they diverge in the thermodynamic limit of an infinite electron gas in three dimensions \cite{Ref1}. This happens at high densities due to the dominance of the particle-hole ring diagrams \cite{Ref4}.

The single particle energies of the HEG when using the Ewald interaction are as follows for occupied single particle states:

\begin{equation}
    \epsilon_i = \frac{\vec{k}_i^2}{2} - \sum_{j\neq i} \langle ij| \hat{v}_2 | ji \rangle - \frac{v_M}{2},
\end{equation}

and for unoccupied single particle states:

\begin{equation}
    \epsilon_a = \frac{\vec{k}_a^2}{2} - \sum_j \langle aj | \hat{v}_2 | ja \rangle.
\end{equation}

Here we have used the common denotation of indices where $a$, $b$, $c$,... represent states which are unoccupied in the Fermi ground state, $i$, $j$, $k$,... represent states which are occupied in the Fermi ground state, and $p$, $q$, $r$,... could represent either \cite{Ref1}. 

We can also define the two-electron matrix elements in integral notation as:

\begin{equation}
    \langle ij | \hat{v}_2 | ab \rangle = \delta_{ia}\delta_{jb} \frac{1}{\Omega^2}\int \int d\vec{r}_1 d\vec{r}_2 \psi^*_i(\vec{r}_1)\psi^*_j(\vec{r}_2)\hat{v}_2(\vec{r}_1, \vec{r}_2)\psi_a(\vec{r}_1)\psi_b(\vec{r}_2),
\end{equation}

where $\hat{v}_2$ is defined in Eq. \ref{two_electron_operator} and $\psi$ are plane waves defined in Sec. 4.2.

By convention, the energies calculated using the HEG will be reported in Hartrees (1 Hartree = $4.35x10^{-18}$ Joules). However, this differs from the unit that will be used for the next system, infinite nuclear matter.
