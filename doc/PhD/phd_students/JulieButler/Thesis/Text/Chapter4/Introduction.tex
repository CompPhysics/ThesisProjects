% What is the infinite matter?
An infinite matter system is simply a system which contains infinite particles and occupies an infinite volume. We will be looking at two infinite matter systems in this chapter. The first, the homogeneous electron gas, which has critical applications in several areas of chemistry and condensed matter physics \cite{Ref1, Ref2, Ref66, Ref67, Ref68, Ref72,Ref88,Ref90,Ref91,Ref92,Ref95}. Since the homogeneous electron gas is a more straightforward infinite matter system, it is also a test bed for developing methods that can be applied to more complex infinite matter systems. The second infinite matter system we will investigate in this chapter is infinite nuclear matter \cite{Ref103, Ref118,Ref120,Ref8,Ref68}. Specifically, pure neutron matter is an infinite nuclear matter system where all the particles are neutrons (\cite{Ref131,Ref39,Ref45,Ref54,Ref55,Ref56,Ref57,Ref62,Ref82}), and symmetric nuclear matter, where half of the particles are protons and half are neutrons (\cite{Ref50,Ref85}). Studies of infinite nuclear matter play an essential role in nuclear physics and astrophysics \cite{Ref108, Ref3, Ref20,Ref28,Ref33,Ref34,Ref35,Ref36,Ref37,Ref38,Ref39,Ref41,Ref58}.

% Coupled cluster considerations for infinite matter
Performing coupled cluster calculations on infinite matter is not a new concept. For coupled cluster calculations of the electron gas, see, for example, Refs. \cite{Ref4,Ref5,Ref2,Ref72}, among many other interesting studies. For infinite matter, there are Refs. \cite{Ref4,Ref5,Ref8,Ref9,Ref20} among others.  However, this work is unique in the methodology we will take to remove some of the errors that result from these calculations, described towards the end of this chapter. It is important to note that for all infinite matter systems investigated in this thesis, the 1-particle 1-hole excitation operator ($\hat{T}_1$) in the cluster operator is zero due to symmetry considerations \cite{Ref8}. It is almost important to note that these infinite matter systems are momentum conserving and that the total momentum is zero \cite{Ref8}.