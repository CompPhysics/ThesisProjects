As pointed out in the pure neutron matter section of this chapter, the main goal of this chapter is not to compare coupled cluster calculations with different levels of approximation, different nuclear interactions, and different infinite nuclear matter systems. Though we have made some of these comparisons, the main point of this chapter is to analyze the accuracy and time savings of the SRE algorithm. As we have seen throughout this chapter, coupled cluster calculations of nuclear systems have very large (and possibly prohibitive) computational times, especially when using realistic nuclear interactions and higher-order coupled cluster approximations. The SRE method, with its ability to make accurate extrapolations with a small amount of training data, can make these studies more feasible by drastically reducing the run times needed to perform the calculations. This is especially important when we realize that studies of the nuclear equation of state (an essential future work for this thesis) will require accurate calculations at different densities and proton fractions. Thus, the development of the SRE method, by reducing the computational time needed to perform accurate calculations, makes large-scale studies of infinite nuclear matter much more feasible, paving the way for novel studies to occur in the future.