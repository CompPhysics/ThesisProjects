Removing the finite size error from truncating the number of particles in the system will be similar to removing the basis incompleteness error.  We will start with a data set that is created by dividing the converged CCD correlation energies by the number of particles in the system for small values of N, resulting in:

\begin{equation}
    y = \frac{\Delta E_{CCD}^{N_k}}{N_k} = \frac{\Delta E_{CCD}^{N_1}}{N_1}, \frac{\Delta E_{CCD}^{N_2}}{N_2}, ... .
\end{equation}

Then we will train a machine learning algorithm using the sequential formatting developed in this last section on the data set we have created.

\begin{equation}
    f_R(\frac{\Delta E_{CCD}^{N_{k-3}}}{N_{k-3}}, \frac{\Delta E_{CCD}^{N_{k-2}}}{N_{k-2}}, \frac{\Delta E_{CCD}^{N_{k-1}}}{N_{k-1}}) = \frac{\Delta E_{CCD}^{N_{k}}}{N_{k}}
\end{equation}

The final step is to use the trained machine learning algorithm to extrapolate this ratio until convergence, resulting in the CCD correlation energy in the thermodynamic limit.  Thus, we have:

\begin{equation}
    \lim_{k\to\infty} = \lim_{k\to\infty} = \Delta E_{CCD}^\infty,
\end{equation}

where $\Delta E_{CCD}^\infty$ is the CCD correlation energy per particle in the thermodynamic limit.  Additionally, since the prediction the machine learning algorithm makes is the result we are looking for, the uncertainty of that prediction does not need to be modified.