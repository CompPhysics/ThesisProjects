%%
%% Automatically generated file from DocOnce source
%% (https://github.com/hplgit/doconce/)
%%
%%


%-------------------- begin preamble ----------------------

\documentclass[%
oneside,                 % oneside: electronic viewing, twoside: printing
final,                   % draft: marks overfull hboxes, figures with paths
10pt]{article}

\listfiles               %  print all files needed to compile this document

\usepackage{relsize,makeidx,color,setspace,amsmath,amsfonts,amssymb}
\usepackage[table]{xcolor}
\usepackage{bm,ltablex,microtype}

\usepackage[pdftex]{graphicx}

\usepackage[T1]{fontenc}
%\usepackage[latin1]{inputenc}
\usepackage{ucs}
\usepackage[utf8x]{inputenc}

\usepackage{lmodern}         % Latin Modern fonts derived from Computer Modern

% Hyperlinks in PDF:
\definecolor{linkcolor}{rgb}{0,0,0.4}
\usepackage{hyperref}
\hypersetup{
    breaklinks=true,
    colorlinks=true,
    linkcolor=linkcolor,
    urlcolor=linkcolor,
    citecolor=black,
    filecolor=black,
    %filecolor=blue,
    pdfmenubar=true,
    pdftoolbar=true,
    bookmarksdepth=3   % Uncomment (and tweak) for PDF bookmarks with more levels than the TOC
    }
%\hyperbaseurl{}   % hyperlinks are relative to this root

\setcounter{tocdepth}{2}  % levels in table of contents

\usepackage[framemethod=TikZ]{mdframed}

% --- begin definitions of admonition environments ---

% --- end of definitions of admonition environments ---

% prevent orhpans and widows
\clubpenalty = 10000
\widowpenalty = 10000

% --- end of standard preamble for documents ---


% insert custom LaTeX commands...

\raggedbottom
\makeindex
\usepackage[totoc]{idxlayout}   % for index in the toc
\usepackage[nottoc]{tocbibind}  % for references/bibliography in the toc

%-------------------- end preamble ----------------------

\begin{document}

% matching end for #ifdef PREAMBLE

\newcommand{\exercisesection}[1]{\subsection*{#1}}


% ------------------- main content ----------------------



% ----------------- title -------------------------

\thispagestyle{empty}

\begin{center}
{\LARGE\bf
\begin{spacing}{1.25}
Developing a modern shell-model code
\end{spacing}
}
\end{center}

% ----------------- author(s) -------------------------

\begin{center}
{\bf Thesis topics for John Bower}
\end{center}

    \begin{center}
% List of all institutions:
\centerline{{\small Department of Physics and Astronomy and National Superconducting Cyclotron Laboratory, Michigan State University}}
\end{center}
    
% ----------------- end author(s) -------------------------

% --- begin date ---
\begin{center}
2016
\end{center}
% --- end date ---

\vspace{1cm}


% !split 
\subsection*{General motivation}

% --- begin paragraph admon ---
\paragraph{}
The nuclear shell model 
plays a central role as a theoretical tool in interpreting nuclear
structure experiments. The aim of this thesis is to build on existing
shell-model codes like the code developed by us and available, with
benchmarks etc at the website of the \href{{https://github.com/ManyBodyPhysics/CENS}}{Computational Environment for
Nuclear Structure} (CENS)
and develop a modern (written in C++ eventually Fortran2008) full
configuration interaction environment for nuclear structure
studies. These codes should be able to 
\begin{itemize}
\item Run on the next generation of supercomputers 

\item Be able to handle both two and three-body interactions 

\item Be able to handle one- and two-body operators 

\item Can be extended to \href{{https://www.duo.uio.no/bitstream/handle/10852/37172/master.pdf?sequence=1}}{FCIQMC} studies of finite nuclei and nuclear matter

\item Be fully open source and accessible
\end{itemize}

\noindent
% --- end paragraph admon ---



% !split
\subsection*{\href{{https://github.com/ManyBodyPhysics/CENS}}{The CENS project}}

% --- begin paragraph admon ---
\paragraph{}
The CENS  site contains 
\begin{itemize}
\item Shell-model code written in C for \href{{https://github.com/ManyBodyPhysics/CENS/tree/master/FCI/serial/IdenticalParticles}}{identical particles} and the \href{{https://github.com/ManyBodyPhysics/CENS/tree/master/FCI/serial/pnCase}}{proton-neutron case}

\item \href{{https://github.com/ManyBodyPhysics/CENS/tree/master/FCI}}{Both parallel and serial versions of the above}

\item Codes for one-body transition probabilities such $M\Lambda$ and $E\Lambda$ transitions but no GT transitions. There is also no code for two-body transition operators and two-body densities

\item Codes for three-body forces with identical particles only
\end{itemize}

\noindent
These codes can serve as starting point together with the already written FCI code by John.
% --- end paragraph admon ---



% !split
\subsection*{Efficient algorithms for bit manipulations}

% --- begin paragraph admon ---
\paragraph{}
In addition to the existing material, there are some useful articles and references for the first steps. 
The first step is to study efficient representations of the Slater determinants for words with more than 64 bits, allowing thereby for shell-model studies 
of systems with more than one major shell. Furthermore, the setup of the Hamiltonian matrix elements plays an important role. 

In order to study these aspects and write an efficient program, the following articles can be of interest
\begin{itemize}
\item \href{{http://arxiv.org/abs/1311.6244}}{Scemana's article on efficient implementation of the Slater-Condon rules}

\item \href{{http://arxiv.org/abs/0810.2644}}{Simen Kvaal's article on developing an FCI code and open source code for quantum dots}
\end{itemize}

\noindent
% --- end paragraph admon ---





% !split
\subsection*{Implementing Lanczos' and Davidson's algorithms}

% --- begin paragraph admon ---
\paragraph{}
\begin{itemize}
\item add references to this

\item baby steps
\end{itemize}

\noindent
% --- end paragraph admon ---



% !split
\subsection*{Addition about FCIQMC}

% --- begin paragraph admon ---
\paragraph{}
\begin{itemize}
\item Add notes here
\end{itemize}

\noindent
% --- end paragraph admon ---






% ------------------- end of main content ---------------

\end{document}

