
% LaTeX Beamer file automatically generated from DocOnce
% https://github.com/hplgit/doconce

%-------------------- begin beamer-specific preamble ----------------------

\documentclass{beamer}

\usetheme{red_plain}
\usecolortheme{default}

% turn off the almost invisible, yet disturbing, navigation symbols:
\setbeamertemplate{navigation symbols}{}

% Examples on customization:
%\usecolortheme[named=RawSienna]{structure}
%\usetheme[height=7mm]{Rochester}
%\setbeamerfont{frametitle}{family=\rmfamily,shape=\itshape}
%\setbeamertemplate{items}[ball]
%\setbeamertemplate{blocks}[rounded][shadow=true]
%\useoutertheme{infolines}
%
%\usefonttheme{}
%\useinntertheme{}
%
%\setbeameroption{show notes}
%\setbeameroption{show notes on second screen=right}

% fine for B/W printing:
%\usecolortheme{seahorse}

\usepackage{pgf}
\usepackage{graphicx}
\usepackage{epsfig}
\usepackage{relsize}

\usepackage{fancybox}  % make sure fancybox is loaded before fancyvrb

\usepackage{fancyvrb}
%\usepackage{minted} % requires pygments and latex -shell-escape filename
%\usepackage{anslistings}
%\usepackage{listingsutf8}

\usepackage{amsmath,amssymb,bm}
%\usepackage[latin1]{inputenc}
\usepackage[T1]{fontenc}
\usepackage[utf8]{inputenc}
\usepackage{colortbl}
\usepackage[english]{babel}
\usepackage{tikz}
\usepackage{framed}
% Use some nice templates
\beamertemplatetransparentcovereddynamic

% --- begin table of contents based on sections ---
% Delete this, if you do not want the table of contents to pop up at
% the beginning of each section:
% (Only section headings can enter the table of contents in Beamer
% slides generated from DocOnce source, while subsections are used
% for the title in ordinary slides.)
\AtBeginSection[]
{
  \begin{frame}<beamer>[plain]
  \frametitle{}
  %\frametitle{Outline}
  \tableofcontents[currentsection]
  \end{frame}
}
% --- end table of contents based on sections ---

% If you wish to uncover everything in a step-wise fashion, uncomment
% the following command:

%\beamerdefaultoverlayspecification{<+->}

\newcommand{\shortinlinecomment}[3]{\note{\textbf{#1}: #2}}
\newcommand{\longinlinecomment}[3]{\shortinlinecomment{#1}{#2}{#3}}

\definecolor{linkcolor}{rgb}{0,0,0.4}
\hypersetup{
    colorlinks=true,
    linkcolor=linkcolor,
    urlcolor=linkcolor,
    pdfmenubar=true,
    pdftoolbar=true,
    bookmarksdepth=3
    }
\setlength{\parskip}{0pt}  % {1em}

\newenvironment{doconceexercise}{}{}
\newcounter{doconceexercisecounter}
\newenvironment{doconce:movie}{}{}
\newcounter{doconce:movie:counter}

\newcommand{\subex}[1]{\noindent\textbf{#1}}  % for subexercises: a), b), etc

%-------------------- end beamer-specific preamble ----------------------

% Add user's preamble




% insert custom LaTeX commands...

\raggedbottom
\makeindex

%-------------------- end preamble ----------------------

\begin{document}

% matching end for #ifdef PREAMBLE

\newcommand{\exercisesection}[1]{\subsection*{#1}}



% ------------------- main content ----------------------



% ----------------- title -------------------------

\title{Developing a modern shell-model code}

% ----------------- author(s) -------------------------

\author{Thesis topics for John Bower\inst{1}}
\institute{Department of Physics and Astronomy and National Superconducting Cyclotron Laboratory, Michigan State University\inst{1}}
% ----------------- end author(s) -------------------------

\date{2016
% <optional titlepage figure>
% <optional copyright>
}

\begin{frame}[plain,fragile]
\titlepage
\end{frame}

\begin{frame}[plain,fragile]
\frametitle{General motivation}

\begin{block}{}
The nuclear shell model 
plays a central role as a theoretical tool in interpreting nuclear
structure experiments. The aim of this thesis is to build on existing
shell-model codes like the code developed by us and available, with
benchmarks etc at the website of the \href{{https://github.com/ManyBodyPhysics/CENS}}{Computational Environment for
Nuclear Structure} (CENS)
and develop a modern (written in C++ eventually Fortran2008) full
configuration interaction environment for nuclear structure
studies. These codes should be able to 
\begin{itemize}
\item Run on the next generation of supercomputers 

\item Be able to handle both two and three-body interactions 

\item Be able to handle one- and two-body operators 

\item Can be extended to \href{{https://www.duo.uio.no/bitstream/handle/10852/37172/master.pdf?sequence=1}}{FCIQMC} studies of finite nuclei and nuclear matter

\item Be fully open source and accessible
\end{itemize}

\noindent
\end{block}
\end{frame}

\begin{frame}[plain,fragile]
\frametitle{\href{{https://github.com/ManyBodyPhysics/CENS}}{The CENS project}}

\begin{block}{}
The CENS  site contains 
\begin{itemize}
\item Shell-model code written in C for \href{{https://github.com/ManyBodyPhysics/CENS/tree/master/FCI/serial/IdenticalParticles}}{identical particles} and the \href{{https://github.com/ManyBodyPhysics/CENS/tree/master/FCI/serial/pnCase}}{proton-neutron case}

\item \href{{https://github.com/ManyBodyPhysics/CENS/tree/master/FCI}}{Both parallel and serial versions of the above}

\item Codes for one-body transition probabilities such $M\Lambda$ and $E\Lambda$ transitions but no GT transitions. There is also no code for two-body transition operators and two-body densities

\item Codes for three-body forces with identical particles only
\end{itemize}

\noindent
These codes can serve as starting point together with the already written FCI code by John. 
\end{block}
\end{frame}

\begin{frame}[plain,fragile]
\frametitle{Efficient algorithms for bit manipulations}

\begin{block}{}
In addition to the existing material, there are some useful articles and references for the first steps. 
The first step is to study efficient representations of the Slater determinants for words with more than 64 bits, allowing thereby for shell-model studies 
of systems with more than one major shell. Furthermore, the setup of the Hamiltonian matrix elements plays an important role. 

In order to study these aspects and write an efficient program, the following articles can be of interest
\begin{itemize}
\item \href{{http://arxiv.org/abs/1311.6244}}{Scemana's article on efficient implementation of the Slater-Condon rules}

\item \href{{http://arxiv.org/abs/0810.2644}}{Simen Kvaal's article on developing an FCI code and open source code for quantum dots}
\end{itemize}

\noindent
\end{block}
\end{frame}

\begin{frame}[plain,fragile]
\frametitle{Implementing Lanczos' and Davidson's algorithms}

\begin{block}{}
\begin{itemize}
\item add references to this

\item baby steps
\end{itemize}

\noindent
\end{block}
\end{frame}

\begin{frame}[plain,fragile]
\frametitle{Addition about FCIQMC}

\begin{block}{}
\begin{itemize}
\item Add notes here
\end{itemize}

\noindent
\end{block}
\end{frame}

\end{document}
