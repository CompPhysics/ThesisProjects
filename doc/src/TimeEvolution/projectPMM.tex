\documentclass{article}
\usepackage{graphicx} % Required for inserting images
\usepackage{hyperref}

\title{Machine learning and non-linear dynamics}
\author{Master of Science thesis project}
\date{November 2024}

\begin{document}

\maketitle  

\section{Scientific aims}


The aim of this master of science project is to study the solution of
time-dependent differential equations such as the time-dependent
Schr\"odinger equation in quantum mechanics using a newly develop
machine learning method, the so-called Parametric Matrix Model (PMM)
approach \cite{us2024}. This method has been shown to be surprisingly
stable and subccesful and performs better (less paramters) in many
cases than most of the standard deep learning methods for both
regression and classification problems.  The PMM has been shown to
have a great potential in solving in particular non-linear dynamics
and time-dependent problems.


One of the first steps in solving any physics problem is identifying
the governing equations.  While the solutions to those equations may
exhibit highly complex phenomena, the equations themselves have a
simple and logical structure.  The structure of the underlying
equations leads to important and nontrivial constraints on the
analytic properties of the solutions.  Some well-known examples
include symmetries, conserved quantities, causality, and analyticity.
Unfortunately, these constraints are usually not reproduced by machine
learning algorithms, leading to inefficiencies and limited accuracy
for scientific computing applications. Current physics-inspired and
physics-informed machine learning approaches aim to constrain the
solutions by penalizing violations of the underlying equations, but
this does not guarantee exact adherence.  Furthermore, the results of
many modern deep learning methods suffer from a lack of
interpretability.  To address these issues, we introduced recently a
new class of machine learning algorithms called parametric matrix
models \cite{us2024}.  Parametric matrix models (PMMs) use matrix
equations similar to the physics of quantum systems and learn the
governing equations that lead to the desired outputs.  Since quantum
mechanics provides a fundamental microscopic description of all
physical phenomena, the existence of such matrix equations is clear.
Parametric matrix models take the additional step of applying the
principles of model order reduction and reduced basis methods to find
efficient approximate matrix equations with finite dimensions.


The plan for this thesis project is as follows:
\begin{itemize}
\item Spring 2025: follow courses and get familiar with the PMM method
and reproduce several of the test examples discussed in \cite{us2024}. 
The notebook at \cite{dannynotebook} can serve as a useful start for getting to know the method.
\item Fall 2025: The main application of the method is to time-dependent quantum mechanical problems. The thesis starts with time-dependent Hartree-Fock theory as discussed in the work of Zanghellini {\em et al.}, see \cite{zanghellini} applied to a system of electrons confined in one-dimensional traps. The aim is to apply the PMM to the study of such dynamical systems. Codes developed by us for the time-dependent Hartree-Fock method will be provided.
\item Spring 2026: With a working code for the time-dependent Hartree-Fock method, the next step is to extend this to more advanced many-body methods such as the time-dependent multiconfiguration Hartree-Fock method. The first step is to reproduce the results of Zanghellini {\em et al.}, see \cite{zanghellini} before applying the formalism and codes to two-dimensional systems of quantum dots. These systems are highly relevant candidates for making quantum components such as various gates. The latter can be used in studies of the time-evolution of entanglement. We expect that the thesis will finalized by May 2026. 
\end{itemize}

\begin{thebibliography}{99}

\bibitem{us2024} Patrick Cook, Danny Jammooa, Morten Hjorth-Jensen, Daniel D. Lee, Dean Lee, Parametric Matrix Models, \url{https://arxiv.org/abs/2401.11694}
\bibitem{dannynotebook} Danny Jammoa, notebook at  \url{https://github.com/dannyjammooa/Parametric-Matrix-Models/tree/main}  
\bibitem{zanghellini} Jürgen Zanghellini et al, J. Phys. B: At. Mol. Opt. Phys. {\bf 37},  763 (2004) and \url{https://iopscience.iop.org/article/10.1088/0953-4075/37/4/004/pdf}
\end{thebibliography}  



\end{document}




