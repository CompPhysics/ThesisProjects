\documentclass{article}
\usepackage{graphicx} % Required for inserting images
\usepackage{hyperref}

\title{Machine learning and artificial intelligence for quantum-mechanical systems}
\author{Master of Science thesis project}
\date{November 2024}

\begin{document}

\maketitle  

\section{Scientific aims}


The aim of this master of science project is to develop many-body
theories for studies of strongly interacting quantum mechanical
many-particle systems using novel methods from deep learning theories,
in particular advanced neural networks and other generative models.

For interacting many-particle systems where the degrees of freedom
increase exponentially, quantum mechanical many-body methods like
quantum Monte Carlo methods, Coupled Cluster theory, Green's function
theories, density functional theories and other, play a central role
in understanding experiments in a wide range of fields, spanning from
atomic and molecular physics and thereby quantum chemistry to
condensed matter physics, materials science, nano-technologies and
quantum technologies and finally processes like fusion and fission in
nuclear physics. This list is definitely not exhaustive as the are
many other areas of applications for quantum mechanical many-body
theories.


Quantum Monte Carlo techniques are widely applicable and have been
used in studies of a large range of systems.  The main difficulty with
QMC calculations of fermionic systems is ensuring the fermionic
antisymmetry is respected. In Diffusion Monte Carlo calculations,
which in principle yield the exact solutions in the limit of lon
simulation times, the prescription to ensure this is fixing the nodes
of the system to prevent the large errors that would come with the
summation of alternating signs. Unfortunately this prescription is not
variational in nature and can in some cases result in convergence to
energies lower than the true ground state of the system. This is one
of the advantages of VMC calculations since they are known to be
variational the resulting wavefunction will always have an energy
larger than or equal to the true ground state wavefunction.

The problem with variational Monte Carlo calculations has been the
choice of trial wave function.  Recently, several research groups have
introduced, with great success, neural networks as a way to represent
the trial wave function. Recent works on infinite nuclear
matter\cite{us2023a,us2024}, the unitary Fermi gas \cite{us2023b}, and
Daniel Haas Beccatini's recent master of science thesis project
\cite{daniel2024} have shown that one can obtain results of equal
accuratness as the theoretical benchmark calculations provided by
diffusion Monte Carlo results.  This has opened up a new area of
research and the present thesis project aims at developing further
deep learning approaches to the studies of strongly interacting
many-body systems.

The plans here are to extend the studies in \cite{daniel2024} to
studies of low-dimensional systems such as quantum dots and the
infinite electron gas in two dimensions. These are systems of great
interest for materials science studies, nano-technologies and quantum
technologies. A proper understanding of the properties of such systems
will play a crucial role in designing for example quantum gates and
circuits.  These systems are studied experimentally at the university
of Oslo at the Center for Materials Science and Nanotechnologies
(SMN). The theoretical activity at the Center for Computing in Science
Education and the Computational Physics research group have through
the last years developed a strong collaboration with several
researchers at the SMN.

The thesis of Daniel Haas Beccatini \cite{daniel2024} with codes and
additional material will serve as an excellent backgroun material.

The plan for this thesis project is as follows:
\begin{itemize}
\item Spring 2025: follow courses and get familair with Monte Carlo methods
and how to use neural networks to solve simpler quantum
mechanical. The software developed in \cite{daniel2024} can serve as a
guidance for developing own code.
\item Fall 2025: Include stochastic resampling \cite{daniel2024} and develop code for studies of both bosonic and fermionic systems.
\item Spring 2026: Apply formalism and code to studies of two-dimensional systems like quantum dots and/or the infinite electron gas in two dimensions. Finalize thesis by May 2026.
\end{itemize}

\begin{thebibliography}{99}

\bibitem{us2023a} Bryce Fore, Jane M. Kim, Giuseppe Carleo, Morten Hjorth-Jensen, Alessandro Lovato, and Maria Piarulli, Dilute neutron star matter from neural-network quantum states, Physical Review Research {\bf 5}, 033062 (2023) and \url{https://journals.aps.org/prresearch/abstract/10.1103/PhysRevResearch.5.033062}
\bibitem{us2023b} Jane Kim, Gabriel Pescia, Bryce Fore, Jannes Nys, Giuseppe Carleo, Stefano Gandolfi, Morten Hjorth-Jensen, Alessandro Lovato, Neural-network quantum states for ultra-cold Fermi gases, Nature Communications Physics {\bf 7}, 148 (2024) and \url{https://www.nature.com/articles/s42005-024-01613-w}
\bibitem{us2024} Bryce Fore, Jane Kim, Morten Hjorth-Jensen, Alessandro Lovato, Investigating the crust of neutron stars with neural-network quantum states, Nature Communications Physics in press  and \url{https://arxiv.org/abs/2407.21207}
\bibitem{daniel2024} Daniel Haas Beccatini, Master of Science thesis, University of Oslo, 2024, Deep Learning Methods for Quantum Many-body Systems, A study on Neural Quantum States, \url{https://www.duo.uio.no/handle/10852/113984}
\end{thebibliography}  



\end{document}





