\documentclass[12pt,a4paper]{article}
\usepackage{amsmath,amsfonts,amssymb}
\usepackage{geometry}
\usepackage{hyperref}
\usepackage{enumitem}
\geometry{margin=2.5cm}

\begin{document}

\begin{center}
    {\Large \textbf{Master Thesis Project in Quantum Information Science and Technology}}\\[0.3cm]
    {\large \textbf{Parametric Matrix Models for Trotterized Unitary Coupled Cluster Dynamics of the Pairing Hamiltonian}}\\[0.2cm]
    \vspace{0.3cm}
    {Duration: One Year (60 ECTS)}
\end{center}

\vspace{1cm}

\section*{1. Introduction}

Quantum computing offers new paradigms for solving strongly correlated
quantum many-body problems in nuclear physics, quantum chemistry, and
condensed matter systems. Among the most prominent variational quantum
algorithms is the \emph{unitary coupled cluster} (UCC) approach, where
a parametrized unitary ansatz
\[
U(\boldsymbol{\theta}) = e^{T(\boldsymbol{\theta}) - T^\dagger(\boldsymbol{\theta})}
\]
is used to prepare approximate ground states of interacting
Hamiltonians, such as the pairing Hamiltonian relevant for
superconductivity and nuclear pairing correlations.

On a digital quantum computer, UCC operators must be decomposed into
sequences of elementary quantum gates using Trotter--Suzuki
factorizations. The finite Trotter step size $\Delta t$ (or, more
generally, the finite number of Trotter steps) introduces systematic
errors. Reducing $\Delta t$ improves accuracy but increases the number
of gates, which is problematic for near-term noisy intermediate-scale
quantum (NISQ) devices.

Recently, Cook \textit{et al.} introduced \emph{Parametric Matrix
Models} (PMMs) as a physics-inspired machine learning framework based
on parametrized Hermitian or unitary matrices and their
eigenvalues/eigenvectors. PMMs are universal function approximators
that work directly with matrix equations, and they have been used to
perform \emph{zero-error Trotter step extrapolation} in quantum
simulations by learning the dependence of observables on the Trotter
step size and extrapolating to the continuum limit.

This master thesis aims at combining the PMM framework with
Trotterized UCC simulations of a pairing Hamiltonian (reduced BCS
Hamiltonian) relevant for nuclear structure and superconductivity. The
project will focus on classical simulations of the quantum circuits
(e.g.\ in Qiskit or similar libraries), generation of Trotterized
data, and the construction and training of PMMs to extrapolate
observables and energies to the $\Delta t \to 0$ limit in an efficient
and interpretable way.

\section*{2. Scientific Background and Motivation}

The pairing Hamiltonian is a minimal yet non-trivial model for
fermionic superfluidity and nuclear pairing:
\[
H = \sum_{p} \varepsilon_p\, N_p - G \sum_{p,q} P_p^\dagger P_q,
\]

where $N_p$ are number operators, $P_p^\dagger$ and $P_p$ are pair
creation and annihilation operators, $\varepsilon_p$ are
single-particle energies, and $G$ is the pairing strength. This
Hamiltonian exhibits strong correlations and can be mapped to a spin
representation suitable for quantum simulation.

The UCC ansatz for the pairing Hamiltonian is constructed from
physically motivated excitation operators that create and annihilate
pairs, and the corresponding unitary transformation is implemented on
a quantum device via Trotter factorizations. For a time-independent
effective Hamiltonian or UCC generator $K$, one typically uses
approximations of the form
\[
e^{-i K \Delta t} \approx \prod_j e^{-i K_j \Delta t},
\]

where $K = \sum_j K_j$ and the $K_j$ are simpler terms whose
exponentials can be directly decomposed into quantum gates. Repeating
this product $n$ times approximates a longer evolution or a more
accurate UCC operator, but also increases circuit depth and noise.

The key challenges are:
\begin{itemize}
    \item Quantifying and mitigating Trotter errors in realistic UCC circuits.
    \item Extrapolating observables (e.g.\ ground-state energies, pair occupation numbers) to the zero Trotter-step limit using data obtained at relatively large $\Delta t$ that are feasible on NISQ devices.
    \item Designing extrapolation schemes that respect the underlying analytic structure of the quantum evolution and that can be efficiently trained from limited data.
\end{itemize}

PMMs provide a natural framework for this problem. By learning an
effective Hermitian matrix model whose eigenvalues and eigenvectors
reproduce the dependence of observables on $\Delta t$ and the UCC
parameters, one can perform robust extrapolations and gain additional
insight into the analytic structure of the Trotterized dynamics.

\section*{3. Objectives}

The overall goal of this thesis is to develop and test a parametric
matrix model for Trotterized UCC simulations of the pairing
Hamiltonian on quantum computers. The work combines quantum many-body
theory, quantum computing, and physics-inspired machine learning.

\subsection*{Objective 1: Theory and Literature Review}

\begin{itemize}
    \item Review the theory of the pairing Hamiltonian (reduced BCS model) and its applications in nuclear physics and superconductivity.
    \item Study the unitary coupled cluster method and its use in quantum algorithms (UCCSD and pairing-specific UCC variants).
    \item Understand Trotter--Suzuki expansions and their role in implementing UCC unitaries on quantum hardware.
    \item Study the theory of Parametric Matrix Models as introduced by Cook \textit{et al.}, including:
          \begin{itemize}
              \item affine Hermitian PMMs and unitary PMMs,
              \item eigenvalue-based and observable-based PMM architectures,
              \item the example of zero-error Trotter step extrapolation presented in the article.
          \end{itemize}
\end{itemize}

\subsection*{Objective 2: Classical Simulation of the Pairing Hamiltonian and UCC Circuits}

\begin{itemize}
    \item Implement the pairing Hamiltonian in a suitable basis (e.g.\ spin representation via Jordan--Wigner or Bravyi--Kitaev transformation).
    \item Construct classical simulations of UCC-type quantum circuits for small systems (few pairs and levels) using a quantum computing framework (e.g.\ Qiskit).
    \item Implement Trotterized approximations to the UCC operator and generate data for energies and selected observables as functions of:
          \begin{itemize}
              \item Trotter step size $\Delta t$,
              \item number of Trotter steps,
              \item pairing strength $G$ and single-particle spacing,
              \item UCC parameters (e.g.\ amplitudes).
          \end{itemize}
\end{itemize}

\subsection*{Objective 3: Design and Implementation of Parametric Matrix Models}

\begin{itemize}
    \item Design PMM architectures tailored to this problem:
          \begin{itemize}
              \item primary matrices as Hermitian functions of input features (e.g.\ $\Delta t$, $G$, variational parameters),
              \item secondary matrices corresponding to effective observables (e.g.\ effective pairing Hamiltonian, pair occupation).
          \end{itemize}
    \item Explore both:
          \begin{itemize}
              \item eigenvalue-based PMMs (learning effective Hamiltonians),
              \item observable-based PMMs (learning expectation values directly).
          \end{itemize}
    \item Implement these PMMs in a numerical environment (Python/NumPy/PyTorch or similar), using the complex-valued gradient methods described in the PMM framework.
\end{itemize}

\subsection*{Objective 4: Zero-Error Trotter Step Extrapolation}

\begin{itemize}
    \item Train PMMs on simulated UCC data obtained at moderate Trotter step sizes $\Delta t$.
    \item Use the trained PMMs to extrapolate observables and energies to $\Delta t \to 0$.
    \item Compare PMM-based extrapolation with:
          \begin{itemize}
              \item polynomial extrapolation,
              \item standard neural network regressors (MLPs),
              \item other baseline methods (e.g.\ spline fits).
          \end{itemize}
    \item Analyze the accuracy and stability of PMM extrapolations for:
          \begin{itemize}
              \item ground-state energy,
              \item occupation numbers and pairing gaps,
              \item possibly time-dependent observables if time evolution is considered.
          \end{itemize}
\end{itemize}

\subsection*{Objective 5: Performance Analysis and Physical Interpretation}

\begin{itemize}
    \item Quantify the efficiency of the PMM in terms of:
          \begin{itemize}
              \item number of trainable parameters,
              \item data requirements,
              \item computational cost (training and inference).
          \end{itemize}
    \item Investigate the analytic structure learned by the PMM (e.g.\ avoided crossings in effective spectra as functions of $\Delta t$ or $G$).
    \item Discuss implications for near-term quantum devices and possible extensions to more complex Hamiltonians or ansätze.
\end{itemize}

\section*{4. Work Plan and Milestones (One-Year Timeline)}

\subsection*{Semester 1 (Months 1--6)}

\textbf{Months 1--2: Literature Review and Theoretical Foundation}
\begin{itemize}
    \item Study the pairing Hamiltonian, UCC methods, and Trotter--Suzuki decompositions.
    \item Review PMM theory and the specific zero-error Trotter extrapolation example.
\end{itemize}
\textbf{Deliverable:} Written summary of theoretical background and key references.

\bigskip

\textbf{Months 3--4: Classical Quantum Circuit Simulation}
\begin{itemize}
    \item Implement small pairing Hamiltonian instances and map them to qubit Hamiltonians.
    \item Build UCC ansätze tailored to pairing (e.g.\ pair excitation operators).
    \item Implement Trotterized UCC circuits in a simulator and generate benchmark data for energies and observables versus $\Delta t$.
\end{itemize}
\textbf{Deliverable:} Verified classical simulation code and initial dataset.

\bigskip

\textbf{Months 5--6: PMM Architecture and Prototype}
\begin{itemize}
    \item Implement basic PMM architectures (Hermitian and unitary variants).
    \item Test PMMs on simplified regression tasks (e.g.\ emulating energies as functions of $\Delta t$ for fixed parameters).
    \item Refine the choice of input features and PMM hyperparameters.
\end{itemize}
\textbf{Deliverable:} Working prototype PMM code and preliminary tests.

\subsection*{Semester 2 (Months 7--12)}

\textbf{Months 7--9: Full PMM Training and Extrapolation Studies}
\begin{itemize}
    \item Generate comprehensive datasets over a range of pairing strengths and UCC parameters.
    \item Train PMMs on Trotterized data and perform $\Delta t \to 0$ extrapolations.
    \item Systematically compare with polynomial and neural network extrapolations.
\end{itemize}
\textbf{Deliverable:} Trained PMM models and detailed numerical analysis of extrapolation quality.

\bigskip

\textbf{Months 10--11: Performance Evaluation and Physical Interpretation}
\begin{itemize}
    \item Analyze the efficiency and robustness of PMMs.
    \item Investigate learned effective spectra and analytic properties.
    \item Explore potential extensions (e.g.\ different ansätze, other interaction channels).
\end{itemize}
\textbf{Deliverable:} Draft of results and discussion chapters for the thesis.

\bigskip

\textbf{Month 12: Thesis Writing and Finalization}
\begin{itemize}
    \item Complete the thesis manuscript, including introduction, methods, results, and conclusions.
    \item Prepare the final oral presentation.
\end{itemize}
\textbf{Deliverable:} Final thesis document and defense presentation.

\section*{5. Expected Outcomes}

\begin{itemize}
    \item A complete computational framework for simulating Trotterized UCC circuits for the pairing Hamiltonian.
    \item A set of parametric matrix models tailored to zero-error Trotter step extrapolation in quantum simulations.
    \item Quantitative benchmarks comparing PMMs to standard extrapolation and machine learning approaches.
    \item Insight into how physics-informed matrix models can improve the reliability and interpretability of quantum algorithms for strongly correlated systems.
\end{itemize}

\section*{6. References}

\begin{enumerate}
    \item P. Cook, D. Jammooa, M. Hjorth-Jensen, D. D. Lee, \textit{Parametric Matrix Models}, Nature Communications 16, 5929 (2025).
    \item A. relevant review on unitary coupled cluster methods for quantum computing (to be added).
    \item Standard references on the pairing Hamiltonian and reduced BCS models in nuclear physics and superconductivity (to be added).
    \item Additional references on Trotter--Suzuki decompositions and NISQ-era quantum algorithms (to be added).
\end{enumerate}

\end{document}
