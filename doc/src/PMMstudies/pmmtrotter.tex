\documentclass[12pt]{article}
\usepackage{amsmath, amssymb, xcolor}

\begin{document}

\begin{center}
{\Large \textbf{Master Thesis Project in Quantum Information Science and Technology}}\\[6pt]
{\large \textbf{Parametric Matrix Models for Trotterized Quantum Dynamics}}\\[6pt]
Duration: 1 year
\end{center}

\section*{Introduction}

Simulating quantum dynamics on quantum computers typically relies on
Trotter or Suzuki--Trotter decompositions of the time-evolution
operator $U(t) = e^{-iHt}$.  While higher-order decompositions reduce
the Trotter error, they also increase the circuit depth and thereby
make real-device execution more demanding.

The recently introduced \emph{Parametric Matrix Models} (PMMs) provide
a new machine-learning architecture whose structure mirrors the matrix
equations of physical systems.  PMMs can emulate eigenvalues,
observables, and even unitary evolution, and show excellent
extrapolation properties.  In particular, PMMs have been shown to
reproduce ``zero-error'' Trotter extrapolations from data taken at
moderate time steps.

This thesis will explore PMMs as a framework for learning effective
Hamiltonians corresponding to different Trotter expansions.  The goal
is to test whether PMMs can systematically reduce Trotter errors,
learn Trotterized unitaries, and predict ideal time evolution from
noisy quantum hardware.  The project will involve both classical
simulations and experiments on real quantum computers (IBM Quantum,
AWS Braket, or Azure Quantum).

\section*{Project Outline}

\textbf{1. Literature Review and Theory}
\begin{itemize}
    \item Review Trotter and Suzuki--Trotter expansions.
    \item Study PMMs as introduced in Cook \emph{et al.}, Nature Comm. (2025).
    \item Understand eigenvalue-based and unitary-based PMM formulations.
\end{itemize}

\textbf{2. PMMs for Trotter Expansion Learning}
\begin{itemize}
    \item Implement affine eigenvalue PMMs and affine observable PMMs.
    \item Construct PMMs that mimic the product structure of Trotterized unitaries.
    \item Train PMMs on simulated quantum data for various Hamiltonians:
    \begin{itemize}
        \item Transverse-field Ising model
        \item Heisenberg and XXZ chains
        \item Fermi-Hubbard model (2--4 qubits)
        \item Lipkin–Meshkov–Glick model
    \end{itemize}
\end{itemize}

\textbf{3. Benchmarking and Model Comparison}
\begin{itemize}
    \item Compare PMM predictions to:
    \begin{itemize}
        \item Exact classical simulation
        \item Standard polynomial extrapolations of Trotter error
        \item Variational quantum algorithms (VQE, Krylov, etc.)
    \end{itemize}
    \item Evaluate performance across system size, entanglement,
    Trotter step size, and Hamiltonian complexity.
\end{itemize}

\textbf{4. Real-Device Quantum Computing Experiments}
\begin{itemize}
    \item Execute Trotterized unitaries for chosen Hamiltonians on real quantum hardware.
    \item Collect data at a range of Trotter step sizes $dt$.
    \item Train PMMs on real-device output data.
    \item Evaluate if PMMs can:
    \begin{itemize}
        \item Extrapolate to the zero-step limit $dt\rightarrow 0$,
        \item Learn effective Hamiltonians $H_{\rm eff}(dt)$,
        \item Compensate for hardware noise.
    \end{itemize}
\end{itemize}

\textbf{5. Extensions and Optional Topics}
\begin{itemize}
    \item PMMs for error mitigation (noise-adapted effective Hamiltonians).
    \item PMMs as surrogates for quantum simulation (hybrid classical--quantum workflows).
    \item Study analytic continuation properties and exceptional points.
\end{itemize}


\section*{Milestones}

\begin{enumerate}
    \item \textbf{Month 1--2:} Literature review, PMM theory, Trotter theory, Qiskit setup.
    \item \textbf{Month 3--4:} Implement PMM codebase (Hermitian + unitary PMMs).
    \item \textbf{Month 5:} Test PMMs on simple 1--2 qubit Hamiltonians.
    \item \textbf{Month 6--7:} Large-scale simulations of Ising, XXZ, and LMG models.
    \item \textbf{Month 8:} Run experiments on IBM Quantum or AWS Braket.
    \item \textbf{Month 9:} Combine real data + PMMs and perform Trotter extrapolation.
    \item \textbf{Month 10:} Benchmark against VQE, polynomial fits, and exact data.
    \item \textbf{Month 11:} Write thesis, finalize figures, summarize results.
    \item \textbf{Month 12:} Complete thesis, prepare presentation.
\end{enumerate}

\section*{Expected Outcomes}

\begin{itemize}
    \item A systematic evaluation of PMMs as Trotter-emulation tools.
    \item A software implementation (Python/Qiskit + PyTorch/Numpy).
    \item A benchmark suite comparing PMMs to conventional extrapolation methods.
    \item Demonstrations of PMM-based extrapolation using real quantum hardware.
\end{itemize}

\end{document}


\section*{6. References}

\begin{enumerate}
    \item P. Cook, D. Jammooa, M. Hjorth-Jensen, D. D. Lee, \textit{Parametric Matrix Models}, Nature Communications 16, 5929 (2025).
    \item Additional references on Trotter--Suzuki decompositions and NISQ-era quantum algorithms, see Yang et al Phys. Rev. A 106, 042401.
\end{enumerate}

\end{document}
