%%
%% Automatically generated file from DocOnce source
%% (https://github.com/hplgit/doconce/)
%%
%%
% #ifdef PTEX2TEX_EXPLANATION
%%
%% The file follows the ptex2tex extended LaTeX format, see
%% ptex2tex: http://code.google.com/p/ptex2tex/
%%
%% Run
%%      ptex2tex myfile
%% or
%%      doconce ptex2tex myfile
%%
%% to turn myfile.p.tex into an ordinary LaTeX file myfile.tex.
%% (The ptex2tex program: http://code.google.com/p/ptex2tex)
%% Many preprocess options can be added to ptex2tex or doconce ptex2tex
%%
%%      ptex2tex -DMINTED myfile
%%      doconce ptex2tex myfile envir=minted
%%
%% ptex2tex will typeset code environments according to a global or local
%% .ptex2tex.cfg configure file. doconce ptex2tex will typeset code
%% according to options on the command line (just type doconce ptex2tex to
%% see examples). If doconce ptex2tex has envir=minted, it enables the
%% minted style without needing -DMINTED.
% #endif

% #define PREAMBLE

% #ifdef PREAMBLE
%-------------------- begin preamble ----------------------

\documentclass[%
oneside,                 % oneside: electronic viewing, twoside: printing
final,                   % draft: marks overfull hboxes, figures with paths
10pt]{article}

\listfiles               %  print all files needed to compile this document

\usepackage{relsize,makeidx,color,setspace,amsmath,amsfonts,amssymb}
\usepackage[table]{xcolor}
\usepackage{bm,ltablex,microtype}

\usepackage[pdftex]{graphicx}

\usepackage[T1]{fontenc}
%\usepackage[latin1]{inputenc}
\usepackage{ucs}
\usepackage[utf8x]{inputenc}

\usepackage{lmodern}         % Latin Modern fonts derived from Computer Modern

% Hyperlinks in PDF:
\definecolor{linkcolor}{rgb}{0,0,0.4}
\usepackage{hyperref}
\hypersetup{
    breaklinks=true,
    colorlinks=true,
    linkcolor=linkcolor,
    urlcolor=linkcolor,
    citecolor=black,
    filecolor=black,
    %filecolor=blue,
    pdfmenubar=true,
    pdftoolbar=true,
    bookmarksdepth=3   % Uncomment (and tweak) for PDF bookmarks with more levels than the TOC
    }
%\hyperbaseurl{}   % hyperlinks are relative to this root

\setcounter{tocdepth}{2}  % levels in table of contents

% prevent orhpans and widows
\clubpenalty = 10000
\widowpenalty = 10000

% --- end of standard preamble for documents ---


% insert custom LaTeX commands...

\raggedbottom
\makeindex
\usepackage[totoc]{idxlayout}   % for index in the toc
\usepackage[nottoc]{tocbibind}  % for references/bibliography in the toc

%-------------------- end preamble ----------------------

\begin{document}

% matching end for #ifdef PREAMBLE
% #endif

\newcommand{\exercisesection}[1]{\subsection*{#1}}


% ------------------- main content ----------------------



% ----------------- title -------------------------

\thispagestyle{empty}

\begin{center}
{\LARGE\bf
\begin{spacing}{1.25}
Nuclear Physics Experiments and  Machine Learning
\end{spacing}
}
\end{center}

% ----------------- author(s) -------------------------

\begin{center}
{\bf Master of Science thesis project${}^{}$} \\ [0mm]
\end{center}

\begin{center}
% List of all institutions:
\end{center}
    
% ----------------- end author(s) -------------------------

% --- begin date ---
\begin{center}
Dec 1, 2019
\end{center}
% --- end date ---

\vspace{1cm}


\subsection{Machine Learning for interpreting Nuclear Physics experiments}


Machine Learning plays nowadays a central role
in the analysis of large data sets in order to extract information
about complicated correlations. This information is often difficult to
obtain with traditional methods. For example, there are about one
trillion web pages; more than one hour of video is uploaded to YouTube
every second, amounting to 10 years of content every day; the genomes
of 1000s of people, each of which has a length of $3.8\times 10^9$
base pairs, have been sequenced by various labs and so on. This deluge
of data calls for automated methods of data analysis, which is exactly
what machine learning provides.  Developing activities in these
frontier computational technologies is thus of strategic importance
for our capability to address future science problems.


These thesis projects aim at using deep learning methods such as
convolutional neural networks, recurrent neural networks, variational
autoenconders, generative adversarial networks (GAN) for both
supervised and unsupervided problems from experimental physics.  Here
in particular we will explore experimental results from nuclear
physics experiments.

There are two main projects:


\paragraph{$\beta$-decay Experiments.}
The classical picture of spherical nuclei is far from the reality of
the true nuclear structure. Shape coexistence is a nuclear phenomenon,
where the nucleus exists in two stable shapes at \href{{https://www.europhysicsnews.org/articles/epn/pdf/2001/01/epn01101.pdf}}{the same excitation
energy}.


Nuclear properties provide unique information on the impetuses that
foster changes to the nuclear structure of rare isotopes. In some
neutron-rich nuclei, $0^{+}$ states are predicted to exhibit shape
coexistence. Therefore they are compelling to study, but \href{{http://iopscience.iop.org/article/10.1088/0954-3899/43/2/024001}}{experimentally
challenging}.
At low energies, where the only energetically allowed decay mode is
$0^{+} \rightarrow 0^{+}$, conversion electron spectroscopy is the
only viable technique to probe their properties.



At the National Superconducting Cyclotron Laboratory at Michigan State
University Sean Liddick's group employs conversion electron
spectroscopy to study these transition rates. When a neutron-rich
nucleus beta decays, a neutron transforms into a proton and emits an
electron $\beta$. The excited nucleus can then interact
electromagnetically with the surrounding orbital electrons. This can
result in the ejection of an internal conversion electron $e^{-}$ from
the
\href{{https://www.sciencedirect.com/science/article/pii/S0065253908608884}}{atom}.
Because this process is essentially simultaneous in time, it is
pivotal to differentiate between the electron $\beta$ emitted from the
nucleus and the internal conversion electron $e^{-}$ emitted from the
atom.

This project attempts to use supervised machine learning algorithms as a
means to distinguish between one and two electron events and predict the
electron(s) corresponding initial position(s) in a scintillator.


\paragraph{Classification in the Active-Target Time Projection Chamber.}
In this thesis project, the aim is to evaluate machine learning methods for
event classification in the Active-Target Time Projection Chamber
(AT-TPC) detector at the National Superconducting Cyclotron Laboratory
(NSCL) at Michigan State University. The AT-TPC detects products from
nuclear physics reactions to study the nuclear structure of rare
isotopes. The detector records many different types of events, but
experimentalists are typically only interested in one reaction
product. We will develop an automated method to single out the desired
reaction product, which may result in more accurate physics results as
well as a faster analysis process. Single-class, binary, and
multi-class classification methods based on deep neural networks will
be developed and tested against earlier and recent experiments at the
NSCL.  This project is a continuation of a previous thesis project which ended in a recent scientific publication.
For more information see the \href{{https://github.com/copperwire/thesis/blob/master/main.pdf}}{Master of Science Thesis of Robert Solli}.


The focus is to  use and explore convolutional neural networks, recurrent neural networks, variational
autoenconders and generative adversarial networks for analyzing nuclear physics experiments.



The milestones are as follows

\begin{enumerate}
\item Spring 2020: Analyze simulated data with Convolutional Networks and reproduce results from simulations

\item Fall 2020: Include other deep learning methods such as reinforcement learning and autoencoders and analyse data from experiments at Michigan State University

\item Spring 2021: Finalize thesis project.
\end{enumerate}

\noindent
The thesis is expected to be handed in May/June  2021.




































% ------------------- end of main content ---------------

% #ifdef PREAMBLE
\end{document}
% #endif

