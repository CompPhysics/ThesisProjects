\documentclass{article}
\usepackage[utf8]{inputenc}
\usepackage{natbib}


\title{Machine Learning for Text Analysis in Physics Education and Political Science}

\author{Gunnar Thorsen Liahjell}

\date{January 2019}


\begin{document}


\maketitle

\section*{Introduction}


Machine Learning (ML) is one of the most exciting and dynamic areas of
modern research and application. The purpose of this thesis is to
explore and develop algortihms for text analysis, with an emphasis on
applications to Physics Education research and studies in Political
Science. Although being disconnected disciplines, the unifying theme
here is a novel application of ML methods to text analysis.

The rich area of text analytics draws ideas from information
retrieval, machine learning, and natural language processing. Each of
these areas is an active and vibrant field in its own right, and
numerous books have been written in each of these different areas.

Text analytics can be split into three main cathegories:

\begin{enumerate}
\item Fundamental algorithms and models: Many
fundamental applications in text analytics, such as matrix
factorization, clustering, and classification, have uses in domains
beyond text. Nevertheless, these methods need to be tailored to the
specialized characteristics of text. 
\item Information retrieval and ranking: Many aspects of
information retrieval and ranking are closely related to text
analytics. For example, ranking Support Vector machines and link-based ranking are often
used for learning from text and text mining. 
\item  Sequence- and natural language-centric text mining: Although
multidimensional representations can be used for basic applications
in text analytics, the true richness of the text representation can be
leveraged by treating text as sequences. This entails the studies of
these advanced topics like sequence embedding, deep learning,
information extraction, summarization, opinion mining, text
segmentation, and event extraction.
\end{enumerate}

The focus of this thesis will be on the third item above and two major
areas of applications will be studied. These include text mining in
Physics Education and text mining in Political Science


\section{Natural Language Processing in Physics Education Research}

This research project is tightly linked with the quantitative
educational ML projects of the newly established center of excellence
in education at the University of Oslo, the Center for Computing in
Science Education (CCSE). Machine Learning algorithms, supervised and
unsupervised applied to regression and classsification problems open
up for a truly quantitative and mathematical rigorous approach to
education research.  A Machine Learning based analysis of texts, in
particular how students respond to exercises and projects, has the
potential to add significant insights on how students develop their
insights about scientific problems and specific disciplines. Here we
will firstly focus on the Physics Education Research Conference
proceedings, which has acted as the “pulse” of Physics Education
Research for over 15 years. Currently receiving hundreds of annual
submissions and having published hundreds of articles, summarizing the
contributions to the field can be incredibly time consuming. Natural
language processing (NLP) and Machine Learning offer a powerful
unsupervised learning tools that can be used to “cluster” different
documents concerning different topics. Data from more than 15 years of
the Physics Education Research Conference proceedings will be
summarized using NLP and ML techniques and changes over time will be
documented.

If this is succesful, we will also analyze data from the USA on how
students develop insights about the scientific method as well as about
a specific physics problem. The data set consists of a large body of
essays written by undergraduate students in physics.

John Aiken from the CCSE will act as co-supervisor on this project.

\section{Machine Learning and text analysis of EU directivies}

Based on the developed algorithms and codes for ML applied to text
analytics, the second aim of this project is to use thr ML algorithm
to help determine where EU directives and regulations originate
from. The download - upload model, states that part of the legislative
process in the EU is member countries "uploading" parts of their
existing national body of legislation to be incorporated to the EU
legislation.  If this is the case one should be able to identify parts
of the native legislative texts from their origin country in the
EU-legislation. Based on this a machine learning algorithm will be
implemented to compare and examine legislative texts.

By taking as input the national legislation of the countries before a
EU regulation is made, and the finished EU-regulation, and search for
similarities, for instance how much of the respective countries
national legislation is to be found in the EU regulation one can use
this as a proxy for assessing which country or countries has the most
influence on the resulting regulation. As all EU-laws are translated
to all languages in the EU, language differences doe not need to be
taken into consideration.


It is expected to find various levels of what Padgett calls synthesis
and emulation \citep{Padgett} (meaning mixes of several states,
synthesis, or more or less copying from one state,emulation),
depending of what field one are to consider, but what is said about
this earlier is qualitative and inconsistent, so this quantitative
approach might serve as a backbone for further research.

To evaluate this model one can follow the similarities into the
download-phase, where the countries of the EU have to implement the
laws following the guidelines set by the directive and see to what
degree the member countries follow it. That is, how compliant the
country is, which is a subject that has been more studied and thus
have more material to compare with. So one can see if this method can
replicate the findings of various articles written on the compliance
of EU countries to EU law \citep{Toshkov} \citep{Tanja}
\citep{toshkov2} \citep{thoman}.

When this is done, we will look at how these factors change over time, to see if the dominant countries become more dominant or less, or if they change. 

Professor Bj\o rn H\o yland from the department of Political Sciences will act as co-supervisor here.

\section{Milestones}

The milestones are as follows
\begin{itemize}
\item Spring 2019: Develop, based on recurrent neural networks, reiforcement learning  and autoencoders, code and theory for analyzing text. In particular, develop code with the aim to extract specific phrases and sentences. 

\item Fall 2019: Start including selected texts from Educational data and the EU and apply the above Machine Learning techniques to these. Start analyzing the data.
\item Spring 2020: Final analysis of data and wrap up of thesis. 
\end{itemize}
The thesis is expected to be handed in May/June  2020.



\bibliographystyle{plain}

\bibliography{literature.bib}


\end{document}


k
