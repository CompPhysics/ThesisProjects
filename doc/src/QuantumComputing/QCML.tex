%%
%% Automatically generated file from DocOnce source
%% (https://github.com/doconce/doconce/)
%% doconce format latex QCML.do.txt --print_latex_style=trac --latex_admon=paragraph
%%


%-------------------- begin preamble ----------------------

\documentclass[%
oneside,                 % oneside: electronic viewing, twoside: printing
final,                   % draft: marks overfull hboxes, figures with paths
10pt]{article}

\listfiles               %  print all files needed to compile this document

\usepackage{relsize,makeidx,color,setspace,amsmath,amsfonts,amssymb}
\usepackage[table]{xcolor}
\usepackage{bm,ltablex,microtype}

\usepackage[pdftex]{graphicx}

\usepackage[T1]{fontenc}
%\usepackage[latin1]{inputenc}
\usepackage{ucs}
\usepackage[utf8x]{inputenc}

\usepackage{lmodern}         % Latin Modern fonts derived from Computer Modern

% Hyperlinks in PDF:
\definecolor{linkcolor}{rgb}{0,0,0.4}
\usepackage{hyperref}
\hypersetup{
    breaklinks=true,
    colorlinks=true,
    linkcolor=linkcolor,
    urlcolor=linkcolor,
    citecolor=black,
    filecolor=black,
    %filecolor=blue,
    pdfmenubar=true,
    pdftoolbar=true,
    bookmarksdepth=3   % Uncomment (and tweak) for PDF bookmarks with more levels than the TOC
    }
%\hyperbaseurl{}   % hyperlinks are relative to this root

\setcounter{tocdepth}{2}  % levels in table of contents

% prevent orhpans and widows
\clubpenalty = 10000
\widowpenalty = 10000

% --- end of standard preamble for documents ---


% insert custom LaTeX commands...

\raggedbottom
\makeindex
\usepackage[totoc]{idxlayout}   % for index in the toc
\usepackage[nottoc]{tocbibind}  % for references/bibliography in the toc

%-------------------- end preamble ----------------------

\begin{document}

% matching end for #ifdef PREAMBLE

\newcommand{\exercisesection}[1]{\subsection*{#1}}


% ------------------- main content ----------------------



% ----------------- title -------------------------

\thispagestyle{empty}

\begin{center}
{\LARGE\bf
\begin{spacing}{1.25}
Quantum Computing and Quantum Machine Learning
\end{spacing}
}
\end{center}

% ----------------- author(s) -------------------------

\begin{center}
{\bf Master of Science Thesis Project${}^{}$} \\ [0mm]
\end{center}

\begin{center}
% List of all institutions:
\end{center}
    
% ----------------- end author(s) -------------------------

% --- begin date ---
\begin{center}
Dec 1, 2022
\end{center}
% --- end date ---

\vspace{1cm}


\subsection*{Quantum Computing and Machine Learning}

\textbf{Quantum Computing and Machine Learning} are two of the most promising
approaches for studying complex physical systems where several length
and energy scales are involved.  Traditional many-particle methods,
either quantum mechanical or classical ones, face huge dimensionality
problems when applied to studies of systems with many interacting
particles. To be able to define properly effective potentials for
realistic molecular dynamics simulations of billions or more
particles, requires both precise quantum mechanical studies as well as
algorithms that allow for parametrizations and simplifications of
quantum mechanical results. Quantum Computing offers now an
interesting avenue, together with traditional algorithms, for studying
complex quantum mechanical systems. Machine Learning on the other hand
allows us to parametrize these results in terms of classical
interactions. These interactions are in turn suitable for large scale
molecular dynamics simulations of complicated systems spanning from
subatomic physics to materials science and life science.

\subsection*{Possible  Projects}

\paragraph{Boltzmann machines, from classical ones to quantum Boltzmann machines (Classical and Quantum Machine Learning).}
Boltzmann Machines (BMs) offer a powerful framework for modeling
probability distributions.  These types of neural networks use an
undirected graph-structure to encode relevant information.  More
precisely, the respective information is stored in bias coefficients
and connection weights of network nodes, which are typically related
to binary spin-systems and grouped into those that determine the
output, the visible nodes, and those that act as latent variables, the
hidden nodes.

Furthermore, the network structure is linked to an energy function
which facilitates the definition of a probability distribution over
the possible node configurations by using a concept from statistical
mechanics, i.e., Gibbs states.  The aim of BM training is to learn a
set of weights such that the resulting model approximates a target
probability distribution which is implicitly given by training data.
This setting can be formulated as discriminative as well as generative
learning task.  Applications have been studied in a large variety of
domains such as the analysis of quantum many-body systems, statistics,
biochemistry, social networks, signal processing and finance

BMs are complicated to train in practice because the loss
function's derivative requires the evaluation of a normalization
factor, the partition function, that is generally difficult to
compute.  Usually, it is approximated using Markov Chain Monte Carlo
methods which may require long runtimes until convergence

Quantum Boltzmann Machines (QBMs) are a natural adaption of BMs to the
quantum computing framework. Instead of an energy function with nodes
being represented by binary spin values, QBMs define the underlying
network using a Hermitian operator, normally a parameterized
Hamiltonian, see references [1,2] below.

Here we will focus on classification problems such as the famous MNIST
data set which contains handwritten numbers from $0$ to $9$.  This can
serve as a starting point.  More data sets can be included at a later
stage.  The next project parallels this but replaces Boltzmann
machines with Autoencoders.

\textbf{Literature:}

\begin{enumerate}
\item Amin et al., \textbf{Quantum Boltzmann Machines}, Physical Review X \textbf{8}, 021050 (2018).

\item Zoufal et al., \textbf{Variational Quantum Boltzmann Machines}, ArXiv:2006.06004.

\item Maria Schuld and Francesco Petruccione, \textbf{Supervised Learning with Quantum Computers}, Springer, 2018.
\end{enumerate}

\noindent
\paragraph{From Classical Autoenconders to Quantum Autoenconders and classification problems (Classical and Quantum Machine Learning).}
Classical autoencoders are neural networks that can learn efficient
low dimensional representations of data in higher dimensional
space. The task of an autoencoder is, given an input $x$, is to map
$x$ to a lower dimensional point $y$ such that $x$ can likely be
recovered from $y$. The structure of the underlying autoencoder
network can be chosen to represent the data on a smaller dimension,
effectively compressing the input.

Inspired by this idea, we would like to, following references [1,2]
below, to introduce Quantum Autoencoders to compress a particular
dataset like the famous MNIST data set which contains handwritten
numbers from $0$ to $9$.

\textbf{Literature:}

\begin{enumerate}
\item Carlos Bravo-Prieto, \textbf{Quantum autoencoders with enhanced data encoding}, see \href{{https://arxiv.org/pdf/2010.06599.pdf}}{\nolinkurl{https://arxiv.org/pdf/2010.06599.pdf}} 

\item Jonathan Romero et al, \textbf{Quantum autoencoders for efficient compression of quantum data}, see \href{{https://arxiv.org/pdf/1612.02806.pdf}}{\nolinkurl{https://arxiv.org/pdf/1612.02806.pdf}}

\item Maria Schuld and Francesco Petruccione, \textbf{Supervised Learning with Quantum Computers}, Springer, 2018. See \href{{https://www.springer.com/gp/book/9783319964232}}{\nolinkurl{https://www.springer.com/gp/book/9783319964232}}
\end{enumerate}

\noindent
\paragraph{Bayesian phase difference estimation (Quantum-mechanical many-body Physics).}
Quantum computers can perform full configuration interaction (full-CI)
calculations by utilising the quantum phase estimation (QPE)
algorithms including Bayesian phase estimation (BPE) and iterative
quantum phase estimation (IQPE). In these quantum algorithms, the time
evolution of wave functions for atoms and molecules is simulated
conditionally with an ancillary qubit as the control, which make
implementation to real quantum devices difficult. Also, most of the
problems in many-body physics discuss energy differences between two
states rather than total energies themselves, and thus
direct calculations of energy gaps in for example atoms and molecules are promising for future
applications of quantum computers to real quantum mechanical many-body problems. In the
race of finding efficient quantum algorithms to solve quantum
chemistry problems, we would like to study  a Bayesian phase difference estimation
(BPDE) algorithm, which is a general algorithm to calculate the
difference of two eigenphases of unitary operators in the several
cases of the direct calculations of energy gaps between selected many-body states
states on quantum computers.

\paragraph{Variational Quantum Eigensolvers (Quantum-mechanical many-body Physics).}
The specific task here is to implenent and study Quantum Computing
algorithms like the Quantum-Phase Estimation algorithm and Variational
Quantum Eigensolvers for solving quantum mechanical many-particle
problems.  Recent scientific articles have shown the reliability of
these methods on existing and real quantum computers, see for example
references [1-5] below.

Here the focus is first on tailoring a Hamiltonian like the pairing
Hamiltonian and/or Anderson Hamiltonian in terms of quantum gates, as
done in references [3-5].

Reproducing these results will be the first step of this thesis
project. The next step includes adding more complicated terms to the
Hamiltonian, like a particle-hole interaction as done in the work of
\href{{http://iopscience.iop.org/article/10.1088/0954-3899/37/6/064035/meta}}{Hjorth-Jensen et
al}.

The final step is to implement the action of these Hamiltonians on
existing quantum computers like \href{{https://www.rigetti.com/}}{Rigetti's Quantum
Computer}.

The projects can easily be split into several parts and form the basis
for collaborations among several students.

\textbf{Literature:}

\begin{enumerate}
\item Dumitrescu et al, see \href{{https://arxiv.org/abs/1801.03897}}{\nolinkurl{https://arxiv.org/abs/1801.03897}}

\item Yuan et al., \textbf{Theory of Variational Quantum Simulations}, see  \href{{https://arxiv.org/abs/1812.08767}}{\nolinkurl{https://arxiv.org/abs/1812.08767}} 

\item Ovrum and Hjorth-Jensen, see \href{{https://arxiv.org/abs/0705.1928}}{\nolinkurl{https://arxiv.org/abs/0705.1928}}

\item Stian Bilek, Master of Science Thesis, University of Oslo, 2020, see \href{{https://www.duo.uio.no/handle/10852/82489}}{\nolinkurl{https://www.duo.uio.no/handle/10852/82489}}

\item Heine Åbø Olsson, Master of Science Thesis, University of Oslo, 2020, see \href{{https://www.duo.uio.no/handle/10852/81259}}{\nolinkurl{https://www.duo.uio.no/handle/10852/81259}}
\end{enumerate}

\noindent
\paragraph{Analysis of entanglement in quantum computers.}
The aim of this project is to study various ways of analyzing
entanglement theoretically by computing for example the von Neumann
entropy of a quantum mechanical many-body system. The systems we are
aiming at here are so-called quantum dots systems which are candidates
from making quantum gates and circuits. The theoretical results are
planned to be linked with experimental interpretations via Quantum
state tomography, which is the standard technique for estimating the
quantum state of small systems.  If possible, this project could be
linked with studies of quantum tomography from the SpinQ quantum
computer at OsloMet.

\textbf{Literature:}

\begin{enumerate}
\item B. P. Lanyon et al, \textbf{Efficient tomography of a quantum many-body system}, Nature Physics \textbf{13}, 1158 (2017), see \href{{https://www.nature.com/articles/nphys4244}}{\nolinkurl{https://www.nature.com/articles/nphys4244}}
\end{enumerate}

\noindent

% ------------------- end of main content ---------------

\end{document}

