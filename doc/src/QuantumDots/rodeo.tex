\section*{Master Thesis Project}

\subsection*{Title}
\textbf{Exploring and Extending the Rodeo Algorithm for Quantum Eigenvalue Problems on Real Quantum Hardware}

\subsection*{Introduction}

Estimating eigenvalues and preparing eigenstates of quantum Hamiltonians are central tasks in quantum many-body physics, quantum chemistry, and nuclear structure theory. The recently proposed \emph{rodeo algorithm} (RA) provides a stochastic, interference-based approach for suppressing unwanted spectral components and isolating the eigenstate associated with a chosen target energy. The algorithm scales exponentially better than traditional techniques such as Quantum Phase Estimation or adiabatic state preparation, assuming sufficient overlap with the target eigenstate.

The article \emph{``Demonstration of the rodeo algorithm on a quantum computer''} presents the first implementation of the RA on real quantum hardware, demonstrating high-precision energy estimation for one-qubit Hamiltonians using IBM Q devices. The method was further validated through Hellmann–Feynman theorem calculations and comparison with direct eigenvector preparation.

This master's thesis aims to (i) reproduce and critically evaluate these results, (ii) extend the RA to multi-qubit Hamiltonians, (iii) compare RA performance to alternative quantum algorithms, and (iv) investigate practical performance on modern quantum hardware with noise mitigation techniques. The thesis will involve substantial computational work, both in classical simulations and real quantum-computer executions.

\subsection*{Project Outline}

\begin{enumerate}

\item \textbf{Theoretical Foundations}
  \begin{itemize}
    \item Study the rodeo algorithm, its stochastic structure, controlled time evolution, and peak-finding strategy.
    \item Review competing algorithms: Quantum Phase Estimation (QPE), Iterative QPE, Variational Quantum Eigensolver (VQE), and quantum Krylov methods.
    \item Analyze the Hellmann–Feynman theorem in the context of quantum computing.
  \end{itemize}

\item \textbf{Reproducing the One-Qubit Demonstration}
  \begin{itemize}
    \item Implement the RA for general one-qubit Hamiltonians.
    \item Classically simulate the success-probability distribution \( P_{0^N}(E) \) using randomized time evolutions.
    \item Run the circuits on IBM Q backends and compare raw data with the published results.
    \item Integrate measurement error mitigation and study its effects.
  \end{itemize}

\item \textbf{Extension to Multi-Qubit Hamiltonians}
  \begin{itemize}
    \item Construct 2–4 qubit Hamiltonians relevant for nuclear physics toy models or spin systems.
    \item Implement Trotterized time evolution and analyze Trotter errors.
    \item Evaluate the RA’s ability to resolve multi-level spectra.
    \item Explore the scaling of RA performance with system size.
  \end{itemize}

\item \textbf{Comparison with Other Quantum Algorithms}
  \begin{itemize}
    \item Implement VQE and Krylov-subspace eigensolvers.
    \item Benchmark accuracy, resource cost, and robustness to noise.
    \item Investigate the RA’s operational advantage under realistic noise.
  \end{itemize}

\item \textbf{Applications on Real Quantum Computers}
  \begin{itemize}
    \item Deploy the algorithms on IBM Q and, if possible, Quantinuum or Amazon Braket backends.
    \item Systematically compare results: noise, gate depth, mid-circuit measurement quality.
    \item Study the effect of advanced error-mitigation techniques:
      \begin{itemize}
        \item measurement calibration,
        \item readout symmetrization,
        \item Zero-Noise Extrapolation (ZNE),
        \item Probabilistic Error Cancellation (PEC).
      \end{itemize}
    \item Extract eigenvalues and use the Hellmann–Feynman approach to compute expectation values.
  \end{itemize}

\item \textbf{Exploratory Directions (Optional)}
  \begin{itemize}
    \item Optimization of the RA time-distribution parameters (e.g.\ non-Gaussian choices).
    \item Machine-learning tuning of initial states to maximize overlap.
    \item Application to effective Hamiltonians arising from nuclear many-body physics.
  \end{itemize}

\end{enumerate}

\subsection*{Milestones and Timeline (12 Months)}

\begin{itemize}
  \item \textbf{Month 1–2:} Literature review; basic RA implementation in classical simulation; reproduction of theoretical formulas.
  \item \textbf{Month 3–4:} Reproduction of the one-qubit energy-scan experiment using simulators; first runs on IBM Q.
  \item \textbf{Month 5–6:} Hellmann–Feynman calculations; error-mitigation benchmarking; detailed comparison to article data.
  \item \textbf{Month 7–8:} Extension to 2–4 qubit Hamiltonians; Trotterization; classical and quantum benchmarks.
  \item \textbf{Month 9:} Comparison with VQE, Krylov, and QPE approaches on identical Hamiltonians.
  \item \textbf{Month 10–11:} Hardware-based performance study; statistical scaling; noise analysis.
  \item \textbf{Month 12:} Final analysis; writing and preparation of the thesis.
\end{itemize}

\subsection*{Expected Outcomes}

\begin{itemize}
  \item A full computational implementation of the rodeo algorithm.
  \item Validation and extension of published one-qubit results.
  \item First exploration of RA applied to multi-qubit Hamiltonians on real hardware.
  \item Comparative study with VQE, Krylov, and QPE methods under noise.
  \item Assessment of error-mitigation strategies for eigenvalue problems.
  \item A comprehensive written thesis integrating theory, simulation, and hardware experiments.
\end{itemize}
