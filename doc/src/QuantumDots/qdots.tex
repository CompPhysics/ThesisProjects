\section*{Master Thesis Project}

\subsection*{Title}
\textbf{Computational Studies of Coulomb–Interaction-Driven Entanglement of Electrons on Helium}

\subsection*{Introduction}

The development of quantum technologies using naturally trapped electrons on the surface of superfluid helium has recently gained considerable momentum due to their extremely long coherence times and highly controllable motional degrees of freedom.  
The PRX Quantum article \emph{``Coulomb Interaction–Driven Entanglement of Electrons on Helium''} (Beysengulov \emph{et al.}, 2024) introduces a realistic two-electron device in which microwave-frequency motional states are confined within electrostatically tunable double-well potentials.  
The Coulomb interaction between the electrons gives rise to strongly correlated motional states and allows entanglement to be generated and controlled via voltage tuning.

This master thesis aims to reproduce, extend, and explore computational models of such trapped electrons, using full configuration–interaction (CI) methods, Hartree or Hartree–Fock approximations, and effective-model Hamiltonians.  
The project combines quantum many-body methods, numerical diagonalization, optimization, and device modelling, with an eye toward providing theoretical guidance for experimental realizations.

\subsection*{Project Description and Goals}

The main goal of this project is to develop a computational framework for studying motional entanglement between two (and potentially more) electrons trapped on helium.  
The model involves electrostatically generated one-dimensional potentials, tunable device parameters, and unscreened Coulomb interactions.  
Numerical diagonalization of the Hamiltonian will enable studies of:

\begin{itemize}
  \item Single-electron motional spectra in electrostatic traps.
  \item Two-electron interaction-induced splittings and avoided crossings.
  \item Generation and quantification of entanglement using von Neumann entropy.
  \item Conditions for realizing specific entangling gates (iSWAP, CZ).
  \item Optimization of trap voltages to achieve gate-relevant level structures.
  \item Comparison between exact CI and simplified effective Hamiltonians.
\end{itemize}

Depending on progress, extensions may include:
\begin{itemize}
  \item Investigation of decoherence channels (ripplons, phonons).
  \item Time-dependent driving and simulation of gate protocols.
  \item Scaling to three or more electrons in a microchannel.
\end{itemize}

All tasks will be implemented computationally using Python (NumPy/SciPy), Julia, or C++ with emphasis on efficient linear algebra and numerical optimization.

\subsection*{Detailed Outline}

\begin{enumerate}
  \item \textbf{Physical System and Literature Review}
    \begin{itemize}
      \item Review trapped electron systems on helium and their role in quantum information.
      \item Study the PRX Quantum article and relevant background.
      \item Understand motional quantization, electrode-generated potentials, and CI methods.
    \end{itemize}

  \item \textbf{Device Modelling and Electrostatic Potentials}
    \begin{itemize}
      \item Implement model potentials based on electrode geometries.
      \item Reproduce the parametrized double-well potentials used in the article.
      \item Explore how varying electrode voltages affects trap anharmonicity and detuning.
    \end{itemize}

  \item \textbf{Single-Particle Basis Construction}
    \begin{itemize}
      \item Solve the one-body Schrödinger equation for left/right wells.
      \item Implement both numerical eigenstates and Hartree (or Hartree–Fock) approximations.
      \item Study basis truncation and convergence.
    \end{itemize}

  \item \textbf{Two-Electron Hamiltonian and Full CI Solver}
    \begin{itemize}
      \item Construct the full Hamiltonian using single-particle product bases.
      \item Compute matrix elements of the soft Coulomb interaction.
      \item Implement exact diagonalization using optimized sparse or dense solvers.
      \item Validate against published spectra.
    \end{itemize}

  \item \textbf{Entanglement and Avoided Crossings}
    \begin{itemize}
      \item Compute von Neumann entropies from reduced density matrices.
      \item Identify and characterize avoided crossings (configuration I, II, III).
      \item Analyze wave-function coefficients and Schmidt decompositions.
    \end{itemize}

  \item \textbf{Optimization of Voltage Configurations}
    \begin{itemize}
      \item Implement gradient or gradient-free optimization (Adam, Nelder–Mead).
      \item Reproduce the paper’s voltage sets for gate configurations.
      \item Explore alternative voltage configurations and robustness.
    \end{itemize}

  \item \textbf{Effective Hamiltonian Comparison}
    \begin{itemize}
      \item Derive the simplified two-mode harmonic-coupling Hamiltonian.
      \item Compute effective coupling strengths and compare to CI results.
      \item Analyze limitations of harmonic approximations and anharmonic corrections.
    \end{itemize}

  \item \textbf{Optional Extensions}
    \begin{itemize}
      \item Time-dependent gate simulations (Floquet, piecewise-constant control).
      \item Studies of decoherence via ripplons and phonons.
      \item Multi-electron extension using tensor-network or CI truncation.
    \end{itemize}

\end{enumerate}

\subsection*{Milestones and Timeline (One Year)}

\begin{itemize}
  \item \textbf{Month 1–2:} Literature review, reproduction of single-electron potentials, basic 1D Schrödinger solver.
  \item \textbf{Month 3–4:} Implementation of CI framework, testing basis convergence.
  \item \textbf{Month 5–6:} Reproduction of configuration I (detuned wells) and entanglement analysis.
  \item \textbf{Month 7–8:} Implementation of voltage optimization and identification of configuration II (iSWAP) and III (CZ).
  \item \textbf{Month 9:} Effective-model derivation and comparison with CI.
  \item \textbf{Month 10–11:} Extensions (time evolution, decoherence modelling, or multi-electron simulations).
  \item \textbf{Month 12:} Writing, verification, and final analysis.
\end{itemize}

\subsection*{Expected Outcomes}

\begin{itemize}
  \item A fully functioning computational toolbox for simulating electrons on helium.
  \item Reproduction and extension of the PRX Quantum results.
  \item Detailed understanding of Coulomb-driven entanglement generation.
  \item Exploration of gate-relevant avoided crossings and device optimization.
  \item A written thesis synthesizing methods, physics, and computational findings.
\end{itemize}
