\documentclass[12pt,a4paper]{article}
\usepackage{amsmath,amsfonts,amssymb}
\usepackage{graphicx}
\usepackage{hyperref}
\usepackage{enumitem}
\usepackage{geometry}
\geometry{margin=2.5cm}

\begin{document}

\begin{center}
    {\Large \textbf{Master Thesis Project in Computational Physics}}\\[0.3cm]
    {\large \textbf{Parametric Matrix Models for Solving the Black--Scholes Equation}}\\[0.2cm]
    \vspace{0.3cm}
    {Duration: One Year (60 ECTS)}
\end{center}

\vspace{1cm}

\section*{1. Introduction}

The Black--Scholes partial differential equation (PDE) is one of the
central equations in quantitative finance.  It describes the fair
price of European options as a function of time, volatility, and
underlying asset price.  Solutions to the PDE can be obtained
analytically in certain cases, but more complex derivatives or
modified market assumptions require numerical methods such as finite
differences or finite elements.

Machine learning methods have recently gained attention as surrogate
approximators for PDE solutions, offering the potential for efficient
evaluation and parameter sensitivity exploration.  However, most
neural-network approaches are not constructed to respect the
underlying analytical properties of the PDE.  They often lack
interpretability, extrapolation reliability, and stability.

In contrast, \textit{Parametric Matrix Models} (PMMs), introduced by
Cook \textit{et al.} (Nature Communications 16, 5929 (2025)), provide
a physics-inspired framework rooted in matrix equations and operator
theory.  The PMM architecture replaces traditional neural networks
with parametrized matrices whose spectra and eigenvectors generate
outputs.
These models:
\begin{itemize}
    \item are universal function approximators,
    \item allow analytic continuation,
    \item incorporate physical constraints directly into their structure,
    \item have excellent extrapolation properties,
    \item often require far fewer trainable parameters than neural networks.
\end{itemize}

This thesis aims to combine the theory of PMMs with the financial
mathematics of the Black--Scholes equation to build a reduced-order
surrogate model for option pricing.  The goal is to investigate
whether PMMs can serve as highly efficient and interpretable models
for PDE-based pricing in computational finance.

\section*{2. Scientific Motivation}

The Black--Scholes PDE is given by
\[
\frac{\partial V}{\partial t}
+ \frac{1}{2}\sigma^2 S^2 \frac{\partial^2 V}{\partial S^2}
+ r S \frac{\partial V}{\partial S}
- rV = 0,
\]
with terminal condition $V(S,T) = \max(S-K,0)$.

Traditional numerical solvers compute the full PDE solution on a
discretized spatial grid.  This is computationally expensive when
evaluating the price repeatedly for different parameter values
$(\sigma, r, K)$.  A PMM surrogate model may achieve:
\begin{itemize}
    \item reduced computational complexity,
    \item efficient multi-parameter interpolation,
    \item automatic incorporation of analyticity of the solution,
    \item excellent performance across a compact domain of input features.
\end{itemize}

Applying PMMs to the Black--Scholes equation also offers a test case
for more complex PDEs in physics, finance, and engineering.

\section*{3. Objectives of the Thesis}

The main objective of this thesis is to develop, implement, and test a
parametric matrix model (PMM) for solving the Black--Scholes PDE. The
work involves theoretical analysis, numerical implementation, and
extensive benchmarking.

\subsection*{Objective 1: Understanding PMMs and Reduced-Order Methods}
\begin{itemize}
    \item Study and summarize the PMM framework introduced by Cook \textit{et al.}  
    \item Analyze connections to eigenvector continuation, reduced basis methods, and implicit-function models.
\end{itemize}

\subsection*{Objective 2: Numerical Foundation of the Black--Scholes PDE}
\begin{itemize}
    \item Derive the Black--Scholes equation and its analytical solution.
    \item Implement a stable finite-difference solver as the reference solution.
\end{itemize}

\subsection*{Objective 3: Parametric Matrix Model for PDE Surrogate Learning}
\begin{itemize}
    \item Construct PMMs where the input features are $(S, t, \sigma, r, K)$ or suitable lower-dimensional encodings.
    \item Explore Hermitian vs.\ unitary PMM formulations.
    \item Investigate how boundary conditions and payoff constraints can be embedded into secondary matrices.
\end{itemize}

\subsection*{Objective 4: Training, Optimization, and Stability}
\begin{itemize}
    \item Implement gradient-based learning using the analytic gradients described in the PMM paper.
    \item Train models on option–price datasets generated from the PDE solver.
    \item Study extrapolation behavior with respect to large or small volatility, deep in/out-of-the-money regimes, and long maturities.
\end{itemize}

\subsection*{Objective 5: Benchmarking and Comparisons}
\begin{itemize}
    \item Compare PMMs with:
    \begin{itemize}
        \item feed-forward neural networks,
        \item physics-informed neural networks,
        \item standard surrogate models (Gaussian processes, Kernel Ridge Regression).
    \end{itemize}
    \item Evaluate:
    \begin{itemize}
        \item accuracy,
        \item computational efficiency,
        \item number of trainable parameters,
        \item extrapolation quality.
    \end{itemize}
\end{itemize}

\subsection*{Objective 6: Extensions}
Possible extensions depending on time:
\begin{itemize}
    \item PMM-based solution of the time-dependent heat equation (via Black--Scholes transformation).
    \item Multi-asset options (two-dimensional PDE).
    \item American option approximation via penalty methods.
\end{itemize}

\section*{4. Work Plan and Milestones (One-Year Timeline)}

\subsection*{Semester 1 (Months 1--6)}

\textbf{Month 1--2: Literature Study}
\begin{itemize}
    \item Read the PMM paper and supporting literature on reduced-basis methods.
    \item Study the derivation and numerical properties of the Black--Scholes PDE.
    \item Begin preliminary numerical experiments.
\end{itemize}

\textbf{Deliverable:} A short report summarizing PMMs and the Black--Scholes PDE.

\bigskip

\textbf{Month 3--4: Numerical Solvers and Data Generation}
\begin{itemize}
    \item Implement a finite-difference solver for Black--Scholes.
    \item Generate datasets for a range of volatilities, interest rates, and maturities.
    \item Test grid-convergence and accuracy.
\end{itemize}

\textbf{Deliverable:} Verified PDE solver and dataset.

\bigskip

\textbf{Month 5--6: PMM Model Construction}
\begin{itemize}
    \item Construct primary and secondary matrices.
    \item Test small-scale PMMs (e.g.\ 5x5, 7x7) on simplified payoff functions.
    \item Begin training using analytic gradients.
\end{itemize}

\textbf{Deliverable:} First working PMM surrogate for one-parameter models.

\vspace{0.5cm}

\subsection*{Semester 2 (Months 7--12)}

\textbf{Month 7--9: Full PMM Training and Analysis}
\begin{itemize}
    \item Train PMMs across full parameter domain.
    \item Assess accuracy, convergence, and extrapolation.
    \item Optimize hyperparameters and matrix sizes.
\end{itemize}

\textbf{Deliverable:} Fully trained PMM for Black--Scholes pricing.

\bigskip

\textbf{Month 10--11: Benchmarking and Comparisons}
\begin{itemize}
    \item Compare PMM with neural networks and other regression models.
    \item Evaluate computational speed and parameter efficiency.
\end{itemize}

\textbf{Deliverable:} Benchmarking report with figures and tables.

\bigskip

\textbf{Month 12: Thesis Writing and Finalization}
\begin{itemize}
    \item Complete thesis manuscript.
    \item Prepare defense presentation.
\end{itemize}

\textbf{Deliverable:} Final thesis and presentation.

\section*{5. Expected Outcomes}

\begin{itemize}
    \item A full numerical implementation of PMMs applied to the Black--Scholes equation.
    \item Demonstration of PMM capabilities as surrogate PDE solvers.
    \item Quantitative comparisons with traditional machine learning and numerical PDE methods.
    \item A master thesis document summarizing:
    \begin{itemize}
        \item theoretical background,
        \item numerical implementations,
        \item parameter studies,
        \item performance analysis.
    \end{itemize}
\end{itemize}

\section*{6. References}

\begin{enumerate}
    \item P. Cook, D. Jammooa, M. Hjorth-Jensen, D. D. Lee, \textit{Parametric Matrix Models}, \textit{Nature Communications}, 16, 5929 (2025).
    \item F. Black and M. Scholes, ``The Pricing of Options and Corporate Liabilities'', \textit{Journal of Political Economy}, 81 (1973).
    \item J. C. Hull, \textit{Options, Futures, and Other Derivatives}, Pearson.
    \item Additional references to be added 
\end{enumerate}

\end{document}
