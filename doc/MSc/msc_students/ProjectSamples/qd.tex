\documentstyle[a4wide]{article}
\newcommand{\OP}[1]{{\bf\widehat{#1}}}

\newcommand{\be}{\begin{equation}}

\newcommand{\ee}{\end{equation}}

\begin{document}

\pagestyle{plain}

\section*{Thesis title: Quantum-mechanical systems in traps and pertinent many-body methods}

The aim of this thesis is to study the structure of quantum dots using
variational and Green's function Monte Carlo techniques, combining results
from Hartree-Fock calculations in order to achieve a as good as possible
variational wave function. The thesis will explore various Monte Carlo
optimalization strategies and use these to define the best possible density functional.
Semiconductor quantum dots are structures where
charge carriers are confined in all three spatial dimensions, 
the dot size being of the order of the Fermi wavelength 
in the host material, typically between  10 nm and  1 $\mu$m.
The confinement is usually achieved by electrical gating of a 
two-dimensional electron gas (2DEG), 
possibly combined with etching techniques. Precise control of the
number of electrons in the conduction band of a quantum dot 
(starting from zero) has been achieved in GaAs heterostructures. 
The electronic spectrum of typical quantum dots
can vary strongly when an external magnetic field is applied, 
since the magnetic length corresponding to typical 
laboratory fields  is comparable to typical dot sizes.
In coupled quantum dots Coulomb blockade effects, 
tunneling between neighboring dots, and magnetization 
have been observed as well as the formation of a
delocalized single-particle state. 




This thesis 
entails the development of variational Monte Carlo (VMC)
and Green's function Monte Carlo  (GFMC) 
programs to solve Schr\"odingers equation
and obtain various expectation values of interest, such as the energy
of the ground state and excited states. 
The GFMC approach allows, in principle, for a numerically
exact solution of Schr\"odingers equation. However, it needs a reasonable
starting point. It is there where a variational Monte Carlo calculation
of the same system provides a variationally optimal trial wave function
of a many-body system and its pertinent energy.
The Slater determinant for the variational wave function is set up using
single-particle wave functions from a Hartree-Fock calculation.
The methods are briefly described in the following section.
These calculations will in turn provide the basis for determining a as good as possible
ground state wave function. This wave function will in turn be used to define the quantum
mechanical density.  The density will be used to construct a density functional for quantum dots
using the adiabatic-connection method as described by Teale {\em et al} in J.~Chem.~Phys.~{\bf 130},
104111 (2009).  The results will be compared with existing density functional for quantum dots.


The reliability of the Monte Carlo  method as function of the externally applied magnetic field
will be compared with ab initio coupled-cluster 
and large-scale diagonalization techniques. The latter
two calculations will be performed by other Master of science students.

The results are expected to be published in leading journals.




A progress plan for this thesis project is given at the end.

\subsection*{Variational Monte Carlo}
The variational quantum Monte Carlo (VMC) has been widely applied 
to studies of quantal systems. 
The recipe consists in choosing 
a trial wave function
$\psi_T({\bf R})$ which we assume to be as realistic as possible. 
The variable ${\bf R}$ stands for the spatial coordinates, in total 
$2N$ if we have $N$ particles present. 
The trial wave function serves then as
a mean to define the quantal probability distribution 
\be
   P({\bf R})= \frac{\left|\psi_T({\bf R})\right|^2}{\int \left|\psi_T({\bf R})\right|^2d{\bf R}}.
\ee
The expectation value of the energy $E$
is given by
\be
   \langle E \rangle =
   \frac{\int d{\bf R}\Psi^{\ast}({\bf R})H({\bf R})\Psi({\bf R})}
        {\int d{\bf R}\Psi^{\ast}({\bf R})\Psi({\bf R})},
\ee
where $\Psi$ is the exact eigenfunction. Using our trial
wave function we define a new operator, 
the so-called  
local energy, 
\be
   E_L({\bf R})=\frac{1}{\psi_T({\bf R})}H\psi_T({\bf R}),
   \label{eq:locale1}
\ee
which, together with our trial PDF allows us to rewrite the 
expression for the energy as
\be
  \langle H \rangle =\int P({\bf R})E_L({\bf R}) d{\bf R}.
  \label{eq:vmc1}
\ee
This equation expresses the variational Monte Carlo approach.

The first part of this thesis deals thus with a VMC calculation of 
spherical quantum dots as functions of the strength of the applied magnetic field.

The trial wave function is a combination of a Slater determinant 
and a correlation part. The Slater determinant will be constructed
using single-particle
wave functions based on Hartree-Fock theory.

\subsection*{Green's function Monte Carlo (GFMC)}
The GFMC method is based on rewriting the 
Schr�dinger equation in imaginary time, by defining
$\tau=it$. The imaginary time Schr�dinger equation is then
\be
   \frac{\partial \psi}{\partial \tau}=-\OP{H}\psi,
\ee
where we have omitted the dependence on $\tau$ and the spatial variables
in $\psi$.

A Green's function Monte Carlo, although it allows for a formally exact
solution of Schr\"odinger's equations, needs a clever starting point 
for the energy. 
This trial energy is initially chosen to be the VMC energy of 
the trial  wave function, and is
updated as the simulation progresses. Use of an optimised 
trial function minimises the difference between the local 
and trial energies, and hence
minimises fluctuations in the calculations. 
A wave function optimised using VMC is ideal for this purpose, 
and in practice VMC provides the best
method for obtaining wave functions that accurately 
approximate ground state wave functions locally. 

The final aim of this thesis is thus to develop a GFMC program for studying
quantum dots, based on a VMC calculation first.

These calculations will in turn provide the basis for determining a as good as possible
ground state wave function. This wave function will in turn be used to define the quantum
mechanical density.  The density will be used to construct a density functional for quantum dots
using the adiabatic-connection method as described by Teale {\em et al} in J.~Chem.~Phys.~{\bf 130},
104111 (2009).  The results will be compared with existing density functionals for various quantum dots.


\section*{Progress plan and milestones}
The aims and progress plan of this thesis are as follows
\begin{itemize}
\item Fall 2009: Develop first a Hartree-Fock code for electrons trapped in a single harmonic oscillator  
in two dimensions.   This part entails developing a code for computing the Coulomb interaction
in two dimensions in the laboratory system.
\item The Hartree-Fock interaction is then used as input to the variational Monte Carlo code.
The results will be compared with large scale diagonalization and coupled-cluster
techniques for 2, 6, 12 and 20 
electrons in a single harmonic oscillator well.
\item Fall 2009: Write a code which solves the variational Monte-Carlo (VMC) problem
      for quantum dots. Both closed-core and open shell quantum dots will be studied. 
Construct  a Green's function Monte Carlo code
      based on the Variational Monte Carlo code. 
      The GFMC code receives as input the optimal 
      variational energy and wave function from the VMC calculation and solves
      in principle the Schr\"odinger equation exactly.
      The Slater determinant used in the VMC calculation includes also
      single-particle wave functions from Hartree-Fock calculations.
 \item Fall 2010: The obtained ground states will in turn be used to define a as exact as possible 
density functional for quantum dots
using the adiabatic-connection method. The density functional can in turn be used to model
systems with a large number of elctrons in quantum dots. Comparisons with a density functional derived from coupled-cluster methods will also be made in order to
test the validity of the Monte Carlo approach.
Possible applications are to mechanical studies of solar cells.
\end{itemize}
 


The thesis is expected to be handed in June 1 2010.

\begin{thebibliography}{999}

\bibitem {ref1}  A.~M.~Teale, S.~Coriani, and T.~Helgaker, J.~Chem.~Phys.~{\bf 130},
104111 (2009).


\end{thebibliography}



\end{document}



