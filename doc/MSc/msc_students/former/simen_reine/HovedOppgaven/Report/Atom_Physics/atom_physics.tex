\chapter{Atomic Physics}
\label{AtomicPhysics}


In this chapter the basic principles and difficulties of atomic
physics are outlined through investigation of the hydrogen and helium
atoms. Applications of quantum mechanics to the atomic problem result in
a partial integro-differential equation. This equation cannot be
solved analytically except for the special case of the hydrogen
atom. The solutions of the hydrogen atom provide useful insights
regarding the nature of the atoms, but difficulties arise when we add
one or more electrons. This is mainly because the strength of the
electron-electron interactions is comparable to the nucleus-electron
interaction.

\section{Basics}

\subsection{The Atomic Problem}
\label{TheAtomicProblem}

The Hamiltonian for an $N$-electron atomic system consists of two terms

\begin{equation}
  \hat{H}(\mathbf{x}) 
  = \hat{T}(\mathbf{x}) 
  + \hat{V}(\mathbf{x}); 
\label{hamiltonOperatorFull}
\end{equation}

the kinetic and the potential energy operator. Here $\mathbf{x} =
\left\{ \mathbf{x}_1, \mathbf{x}_2, \dots \mathbf{x}_N \right\}$  is
the spatial and spin degrees of freedom associated with the different
particles. The classical kinetic energy  

\begin{equation*}
  T= \frac{\mathbf{P^2}}{2m} + \sum_{j=1}^N \frac{\mathbf{p}_j^2}{2m}
\end{equation*}

is transformed to the quantum mechanical kinetic energy operator by 
operator substitution of the momentum ($p_k \to -i\hbar
\partial/\partial x_k$)

\begin{equation}
  \hat{T}(\mathbf{x}) = -\frac{\hbar^2}{2M}\nabla^2_0
  -\sum_{i=1}^{N}\frac{\hbar^2}{2m}\nabla^2_i.
\label{kineticEnergyOperatorFull}
\end{equation}

Here the first term is the kinetic energy operator of the nucleus,
the second term is the kinetic energy operator of the electrons,
$M$ is the mass of the nucleus and $m$ is the electron mass. The
potential energy operator is given by

\begin{equation}
  \hat{V}(\mathbf{x}) = 
  - \sum_{i=1}^{N} \frac{Ze^2}{(4\pi \epsilon_0)r_i}
  + \sum_{i=1,i<j}^{N} \frac{e^2}{(4\pi \epsilon_0)r_{ij}},
\label{potentialEnergyOperatorFull}
\end{equation}

where the $r_i$'s are the electron-nucleus distances and the
$r_{ij}$'s are the inter-electronic distances. 
\newline
%
\newline
We seek to find controlled and well understood approximations in order
to reduce the complexity of the above equations. The
\emph{Born-Oppenheimer approximation} is a commonly used
approximation, in which the motion of the nucleus is disregarded.


%*                The Born-Oppenheimer Approximation                *
\subsection{The Born-Oppenheimer Approximation}

In a system of interacting electrons and a nucleus there will usually
be little momentum transfer between the two types of particles due to
their differing masses. The forces between the particles are 
of similar magnitude due to their similar charge. If one assumes
that the momenta of the particles are also similar, the nucleus
must have a much smaller velocity than the electrons due to its far
greater mass. On the time-scale of nuclear motion, one can therefore
consider the electrons to relax to a ground-state given by the
Hamiltonian of eqs.~(\ref{hamiltonOperatorFull}),
(\ref{kineticEnergyOperatorFull}) and
(\ref{potentialEnergyOperatorFull}) with the nucleus at a fixed
location. This separation of the electronic and nuclear degrees of
freedom is known as the Born-Oppenheimer approximation. 
\newline
%
\newline
In the center of mass system the
kinetic energy operator reads (ref. \cite{bransden1983})

\begin{equation}
  \hat{T}(\mathbf{x}) = -\frac{\hbar^2}{2(M+Nm)}\nabla^2_{CM}
  -\frac{\hbar^2}{2\mu}\sum_{i=1}^{N}\nabla^2_i
  -\frac{\hbar^2}{M}\sum_{i>j}^{N}\nabla_i\cdot\nabla_j,
  \label{centerOfMassKineticEnergyOperator}
\end{equation}

while the potential energy operator remains unchanged. Note that the
Laplace operators $\nabla^2_i$ now are in the center of mass reference
system.
\newline
%
\newline
The first term of eq.~(\ref{centerOfMassKineticEnergyOperator})
represents the kinetic energy operator of the center of mass. The
second term represents the sum of the kinetic energy operators of the
$N$ electrons, each of them having their mass $m$ replaced by the
reduced mass $\mu = mM/(m+M)$ because of the motion of the
nucleus. The nuclear motion is also responsible for the third term,
or the \emph{mass polarization} term.
\newline
%
\newline
The nucleus consists of protons
and neutrons. The proton-electron mass ratio is about
$1 / 1836$ and the neutron-electron mass ratio is about
$1 / 1839$, so regarding the nucleus as stationary is a natural
approximation. Taking the limit $M\to \infty$ in
eq.~(\ref{centerOfMassKineticEnergyOperator}), the kinetic energy 
operator reduces to

\begin{equation}
  \hat{T} = -\sum_{i=1}^{N}\frac{\hbar^2}{2m}\nabla^2_i
\end{equation}

The Born-Oppenheimer approximation thus disregards both the kinetic
energy of the center of mass as well as the mass polarization term.
The effects of the Born-Oppenheimer approximation are quite small and
they are also well accounted for.
However, this simplified electronic Hamiltonian remains very difficult
to solve, and analytical solutions do not exist for general systems
with more than one electron. The Born-Oppenheimer approximation will
be used for the rest of this thesis.  
\newline
%
\newline
The first term of eq.~(\ref{potentialEnergyOperatorFull}) is the
nucleus-electron potential and the second term is the
electron-electron potential. The inter-electronic potentials are the
main problem in atomic physics. Because of these terms, the
Hamiltonian cannot be separated into one-particle parts, and the
problem must be solved as a whole. A common approximation is to regard
the effects of the electron-electron interactions either as averaged
over the domain or by means of introducing a density functional, such as
by Hartree-Fock (HF) or Density Functional Theory (DFT). These
approaches are actually very efficient, and about $99\%$ or more of
the electronic energies are obtained for most HF calculations.
Other observables are usually obtained to an accuracy of about
$90-95\%$ (ref. \cite{helgaker2002}).  The main effort of the
advanced numerical procedures is to reduce the errors these
approximations induce. These issues will be discussed in detail in
chapter \ref{NumericalApproaches}. But first we simplify the atomic
problem further by using atomic units. 

%*                          Atomic Units                           *
\subsection{Atomic Units}

Numerical methods require proper scaling of the system in
question. In atomic systems we scale to atomic units by setting
$m=e=\hbar=4\pi\epsilon_0=1$, see table \ref{atomicUnits}. 

\begin{table}[hbtp]
\begin{center} {\large \bf Atomic Units} \\ 
$\phantom{a}$ \\
\begin{tabular}{llc}
\hline\\ 
{\bf Quantity}                 & {\bf SI}               & {\bf Atomic unit}\\
Electron mass, $m$               & $9.109\cdot 10^{-31}$ kg & 1 \\
Charge, $e$                      & $1.602\cdot 10^{-19}$ C  & 1 \\
Planck's reduced constant, $\hbar$& $1.055\cdot 10^{-34}$ Js& 1 \\       
Permittivity, $4\pi\epsilon_0$   & $1.113\cdot 10^{-10}$ C$^2$ J$^{-1}$ m$^{-1}$&1\\
Energy, $\frac{e^2}{4\pi\epsilon_0 a_0}$ & $27.211$ eV       & 1 \\
Length, $a_0=\frac{4\pi\epsilon_0 \hbar^2}{me^2}$&$0.529\cdot10^{-10}$ m&1\\ [10pt]      
\hline
\end{tabular} 
\end{center}
\caption{Scaling from SI to atomic units}
\label{atomicUnits}
\end{table}

In this way the atomic problem is simplified to

\begin{equation}
  \left[-\sum_{i=1}^N \frac{1}{2} \nabla^2_i 
    - \sum_{i=1}^N \frac{Z}{r_i} + \sum_{i<j}^N \frac{1}{r_{ij}} 
    \right] \Psi(\mathbf{x}) = E \Psi(\mathbf{x}).
  \label{SchrodingerBornOppenheimerAtomicUnits}
\end{equation}

This is the equation we want to solve in atomic physics. The
introduction of atomic units serves two purposes. In addition to
making the equation easier to work with because of the neglected
units, the atomic units also have an important numerical feature. When
solving numerical problems the different quantities involved must have
proper scaling; they must be in the same order of magnitude. Failing
to do so may result in loss of numerical precision.


%*******************************************************************
%*                      The Hydrogen Atom                          *
%*******************************************************************
\section{The Hydrogen Atom}
\label{TheHydrogenAtom}

The solutions of the hydrogen atom form the basis of our
understanding of the many-electron atom. The Hamiltonian of the
hydrogen atom reads 

\begin{equation}
  \hat{H} = -\frac{1}{2} \nabla^2  + V(r),
  \label{HydrogenHamiltonian}
\end{equation}

where $V(r) = - Z/r$. The nucleus charge $Z$ equals unity for the
hydrogen atom, but since the solutions to hydrogen-like atoms are
important, we keep it throughout our calculations. In polar coordinates
the Laplacian is given by (ref. \cite{rottmann2003})

\begin{equation}
  \nabla^2 = \frac{1}{r^2}\left[
    \frac{\partial}{\partial r} \left( r^2 \frac{\partial}{\partial r}\right)+
    \frac{1}{sin^2 \theta} \frac{\partial^2}{\partial \phi}+
    \frac{1}{sin \theta}\frac{\partial}{\partial \theta} \left( sin \theta
    \frac{\partial}{\partial \theta} \right)
    \right].
\label{laplacianRottmann}
\end{equation}

Introduction of this term, combined with the fact that the potential is
spherically symmetric, allows us to separate the radial part and the
angular part of the equation; $\Psi(r,\theta, \phi) =
R(r) \cdot Y(\theta,\phi)$. The solutions of the angular equation are
known as the spherical harmonics $Y_{lm_l}(\theta, \phi)$. The first
few spherical harmonics are listed in table \ref{sphericalHaromical}.
\newline

\begin{table}[hbtp]
\begin{center} {\large \bf Spherical Harmonics} \\ 
$\phantom{a}$ \\
\begin{tabular}{ccccc}
\hline\\ 
$m_l\backslash l$ & \phantom{AA}0\phantom{AA}
& \phantom{AA}1\phantom{AA} & \phantom{AA}2\phantom{AA} &
\phantom{AA}3\phantom{AA} \\ 
\hline\\ 
+3 &                      &
&
&$-\frac{1}{8}(\frac{35}{\pi})^{1/2}sin^3\theta e^{+ 3i\phi}$
\\ [7pt] 

+2 &                      &
&$\frac{1}{4}(\frac{15}{2\pi})^{1/2}sin^2\theta e^{+ 2i\phi}$
&$\frac{1}{4}(\frac{105}{2\pi})^{1/2}cos\theta sin^2\theta e^{+ 2i\phi}$     \\ [7pt]

+1 &  
&$-\frac{1}{2}(\frac{3}{2\pi})^{1/2}sin\theta
e^{+i\phi}$&$-\frac{1}{2}(\frac{15}{2\pi})^{1/2}cos\theta sin\theta
e^{+ i\phi}$&$-\frac{1}{8}(\frac{21}{2\pi})^{1/2}(5cos^2\theta
-1)sin\theta e^{+ i\phi}$\\ [7pt] 

 0 &$\frac{1}{2\pi^{1/2}}$&$\frac{1}{2}(\frac{3}{\pi})^{1/2}cos\theta$
 &$\frac{1}{4}(\frac{5}{\pi})^{1/2}(3cos^2\theta-1)$
 &$\frac{1}{4}(\frac{7}{\pi})^{1/2}(2-5sin^2\theta)cos\theta$
 \\ [7pt] 

-1 &  
 &$+\frac{1}{2}(\frac{3}{2\pi})^{1/2}sin\theta
 e^{-i\phi}$&$+\frac{1}{2}(\frac{15}{2\pi})^{1/2}cos\theta sin\theta
 e^{- i\phi}$&$+\frac{1}{8}(\frac{21}{2\pi})^{1/2}(5cos^2\theta
 -1)sin\theta e^{- i\phi}$\\ [7pt] 

-2 &                      &
 &$\frac{1}{4}(\frac{15}{2\pi})^{1/2}sin^2\theta e^{- 2i\phi}$
 &$\frac{1}{4}(\frac{105}{2\pi})^{1/2}cos\theta sin^2\theta e^{- 2i\phi}$     \\ [7pt]

-3 &                      &
&
&$+\frac{1}{8}(\frac{35}{\pi})^{1/2}sin^3\theta e^{- 3i\phi}$
\\ [7pt] 
\hline
\end{tabular} 
\end{center}
\caption{Spherical harmonics $Y_{lm_l}$ for the lowest $l$ and $m_l$
  values (taken from ref. \cite{atkins2003}).} 
\label{sphericalHaromical}
\end{table}

The spherical harmonics introduce two quantum numbers 
$l = 0,1,2,\dots$ and $m_l = -l, -(l-1), \dots, (l-1), l$. 
These quantum numbers are called the 
\emph{orbital angular momentum} and the \emph{magnetic quantum
number}, respectively. It is worth noticing that the spherical
harmonics are independent of the shape of a spherical symmetric
potential $V(r)$. The radial and the angular part are interconnected
through the separation constants introduced when separating the
equations. For the radial wave-function 

\begin{equation}
  \left[ -\frac{1}{2} \frac{1}{r^2}\frac{\partial}{\partial r} 
    \left( r^2 \frac{\partial}{\partial r}\right)  
    + V(r) + \frac{l(l+1)}{r^2} \right]
  R(r) = E R(r).
\label{hydrogenRadialEquation}
\end{equation}

we therefore have a term including the angular momentum $l$.
The first few non-normalized radial solutions of equation
(\ref{hydrogenRadialEquation}) are listed in table
\ref{hydrogenRadialFunctions}.
\newline


\begin{table}[hbtp]
\begin{center} {\large \bf Hydrogen-Like Atomic Radial Functions} \\ 
$\phantom{a}$ \\
\begin{tabular}{cccc}
\hline\\ 
$l\backslash n$ & \phantom{AA}1\phantom{AA}
& \phantom{AA}2\phantom{AA} & \phantom{AA}3\phantom{AA}  \\ 
\hline\\ 
0 & $e^{-Zr}$ & $(2-r)e^{-Zr/2}$ & $(27-18r+2r^2)e^{-Zr/3}$ \\[7pt]
1 & & $re^{-Zr/2}$ & $r(6-r)e^{-Zr/3}$\\[7pt]
2 & & & $r^2e^{-Zr/3}$ \\[7pt]
\hline
\end{tabular} 
\end{center}
\caption{The first few radial functions of the hydrogen-like atoms
  (taken from ref. \cite{rohlf1994}).} 
\label{hydrogenRadialFunctions}
\end{table}

The states

\begin{equation}
  \Psi_{nlm_l}(r,\theta, \phi) =R_{nl}(r) \cdot Y_{lm_l}(\theta,\phi),
\label{totalHydrogenWavefunction}
\end{equation}

now have three quantum numbers $n$, $l$, $m_l$, where the
\emph{principal quantum number} can take any positive integer
$n=1,2,\dots$. By the introduction of 
quantum numbers the electron may only occupy distinct states or
\emph{eigenstates}, and the corresponding \emph{eigenenergy} is thus
quantized. The eigenenergies

\begin{equation*}
  E_{n} = -\frac{1}{2n^2},
\end{equation*}

are independent of the orbital angular momentum and the magnetic
quantum number. The eigenstates are therefore \emph{degenerate};
several distinct states share the same energy $E_n$. For a given $n$
we have a degeneracy both with respect to varying values of $l$ and of
$m_l$. Furthermore, we have yet another degeneracy associated with the
two possible values of the electronic spin $m_s$ ($=\pm 1/2$).
\newline
%
\newline
The degeneracy due to the magnetic and spin quantum numbers becomes
clear when applying a magnetic field. The magnetic field removes the
degeneracy by splitting the energy levels. 
The degeneracy of different orbital angular momenta of the hydrogen
atom is removed in multi-electronic systems. 
Higher angular momentum lead to higher energy. 
This result 
is manifested by the way the atoms are classified in the periodic
table. States with different orbital angular momenta are for
historical reasons assigned the letters $s$, $p$, $d$, $f$, etc. where
$s$ corresponds to $l=0$, $p$ to $l=1$ and so forth. For example
$2p$ is assigned to a state with $n=2$ and $l=1$. The orbitals are
energetically arranged in the order $1s$, $2s$, $2p$, $3s$, $3p$,
$4s$, $3d$, $4p$, $5s$, $4d$, $5p$, $6s$, $4f$, $5d$, $6p$, etc.,
which explains the general trends of the periodic table.
\newline
%
\newline
A problem with the spherical harmonics of table
\ref{sphericalHaromical} is that they are complex. The introduction of
\emph{solid harmonics}, see ref. \cite{helgaker2002}, allows the use
of real orbital wave-functions for a wide range of applications. The
complex solid harmonics ${\cal Y}_{lm_l}(\mathbf{r})$ are related to
the spherical harmonics  $Y_{lm_L}(\mathbf{r})$ through

\begin{equation*}
  {\cal Y}_{lm_l}(\mathbf{r}) = r^l Y_{lm_l}(\mathbf{r}).
\end{equation*}

By factoring out the leading $r$-dependency of the radial-function
(see for example \cite{shankar1994} \newline \newline)

\begin{equation*}
  {\cal R}_{nl}(\mathbf{r}) = r^{-l} R_{nl}(\mathbf{r}),
\end{equation*}

we obtain a relationship similar to that of
eq.~(\ref{totalHydrogenWavefunction}), namely

\begin{equation*}
  \Psi_{nlm_l}(r,\theta, \phi) %=R_{nl}(r) \cdot Y_{lm_l}(\theta,\phi)
  = {\cal R}_{nl}(\mathbf{r})\cdot{\cal Y}_{lm_l}(\mathbf{r}).
%\label{totalSolidHydrogenWavefunction}
\end{equation*}

For the theoretical development of the \emph{real solid harmonics} see
ref. \cite{helgaker2002}. Here Helgaker \emph{et al} first 
express the complex solid harmonics, $C_{lm_l}$, by (complex) Cartesian
coordinates, and arrive at the real solid harmonics, $S_{lm_l}$, through
the unitary transformation

\begin{equation*}
  \left( \begin{split} &\phantom{i} S_{lm_l} \\ 
    &S_{l,-m_l} \end{split} \right) 
  = \frac{1}{\sqrt{2}} \left(        \begin{split}
    (-1)^m_l \phantom{a} & \phantom{aa} 1 \\ 
    -(-1)^m_l i & \phantom{aa} i       \end{split} \right)  
  \left( \begin{split} &\phantom{i} C_{lm_l} \\ 
    &C_{l,-m_l} \end{split} \right).
\end{equation*}

This transformation will not alter any physical quantities that are
degenerate in the subspace consisting of opposite magnetic quantum
numbers (the angular momentum $l$ is equal for both these cases). This
means for example that the above transformation does not alter the
energies, unless an external magnetic field is applied to the
system. Henceforth, we will use the solid harmonics, and note that
changing the spherical potential beyond the Coulomb potential will not
alter the solid harmonics. The lowest-order real solid harmonics are
listed in table \ref{solidHarmonics}.

\begin{table}[hbtp]
\begin{center} {\large \bf Real Solid Harmonics} \\ 
$\phantom{a}$ \\
\begin{tabular}{ccccc}
\hline\\ 
$m_l\backslash l$ & \phantom{AA}0\phantom{AA}
& \phantom{AA}1\phantom{AA} & \phantom{AA}2\phantom{AA} &
\phantom{AA}3\phantom{AA} \\ 
\hline\\ 
+3& & &
&$\frac{1}{2}\sqrt{\frac{5}{2}}(x^2-3y^2)x$ \\ [7pt] 
+2& & &$\frac{1}{2}\sqrt{3}(x^2-y^2)$&$\frac{1}{2}\sqrt{15}(x^2-y^2)z$
\\ [7pt] 
+1& &x&$\sqrt{3}xz$
&$\frac{1}{2}\sqrt{\frac{3}{2}}(5z^2-r^2)x$ \\ [7pt] 
0&1&y&$\frac{1}{2}(3z^2-r^2)$       &$\frac{1}{2}(5z^2-3r^2)x$ \\
 [7pt] 
-1& &z&$\sqrt{3}yz$
&$\frac{1}{2}\sqrt{\frac{3}{2}}(5z^2-r^2)y$ \\ [7pt] 
-2& & &$\sqrt{3}xy$                  &$\sqrt{15}xyz$ \\ [7pt] 
-3& & &
&$\frac{1}{2}\sqrt{\frac{5}{2}}(3x^2-y^2)y$ \\ [7pt] 
\hline
\end{tabular} 
\end{center}
\caption{The first-order real solid harmonics ${\cal Y}_{lm_l}$ (taken from
  ref. \cite{atkins2003}).} 
\label{solidHarmonics}
\end{table}






%*****************************************************************
%*                      The Helium Atom                          *
%*****************************************************************
\section{The Helium Atom}
\label{TheHeliumAtom}

The helium atom cannot be solved analytically. The numerical
solutions, however, are in excellent agreement with
experiments, see for example ref. \cite{coldwell1997}. We will not
go into the details of such accurate approaches, but rather illustrate
how to generate an approximate wave-function through application of 
perturbative and variational methods.
\newline
%
\newline
The Hamiltonian of the helium atom is

\begin{equation}
  \hat{H} = -\frac{1}{2} \nabla_1^2 - \frac{1}{2} \nabla_2^2  -
  \frac{2}{r_1} - \frac{2}{r_2} + \frac{1}{r_{12}}.
  \label{HeliumHamiltonian}
\end{equation}


%*                   The Perturbative Approach                   *
\subsection{The Perturbative Approach}
In the perturbative approach controlled approximations are made so
that the initial problem is transformed to an \emph{unperturbed} problem
where the solutions are easy to obtain. The controlled approximations
are treated as \emph{perturbations} of the unperturbed problem and
added into the system by including higher and higher corrections of
the perturbation. A requirement of the perturbative approach is that
the perturbations are small compared to the unperturbed values.
Straightforward perturbation of the helium atom is acquired if we
first disregard the electron-electron repulsion, and then add it as a
perturbative correction. 
\newline
%
\newline
Without the inter-electronic 
repulsion we get a separable Hamiltonian

\begin{equation*}
  \hat{H} = -\frac{1}{2} \nabla_1^2 - \frac{1}{2} \nabla_2^2  -
  \frac{2}{r_1} - \frac{2}{r_2} = \hat{h}_1 + \hat{h}_2,
\end{equation*}

and we may solve the two one-particle equations independently. The
unperturbed wave-function becomes the product of two hydrogen atom
solutions (given by equation (\ref{totalHydrogenWavefunction}))


\begin{equation*}
  \Psi_{n_1 l_1 m_{l,1} n_2 l_2 m_{l,2}}^{(0)}
  (r_1, \theta_1, \phi_1, r_2, \theta_2, \phi_2)=
  \Psi_{n_1 l_1 m_{l,1}}(r_1, \theta_1, \phi_1)
  \Psi_{n_2 l_2 m_{l,2}}(r_2, \theta_2, \phi_2)
\end{equation*}

with the nucleus charge $Z=1$ replaced by $Z=2$. This gives the
energy

\begin{equation}
  E_{n_1n_2}^{(0)} = -2\left(\frac{1}{n_1^2}+\frac{1}{n_2^2}\right),
\label{unperturbedHeliumEnergy}
\end{equation}

where the superscript ${(0)}$ indicates the unperturbed energy. For
the ground state we define the product   

\begin{equation*}
  \Psi_{1,0,0}(r_1, \theta_1, \phi_1) 
  \Psi_{1,0,0}(r_2, \theta_2, \phi_2) 
  \equiv \Psi_{1s}(1) \Psi_{1s}(2),
\end{equation*}

The first order correction to the energy is then

\begin{equation}
  J \equiv E_{n_1n_2}^{(1)} - E_{n_1n_2}^{(0)} 
%  = \int \Psi_{1s}(1)^* \Psi_{1s}(2)^* \frac{1}{r_{12}}
%  \Psi_{1s}(1) \Psi_{1s}(2) d\tau_1 d\tau_2 
  = \int |\Psi_{1s}(1)|^2 \frac{1}{r_{12}}
  |\Psi_{1s}(2)|^2 d\tau_1 d\tau_2 ,
\label{CoulombIntegral}
\end{equation}

which is called the \emph{Coulomb integral} and often denoted by
$J$. The Coulomb integral is commonly encountered in the approximative 
methods of many-body quantum mechanics, and has an easy
interpretation. The term $|\Psi_{1s}(1)|^2 d\tau_1$ is the probability
of finding electron $1$ in the volume element $d\tau_1$, and when
multiplied with the charge $-1$ (in atomic units) it represents the
\emph{charge density} of that region. Similarly $-|\Psi_{1s}(2)|^2
d\tau_2$ is the charge density of electron $2$ in the volume element
$d\tau_2$. The Coulomb integral of eq.~(\ref{CoulombIntegral}) may
therefore be interpreted as the averaged contribution of the Coulomb
repulsion between the two electrons. The value of the Coulomb integral
is $J = 1.25$ according to ref. \cite{atkins2003}. This gives a first
order approximation to the ground state energy

\begin{equation*}
  E_0^{(1)} = - 2 - 2 + 1.25 = - 2.75.
\end{equation*}

This result is not in perfect agreement with the experimental value
$E_0=-2.9037$. However, it is a clear indication that we are on the
right track. One of the reasons for the disagreement is that the
perturbation is not at all small, so first-order perturbation theory
cannot be expected to lead to a reliable result.


%*                   The Variational Approach                   *
\subsection{The Variational Approach}

A different way of solving the helium atom is the use of a
\emph{variational} approach. Here we start out by guessing a
parametrized form of a \emph{trial wave-function}
$\Psi_{\mathbf{\alpha}}$, where  
$\mathbf{\alpha} = (\alpha_1, \alpha_2,\dots,\alpha_M)$  
denotes the set of variation parameters. Then we optimize these
parameters in accordance with the 
\emph{variational principle}; the energy expectation value of a
variational wave-function provides an upper bound to the true ground
state energy

\begin{equation*} 
  \frac{\int \Psi_{\mathbf{\alpha}}^* \hat{H}
  \Psi_{\mathbf{\alpha}} d\tau}{\int \vert
  \Psi_{\mathbf{\alpha}}\vert^2 d\tau} \ge E_0.
\end{equation*}

The variational principle of quantum mechanics may be derived by
expanding a normalized trial wave-function, $\Psi_{\mathbf{\alpha}}$,
in terms of the exact orthonormal eigenstates $\left\{ \psi_i
\right\}$ of the Hamiltonian

\begin{equation*} 
  \psi_{\mathbf{\alpha}}=\sum_{i=0}^{\infty} c_{i} \psi_i,
\end{equation*}

where the expansion coefficients $c_{i}$ are normalized

\begin{equation*} 
  \sum_{i=0}^{\infty} \vert c_{i}\vert^2=1. 
\end{equation*}

The expectation of the many-body Hamiltonian $\hat{H}$ is
then evaluated as

\begin{equation*} 
  \langle E_{\mathbf{\alpha}} \rangle =  
 \sum_{i=0}^{\infty} \vert c_{i}\vert^{2} \epsilon_{i},
\end{equation*}

where $\epsilon_{i}$ and $\psi_i $ fulfills the stationary
Schr\"odinger equation

\begin{equation*} 
  \hat{H} \psi_i = \epsilon_{i} \psi_i.
\end{equation*}

The expectation value of the trial energy
must therefore be greater than or equal to the true ground state
energy

\begin{equation*} 
  \langle E_{\mathbf{\alpha}} \rangle \ge \langle E_0 \rangle = \epsilon_0,
\end{equation*}

as $\epsilon_{i} \ge \epsilon_{0}$. The variational energy
computed using $\psi_{\mathbf{\alpha}}$ thus provides an upper bound
for the true ground state energy. Therefore, our strategy is to 
search for the variational parameters that give us the lowest
variational energy. 
\newline
%
\newline
The difficulties in the variational method are to find a good
variational wave-function, to evaluate the energy expectation value and
to find the energy minimum in parameter space. There are several ways
to generate the trial wave-function, and we will return to some of
these in chapter \ref{NumericalApproaches}. One simple approach is to
start with a product of two variational hydrogen $1s$ solutions 

\begin{equation*} 
  \psi_{\alpha} 
  = e^{\alpha r_1}e^{\alpha r_2} = e^{\alpha (r_1+r_2)}.
\end{equation*}

We then need to minimize the expectation value

\begin{equation*} 
  \langle E_{\alpha} \rangle = 
  \frac{\int e^{\alpha (r_1+r_2)} \hat{H} e^{\alpha (r_1+r_2)}
  d\tau_1d\tau_2}
       {\int e^{2\alpha (r_1+r_2)} d\tau_1d\tau_2},
\end{equation*}

with respect to the parameter $\alpha$.\footnote{Notice that by
  setting $\alpha=2$ we reproduce the perturbative result.}
This minimization is not trivial. The integration domain is the
six-dimensional configuration space. We start by a transformation to
polar coordinates. For the ground state there are no angular
dependencies $\partial\psi_{\alpha}/\partial\phi
=\partial\psi_{\alpha}/\partial\theta = 0$, so these terms may be
removed altogether from the Hamiltonian. The angular terms of the
numerator are therefore cancelled by the equal angular terms of the
determinator. We are left with the two-dimensional integral  

\begin{equation} 
  \langle E_{\alpha} \rangle = 
  \frac{\int\limits_0^{\infty}\int\limits_0^{\infty} e^{\alpha
      (r_1+r_2)} \hat{H}_r  e^{\alpha (r_1+r_2)} r_1^2 r_2^2 dr_1 dr_2}
       {\int\limits_0^{\infty}\int\limits_0^{\infty} e^{2\alpha
	   (r_1+r_2)} r_1^2 r_2^2 dr_1 dr_2},
\label{variationalHeliumApproach}
\end{equation}

where the radial Hamiltonian $\hat{H}_r$ is obtained from
eqs.~(\ref{laplacianRottmann}) and (\ref{HeliumHamiltonian})

\begin{equation*}
  \hat{H}_r = -\frac{1}{2r_1^2} \frac{\partial}{\partial r_1} \left( r_1^2
  \frac{\partial}{\partial r_1}\right) - \frac{1}{2r_2^2}
  \frac{\partial}{\partial r_2} \left( r_2^2 \frac{\partial}{\partial
  r_2}\right) - \frac{2}{r_1} - \frac{2}{r_2} + \frac{1}{r_{12}}.
\end{equation*}



Eq.~(\ref{variationalHeliumApproach}) can be reduced to the following,
see \ref{}, \newline \newline !!! Referanse her !!! \newline \newline

\begin{equation}
  \langle E_{\alpha} \rangle = \alpha^2 - \frac{27}{8}\alpha,
\end{equation}

which has a minima for $\alpha = 27/16 = 1.6875$, namely 
$\langle E_{1.6875} \rangle = -2.84766$. 
\newline
%
\newline
Compared to the perturbative approach we have gained some, but not
all of the correlation. However, adding another term to the trial
wave-function

\begin{equation*} 
  \psi_{\alpha,\beta} 
  = e^{-\alpha (r_1+r_2)} exp \left\{ \frac{r_{12}} {2(1+\beta
  r_{12})} \right\}
\end{equation*}

and optimizing (see table \ref{energyMinimaCheck}) we arrive at
$E_{\alpha,\beta} = -2.8901 \pm 0.0003$. As 
can be seen by comparing these results with the experimental value
$-2.9037$, the addition of the simple electron-electron exponent is able
to regain approximately $78\%$ of the correlation compared to the
first hydrogenic trial wave-function. 
\newline
%
\newline
Variational calculations depend crucially on the form of the trial
wave-function used. By selecting trial wave-functions on physically
motivated grounds, accurate wave-functions may be obtained. Commonly,
wave-functions obtained from a Hartree-Fock or similar calculations are
used. Then additional parameters are added, building in additional
physics such as known limits and derivatives of the many-body
wave-function. The additional variational freedom is then exploited to
further optimize the wave-function.

%**************************************************************
%*                 Beyond the Helium Atom                     *
%**************************************************************
\section{Beyond the Helium Atom}

Before starting the description of the most common methods used to
solve the many-body problem we need to adress some elementary theory
regarding this problem. First, we must
establish some rules regarding the construction of physically reliable
wave-functions for systems with more than one electron. 
\newline
%
\newline
The \emph{Pauli principle} was recognized by Wolfgang Pauli
(ref. \cite{atkins2003}):
\newline

{\bf \large The Pauli Principle}
\emph{
The total wave-function
must be antisymmetric under the interchange 
of any pair of identical fermions and symmetric under the
interchange of any pair of identical bosons.
\newline
}

A result of the Pauli principle is the so-called \emph{Pauli exclusion
  principle}:
\newline

{\bf \large The Pauli Exclusion Principle}
\emph{
  No two electrons can occupy the same state.
\newline
}

Overall wave-functions that satisfy the Pauli principle are often
written as \emph{Slater Determinants}.

\subsection{The Slater Determinant}

Again we turn our attention to the helium atom. It was assumed that
the two electrons were both in the $1s$ state. This fulfills the Pauli
exclusion principle as the two electrons in the ground state have
different intrinsic spin. However, the wave-functions we used in both
the perturbative and variational approach were not antisymmetric with
respect to interchange of the different electrons. This is not totally 
true as we only included the spatial part of the wave-function.
For the helium ground state the spatial part of the wave-function is
symmetric and the spin part is anti-symmetric. The product is
therefore anti-symmetric as well. The Slater-determinant consists of
single-particle \emph{spin-orbital}s; joint spin-space states of the
electrons

\begin{equation*} 
  \Psi_{1s}^{\uparrow}(1) = \Psi_{1s}(1)\uparrow(1),
\end{equation*}
 
and similarly

\begin{equation*} 
  \Psi_{1s}^{\downarrow}(2) = \Psi_{1s}(2)\downarrow(2).
\end{equation*}

Here the two spin functions are given by 

\begin{equation*}
  \uparrow(I) = \left\{ 
  \begin{array}{cl}
    1 & \text{ if $m_s(I)=\frac{1}{2}$} \\ [4pt]
    0 & \text{ if $m_s(I)=-\frac{1}{2}$}
  \end{array}
  \right.,
\end{equation*}

and

\begin{equation}
  \downarrow(I) = \left\{ 
  \begin{array}{cl}
    0 & \text{ if $m_s(I)=\frac{1}{2}$} \\ [4pt]
    1 & \text{,if $m_s(I)=-\frac{1}{2}$}
  \end{array}
  \right.,
\label{heliumSlaterDeterminant}
\end{equation}

with $I=1,2$.

The ground state can then be expressed by the following determinant

\begin{equation*} 
  \Psi(1,2) = \frac{1}{\sqrt(2)} \left|
  \begin{array}{cc}
    \Psi_{1s}(1)\uparrow(1) & \Psi_{1s}(2)\uparrow(2) \\ [4pt]
    \Psi_{1s}(1)\downarrow(1)  & \Psi_{1s}(2)\downarrow(2)
  \end{array}
  \right|.
\end{equation*}

This is an example of a \emph{Slater determinant}. This determinant is
antisymmetric since particle interchange is identical to an
interchange of the two columns. For the ground state the spatial
wave-function is symmetric. Therefore we simply get 

\begin{equation*} 
  \Psi(1,2) = \Psi_{1s}(1) \Psi_{1s}(2) \left[ 
    \uparrow(1)\downarrow(2) - \uparrow(2) \downarrow(1) \right].
\end{equation*}

The spin part of the wave-function is here anti-symmetric. This has no
effect when calculating physical observables because the sign of the
wave-function is squared in all expectation values.
\newline
%
\newline
The general form of a Slater determinant composed of $n$
orthonormal orbitals $\left\{ \phi_i \right\}$ is

\begin{equation}
  \Psi = \frac{1}{\sqrt{N!}}\left| 
  \begin{array}{cccc}
    \phi_1(1) & \phi_1(2) & \dots  &\phi_1(N) \\ [4pt]
    \phi_2(1) & \phi_2(2) & \dots  &\phi_2(N) \\ [4pt] 
    \vdots    & \vdots    & \ddots &\vdots    \\ [4pt]
    \phi_N(1) & \phi_N(2) & \dots  &\phi_N(N)
  \end{array}
  \right|.
\label{SlaterDeterminantDefinition}
\end{equation}

The introduction of the Slater determinant is very important for
treatment of many-body systems, and in this thesis it is the 
principal building block of each variational wave-function used.
As long as we express the wave-function in terms of either one Slater
determinant or a linear combination of several Slater determinants,
the Pauli principle is satisfied. When constructing
many-electron wave-functions this picture provides an easy way to
include many of the physical features. One problem with the Slater
matrix is that it is computationally demanding. Limiting the number of
calculations will be one of the most important issues concerning the
implementation of the Slater determinant. This will be discussed in
detail in chapter \ref{Implementation}.

