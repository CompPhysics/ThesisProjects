\chapter{Conclusion}

In this thesis we have worked out the basic principles regarding the
use of Monte Carlo integration to solve many-body quantum mechanical
systems. A fast and reliable code has been developed, and principal
advances for future development have been identified. In this section
we summarize the insights gained through this work.
\newline
%
\newline
Quantum Monte Carlo methods provide powerful tools for solving the
many-body Schr\"o\-dinger equation. As a part of developing the QMC
machinery, Variation Monte Carlo (VMC) is a natural starting point. 
Here the basics of the VMC method have been developed, and the method 
has been applied to the atomic problem. The results given in the
previous chapter indicate that the main building blocks of the VMC
method are very efficient. Furthermore, the principal limitations
have been identified, and suggestions for future work have been
outlined. 
\newline
%
\newline
Several key factors regarding the program code need further
development. The main factor is related to auto-correlation
effects. The Metropolis 
algorithm needs to be sophisticated so that the walkers span the
phase space more efficiently. We further identified the
need for different step lengths at different length scales; the
electrons close to the nucleus should be moved shorter than the
electrons further away. A suggestion was made regarding possible
labeling of the individual electrons, as the different electron
permutations of the probability distribution simply resulted in
symmetrical subspaces.
\newline
%
\newline
Another key factor for future development is to allow greater
flexibility in the variational form of the trial wave-function; to
allow electron-nucleus and electron-electron-nucleus correlations and
also to allow linear combinations of Slater determinants.
\newline
%
\newline
An additional and natural extension is to implement the Diffusion
Monte Carlo (DMC) method. This method produces excellent results for
the electronic energies, and its foundations are laid when developing
the VMC method. Here an ensemble of walkers is moved, and making many
walkers is a simple matter. The diffusion, drift and branching terms
should also be simple to implement.
\newline
%
\newline
The energy and variance optimization schemes have been
studied, but we have omitted to mention effective procedures for
optimization of \emph{many} parameters simultaneously. High-dimensional
optimization is a common procedure in many areas, and can be
incorporated into the variance optimization scheme. The basic building
blocks of the variance optimization scheme has been developed and can
be used even for many-dimensional parameter optimization. However,
such procedures requires high accuracy in the different variations.
This emphasizes the importance of reducing auto-correlation effects
and obtaining better trial wave-functions.  Another easy and natural
extension to the current program is to parallelize the code also for
variance optimization.
\newline
%
\newline
The above extensions would result in a very efficient tool for
many-body quantum mechanics, that can be applied to a wide variety of
problems. For example for materials and molecules the Slater
determinant can be reduced to a block diagonal matrix by identifying
non-overlapping parts. Similarly, the terms in the Jastrow
factor approach a constant with increasing distance, and such terms
may also be removed. This procedure results in a linear dependency
with the number of atoms involved. For large molecules an interesting
application could be to apply the VMC method to Density Functional
Theory (DFT) determinants.
\newline
%
\newline
The VMC method is one of the several quantum mechanical many-body
methods. In particularly, the VMC method provides an easy and
efficient tool for including particle-particle correlations based on
physical principles. As one of the main building blocks of the QMC
machinery it deserves further investigation!

%\newline
%\begin{figure}[hbtp]
%\begin{center}
%  \input{Conclusion/slaterDeterminantNeglectContribution}
%  \caption{Plot of two selected hydrogen radial solutions; $\Psi_{1s}$
%  and $\Psi_{3d}$, for the nucleus charge $Z=10$.
%  }
%  \label{autoCorrelationTimeScaleIllustration}
%\end{center}
%\end{figure}

