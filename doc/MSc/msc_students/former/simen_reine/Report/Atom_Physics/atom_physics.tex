\chapter{Atom Physics}

\section{The Basics}

\subsection{The Atomic Problem}
\label{TheAtomicProblem}

An atomic system consists of $N$ electrons and a nucleus. 
The Hamiltonian is split in two terms, the kinetic and the potential
energy operator:

\begin{equation}
  \hat{H}(\mathbf{x}) 
  = \hat{T}(\mathbf{x}) 
  + \hat{V}(\mathbf{x})
\end{equation}

The kinetic energy operator is given by

\begin{equation}
  \hat{T}(\mathbf{x}) = -\frac{\hbar^2}{2M}\nabla^2_0
  -\sum_{i=1}^{N}\frac{\hbar^2}{2m}\nabla^2_i
\end{equation}

The first term is the kinetic energy operator of the nucleus,
and the second term is the kinetic energy operators of the electrons.
$M$ is the mass of the nucleus and $m$ is the electron mass. The
potential operator is given by

\begin{equation}
  \hat{V}(\mathbf{x}) = 
  - \sum_{i=1}^{N} \frac{Ze^2}{(4\pi \epsilon_0)r_i}
  + \sum_{i=1,i<j}^{N} \frac{e^2}{(4\pi \epsilon_0)r_{ij}}
\end{equation}

where the $r_i$'s are the electron-nucleus distances and the
$r_{ij}$'s are the inter-electronic distances. In the center of mass
system the kinetic energy operator reads (\cite{bransden1983})

\begin{equation}
  \hat{T}(\mathbf{x}) = -\frac{\hbar^2}{2(M+Nm)}\nabla^2_{CM}
  -\frac{\hbar^2}{2\mu}\sum_{i=1}^{N}\nabla^2_i
  -\frac{\hbar^2}{M}\sum_{i>j}^{N}\nabla_i\cdot\nabla_j
  \label{centerOfMassKineticEnergyOperator}
\end{equation}

while the potential energy operator remains unchanged. Note that the
Laplace operators $\nabla^2_i$ now is in the center of mass reference
system. 

The first term of eq. (\ref{centerOfMassKineticEnergyOperator})
represents the kinetic energy operator of the center of mass. The
second term represents the sum of the kinetic energy operators of the
$N$ electrons, each of them having their mass $m$ replaced by the
reduced mass $\mu = \frac{mM}{m+M}$ because of the motion of the
nucleus. The nuclear motion is also responsible for the third term,
which is often called the \emph{mass polarization} term.

%*                The Born-Oppenheimer Approximation                *
\subsection{The Born-Oppenheimer Approximation}

A common first approximation to the atomic problem is the 
{\bf Born-Oppenheimer approximation}. The nucleus consists of protons
and neutrons. The proton-electron mass ratio is about
$\frac{1}{1836}$ and the neutron-electron mass ratio is about
$\frac{1}{1839}$, so a natural approximation is to regard the nucleus
as stationary. Taking the limit $M\to \infty$ in
eq. (\ref{centerOfMassKineticEnergyOperator}), the kinetic energy
operator reduce to

\begin{equation}
  \hat{T} = -\sum_{i=1}^{N}\frac{\hbar^2}{2m}\nabla^2_i
\end{equation}

The Born-Oppenheimer approximation thus disregard both the kinetic
energy of the center of mass as well as the mass polarization term.

!!! What does this imply?
The kinetic energy of the center of mass has little importance for the
physical and chemical properties of interest in atom physics.
What about the mass polarization term?
!!!


%*                          Atomic Units                           *
\subsection{Atomic Units}

Numerical methods requires proper scaling of the system in
question. In atomic systems we scale to atomic units by setting
$m=e=\hbar=4\pi\epsilon_0=1$, see table \ref{atomicUnits}. 

\begin{table}[hbtp]
\begin{center} {\large \bf Atomic Units} \\ 
$\phantom{a}$ \\
\begin{tabular}{lll}
\hline\\ 
{\bf Quantity}                 & {\bf SI}               & {\bf Atomic unit}\\
Electron mass, $m$               & $9.109\cdot 10^{-31} kg$ & 1 \\
Charge, $e$                      & $1.602\cdot 10^{-19} C$  & 1 \\
Plack's reduced constant, $\hbar$& $1.055\cdot 10^{-34} J s$& 1 \\       
Permittivity, $4\pi\epsilon_0$   & $1.113\cdot 10^{-10} C^2 J^{-1} m^{-1}$&1\\
Energy, $\frac{e^2}{4\pi\epsilon_0 a_0}$ & $27.211 eV$       & 1 \\
Lenght, $a_0=\frac{4\pi\epsilon_0 \hbar^2}{me^2}$&$0.529\cdot10^{-10} m$&1\\       
\hline
\end{tabular} 
\end{center}
\caption{Scaling from SI to atomic units}
\label{atomicUnits}
\end{table}

Applying the Born-Oppenheimer approximation to the atomic problem of
section \ref{TheAtomicProblem}, we are to slove

\begin{equation}
  \left[-\sum_{i=1}^N \frac{1}{2} \nabla^2_i 
    - \sum_{i=1}^N \frac{Z}{r_i} + \sum_{i<j}^N \frac{1}{r_{ij}} 
    \right] \Psi(\mathbf{x}) = E \Psi(\mathbf{x})
  \label{SchrodingerBornOppenheimerAtomicUnits}
\end{equation}

This linear ODE (ordinary differential equation) has no analytical
solution for $N \ge 2$.

%*                      The Hydrogen Atom                          *
\subsection{The Hydrogen Atom}

Before we proceed with the many-body atomic problem we look at the
hydrogen atom. The Hamiltonian of the hydrogen atom given by

\begin{equation}
  \hat{H} = -\frac{1}{2} \nabla^2  - \frac{1}{r_i} 
  \label{HydrogenHamiltonian}
\end{equation}

where the Born-Oppenhimer approximation has been applied. This
equation analytically 
