\documentstyle[a4wide]{article}
\newcommand{\OP}[1]{{\bf\widehat{#1}}}

\newcommand{\be}{\begin{equation}}

\newcommand{\ee}{\end{equation}}

\begin{document}

\pagestyle{plain}

\section*{Thesis title: Monte-Carlo simulations of atoms}
The aim of this thesis is to study the structure of several light atoms using
variational and Green's function Monte Carlo techniques, combining results
from Hartree-Fock calculations in order to achieve a as good as possible
variational wave function. The thesis will explore various Monte Carlo
optimalization strategies.

This thesis 
entails the development of variational Monte Carlo (VMC)
and Green's function Monte Carlo  (GFMC) 
programs to solve Schr\"odingers equation
and obtain various expectation values of interest, such as the energy
of the ground state and excited states. 
The GFMC approach allows, in principle, for a numerically
exact solution of Schr\"odingers equation. However, it needs a reasonable
starting point. It is there where a variational Monte Carlo calculation
of the same system provides a variationally optimal trial wave function
of a many-body system and its pertinent energy.
The Slater determinant for the variational wave function is set up using
single-particle wave functions from a Hartree-Fock calculation.
The methods are briefly described in the following section.

A progress plan for this thesis project is given at the end.

\subsection*{Variational Monte Carlo}
The variational quantum Monte Carlo (VMC) has been widely applied 
to studies of quantal systems. 
The recipe consists in choosing 
a trial wave function
$\psi_T({\bf R})$ which we assume to be as realistic as possible. 
The variable ${\bf R}$ stands for the spatial coordinates, in total 
$3N$ if we have $N$ particles present. 
The trial wave function serves then as
a mean to define the quantal probability distribution 
\be
   P({\bf R})= \frac{\left|\psi_T({\bf R})\right|^2}{\int \left|\psi_T({\bf R})\right|^2d{\bf R}}.
\ee
The expectation value of the energy $E$
is given by
\be
   \langle E \rangle =
   \frac{\int d{\bf R}\Psi^{\ast}({\bf R})H({\bf R})\Psi({\bf R})}
        {\int d{\bf R}\Psi^{\ast}({\bf R})\Psi({\bf R})},
\ee
where $\Psi$ is the exact eigenfunction. Using our trial
wave function we define a new operator, 
the so-called  
local energy, 
\be
   E_L({\bf R})=\frac{1}{\psi_T({\bf R})}H\psi_T({\bf R}),
   \label{eq:locale1}
\ee
which, together with our trial PDF allows us to rewrite the 
expression for the energy as
\be
  \langle H \rangle =\int P({\bf R})E_L({\bf R}) d{\bf R}.
  \label{eq:vmc1}
\ee
This equation expresses the variational Monte Carlo approach.

The first part of this thesis deals thus with a VMC calculation of light atoms
from He to Ne, where the main goals are to study various approximations
to the trial wave function  and testing optimalization techniques based 
on energy and variance optimalization.

The trial wave function is a combination of a Slater determinant 
and a correlation part. The Slater determinant will be constructed
using both Hydrogen-like single-particle wave functions and single-particle
wave functions based on Hartree-Fock theory.

\subsection*{Green's function Monte Carlo (GFMC)}
The GFMC method is based on rewriting the 
Schr�dinger equation in imaginary time, by defining
$\tau=it$. The imaginary time Schr�dinger equation is then
\be
   \frac{\partial \psi}{\partial \tau}=-\OP{H}\psi,
\ee
where we have omitted the dependence on $\tau$ and the spatial variables
in $\psi$.

A Green's function Monte Carlo, although it allows for a formally exact
solution of Schr\"odinger's equations, needs a clever starting point 
for the energy. 
This trial energy is initially chosen to be the VMC energy of 
the trial  wave function, and is
updated as the simulation progresses. Use of an optimised 
trial function minimises the difference between the local 
and trial energies, and hence
minimises fluctuations in the calculations. 
A wave function optimised using VMC is ideal for this purpose, 
and in practice VMC provides the best
method for obtaining wave functions that accurately 
approximate ground state wave functions locally. 

The final aim of this thesis is thus to develop a GFMC program for studying
light atoms, based on a VMC calculation first.

\section*{Progress plan and milestones}
The thesis is expected to be finished towards the end  of the fall
semester of 2003
\begin{itemize}
\item Fall 2002: Follows FYS 305 lectures and an advanced seminar 
      on Computational Physics (Vekttallgivende seminar).
      Write a code which solves the variational Monte-Carlo (VMC) problem
      for light atoms, typically with up to 10 electrons. 
      This code has to include a Slater determinant based on Hydrogen-like
      single-particle wave functions and includes energy and variance
      optimalization tecnhiques.
\item Spring 2003:  Construction of a Green's function Monte Carlo code
      based on the Variational Monte Carlo code. Extension of the simple
      Metropolis sampling are included such as importance sampling through 
      the Fokker-Planck formalism.
      The GFMC code receives as input the optimal 
      variational energy and wave function from the VMC calculation and solves
      in principle the Schr\"odinger equation exactly.
      The Slater determinant used in the VMC calculation includes also
      single-particle wave functions from Hartree-Fock calculations.
\item Fall 2003: Writeup of thesis, thesis exam and exam in Fys 305.

\end{itemize}

\end{document}


