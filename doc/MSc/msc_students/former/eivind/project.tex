\documentstyle[a4wide]{article}

\begin{document}

\pagestyle{plain}

\section*{In-medium nucleon-nucleon scattering up to 1 GeV}

\subsection*{Introduction}
Experiments on
nucleon-nucleon (NN) 
scattering, be it either proton-proton or proton-neutron scattering, 
have throughout the last five decades been
of crucial importance for our understanding of the strong interaction.
At present, we are still lacking a quantitative description
of the nuclear force from first principles, that is from an interaction
based on quarks and gluons.  
Since quantum chromodynamics (QCD) is commonly
accepted as the theory of the strong interaction, the
NN interaction $V$ is completely determined by the underlying
quark-quark dynamics in QCD. However, due to the non-perturbative
character of QCD at low energies, one is still far from a
quantitative understanding of the NN interaction from the QCD point of
view. This problem is circumvented by introducing
models containing some of the properties of QCD, such as
confinement and chiral symmetry breaking. 
From  such models, one can assume that
low-energy nuclear physics phenomena can be fairly well described
in terms of hadrons like nucleons, isobars and various mesons,
which are to be understood as effective descriptions of complicated
multiquark interactions.
Other models which
also seek to approximate QCD also indicate that an effective theory in terms
of hadronic degrees of freedom may very well be the most appropriate
picture for low energy nuclear physics.
Although there is no unique prescription for how to construct
a free NN interaction, a description
of the NN interaction in terms of various meson exchanges is presently
the most quantitative representation of the NN interaction
in the energy regime of nuclear physics.
We will assume that meson-exchange is an appropriate picture at low
and intermediate energies.  The parameters which enter the theory
are thence meson-nucleon coupling constants, meson masses and cutoffs
used in the evaluation of various Feynman diagrams. Some of these
parameters are constrained from experiments, such as meson masses
and some of the coupling constants. However, the other parameters
are determined through the requirement that the modelled NN interaction
fits the 
experimental phase shifts.
The latter are related to the underlying model for the NN interaction
by solving the so-called Bethe-Salpeter equation, or in its
non-relativistic form, the Lippman-Schwinger equation.

This thesis project deals with the solution of the above equation
for incoming energies of the interacting particles up to
1 GeV.

An outline of the thesis project is given below.
A study progression plan follows thereafter, with milestones.



\subsection*{Solution of the Lippman-Schwinger equation up to 1 GeV}
Traditionally, the nucleon-nucleon interaction has been solved 
for lab energies of the incoming particles up to approximately
$300-400$ MeV. At such energies, resonances like e.g., the isobar
$\Delta $ do not appear. 
The inclusion of such resonances lead to the need of taking care of
both imaginary and real contributions in the solution of the
Lippman-Schwinger or Bethe-Salpeter equations.  

\begin{itemize}
\item The first part of this thesis deals with technicalities which arise
from such resonances. Especially, two methods for calculating the 
scattering matrix will be studied, one after Kowalski, see the reference
list, and another one which uses the so-called standard subtraction
trick, see chap 5 of the text of Brown and Jackson in the reference list.

The first aim is to study the numerical stability of a scattering equation
with complex energy denominators and complex interaction
matrices. 
The first part of the master degree project has as a final aim to
reproduce nucleon-nucleon scattering data up to 1 GeV in energy.
The reason behind this numerical stability study, is that the way poles
in the scattering equation are dealt with can have consequences 
for the solution of the scattering in a nuclear medium.

\item
The latter brings us to the next step of this project, namely the inclusion
of a Pauli operator, which prevents scattering  to forbidden intermediate
states, in the scattering equation. 


The second aim of this project is then to study in-medium, that is
nuclear matter, scattering for interacting particles up to 1 GeV.
In-medium scattering has important consequences for our understanding
of effective interactions in matter.
\end{itemize}


\section*{Progress plan}
The thesis is expected to be defended in the beginning of the spring
semester of 2003, 18 months after the approval of this application.
\begin{itemize}
\item Fall  2001: Exam in Fys 303 (Relativistic quantum physics).
                   Follows lectures on the nucleon-nucleon interaction
                   for the special curriculum and Fys 305 (many-body physics)
\item Spring 2002:  Finish part 1 discussed above, that is to have a 
      workable code for the complex scattering equation up to 1 GeV.

\item Fall 2002: Extend the code in order to study in-medium scattering up to
      1 GeV. Start writing on thesis.
\item Spring 2003: Thesis exam and exams in Fys 305 and special curriculum.

\end{itemize}
\section*{Literature}
\begin{enumerate}
\item  G.E.\ Brown and A.D.\ Jackson,
The Nucleon-Nucleon Interaction,
(North-Holland, Amsterdam, 1976).
\item K.L.\ Kowalski, Phys.\ Rev.\ Lett.\ {\bf 15} (1965) 798. See also
chapter 5 of the previous item.
\item R.\ Machleidt, K.\ Holinde and C.\ Elster, Phys.\ Repts
 149 (1987) 1.
\item R.\ Machleidt, Adv.\ Nucl.\ Phys.\  19 (1989) 189.
\end{enumerate}
\end{document}












