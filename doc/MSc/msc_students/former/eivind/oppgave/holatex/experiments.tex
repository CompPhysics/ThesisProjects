\kap{Experimental data}\label{chap:Experimental data}


The process in receiving the experimental data for the binding energy of nuclears
is straight forward and very accurate.
The experimental value of the binding energy of the Deuteron(${}^2$H) is $2.224589\pm0.000002$  MeV
This binding energy is found by measuring the energy of the $\gamma$-ray photon that is emitted when a free np-system is 
being bound.
\begin{equation}
p+n\rightarrow {}^2{\text{H}}+\gamma
\end{equation}
One can also find an binding estimate for the opposite process where a $\gamma$-ray photon breaks apart the deuteron
This method is called photodissociation. A third way to measure the binding energy is by doing spectroscopy and using the
mass doublet method described in \cite{roed}. 
These three methods are all in excellent agreement, but the last two methods are not as accurate as the first one.
\nl
Evaluating phase shifts, mixing parameters and inelasticities are on the other hand a much more complicated process.
The observable in a NN-scattering like the partial cross sections and polarizations give us enough
parameter to describe the scattering experimentally. But in order to find the phase shifts, the mixing parameters  and
inelasticities, one have to 
do a partial wave decomposition. This method is far from being accurate, and the errors are getting bigger with
the energy in the scattering. The accuracy is getting better and better as new experimental and better data
are used. So every now and then a new updated version is made. The
same people are usually publishing these updates, since they all-ready a computer program. Therefor they only 
have to update the observable with new and better ones.

It is also possible to go the other way from phase shifts and inelasticities to cross section. For example
obs.f written by ??? calculates the scattering observable from the phase shifts.

This process is difficult, and is perhaps illustrated in R.A.Arnts
phase shifts analysis from 2000 ~\cite{PRC62faseskift3GeV}.
Where the \state{3}{D}{1} state differences in the inelasticity between experiments and theory 
can be due to a sign convention (REF FIGUR!!!!).
Machleidt who wrote the theoretical "nna13" potential, strongly believes that Arndt have done a mistake in his 
sign convention for this state.
\nl
How the accuracy in phase shift variates, can be seen from phase shift analysis from 1993 
~\cite{PRC48faseskift350MeV}.
%\nl
%\begin{flushleft}
\begin{table}
\begin{tabular}{|l|c|c|c|c|}
\hline
 & \multicolumn{2}{|c|}{ pp phase shifts} &\multicolumn{2}{|c|}{ np phase shifts}\\
 \hline
$T_{lab}$& \state{1}{S}{0} & \state{3}{F}{2}& \state{1}{S}{0} & \state{3}{S}{1}\\
\hline
1 MeV& $32.684\pm0.005$& 0.000&$62.068\pm0.030$& $147.747\pm0.010$\\
\hline
100 MeV&$24.97\pm0.08$& $0.817\pm0.004$&$26.78\pm0.38$&$43.23\pm0.14$\\
\hline
200 MeV&$6.55\pm0.16$& $1.424\pm0.034$&$8.94\pm0.39$&$21.22\pm0.15$\\ 
\hline
350 MeV& $-11.13\pm 0.46$& $1.04\pm 0.16$&$-10.59\pm 0.62$&$0.502\pm0.32$\\
\hline
\end{tabular}
\caption{
\label{tableexperiment} 
Experimental proton-proton and neutron-neutron phase shifts.
Uses phase shift data from ~\cite{PRC48faseskift350MeV}.}
\end{table}
%\end{flushleft}


We see from table~\ref{tableexperiment} that the phase shift accuracy is better for pp than np scattering.
This is due to the fact that it is much easier to handle protons than neutrons. 
For example both giving the nucleons a known initial energy, and observing the
nucleons after the scattering are much more complicated for neutrons than protons.
nn-scattering data does not exist because of these difficulties with the neutron.
For the \state{1}{S}{0} states the error estimate of the phase shift is less than 10$\%$
when compared with the phase shift. 
The errors are about four times as big in the nn-scattering as it is for pp-scattering. 
The \state{3}{F}{2} state and \state{3}{S}{1} illustrates smaller phase shifts where
the the error estimate is about 50$\%$ of the phase shift. In 
\ref{fig:hiEnergy} we have the same problem with the inelasticity in the
\state{3}{D}{2} state. In this state the inelasticities are so small compared with their corresponding
uncertainty, so Arndt have set them to zero in his article ~\cite{PRC62faseskift3GeV}.


Experimental phase shifts for higher energies up to 3 GeV are given in ~\cite{PRC62faseskift3GeV}.
\nl
Phase shifts in-medium are still not calculated experimentally. 
\nl
Polarization scattering experiments are done by a polarizing the particles spin, so that one can study the
spin dependence in the scattering.
 
