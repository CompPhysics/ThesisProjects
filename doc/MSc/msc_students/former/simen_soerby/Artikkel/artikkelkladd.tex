Artikkelkladd:

The girls have quite a nasty looking expression for the forces in play:

\begin{equation}
F = F_D + f_c e^{-(\frac{t}{t_c})^2} - f_v v - D
\end{equation}

where $F_D$ is a constant driving force, $f_c e^{-(\frac{t}{t_c})^2}$ is an added, yet highly dampened, force representing a push-off and a smaller effective area of the runner in the start of the run, $f_v v$ represents the runners physiological limitations and $D$ represents the air resistance, given by: 

\begin{equation}
D = \frac{1}{2}(1-\frac{1}{4}e^{-(\frac{t}{t_c})^2})\rho C_D A(v-w)^2
\end{equation}

However, they debate the acceleration at $t=0$, in which $v=0$ and $e^{-(\frac{t}{t_c})^2}=1$. The wind factor $w$ is also set to zero at this point. This gives us, with small amounts of calculations: 

\begin{equation}
a_{t=0} = \frac{F_{t=0}}{m} = \frac{F_D + f_c}{m} = \frac{900\texttt{ N}}{80\texttt{ kg}} = 11,25 \texttt{ m/s}^2
\end{equation}

Question: Why not make use of analytical mathematics?

After a brief experiment with a falling pen and some discussion with myself involved, both students agree that 11 m/s$^2$ might be the correct answer after all. Still, I won't let them conclude without having some ``hard evidence'' with the help of mathematics. I convince them to have another look:

\begin{description}
\small
\item[I] What about the other term that was added? The force term? How big is that in the beginning? 
\item[J1] The term ... 2.219? 
\item[J2] It's pretty big in the beginning! 
\item[J1] It's really hopeless to see! We can check. 
\item[] [...]
\item[J1] Time equals zero? So it becomes ... The exponential to zero, whats that? [...] Yes, he gets an acceleration of eleven. So I'm wrong?
\item[J2] I hope! 
\end{description}