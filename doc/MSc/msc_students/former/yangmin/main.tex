\documentclass[a4paper,11pt]{book}

%-----------------PREAMBLE---------------%
%Graphics
\usepackage{graphicx}
\usepackage{color}
\DeclareGraphicsRule{.pdftex}{pdf}{.pdftex}{}
\usepackage[]{subfigure}
\usepackage{algorithm}
\usepackage{algorithmic}
\usepackage{enumerate}
\usepackage{bbold}

\usepackage{tikz}
\usetikzlibrary{decorations.markings,matrix,arrows}
\usetikzlibrary{trees}
\usetikzlibrary{decorations.pathmorphing}
\usetikzlibrary{decorations.markings}


%Text
\usepackage{appendix}
\usepackage{empheq}
\usepackage{listings}
\lstset{ %
language=C,                % the language of the code
basicstyle=\footnotesize,       % the size of the fonts that are used for 
numbersep=5pt,                  % how far the line-numbers are from the code
backgroundcolor=\color{white},  % choose the background color. You must add \usepackage{color}
showspaces=false,               % show spaces adding particular underscores
showstringspaces=false,         % underline spaces within strings
showtabs=false,                 % show tabs within strings adding particular underscores
frame=single,                   % adds a frame around the code
tabsize=2,                      % sets default tabsize to 2 spaces
captionpos=b,                   % sets the caption-position to bottom
breaklines=true,                % sets automatic line breaking
breakatwhitespace=false,        % sets if automatic breaks should only happen at whitespace
title=\lstname,                 % show the filename of files included with \lstinputlisting;
                                % also try caption instead of title
escapeinside={\%*}{*)},         % if you want to add a comment within your code
morekeywords={*,...}            % if you want to add more keywords to the set
}

%For tables
\usepackage{multirow} 

%---------------------------------------------------
%customizing page body layout using the geometry package:
%---------------------------------------------------
\usepackage[vscale=0.8,left=3cm,right=2cm,footskip=2.5cm, headsep=1cm]{geometry}
%-----------------------------------------------------------
%customizing headers and footers using the fancyhdr package:
%-----------------------------------------------------------
\usepackage{fancyhdr}
%redefining plain style,(necessary to remove default page-style setings):
\fancypagestyle{plain}{
\fancyhf{}
\renewcommand{\headrulewidth}{0pt}
\renewcommand{\footrulewidth}{0pt}}
%seting new style
\pagestyle{fancy}
\fancyhf{} %clears both head and foot
\renewcommand{\chaptermark}[1]{\markboth{\chaptername\ \thechapter.\ #1}{}}
\renewcommand{\sectionmark}[1]{\markright{\thesection\ #1}{}}
\fancyhead[LE]{\footnotesize \bfseries \leftmark}
\fancyhead[RO]{\footnotesize \bfseries \rightmark}
\headheight 15pt
\fancyfoot[LE,RO]{\bfseries \thepage}

%----------------------------------------
%redefining the \cleardoublepage command
%----------------------------------------
%automatically clears the (even) blank side following each chapter
\makeatletter
\def\cleardoublepage{\clearpage\if@twoside \ifodd\c@page\else
  \hbox{}
  %\vspace*{\fill}
  %If you want the blank page following the chapters
  %to have some content, this is added here. 
  %\vspace{\fill}
  \thispagestyle{empty}
  \newpage
  \if@twocolumn\hbox{}\newpage\fi\fi\fi}
\makeatother

%-------------------------------
%section-setings,no indentation:
%-------------------------------
%\usepackage{parskip}
%\setlength{\parindent}{0pt}
%\setlength{\parskip}{\baselineskip}

%-------------
%math setings:
%-------------
\usepackage{amsmath}
\usepackage{amsfonts}
\usepackage{amssymb}
\usepackage{simplewick} 
\usepackage{amsthm}  %Gives us abillity to begin proof

%changes refrences into links:
\usepackage{hyperref}

%Tabulars
\usepackage{booktabs,tabularx}

%---------------ENDING PREAMBLE---------------%

%----Note----%
%\footnote{}
%\emph{}
%\label{} - \ref{} - \pageref{} 

%--------------COSTUMIZE NEWCOMMANDS-----------------%
\newcommand{\bra}[1]{\langle{#1}|}
\newcommand{\ket}[1]{|{#1}\rangle}
\newcommand{\braket}[2]{\langle{#1}|{#2}\rangle}
\newcommand{\ql}{\left}
\newcommand{\qr}{\right}
\newcommand{\hh}[1]{\widehat{#1}}
\newcommand{\p}[1]{\hat{#1}}
\newcommand{\da}[1]{{#1}^{\dagger}}
%\newcommand{\bb}[1]{{\bf{#1}}}
\newcommand{\dd}[1]{\delta_{#1}}
\newcommand{\vv}[2]{\bra{#1}v\ket{#2}}
\newcommand{\vva}[2]{\bra{#1}|\hat{v}|\ket{#2}}


\renewcommand{\vec}[1]{\mathbf{#1}}
\newcommand{\bb}[1]{\, \ensuremath{\mathbf{#1}}}
%--------------COSTUMIZE amsthm package-----------------%
\newtheorem{theorem}{Theorem}[section]
\newtheorem{lemma}[theorem]{Lemma}
\newtheorem{proposition}[theorem]{Proposition}
\newtheorem{corollary}[theorem]{Corollary}


\providecommand{\e}[1]{\ensuremath{\times 10^{#1}}}

\begin{document}
\tableofcontents

%\maketitle
%\thispagestyle{empty}


%\input{./mpi.tex}


%\chapter*{Preface}
%\chapter*{Abstract}
%\chapter*{Notation}
%\section*{Symbols}
\section*{Abbreviations}
\begin{table}[ht]
\begin{tabular}{l l}
EPV & Exclusion-Principle-Violationg\\
Eq & Equation\\
Fig & Figure\\
BCH & Baker-Campbell-Hausdorff expansion\\
SD & Slater Determinant\\
PEP & Pauli Exclusion Principle\\ 
HF & Hartree-Fock Method\\
RHF & Restriced Hartree-Fock Method\\
DFT & Density functional Theory\\
MBPT & Many-Body Perturbation Theory\\
CC & Coupled Cluster\\
CCM & Coupled Cluster Method\\
CCS & Coupled Cluster Method with Single Excitations\\
CCD & Coupled Cluster Method with Double Excitations\\
CCSD & Coupled Cluster with Single and Double Excitations\\
CCSDT & Coupled Cluster with Single, Double and Triple Excitations\\
CCSDTQ & Coupled Cluster with Single, Double,Triple and Quadrouple Excitations\\
DMC & Diffusion Monte Carlo \\
\end{tabular}
\label{table:abbreviations}
\end{table}

%%%%%%%%%%%%%%%%%%%%%%%%%%% INTRODUCTION %%%%%%%%%%%%%%%%%%%%%%%%%%%%%%%%%%%%%%%%%%%%%%%%%%%
\chapter*{Acknowlegdement}

I am very thankful to my supervisor, professor Morten Hjort-Jensen, who have taught me ever since 2009. He have been a mentor and helped me far beyond what I would expect from a supervisor. His enthusiasm and humor have been a motivation. I am very lucky to have met Morten. And I hope I have in return fullfiled his expectations. I am deeply grateful to the many hours Simen Kvaal have spent helping me to implement the transformations and writing replies to my questions. 

It have been both interesting and challenging to work with such a likable person such as Marte Hoel J\o rgensen, whom I worked with on the coding part. I admire her professionalism and discipline. And she have helped me to see things from other perspectives.

I would also thank Gustav Jansen and \O yvind Jensen for their insights and suggestions that have helped me understand the nature of my problems. Magnus Pedersen Lohne have been very helpful and responded everytime on email. Thanks to my fellow students David Sk\aa lid Amundsen, Andreas Nakkerud, J\o rgen Tr\o mborg, Christoffer Hirth, J\o rgen H\o gberget, who have helped me and made my time at the University of Oslo ver enjoyable.
 
I have enjoyed the physics aswell as non-physics discussions with my friends Henrik and Michael, they have broaden my perspectives on what is fruit or not. And it was certainly fruitful to say the least. It have been a challenging year, and skiing have been very relaxing, thanks to Patrick, Morten, Christoffer, Thomas, Henrik and Mikael for joining me for some memorable trips in Lillomarka.

Finally, I want to give credits to my mother, father, and my sister. Without their encouragement and support there would be no thesis. 

\begin{quotation}
\hfill Yang Min Wang
\end{quotation}
\hfill
\hfill
\hfill
\begin{quotation}
\it ``The highest forms of understanding we can achieve is laughter and human compassion.''
\end{quotation}
\hfill R. P. Feynman.


\part{INTRODUCTION}

\chapter{Introduction}
The goal of this thesis is to show how we can study the double quantum dot potential using the CCSD method. During this process we have used Magnus Pedersen Lohnes Master's Thesis \cite{lohne} as an starting point, and a lot of work have been to optimalize his code. I have been working with Marte Hoel Joergensen \cite{marte} with the optimalization. And we have developed a method of using a Hartree-Fock calculation as an input to improve the CCSD calculations. 

This thesis can be regarded as a academic curiosity. However we will show the versatilty of the CCSD method and the generality of our potential, could be used in a real world scenario. This is the main reason why we want to study the quantum dot. People have applied DFT method to describe this system \cite{introdft}, but we want to describe it from a deeper foundation. That is why we use the ab initio CCSD method. And some of the physics behind the quantum dot have been used develop everything from devices such as single electron transistors and quantum dot lasers \cite{introqdotlaser} to ``artificial atoms'' \cite{ashoori}. They share the similar properties such as shell structure and magic numbers. Lately one of the important applications is the use of colloidal quantum dot for detecting canser cells \cite{gao}. For this we need a proper theoretical description of the electronic system, exchange coupling, correlation energies and ground state energies.

The quantum dot are esentially a device that can trap electrons, the typical size of these devices are between a micrometer to few nanometers. In these quantum dots we can quantum mechanical effects such as discrete energylevels. Our quantum mechanical model for the quantum dot is a parabolic quantum well which traps the electrons in two dimentions. Such a model does not exist in nature but is a starting point to understand realistic quantum dots. They are in fact crystalline and have periodic potentials. That is the reason why we are intereseted in the double dot. This could give us insights of the physics in more real life systems. 
 


%%%%%%%%%%%%%%%%%%%%%%%%%%%%%%%%%%%%%%%%%%%%%%%%%%%%%%%%%%%%%%%%%%%%%%%%%%%%%%%%%%%%%%%%%%%%
%%%%%%%%%%%%%%%%%%%%%%%%%%%%%%%%%%%% PART FUNDAMENTALS %%%%%%%%%%%%%%%%%%%%%%%%%%%%%%%%%%%%%
%%%%%%%%%%%%%%%%%%%%%%%%%%%%%%%%%%%%%%%%%%%%%%%%%%%%%%%%%%%%%%%%%%%%%%%%%%%%%%%%%%%%%%%%%%%%
 
\part{FUNDAMENTALS}

\chapter{Quantum Mechanics for single-particle systems}
\begin{quotation}
Quantum mechanics is that mysterious, confusing discipline, which really no one of us understands but which we know how to use. It works perfectly, as far as we can tell, in describing physical reality, but it is a ``counter-intuitive discipline'',as the social scientists would say. Quantum mechanics is not a theory, but rather a framework within which we believe any correct theory must fit. \it{Murray Gell-Mann, in Mulvey (1981)}
\end{quotation}


\section{Introduction}
I in the late 19th century scientists had problems with their classical understanding of the ways things are. Some experiments showed up that did not coicide with the current theories. And the beginning of quantum mechanics was when Max Planck published in (1900) a theory of black-body radiation. His explained that atoms can absorb and emit discrete quanta of radiation with energy $\epsilon = h f$, where $f$ is the radiation frequency and $h$ is the fundamental constant called Planck's constant.

\begin{equation}
h \approx 6.626 \times 10^{-34} Js
\end{equation}
%
In classical physics we separate between the particle and waves, a classical particle cannot be wavelike and particle like at the same time. But as some the experiments have shown this does not reflect the reality (Young's famous double slit experiment). Quantum mechanics can sometimes be counterintuitive in that regard, but then again, Newton's theory of gravity must have been difficult to grasp at first. How do we know about these``invisible'' forces which we cannot see? 

Another problem poeple had was the atoms would be unstable if we had electrons that where particle-like and orbited around defined orbits. The orbiting electrons would radiate electromagnetic energy and eventually fall into the nucleus. We need to have a wavelike description of the atomic electrons in order to explain their stability. Wave-like electrons are confined inside the atom, and at the lowest state, the ground state, the electron cannot readiate away its energy and fall into the nucleus. This gave us a whole new way of looking at nuclear physics. It revolutionized the 20th century physics.

\subsection{Measurement and Observables}
Measurements are done by a subject, usually called an \emph{observer} who have an instrument which takes measurements on an \emph{object}. During this process we have disturbances and never ideal conditions, in classical physics the disturbance is directly associated with measurement itself and can be made arbitrarily small, depending of how good the engineers are. If we wanted to measure length with 5 decimal precision, we could just get a more precise ruler. But this is not the case in quantum mechanics. Each time we do measurements in quantum mechanics we have a probability distribution of different outcomes. This have nothing to do with the instrument itself, it is a part of the the instrisic nature of formulation of the quantum mechanics called the \emph{Heisenberg uncertainty principle}.

\begin{equation}
 \Delta x \Delta p \geq \frac{\hbar}{2}
 \label{heisenberg}
\end{equation}
%
This means that if we determine the momentum exactly $\Delta x = 0$, then the momentum is totaly uncertain. The concept that particles exist with definite position and momentum is an idealistic classical concept.   

\subsection{The Schr\"{o}dinger equation} 
The Schr\"{o}dinger equation is the quantum mechanical eqvivalent to Newton's Second Law. It describes the motion of a quantum mechanical particle. This is a partial differential equation which describes how the wave function, representing the particle, \emph{flows}. The timedependent Schr\"{o}dinger equation for a particle moving in a three-dimentional potential is:

\begin{equation}
 i \hbar \frac{\partial \Psi}{\partial t} = \hat{H}\Psi
\label{schrodinger}
\end{equation}
%
Where $H$ is called the Hamilton operator and $\Psi$ is the wavefunction:
 
\begin{equation}
 \hat{H} = -\frac{\hbar^2}{2m} \nabla^2 + V(\vec{r})
\end{equation}
%
It is important to point out that this equation describes a non realtivistic motion of a quantum particle i.e. $E >> m_0 c^2$, where $m_0$ is the restmass of the quantum particle. 

\subsection{Probabilities}
Max Born introduced an interpretation of the Schr\"{o}dinger wave function $\Psi(\vec{r},t)$. He pointed out that the probability of detecting a particle at a particular location and time is proportional to $|\Psi(\vec{r},t)|^2$. Thus $|\Psi(\vec{r},t)|^2$ is often viewed as a probability density at the position $r$ and time $t$. And the wavefunction is often reffered to as a probability amplitude.

We can normalize our probability density to sum all the possible position of the particle to one:

\begin{equation}
 \int |\Psi(\vec{r},t)|^2 d^3 \vec{r} = 1
\label{normalization}
\end{equation}
%
The wavefunction could also be a function of momentum and thereby describe the probability for finding a momentum in a certain range.

\section{Postulates of Quantum Mechanics}
We could argue that Newton's law of motion is the postulates in classical mechanics. What are Newton's laws derivable from?. So likewise quantum mechanics is based on some fundamental ``laws of nature'' that must be underivable. 

A postulate is a statement made without any proof, an ``underived'' statement``. And in physics a postulate could be translated into a proposal which could either be verified or falsified based on experiment \cite{bowman}.

\begin{quotation}
{\bf Postulate 0 }{\it We describe a system by is its state vector $\ket{s}$, an observable q and is a hermitian operator $\hat{Q}$ which operates on any $\ket{s}$}
\end{quotation}

\begin{quotation}
{\bf Postulate 1 }{\it The time evolution of the quantum state $\Psi$ is goverened by the timedependent Schr\"{o}dinger equation. $\Psi$ is a function of position coordinates $q_n$ and the time $t$. $\hat{H}$ is the Hamiltonian}
\begin{equation}
 i\hbar\frac{\partial \Psi}{\partial t} = \hat{H} \Psi
\label{eq:tdseq}
\end{equation}
\end{quotation}

\begin{quotation}
{\bf Postulate 2 }{\it The possible results of measurement of an observable $Q$, are its eigenvalues $q_i$ of its operator $\hat{Q}$.}
\end{quotation}

\begin{quotation}
{\bf Postulate 3 }{\it If the results of the measurements is found to be $q_i$, then after the measurement the system will ''collpase`` into a corresponding eigenstate $\ket{q_i}$.}
\end{quotation}

\begin{quotation}
{\bf Postulate 4 }{\it The expectation value of an observable is given by}
\begin{equation}
\langle F \rangle = \int \psi^* \hh{F} \psi \, d \tau \qquad \mbox{(coordinate representation)}
\end{equation}
\end{quotation}


\section{Different Representations of Quantum Mechanics}
As we have seen, originally the state $\Psi$ was a solution to the Schr\"{o}dinger equation, which was a partial differential equation, second order in space and first order in time. So the ''state function`` $\Psi$ have to be a function of time and space coordinates. We will be refer to this as the Schr\"{o}dinger representation. Quantum mechanics could be more abstract and more convenient shown by Dirac \cite{dirac}. Another way to represent it is by matrix mechanics introduced by Heisenberg.
Those are the main ways which we could represent quantum mechanics and they are mathematically equivalent.

\subsection{Dirac's notation}
An abstract quantum mechanical state $\Psi$ is represented by a ''bra`` vector $\bra{\Psi}$ or a ''ket`` vector $\ket{\Psi}$. The distinction between the forms lies in the context in which they are used and will become clearer when we show this. The scalar product in Diracs notation is the ''bracket``, which is very convenient compared to write the integral in the Schr\"{o}dinger representation.

\begin{equation}
\bra{\Phi}\ket{\Psi} \equiv \int \Phi^*(\vec{r})\Psi(\vec{r}) d\vec{r}
\label{eq:bracket}
\end{equation}
%
The expectation value of an operator is:

\begin{equation}
\bra{\Phi}\hat{Q}\ket{\Psi} \equiv \int \Phi^*(\vec{r})\hat{Q}\Psi(\vec{r})d\vec{r}
\label{eq:dirac_expect} 
\end{equation}
%
The time-independent Schr\"{o}dinger equation in the Dirac notation:

\begin{equation}
\hat{H} \ket{E_i} = E_i \ket{E_i}
\end{equation}
%
In this example we see that the label $E_i$ inside the ket, tells us about its eigenvalue. The spectra of the eigenvalues to an operator could be discrete or continous like the operator $\hat{x}$: 

\begin{equation}
 \hat{x}\ket{x} = x \ket{x}
\label{eq:xrep}
\end{equation}
%
If we wish to get back to the Schr\"{o}dinger representation $\Psi(x)$, we could do a ''projection`` of $\ket{\Psi}$ on the eigenfunction $\ket{x}$:

\begin{equation}
 \Psi(x) \equiv \braket{x}{\Psi}
\label{eq:coordinatrep}
\end{equation}
%
Similary the complex conjugate:
\begin{equation}
 \Psi^*(x) \equiv \braket{\Psi}{x}
\label{eq:ccoordinatrep}
\end{equation}

\subsection{Heisenberg's matrix formulation}
\label{Heisenberg}

The state vector $\ket{\Psi}$ is represented as a vector by its projection on a complete set of basis states $\ket{E_1},\ket{E_2}...$:

\begin{equation}
 \ket{\Psi} \equiv \left(\begin{array}{c} \braket{E_1}{\Psi} \\ \braket{E_2}{\Psi} \\ ... \\ ... \end{array}\right)
\label{eq:heisenbergstate}
\end{equation}
%
Similary we represent the operators by:

\begin{equation}
 \hat{H} \equiv 
\left(\begin{array}{cccc} 
\bra{E_1}\hat{H}\ket{E_1} & \bra{E_1}\hat{H}\ket{E_2} & \bra{E_1}\hat{H}\ket{E_3} & ... \\ 
\bra{E_2}\hat{H}\ket{E_1} & \bra{E_2}\hat{H}\ket{E_2} & ... & ... \\
... & ... & ... & ... \\
... & ... & ... & ... \\
\end{array}\right)
\label{eq:heisenbergopeator}
\end{equation}
%
Since the basis states $\ket{E_n}$ are the eigenstates of $\hat{H}$, its matrix representation is diagonal, and the diagonal elements are eigenvalues of $\hat{H}$. To diagonalize an operator in matrix representation is therefore the same as solving the time-independent Schr\"{o}dinger equation. 

This representation is very useful when dealing with angular momentum and spin.

\section{Angular Momentum and Spin}

In classical physics angular momentum is defined as:
\begin{equation}
 \vec{L} = \vec{r} \times \vec{p}
 \label{angular momentum}
\end{equation}
$\vec{r}$ is the displacement vector from the origin and $\vec{p}$ is the linear momentum. 
We can write out the components of the angular momentum operator in quantum mechanics by using the substitution
\[
 p_x \rightarrow -i\hbar \frac{\partial}{\partial x}
\]
%
Gouldsmit and Uhlenbeck (in 1925) introduced the concept of interal, purely quantum mechanical, angular momentum called \emph{spin}. This was later experimentally confirmed by the Stern-Gerlach experiment (1922).

The spin eigenstates are:
\begin{equation}
\hat{S}_z \ket{l,m} = m \hbar \ket{l,m} 
 \label{zspin}
\end{equation}

\begin{equation}
\hat{S}^2 \ket{l,m} = l(l+1) \hbar^2 \ket{l,m}
 \label{totalspin}
\end{equation}
%
And their commutation relation reads

\begin{equation}
  \left[\hh{S}_x,\hh{S}_y \right] = i\hbar\hh{S}_z  \qquad \left[ \hh{S}_y,\hh{S}_z \right] = i\hbar\hh{S}_x \qquad \left[ \hh{S}_z, \hh{S}_x \right] = i\hbar \hh{S}_y
\end{equation}
%
This basically means that we cannot determine eigenvalues of two different components, for example $\hh{S}_z$ and $\hh{S}_x$ simultaneously, they are \emph{incompatible} observables. But we can determine one of the directions and $\hh{S}^2$. The spin quantum numbers are \cite{griffiths}

\begin{align}
  s = 0,\frac{1}{2},1,\frac{3}{2},... \\
  m_s = -s,-s+1,...,s-1,s
\end{align}
%
Where $s$ is defined as the \emph{spin} of the particle. Electrons have $1/2$ spin while photons have $1$. 

\subsection{The wavefunction}
The total wavefunction of a state vector mentioned in Postulate 0 are composed of a spatial part $\phi(x,y,z)$ and a spin part $\ket{\chi}$. They are from different Hilbert spaces and mathematically the state vector are a tensor product of these two. 

\begin{equation}
 \ket{\psi} \equiv \psi(r) \equiv \phi(x,y,z) \otimes \ket{\chi}
\end{equation}


\section{Simple systems}

\subsection{Particle in a infinite potential well}
One of the simplest single-particle systems we could solve exact is the infinite one-dimentional potential well, defined by:

\begin{equation}
V(x) = \left\{ \begin{array}{rl}
 -x &\mbox{ if $0\geq x \geq a$} \\
  \infty, &\mbox{ otherwise}
       \end{array} \right.
 \label{def:infinitepotential}
\end{equation}
%
and we want to solve the Schr\"{o}dinger equation with respect to this potential

\begin{equation}
-\frac{\hbar^2}{2m}\frac{d^2 \psi}{dx^2} + V(x) = E \psi
 \label{eq:schrodingerinfinitewell}
\end{equation}
%
We can rewrite this equation

\begin{equation}
\frac{d^2 \psi}{dx^2} = \frac{2m}{\hbar^2}\left[V(x)-E\right]\psi
 \label{eq:rewriteschrodingerinfinitewell}
\end{equation}
%
For the case $x>a$ and $x<0$ where $E < V(x)$, then we see that the second derivate always have the same sign, and therefore cannot be normalized (ref postulate). Such wavefunctions cannot exist in this range. Therefore the wavefunctions only exist in the range $0\geq x \geq a$ and continuity of $\psi$ requires the boundary conditions $\psi(0)=0$ and $\psi(L)=0$. 

Using standard methods of solving second order differential equations (see \cite{boas} for details) we get 

\begin{equation}
\psi(x) = A\sin kx
 \label{eq:infinitewellsolution}
\end{equation}

\begin{equation}
k \equiv \frac{\sqrt{2mE}}{\hbar} 
 \label{def:k}
\end{equation}
%
Normalizing in order to get the constant $A$:

\begin{equation}
\int_0^a |A|^2 \sin^2 kx \, dx = |A|^2 \frac{a}{2} = 1 \Rightarrow |A|^2 = \frac{2}{a}
 \label{deriv:normalizing}
\end{equation}
%
We then choose to use the positive root 

\begin{equation}
\psi(x) = \sqrt{\frac{2}{a}} \sin kx
 \label{eq:infinitewellsolution2}
\end{equation}
%
Next we want to determine the konstant $k$. From boundary condition we know that $\sin ka = 0$. Then possible values for $k$ are

\begin{equation}
 k = \frac{n \pi}{a}, \qquad \mbox{with $n = 1,2,3...$}
 \label{eq:kvalues}
\end{equation}
%
and hence the possible values of $E$ from (\ref{def:k}) are discrete:

\begin{equation}
E_n = \frac{\hbar^2 k^2}{2m} = \frac{n^2\pi^2\hbar^2}{2ma^2}
 \label{eq:energies}
\end{equation}


%%------Under construction--------

\subsection{Particle in a harmonic oscialltor potential well}
\label{onebodyharmonicoscillatorpotential}
The next simplest problem to solve is the \emph{harmonic oscillator} in one-dimentional case. The Hamiltonian we have are

\begin{equation}
  \hh{H} = \frac{\hat{p}^2}{2m} + \frac{1}{2}\omega^� \hat{x}^2
\end{equation}
%
Wher $\hat{x} = x$ is the position operator and the $\hat{p}$ is the momentum operator, given by

\begin{equation}
  \hat{p} = -i\hbar \frac{d}{dx}
\end{equation}

There are two main approaches to solve this, one of them is the analytical approach \cite{griffiths} and we get the solutions in Schr\"{o}dinger representation,

\begin{equation}
  \psi_n(x) = \sqrt{\frac{1}{2^n n!}} \left(\frac{m\omega}{\pi \hbar}\right)^{1/4} e^{-\beta^2 /2} H_n(\beta), \qquad \beta = \sqrt{\frac{m\omega}{\hbar}}x
\end{equation}

And the eigenvalues,

\begin{equation}
  E_n = \left(n + \frac{1}{2}\right) \hbar \omega, \qquad n = 0,1,2,3...
\end{equation}

The other way is algebraically \cite{griffiths} and we get solutions in Dirac formalism:

\begin{equation}
  \ket{\psi_n} = \frac{1}{\sqrt{n!}} (a_+)^n \ket{0}
\end{equation}
%
Where $\ket{0}$ is our ground state and $a_+ = (a_-)^\dagger$ is the creation operator
%
\begin{equation}
  a_{\pm} = \frac{1}{\sqrt{2\hbar m \omega}} (\mp i\hat{p} + m\omega \hat{x})
\end{equation}



%%-----------------  MATHEMATICAL DESCRIPTION OF THE QD -------------------
\chapter{The Quantum Mechanics behind Quantum Dots}
In this chapter we will give an quantum mechanical description for the 2-dimentional quantum dot. We will start by solving the Schr\"{o}dinger equation for a single-electron parabolic quantum dot with an applied magnetic field. 

Recently development in techniques of quantum dot growth have made it possible to create better solar cells. But also other exotic phenomena as quantum computing. Quantum dot is a semiconductor on the nanoscale. It can trap one or group of electrons in a spacially confined potential. The size of quantum dot ranges from a few hundrer to many thousand of atoms \cite{qdotshur}. And it can confine everything from one electron to hundreds. 

\section{Description of the Quantum Dot}
As we have seen in the chapter 1, quantum dots are artificially created. And there are varius techniques and methods for creating quantum dots, which gives us them different properties. In this thesis we will concentrate on the quantum dots created inside the Gallium Arsenide (GaAs) semiconductors. The semiconductor is sandwiched between layers of Aluminum Gallium Arsenide (AlGaAs) semiconductor material which has a bigger bandgap. This acts like an insulator and results in a confinement in the vertical direction. Our choice for the confinement potential is a parabolic harmonic oscillator $\omega_x = \omega_y = \omega$. Both numerical \cite{numerical1, numerical2} and experimental \cite{experimental1, experimental2} studies have shown that this is a reasonable approximation. And in our case the electrons inside will only feel the Coulomb interaction. The Hamiltonian then becomes 
\begin{equation}
\hh{H} = \sum_{i=1}^N \left(-\frac{\hbar^2}{2 m^*} \nabla_i^2 + \frac{1}{2} m^* \omega^2 r_i^2 \right) + \frac{e^2}{4\pi \epsilon_0 \epsilon_r} \sum_{i<j}^N \frac{1}{r_{ij}}
  \label{Hamiltonian for QD}
\end{equation}
The $e$ is the electron charge, $\epsilon_0$ is the vacuum permittivity, $\epsilon_r$ is the relative permittivity, and $r_{ij} = |r_i - r_j|$, $\omega$ is the oscillator frequency, and $r_i$ is the distance from electron $i$ to the potential minimun $(r=0)$.
It is important to notice that the $m^*$ here is the effective electron mass and not be mistaken by the Newtonian reduced mass which is a classical phenomena. This is an simplification we have made to our problem. The effective electron mass differs from the free-electron mass $m$ and its isotropic and independent of both the position and the energy of the electron. The effective mass is the result of motion of an electron in a periodic potential \cite{quantumheterostructures}. For example in GaAs the electrons appear to carry mass that is only $7\%$ of the free-electron mass \cite{ashoori}.  

\section{The One-electron Quantum Dot}
We start by derving the wavefunctions for a general spherical symmetric potential $V(r) = V(-r)$ in two dimentions. Our Hamiltonian in cartesian coordinates reads
\begin{equation}
  -\frac{\hbar^2}{2m^*} \nabla^2 \psi(x,y) + V(x,y) \psi(x,y) = E \psi(x,y)
   \label{Hamiltonian cartesian coordinates}
\end{equation}
%
We want to do rewrite this using polar coordinates
%
\begin{align}
x &= r \cos \theta \\
y &= r \sin \theta \\
r &= \sqrt{x^2 + y^2}
\end{align}
The Laplacian them becomes
\begin{equation}
\nabla^2 = \frac{1}{r} \frac{\partial}{\partial r}\left(r\frac{\partial}{\partial r}\right) + \frac{1}{r^2} \frac{\partial}{\partial \theta^2} 
  \label{laplacian in 2D}
\end{equation}
Inserting this into Eq. (\ref{Hamiltonian cartesian coordinates}) gives us the Schr\"{o}dinger equation in polar coordintes
\begin{equation}
  -\frac{\hbar^2}{2m^*} \left(\frac{\partial^2}{\partial r^2} + \frac{1}{r} \frac{\partial}{\partial r} + \frac{1}{r^2} \frac{\partial^2}{\partial \theta^2}\right) \psi(r,\theta) + V(r) \psi(r,\theta) = E \psi(r,\theta)
   \label{Hamiltonian polar coordinates}
\end{equation}
Introduction a solution on the form $\psi(r,\theta) = R(r) Y(\theta)$ and multiplying by $\frac{2m^*}{\hbar^2 R(r)Y(\theta)}r^2$ in Eq. (\ref{Hamiltonian polar coordinates}) we obtain

\begin{equation}
\frac{r^2}{R(r)}\left[\frac{d^2 R}{d r^2} + \frac{1}{r} \frac{dR}{dr} + \frac{2m^*}{\hbar^2}(E-V(r))R(r) \right] = -\frac{1}{Y(\theta)}\frac{\partial^2 Y(\theta)}{\partial \theta^2}
  \label{separable Hamiltonian 2}
\end{equation}
The left side of this equation depends on $r$ while the right side depends on $\theta$. This can only be satisfyied if each term is equal to a constant $k=m^2$
%
\begin{align}
\frac{r^2}{R(r)}\left[\frac{d^2 R}{d r^2} + \frac{1}{r} \frac{dR}{dr} + \frac{2m^*}{\hbar^2}(E-V(r))R(r) \right] &= -m^2 \label{separation constants1} \\
\frac{1}{Y(\theta)}\frac{\partial^2 Y(\theta)}{\partial \theta^2} &= m^2 
  \label{separation constants2}
\end{align}
%
The solution to anuglar part Eq. (\ref{separation constants2}) is 
\begin{equation}
Y(\theta)  = Ce^{im\theta}  
\end{equation}
%
Normalization gives us the constant $C$
\begin{equation}
C^2 = \frac{1}{\int_0^{2\pi}Y(\theta)^2 \, d\theta} = \frac{1}{2\pi}
  \label{constant C}
\end{equation}
%
The normalized solution for the angular part is
\begin{equation}
Y(\theta) = \frac{1}{\sqrt{2\pi}} e^{im\theta}
  \label{normalized angular part}
\end{equation}
%
The total wavefunction must satisfy the physical condition that $\psi(r,\theta) = \psi(r,\theta+2\pi)$ This makes a restriction on the quantum number $m$ which can take integral values
\begin{equation}
m = 0, \pm 1, \pm 2, ...
  \label{m quantum number}
\end{equation}
For the Eq. (\ref{separation constants1}) we can simplify by defining
\begin{equation}
  u(r) = \sqrt{r}R(r) \Rightarrow R(r) = \frac{u(r)}{\sqrt{r}}
  \label{radialequation function}
\end{equation}
which yields
\begin{equation}
  -\frac{\hbar}{2m^*}\frac{d^2 u}{dr^2} + \left[ V(r) + \frac{\hbar^2}{2m^*} \frac{m^2-\frac{1}{4}}{r^2} \right] u(r) = E u(r)
\end{equation}
This is the radial equation. And it have the same form as the one-dimentional time-independent Schr\"{o}dinger equation with an effective potential
\begin{equation}
V_{eff} = V(r) + \frac{\hbar^2}{2m^*}\frac{m^2-\frac{1}{4}}{r^2}
  \label{effective potential}
\end{equation}
The radial function Eq. (\ref{radialequation function}) must satisfy the normalization conditions
\begin{equation}
\int_0^\infty |u(r)|^2 dr = 1
  \label{normalization radial}
\end{equation}
For this to be normalizable we must require the boundary conditions 
\begin{equation}
  \text{$u(0) = C$ and $u(\infty) = 0$}, \quad \text{where $C$ is a constant}
\end{equation}
%
Finally the general solutions to the spherical symmetrical potential is
\begin{equation}
\psi(r,\theta) = R(r) \frac{1}{\sqrt{2\pi}} e^{im\theta} 
  \label{final solution to spherical}
\end{equation}

\subsection{Parabolic quantum dot with influence of an electromagnetic field}
\label{parabolic quantum dot}

As an academic exercise we shall solve the one-electron Schr\"{o}dinger equation for a quantum dot in 2 dimentions. Most of the derivation here are taken from \cite{jackson}. The classical Hamiltonian of a charged electron in a electromagnetic field reads
  
  \begin{equation}
   H = \frac{1}{2m}(\bb{p} - e\bb{A})^2 + e\phi 
   \label{electromagnetic Hamiltonian}
 \end{equation}
%
with $\bf{A}$ and $\phi$ as the electromagnetic vector and scalar potentials, $m$ is the electron mass, $e$ is the charge, and $\bb{p}$ is the momentum vector. The electromagnetic fields are related to the potentials  
\begin{align}
\bb{E} &= - \frac{1}{c} \frac{\partial \bb{A}}{\partial t} - \nabla \phi \\ 
\bb{B} &= \nabla \times \bb{A}
  \label{elmag potentials}
\end{align}
where $\bb{E}$ is electric field, and $\bb{B}$ is the magnetic field, which satisfies Maxwells equations. The quantum mechanical Hamiltonian consist of and additional term which couples the spin to the electromagnetic field $-\mu\cdot \bb{B}$, where $\mu$ is the magnetic moment of the electron. 
\begin{equation}
  \hh{H} = \frac{1}{2m^*} (\hh{p} - e\bb{A})^2 + e\phi +\frac{1}{2}m^* \omega_0^2 r^2 - \hh{\mu} \bb{B}
  \label{Total Hamiltonian}
\end{equation}
Here the $\hh{p}$ is a quantum mechanical momentum operator. The time-independent Schr\"{o}dinger we want to solve is 
\begin{align}
\frac{1}{2m^*} (\hh{p} - e\bb{A})^2 + e\phi +\frac{1}{2}m^* \omega_0^2 r^2 - \hh{\mu} \cdot \bb{B} \psi(\bb{r}) & = E \psi(\bb{r})
  \label{Time-independent SE}
\end{align}
\begin{align}
H_r &= \frac{1}{2m^*} (\hh{p} - e\bb{A})^2  +\frac{1}{2}m^* \omega_0^2 r^2\label{spatial hamiltonian}\\
H_s &=  - \hh{\mu} \cdot \bb{B} \label{spin hamiltonian}
\end{align}
%
where the wavefunction $\psi(\bb{r})$ also include the spin part which are decoupled with the spatial part, i.e. their expectation are uncorrolated, and we can therefore write the wavefunction as a product of each.
\begin{equation}
 \psi(\bb{r}) = R(r) \otimes \ket{m_s}  
  \label{wavef spin}
\end{equation}
%
Here the quantum number $m_s=\pm \frac{1}{2}$, since electrons are fermions. Inserting this into Eqs.(\ref{spatial hamiltonian}) and (\ref{spin hamiltonian}) we obtain two equations, one spatial and one spin dependent
\begin{align}
  \left(\frac{1}{2m^*} (\hh{p} - e\bb{A})^2 + e\phi +\frac{1}{2}m^* \omega_0^2 r^2 \right)R(r) &= E_r R(r) \\
  -(\mu \cdot \bb{B}) \ket{m_s} &= E_s \ket{m_s}
\end{align}
%
Which gives us the total energy 
%
\begin{equation}
E = E_r + E_s + e\phi
  \label{total energy}
\end{equation}
%
We want to do a gauge transformation on the potential $\bb{A}$ with the Coulomb gauge condition. This will not change the potentials $\bb{E}$ and $\bb{B}$.
%
\begin{equation}
\bb{\nabla} \cdot \bb{A} = 0
  \label{gauge}
\end{equation}
%
A choice that satisfy this condition is 
\begin{equation}
\vec{A} = \frac{1}{2} \bb{B} \times \left(x\,\bb{i} + y \, \bb{j} \right)
  \label{coulomb gauge}
\end{equation}
%
We want to expand the first term in the spatial Hamiltonian $\hh{H}_r$ Eq. (\ref{spatial hamiltonian}) using the condition Eq. (\ref{coulomb gauge})

\begin{align}
(\hh{p}^2 - e\bb{A})^2 &= \hh{p}^2 - e(\hh{p}\bb{A} + \bb{A}\hh{p}) + e^2 \bb{A}  \\
& = \hh{p}^2 - 2e \bb{A}\cdot\hh{p}+ e^2 \bb{A}^2 \label{second deriv} \\
& = \hh{p}^2 - e\mathbf{B}\cdot \hh{L} + \frac{e^2}{4}\left(\mathbf{B} \times (x \bb{i} + y \bb{j}) \right)^2 \label{thirdeq}  
\end{align}
%
In Eq. (\ref{second deriv}) we have used that $\hh{p}$ and $\bb{A}$ commute because of the coulomb gauge. 
\begin{equation}
\hh{p} \cdot \mathbf{A} \psi = -i\hbar\nabla \cdot \left(\mathbf{A} \psi \right) = -i\hbar\left( \underbrace{\nabla \cdot \mathbf{A}}_{=0} + \mathbf{A} \cdot \nabla \psi \right) = \mathbf{A} \cdot \left(-i\hbar\nabla \psi \right) = \bb{A}\cdot \hh{p} \psi
\end{equation}
And in Eq. (\ref{thirdeq}) we have used the relation 
\begin{equation}
 \left( \mathbf{B} \times \vec{r} \right) \cdot \mathbf{p} = \mathbf{B} \cdot \left( \vec{r}  \times \mathbf{p} \right) =  \mathbf{B} \cdot \mathbf{L}
  \label{math relation cross}
\end{equation}
%
The applied magnetic field is constant and homogeneous along the z-axis. $\bb{B} = B_0 \bb{k}$. Then our spatial Hamiltonian $\hh{H}_r$ simplifies to 
\begin{equation}
\hh{H}_r = \frac{1}{2m^*} \left[\hh{p}^2 - e\mathbf{B_0}\left(x\hh{p}_y - y \hh{p}_x \right) + \frac{e^2B_0^2}{4}\left( x^2 + y^2 \right)\right]
 + \frac{1}{2} m^* \omega_0^2 (x^2+y^2)
  \label{total hamiltonian 2}
\end{equation}
%
Introducing
\begin{equation}
  \omega_B \equiv \frac{eB_0}{2m^*}
\end{equation}
%
And 
\begin{equation}
  \omega \equiv \omega_0 + \omega_B^2
\end{equation}
%
The Hamiltonian becomes 
\begin{equation}
\hh{H}_r = \frac{1}{2m^*} \left(\hh{p}^2 - eB_0\hh{L}_z \right) + \frac{1}{2} m^* \omega^2 \left(x^2+y^2\right) \label{final Hamil}
\end{equation}
%
Where $\hh{L}_z$ is the angular momentum in the $z$-direction. $L_z = x\hh{p}_y - y \hh{p}_x$ \cite{hemmer}. The next step is tranform the Hamiltonian Eq. (\ref{final Hamil}) from a cartesian coordinate representation to a polar coordinate representation. Then the angular momentum can be expressed
as \cite{hemmer} 
\begin{equation}
\hh{L}_z = -i\hbar \frac{\partial}{\partial \theta}
  \label{Lz in polar}
\end{equation}
%
Which yields the time-independent Schr\"{o}dinger equation for $\hh{H}_r$ Eq. (\ref{spatial hamiltonian})
\begin{equation}
\left[ - \frac{\hbar^2}{2m^*} \left(\frac{\partial^2}{\partial r^2} + \frac{1}{r} \frac{\partial}{\partial r} + \frac{1}{r^2}\frac{\partial^2}{\partial \theta^2} - \frac{ieB_0}{\hbar} \frac{\partial}{\partial \theta}\right) + \frac{1}{2}m^*\omega^2 r^2 \right] \psi(r,\theta) = E \psi(r,\theta)
  \label{Hamilton in polar}
\end{equation}
This is almost the same as Eq. (\ref{Hamiltonian polar coordinates}) which we have solved, except now have an additional term $\frac{ieB_0}{\hbar} \frac{\partial}{\partial \theta}$  caused by the magnetic field. But this can be seprated from the radial equation, since it only depends on the angle $\theta$. 
%
The wavefunction can still be seprated in an angular and a radial part, and we will use the same ansatz as Eq. (\ref{final solution to spherical})
\begin{equation}
\psi(r,\theta) = R(r) \frac{1}{\sqrt{2\pi}} e^{im\theta}, \qquad m=0,\pm 1, \pm 2,... 
  \label{wave function ansatz}
\end{equation}
%
Inserting this to Eq. (\ref{Hamiltonian polar coordinates}) 
\begin{align}
\left[ - \frac{\hbar^2}{2m^*} \left(\frac{\partial^2}{\partial r^2} + \frac{1}{r} \frac{\partial}{\partial r} + \frac{1}{r^2}\frac{\partial^2}{\partial \theta^2} + \frac{meB_0}{\hbar} \frac{\partial}{\partial \theta}\right) + \frac{1}{2}m^*\omega^2 r^2 \right] R(r) = E_r R(r)
  \label{Hamilton in polar 2}
\end{align}
%
The solution of this radial equation is
%
\begin{equation}
R_{nm}(r) = \sqrt{\frac{2n!}{(n+|m|)!}}\beta^{\frac{1}{2}(|m|+1)}r^{|m|}e^{-\frac{1}{2}\beta r^2} L_n^{|m|} (\beta r^2)
  \label{Laguerre solutions}
\end{equation}
%
Here the subscript $n$ denote the principal quantum number, and $m$ is the angular momentum number
\begin{align}
  n &= 0, 1, 2, 3, .... \\
  m &= 0, \pm 1, \pm, 2, \pm 3,...
\end{align}

$L_n^{|m|}$ is the associated Laguerre polynomials \cite{arfken}, and $\beta$ is defined as
\begin{equation}
\beta = \frac{m^*\omega}{\hbar}
  \label{definition beta}
\end{equation}
%
The final eigenfunction to the spatial Hamiltonian $H_r$ is then
%
\begin{equation}
 \psi(r,\theta) = \sqrt{\frac{n!}{\pi(n+|m|)!}}\beta^{\frac{1}{2}(|m|+1)}r^{|m|}e^{-\frac{1}{2}\beta r^2} L_n^{|m|} (\beta r^2) e^{im\theta}
 \label{final for spatial Hamiltonian}
\end{equation}
%
With the corresponding eigenvalues
%
\begin{equation}
E_r = (1+|m|+2n)\hbar \omega + m\hbar \omega_B
  \label{corresponding eigenvalues}
\end{equation}
%
See Appendix in \cite{lohne} and \cite{drosdal} for details. Now we consider the spin Hamiltonian $H_s$ Eq. (\ref{spin hamiltonian}). The quantum mechanical magnetic moment $\hh{mu}$ is given by \cite{griffiths}

\begin{equation}
\hh{\mu} = \frac{eg}{2m^*} \hh{S}
  \label{magnetic moment}
\end{equation}
%
where $g$ is the $g$-factor and approximate $2$ for the electron. $\hh{S}$ is the spin operator. Since the magnetic field is $\mathbf{B} = B_0 \bb{k}$, the Schr\"{o}dinger equation reads
%
\begin{equation}
-\frac{egB_0}{2m^*}S_z\ket{m_s} = E_s \ket{m_s}
  \label{SL for the spin part}
\end{equation}
%
$\hh{S}_z$ is the $z$-component of the total spin $\hh{S}$. The eigenvectors and eigenvalues of this operator are given by [Insert ref]
%
\begin{equation}
E_s = -\frac{eg\hbar B_0}{2m^*}m_s = gm_s \hbar \omega_B
  \label{eigenvalue for spin}
\end{equation}
%
The total eigenvalue for the system becomes
\begin{equation}
E = (1+|m|+2n)\hbar \omega + m \hbar \omega_B + g m_s \hbar \omega_B + e\phi
  \label{total energyeigenvalue}
\end{equation}
And the corresponding eigenstate
\begin{equation}
\psi_{nmm_s}(\bb{r},\theta) = \sqrt{\frac{n!}{\pi(n+|m|)!}}\beta^{\frac{1}{2}(|m|+1)}r^{|m|}e^{-\frac{1}{2}\beta r^2} L_n^{|m|} (\beta r^2) e^{im\phi} \otimes \ket{m_s}
  \label{total eigenstate}
\end{equation}
%
Without any external magnetic field $\bb{B} = 0$, the energy becomes spin-indenpendent because $\omega_B = 0$
\begin{equation}
E_{nm}^0 = (1+|m|+2n) \hbar \omega_0 
  \label{spin independent energy}
\end{equation}
%
In the spirit of perturbation theory we denote this with a superscript $0$. We will have a degeneracy in spin since the Hamiltonian is spin-independent. For each pair of the quantum numbers $\{n,m\}$ we have two different quantum states, one with $m_s=-\frac{1}{2}$ and the other with $m_s=\frac{1}{2}$.
\begin{equation}
R \equiv (1+|m|+2n)
  \label{shellnumber}
\end{equation}
%
$R$ is defined as the \emph{shellnumber}, it correspond to the energy level and the degeneracy for each level $R$ is
\begin{equation}
  g(R) = 2R
\end{equation}
%
This system have a \emph{shell structure}, Figure \ref{fig:shellnumber}, i.e. the energy levels are equidistant from each other and we have a defined degeneracy. This is similar to the shell model in nuclear physics, in which Goeppert-Mayer, Wiger and Jensen was awarded the nobelprize (1963).
%
\begin{figure}
\centering
\scalebox{0.6}{\input{shell.pdftex_t}}
\caption{Shell structure of a single-electron parabolic quantum dot for $\bb{B=0}$, where $R$ is the shell number defined in Eq. (\ref{shellnumber}) and $m$ is the angular quantum number. The arrows $\uparrow \downarrow$ denote the spin quantum number $m_s=\pm \frac{1}{2}$}
\label{fig:shellnumber}
\end{figure}
%
The total number of spin-orbitals for a given shellnumber $R$ is 
%
\begin{equation}
  N = \sum_{R=0}^{R'} g(R) = 2R' + 2(R'-1) + 2(R'-3) + ... + 2
\end{equation}
%
We have tabulated some of the values in Table \ref{tab:shellfilling} 

\begin{table}[H]
\centering 
\begin{tabular}{rrr}
\toprule
$R$ & $g(R)$ & $N$\\
\midrule
1	& 2 	 &  2 \\
2	& 4 	 &  6 \\
4	& 8 	 &  20  \\
8	& 16 	 &  72 \\
10	& 20 	 &  110 \\
15	& 30 	 &  240 \\
20	& 40 	 &  420 \\
\bottomrule
\end{tabular}
\caption{Some values for different shellnumber $R$, where $g(R)$ is the degeneracy. $N$ is the total number of single-electron spin-orbitals occupied in $R$-number of shelles. This is often refered to \emph{magic numbers} which indicate the number of spin-orbitals need to complete the shells.}
\label{tab:shellfilling}
\end{table}
%
If we take a look at Eq. (\ref{total energyeigenvalue}), we see that in a presense of a magnetic field, the degenerate energy levels would split because of the sign of $m$. If we simplify by neglecting the spin quantum number $m_s = 0$, and setting the constant $e\phi = 0$, we can express the energy in Eq. (\ref{total energyeigenvalue}) by 

\begin{equation}
\frac{E_{nm}}{\hbar \omega_0} = (1+|m|+2n) \sqrt{1+\frac{\omega_B^2}{\omega_0^2}} + m\frac{\omega_B}{\omega_0} 
  \label{fock darwin}
\end{equation}

If we plot $\frac{E_{nm}}{\hbar \omega_0}$ as a function of $\frac{\omega_B}{\omega_0}$ we get the Fock-Darwin energy spectrum, which was first solved by V.Fock \cite{fock} and later by C.G. Darwin \cite{darwin}. 

\begin{figure}[H]
\centering
\scalebox{0.7}{\includegraphics{fockdarwin.eps}}
\caption{Fock-Darwin energy spectrum for a single-electron quantum dot. When $\omega_B = 0$ all the quantum states with equal $R$ is degenerate, after $\omega_B$ increases the levels begin to split due to the angular momentum contribution from $m\hbar \omega_B$. The degeneracy can reappear for certain levels and certain $\omega_B$'s. The energy levels will shift back and forth between $(n,m)$ pairs, but they all appear to reach an asymptote in the high field limit, forming the famous Landau levels \cite{landaulevels}, indicated by the red line. The landau levels are plottet for $N_L =0,1,2$ in Eq. (\ref{landaulevelsplot})}
\end{figure}
%
The energy of the different states will then decrease or increase with stronger magnetic field $\omega_B$, depending on $m$. States that belong to different shells for $\bb{B}=0$ will become degenerate, and when the magnetic field increases even more, we would reach a asymptote for the energies as we clearly see in Figure \ref{fock darwin}. These energy levels are popularly called Landau levels, which he discovered in 1930 at an age of 22 \cite{landauonline}. 

We are interested in the energy levels when $\omega_B \rightarrow \infty$ for the lowest energy levels $m<0$

\begin{equation}
\lim_{\omega_B \rightarrow \infty} E_{nm} = (1+2n)\hbar\omega_B
\end{equation}
%
The Landau levels appear when $N_L \equiv n = 0,1,2,3...$. And the energy only depends on $n$ in the high limit of $\bb{B}$

\begin{equation}
E_L \approx (1+2N_L) \hbar \omega_B
\label{landaulevelsplot}
\end{equation}

\subsection{Scaling the Hamiltonian}
\label{section:scaling the hamiltonian}
When we do computations its a good thing too have dimentionless parameters. The fewer things that can go wrong in the calculation the better. Therefore we want to rescale our manybody Hamiltonian Eq. (\ref{Hamiltonian for QD}) so that its dimentionless. The following derivation is based on examples \cite{scaling,lohne}

\begin{align}
  \omega  & = \omega_c \bar{\omega} \\
  \bb{r}  & = r_c \bb{\bar{r}} \\
  \bb{\nabla}  & = \frac{1}{r_c} \bb{\bar{\nabla}} \\
  r_i^2 & = r_c^2 \bar{r}_i^2 \\
  r_{ij} & = r_c \bar{r}_{ij}
\end{align}
%
Here the variables with the subscript $c$ is just a constant with the same dimentions as the variable we want to rescale. The variables with a bar is the dimentionless variables that we want. Inserting these into our Hamiltonian Eq. (\ref{Hamiltonian for QD}) gives

\begin{equation}
\hh{H} = -\frac{\hbar^2}{2m^*r_c^2} \sum_{i=1}^N \bar{\nabla}_i^2 + \frac{1}{2}m^* \omega_c^2 \bar{\omega}^2 r_c^2 \sum_{i=1}^N \bar{r}_i^2 + \frac{\hbar}{\epsilon r_c} \sum_{i<j}^N \frac{1}{\bar{r}_{ij}}
  \label{rescaled Qdot Hamiltonian}
\end{equation}
%
Where 
\begin{equation}
  \epsilon = \frac{4\pi \epsilon_0 \epsilon_r}{e^2}
\end{equation}
%
Furthermore we define the oscillator length to be
\begin{equation}
r_c = \sqrt{\frac{\hbar}{m^* \omega}}
  \label{oscillator length}
\end{equation}
Inserting this into our Hamiltonian Eq. (\ref{rescaled Qdot Hamiltonian})
\begin{equation}
\hh{H} = -\frac{\omega_c \bar{\omega} \hbar}{2} \sum_{i=1}^N \bar{\nabla}^2_i + \frac{\hbar}{2} \omega_c \bar{\omega} \sum_{i=1}^N \bar{r}^2_i + \frac{\hbar}{\epsilon} \sqrt{\frac{m^*\omega_c \bar{\omega}}{\hbar}} \sum_{i<j}^N \frac{1}{\bar{r}_{ij}}
  \label{rerescaled Hamiltonian2}
\end{equation}
%
We want to scale the Hamiltonian aswell so that it have units of the Hartree energy $E_h$ \cite{hartrees}
\begin{equation}
  \bar{H} =  \hh{H} / E_h
\end{equation}
%
Where the Hartree energy is defined as
%
\begin{equation}
E_h = m^* \left(\frac{e^2}{4\pi \epsilon_0 \epsilon_r \hbar}\right)^2 = \frac{m^*}{\epsilon^2}
  \label{E hartree}
\end{equation}
%
The scaled Hamiltonian becomes
\begin{equation}
\bar{H} = -\frac{\omega_c\bar{\omega} \hbar \epsilon^2}{2m^*} \sum_{i=1}^N \bar{\nabla}^2_i + \frac{\hbar \epsilon^2}{2m^*} \omega_c \bar{\omega} \sum_{i=1}^N \bar{r}^2_i + \frac{\hbar\epsilon}{m^*} \sqrt{\frac{m^*\omega_c \bar{\omega}}{\hbar}} \sum_{i<j}^N \frac{1}{\bar{r}_{ij}}
\end{equation}
%
To make it dimentionless we have to defined
%
\begin{equation}
\omega_c = \frac{m^*}{\hbar \epsilon}
  \label{dimentional condition}
\end{equation}
%
Which is fine since $\omega_c$ have the dimention of $[1/s]$ which is the same as $\omega$. Our final dimentionless N-electron scaled Hamiltonian reads

\begin{equation}
\bar{H} = -\frac{\omega_c}{2} \sum_{i=1}^N \bar{\nabla}^2 + \frac{1}{2} \omega_c \sum_{i=1}^N \bar{r}_i^2 + \sqrt{\omega_c} \sum_{i<j}^N \frac{1}{\bar{r}_{ij}}
  \label{N-body Hamiltonian scaled}
\end{equation}



\section{The double quantum dot}
\label{sec:The double quantum dot}
\label{section:the double dot potential}
The rapid development of nanotechnology has made the quantum computer a more realistic achievement. One of the possible candidates to the quantum bit is a double quantum dot. Expermients of this have been done both with a trapped nuclei \cite{doublenuclei} and electrons \cite{doubleelectron}. We will investigate a model based on \cite{doubledot}. Our one-electron Hamiltonian is the same as before but now with a change in the potential

\begin{equation}
  \hh{H} = -\frac{\hbar^2}{2m^*} \nabla_i^2 + V_c(x,y) 
  \label{one electron hamilton double}
\end{equation}
where the confinement potential is
\begin{equation}
V_c(x,y) = \frac{1}{2} m^* \omega_0^2 \cdot \left[x^2+y^2 - 2L_x |x| + L_x^2 \right]
  \label{double potential}
\end{equation}
%
From \cite{doubledot} we use GaAs material parameters $m^* = 0.067m_e$, and the confinement strength $\hbar \omega_0 = 3.0$ meV. Which correspond to a harmonic oscillator length of $\sqrt{\hbar/\omega_0 m^*} \approx 5.3$ nm. And with the minima separated from a distance $2L_x$ from each other. The values of $V_c(x,0)$ are plotted in Figure \ref{doubledotfigure}.

\begin{figure}[H]
\centering
\scalebox{0.7}{\includegraphics{doubledot.eps}}
\caption{Confinement potential for $V_c(x,0)$ and $L_x=50$ nm}
\label{doubledotfigure}
\end{figure}
%
We will then solve the two-dimentional Schr\"{o}dinger equation by using a finite difference method with a three-point Laplacian \cite{computerjensen}. This will give us an Hermitian matrix for which the eigenvalues and eigenvectors can be solved using the QR-algorithm \cite{faires}. The algorithm was independently introduced in 60s by Kublanovskaya \cite{QR2} and Francis \cite{francis}. And have been recognized as one of the 21st century most important algorithms \cite{mostimportant}. 




\subsection{The QR-Algorithm}
\label{section:The QR-Algorithm}
The idea is surprisingly simple, first step is to factor the matrix $A$ into a product of an orthogonal matrix $Q_1$ and positive upper triangular matrix $R_1$. This also refered to as an QR-decomposition and solved by using the Gram-Schmidt algorithm \cite{linalg}.
\begin{equation}
A_1 = Q_1 R_1
  \label{first iteration}
\end{equation}
Next step is to multiply $Q$ and $R$ in a reversed order
\begin{equation}
A_2 = R_1 Q_1
  \label{second iteration}
\end{equation}
%
We repeat this process by finding the $Q$ and $R$ values to $A_2$
\begin{equation}
A_2 = Q_2 Q_2
\end{equation}
%
The complete algorithm can be written as
\begin{equation}
  A = Q_1R_1,\qquad R_kQ_k = A_{k+1} = Q_{k+1}R_{k+1}, \qquad k=1,2,3,...
\end{equation}
%
Where $ R_k,Q_k$ is from the previous steps, and $Q_{k+1}$ is still an ortogonal matrix ($Q^TQ=1$) and $R_{k+10}$ is an positive upper triangular.
%
\begin{algorithm}
\caption{\emph{QR-Algorithm}}
\begin{algorithmic}
\FOR{$i=1 \to n$}
\STATE $Q_iR_i = A_i$ 
\STATE $A_{i+1} = R_iQ_i$ 
\ENDFOR
\end{algorithmic}
\end{algorithm}
%
The iteration will finally create a matrix $\tilde{A}$ whose diagonal entries are eigenvalues of $A$. The reason why this works is because all the $A_k$ are similar to each other and therefore they have a common set of eigenvalues with different eigenvectors, i.e. 
\begin{proof}
\bf{If} $Ax = \lambda x$ and $\tilde{A} = S^TA S $ ($S S^T=1$) $\Rightarrow S^T A S x = \lambda x \Rightarrow  A(Sx) = \lambda (Sx)$
\label{similartransformation}
\end{proof}
%
And 
%
\begin{equation}
A_{k+1} = R_k Q_k = Q_k^T (Q_k R_k) Q_k = Q_k^T A_k Q_k 
  \label{shows the similarity transform}
\end{equation}

We will not go in to rigorous details of why this works, but readers are recommended to read \cite{qrproof}. Instead we will give an numerical example, given a Hermitian matrix 
\begin{equation}
A = 		\left[\begin{array}{ccc}
			5 & 4 & 1\\
			4 & 3 & 2\\
			1 & 2 & 1\\
		\end{array}	\right]
		\label{Hermitian A}
\end{equation}
the \emph{exact} eigenvalues found by $\texttt{eig(A)}$ in \textsc{matlab} are 
%
\begin{equation}
\begin{matrix}
 \lambda_1 & = &  8.6625 \\
 \lambda_2 & = &  1.1444 \\
 \lambda_3 & = & -0.8070
\end{matrix}
\label{eigenvalues}
\end{equation}
%
Then the initial QR-factorization $A_1 = Q_1R_1$ produces 
\begin{equation}
Q_1 = 		\left[\begin{array}{ccc}
    0.7715  & -0.0392 &  -0.6350 \\
    0.6172 & -0.1960  &  0.7620 \\
    0.1543  &  0.9798  &  0.1270
		\end{array}\right] 
				\qquad 
		R_1 = 		\left[\begin{array}{ccc}
    6.4807  &  5.2463  &  2.1602 \\
         0  &  1.2150  &  0.5487 \\
         0  &       0  &   1.0160
		\end{array}\right] 
\end{equation}
Which in turn gives us the new $A_2$
\begin{equation}
A_2 = R_1Q_1 = \left[\begin{array}{ccc}
    8.5714 &   0.8346  &  0.1568 \\
    0.8346 &   0.2995  &  0.9955 \\
    0.1568  &  0.9955  &  0.1290		
		\end{array}	\right]
\end{equation}
%
We continue this procedure
%
\begin{equation}
A_3 = R_2Q_2 = \left[\begin{array}{ccc}
    8.6611  &  0.1033 &   0.0168 \\
    0.1033  &  0.5392 &   0.9035 \\
    0.0168  &  0.9035 &  -0.2003		
		\end{array}	\right]
\end{equation}
%
\begin{equation}
A_4 = R_3Q_3 = \left[\begin{array}{ccc}
    8.6625  &  0.0131  &  0.0017 \\ 
    0.0131 &   0.7871  &  0.7548 \\
    0.0017  &  0.7548 &  -0.4496 
		\end{array}	\right]
\end{equation}
%
After $17$ iterations the off-diagonal elements are practically zero, and the eigenvalues on the diagonal correspond to the eigenvalues found in (\ref{eigenvalues}). 
\begin{equation}
A_{17} = R_{16}Q_{16} = \left[\begin{array}{ccc}
    8.6625  &   0.0000  & -0.0000 \\
    0.0000  &  1.1444   & 0.0098 \\
    0.0000  &  0.0098  & -0.8069
		\end{array}	\right]
\end{equation}
%
But if we want to increase the precision, this method becomes rather slow. As we can see, the convergence of this method is not that impressive, assuming \textsc{matlab} is using the Householder's QR factorization method \cite{bible}, we are going to have $O(n^3)$ flops per iteration, in addiction we have a matrix-matrix multiplication.

One of the optimazations we could do is to use Householder's method for tridiagonalization \cite{computerjensen}, i.e. we want to find a tridiagonal matrix $T$ that is a similar transformed of the matrix $A$.

\begin{equation}
  T = S^T A S, \qquad S = S_1S_2...S_{n-2} 
  \label{similar}
\end{equation}
%
This would speed up the iteration since we have fewer off-diagonal elements to worry about. A second optimazation is the QR-algorithm. we cpuld improve the convergence by introducing a \emph{shift} on the diagonal. This is popularly called \emph{The accelerated QR-algorithm} or \emph{The shifted QR-algorithm} \cite{kincaid}

For a general tridiagonal matrix

\begin{equation}
T_m = \left[\begin{array}{cccccc}
      \alpha_1^{m}  &  \beta_1^{m}   & 0 	    & 0      & \hdots & 0  \\ 
      \beta_1^{m}   &  \alpha_2^{m}  & \beta_2^{m}  & 0 \\ 
                 &  \beta_2^{m}   & \alpha_3^{m} & \beta_3 & & \vdots  \\ 
       \vdots       &                &  \ddots      & \ddots &  \ddots  \\
           &      	     &              &  \beta_{n-2}^m & \alpha_{n-1}^m  & \beta_{n-1}^m \\ 
        0	    &      &      \hdots          &       & \beta_{n-1}^n & \alpha_n^m  \\ 
		\end{array}	\right]
\end{equation}

\begin{algorithm}
\caption{\emph{The accelerated QR-Algorithm}}
\begin{algorithmic}
\FOR{$i=1 \to n$}
\STATE $T_m - \alpha_n^m I = Q_m R_m $ 
\STATE $T_{m+1} = R_m Q_m + \alpha_{n}^m I$ 
\ENDFOR
\end{algorithmic}
\end{algorithm}
%
It was interesting to see how well this algorithm works compared to a, and have therefore mad a plot for comparison in Figure \ref{fig:QRsummary}

\begin{figure}[H]
\centering
\scalebox{0.7}{\includegraphics{QR.eps}}
\caption{This a plot of different eigenvalues as a function of iteration. The blue line with diamond points represent our normal QR-algorithm on matone electron hamilton doublerix $A$ Eq. (\ref{Hermitian A}), and the red line with points represent the accelerated QR-algorithm on a tridiagonalized matrix $A$. The improved QR-algorithm converges faster for the lower eigenvalues}
\label{fig:QRsummary}
\end{figure}


The final issue is the numerical precision or more correctly, the numerical imprecision. It is important to know the error bound when we do this type of calculations. The details of the mathematics can be read in \cite{lyche}. We will take use of the Hoffman-Wielandt Theorem which states that

\begin{theorem}
  Let $A$ and $E$ be a real symmetric $n\times n$. And let $T = A + E$ with eigenvalues $\{ \gamma_i \}$. And   $\{ \lambda_i \}$ the eigenvalues of $A$, arranged in increasing order. Then
  \begin{equation}
      \left[\sum_{j=1}^n \left(\lambda_i - \gamma_i \right) \right]^{\frac{1}{2}} \leq F(E)     
  \end{equation}
 Where $F(E) = \left(\sum_{ij}^n |a_{ij}|^2 \right)^{1/2}$ is the Frobenius norm of $E$
\end{theorem}
%
Let $T$ be a tridiagonal matrix and let $\tilde{T}$ be the new matrix obtained by deleting $\beta_{n-1}$ from the off-diagonal positions $(n-1,n)$ and $(n,n-1)$ of T. And let $\{\lambda_j\}$ and $\{\tilde{\lambda}_i\}$ denote the eigenvalues respectivly. Then from the Wielandt-Hoffman theorem

\begin{equation}
  \left[\sum_{j=1}^n \left(\lambda_i - \tilde{\lambda}_i \right) \right]^{\frac{1}{2}} \leq F(T-\tilde{T}) = \sqrt{2}|\beta_{n-1}|
\end{equation}
%
Since $\beta_{n-1}$ is the only term that was left after the subtraction. The conclusion of this is that we are closest to the exact eigenvalues when the off-diagonals are smallest.

\subsection{Discretizing the Schr\"{o}dinger equation}
Our one-electron Hamiltonian from Eq. (\ref{one electron hamilton double}) reads

\begin{equation}
  \hh{H} = -\frac{\hbar^2}{2m^*} \nabla_i^2 +  \frac{1}{2} m^* \omega_0^2 \cdot \left[x^2+y^2 - 2L_x |x| + L_x^2 \right]
  \label{rerescaled Hamiltonian}
\end{equation}
Using the rescaled parameters in section \ref{section:scaling the hamiltonian} we get
\begin{equation}
  \bar{H} = -\frac{\omega_c}{2} \nabla^2 +  \frac{1}{2} \omega_c^2 \cdot V_c(\bar{x},\bar {y}) 
\end{equation}
%
The time-independent Schr\"{o}inger equation of this Hamiltonian is then
\begin{equation}
-\frac{\omega_c}{2} \left[\frac{\partial}{\partial \bar{x}^2} +\frac{\partial}{\partial \bar{y}^2} \right]u(\bar{x},\bar{y}) +  \frac{1}{2} \omega_c^2 \cdot V_c(\bar{x},\bar{y}) u(\bar{x},\bar{y})  = E u(\bar{x},\bar{y}) 
  \label{tisl scaled hamilton}
\end{equation}
%
Where $u(\bar{x},\bar{y})$ is the single-electrons eigenfunctions. From here on the dimentionless coordinates $\bar{x}$ and $\bar{y}$ would be refered to as $x$ and $y$. And we would set the constant $\omega_c = 1$. The differential equation can be solved as a matrix diagonalization problem. By subracting two Taylor series we get the numerical second derviate \cite{computerjensen}

\begin{align}
  u(x+h) + u(x-h) &= 2u(x) + h^2 f''(x) + O(h^4) \\
  u(y+h) + u(y-h) &= 2u(y) + h^2 f''(y) + O(h^4) 
\end{align}

\begin{equation}
 f''(x) = \frac{u(x+h)-2u(x)+u(x-h)}{h^2} + O(h^2) \\
 f''(y) = \frac{u(y+h)-2u(y)+u(y-h)}{h^2} + O(h^2) 
 \label{discretize}
\end{equation}
%
For the two-dimentional case the Laplacian becomes
\begin{equation}
  \nabla^2  \approx \frac{1}{h_x^2} \left(u_{i-1,j} - 2u_{i,j} + u_{i+1,j} \right) +  \frac{1}{h_y^2} \left(u_{i,j-1} - 2u_{i,j} + u_{i,j+1} \right) + O(h^2)
  \label{poisson insert}
\end{equation}
%
Where we have used the more compact way of writing $u(x\pm h,y) = u_{i \pm 1, j}$, $u(x,y\pm h) = u_{i, j\pm1}$. An example could be a grid with integration points $n_x = 3$, and step $h=h_x=h_y=\frac{x_{\max}-x_{\min}}{n_x} = \frac{1}{3}$, $x=x_{\min}+ih$, $y=y_{\min}+jh$, where $(i,j)\in\{1,2,3\}$. See Figure \ref{unitgrid}
%
\begin{figure}[H]
\centering
\input{unitgrid}
\caption{The discretized unit grid in our example. The indices are arranged in lexical order}
\label{unitgrid}
\end{figure}
%
Inserting Eq. (\ref{poisson insert}) into the Schr\"{o}dinger equation Eq. (\ref{tisl scaled hamilton}), gives
\begin{equation}
  d_{ij} u_{i,j} - e u_{i,j-1} - e u_{i,j+1} - e u_{i-1,j} - e u_{i+1,j} = \lambda u_{i,j}
\end{equation}
%
Where we have defined
\begin{equation}
  d_{ij} = \frac{4}{h^2} + V_{ij} \qquad e = -\frac{1}{h^2}
\end{equation}
%
The left side is the discretized poisson equation in two dimensions. This gives us a set of linear equations which can be written as a matrix eigenvalue problem.

\begin{equation}
 \left[\begin{array}{ccccccccc}
    -4   &  1  &    0  &   1  &      &      &      &      &     \\
     1   & -4  &    1  &   0  &   1  &      &      &      &     \\
     0   &  1  &   -4  &   0  &   0  &   1  &      &      &     \\
     1   &  0  &    0  &  -4  &   1  &   0  &   1  &      &     \\
         &  1  &    0  &   1  &  -4  &   1  &   0  &   1  &     \\
         &     &    1  &   0  &   1  &  -4  &   0  &   0  &   1 \\
         &     &       &   1  &   0  &   0  &  -4  &   1  &   0 \\
         &     &       &      &   1  &   0  &   1  &  -4  &   1 \\
         &     &       &      &      &   1  &   0  &   1  &  -4
		\end{array}	\right]
    \left[\begin{array}{c}
     u_{1,1}    \\
     u_{2,1}    \\
     u_{3,1}    \\
     u_{1,2}    \\
     u_{2,2}    \\
     u_{3,2}    \\
     u_{1,3}    \\
     u_{2,3}    \\
     u_{3,3}   
	\end{array}\right]		
	=
	\lambda
	  \left[\begin{array}{c}
     u_{1,1}    \\
     u_{2,1}    \\
     u_{3,1}    \\
     u_{1,2}    \\
     u_{2,2}    \\
     u_{3,2}    \\
     u_{1,3}    \\
     u_{2,3}    \\
     u_{3,3}   
	\end{array}\right]
\end{equation}
%
This is a sparse symmeteric matrix which is almost tridiagonal. In reality we would choose larger integration points and the number of entries in the matrix grows with $n^4$. And lot of those entries are zero.

Since the $x$-coordinates are separable, i.e.

\begin{align}
  \hh{H}(x,y) &= \hh{H}_X(x) + \hh{H}_Y(y) \\ \Rightarrow E &= E_X + E_Y
\end{align}
%
We do not need to discretize in two dimentions, since we already know the eigenvalues for $\hh{H}_Y$, they are the same as for the one-dimentional case, see section \ref{onebodyharmonicoscillatorpotential}. 
The laplacian in one dimention is (Eq. \ref{discretize})

\begin{equation}
  \frac{d^2}{dx^2} = \frac{u_{i-1} - 2u_i + u_{i+1} }{h^2} + O(h^2)
  \label{onedimdiscretized}
\end{equation}
%
Where $h$ is our step defined as 
\begin{equation}
h = \frac{x_{max}-x_{max}}{N}
  \label{step}
\end{equation}
%
Where $N$ is number of steps or gridpoints. Inserting Eq. \ref{onedimdiscretized} into the one-dimentional Schr\"{o}dinger equation
%
\begin{equation}
-\frac{u_{i+1} - 2u_{i} + u_{i-1}}{h^2} + V_i u_i = E_X u_i
\end{equation}
%
This can be written as an tridiagonal matrix eigenvalue equation.
%
\begin{equation}
  d_iu_i + e_{i-1}u_{i-1} + e_{i+1}u_{i+1} = \lambda u_i
  \label{onedim discrete}
\end{equation}
%
Where
%
\begin{align}
  e_i  = -\frac{1}{h^2}, \qquad  d_i  = \frac{1}{h^2} + V_i
\end{align}

\begin{equation}
 \left[\begin{array}{ccccccc}
    d_1   &  e_1   &  0  &   0  &  \cdots  &   0  &   0\\
    e_1   & d_2  &  e_2  &   0  &  \cdots  &   0   &  0 \\
     0    & e_2  &  d_3  &  e_3  &  \cdots  &   0   &  0 \\
 \vdots   & \vdots  &  \ddots  &   \ddots  &  \ddots  &   0   &  0 \\
     0    & 0  &  0  &   0  &  \cdots  &   d_{N-2}   &  e_{N-1}  \\		
     0    & 0  &  0  &   0  &  \cdots  &   e_{N-1}   &  d_N	
		\end{array}	\right]
    \left[\begin{array}{c}
     u_{1}    \\
     u_{2}    \\
     u_{3}    \\
    \vdots    \\	
     u_{N-1}   
	\end{array}\right]		
	=
	E_X
	  \left[\begin{array}{c}
    u_{1}    \\
     u_{2}    \\
     u_{3}    \\
    \vdots    \\	
     u_{N-1}   
	\end{array}\right]
\label{tridiagonal}
\end{equation}


 
\chapter{Quantum Mechanics for Many-Body Systems}
In this chapter, we introduce the notation that is common in many-body physics. We will also introduce the formalism of second-quantization and Wick's theorem.

\section{Introduction}
The underlying fundamentals of quantum mechanics as we know today have not been changed much since its birth. We still need to know the Hamiltonian and solve the Schr\"{o}dinger equation. But how we use this theory is a different story. In the real world we have deal with systems of more than one quantum particle, we therefore need to expand our Schr\"{o}dinger equation to include those particles, we call it the manybody Schr\"{o}dinger equation. The degrees of freedom increase as our system size get bigger. We are not able to solve our manybody Schr\"{o}dinger equation with conventional techniques, not analytically nor numerically. Another limiting factor is our knowlegde of the interactions, the exact Hamiltonian is not known.

Therefore we have to make some assumptions and use approximations, and here is where the \emph{many-body methods} comes in. The first approximation we use is to the Hartree-Fock method on closed shells, assuming that all our electron interact in the same way. The next method we are going to use is the coupled cluster (CC) method. They are all different in the sense that they have their regions of effectiveness. HF is very fast and gives reasonable result compared to  when we have small systems for very closed shells and the ground state is stable. While the CC method have applications for systems up to 40 electrons, but have problems with convergence and non-variational.


%26.01.2011
%Many of manybody methods we are called \emph{ab initio} methods. That means the calculation start from some fundamental interaction. These methods are not based on any empirical assumptions or parametrizations %[ask Morten] What about when we fit trial wave functions to STO's and parametrize the trial wave function. The whole HF is based on educated guess? empirical guess? some knowelgde about empiri gives us the guess? and the Electron Correlation energy Diagram to show how good HF is compared to other methods see page 32 in Patrick.
\section{The Many-Body Problem}
\label{sec:many-body problem}
Let us assume we have a non-relativistic isolated system of $N$ particles. And assume we can describe the system with a time-independent Hamiltonoperator $\hat{H}$, then we could reduce the problem to solve the time-independent Schr\"{o}dinger equation:

\begin{equation}
 \hat{H}(r_1, r_2,...,r_N)\psi_\lambda(r_1,r_2,...,r_N) = E_\lambda \psi_\lambda(r_1, r_2,...,r_N)
\label{def:manybodyschrodinger}
\end{equation}
%
where the $r_i$ represent particle $i$ with spin $\ket{m_s}$. $\lambda$ denotes the set of quantum numbers for particles 1,...N. 

The many-body wave fuction $\Psi_\lambda$ is a N-body \emph{vector} in the composite Hilbert space:

\begin{equation}
\psi_\lambda \in \mathcal{H}_N := \mathcal{H}_1 \oplus \mathcal{H}_1 \oplus \ldots \oplus \mathcal{H}_1 
  \label{maybodyinhilbert}
\end{equation}
%
Or

\begin{equation}
\ket{\Psi_\lambda} = \ket{\psi_1}  \oplus \ket{\psi_2} \oplus ... \oplus \ket{\psi_N} \equiv \ket{\psi_1\psi_2 ... \psi_N}
 \label{def:productstates}
\end{equation}
%
where $\ket{\psi_i}$ is a state in a single-particle Hilbert space $\mathcal{H}_1$, which is the space of square integrable function over  
$\mathbb{R}^d \oplus (\sigma)$, or formally:

\begin{equation}
\mathcal{H}_1:= L^2(\mathbb{R}^d \oplus (\sigma))
\label{hilbertspacedim}
\end{equation}
%

\subsection{The Electronic Hamiltonian}
We want to describe our physical system with a \emph{ab initio} method, which basically means that we want our Hamiltonian to include the basic forces with no parametrization, in atomic units ($\hbar = c = m_e = 1$) the Hamiltonian is:

\begin{equation}
\hat{H}  = \hat{T} + \hat{V} 
 \label{eq:tplussv}
\end{equation}

Where $\hat{T}$ is the total kinetic energy operator, and $\hat{V}$ is the total potential energy operator.
 
\begin{equation}
\hat{T} = \sum_{k} \hat{t}_k 
 \label{eq:kinetic}
\end{equation}
%
$\hat{t}_k$ is the kinetic energy operator for particle $k$. 

In general we have:

\begin{equation}
 \hat{V} = \hat{V}_1 + \hat{V}_2 + ...  
 \label{eq:potential}
\end{equation}
 %
where 

\begin{equation}
 \hat{V}_n = \frac{1}{n!} \sum_{abc..z} \hat{v}^{(n)}_{abc...z}
\end{equation}
$n!$ is because we have indistignuishable particles. For systems of electrons like quantum dots, we will truncate our total potential operator to include up to the two-body potential operator $\hat{V}_2$. But some papers in nuclear physics (ref(papers of three body force)) have proven that the three-body force is a important contributor to the binding energy????. Then the electronic Hamiltonian the reads (in atomic units)

\begin{equation}
\hat{H} = \sum_{k} \hat{t}_k + \frac{1}{2} \sum_{ij} \hat{v}_{ij} 
 \label{def:electronic hamiltonian}
\end{equation}

where (in atomic units)

\begin{equation}
\hat{h}_k = -\frac{1}{2}\nabla^2_k + \sum_{A} \hat{c}_{kA}, \qquad \hat{v}_{ij} = \frac{1}{r_{ij}}
 \label{def:electronic hamiltonian parts}
\end{equation}
%
Where the last sum is the coulomb contribution of interaction of the single-particles with the \emph{core} particles $A$, e.g. electrons around a proton core. 


%% Need more text or else underfull \vbox












\subsection{Identical Particles}
In quantum mechanics particles are indistinguishable, we could not tell which of the electrons is in which state. This means that the expectation value would have to be the same when we interchange the coordinates of particle $i$ and $j$. 

\begin{equation}
|\Psi_{\lambda}(r_1,r_2,..,r_i,..,r_j,..,r_N)|^2 = |\Psi_{\lambda}(r_1,r_2,..,r_j,..,r_i,..,r_N)|^2
 \label{eq:interchange}
\end{equation}
%
Gives us possible antisymmetric ($-$) and symmetric ($+$) solutions

\begin{equation}
\Psi_{\lambda}(r_1,r_2,..,r_i,..,r_j,..,r_N) = \pm \Psi_{\lambda}(r_1,r_2,..,r_j,..,r_i,..,r_N)
\end{equation}
%
We will later refer to the symmetric solution as bosons, and the other as fermions.
Introducing the Permutation operator 

\begin{equation}
\hat{P}_{ij}\Psi_\lambda(r_1,r_2,..,r_i,..,r_j,..,r_N) = \Psi_\lambda(r_1,r_2,..,r_j,..,r_i,..,r_N) 
 \label{eq:poperator}
\end{equation}
%
The eigenvalue equation for $\hat{P}$ gives 

\begin{equation}
 \hat{P}_{ij}\Psi_\lambda(r_1,r_2,..,r_i,..,r_j,..,r_N) = \beta \Psi_\lambda(r_1,r_2,..,r_i,..,r_j,..,r_N) 
 \label{eq:poperatoreq}
\end{equation}

\begin{equation}
\beta = \pm 1 \qquad \text{Since} \qquad \hat{P}^2_{ij} = 1
 \label{eq:beta} 
\end{equation}
%
wavefunctions with $(\beta = +1)$ are the bosons, and $(\beta = -1)$ are the fermions. The Hamiltonian is invariant under the interchange of particles and therefore commutes with the permutation operator.

\begin{proof}
\begin{equation}
\hat{P}_ {jk} \hat{H} \Psi = \hat{P}_{jk}(\hat{H}_{jk}\Psi_{jk}) =  \hat{H}_{kj}\Psi_{kj} 
\label{eq:overst}
\end{equation}
%
\begin{equation}
\hat{H} \hat{P}_{jk} \Psi = \hat{H}_{jk}\hat{P}_{jk}\Psi_{jk} = \hat{H}_{jk} \Psi_{kj}
\label{eq:nederst}
\end{equation}
%
We then subtract equation (\ref{eq:overst}) with (\ref{eq:nederst}) 
%
\begin{equation}
 \hat{P}_{jk}(\hat{H}\Psi) - \hat{H}(\hat{P}_{jk}\Psi) = \hat{H}_{kj}\Psi_{kj} - \hat{H}_{jk} \Psi_{kj}
\end{equation}
%
or equivalently
%
\begin{equation}
 \left[\hat{P}_{jk},\hat{H} \right] = \left(\hat{H}_{kj} - \hat{H}_{jk}\right)\Psi_{kj}
\end{equation}
\end{proof}
%
\noindent The permutation operator commutes with the Hamiltonian if and only if $\hat{H}_{kj} = \hat{H}_{jk}$.

According to (\ref{eq:kinetic}) and (\ref{eq:potential}) Hamiltonian is a sum of of onebody and two-body operators. The sums converge and are therefore interchangable with respect to particles without changing our Hamiltonian.

\begin{equation}
H_{jk} = H_1... + H_j + ... + H_k + ... = H_1... + H_k + ... + H_j + ... = H_{kj}
 \label{eq:HkjequalsHjk}
\end{equation}
%
Then it follows that $\hat{H}$ and $\hat{P}$ are compatible observable (\cite{griffiths}[111]). i.e. there exist eigenfunctions for $\hat{H}$ that are also eigenfunctions of $\hat{P}$. We know we need to construct symmetric wavefunctions for the bosons and antisymmetric for the fermions. One way of doing this is using symmetrizer operator for bosons

\begin{equation}
\hat{S} = \frac{1}{N!} \sum_p \hat{P}
 \label{def:symmetrizer}
\end{equation}
%
where $p$ is the called the permutation number and is the \emph{set} of all possible permutations including the empty set. $p = \{\emptyset,[1,2],[1,3],[2,3]\}$ ------[[[Uncertain]]]------ for three particles.
The normalized symmetric state $\Phi_S$ is then given by

\begin{equation}
\Phi_S(r_1,r_2,...,r_N) = \sqrt{\frac{N!}{n_\alpha!n_\beta!...n_\gamma!}} \hat{S} \phi_\alpha(r_1)\phi_\beta(r_2)...\phi_\gamma(r_N)
 \label{def:symmetricwave}
\end{equation}
%
Similary we have the antisymmetrizer operator for fermions

\begin{equation}
\hat{A} = \frac{1}{N!} \sum_p (-1)^p \hat{P}
 \label{def:antisymmetrizer}
\end{equation}
%
and the normalized antisymmetric states

\begin{equation}
\Phi_{AS}(r_1,r_2,...,r_N) = \sqrt{N!} \hat{A} \psi_\alpha(r_1)\psi_\beta(r_2)...\psi_\gamma(r_N)
 \label{def:antisymmetricwave}
\end{equation}
%
or equivalently

\begin{equation}
  \Phi_{\alpha\beta...\gamma}(r_1,r_2,...,r_N) = \frac{1}{\sqrt N!} 
\left|\begin{array}{cccc} 
\phi_\alpha(r_1) & \phi_\beta(r_1) & ... & \phi_\gamma(r_1) \\
\phi_\alpha(r_2) & \phi_\beta(r_2) & ... & \phi_\gamma(r_2) \\
\vdots & \vdots & \vdots & \vdots \\
\phi_\alpha(r_N) & \phi_\beta(r_N) & ... & \phi_\gamma(r_N) \\
\end{array} \right|
\label{def:slaterdeterminant}
\end{equation}
%
This is first introduced by J.C. Slater \cite{slaterdeterminant} in 1929 and popularily called a Slater determinant. It obeys the Pauli Exclusion Principle (PEP): The determinant would be zero if two of the single-particle wavefunctions have the same quantum numbers $\alpha, \beta, ... \gamma$. 


The most general way of writing our wavefunction of the N-fermion system is to have a linear combination of the Slater determinants (\ref{def:slaterdeterminant}).

\begin{equation}
\Psi_\lambda(r_1,r_2,....,r_N) = \sum_{\alpha\beta...\gamma} C_{\alpha\beta...\gamma}^{\lambda}  \Phi_{\alpha\beta...\gamma}(r_1,r_2,...,r_N)
 \label{def:generalslaterdeterminant}
\end{equation}

\section{Second Quantization}
\label{sec:second quantization}
The second-quantization formalism was first introduced by Dirac (1927) and extended to fermion systems by Jordan and Klein (1927) and by Jordan and Wigner (1928) \cite{bartlett}. The formalism of \emph{second quantisation} is just a simplification in the description of many-body system, a reformulation of the original Schr\"{o}dinger equation. The quantum mechanical states are represented by annihilation and creation operators working on the physical vacuum state. 

We will look at fermionic systems, therefore we willl restric the many-particle functions to be antisymmetric and choose the \emph{Slater determinant} (\ref{def:slaterdeterminant}) as our candidate. And introduce the occupancy notation for Slater determinants

\begin{equation}
\Phi_{\alpha_1 \alpha_2...\alpha_N} \equiv \ket{\alpha_1\alpha_2...\alpha_N}
 \label{def:state}
\end{equation}
%
Note: this is not the same as the product states in (\ref{def:productstates}). This is antisymmetrized 

\begin{equation}
\ket{\alpha_1..\alpha_i\alpha_j..\alpha_N} = -\ket{\alpha_1..\alpha_j\alpha_i..\alpha_N}
 \label{def:antisymmetricstate}
\end{equation}
%
And the state ''lies`` in what we called the Fock space,  which is a tensor product space of antisymmetric Hilbert spaces:

\begin{equation}
\mathcal{F}_N = \bigoplus_{n=0}^N \mathcal{H}_n^{AS}
 \label{def:fockspace}
\end{equation}
%
In this case we have a state in a $N$-dimentional Fock space. 

\subsection{Creation and Annihilation Operators}
The creation and annihilation operators are mappings between different $N$ and $N\pm1$ dimentional Hilbert spaces,

\begin{equation}
a_{\alpha}^{\dagger}: \mathcal{H}_{N}^{AS} \rightarrow \mathcal{H}_{N+1}^{AS} 
 \label{def:creation}
\end{equation}

\begin{equation}
a_{\alpha}: \mathcal{H}_N^{AS} \rightarrow \mathcal{H}_{N-1}^{AS}
 \label{def:annihilation}
\end{equation}
%
where 

\begin{equation}
 \alpha \in \mathcal{H}_1
\end{equation}
%
A creation operator $a_{\alpha}^{\dagger}$ will create a fermion with quantum number(s) $\alpha$ from the antisymmetric state (\ref{def:state})
 

\begin{equation}
a_{\alpha}^{\dagger} \ket{0} = \ket{\alpha}
 \label{ex:creation}
\end{equation}
%
$\ket{0}$ is the vacuum state. If $\alpha$ is already occupied, the result is zero due to PEP. 

\begin{equation}
a_{\alpha}^{\dagger} \ket{\alpha} = 0
 \label{ex:creationpep}
\end{equation}
%
An annihilation operator $a_{\alpha}$ will remove a fermion with quantum number(s) $\alpha$ from the antisymmetric state (\ref{def:state})

\begin{equation}
a_{\alpha}\ket{\alpha} = \ket{0}
 \label{ex:annihilation}
\end{equation}

\begin{equation}
a_{\alpha}\ket{0} = 0
 \label{ex:annihilationvacuum}
\end{equation}
%
If $\alpha$ does not exist, the result is zero due to annihilation of a vacuum state. 

\begin{equation}
a_{\alpha} \underbrace{\ket{\alpha_1\alpha_2...\alpha_N}}_{\alpha \notin} = 0
 \label{ex:annihilationpep}
\end{equation}
%
Our Slater determinant (\ref{def:state}) can now be written as a product of creation operators

\begin{equation}
\ket{\alpha_1\alpha_2...\alpha_N} = \prod_{i=1}^N a_{\alpha_i}^\dagger \ket{0}
 \label{eq:slatercreationproduct}
\end{equation}
%
Using the antisymmetry of the states (\ref{def:antisymmetricstate}) we can show that

\begin{equation}
a_{\alpha_i}^\dagger a_{\alpha_k}^\dagger = - a_{\alpha_k}^\dagger a_{\alpha_i}^\dagger 
 \label{eq:creationcommute}
\end{equation}
%
leading to the anticommutation rule for creation operators

\begin{equation}
\{a_{\alpha}^\dagger ,a_{\beta}^\dagger \} = a_{\alpha}^\dagger a_{\beta}^\dagger  + a_{\beta}^\dagger a_{\alpha}^\dagger = 0
 \label{eq:creationanticommute}
\end{equation}
%
Note: If $\alpha = \beta$ we would also get zero because of PEP. The hermitian conjugate (adjoint) of $a_{\alpha}^\dagger$ is the annihilation operator,

\begin{equation}
\left(a_{\alpha}^\dagger \right)^\dagger = a_\alpha
 \label{def:creationadjoint]}
\end{equation}
%
We have the following anticommutation relation for the annihilation operators (see \cite{bartlett} for details)

\begin{equation}
\{a_\alpha, a_\beta \} = a_\alpha a_\beta + a_\beta a_\alpha = 0
 \label{eq:annihilationanticommute}
\end{equation}
%
and 

\begin{equation}
\{a_{\alpha}^\dagger, a_\beta \} = \{a_{\alpha}, a_\beta^\dagger \}  = \delta_{\alpha \beta}
 \label{eq:creationannihilationanticommute}
\end{equation}
%
where $\delta_{\alpha \beta}$ is $0$ if $\alpha \neq \beta$ and $1$ if $\alpha = \beta$.

\subsection{Representation of Operators}
Now that we have formalism for our states, we want to calculate matrix elements and expectation values of a \emph{many-body} operators.

Starting with the number-operator. It is a way to test our many-body formalism conserve the number of particles. 

\begin{equation}
\hat{N} = \sum_\alpha a_{\alpha}^\dagger a_{\alpha} 
 \label{def:numberoperator}
\end{equation}
%
And operating this on a state gives us the eigenvalue of $n$, which is the number of fermions in that state.

\begin{equation}
\hat{N} \ket{\alpha_1 \alpha_2... \alpha_N} = \sum_\alpha a_{\alpha}^\dagger a_{\alpha} \ket{\alpha_1 \alpha_2... \alpha_N} = n  \ket{\alpha_1 \alpha_2... \alpha_N} 
 \label{def:numberoperatorstate}
\end{equation}

Because from  (\ref{ex:creation}),(\ref{ex:annihilationpep}) and (\ref{ex:annihilationpep}) we get 

\begin{equation}
 a_{\alpha}^\dagger a_{\alpha} \ket{\alpha_1 \alpha_2... \alpha_N}= \left\{ \begin{array}{ll}
 0 & \mbox{$\alpha \notin \{\alpha_i\}$} \\
  \ket{\alpha_1 \alpha_2... \alpha_N} & \mbox{$\alpha \in \{\alpha_i\}$}
       \end{array} \right.
 \label{def:becausenumberoperators}
\end{equation}
% 
The number operator is a one-body operator since it acts on one single-particle state at a time. Another type is (what kind of type?) A general one-body operator? (For details see...) \cite{bartlett} mentions this to be a symmetric.

\begin{equation}
\hat{F} = \sum_{\alpha \beta} \bra{\alpha} \hat{f} \ket{\beta} \ket{\alpha}\bra{\beta}
 \label{def:onebodyoperator}
\end{equation}
%
where $\ket{\alpha},\ket{\beta}$ is the chosen single-particle basis. It can be rewritten and expressed with creation and annihilation operators (see Appendix derivation). The second quantized form of $\hat{F}$

\begin{equation}
\hat{F} = \sum_{\alpha \beta} \bra{\alpha} \hat{f} \ket{\beta} a_{\alpha}^\dagger a_{\beta}
 \label{def:rewrittenonebodyoperator}
\end{equation}


%One spesial case is the one-body operator for the kinetic energy plus an external one-body %potential.

%\begin{equation}
%\hat{H}_0 = \sum_{\alpha \beta} \bra{\alpha} \hat{h}_0 \ket{\beta} a_{\alpha}^\dagger a_{\beta}
% \label{def:spesialonebodyoperator}
%\end{equation}

%where $\hat{h}_0 = \hat{t} + \hat{u}$. 

The operator $\hat{F}$ removes a fermion from the state $\beta$ and creates a new one in state $\alpha$. This transition is given by the probability amplitude $\bra{\alpha}\hat{f}\ket{\beta}$.

Generally we can do this for a N-body operator. But we will just consider a two-body operator at the most

\begin{equation}
\hat{V} = \sum_{\alpha \beta \gamma \delta} \bra{\alpha \beta} v \ket{\gamma \delta}   \ket{\gamma \delta} \bra{\alpha \beta}
 \label{def:twobodyoperator}
\end{equation}
%
For a $N$-particle system we have

\begin{equation}
V_N = \sum_{i<j=1}^N \hat{v}_{ij} = \frac{1}{2} \sum_{i\neq j}^N \hat{v}_{ij}
 \label{def:N-twobodyoperator}
\end{equation} 
%
This can be used to rewrite $\hat{V}$ to

\begin{align}
\hat{V} &= \frac{1}{2} \sum_{\alpha\beta\gamma\delta}\bra{\alpha \beta} v \ket{\gamma \delta} a_{\alpha}^\dagger a_{\beta}^\dagger a_{\delta} a_{\gamma} \\
        &= \frac{1}{4} \sum_{\alpha\beta\gamma\delta} \bra{\alpha \beta}|v|\ket{\gamma \delta} \, a_{\alpha}^\dagger a_{\beta}^\dagger a_{\delta} a_{\gamma} 
 \label{eq:secondquantized}
\end{align}
%
Where we have defined the antisymmetric matrix element to be.
\begin{align}
\bra{\alpha \beta} |v| \ket{\gamma \delta} = \bra{\alpha \beta} v \ket{\gamma \delta} - \bra{\alpha \beta} v \ket{\delta\gamma} 
 \label{def:antisymmetrizedmatrix}
\end{align}
%Note that if alpha = beta, gamma, delta we would automatically get 0, no need to have the condition alpha != beta like in the def:N-twobodyoperator. That is restriction of the summation index are not needed because of PEP
see \cite{manybodyjensen} and \cite{dick} for details of this derivation. The interpretation of the operator $\hat{V}$ is that it removes two fermions in the states $\gamma$ and $\delta$, and creates two others in states $\alpha$,$\beta$. This is done with probability amplitude  $\frac{1}{4}\bra{\alpha \beta} v \ket{\gamma \delta}_{\text{AS}}$. 
%
But the interesting here is to calculate expectation values of the operator (\ref{eq:secondquantized}). Let us find the expectation value of $\hat{V}$ with respect to the two-particle product states $\ket{\alpha_1 \alpha_2}$ and $\ket{\beta_1}{\beta_2}$ 

\begin{align}
\bra{\alpha_1\alpha_2}\hat{V}\ket{\beta_1\beta_2} &= \frac{1}{4} \sum_{\alpha\beta\gamma\delta} 
 \bra{\alpha \beta} |v| \ket{\gamma \delta}\bra{\alpha_1 \alpha_2} a_{\alpha}^\dagger a_{\beta}^\dagger a_{\delta} a_{\gamma} \ket{\beta_1\beta_2} \\ 
  & = \frac{1}{4} \sum_{\alpha\beta\gamma\delta} \bra{\alpha \beta} |v| \ket{\gamma \delta}\bra{0} a_{\alpha_1} a_{\alpha_2} a_{\alpha}^\dagger a_{\beta}^\dagger a_{\delta} a_{\gamma} a_{\beta_1}^\dagger a_{\beta_2}^\dagger \ket{0}
\end{align}
%Derivate the expectation value
Using the anticommutation relations (\ref{eq:creationanticommute}),(\ref{eq:annihilationanticommute}) and (\ref{eq:creationannihilationanticommute}), we get

\begin{align}
 &\bra{0} a_{\alpha_1} a_{\alpha_2} a_{\alpha}^\dagger a_{\beta}^\dagger a_{\delta} a_{\gamma} a_{\beta_1}^\dagger a_{\beta_2}^\dagger \ket{0} \nonumber \\
&= \bra{0} a_{\alpha_1} a_{\alpha_2} a_{\alpha}^\dagger a_{\beta}^\dagger \left( a_{\delta} \delta_{\gamma\beta_1} a_{\beta_2}^{\dagger} -  a_\delta a_{\beta_1}^\dagger a_\gamma a_{\beta_2}^\dagger \right) \ket{0} \\ 
&= \bra{0} a_{\alpha_1} a_{\alpha_2} a_{\alpha}^\dagger a_{\beta}^\dagger \left( \delta_{\gamma\beta_1}\delta_{\delta \beta_2} - \delta_{\gamma \beta_1} a_{\beta_2}^{\dagger} a_{\delta} - a_{\delta} a_{\beta_1}^{\dagger} \delta_{\gamma \beta_2} + a_\delta a_{\beta_1}^{\dagger}a_{\beta_2}^{\dagger} a_\gamma \right) \ket{0}\\
&=\bra{0} a_{\alpha_1} a_{\alpha_2} a_{\alpha}^\dagger a_{\beta}^\dagger \left(\delta_{\gamma\beta_1}\delta_{\delta \beta_2} - \delta_{\gamma \beta_1} a_{\beta_2}^{\dagger} a_{\delta} - \delta_{\delta\beta_1}\delta_{\gamma\beta_2} \delta_{\gamma \beta_2} + \delta_{\gamma\beta_2}a_{\beta_1}^{\dagger}a_\delta + a_\delta a_{\beta_1}^{\dagger}a_{\beta_2}^{\dagger} a_\gamma \right) \ket{0}
\end{align}
% 
The only terms that survive is the terms with only kronecker deltas, because all the other terms have an annihilation operator to the left which gives zero with the vacuum state (\ref{ex:annihilationvacuum})

\begin{equation}
\bra{0} a_{\alpha_1} a_{\alpha_2} a_{\alpha}^\dagger a_{\beta}^\dagger a_{\delta} a_{\gamma} a_{\beta_1}^\dagger a_{\beta_2}^\dagger \ket{0} = \left(\delta_{\gamma\beta_1}\delta_{\delta\beta_2} - \delta_{\delta\beta_1}\delta_{\gamma\beta_2}\right)\bra{0} a_{\alpha_2} a_{\alpha_1} a^{\dagger}_{\alpha} a^{\dagger}_{\beta} \ket{0}
\label{eq:vacuumexpectation}
\end{equation}
%
Similary we can rewrite 

\begin{equation}
\bra{0} a_{\alpha_2}a_{\alpha_1}a_{\alpha}^{\dagger}a_{\beta}^{\dagger} \ket{0} = \delta_{\alpha \alpha_1}\delta_{\beta \alpha_2} - \delta_{\beta \alpha_1}\delta_{\alpha \alpha_2}
 \label{eq:vacuumexpectation2}
\end{equation}
%
This gives us the following expectation value

\begin{align}
\bra{\alpha_1\alpha_2}\hat{V}\ket{\beta_1\beta_2} &= \frac{1}{2} \left[ \bra{\alpha_1\alpha_2}v\ket{\beta_1\beta_2} - \bra{\alpha_1\alpha_2}v\ket{\beta_2\beta_1} - \bra{\alpha_2\alpha_1}v\ket{\beta_1\beta_2} + \bra{\alpha_2\alpha_1}v\ket{\beta_2\beta_1} \right]\\
 &= \bra{\alpha_1\alpha_2}v\ket{\beta_1\beta_2} - \bra{\alpha_1\alpha_2}v\ket{\beta_2\beta_1}\\
 &= \bra{\alpha_1\alpha_2}v\ket{\beta_1\beta_2}_{\text{AS}} 
 \label{res:twobodyexpectation}
\end{align}
%
As we see, this can be very tedious and unefficient as we have to write out contributions that gives us zero. But we can use Wick's theorem to more easily find those terms that gives us contribution. This will be our next topic. The second-quantized form of electronic Hamiltonian from (\ref{def:electronic hamiltonian},\ref{def:electronic hamiltonian parts}) is then %badness underfull 

\begin{equation}
\hat{H} = \sum_{ij} \bra{i}\hat{h}\ket{j}a_i^{\dagger} a_j+ \frac{1}{4} \sum_{ijkl} \bra{ij}|\hat{v}|\ket{kl} a_i^{\dagger} a_j^{\dagger} a_l a_k
 \label{def:secondquantized hamiltonian}
\end{equation}
%
And its vacuum expectation value:
\begin{equation}
\bra{0}\hat{H}\ket{0} = \sum_{i} \bra{i}h\ket{i} + \frac{1}{4} \sum_{ij} \bra{ij}|v|\ket{ij}
 \label{eq:second-qunatized vacuum expectation value}
\end{equation}


\subsection{Wick's Theorem}
Originally Gian-Carlo Wick established this method (1950) in order to evalute the $S$-matrix in quantum field theory (see for \cite{wick} and \cite{chang} for details). He introduced two concepts \emph{normal ordering} and \emph{contractions}. Normal ordering is just a way to write products of annihilation and creation operators in a systematic manner. 

The operators $\hat{A},\hat{B},\hat{C},...$ represents both creation and annihilation operators. Then the \emph{normal ordering} of the operators $\{\hat{A}\hat{B}\hat{C}...\}$ are defined as the rearrangement such that all of the annihilation operators are to the left of the creation operators, multiplied with a phase factor which is $-1$ for each permutation of nearest neighboor operators.

\begin{equation}
\left \{ \hat{A}\hat{B}...\hat{U}\hat{V} \right \} \equiv (-1)^{p} u^{\dagger}v^{\dagger}w^{\dagger}...cba 
 \label{def:normalorder}
\end{equation}
%
The superscript $p$ denotes the number of permutations needed to bring the original operator product into the normal ordered form. 
\\   %How to get Example \newline without badness???
Example:
\begin{align}
\{a^{\dagger}b\} &= a^{\dagger}b, \qquad \{ab^{\dagger}\} = -b^{\dagger}a, \nonumber \\
\{ab\} &= ab = -ba, \\
\{a^{\dagger}bc^{\dagger}d\} & = a^{\dagger} c^{\dagger} db = c^{\dagger}a^{\dagger} bd = -a^{\dagger} c^{\dagger}bd =  -c^{\dagger} a^{\dagger}db \nonumber
  \label{ex:normalorder}
\end{align}
Note that the normal ordered form of operators is not unique since creation and annihilation operators can permute among themselves. Also note one of the important properties of normal ordered operators is that its vacuum expectation value is zero. 
\begin{equation}
\bra{0}\ql\{\hh{A}\hh{B}...\qr\}\ket{0} = 0
 \label{show:vacuumexpectnormalorder}
\end{equation}
%
Because of 
\begin{equation}
a_\alpha\ket{0} = 0
\label{eq:becauseannihilation} 
\end{equation}
\begin{equation}
\bra{0}a^\dagger_\alpha = 0
\label{eq:becausecreation} 
\end{equation}
%
A contraction between two operators are defined as
\begin{equation}
\contraction{}{\hh{A} }{}{\hh{B}}
\hh{A}\hh{B} \equiv \hh{A} \hh{B} - \ql \{ \hh{A} \hh{B} \qr \}
 \label{def:contraction}
\end{equation}
%
And we have only four possibile contractions
\begin{align}
\contraction{}{\hat{a}}{{}_{\alpha}^{\dagger}}{a}
a_{\alpha}^{\dagger}a_{\beta}^{\dagger} &= a_{\alpha}^{\dagger}a_{\beta}^{\dagger} - a_{\alpha}^{\dagger}a_{\beta}^{\dagger} = 0 \\
\contraction{}{a}{{}_{\alpha}}{a}
a_{\alpha}a_{\beta} &= a_{\alpha}a_{\beta} = 0 \\
\contraction{}{\hat{a}}{{}_{\alpha}^{\dagger}}{a}
a_{\alpha}^{\dagger}a_{\beta} &= a_{\alpha}^{\dagger}a_{\beta} - a_{\alpha}^{\dagger}a_{\beta} = 0 \\
\contraction{}{\hat{a}}{{}_{\alpha}^{\dagger}}{a}
a_{\alpha}a_{\beta}^{\dagger} &= a_{\alpha}a_{\beta}^{\dagger} - ( - a_{\beta}^{\dagger}a_{\beta}^{\dagger} ) = \delta_{\alpha\beta} \qquad \text{from (\ref{eq:creationannihilationanticommute})}
 \label{eq:contractionpossibilities}
\end{align}
%
We can have contractions between operators inside a normal ordered product,
\begin{equation}
\contraction{\ql \{ \qr. \hh{A}\hh{B}\hh{C},.}{\hh{P}}{}{\hh{Q}} 
\contraction{\ql \{ \qr. \hh{A}\hh{B}\hh{C}...\hh{P}\hh{Q},.}{\hh{X}}{}{\hh{Y}} 
\contraction{\ql \{ \hh{A}\hh{B}\hh{C}...\hh{P}\hh{Q}...\hh{X}\hh{Y}...\qr \} = (-1)^{p} }{\hh{P}}{}{\hh{Q}} 
\contraction{\ql \{ \hh{A}\hh{B}\hh{C}...\hh{P}\hh{Q}...\hh{X}\hh{Y}...\qr \} = (-1)^{p} \hh{P}\hh{Q}}{\hh{X}}{}{\hh{Y}} 
\ql \{ \hh{A}\hh{B}\hh{C}...\hh{P}\hh{Q}...\hh{X}\hh{Y}...\qr \} = (-1)^{p} \hh{P}\hh{Q} \hh{X}\hh{Y} \ql \{ \hh{A}\hh{B}\hh{C}...\qr \}
 \label{def:contrationinsidenormalorder}
\end{equation}
%
Wick's theorem states that we can express any product of creation and annihilation operators as sum of normal ordered products with all possible ways of contractions, i.e.

\begin{align}
\hh{A}\hh{B}\hh{C}\hh{D}...\hh{V}\hh{X}\hh{Y}\hh{Z} &= \ql\{ \hh{A}\hh{B}\hh{C}\hh{D}...\hh{V}\hh{X}\hh{Y}\hh{Z} \qr \}  \nonumber\\
\contraction{ + \sum_{(1)} \ql\{ \qr.}{\hh{A}}{}{..}
& + \sum_{(1)} \ql\{ \hh{A}\hh{B}\hh{C}\hh{D}...\hh{V}\hh{X}\hh{Y}\hh{Z} \qr \} \nonumber\\
\contraction{ + \sum_{(1)} \ql\{ \qr.}{\hh{A}}{,.}{\hh{C}}  %A little cheating whith ,.
\contraction[2ex]{ + \sum_{(1)} \ql\{ \qr.\hh{A}}{\hh{B}}{,.}{\hh{D}}  %A little cheating whith ,.
& + \sum_{(2)} \ql\{ \hh{A}\hh{B}\hh{C}\hh{D}...\hh{V}\hh{X}\hh{Y}\hh{Z} \qr \} \nonumber\\
&+...\nonumber\\
\contraction{ + \sum_{(N/2)} \ql\{ \qr.}{\hh{A}}{,.}{\hh{C}}  %A little cheating whith ,.
\contraction[2ex]{ + \sum_{(N/2)} \ql\{ \qr.\hh{A}}{\hh{B}}{,.}{\hh{D}}  %A little cheating whith ,.
\contraction{ + \sum_{(N/2)} \ql\{ \qr.\hh{A}\hh{B}\hh{C}\hh{D}...}{\hh{V}}{.,}{\hh{Y}}
\contraction[2ex]{ + \sum_{(N/2)} \ql\{ \qr.\hh{A}\hh{B}\hh{C}\hh{D}...\hh{V}}{\hh{X}}{,.}{\hh{Z}}
& + \sum_{(N/2)} \ql \{ \hh{A}\hh{B}\hh{C}\hh{D}...\hh{V}\hh{X}\hh{Y}\hh{Z} \qr \} 
\label{def:wicksteorem}
\end{align}

$\sum_{(m)}$ means sum over all terms with $m$ number of contractions. $N$ is the total number of creation and annihilation operators. If there are different numbers of creation and annihilation operators, the vacuum expectation value would be zero, because of Eq. (\ref{eq:becauseannihilation}) and Eq. (\ref{eq:becausecreation}). If $N$ odd one of the operators would not be contracted and we would get zero aswell. In order to get contribution one must contract all of the operators. For details of the proof see \cite{peskin} or \cite{bartlett}. 

The generalized Wick's theorem follows directly from Wick's theorem and states that the normal ordered product of operators strings $\{...\}$ are the same as the sum of normal ordered product of the total group with all possible ways of contractions, i.e. 

\begin{align}
\ql\{\hh{A}\hh{B}\hh{C}\hh{D}..\qr \} \ql \{\hh{V}\hh{X}\hh{Y}\hh{Z}..\qr \} & = \ql \{ \hh{A}\hh{B}\hh{C}\hh{D}..\hh{V}\hh{X}\hh{Y}\hh{Z}\qr\} \nonumber \\
\contraction{+\sum_{(1)} \ql \{ \qr .}{\hh{A}}{\hh{B}\hh{C}\hh{D},}{\hh{V}} 
& +\sum_{(1)} \ql \{ \hh{A}\hh{B}\hh{C}\hh{D}..\hh{V}\hh{X}\hh{Y}\hh{Z}\qr\} \nonumber \\
\contraction{+\sum_{(2)} \ql \{ \qr .}{\hh{A}}{\hh{B}\hh{C}\hh{D},}{\hh{V}} 
\contraction[2ex]{+\sum_{(2)} \ql \{ \qr . \hh{A}}{\hh{B}}{\hh{C}\hh{D},\hh{V}}{\hh{X}} 
& +\sum_{(2)} \ql \{ \hh{A}\hh{B}\hh{C}\hh{D}..\hh{V}\hh{X}\hh{Y}\hh{Z}\qr\} \nonumber \\
& + ... \nonumber \\
\contraction{+\sum_{(N/2)} \ql \{ \qr .}{\hh{A}}{\hh{B}\hh{C}\hh{D}.}{V}
\contraction[2ex]{+\sum_{(N/2)} \ql \{ \qr .\hh{A}}{\hh{B}}{\hh{C}\hh{D}..\hh{V}}{\hh{X}}
\contraction[3ex]{+\sum_{(N/2)} \ql \{ \qr .\hh{A}\hh{B}}{\hh{C}}{\hh{D}..\hh{V}\hh{X}}{Y}
\contraction[4ex]{+\sum_{(N/2)} \ql \{ \qr .\hh{A}\hh{B}\hh{C} }{\hh{D}}{..\hh{V}\hh{X}\hh{Y}}{Z}
& +\sum_{(N/2)} \ql \{ \hh{A}\hh{B}\hh{C}\hh{D}..\hh{V}\hh{X}\hh{Y}\hh{Z}\qr\} \nonumber \\
 \label{def:generalizednormalorder}
\end{align}
%
Note that as before the only contribution to the vacuum expectation value comes from the full contractions, only the last sum will give contribution. There are no \emph{internal}  contractions, i.e. contraction between pair of operators inside each operator string $\{..\}$. 

As an example, we can now use Wick's theorem to find the vacuum expectation value of the following products:

\begin{align}
\contraction{\bra{0} a_{i}a_{j}^{\dagger} \ket{0} .;;  \ql \{ \qr.}{a}{{}_{i}}{a}
\bra{0} a_{i}a_{j}^{\dagger} \ket{0} &=   \ql \{ a_{i}a_{j}^{\dagger} \qr \} =\delta_{ij}  \label{eq:vacuumexpectation2redo one-body} \\
%
\contraction{\bra{0} a_{\alpha_2}a_{\alpha_1}a_{\alpha}^{\dagger}a_{\beta}^{\dagger} \ket{0} = \ql \{ \qr.}{a}{{}_{\alpha_2}a_{\alpha_1} }{a}
%
\contraction[2ex]{\bra{0} a_{\alpha_2}a_{\alpha_1}a_{\alpha}^{\dagger}a_{\beta}^{\dagger} \ket{0} = \ql \{ a_{\alpha_2} \qr.}{a}{{}_{\alpha_1}a_{\alpha}^{\dagger}}{a}
%
\contraction{\bra{0} a_{\alpha_2}a_{\alpha_1}a_{\alpha}^{\dagger}a_{\beta}^{\dagger} \ket{0} = \ql \{a_{\alpha_2}a_{\alpha_1}a_{\alpha}^{\dagger}a_{\beta}^{\dagger} \qr \} + \ql \{ \qr. a_{\alpha_2}}{a}{{}_{\alpha.}}{a}
%
\contraction[2ex]{\bra{0} a_{\alpha_2}a_{\alpha_1}a_{\alpha}^{\dagger}a_{\beta}^{\dagger} \ket{0} = \ql 
\{a_{\alpha_2}a_{\alpha_1}a_{\alpha}^{\dagger}a_{\beta}^{\dagger} \qr \} + \ql \{ \qr. }{a}{{}_{\alpha_2}a_{\alpha_1}a^{\dagger}_{\alpha}}{a}
%
\contraction{\bra{0} a_{\alpha_2}a_{\alpha_1}a_{\alpha}^{\dagger}a_{\beta}^{\dagger} \ket{0} = \ql \{a_{\alpha_2}a_{\alpha_1}a_{\alpha}^{\dagger}a_{\beta}^{\dagger} \qr \} + \ql \{a_{\alpha_2}a_{\alpha_1}a_{\alpha}^{\dagger}a_{\beta}^{\dagger} \qr \} 
+ ,,.}{a}{{}_{\alpha_1}}{a}
%
\contraction{\bra{0} a_{\alpha_2}a_{\alpha_1}a_{\alpha}^{\dagger}a_{\beta}^{\dagger} \ket{0} = \ql \{ a_{\alpha_2}a_{\alpha_1}a_{\alpha}^{\dagger}a_{\beta}^{\dagger} \qr \} + \ql \{a_{\alpha_2}a_{\alpha_1}a_{\alpha}^{\dagger}a_{\beta}^{\dagger} \qr \} 
+ \ql \{ \qr. a_{\alpha_2}a_{\alpha.}.}{a}{{}^{\dagger}_{\alpha}}{a}
%
\bra{0} a_{\alpha_2}a_{\alpha_1}a_{\alpha}^{\dagger}a_{\beta}^{\dagger} \ket{0}&= \ql \{a_{\alpha_2}a_{\alpha_1}a_{\alpha}^{\dagger}a_{\beta}^{\dagger} \qr \} + \ql \{a_{\alpha_2}a_{\alpha_1}a_{\alpha}^{\dagger}a_{\beta}^{\dagger} \qr \} 
+ \ql \{\underbrace{a_{\alpha_2}a_{\alpha_1}a_{\alpha}^{\dagger}a_{\beta}^{\dagger}}_{=0} \qr \}  \nonumber \\
%
& = \delta_{\alpha \alpha_1}\delta_{\beta \alpha_2} - \delta_{\beta \alpha_1}\delta_{\alpha \alpha_2}  \label{eq:vacuumexpectation2redo}
\end{align}
%
% And remember that the vacuum expectation value of no fully contracted terms always gives zero 
% \begin{equation}
% \bra{0} \ql \{a_{\alpha_2}a_{\alpha_1}a_{\alpha}^{\dagger}a_{\beta}^{\dagger} \qr \} \ket{0} = (-1)^{2+2} \underbrace{\bra{0} a_{\alpha}^{\dagger}}_{=0}a_{\beta}^{\dagger} a_{\alpha_2}\underbrace{a_{\alpha_1}\ket{0}}_{=0} = 0
%  \label{def:vacuum of no fully contracted normal order}
% \end{equation}
%



\subsection{Particle-Hole Formalism}
\label{sec:particle-hole formalism}
One of the advantages with second quantization is that we can easily introduce a reference SD: $\ket{c}$ instead of using the physical vacuum state $\ket{0}$. This reduces the dimentionality of the problem and the new referece state would be defined by a boldfaced zero

\begin{equation}
\ket{\bf 0} \equiv \ket{\Phi_0} = \ket{ijk...n} 
 \label{def:referencestate}
\end{equation}
%
The new reference state $\ket{\bf 0}$ is also refered to as the \emph{Fermi vacuum}, this is now our Fermi level which is the level of our last occupied quantum state, usually the highest occupied orbital ---[[[What do we mean by orbital or spinorbital]]]---. 
\\
In this representation we have hole states in addition to particle states. We will define what the hole-states are after the following example. Assume that we have three states states that are successively filled with $n-1$,$n$ and $n+1$ single-particle states $\alpha_i$

\begin{align}
\ket{\Phi_0} &\equiv \ket{\alpha_1\alpha_2...\alpha_{n}} &\qquad \text{(reference state)} \label{def:reference state} \\
\ket{\Phi_{\alpha_1}} &\equiv \ket{\alpha_2\alpha_3...\alpha_{n}} = a_{\alpha_1} \ket{\Phi_0} &\qquad \text{(creation of a hole)} \label{def:creation of a hole} \\
\ket{\Phi^{\alpha}} &\equiv \ket{\alpha\alpha_1\alpha_2...\alpha_{n}} = a_{\alpha}^{\dagger} \ket{\Phi_0} &\qquad \text{(creation of a particle)}
 \label{def:creation of a particle}
\end{align}
%
And assume that the energies of the single-particle orbitals is such that (see Figure (\ref{fig:particleholerepresentation}))
\begin{equation} \epsilon_{\alpha_{n+1}}>\epsilon_{\alpha_n}>\epsilon_{\alpha_{n-1}}>...\epsilon_{\alpha_{2}}>\epsilon_{\alpha_{1}}
 \label{def:singleparticleenergies} 
\end{equation}
%
Let us then define our Fermi level to be $\alpha_n$, a \emph{hole} is then a state that is below or on (at??) the Fermi level $\alpha_i \leq \alpha_n$. And a particle is a state above $\alpha_i > \alpha_n$. 

When change our reference state from the physical vacuum state $\ket{0}$ to a particle-hole vaccum $\ket{\bf 0}$, we have to introduce new operators aswell. 

\begin{equation}
a_{\alpha} \ket{\bf 0} \neq 0 
 \label{show:neednewoperators}
\end{equation}
%
since $\alpha \in \ket{\bf 0}$, while for the physical vacuum we have $a_\alpha \ket{0} = 0$ for all $\alpha$. The new operators need to have the relation $b_\alpha\ket{\bf 0} = 0$.

The new operators are called \emph{quasi-} annihilation and creation operators, viz

\begin{equation}
b_{\alpha}^{\dagger} = \left\{ \begin{array}{rl}
 a_\alpha^{\dagger}, & \alpha > \alpha_F\\
 a_\alpha, & \alpha \leq \alpha_F
       \end{array} \right.
 \label{def:quasi-creation}
\end{equation} 

\begin{equation}
b_{\alpha} = \left\{ \begin{array}{rl}
 a_\alpha, & \alpha > \alpha_F \\
 a_\alpha^{\dagger}, & \alpha \leq \alpha_F
       \end{array} \right.
 \label{def:quasi-annihilation}
\end{equation}
%
Where $\alpha_F$ is the Fermi level representing the last occupied single-particle orbit of the reference state $\ket{c}$. 
%
And we have the following anticommutation relations
\begin{align}
\ql \{ b_{\alpha}, b_{\beta} \qr \} &= 0  \label{def:quasioperator1}\\
\ql \{ b_{\alpha}^{\dagger}, b_{\beta}^{\dagger} \qr \} &= 0  \label{def:quasioperator2}\\
\ql \{ b_{\alpha}^{\dagger}, b_{\beta}  \qr \} &= \delta_{\alpha \beta}  \label{def:quasioperator3}
\end{align}
%
The reference state is normalized

\begin{equation}
 \braket{c}{c} = 1
\end{equation}
%
A quasi particle state is defined by a state which has one or more particles/holes  added to the reference state $\ket{c}$.
%
\begin{equation}
\ket{abcd...ijkl...pqrs...} \equiv b_a^{\dagger}b_b^{\dagger}b_c^{\dagger}b_d^{\dagger}...b_i^{\dagger}b_j^{\dagger}b_k^{\dagger}b_l^{\dagger}...b_p^{\dagger}b_q^{\dagger}b_r^{\dagger}b_s^{\dagger}\ket{c}
 \label{def:manybody-quasiparticle}
\end{equation}
%
The convenction is that indices $i,j,k,l...$ indicates states which are occupied by holes. Indices $a,b,c,d...$ indicates the states which are occupied by particles. And $p,q,r,s...$ indicate any state. We are going to simplify the notation further

\begin{align}
 b_i^{\dagger} &= a_i           = i         \,\,\,\,\,\qquad \text{(creation of a hole = removing a state below $\alpha_F$)} \nonumber \\
 b_i           &= a_i^{\dagger} = i^\dagger \,\,\qquad \text{(annihilation of a hole = creating a state below $\alpha_F$)} \nonumber \\
 b_a^{\dagger} &= a_a^{\dagger} = a^\dagger \qquad \text{(creation of a particle = adding a state above $\alpha_F$) } \nonumber \\
 b_a           &= a_a           = a         \,\,\qquad \text{(annihilation of a particle = removing a state above $\alpha_F$)}
\label{def:simplyfied notation of quasi-operators}
\end{align}
%
And the following contractions
\begin{align}
\contraction{}{p}{{}^{\dagger}}{q}
\contraction{p^{\dagger}q^{\dagger} = }{p}{}{q}
p^{\dagger}q^{\dagger} &= pq = 0  \label{eq:contraction 1} \\
\contraction{}{i}{{}^{\dagger}}{j}
i^{\dagger}j &= \delta_{ij} \label{eq:contraction 2} \\
\contraction{}{i}{}{j}
ij^{\dagger} &= 0 \label{eq:contraction 3} \\
\contraction{}{\hat{a}}{}{b}
ab^{\dagger} &= \delta_{ab} \label{eq:contraction 4}\\
\contraction{}{\hat{a}}{{}^{\dagger}}{b}
a^{\dagger}b &= 0 \label{eq:contraction 5}
\end{align}

%\begin{equation}
% \text{\scalebox{0.5}{\input{occupation_orbitals2.pdftex_t}}}
%\end{equation}

\begin{figure}
\centering
     \scalebox{0.7}{\input{occupation_orbitals2.pdftex_t}}
  \caption{The deshed line represent our Fermi level}
  \label{fig:particleholerepresentation}
\end{figure}

\section{The Normal-Ordered Hamiltonian}
We want to rewrite the second-quantized form of the electronic Hamiltonian (\ref{def:secondquantized hamiltonian}) 

\begin{equation}
\hat{H} = \sum_{pq} \bra{p}h\ket{q}p^{\dagger}q + \frac{1}{4}\sum_{pqrs} \bra{pq}|v|\ket{rs}p^{\dagger}q^{\dagger}sr 
 \label{def:normal-ordered hamiltonian}
\end{equation}
%
using Wick's theorem, 

\begin{align}
\contraction{p^{\dagger}q.;; \ql \{ p^{\dagger}q \qr \} + \ql \{ \qr.}{p}{{}^{\dagger}}{q}
 p^{\dagger}q &= \ql \{ p^{\dagger}q \qr \} +  \underbrace{\ql \{ p^{\dagger}q \qr \}}_{ p=i,q=j}  = \ql \{ p^{\dagger}q \qr \}  + \delta_{ij}  \label{eq:normalordere one-particle} \\
%
\contraction{p^{\dagger}q^{\dagger}sr = \ql \{ p^{\dagger}q^{\dagger}sr \qr \} ;; \ql \{ \qr.}{p}{{}^\dagger q^\dagger}{s}
%
\contraction{p^{\dagger}q^{\dagger}sr = \ql \{ p^{\dagger}q^{\dagger}sr \qr \} + \ql \{ p^{\dagger}q^{\dagger}sr \qr \} ;; \ql \{ \qr .}{p}{{}^\dagger q^\dagger s}{r}
%
\contraction{p^{\dagger}q^{\dagger}sr = \ql \{ p^{\dagger}q^{\dagger}sr \qr \} + \ql \{ p^{\dagger}q^{\dagger}sr \qr \} + \ql \{ p^{\dagger}q^{\dagger}sr \qr \} .. \ql \{ p^{\dagger} \qr .}{q}{{}^\dagger}{s}
%
\contraction{p^{\dagger}q^{\dagger}sr = \ql \{ p^{\dagger}q^{\dagger}sr \qr \} + \ql \{ p^{\dagger}q^{\dagger}sr \qr \} + \ql \{ p^{\dagger}q^{\dagger}sr \qr \} .. \ql \{ p^{\dagger} \qr .}{q}{{}^\dagger}{s}
%
\contraction{p^{\dagger}q^{\dagger}sr = \ql \{ p^{\dagger}q^{\dagger}sr \qr \} + \ql \{ p^{\dagger}q^{\dagger}sr \qr \} + \ql \{ p^{\dagger}q^{\dagger}sr \qr \} + \ql \{ p^{\dagger}q^{\dagger}sr \qr \}.. \ql \{ p^{\dagger} \qr .}{q}{{}^\dagger s}{r}
%%%%
p^{\dagger}q^{\dagger}sr &= \ql \{ p^{\dagger}q^{\dagger}sr \qr \} + \ql \{ p^{\dagger}q^{\dagger}sr \qr \}  + \ql \{ p^{\dagger}q^{\dagger}sr \qr \}  + \ql \{ p^{\dagger}q^{\dagger}sr \qr \} +  \ql \{ p^{\dagger}q^{\dagger}sr \qr \}  \nonumber \\
%%%%
\contraction{... \ql \{ \qr.}{p}{{}^{\dagger} \da{q}}{s}
\contraction[2ex]{... \ql \{ p^\dagger \qr. }{q}{{}^\dagger s}{r}
%
\contraction[2ex]{... \ql \{ p^{\dagger}q^{\dagger}sr \qr \}  + \ql \{ \qr.}{p}{{}_{\dagger} \da{q}	s}{r}
\contraction{... \ql \{ p^{\dagger}q^{\dagger} sr \qr \} +  \ql \{ p^\dagger \qr. }{q}{.}{S}
%
&+ \ql \{ p^{\dagger}q^{\dagger}sr \qr \}  + \ql \{ p^{\dagger}q^{\dagger}sr \qr \} \nonumber  \\
%%%%
\contraction{ = \ql \{ p^{\dagger}q^{\dagger}sr \qr \} - ,,}{p}{\da{}}{s}
\contraction{ = \ql \{ p^{\dagger}q^{\dagger}sr \qr \} - ,,,,,p^\dagger s \ql \{ q^\dagger r \qr \} + }{p}{\da{}}{r}
\contraction{ = \ql \{ p^{\dagger}q^{\dagger}sr \qr \} - ,,,,,,,...p^\dagger s \ql \{ q^\dagger r \qr \} + p^\dagger r \ql \{ q^\dagger s \qr \} + }{q}{\da{}}{s}
\contraction{ = \ql \{ p^{\dagger}q^{\dagger}sr \qr \} - ,,,,,,,,,,,,p^\dagger s \ql \{ q^\dagger r \qr \} + p^\dagger r \ql \{ q^\dagger s \qr \} + q^\dagger s \ql \{ p^\dagger r \qr \} - }{p}{\da{}}{s}
%%%
& = \ql \{ p^{\dagger}q^{\dagger}sr \qr \} - \underbrace{p^\dagger s}_{p=i,s=j} \ql \{ q^\dagger r \qr \} + \underbrace{p^\dagger r}_{p=i, r=j} \ql \{ q^\dagger s \qr \} + \underbrace{q^\dagger s}_{q=i,s=j} \ql \{ p^\dagger r \qr \} - \underbrace{q^\dagger r}_{q=i,r=j} \ql \{ p^\dagger s \qr \} \nonumber \\
%%%
\contraction{- }{p}{\da{}}{s}
\contraction{.... \da{p}s}{q}{\da{}}{r}
%
\contraction{- \da{p}s\da{q}r + }{p}{\da{}}{r}
\contraction{- \da{p}s\da{q}r + \da{p}r}{q}{\da{}}{s}
& - \da{p}s\da{q}r + \da{p}r\da{q}s \qquad \text{only contribution when $p=i,s=j,q=k,r=l$} \nonumber  \\
%%%
& = \ql \{ p^{\dagger}q^{\dagger}sr \qr \} - \delta_{ij}\ql \{ q^\dagger r \qr \} + \delta_{ij}\ql \{ q^\dagger s \qr \} + \delta_{ij} \ql \{ p^\dagger r \qr \} - \delta_{ij}\ql \{ p^\dagger s \qr \} \nonumber \\
& = \delta_{ij}\delta_{kl} + \delta_{ij}\delta_{kl}
\end{align}
%
Here we have done contractions relative to a reference state $\ket{\bf 0}$ and followed the relations Eq. (\ref{eq:contraction 1}-\ref{eq:contraction 5}). Then the normal-ordered one-body Hamiltonian is

\begin{equation}
\hat{H}_1 = \sum_{pq} \bra{p}h\ket{q} \ql \{ p^{\dagger}q \qr \}  + \sum_{i}\bra{i}h\ket{i}
 \label{def:normal-ordered hamiltonian part 1}
\end{equation}
%
And the two-body 
\begin{align}
\hat{H}_2 &= \frac{1}{4} \sum_{pqrs} \bra{pq}|v|\ket{rs}  \ql \{ p^{\dagger}q^{\dagger}sr \qr \} - \frac{1}{4} \sum_{qri} \bra{iq}|v|\ket{ri}  \ql \{ p^{\dagger} r \qr \}  + \frac{1}{4}  \sum_{qsi} \bra{iq}|v|\ket{is} \ql \{ q^{\dagger}s \qr \} \nonumber \\
%
& + \frac{1}{4} \sum_{pri} \bra{pi}|v|\ket{ri} \ql \{ p^{\dagger} r \qr \} - \frac{1}{4} \sum_{psi} \bra{pi}|v|\ket{is}\ql \{ p^{\dagger} s \qr \} + \frac{1}{4}\sum_{ij}\ql [ \bra{ij}|v|\ket{ij} - \bra{ij}|v|\ket{ji}\qr ] 
 \label{def:normal-ordered hamiltonain part 2}
\end{align}
%
The second term are equal to fourth term. And the third terms is equal to the fifth term.  

\begin{align}
 \hat{H} &= \sum_{pq} \bra{p}h\ket{q} \ql \{ p^{\dagger}q \qr \} + \sum_{i}\bra{i}h\ket{i} + \frac{1}{4} \sum_{pqrs} \bra{pq}|v|\ket{rs} \ql \{ p^{\dagger}q^{\dagger}sr \qr \}   \nonumber \\ 
&+ \frac{1}{2} \sum_{pri} \bra{ip}|v|\ket{ri}\ql \{ p^{\dagger} r \qr  \} - \frac{1}{2} \sum_{psi} \bra{pi}|v|\ket{is}\ql \{ p^{\dagger} s \qr \} + \frac{1}{2} \sum_{ij} \bra{ij}|v|\ket{ij}
\label{def:second-ordered Hamiltonian1}
\end{align}
%
We have changed summation variable $s,r\rightarrow q$

\begin{align}
 \hat{H} &= \sum_{pq} \bra{p}h\ket{q} \ql \{ p^{\dagger}q \qr \} + \sum_{i}\bra{i}h\ket{i} + \frac{1}{4} \sum_{pqrs} \bra{pq}|v|\ket{rs} \ql \{ p^{\dagger}q^{\dagger}sr \qr \}   \nonumber \\ 
& + \sum_{pq} \ql (  \bra{p}h\ket{q} + \sum_{i}\bra{ip}|v|\ket{qi}  \qr ) \ql \{ p^{\dagger} q \qr \} + \frac{1}{2} \sum_{ij} \bra{ij}|v|\ket{ij}
 \label{def:second-ordered Hamiltonian2}
\end{align}
%
We define the following
\begin{align}
f_q^p         &\equiv  \bra{p}h\ket{q} + \sum_{i}\bra{ip}|v|\ket{qi}  \label{def:f_p^q} \\
\hh{F}_N      &\equiv  \sum_{pq} f_q^p  \ql \{ p^{\dagger} q \qr \}    \label{def:F_N}     \\
\hh{V}_N      &\equiv  \frac{1}{4} \sum_{pqrs} \bra{pq}|v|\ket{rs} \ql \{ p^{\dagger}q^{\dagger}sr \qr \}\label{def:V_N}  \\
\hh{H}_N      &\equiv  \hh{F}_N + \hh{V}_N \label{def:H_N} 
\end{align}
%
From Eq. (\ref{def:second-ordered Hamiltonian2}) we see that only the last term survives and expectation value would be
\begin{equation}
\bra{\bf 0}\hh{H}\ket{\bf 0} = \sum_i \bra{i}h\ket{i} + \frac{1}{2} \sum_{ij} \bra{ij}|v|\ket{ij} 
 \label{eq:second-qunatized reference expectation value}
\end{equation}
%
Finally the Hamiltonian reads
\begin{equation}
\hh{H} = \hh{H}_N + \bra{\bf 0}\hh{H}\ket{\bf 0}
 \label{eq:final normal ordered hamiltonian}
\end{equation}
%
And the normal-ordered electronic Hamiltonian 
\begin{equation}
\hh{H}_N = \hh{H} - \bra{\bf 0}\hh{H}\ket{\bf 0}
 \label{eq:normal-ordered hamiltonian}
\end{equation}
%
This is nothing but a shift by a constant for the expectation value. The usefullness of this will be clearer when we introduce the Coupled Cluster Theory.


%%%%%%%%%%%%%%%%%%%%%%%%%%%%%%%%%%%%%%%%%%%%%%%%%%%%%%%%%%%%%%%%%%%%%%%%%%%%%%%%%%%%%%%%
%%%%%%%%%%%%%%%%%%%%%%%%%%%%%%%   PART MANY-BODY METHODS        %%%%%%%%%%%%%%%%%%%%%%%%
%%%%%%%%%%%%%%%%%%%%%%%%%%%%%%%%%%%%%%%%%%%%%%%%%%%%%%%%%%%%%%%%%%%%%%%%%%%%%%%%%%%%%%%%
\part{MANY-BODY METHODS}

\chapter{Hatree-Fock Method}
%In this chapter we shall...................
%Hartree-Fock theory and DFT are simplifications of the manybody problem with moving particles in a potential field. Physical systems such as atoms, molecules and  nuclei. In such systems particles are not only affected by the main potential set up by a \emph{core}, but aswell as the field generated by the other particles. 

%The first approximation is to assume that the kinetic energy of the \emph{core} is negligible compared to the particles surrounding it. Then we just have to solve the Schr\"{o}dinger equation for particles instead of the whole system, particles$+$core. This is the philosophy of \emph{Born-Oppenheimer approximation}.

The exact solution to the Schr\"{o}inger equation cannot be obtained in most of the systems of chemical interest (see \cite{helgaker}). The number of particles involved in those problems are just to big for computers to solve. Therefore approximations are needed. The approximations would of course depend on the physical system. We could just try some potential and interactions that works. But the many-body theories have hierachcal structure when it comes to approximations. The advantages of this is the fact that we build up experience about the given models. And we can do benchmarks and comparison tests. This allows too differentiate between different type of \emph{effects} in the theories. In this chapter we are going to look into one of the first approximations, the Hartree-Fock method.

\section{Introduction}
\label{sec:Hatree-Fock Method:Introduction}
The independent particle picture is an assumption that the electron-electron interaction is rather weak. And each electron could be viewed as independent particles which sees an effective field set up by the other electrons. This is the philosophies of Hartree-Fock theories and Density functional theories. 

The Hartree-Fock method is an optimization problem using Lagrange multipliers as the mathematical tool. The integral we want to minimize are in general

\begin{equation}
E[\Phi] = \int_a^b f(\Phi(r),\frac{\partial \Phi}{\partial r}, r)d^3 r
 \label{def:integral we want to minimize}
\end{equation}
%
We want to find a function $\Phi$ that minimizes the \emph{functional} $E[\Phi]$. The set of functions $V$ have the following conditions: $V = \ql \{ \Phi:[-\infty,\infty] \rightarrow \mathbb{R}:\text{$\Phi$ is continous and differentiable}: \braket{\Phi}{\Phi} = 1 \qr \}$. The unkowns are the functions in $V,\frac{\partial \Phi}{\partial r}$ and $r$. Although the integral limits $a,b$ are defined, the integration path is not. We want to find a path with a given set of unknowns such that $\delta E = 0$. This will give us minima, maxima or saddle points. So we have check if we have a minima after finding the solution.

The quantum mechanical functional we want to minimize is

\begin{equation}
E[\Phi] = \frac{\bra{\Phi}\hat{H}\ket{\Phi}}{\braket{\Phi}{\Phi}} =  \frac{\int \Phi^* \hat{H} \Phi d\tau}{\int \Phi^* \Phi d \tau}
 \label{def:energyfunctional}
\end{equation}
If $\Phi$ is an eigenstate of the $\hat{H}$, then the functional will be stationary
\begin{equation}
\hat{H} \ket{\Phi} = E_0 \ket{\Phi} \Rightarrow E[\Phi] = E_0 \Rightarrow \delta E[\Phi] = 0 
 \label{def:eigenstate then stationary}
\end{equation}
%
Conversely we can show a stationary point is an eigenstates of the Hamiltonian (\ref{def:eigenstate then stationary}), and therefore a solutions of the Schr\"{o}dinger equation. 
%
\begin{proof}
From (\ref{def:energyfunctional}) and the fact that setting $\delta E[\Phi]$ = 0 gives us
\begin{align}
0 &= \delta [ \bra{\Phi}\hat{H}\ket{\Phi} - E[\Phi] \ql (\braket{\Phi}{\Phi} - 1 \qr ) ] \nonumber \\
0 &= \delta [ \bra{\Phi} \ql( \hat{H}-E[\Phi] \qr ) \ket{\Phi}+ E[\Phi] ] \nonumber \\
0 &=  \bra{\delta \Phi} \ql( \hat{H}-E[\Phi] \qr ) \ket{\Phi}+  \bra{\Phi} \ql( \hat{H}-E[\Phi] \qr ) \ket{\delta \Phi}+ \delta E[\Phi] \nonumber \\
0 &=  \bra{\delta \Phi} \ql( \hat{H}-E[\Phi] \qr ) \ket{\Phi}+  \bra{\Phi} \ql( \hat{H}-E[\Phi] \qr ) \ket{\delta \Phi} 
\label{proof:stationary points are eigenvalues}
\end{align}
We can add a phase $\delta \Phi \rightarrow i\delta \Phi$. Adding a phase to a wave function will not change the expectation value. Then we get two se of equations 
\begin{align}
I. \qquad & \bra{\delta \Phi} \ql( \hat{H}-E \qr ) \ket{\Phi}+  \bra{\Phi} \ql( \hat{H}-E \qr ) \ket{\delta \Phi}  = 0\\
II. \qquad& \bra{\delta \Phi} \ql( \hat{H}-E \qr ) \ket{\Phi}-  \bra{\Phi} \ql( \hat{H}-E \qr ) \ket{\delta \Phi} = 0
 \label{proof:arbritrary phase}
\end{align}
We have multiplied with $i$ in (II) and remember that $i \ket{\delta \Phi} = -i\bra{\delta \Phi}$. And adding them together
\begin{equation}
\bra{\delta \Phi} \ql( \hat{H}-E \qr ) \ket{\Phi} = 0
 \label{proof:adding I with II}
\end{equation}
The choice of $\bra{\delta \Phi}$ is arbritrary, so the equation must hold for all possible $\bra{\delta \Phi}$'s. Therefore this can only be satisfied if $\Phi$ is an eigenfunction to $\hat{H}$.
\end{proof}
%
\noindent This is equivalent to Rayleigh-Ritz principle \cite{arfken} that tells us that the functional $\Omega_{RR} = \bra{\Psi}\hat{H}-E\ket{\Psi}$  is stationary at any eigenfunctions $H\ket{\Phi_m} = E_m \ket{\Phi_m}$. 
%
One important feature of the functional (\ref{def:energyfunctional}) is that errors are in the second order in $\delta \Phi$. 
 
\begin{proof}
Our trial wavefunction is $\ket{\phi} = \ket{\psi_0} + \lambda\ket{\psi}$. $\ket{\psi_0}$ is a stationary state, $\ket{\psi}$ is our guess, and $\lambda$ is complex number. Using the relations
\begin{align}
(\hat{H}-E_0) \ket{\psi_0} &= 0 \nonumber \\
(\hat{H}-E_0) [\ket{\phi} - \lambda \ket{\psi} ] &= 0 \nonumber \\
(\hat{H}-E_0) \ket{\phi} &= \lambda (\hat{H}-E_0 ) \ket{\psi} 
 \label{proof:need relations}
\end{align}
%
We get
\begin{align}
E[\Phi] - E_0 & = \frac{\bra{\phi}(\hat{H}-E_0)\ket{\phi}}{\braket{\phi}{\phi}} \nonumber \\ 
&= \frac{\bra{\lambda\psi}(\hat{H}-E_0)\ket{\lambda\psi}}{\braket{\phi}{\phi}} \nonumber \\
&= |\lambda|^2 \frac{\bra{\psi}(\hat{H}-E_0)\ket{\psi}}{\braket{\phi}{\phi}}
\end{align}
\end{proof}

\section{Derivation of the Hartree-Fock equations}
The HF ansatz reads

\begin{equation}
\Phi_{\text{HF}} = \ket{pqrs...}
 \label{def:Hartree Fock ansatz}
\end{equation}
%
Where $pqrs...$ are the HF spin-orbitals. $\Phi_{\text{HF}}$ is a SD with $N$ states written in second quantization formalism. We could vary these directly, giving us the Hartree-Fock-Roothaan method \cite{roothaan}. Another possibility is to expand our spin-orbitals $\ket{a}$ as a linear combination of a finite number of basis states $\ket{\alpha}$. (A unitary transformation of the states $\ket{a}$?) The sum goes to infinite in general but we do a truncation. (Why does this work? with the truncation).
\begin{equation}
 \ket{a} = \sum_{\alpha}^n C_{a\alpha} \ket{\alpha}
 \label{def:spin-orbital expantion}
\end{equation}
%
$C_{a\alpha}$ is the expansion coefficient of an unitary matrix $C^{\dagger}C = 1$. Writing the Hamiltonian in this basis up to two-body interactions gives

\begin{equation}
\hat{H} = \sum_{ab} \bra{a}h\ket{b}a_a^{\dagger}a_b + \frac{1}{2} \sum_{abcd} \bra{ab}v\ket{cd} a_a^{\dagger}a_b^{\dagger} a_d a_c
 \label{def:Hamiltonian in second quantization}
\end{equation}
Note that the Hamiltionian is in second quantization form (see section \ref{sec:second quantization} and \cite{bartlett} for details). $h$ is just the one-body operator, and $v$ is the two-body operator. 
%
We will restrict our system to a closed shell system (RHF-equations, see cite...), i.e. all possible spin-orbital levels are filled. Our trial wavefunction $\ket{\Phi_{\text{HF}}}$ will be the \emph{Fermi vacuum} $\ket{\bf0}$ (see section \ref{sec:particle-hole formalism}). We will use Wick's theorem to evaluate $\bra{\bf 0}\hat{H}\ket{\bf 0}$. First the Hamiltonian (\ref{def:Hamiltonian in second quantization}) have to be rewritten in the quasi-operator representation
\begin{equation}
\hat{H} = \sum_{ij} \bra{i}h\ket{j}b_ib_j^{\dagger} + \frac{1}{2} \sum_{ijkl} \bra{ij}v\ket{kl} b_ib_j b_l^{\dagger} b_k^{\dagger}
 \label{def:Hamiltonian in quasi-operator representation}
\end{equation}
Wick's theorem on quasi-operators
\begin{align}
\bra{\bf 0} b_{i}b_{j}^{\dagger} \ket{\bf 0} &= \delta_{ij} \qquad &\text{from (\ref{eq:vacuumexpectation2redo one-body})} \\
\bra{\bf 0} b_ib_j b_l^{\dagger} b_k^{\dagger} \ket{\bf 0} &= \delta_{kj}\delta_{li} - \delta_{lj}\delta_{ki} \qquad &\text{from (\ref{eq:vacuumexpectation2redo})} 
 \label{eq:evaluate quasi-operators}
\end{align}
Lead us to the following energy functional
\begin{align}
E[\Phi_{\text{HF}}] &= \bra{\bf 0}\hat{H}\ket{\bf 0} = \sum_{i} \bra{i}h\ket{i} + \frac{1}{2}   \sum_{ijkl} [\bra{ij}v\ket{ij} - \bra{ij}v\ket{ji}] \nonumber \\
        &= \sum_{i} \bra{i}h\ket{i} + \frac{1}{2}   \sum_{ij} \bra{ij}v\ket{ij}_{\text{AS}}
 \label{eq:energyfunctional}
\end{align}
%
Inserting the new basis states (\ref{def:spin-orbital expantion}) in this expression yields 
%
\begin{equation}
E[\Phi_{\text{HF}}] = \sum_{a}^N \sum_{\alpha \beta}^n C_{a\alpha}^*C_{b\beta} \bra{\alpha}h\ket{\beta} + \frac{1}{2} \sum_{ab}^N\sum_{\alpha\beta\gamma\delta}^n C_{a\alpha}^*C_{b\beta}^* C_{a\gamma}C_{b\delta} \bra{\alpha\beta}v\ket{\gamma\delta}_{\text{AS}}
 \label{eq:energyfunctional2}
\end{equation}
%
The second-quantized form of an operator is unchanged under a unitary transformation of the basis (see \cite{bartlett}). 

As mention the method of choice for minimizing $E[\Phi_{HF}]$ is the Lagrange multipliers method (see \cite{boas}).

In general we want to find a stationary point $p=(x_1,x_2,...,x_n)$ of a function $f(p)$ with multiple constraints
\begin{align}
g_1(p) &= 0 \nonumber \\
g_2(p) &= 0 \nonumber \\
\vdots  \\
g_N(p) &= 0  \nonumber 
 \label{def:constraints}
\end{align}
%
Then a point $p$ is a stationary if and only if 
%
\begin{equation}
\nabla_{\nu} f(p) - \sum_{i=1}^N \lambda_i \nabla_\nu g_i(p) = 0, \qquad \forall \nu \in p
 \label{eq:langrange multiplier}
\end{equation}
%
$\lambda_i$ is the Langrangian multipliers. We get a system of $N\cdot n$ equations to be solved for the $(N+n)$-set of variables  $\lambda_1,\lambda_2...\lambda_N,x_1,x_2,...x_n$

In our case the constraint is the HF spin-orbitals are orthonormal (Because we have done a unitary transformation?)(which they have to be? No Since a og b can have same spin, depends on the physical system)

\begin{equation}
\braket{a}{b} = \sum_{\alpha}^n C_{a\alpha}^*C_{b\alpha} = \delta_{ab}
 \label{def: Hartree Fock constraint}
\end{equation}
%
We want to find the stationary point of the energy functional (\ref{eq:energyfunctional2}) with respect to a specific coefficient $C_{k\mu}^*$ in (\ref{def:spin-orbital expantion}). We dont have to do the same for $C_{k\mu}$ since they only differ by a phase. This means that for our HF spin-orbitals $\ket{a} \rightarrow i\ket{a}$ will not change the expectation value. 

We define
\begin{align}
F &\equiv E[\Phi_{\text{HF}}] - \sum_a^N \lambda_a g_a \nonumber \\
  &= E[\Phi_{\text{HF}}] - \sum_a^N \lambda_a \sum_\alpha^n C_{a\alpha}^*C_{a\alpha}
 \label{def:Big F}
\end{align}
%
Then we take the partial derivate of $F$ with respect to $C_{k\mu}^*$ using Eq. (\ref{eq:langrange multiplier}), yielding

\begin{align}
\frac{\partial}{\partial C_{k\mu}^*}  &\ql(  \sum_{a}^N\qr.  \sum_{\alpha \beta}^n C_{a\alpha}^*C_{b\beta} \bra{\alpha}h\ket{\beta} + \frac{1}{2} \sum_{ab}^N\sum_{\alpha\beta\gamma\delta}^n C_{a\alpha}^*C_{b\beta}^*  C_{a\gamma}C_{b\delta} \bra{\alpha\beta}v\ket{\gamma\delta}_{\text{AS}} \nonumber - \ql. \sum_a^N \lambda_a \sum_\alpha C_{a\alpha}^*C_{a\alpha}  \qr ) \\
 &= \sum_{\alpha \beta}^n C_{k\beta} \bra{\alpha}h\ket{\beta} + \frac{1}{2} \sum_{a}^N \sum_{\alpha\beta\gamma\delta}^n C_{a\beta}^*C_{k\gamma}  C_{a\delta} \bra{\alpha\beta}v\ket{\gamma\delta}_{\text{AS}} - \lambda_k \sum_\alpha^n C_{k\alpha} \\
&= \sum_{\alpha}^n\ql (\sum_\beta^n C_{k\beta} \bra{\alpha}h\ket{\beta} + \frac{1}{2} \sum_{a}^N \sum_{\beta\gamma\delta}^n C_{a\beta}^*C_{k\gamma}  C_{a\delta} \bra{\alpha\beta}v\ket{\gamma\delta}_{\text{AS}} - \lambda_k C_{k\alpha} \qr ) = 0  \label{eq:partial of F} 
\end{align}
%
The factor $1/2$ dissapears because of the product rule when we derivate $C_{a\alpha}^*C_{b\beta}^*  C_{a\gamma}C_{b\delta}$, we get two cases, when $\{a = k, \alpha = \mu \}$ and $\{b = k, \beta = \mu\}$. We rewrite Eq. (\ref{eq:partial of F})

\begin{equation}
\sum_\gamma^n C_{k\gamma} \bra{\alpha}h\ket{\gamma} + \frac{1}{2} \sum_{a}^N \sum_{\beta\gamma\delta}^n C_{a\beta}^*C_{k\gamma}  C_{a\delta} \bra{\alpha\beta}v\ket{\gamma\delta}_{\text{AS}}  = \lambda_k C_{k\alpha}
  \label{eq: rewritten equation} 
\end{equation}
Where we have changed the summation index $\beta \rightarrow \gamma$ in the first sum. And this than be simplified further

\begin{equation}
\sum_\gamma^n \ql ( \bra{\alpha}h\ket{\gamma} + \frac{1}{2} \sum_{a}^N \sum_{\beta\delta}^n C_{a\beta}^*  C_{a\delta} \bra{\alpha\beta}v\ket{\gamma\delta}_{\text{AS}} \qr ) C_{k\gamma} = \lambda_k C_{k\alpha}
 \label{def: simplified HF equation}
\end{equation}
%
By defining the Hartree-Fock Hamiltonian as 
\begin{equation}
h_{\alpha \gamma}^{\text{HF}} \equiv \bra{\alpha}h\ket{\gamma} + \frac{1}{2} \sum_{a}^N \sum_{\beta\delta}^n C_{a\beta}^*  C_{a\delta} \bra{\alpha\beta}v\ket{\gamma\delta}_{\text{AS}}
 \label{def: HF Hamiltonian}
\end{equation}
We finally obtain the Hartree-Fock equation
\begin{equation}
\sum_\gamma^n h_{\alpha\gamma}^{\text{HF}} C_{k\gamma} = \lambda_k C_{k\alpha}
 \label{def: The final HF equation}
\end{equation}
%
This is a non-linear set of equations, meaning that the set of equations changes with solutions we find. One way to solve this is by the \emph{self-consisten field method}. 



%
%\emph{the Hylleraas-Undheim theorem} \cite{hylleraas} and \cite{helgaker}
%
%More Hartree-Fock Approximation (see Heinnonen 59 kapt 7) And about Restricted Hartree-Fock approximation and the symmetry dilemma.
%
%The quasi-particle concept Heinnonen 313 kapt 26
%
%Time-independent perturbation theory Heinnonen 197 kapt 17
%
%Particle and hole operators and Wick's theorem kapt 19,20,21,22...
\section{Self-Consistent Field Method}
This is an iterative method that involves diagonalizing the $h_{\alpha\beta}^{\text{HF}}$-matrix for each iteration and summing the eigenvalues. It reaches selfconsitensy  

In a quantum mechanical system of many particles, particles are interacting with each other and there is \emph{correlation} between them. Therefore we have to find a wavefunction which describe the system as an whole. Our guess is that this wavefunction could have the form of an Slater-Determinant which is product of the single-particle wavefunctions, which depends on both spin and positional coordinates. 

The single-particle wavefunctions must be \emph{self-consistent}, i.e. they are solutions of the  Schr\"{o}dinger equations for the average field produced by the other particles, and they determine the average potential of the field.

One of the earliest methods relating to a self-consistent field was the Thomas-Fermi method (L. Thomas and E. Fermi) (ref). It did not use wave functions but the density distribution of the particles. The self-consistent Thomas-Fermi field explains the sequence in which electron shells are filled in atoms. 

The Hartree method does not take into account the PEP, an improved self-consistent field was given by the Hartree-Fock method, which was introduced by V.A. Fock in 1930 (ref). He used the SD instead of just the product of single-particle wavefunctions. We use this method to find the single-particle states that gives us minimum in energy.

Hartree-Fock method gives contribution for pairs of particles with opposite spins, but not the case of when the pairs of the spins are parallel? (Attraction?). A generalization of the Hartree-Fock method that take this into account aswell was developed by N. N. Bogoliubov in 1958 (ref?). He used this in the theory of superconductivity and the theory of heavy nuclei.

Historically, the concept of a self-consistent field was introduced in 1907 by the French physiscist P. Weiss \cite{weiss}. He found that the magnetic moment of each atom in a ferromagnetic material is proportional to magnetic moment and is thus \emph{self-consistent}. %(Vi har en partikkel som ser et gjennomsnittlig felt dannet av andre partikler, men som er med p� � skape dette feltet.)

The interaction, averaged in a certain way, between a given particle and the other particles of a quantum-mechanical system consisting of many particles. Because the problem of many interacting particles is very complex and has no exact solution, calculations are done by approximate methods. One of the most often used approximate methods of quantum mechanics is based on the introduction of a self-consistent field, which permits the many-particle problem to be reduced to the problem of a single particle moving in the average self-consistent field produced by the other particles. 

A concept used to find approximate solutions to the many-body problem in quantum mechanics. The procedure starts with an approximate solution for a particle moving in a single-particle potential, which derives from its average interaction with all the other particles. This average interaction is determined by the wave functions of all the other particles. The equation describing this average interaction is solved and the improved solution obtained is used in the calculation of the interaction term. This procedure is repeated for wave functions until the wave functions and associated energies are not significantly changed in the cycle, self-consistency having been attained. In atomic theory the Hartree-Fock procedure makes use of self-consistent fields. 

We would refer to this as the restricted Hatree-Fock (RHF) method. Which is used in calculations of a \emph{closed-shell} system, i.e. all the levels are filled with electrons. For \emph{open-shell} systems unrestricted Hartree-Fock (UHF) is the method of choice (see \cite{uhf}).

We would like to know how well the HF method preform. Therefore we introduce the term \emph{correlation energy}. 

\begin{equation}
E_{corr} = E_{exact} - E_{HF}
 \label{def:correlation energy}
\end{equation}

It is the difference between the exact nonrelativistic energy (i.e. the FCI energy) and the Hartree-Focl energy. This gives us some indication of how much the electron-electron interaction contributes to our system.

\begin{figure}
\centering
  \scalebox{0.75}{
    \input{electron-correlation.pdftex_t}
}
\caption{A schematic overview over different energies for different approximations}
 \label{fig:Electron correlation energy}
\end{figure}


\section{Outline of the Algorithm}

\begin{figure}
\centering
\includegraphics{HF-flowchart.pdf}
\caption{Flowchart of Hartree-Fock Algorithm. Self-consistency is the condition that $E[\Phi_{new}] - E[\Phi_{old}] = 0$ }
 \label{fig:HF flow chart}
\end{figure}


%%%%%%%%%%%%%%%%%%%%%%%%%%%%%%%%%%%%%%%%%%% Comments %%%%%%%%%%%%%%%%%%%%%%%%%%%%%%%%%%%%%%

% Thjiissen:
% Hartree-Fock method a variational method? We have a SD.
% Individual one-electron wave functions are called \emph{orbitals}.
% The Schr\"{o}dinger equation have a potential determined by the electrons. A coupling between the orbitals via the potential makes the equation (SL?) nonlinear in the orbitals?. 
% 
% The solution to this \emph{nonlinear in the orbitals} is found iteratively in a self-consistency procedure??? 
% 
% The Hartree-Fock method is close the mean-field approach in Statmech???
% 
% Hartree-Fock method is very popular among chemists, and it has also been applied to solids. [See referanser...]
% 
% Hartree-Fock method used in fermionic systems, 
% 
% The Coulomb repulsion between the electrons is represented in an averaged way. 
% The Pauli Principle is included in the HF-theory.
% 
% Even with the position of the nuclei kept fixed (Born-Oppenheimer approximation). But still we have problems with too many degrees of freedom. Especially the the interaction term makes the problem difficult. (In what way?)
% 
% One common method is to ''uncouple`` the problem in which the interaction of one electron with the remaining ones is incorporated in an averaged way into a potential felt by the electron 
% 
% Def: Spin-orbitals are functions depending on the spatial and spin coordinates of one electron (page 48)
% 
% However the Hamiltonian does not act on the spin-dependent part of the spin-orbital (a quantum state |alpha>). Then spatial and the spin part can be seperated into two products. 
% 
% Hartree potential/Hartree term is something unphysical: it contains a coupling between orbital \emph{k} and itself.
% 
% Hartree derived his equation 4.11 in 1927, it neglects exchange as well as other correlations (which?)
% 
% Fock took the antisymmetry into account, exchange term. 
% 
% It is clear that the nonlinear Hartree-Fock equation, must be solved by a selfconsistency iterative procedure. (Describe) This means that we start with a finite number of states N. A guess. These are used to generate a new Fock operator which is diagonalized again and we get new states. This procedure is repeated until the wanted convergence is achieved.
% 
% CI (Configuration Interaction) is a systematic way of improving upon Hartree-Fock theory.
% 
% The exchange term vanished for orthogonal states, pair of opposite spin do not feel this term.
% 
% The derivation of the Hartree-Fock equation is based on a varitaional calculation for the Schr\"{o}dinger equation. 
% 
% We can use Koopman's theorem to approximate excitation energies from a ground state calculation.
% 
% In closed shells systems, all the levels are occupied by two electrons with opposite spin, but in open-shell system there some levels that are partially filled. 
% 
% Calculation with spin-orbitals paired are called restricted Hartree-Fock (RHF). Those in which all spin-orbitals are allowed to have a different spatial dependence are called unrestricted Hartree-Fock (UHF).
% 
% Magnus: \cite{lohne}  The foundation of HF is the variational principle of Rayleigh and Ritz (RR) (See E.K.U Gross)
% 
% Derived the Hartree Fock Equations
% 
% Patrick: \cite{merlot}
% Quantum Dots can be seen as an many-particle problem. (Show how?)
% 
% Hartree-Fock method is minimazation problem based on a mathematical technique known as the Lagrange multipliers, we want to minimize the energy.


% 
% Assume that we have a hermitian Hamiltonian operator that can be separated into two parts
% 
% \begin{equation}
% \hat{H} = \hat{H}_0 + \lambda\hat{V} 
%  \label{def: two part Hamiltonian}
% \end{equation}
% %
% Where $H_0$ is the zero-order and $\lambda\hat{V}$ the perturbation. $\lambda$ is called an \emph{order parameter} which is used to classify the order of the contributions. We assume that $\Psi_n$ is the eigenstate to (\ref{def: two part Hamiltonian}), and $\Phi_n$ the eigenstate for the zero-order $H_0$. And that they are ortonormal.
% 
% \begin{theorem}
% \emph{(Bartlett)}
% \label{thm:Lagrange}
% If $\Phi_n$ is nondegenerate, then its possible to find the solutions 
% \begin{equation}
%  \lim_{\lambda \rightarrow 0} \Psi_n = \Phi_n, \qquad  \lim_{\lambda \rightarrow 0} E_n = E_n^{(0)}
% \end{equation}
% \end{theorem}
% 
% \subsection{Projection Operators}
% We define the projection operators
% 
% \begin{align}
% \hat{P} &= \ket{\Phi_0}\bra{\Phi_0} \\
% \hat{Q} &= \hat{1} - \hat{P} = \sum_{i=1} \ket{\Phi_i}\bra{\Phi_i}
%  \label{def:projection operators}
% \end{align}
% %
% Rayleigh-Schr\"{o}dinger perturbation theory (RSPT) is extensive order by order. This is related to the fact that RSPT is a true order-by-order expansion, unlike BWPT, which has the infinite-order energy in all denominators \cite{bartlett} ????.
% 
% Time-independent perturbation theory Heinnonen 197 kapt 17
% 
% \subsection{Rayleigh-Schr\"{o}dinger Perturbation Theory}
% \subsection{Brioullin-Wigner Perturbation Theory}
% \subsection{Introduction to Diagrammatic notation}
% % 
% 
% %%%%%%%%%%%%%%%%%%%%%%%%%%%%%%%%%%%%%%%%%%%%%%%%%%%%%%%%%%%%%%%%%%%%%%%%%
% %%%%%%%%%%%%%%%%%%%%%%%%%%%%%% CCSD %%%%%%%%%%%%%%%%%%%%%%%%%%%%%%%%%%%%%
% %%%%%%%%%%%%%%%%%%%%%%%%%%%%%%%%%%%%%%%%%%%%%%%%%%%%%%%%%%%%%%%%%%%%%%%%%
\chapter{Coupled Cluster Theory}
In this chapter we will introduce the Coupled Cluster Theory \cite{crawford} and derive the Coupled Cluster Equations for Single and Double Exitations (CCSD). 

\section{Introduction}
The theoretical framework of Coupled Cluster Theory was developed in the late 1960s by Coester and K\"{u}mmel. He applied it to problems in nuclear and subnuclear physics. Later it was introduced into quantum chemistry in the 1960s by \v{C}\'i\v{z}ek \cite{cizek1,cizek2}. K\"{u}mmel said that he found it remarkable that a quantum chemist would open an issue of a nuclear physics journal \cite{kummel-ccsd}. Then Monkhorts \cite{monkhorst} developed the CC response theory for calculating the molecular properties. And with the revolution of microprocessors the computational power increased rapidly. People like Pople et al. \cite{pople-ccsd} and Bartlett et al. \cite{bartlett-ccsd} began to look into more realistic systems by developing the spin-orbital CCD programs. Since then people have tried to develop efficient CCSD energy codes, and inclusion of higher excitations in the wavefunction. And there is still onging work with develping methods for open shell calculations (Equation of Motion CCSD) \cite{krylov}.


Later Bishop applied and developed the CCM theory for use in condensed matter physics, electron gas at both high and low densities. (See link) 

The CCM is a numerical method for solving the many-body problem (see section \ref{sec:many-body problem}). It is a popular choice with respect to computability ($\sim 100$ nucleons) (cite). Other \emph{ab intion} many-body methods such as Green's Function Monte Carlo have great precision compared to experiment but have expoential growth in computational need \cite{jensen-ccsd}.

\section{The Coupled Cluster Formalism}
In this section we will discuss the notation that are going to be used and derive the theory behind the CC equations.

\subsection{Reference state and operators}
We will use the particle-hole formalism (see section \ref{sec:particle-hole formalism}). The reference state are defined as

\begin{equation}
\ket{\bf 0} \equiv \ket{\Phi_0}
 \label{def:ccsd reference state}
\end{equation}

Where $\ket{\Phi_0}$ is a SD with single-particle orbitals up to the Fermi-level,

\begin{equation}
 \ket{\Phi_0} = a^{\dagger}_{\alpha_1}a^{\dagger}_{\alpha_2}...a^{\dagger}_{\alpha_F}\ket{0}
\label{def:old reference state}
\end{equation}

Where $\alpha_F$ denotes the quantum number that is on the Fermi-level. annihiliation and creation operators with subscripts $ijk...$ will denote hole states, and the subscripts $abc...$ will denote the particle states. Subscripts that are $pqr...$ are in either of these groups. For the creation and annihilation operators we will use the standard ones $a_\alpha$ and $a_\alpha^{\dagger}$ with respect to the physical vacuum $\ket{0}$. But we will keep the \emph{quasi}-annihilation $b_\alpha$ and creation operator $b_\alpha^{\dagger}$.   (\ref{def:quasi-creation},\ref{def:quasi-annihilation}) in mind when we use Wick's theorem.

Products of annihiliation and creation operators can be used to construct different states $\ket{\Phi^a}$ with respect to our reference state $\ket{\Phi_0}$ as shown in (\ref{def:creation of a hole},\ref{def:creation of a particle}). The subscripts will denote the hole states, and the superscripts will denote the particle states.  

%\begin{figure}
% \label{def:}
%\end{figure}

A general state can be represented by 

\begin{equation}
\ket{\Phi_{ijk...}^{abc...}}
 \label{def:general reference state}
\end{equation}
%
Where total number of particles in a system are given by 
%
\begin{equation}
N = N' + n_p - n_h
 \label{eq:total number of particles}
\end{equation}
%
$N'$ is the number of of particles in the reference state. And $n_p$ is the number of particles, $n_h$ is the number of holes.
We define the single-orbital exitation operator (cluster operator) as 

\begin{equation}
t_i \equiv \sum_a t_i^a a_a^{\dagger} a_i 
 \label{def:single-orbital cluster operator}
\end{equation}
%
And the two-orbital exitation operator  
\begin{equation}
t_{ij} \equiv \frac{1}{2} \sum_a t_{ij}^{ab} a_a^{\dagger}a_b^{\dagger} a_j a_i 
 \label{def:two particle orbital cluster operator}
\end{equation}
%
In genereal the $n$-orbital exitation operator is defined as 
\begin{equation}
t_{ijk...}^{abc...} \equiv \frac{1}{n!} \sum_{\underbrace{\mbox{\scriptsize $abc...$}}_n} t_{ijk...}^{abc...} a_a^{\dagger}a_b^{\dagger}a_c^{\dagger}...a_j a_i a_k...
 \label{def:n-ortbital exitation operator}
\end{equation}
%
$t_{ij..}^{ab..}$ is the $n$-orbital excitation amplitude. The factor $\frac{1}{n!}$ is due to the fact that we have an unrestricted summation over the particle states, and there are $n!$ ways of permuting $n$-particles which give rise to the same final state. 
%
The total excitation amplitdes is the sum of all possible excitations
\begin{align}
\hh{T}_1 &\equiv \sum_i \hat{t_i} \label{def:T_1} \\
\hh{T}_2 &\equiv \frac{1}{2} \sum_{ij} t_{ij}^{ab} \label{def:T_2} \\
& \vdots \qquad \vdots \nonumber \\
\hh{T}_n &\equiv \sum_{\underbrace{\mbox{\scriptsize $ijk...$}}_n} t_{\underbrace{ \mbox{\scriptsize $ijk...$}}_n}^{\overbrace{\mbox{\scriptsize $abc...$}}^{n}}
 \label{def:T_n}
\end{align}

\subsection{The Exponential Ansatz}
Our ansatz is that the exact wavefunction $\ket{\Psi}$ can be written as 

\begin{equation}
\ket{\Psi} = e^{\hat{T}} \ket{\Phi_0}
 \label{def:exponential ansatz}
\end{equation}
%
Where $\hat{T}$ is the total excitation operator
\begin{equation}
\hh{T} \equiv \sum_n^{\infty} \hh{T}_n 
 \label{def:total excitation operator}
\end{equation}
%
and $\ket{\Phi_0}$ is the refernce state. (The exponential ansatz manifest the extensitivity property??? see Bartlett). The motivation for this ansatz aswell???? 

This would be the exact solution to our many-body Schr\"{o}dinger equation and Why??? Proof??? (ref). But we have to truncate this sum and these truncations give rise to errors in the CC calculation. The fewer term we truncate the closer we are to the exact solution. Different CC schemes are determined by the level of truncation,

\begin{align}
\hh{T} &= \hh{T}_1 \qquad \mbox{(CCS)} \\
\hh{T} &= \hh{T}_1 + \hh{T}_2 \qquad \mbox{(CCSD)} \\
\hh{T} &= \hh{T}_1 + \hh{T}_2 + \hh{T}_3 \qquad \mbox{(CCSDT)} \\ 
\hh{T} &= \hh{T}_1 + \hh{T}_2 + \hh{T}_3 + \hh{T}_3 \qquad \mbox{(CCSDTQ)}
 \label{def:CC schemes}
\end{align}
%
We will focus on the CCSD scheme in this thesis.


\section{Energy equation for the CCSD}  

%Natural Truncation of the Exponential Ansatz......... page 22 Crawford

The problem we want to solve is the manybody Sch\"{o}dinger equation defined in Eq. (\ref{def:manybodyschrodinger})
\begin{equation}
\hh{H} \ket{\Psi} = E \ket{\Psi}
 \label{def:CC schrodinger}
\end{equation}
%
With our Exponential ansatz

\begin{equation}
\hh{H} e^{\hh{T}} \ket{\Phi_0} = E e^{\hh{T}}\ket{\Phi_0}
 \label{def:CC schrodinger ansatz}
\end{equation}
%
We can left-multiply to get the energy eigenvalue

\begin{equation}
\bra{\Phi_0}\hh{H}e^{\hh{T}}\ket{\Phi_0} = E
 \label{def:intermediate normalization}
\end{equation}
%
Where $\braket{\Phi_0}{\Phi_{CC}}=1$, because of $\braket{\Phi_0}{\Phi_0}=1$ and $\braket{\Phi_{ijk...}^{abc...}}{\Phi_0}=0$ . This can be taken further by multiplying with a general excited sates 
 
\begin{equation}
\bra{\Phi_{ijk...}^{abc..}}\hh{H}\ket{\Phi_0} = \bra{\Phi_{ijk...}^{abc..}}E e^{\hh{T}}\ket{\Phi_0}
\label{def:intermediate normalization excited} 
\end{equation}
%
This gives us a set of equations involving the excitation amplitude $t_{ijk..}^{abc..}$. And they are non-linear because of $e^{\hh{T}}$. But formally they are solved exactly when $\hat{T}$ is not truncated (ref?????). We want to decouple the energy and the excitation amplitudes. This can be done by multiplying with $\bra{\Phi_0} e^{\hat{T}}$.

\begin{align}
\bra{\Phi_0}e^{-\hh{T}} \hh{H} e^{-\hh{T}} \ket{\Phi_0}&=E
 \label{eq:mathematical foresight 1} \\
\bra{\Phi_{i}^{a}} e^{-\hh{T}} \hh{H} e^{\hh{T}} \ket{\Phi_0} &= 0
 \label{eq:mathematical foresight 2}  \\
\bra{\Phi_{ij}^{ab}} e^{-\hh{T}} \hh{H} e^{\hh{T}} \ket{\Phi_0} &= 0  \label{eq:mathematical foresight 22}\\
\vdots \nonumber\\
  \bra{\Phi_{ijk...}^{abc..}}e^{-\hh{T}} \hh{H} e^{\hh{T}} \ket{\Phi_0} &=0 \label{eq:mathematical foresight 3}
\end{align}
%
Which are the CC equations. The first one Eq. (\ref{eq:mathematical foresight 1}) is the CC energy equation. The next ones are the \emph{amplitude} equations, which we have to solve to get the excitation amplitudes $t_{ijk...}^{abc...}$.

We can write the similarity transformed Hamiltonian as a nested sum of commutators using the Baker-Campbell-Hausdorff relation (see Appendix \ref{dix:derivation of baker-campbell-hausdorff}).

\begin{equation}
e^{-\hh{T}} \hat{H} e^{\hh{T}} =  \hh{H} + \ql[\hh{H}, \hh{T}  \qr] + \frac{1}{2!} \ql[\ql[\hh{H}, \hh{T}  \qr],\hh{T} \qr] + \frac{1}{3!} \ql[ \ql[\ql[\hh{H}, \hh{T}  \qr],\hh{T} \qr],\hh{T} \qr] + ...
 \label{def:BCH formula}
\end{equation}
%
Instead of the Hamiltonian will are going to use the normal-ordered $\hh{H}_N$ to derive the CCSD equations, and truncate such that $\hh{T} = \hh{T}_1 + \hh{T}_2$. We define the similarity transformed normal-ordered  Hamiltonian, $\bb{H} \equiv e^{-\hh{T}} \hh{H}_N e^{\hh{T}}$. %Inserting for the normal-ordered Hamiltonian Eq. (\ref{eq:final normal ordered hamiltonian}) gives
% \begin{align}
% \bb{H} &=  e^{-\hh{T}}\ql[\hh{H}_N + \bra{\Phi_0} \hh{H} \ket{\Phi_0}\qr] e^{\hh{T}} \nonumber \\
% &= e^{-\hh{T}}\hh{H}_Ne^{\hh{T}} +  \bra{\Phi_0} \hh{H} \ket{\Phi_0}
%  \label{def:inserting normal ordered}
% \end{align}
%
Inserting this into Eq. (\ref{eq:mathematical foresight 1})-(\ref{eq:mathematical foresight 22}) 

\begin{align}
\bra{\Phi_0} \bb{H} \ket{\Phi_0} + \bra{\Phi_0}\hh{H}\ket{\Phi_0} &= E
 \label{eq:inserting this into 1} \\
\bra{\Phi_i^a}\bb{H} \ket{\Phi_0} & = 0  \label{eq:inserting this into 2} \\
\bra{\Phi_{ij}^{ab}}\bb{H} \ket{\Phi_0} & = 0  \label{eq:inserting this into 3} 
\end{align}
%
We define the CCSD energy as 

\begin{equation}
E_{\text{CCSD}} = \bra{\Phi_0} \bb{H} \ket{\Phi_0} + E_0
 \label{def:CCSD energy} 
\end{equation}
%
Where $E_0 \equiv \bra{\Phi_0}\hh{H}\ket{\Phi_0}$ is the SCF energy (???). There are two ways to derive the set of energy equations (\ref{eq:inserting this into 1})-(\ref{eq:inserting this into 2}), the diagramatic and the algebraic. We are going to do the latter first. 

\subsection{The Algebraic Approach}
Now we want to do a BCH-expansion on $\bb{H}$
%
\begin{equation}
\bb{H} = \hh{H}_N + \ql[\hh{H}_N, \hh{T}\qr] + \frac{1}{2!}\ql[\ql[\hh{H}_N, \hh{T}\qr], \hh{T}\qr] 
 \label{eq: bb H BCH-expansion}
\end{equation}
%
The expansion goes to infinity but terminates since $\hh{H}_N$ (Eq. \ref{def:second-ordered Hamiltonian2}) is a two-particle operator and therefore can at most de-excite a state that have been twofold excited. This is important because the ortonormality of the reference state $\ket{\Phi_0}$. Therefore we will never get contributions from $\hh{T}^3$ and higher, because $\bra{\Phi_0}\hh{H}_N T_1^3 \ket{\Phi_0} = 0$. This is due to the type of Hamilton operator and not electron numbers. Inserting for $\hh{T} = \hh{T}_1 + \hh{T}_2$
\begin{align}
\bb{H} &= \hh{H}_N + \ql[\hh{H}_N, \hh{T}_1\qr] + \ql[\hh{H}_N, \hh{T}_2\qr] + \frac{1}{2} \ql[\ql[\hh{H}_N, \hh{T}_1\qr] , \hh{T}_1\qr] \nonumber \\ 
&+  \frac{1}{2} \ql[\ql[\hh{H}_N, \hh{T}_2\qr] , \hh{T}_2\qr] + \frac{1}{2}\ql[\ql[\hh{H}_N, \hh{T}_1\qr] , \hh{T}_2\qr] + \frac{1}{2}\ql[\ql[\hh{H}_N, \hh{T}_2\qr] , \hh{T}_1\qr] 
 \label{eq:bar H similarity series}
\end{align}
%
We know that the $\hh{T}_n$ operators commute with each other beacause of commutation relations Eq.  (\ref{def:quasioperator1})-(\ref{def:quasioperator3}). This gives us the following relation

\begin{equation}
\ql[ \ql [\hh{H}_N,\hh{T}_2 \qr],\hh{T}_1 \qr] = \ql[ \ql [\hh{H}_N,\hh{T}_1 \qr],\hh{T}_2 \qr]
 \label{def:commutation relation}
\end{equation}
%
And the following Hamiltonian
\begin{align}
\bb{H} &= \overbrace{\hh{H}_N}^{\textbf{1th term}} + \overbrace{\ql[\hh{H}_N, \hh{T}_1\qr]}^{\textbf{2nd term}}+ \overbrace{\ql[\hh{H}_N, \hh{T}_2\qr]}^{\textbf{3rd term}}+  \overbrace{\frac{1}{2} \ql[\ql[\hh{H}_N, \hh{T}_1\qr] , \hh{T}_1\qr]}^{\textbf{4th term}} \nonumber \\ 
&+  \underbrace{\frac{1}{2}  \ql[\ql[\hh{H}_N, \hh{T}_2\qr] , \hh{T}_2\qr]}_{\textbf{5th term}} + \underbrace{\ql[\ql[\hh{H}_N, \hh{T}_1\qr] , \hh{T}_2\qr]}_{\textbf{6th term}} 
 \label{eq:following hamiltonian}
\end{align}
%
We need to express these commutation relations in a second-quantized form which leads us to the algebraic approach.

\subsubsection*{First term}
The first term gives us the contribution 

\begin{empheq}[box=\fbox]{equation}
\bra{\Phi_0} \hh{H}_N \ket{\Phi_0} = 0 \rightarrow E_{\text{CCSD}}^{(1)}  
 \label{eq:first term}
\end{empheq}


\subsubsection*{Second term}
\begin{equation}
\ql[\hh{H}_N, \hh{T}_1\qr] =  \ql[\hh{F}_N, \hh{T}_1\qr] +  \ql[\hh{V}_N, \hh{T}_1\qr]   
 \label{def:second commutator} 
\end{equation}
%
\begin{align}
\hh{V}_N\hh{T}_1&= \frac{1}{4}\sum_{pqrs}\sum_{ia} \bra{pq}|v|\ket{rs} t_i^a \{a_p^{\dagger}a_q^{\dagger}a_s a_r\}\{a_a^\dagger a_i\}
 \label{def:V_N T_1}
\end{align}
%
There are four operators in the first operator string and two in the second. Using the general Wick's theorem we see that we cannot obtain fully contracted terms and thus does not contribute to the CCSD energy. Contractions within the operator strings gives zero since they are already on a normalordered form. 

\begin{equation}
\bra{\Phi_0}[ \hh{V}_N,\hh{T}_1 ]\ket{\Phi_0} = 0 \rightarrow E_{\text{CCSD}}^{(2)} 
 \label{eq:second term T_1 V_1 energy}
\end{equation}
%
Let us find a secondquantized form of the first commutator 

\begin{align}
\hh{F}_N\hh{T}_1 &= \sum_{pq}\sum_{ia}f_q^p t_i^a \{a_p^{\dagger}a_q\}\{a_a^\dagger a_i\}  \label{def:F_N T_1}\\
\hh{T}_1\hh{F}_N &= \sum_{pq}\sum_{ia}f_q^p t_i^a \{a_a^\dagger a_i\}\{a_p^{\dagger}a_q\}
 \label{def:T_1 F_1}
\end{align}
%
From section \ref{sec:particle-hole formalism} we have the following contraction relations

\begin{align}
\contraction{}{a}{{}^\dagger}{a} 
a_{i}^\dagger a_j &= \delta_{ij} \label{rel:1}\\
\contraction{}{a}{{}_a}{a}            
a_{a} a_b^\dagger &= \delta_{ab} \label{rel:2}\\
\contraction{}{a}{{}^\dagger}{a}
a_{a}^\dagger a_b &= 0           \label{rel:3}\\
\contraction{}{a}{{}^\dagger}{a}
a_{i} a_j^\dagger &= 0           \label{rel:4}
\end{align}
%
Using the generalized form of Wick's theorem from Eq. (\ref{def:generalizednormalorder}) on the operator strings
%
\begin{align}
\contraction{\{a_p^{\dagger}a_q\}\{a_a^{\dagger} a_i\} = \{a_p^{\dagger}a_q a_a^\dagger a_i\} + \{}{a}{{}^{\dagger}a_q a_a^{\dagger} }{a}
\contraction{\{a_p^{\dagger}a_q\}\{a_a^{\dagger} a_i\} = \{a_p^{\dagger}a_q a_a^\dagger a_i\} + \{a_p^{\dagger} a_q a_a^{\dagger} a_i\} + \{a_p^{\dagger} }{a}{{}_{}^\dagger}{a}
\contraction[2ex]{\{a_p^{\dagger}a_q\}\{a_a^{\dagger} a_i\} = \{a_p^{\dagger}a_q a_a^\dagger a_i\} + \{a_p^{\dagger} a_q a_a^{\dagger} a_i\} + \{a_p^{\dagger} a_q a_a^{\dagger} a_i\}  + \{}{a}{{}_{}^\dagger a_q a_a}{a}
\contraction{\{a_p^{\dagger}a_q\}\{a_a^{\dagger} a_i\} = \{a_p^{\dagger}a_q a_a^\dagger a_i\} + \{a_p^{\dagger} a_q a_a^{\dagger} a_i\} + \{a_p^{\dagger} a_q a_a^{\dagger} a_i\}  + \{a_p^{\dagger}}{a}{{}_q}{a}
%
\{a_p^{\dagger}a_q\}\{a_a^{\dagger} a_i\} &= \{a_p^{\dagger}a_q a_a^\dagger a_i\} + \{a_p^{\dagger} a_q a_a^{\dagger} a_i\} + \{a_p^{\dagger} a_q a_a^{\dagger} a_i\}  + \{a_p^{\dagger}a_q a_a^\dagger a_i\} \nonumber \\
%
&= \{a_p^{\dagger}a_q a_a^\dagger a_i\} + \delta_{pi}\{a_q a_a^\dagger\} + \delta_{qa} \{a_p^{\dagger} a_i \} + \delta_{pi}\delta_{qa} \label{eq:generalized wick on F_N T_1} \\
%
\{a_a^{\dagger} a_i\}\{a_p^{\dagger} a_q\} &= \{a_a^{\dagger} a_i a_p^{\dagger} a_q\} = \{a_p^{\dagger} a_q a_a^{\dagger} a_i\} \label{eq:T_1 F_1 contraction}
\end{align}
%
The non-contracted terms cancels and the commutation relation of reads

\begin{equation}
\ql[\hh{F}_N,\hh{T}_1\qr] = \sum_{qia} f_q^i t_i^a \{a_q a_a^{\dagger} \} + \sum_{pia} f_a^p t_i^a \{a_p^{\dagger} a_i\} +  \sum_{ia} f_i^a t_a^i 
 \label{eq:second term F_N T_1 commuation}
\end{equation}
%
And the contribution from the second term to the energy 

\begin{empheq}[box=\fbox]{equation}
\bra{\Phi_0} [\hh{H}_N,\hh{T}_1] \ket{\Phi_0} = \sum_{ia}f_a^i t_i^a \rightarrow E_{\text{CCSD}}^{(2)}  
 \label{eq:second term energy}
\end{empheq}


\subsubsection*{Third term}
\begin{equation}
\ql[\hh{H}_N, \hh{T}_2\qr] =  \ql[\hh{F}_N, \hh{T}_2\qr] +  \ql[\hh{V}_N, \hh{T}_2\qr]   
 \label{def:third commutator} 
\end{equation}
%
Using the same argumentation we had for $[\hh{V}_N,\hh{T}_1]$, we see that $[\hh{F}_N,\hh{T}_2]$ does not contribute to the energy

\begin{align}
\hh{F}_N\hh{T}_2 &= \frac{1}{4} \sum_{abji}\sum_{pq} f_p^q t_{ij}^{ab} \{a_p^\dagger a_q\}\{a_a^{\dagger}a_b^{\dagger}a_j a_i\}
 \label{eq:F_N T_2}
\end{align}

\begin{equation}
\bra{\Phi_0}[\hh{F}_N,\hh{T}_2] \ket{\Phi_0} = 0 \rightarrow E_{\text{CCSD}}
 \label{eq:third term T_1 V_1}
\end{equation}
% 
The second part of the commutation relation of the third term gives

\begin{align}
\hh{V}_N\hh{T}_2 &= \frac{1}{16}\sum_{pqrs}\sum_{ijab} t_{ij}^{ab} \bra{pq}|v|\ket{rs} \{a_p^{\dagger}a_q^\dagger a_s a_r \}\{a_a^\dagger a_b^\dagger a_j a_i\} \\
\hh{V}_2\hh{V}_N &= \frac{1}{16}\sum_{pqrs}\sum_{ijab} t_{ij}^{ab} \bra{pq}|v|\ket{rs} \{a_a^\dagger a_b^\dagger a_j a_i\} \{a_p^{\dagger}a_q^\dagger a_s a_r \}
 \label{eq: third term commuation V_N T_2}
\end{align}
%
Using the generalized Wick's theorem 
%
\begin{align}
\contraction[4ex]{a_p^{\dagger}a_q^\dagger a_s a_r \}\{a_a^\dagger a_b^\dagger a_j a_i\} =   \{a_p^{\dagger}a_q^\dagger a_s a_r a_a^\dagger a_b^\dagger a_j a_i\} + \{a }{a}{{}_p^{\dagger}a_q^\dagger a_s a_r a_a^\dagger a_b^\dagger a_j }{a}
\contraction[3ex]{a_p^{\dagger}a_q^\dagger a_s a_r \}\{a_a^\dagger a_b^\dagger a_j a_i\} =   \{a_p^{\dagger}a_q^\dagger a_s a_r a_a^\dagger a_b^\dagger a_j a_i\} + \{a a_p^{\dagger} }{a}{{}_q^\dagger a_s a_r a_a^\dagger a_b^\dagger}{a}
\contraction[2ex]{a_p^{\dagger}a_q^\dagger a_s a_r \}\{a_a^\dagger a_b^\dagger a_j a_i\} =   \{a_p^{\dagger}a_q^\dagger a_s a_r a_a^\dagger a_b^\dagger a_j a_i\} + \{a a_p^{\dagger} {a}_q^\dagger}{a}{{}_s a_r a_a^\dagger}{a}
\contraction{a_p^{\dagger}a_q^\dagger a_s a_r \}\{a_a^\dagger a_b^\dagger a_j a_i\} =   \{a_p^{\dagger}a_q^\dagger a_s a_r a_a^\dagger a_b^\dagger a_j a_i\} + \{a a_p^{\dagger} {a}_q^\dagger a_s}{a}{{}_r }{a}
%
%
\contraction[4ex]{a_p^{\dagger}a_q^\dagger a_s a_r \}\{a_a^\dagger a_b^\dagger a_j a_i\} =   \{a_p^{\dagger}a_q^\dagger a_s a_r a_a^\dagger a_b^\dagger a_j a_i\} + \{a_p^{\dagger}a_q^\dagger a_s a_r a_a^\dagger a_b^\dagger a_j a_i\} + \{a }{a}{{}_p^{\dagger}a_q^\dagger a_s a_r a_a^\dagger a_b^\dagger a_j }{a}
\contraction[3ex]{a_p^{\dagger}a_q^\dagger a_s a_r \}\{a_a^\dagger a_b^\dagger a_j a_i\} =   \{a_p^{\dagger}a_q^\dagger a_s a_r a_a^\dagger a_b^\dagger a_j a_i\} + \{a_p^{\dagger}a_q^\dagger a_s a_r a_a^\dagger a_b^\dagger a_j a_i\} + \{a a_p^{\dagger} }{a}{{}_q^\dagger a_s a_r a_a^\dagger a_b^\dagger}{a}
\contraction[2ex]{a_p^{\dagger}a_q^\dagger a_s a_r \}\{a_a^\dagger a_b^\dagger a_j a_i\} =   \{a_p^{\dagger}a_q^\dagger a_s a_r a_a^\dagger a_b^\dagger a_j a_i\} + \{a_p^{\dagger}a_q^\dagger a_s a_r a_a^\dagger a_b^\dagger a_j a_i\} + \{a a_p^{\dagger} {a}_q^\dagger}{a}{{}_s a_r}{a}
\contraction{a_p^{\dagger}a_q^\dagger a_s a_r \}\{a_a^\dagger a_b^\dagger a_j a_i\} =   \{a_p^{\dagger}a_q^\dagger a_s a_r a_a^\dagger a_b^\dagger a_j a_i\} + \{a_p^{\dagger}a_q^\dagger a_s a_r a_a^\dagger a_b^\dagger a_j a_i\} + \{a a_p^{\dagger} {a}_q^\dagger a_s}{a}{{}_r a_a^\dagger}{a}
%
%
\{a_p^{\dagger}a_q^\dagger a_s a_r \}\{a_a^\dagger a_b^\dagger a_j a_i\} &=   \{a_p^{\dagger}a_q^\dagger a_s a_r a_a^\dagger a_b^\dagger a_j a_i\} +  \{a_p^{\dagger}a_q^\dagger a_s a_r a_a^\dagger a_b^\dagger a_j a_i\} + \{a_p^{\dagger}a_q^\dagger a_s a_r a_a^\dagger a_b^\dagger a_j a_i\}  \nonumber\\
%
%
\contraction[4ex]{+\{ }{a}{{}_p^{\dagger}a_q^\dagger a_s a_r a_a^\dagger a_b^\dagger }{a}
\contraction[3ex]{+\{ a_p^\dagger }{a}{{}_q^\dagger a_s a_r a_a^\dagger a_b^\dagger a_j}{a}
\contraction[2ex]{+\{ a_p^\dagger a_q^\dagger}{a}{{}_s a_r a_a^\dagger}{a}
\contraction{+\{ a_p^\dagger a_q^\dagger a_s}{a}{{}_r }{a}
%
%
\contraction[4ex]{+\{a_p^{\dagger}a_q^\dagger a_s a_r a_a^\dagger a_b^\dagger a_j a_i\} + \{ }{a}{{}_p^{\dagger}a_q^\dagger a_s a_r a_a^\dagger a_b^\dagger }{a}
\contraction[3ex]{+\{a_p^{\dagger}a_q^\dagger a_s a_r a_a^\dagger a_b^\dagger a_j a_i\}+\{ a_p^\dagger }{a}{{}_q^\dagger a_s a_r a_a^\dagger a_b^\dagger a_j}{a}
\contraction[2ex]{+\{a_p^{\dagger}a_q^\dagger a_s a_r a_a^\dagger a_b^\dagger a_j a_i\}+\{ a_p^\dagger a_q^\dagger}{a}{{}_s a_r}{a}
\contraction{+\{a_p^{\dagger}a_q^\dagger a_s a_r a_a^\dagger a_b^\dagger a_j a_i\}+\{ a_p^\dagger a_q^\dagger a_s}{a}{{}_r a_a^\dagger}{a} 
%
%
&+ \{a_p^{\dagger}a_q^\dagger a_s a_r a_a^\dagger a_b^\dagger a_j a_i\}  + \{a_p^{\dagger}a_q^\dagger a_s a_r a_a^\dagger a_b^\dagger a_j a_i\} + ...
 \label{eq: third term generalized wick} \\
%
% 
&= \{a_p^{\dagger}a_q^\dagger a_s a_r a_a^\dagger a_b^\dagger a_j a_i\} + \dd{pi}\dd{qj}\dd{sb}\dd{ra} - \dd{pi}\dd{qj}\dd{sa}\dd{rb}  \nonumber \\ 
&- \dd{pj}\dd{qi}\dd{sb}\dd{ra} + \dd{pj}\dd{qi}\dd{sa}\dd{rb}  + ... \nonumber \\
%
%
\{a_a^\dagger a_b^\dagger a_j a_i\} \{a_p^{\dagger}a_q^\dagger a_s a_r \}  &= \{a_a^\dagger a_b^\dagger a_j a_i a_p^{\dagger}a_q^\dagger a_s a_r \} = \{a_p^{\dagger}a_q^\dagger a_s a_r a_a^\dagger a_b^\dagger a_j a_i\}
%
%
\end{align}
%
The non-contracted term cancels, but we are only interested in the fully contracted terms
\begin{align}
[\hh{V}_N,\hh{T}_2] &= \frac{1}{16} \sum_{ijab} \ql[\vva{ij}{ab} - \vva{ij}{ba} - \vva{ji}{ab} + \vva{ji}{ba} \qr] t_{ij}^{ab} \nonumber \\
&= \frac{1}{4} \sum_{ijab} \vva{ij}{ab} t_{ij}^{ab}
 \label{eq:commutation V_N T_2 final}
\end{align}
An the contribution to the energy

\begin{empheq}[box=\fbox]{equation}
\bra{\Phi_0} [\hh{H}_N,\hh{T}_2] \ket{\Phi_0} =\frac{1}{4} \sum_{ijab} \vva{ij}{ab}t_{ij}^{ab} \rightarrow E_{\text{CCSD}}^{(3)} 
 \label{eq:third term energy}
\end{empheq}

\subsubsection*{Fourth term}
\begin{equation}
\ql[\ql[\hh{H}_N, \hh{T}_1\qr], \hh{T}_1\qr] =  \ql[\ql[\hh{F}_N, \hh{T}_1\qr], \hh{T}_1\qr] +  \ql[\ql[\hh{V}_N, \hh{T}_1\qr], \hh{T}_1\qr]  
 \label{def:fourth commutator} 
\end{equation}
%
From Eqs. (\ref{def:T_2}) and (\ref{eq:second term F_N T_1 commuation}) we obtain these expressions

\begin{equation}
\ql[\hh{F}_N,\hh{T}_1\qr]\hh{T}_1 = \sum_{qia}\sum_{jb}  f_q^i t_i^a t_j^b \{ a_q a_a^{\dagger}\}\{a_b^{\dagger} a_j\} + \sum_{pia}\sum_{jb}   f_a^p t_i^a t_j^b\{a_p^{\dagger} a_i\} \{a_b^{\dagger} a_j\} +  \sum_{ia}\sum_{jb}  f_i^a t_a^i t_j^b \{a_b^{\dagger} a_j \} 
 \label{eq:[F_N,T_1]T_1}
\end{equation}
\begin{equation}
\hh{T}_1 \ql[\hh{F}_N,\hh{T}_1\qr]= \sum_{qia}\sum_{jb}  f_q^i t_i^a t_j^b \{a_b^{\dagger} a_j\} \{a_q a_a^{\dagger}\}+ \sum_{pia}\sum_{jb}   f_a^p t_i^a t_j^b\{a_b^{\dagger} a_j\} \{a_p^{\dagger} a_i\} +  \sum_{ia}\sum_{jb}  f_i^a t_a^i t_j^b \{a_b^{\dagger} a_j \} 
 \label{eq:T_1[F_N,T_1]}
\end{equation}
%
The last two terms cancels each other in a commutation, and all possible contractions is zero. In Eq.  (\ref{eq:T_1[F_N,T_1]}) the particle creation operator $a_b^\dagger$ gives zero in contraction with a particle or hole operator from the left, because of Eq. (\ref{rel:3}). And a in Eq. (\ref{eq:[F_N,T_1]T_1}) the contractions between a particle and hole operator pairs always gives zero. Therefore there is no contribution to the energy from this part of the fourth term.

\begin{align}
\ql[\hh{V}_N, \hh{T}_1\qr]\hh{T}_1  &=  \frac{1}{4}\sum_{pqrs}\sum_{ia}\sum_{jb} \bra{pq}|v|\ket{rs} t_i^at_j^b \{a_p^{\dagger}a_q^{\dagger}a_s a_r\}\{a_a^\dagger a_i\}\{a_b^\dagger a_j\} \\
&-\frac{1}{4} \sum_{pqrs}\sum_{ia}\sum_{jb} \vva{pq}{rs} t_i^at_j^b \{a_a^\dagger a_i\} \{a_p^{\dagger}a_q^{\dagger}a_s a_r\} \{a_b^\dagger a_j\} \label{eq:[V_N,T_1]T_1 1}\\
%
%
%
\hh{T}_1\ql[\hh{V}_N, \hh{T}_1\qr]  &= \frac{1}{4}\sum_{pqrs}\sum_{ia}\sum_{jb} \bra{pq}|v|\ket{rs} t_i^at_j^b \{a_b^\dagger a_j\} \{a_p^{\dagger}a_q^{\dagger}a_s a_r\}\{a_a^\dagger a_i\}\\ 
&-  \frac{1}{4}\sum_{pqrs}\sum_{ia}\sum_{jb} \bra{pq}|v|\ket{rs} t_i^at_j^b \{a_a^\dagger a_i\}\{a_b^\dagger a_j\} \{a_p^{\dagger}a_q^{\dagger}a_s a_r\}
 \label{eq:[V_N,T_1]T_1 2}
\end{align}
%
We see that the only term that contribute to the expectation value is the first term in Eq. (\ref{eq:[V_N,T_1]T_1 1}). All the other terms have a particle creation operator in the leftmost operator string, and the contrations with this operator therefore leads to zero from Eq. (\ref{rel:3}). Using the generalized Wick's theorem on the first term in Eq. (\ref{eq:[V_N,T_1]T_1 1}) 
%
\begin{align}
\contraction[4ex]{\{a_p^{\dagger}a_q^\dagger a_s a_r \}\{a_a^\dagger a_i\}\{a_b^\dagger a_j\} =  \{ }{a}{{}_p^{\dagger}a_q^\dagger a_s a_r a_a^\dagger a_i a_b^\dagger }{a}
\contraction[3ex]{\{a_p^{\dagger}a_q^\dagger a_s a_r \}\{a_a^\dagger a_i\}\{a_b^\dagger a_j\} = \{a_p^{\dagger}}{a}{{}_q^\dagger a_s a_r a_a^\dagger}{a}
\contraction[2ex]{\{a_p^{\dagger}a_q^\dagger a_s a_r \}\{a_a^\dagger a_i\}\{a_b^\dagger a_j\} = \{ a_p^{\dagger} a_q^\dagger}{a}{a_r a_a^\dagger a_i .}{a}
\contraction{\{a_p^{\dagger}a_q^\dagger a_s a_r \}\{a_a^\dagger a_i\}\{a_b^\dagger a_j\} = \{ a_p^{\dagger} a_q^\dagger a_s}{a}{{}_r}{a}
%
%
\contraction[4ex]{\{a_p^{\dagger}a_q^\dagger a_s a_r \}\{a_a^\dagger a_i\}\{a_b^\dagger a_j\} = \{a_p^{\dagger}a_q^\dagger a_s a_r a_a^\dagger a_i a_b^\dagger a_j\} + \{ }{a}{{}_p^{\dagger}a_q^\dagger a_s a_r a_a^\dagger a_i a_b^\dagger }{a}
\contraction[3ex]{\{a_p^{\dagger}a_q^\dagger a_s a_r \}\{a_a^\dagger a_i\}\{a_b^\dagger a_j\} = \{a_p^{\dagger}a_q^\dagger a_s a_r a_a^\dagger a_i a_b^\dagger a_j\} + \{a_p^{\dagger}}{a}{{}_q^\dagger a_s a_r a_a^\dagger}{a}
\contraction[2ex]{\{a_p^{\dagger}a_q^\dagger a_s a_r \}\{a_a^\dagger a_i\}\{a_b^\dagger a_j\} = \{a_p^{\dagger}a_q^\dagger a_s a_r a_a^\dagger a_i a_b^\dagger a_j\} + \{ a_p^{\dagger} a_q^\dagger}{a}{a_r .}{a}
\contraction{\{a_p^{\dagger}a_q^\dagger a_s a_r \}\{a_a^\dagger a_i\}\{a_b^\dagger a_j\} = \{a_p^{\dagger}a_q^\dagger a_s a_r a_a^\dagger a_i a_b^\dagger a_j\} + \{ a_p^{\dagger} a_q^\dagger a_s}{a}{a_a^\dagger a_i .}{a}
%
%
\{a_p^{\dagger}a_q^\dagger a_s a_r \}\{a_a^\dagger a_i\}\{a_b^\dagger a_j\} &=   \{a_p^{\dagger}a_q^\dagger a_s a_r a_a^\dagger a_i a_b^\dagger a_j\} + \{a_p^{\dagger}a_q^\dagger a_s a_r a_a^\dagger a_i a_b^\dagger a_j\} \\
%
%
\contraction[4ex]{+ \{}{a}{{}_p^{\dagger}a_q^\dagger a_s a_r a_a^\dagger}{a}
\contraction[3ex]{+ \{ a_p^\dagger}{a}{{}_q^\dagger a_s a_r  a_a^\dagger a_i a_b^\dagger}{a}
\contraction[2ex]{+ \{a_p^\dagger a_q^\dagger}{a}{a_r  a_a^\dagger a_i .}{a}
\contraction{+ \{ a_p^\dagger a_q^\dagger a_s}{a}{{}_r}{a}
%
%
\contraction[4ex]{+ \{a_p^{\dagger}a_q^\dagger a_s a_r a_a^\dagger a_i a_b^\dagger a_j\} + \{}{a}{{}_p^{\dagger}a_q^\dagger a_s a_r a_a^\dagger}{a}
\contraction[3ex]{+ \{a_p^{\dagger}a_q^\dagger a_s a_r a_a^\dagger a_i a_b^\dagger a_j\} + \{ a_p^\dagger}{a}{{}_q^\dagger a_s a_r  a_a^\dagger a_i a_b^\dagger}{a}
\contraction[2ex]{+ \{a_p^{\dagger}a_q^\dagger a_s a_r a_a^\dagger a_i a_b^\dagger a_j\} +  \{a_p^\dagger a_q^\dagger}{a}{a_r . }{a}
\contraction{+ \{a_p^{\dagger}a_q^\dagger a_s a_r a_a^\dagger a_i a_b^\dagger a_j\} + \{ a_p^\dagger a_q^\dagger a_s}{a}{{a}_r a_i^\dagger . }{a}
%
%
&+ \{a_p^{\dagger}a_q^\dagger a_s a_r a_a^\dagger a_i a_b^\dagger a_j\} + \{a_p^{\dagger}a_q^\dagger a_s a_r a_a^\dagger a_i a_b^\dagger a_j\} + ...\\
%
%
&= -\dd{pj}\dd{qi}\dd{ra}\dd{sb} + \dd{pj}\dd{qi}\dd{rb}\dd{sa} + \dd{pi}\dd{qj}\dd{ra}\dd{sb} - \dd{pi}\dd{qj}\dd{rb}\dd{sa} + ...
\end{align}
%
The expectation value of the fourth term is then

\begin{align}
\bra{\Phi_0} [[\hh{V}_N,\hh{T}_1 ],\hh{T}_1 ]\ket{\Phi_0} &= \frac{1}{4} \sum_{ia}\sum_{jb} t_i^a t_j^b \ql[\vva{ij}{ba} - \vva{ij}{ab} - \vva{ji}{ba} + \vva{ji}{ab} \qr] \\
&= \sum_{ijab} t_i^a t_j^b\vva{ij}{ab} 
 \label{eq:commutator expectation 4th}
\end{align}
%
The contribution to the energy is then

\begin{empheq}[box=\fbox]{equation}
\bra{\Phi_0} \frac{1}{2} [[\hh{H}_N,\hh{T}_1 ],\hh{T}_1 ] \ket{\Phi_0} =\frac{1}{2} \sum_{ijab} t_i^a t_j^b\vva{ij}{ab}  \rightarrow E_{\text{CCSD}}^{(4)}
 \label{eq:fourth term energy}
\end{empheq}

\subsubsection*{Fifth term}

\begin{equation}
[[\hh{H}_N,\hh{T}_2 ],\hh{T}_2] = [[\hh{F}_N,\hh{T}_2 ],\hh{T}_2] +  [[\hh{V}_N,\hh{T}_2 ],\hh{T}_2] 
 \label{eq:fith term}
\end{equation}
%
The expectation value of $[\hh{F}_N,\hh{T}_2 ]$ is zero Eq. (\ref{eq:F_N T_2}), and we are left with finding $[\hh{V}_N,\hh{T}_2 ]\hh{T}_2$,

\begin{align}
[\hh{V}_N,\hh{T}_2 ]\hh{T}_2 &= \frac{1}{16} \sum_{pqrs}\sum_{abij}\sum_{cdkl}\vva{pq}{rs}t_{ij}^{ab}t_{kl}^{cd} \{a_p^\dagger a_q^\dagger a_s a_r\}\{a_a^\dagger a_b^\dagger a_j a_i\}\{a_c^\dagger a_d^\dagger a_l a_k\} \\
&- \frac{1}{16} \sum_{pqrs}\sum_{abij}\sum_{cdkl}\vva{pq}{rs}t_{ij}^{ab}t_{kl}^{cd} \{a_a^\dagger a_b^\dagger a_j a_i\}\{a_p^\dagger a_q^\dagger a_s a_r\}\{a_c^\dagger a_d^\dagger a_l a_k\} 
\end{align}
%
There are in total four hole-annihilation operators $a_s,a_r,a_k,a_l$ and only two-hole creation operators $a_p^\dagger,a_q^\dagger$, this will not be fully contracted. For the same reason $\hh{T}_2[\hh{V}_N,\hh{T}_2 ]$ will not contribute to the energy.

\begin{empheq}[box=\fbox]{equation}
\bra{\Phi_0} \frac{1}{2} [[\hh{H}_N,\hh{T}_2 ],\hh{T}_2] \ket{\Phi_0} = 0  \rightarrow E_{\text{CCSD}}^{(5)}
 \label{eq:fifth term energy}
\end{empheq}

\subsubsection*{Sixth term}
The mixed term $\hh{H}_N\hh{T}_1\hh{T}_2$ have three annihilation-creation pairs (one from $\hh{T}_1$ and two from $\hh{T}_2$), while $\hh{H}_N$ only have two. This term can therefore never be fully contracted and gives no contribution to the energy

\begin{empheq}[box=\fbox]{equation}
\bra{\Phi_0} \frac{1}{2} [[\hh{H}_N,\hh{T}_1 ],\hh{T}_2 ] \ket{\Phi_0} = 0 \rightarrow E_{\text{CCSD}}^{(6)}
 \label{eq:sixth  term energy}
\end{empheq}
%

\subsubsection{Final term}
The final expression becomes 

\begin{empheq}[box=\fbox]{equation}
E_{\text{CCSD}} - E_0 =  \sum_{ia}f_a^i t_i^a + \frac{1}{4} \sum_{ijab} \vva{ij}{ab}t_{ij}^{ab} + \frac{1}{2} \sum_{ijab} \vva{ij}{ab} t_i^a t_j^b 
 \label{eq:final energy}
\end{empheq}
%
This equation is valid for CCSDT and other schemes. Since the higher terms $\hh{T}_3$ and $\hh{T}_4$ etc. cannot produce fully contracted terms with the two-body Hamiltonian as we have seen. However the higher order excitation operators contributes through the amplitude equations Eqs. (\ref{eq:mathematical foresight 2})-(\ref{eq:mathematical foresight 3}).

\section{Introduction to Coupled Cluster Diagrams}
Even though Wick's theorem have simplyfied the method of evaluating expectation values. However the algebraic approach to the amplitude equations would involve even longer operator strings and be very time consuming. Another method is the digramatic approach popularized by Kucharski and Bartlett \cite{bartlett-kucharski}. We will use their approach to derive the energy equation Eq. (\ref{eq:final energy}) and the amplitude equations.
   The particle-hole formalism is still in use, lines represent a particle or a hole with respect to the reference state $\ket{\Phi_0}$. Lines with the downard arrows represent the hole states. And lines with the upward arrows represent the particle states (see Fig.  \ref{fig:particle-hole representation}).  

\begin{figure}[h!]
\centering
\input{Figurer/Tikz/particle-hole-representation}  %latex'en klarer ikke mellomrom!!!!
\caption{Representation of particle and holes}
 \label{fig:particle-hole representation}
\end{figure}
%
Notice that the covention is to write the rightmost particle/hole first, this have no physical significance since we have a sum over all particles and holes. Its just a phase factor of (-1) difference, $\ket{\Phi_{ij}^{ab}} = -\ket{\Phi_{ji}^{ab}} = \ket{\Phi_{ji}^{ab}}$. Since we already have defined the directions of particles and holes, we want to be consistent with the algebraic ordering of the operators, and have a convecntion that the rightmost operators starts always at the bottom this page which is the reference state $\ket{\Phi_0}$. 

The matrix elements of an interaction with a one-body operator $\bra{b}h\ket{a}$ is represented with a vertex (the black dot), $\bullet---\times$. Where the dashed line indicate our interaction potential.

\begin{figure}[h!]
\centering
\subfigure[$\equiv \bra{b}u\ket{a}\{a_b^\dagger a_a \}$]{
\input{Figurer/Tikz/oneparticle-pp}   %HUSK ALLTID \input() P� .TEX FILER!!!!!
\label{fig:oneparticle-pp}
}
\subfigure[$\equiv \bra{j}u\ket{i}\{a_i^\dagger a_j \}$]{
\input{Figurer/Tikz/oneparticle-hh}
\label{fig:oneparticle-hh}
}
\subfigure[$\equiv \bra{a}u\ket{i}\{a_a^\dagger a_i \}$]{
\input{Figurer/Tikz/oneparticle-ph}
\label{fig:oneparticle-ph}
}
\subfigure[$\equiv \bra{i}u\ket{a}\{a_i^\dagger a_a \}$]{
\input{Figurer/Tikz/oneparticle-hp}
\label{fig:oneparticle-hp}
}

\caption[Optional caption for list of figures]{Diagramatic representation of one-body operators. As we can see from Fig. \ref{fig:one particle operators} the quasi-creation operators lie above the interaction line while the quasi-annihilation operators is below.}
\label{fig:one particle operators}
\end{figure}

\subsection*{Diagram rules part 1}

\begin{quote}
{\bf Rule 1:} \em When expressing operators: Any unlabeled particle/hole-line is summed over
\end{quote}

\begin{quote}
{\bf Rule 2:} \em Incoming lines in a vertex $\bullet---\times$ represent an annihilation operator and correspond to the $\ket{}$ $ket$-part of the matrix element. Outgoing lines represent an creation operator and correspond to the $\bra{}$ $bra$-part of the matrix element. %And the rightmost incoming/outgoing line is always written first, 

\begin{equation}
\bra{\text{leftmost-out ... rightmost-out}}u\ket{\text{leftmost-in ...  rightmost-in}}
\end{equation}

\end{quote}

\begin{quote}
{\bf Rule 3:} \em Lines that follow the same direction can be contracted. And a phase factor of $-1$ is multiplied when contracting two holes
\end{quote} 

\begin{figure}[h!]
\centering
\subfigure[{$\equiv \contraction{}{a}{{}_a}{a}a_a a_b^\dagger$}]{
\input{Figurer/Tikz/contraction-pp}  %HUSK ALLTID \input() P� .TEX FILER!!!!!
\label{fig:contraction-pp}
}
\hspace{4cm}
\subfigure[{$\equiv \contraction{}{a}{{}_i^\dagger}{a} a_i^\dagger a_j$}]{
\input{Figurer/Tikz/contraction-hh}
\label{fig:contraction-hh}
}
\label{fig:one particle contractions}
\caption{Non-zero contractions}
\end{figure}
%
\noindent Our one-body operator $\hat{F}_N$ Eq. (\ref{def:F_N}) is then a sum of all the different diagrams in Fig. \ref{fig:one particle operators} and over all the indices. 

\begin{align}
\hh{F}_N &=& \sum_{ab} f_b^a \{a_a^\dagger a_b \} \quad &+& \sum_{ij} f_j^i \{a_i^\dagger a_j \} \quad &+& \sum_{ia} f_a^i \{a_i^\dagger a_a \}\quad &+& \sum_{ai} f_i^a \{a_a^\dagger a_i \}   \nonumber \\
%%
&\equiv& \begin{matrix}\scalebox{0.6}{\input{Figurer/Tikz/oneparticle-pp-noindex}}\\ \epsilon_1 = 0\end{matrix} \quad &+& \begin{matrix}\scalebox{0.6}{\input{Figurer/Tikz/oneparticle-hh-noindex}} \\ \epsilon_2 = 0 \end{matrix} \quad  &+&
\begin{matrix}\scalebox{0.6}{\input{Figurer/Tikz/oneparticle-hp-noindex}}\\ \\ \\\epsilon_3 = -1\end{matrix} \quad  &+&
\begin{matrix}\scalebox{0.6}{\input{Figurer/Tikz/oneparticle-ph-noindex}}\\ \\ \\ \epsilon_4 = +1\end{matrix} \label{diagram:F_N} \\
\end{align}
%
$\epsilon_n$ is the excitation level of diagram nr $n$ and defined by the difference between the number of quasi-creation operators $\mathcal{N}_c$ and the quasi-annihilation operators $\mathcal{N}_a$ divided by 2.

\begin{equation}
\epsilon = \frac{\mathcal{N}_c - \mathcal{N}_a}{2}
 \label{def:excitation level}
\end{equation}
%
A two-body operator will be represented by two vertices, $\bullet --- \bullet$ and the matrix elements are defined as before by diagram the rule 2. The diagramatic representation of $\hh{V}_N$ then becomes.

\begin{align}
\hh{V}_N &=& \frac{1}{4} \sum_{abcd} \vva{ab}{cd} \{a_a^\dagger a_b^\dagger a_d a_c \} \,\, &+& \frac{1}{4} \sum_{ijkl} \vva{ij}{kl} \{a_i^\dagger a_j^\dagger a_l a_k \} \, &+&  \sum_{iabj} \vva{ia}{bj} \{a_i^\dagger a_a^\dagger a_j a_b \} \nonumber \\
%%
&+& \frac{1}{2} \sum_{aibc} \vva{aj}{bc} \{ a_a^\dagger a_i^\dagger a_c a_b \} \,\,  &+& \frac{1}{2} \sum_{ijka} \vva{ij}{ka} \{ a_i^\dagger a_j^\dagger a_a a_k \} \,\,   &+& \frac{1}{2} \sum_{abci} \vva{ab}{ci} \{ a_a^\dagger a_b^\dagger a_i a_c \} \nonumber \\
%%
&+& \frac{1}{2} \sum_{iakl} \vva{ia}{kj} \{ a_i^\dagger a_a^\dagger a_k a_j \} \,\,   &+& \frac{1}{4} \sum_{abij} \vva{ab}{ij} \{ a_a^\dagger a_b^\dagger a_j a_i \} \,\,  &+& \frac{1}{4} \sum_{ijab} \vva{ij}{ab} \{ a_i^\dagger a_j^\dagger a_b a_a \}  \nonumber \\
%%
&\equiv& \begin{matrix}\scalebox{0.6}{\input{Figurer/Tikz/V1}}\\ \epsilon_1 = 0 \end{matrix} \quad  &+& 
\begin{matrix}\scalebox{0.6}{\input{Figurer/Tikz/V2}}\\ \epsilon_2 = 0 \end{matrix} \quad  &+&
\begin{matrix}\scalebox{0.6}{\input{Figurer/Tikz/V3}}\\ \epsilon_3 = 0 \end{matrix} \quad  \nonumber \\
%%
&+& \begin{matrix}\scalebox{0.6}{\input{Figurer/Tikz/V4}}\\ \epsilon_4 = -1 \end{matrix} \quad  &+& 
\begin{matrix}\scalebox{0.6}{\input{Figurer/Tikz/V5}}\\ \epsilon_5 = -1 \end{matrix} \quad  &+&
\begin{matrix}\scalebox{0.6}{\input{Figurer/Tikz/V6}}\\ \epsilon_6 = +1 \end{matrix} \quad  \nonumber \\
%%
&+& \begin{matrix}\scalebox{0.6}{\input{Figurer/Tikz/V7}}\\ \epsilon_7 = +1 \end{matrix} \quad  &+& 
\begin{matrix}\scalebox{0.6}{\input{Figurer/Tikz/V8}}\\\\\\ \epsilon_8 = +2 \end{matrix} \quad  &+&
\begin{matrix}\scalebox{0.6}{\input{Figurer/Tikz/V9}}\\\\\\ \epsilon_9 = -2 \end{matrix} \quad  \nonumber \\
\label{diagram:V_N}
\end{align}
%
Notice that the $\hh{V}_N$ is a sum over indices $p,q,r,s$ which could either be a particle or hole, we would have $2^4=16$ sums, but some of the sums are equivalent. For instance we have used that 

\begin{align}
\frac{1}{4} \sum_{iabj} \vva{ia}{bj} \{a_i^\dagger a_a^\dagger a_j a_b \} \quad &+& \frac{1}{4} \sum_{iabj} \vva{ai}{bj} \{a_a^\dagger a_i^\dagger a_j a_b \} \quad  &+& \frac{1}{4} \sum_{iabj} \vva{ia}{jb} \{a_i^\dagger a_a^\dagger a_b a_j \} \nonumber \\
%%
&+& \frac{1}{4} \sum_{iabj} \vva{ai}{bj} \{a_a^\dagger a_i^\dagger a_b a_j \} \quad  &=&  \sum_{iabj} \vva{ai}{jb} \{a_i^\dagger a_a^\dagger a_j a_b \}
 \label{proof:V_N term 3} 
\end{align}
%
Which is term 3 in $\hh{V}_N$. Each vertex is unique $A\bullet --- \bullet B$, i.e. we can therefore permute the lines leaving or entering the vertex $A$ or $B$, it will give different diagrams but the same matrix element with a phase factor. For example, the third diagram in $\hh{V}_N$ which is the sum over $\vva{ia}{jb}\{a_i^\dagger a_a^\dagger a_j a_b \}$ can be written in four different ways which is the diagramatic equivalent of Eq. (\ref{proof:V_N term 3}).

\begin{align}
\begin{matrix}\scalebox{0.6}{\input{Figurer/Tikz/V3proof1}}\end{matrix} =
\begin{matrix}\scalebox{0.6}{\input{Figurer/Tikz/V3proof4}}\end{matrix} = -
\begin{matrix}\scalebox{0.6}{\input{Figurer/Tikz/V3proof2}}\end{matrix} = - 
\begin{matrix}\scalebox{0.6}{\input{Figurer/Tikz/V3proof3}}\end{matrix} 
\label{diagram:V_N proof}
\end{align}
%
The diagrams are antisymmetric with respect to permutation of hole and particle pairs. Permutation means placing them in different vertices. 

In addition we have the cluster operator $\hh{T} = \hh{T}_1 + \hh{T}_2$ 

\begin{align}
\hh{T} &=& \sum_{ia} t_i^a \{ a_a^\dagger a_i \} \qquad &+& \frac{1}{4} \sum_{ijab} t_{ij}^{ab} \{ a_a^\dagger a_b^\dagger a_j a_i \} \nonumber \\
&\equiv& \begin{matrix}\scalebox{0.6}{\input{Figurer/Tikz/T1}}\\ \epsilon_{\hh{T}_1} = +1 \end{matrix}  \qquad &+& 
\begin{matrix}\scalebox{0.6}{\input{Figurer/Tikz/T2}}\\ \epsilon_{\hh{T}_2} = +2 \end{matrix}
 \label{diagram:cluster operator}
\end{align}
%
Where the interaction is represented by a solid bar ''\, --- \,``. 

Now that we have established diagrams for the operators, we can now show how the matrix elements of operatos between SDs are represented.

\begin{align}
\bra{\Phi_i^a}\hh{T}_1\ket{\Phi_0} = \quad \begin{matrix}\scalebox{0.8}{\input{Figurer/Tikz/T1matrix1}}\end{matrix} \rightarrow \begin{matrix}\scalebox{0.8}{\input{Figurer/Tikz/T1matrix2}}\end{matrix}
 \label{diagram:T1matrix}
\end{align}
%
As mentioned earlier the diagrams are read from bottom to top. We have the reference state on bottom, which is \emph{white space} followed by an N-body operator with an interaction in the middle and the finally the $bra$-state on top.

\begin{align}
\bra{\Phi_{ij}^{ab}}\hh{T}_1\ket{\Phi_0} = \quad \begin{matrix}\scalebox{0.8}{\input{Figurer/Tikz/T2matrix1}}\end{matrix} \rightarrow \begin{matrix}\scalebox{0.8}{\input{Figurer/Tikz/T2matrix2}}\end{matrix}
 \label{diagram:T2matrix}
\end{align}
%
All the lines in the N-body operator have to be contracted in order to give nonzero matrix elements, additional particle/hole lines which are not involved in the interaction will be written on the right side.

\begin{align}
\bra{\Phi_{ikl}^{acd}}\hh{V}_N\ket{\Phi_{jkl}^{bcd}} = \quad \begin{matrix}\scalebox{0.8}{\input{Figurer/Tikz/V3matrix}}\end{matrix}
 \label{diagram:V3matrix}
\end{align}
%
Diagram 3 in $\hh{V}_N$ is therefore the only nonzero contribution for the given the SD. Diagrams that represents the energy equations can be created in the same way. We must remember our reference state is white space, so we cannot have lines above or below our interaction, i.e. the \emph{excitation level}  have to be zero. This will be the condition that gives us the natural truncation to include only $\hh{T}^2$ diagrams.0 Since $\hh{T}^3$ have at least an excitation level of 3, but $\hh{H}_N$ have at an excitation level between $-2$ and $2$. 

\subsection{Energy Equation}

We can write out the commutators in Eq. (\ref{eq:bar H similarity series}) and only include those terms in which the Hamiltonian is to the left of the cluster operators. The reason for this is because of Wicks theorem, we cannot get a full contraction when we have a  particle creation operator $a_a^\dagger$ to the left, as we have seen an example of in Eqs. (\ref{def:T_1 F_1}) and (\ref{eq:T_1 F_1 contraction}), i.e. $\bra{\Phi_0}\hh{T}_1\hh{H}_N \ket{\Phi_0} = 0$.

\begin{align}
\bb{H}_c = \hh{H}_N + \hh{H}_N\hh{T}_1 + \hh{H}_N\hh{T}_2 + \frac{1}{2} \hh{H}_N \hh{T}_1^2 + \frac{1}{2} \hh{H}_N \hh{T}_2^2 + \hh{H}_N \hh{T}_1 \hh{T}_2
 \label{eq:rewrite bar H similartiy series}
\end{align}
%
The subscript $c$ indicates that we have a \emph{connceted cluster} form of the similarity-transformed Hamiltonian. 

Since both $\hh{T}_2$ and $\hh{T}_1\hh{T}_2$ have excitation levels bigger than $+2$, we would not get contributions from the last two terms.  We are going to consider the first nonzero contribution to the coupled cluster energy

\begin{equation}
E_{CCSD} - E_0 = \bra{\Phi_0} \hh{H}_N \hh{T}_1 \ket{\Phi_0} + \bra{\Phi_0} \hh{H}_N \hh{T}_2   \ket{\Phi_0} + \frac{1}{2} \bra{\Phi_0} \hh{H}_N \hh{T}_1^2 \ket{\Phi_0}   
 \label{def:CCSD energy c}
\end{equation}
%
A $\hh{T}_1$-diagram have an excitation level of $+1$, we have to combine this with a diagram in $\hh{H}_N$ that have have an excitation level of $-1$ to get an nonzero contribution. There are several diagrams in $\hh{F}_N$ and $\hh{V}_N$ that fullfill this criteria, but only diagram 3 in $\hh{F}_N$ that have the reference state on top.

\begin{align}
\bra{\Phi_0} \hh{F}_N \hh{T}_1 \ket{\Phi_0} = \quad \begin{matrix}\epsilon = -1 \\ \scalebox{0.6}{\input{Figurer/Tikz/F_NT_1part1}}\\ \epsilon = +1\end{matrix} \quad \rightarrow \qquad \begin{matrix}\scalebox{0.6}{\input{Figurer/Tikz/F_NT_1part2}} \\ \epsilon = 0\end{matrix}
\label{diagram:F_NT_1}
\end{align}
%
Next we consider $\bra{\Phi_0} \hh{H}_N \hh{T}_2 \ket{\Phi_0}$, here the $\hh{T}_2$ have an excitation level of $+2$, which has to be combined with diagram 9 in $\hh{V}_N$ with an excitation level of $-2$ and reference state on top.

\begin{align}
\bra{\Phi_0} \hh{V}_N \hh{T}_2 \ket{\Phi_0} = \quad \begin{matrix}\epsilon = -2 \\ \scalebox{0.6}{\input{Figurer/Tikz/V_NT_2part1}}\\ \epsilon = +2\end{matrix} \qquad \rightarrow \qquad \begin{matrix}\scalebox{0.6}{\input{Figurer/Tikz/V_NT_2part2}} \\ \epsilon = 0\end{matrix}
\label{diagram:V_NT_2}
\end{align}
%
The only contribution left is $\bra{\Phi_0} \hh{H}_N \hh{T}_1^2 \ket{\Phi_0}$, the excitation level of $\hh{T}_1$ is the same as the sum of each which is $+2$. This can only be coupled to diagram 9 in $\hh{V}_N$. 

\begin{align}
\frac{1}{2} \bra{\Phi_0} \hh{V}_N \hh{T}_1^2 \ket{\Phi_0} = \quad \begin{matrix}\epsilon = -2 \\ \scalebox{0.6}{\input{Figurer/Tikz/V_NT_1T_1part1}}\\ \epsilon = +2\end{matrix} \qquad \rightarrow \qquad \begin{matrix}\scalebox{0.6}{\input{Figurer/Tikz/V_NT_1T_1part2}} \\ \epsilon = 0\end{matrix}
\label{diagram:V_NT_1^2}
\end{align}
%
The factor $1/2$ seem to be very arbritrary, but the Diagrammatic rules are consistent with using Wick's theorem and this factor is taken care of when we imply the diagrammatic rules. The energy equation in the diagramatic form is then

\begin{align}
E_{CCSD} - E_0 = \quad \begin{matrix}\scalebox{0.6}{\input{Figurer/Tikz/CCSDenergypart1}}\end{matrix} \quad + \qquad \begin{matrix}\scalebox{0.6}{\input{Figurer/Tikz/CCSDenergypart2}}\end{matrix} \quad + \quad \begin{matrix}\scalebox{0.6}{\input{Figurer/Tikz/CCSDenergypart3}}\end{matrix}
\label{diagram:E_CCSD}
\end{align}
%
Haven written out the energy equation in a diagramatic form we now want to translate this into the algebraic equations we got from Eq. (\ref{eq:final energy}). First we have to introduce some rules to how diagrams can be interpreted.

\subsection*{Diagram rules part 2}

\begin{quote}
{\bf Rule 4:} \em Label the hole lines with indices \boldmath$ijk$.., and particle lines with indices  \boldmath$abc$..
\end{quote} 

\begin{quote}
{\bf Rule 5:} \em Each interaction line contributes with an matrix element $f_{in}^{out}$ or a amplitude $t_{ijk..}^{abc..}$, it is consistent with \text{\bf Rule 2} 
\end{quote} 

\begin{quote}
{\bf Rule 6:} \em Sum over all incdices that is associated with lines that begin and end at interaction lines (internal lines). 
\end{quote}

\begin{quote}
{\bf Rule 7:} \em The phase of the diagram is \boldmath$(-1)^{(l+h)}$, where \boldmath$h$ is number of hole lines. \boldmath$l$ is the number of \boldmath$loops$ that we have in our diagram. A \boldmath$loop$ is a route along a series of lines that returns to its beginning or begins at one external line and ends at another \cite{crawford}.
\end{quote}

\begin{figure}[h!]
\centering
\subfigure[$l=2$,\,\,$h=2$]{
  \scalebox{0.8}{\input{Figurer/Tikz/loopex1}}  
\label{fig:loop1}
}
\hspace{1cm}
\subfigure[$l=1$,\,\,$h=2$]{
 \scalebox{0.8}{\input{Figurer/Tikz/loopex2}}
\label{fig:loop2}
}
\caption[Optional caption for list of figures]{Examples of some loops}
\label{fig:loop examples}
\end{figure}

\begin{quote}
{\bf Rule 8:} \em Each pair of equivalent vertices and equivalent gives a factor of $\frac{1}{2}$ to be multiplied onto the algebraic expression. Particle/hole-lines that begins and end at the same interactions are  equivalent lines. Equivalent vertices are connected by two $\hh{T}_N$ operators to  $\hh{V}_N$ in the exact same way, i.e. same incoming and outgoing arrows for  
\end{quote}

\begin{quote}
{\bf Rule 9:} \em Each pair of \boldmath$unique$ external hole or particle lines give rise to a permutation operator $P(p,q)$. $Unique$ means that the external hole and particles enters/leaves different interactions lines. This rule makes the total diagram antisymmetric and includes the Pauli Principle which are already inclued in the interaction. 

\begin{equation} 
P(p,q)f(p,q) = f(p,q) - f(q,p)
 \label{def:permutation operator}
\end{equation}


\begin{figure}[h!]
\centering
  \scalebox{0.8}{ \input{Figurer/Tikz/unique-ex1}}  
\caption{\boldmath Here $i$ and $j$ enters two different interactions and they are therefore a $unique$ hole pair. We have to multiply with $P(i,j)$}
\label{fig:unique examples}
\end{figure}
%
\end{quote}
%
Now that we have established the diagram rules we can express them in an algebraic form

\begin{equation}
\begin{matrix}\scalebox{0.6}{\input{Figurer/Tikz/CCSDenergy1}}\end{matrix} = \sum_{ia} f_a^i t_i^a 
 \label{diagram:F_N algebraic}
\end{equation}
%
In this diagram we have two loops and two hole lines, one amplitude, one matrix elements and two internal indicies that have to be summed over which gives us a factor $+1$ in total.
\begin{equation}
\begin{matrix}\scalebox{0.6}{\input{Figurer/Tikz/CCSDenergy2}}\end{matrix} = \frac{1}{4} \sum_{ijab} \vva{ij}{ab} t_i^a 
 \label{diagram:V_NT_2 algebraic}
\end{equation}
%
In this diagram we have two loops and two hole lines, two amplitudes, one matrix element and four internal indicies that have to be summed over. And two pair of equivalent lines that gives us a factor $+\frac{1}{4}$ in total.

\begin{equation}
\begin{matrix}\scalebox{0.6}{\input{Figurer/Tikz/CCSDenergy3}}\end{matrix} = \frac{1}{2} \sum_{ijab} \vva{ij}{ab} t_i^a t_j^b
 \label{diagram:V_NT_1^2 algebraic}
\end{equation}
%
In this diagram we have two loops and two hole lines, two amplitudes, two matrix elements and four internal indicies that have to be summed over. And one pair of equivalent vertices that gives us a factor of $+\frac{1}{2}$
This is the same expressions as Eq. (\ref{eq:final energy}) when we used Wick's theorem.

\section{The Amplitude Equations}
We will use the same procedure for amplitude equations. The \emph{connected cluster} form of the similartiy-transformed Hamiltonian for both the singles and doubles equations

\begin{equation}
\bra{\Phi_i^a} \hh{H}_N \left(1 + \hh{T}_1 + \hh{T}_2 + \frac{1}{2} \hh{T}_1^2 + \hh{T}_1\hh{T}_2 + \frac{1}{3!} \hh{T}_1^3 \right) \ket{\Phi_0} = 0
\label{def:H_c for T_1}
\end{equation}

\begin{equation}
\bra{\Phi_{ij}^{ab}} \hh{H}_N \left(1 + \hh{T}_1 + \hh{T}_2 + \frac{1}{2} \hh{T}_1^2 + \hh{T}_1\hh{T}_2 + \frac{1}{2} \hh{T}_2^2 + \frac{1}{2} \hh{T}_1^2 \hh{T}_2 + \frac{1}{3!} \hh{T}_1^3 + \frac{1}{4!} \hh{T}_1^4 \right) \ket{\Phi_0} = 0
\label{def:H_c for T_2}
\end{equation}

\setlength{\fboxsep}{15pt}
\begin{figure}[h!]
  \centering
  \fbox{$\begin{matrix}\scalebox{0.4}{\input{Figurer/Tikz/S1}} 
& \scalebox{0.4}{\input{Figurer/Tikz/S2a}}
&\scalebox{0.4}{\input{Figurer/Tikz/S2b}}
&\scalebox{0.4}{\input{Figurer/Tikz/S2c}}
&\scalebox{0.4}{\input{Figurer/Tikz/S3a}}
\\S_1 & S_{2a} & S_{2b} & S_{2c} & S_{3a}\\
&\scalebox{0.4}{\input{Figurer/Tikz/S3b}}
&\scalebox{0.4}{\input{Figurer/Tikz/S3c}}
& \scalebox{0.4}{\input{Figurer/Tikz/S4a}}
& \scalebox{0.4}{\input{Figurer/Tikz/S4b}}
\\& S_{3b} & S_{3c} & S_{4a} & S_{4b}\\ 
&
& \scalebox{0.4}{\input{Figurer/Tikz/S4c}}
& \scalebox{0.4}{\input{Figurer/Tikz/S5a}}
& \scalebox{0.4}{\input{Figurer/Tikz/S5b}}
\\& & S_{4c} & S_{5a} & S_{5b}
\\
&
&
&\scalebox{0.4}{\input{Figurer/Tikz/S5c}}
&\scalebox{0.4}{\input{Figurer/Tikz/S6}}
\\& & & S_{5c} & S_{6}
\end{matrix}$}
  \caption{Diagrams of the $\hh{T}_1$ amplitude equation Eq. (\ref{def:H_c for T_1})}
  \label{fig:Diagrams of T1}
\end{figure}


\begin{table}[h!]
\caption{Algebraic expression of $\hh{T}_1$-amplitude diagrams from Fig. \ref{fig:Diagrams of T1}}
\centering
\begin{tabular}{c c c c l}
\hline\hline
Interaction & Contraction & $\epsilon$ & Diagram & Expression \\   
\hline
$\bra{\Phi_i^a}\hh{H}_N\ket{\Phi_0}$         &$\hh{F}_{N,4}$ & $+1$ & $S_1$     & $\quad \qquad f_i^a$\\
\hline
$\bra{\Phi_i^a}\hh{H}_N\hh{T}_1\ket{\Phi_0}$ &$\hh{F}_{N,1}$ & $0$  &$S_{2a}$   & $\begin{aligned}\quad \sum_c f_c^a t_i^c\end{aligned}$\\
$\bra{\Phi_i^a}\hh{H}_N\hh{T}_1\ket{\Phi_0}$ &$\hh{F}_{N,2}$ & $0$  &$S_{2b}$   & $\begin{aligned}\quad \sum_k f_i^k t_k^a\end{aligned}$\\
$\bra{\Phi_i^a}\hh{H}_N\hh{T}_1\ket{\Phi_0}$ &$\hh{V}_{N,3}$ & $0$  &$S_{2c}$   & $\begin{aligned}\quad \sum_{kc} \vva{ka}{ci} t_k^c\end{aligned}$\\
\hline
$\bra{\Phi_i^a}\hh{H}_N\hh{T}_2\ket{\Phi_0}$ &$\hh{F}_{N,3}$ & $-1$  &$S_{3a}$  & $\begin{aligned}\quad \sum_{kc}  f_c^k t_{ik}^{ac}\end{aligned}$\\
$\bra{\Phi_i^a}\hh{H}_N\hh{T}_2\ket{\Phi_0}$ &$\hh{V}_{N,4}$ & $-1$  &$S_{3b}$  & $\begin{aligned}\quad \frac{1}{2}\sum_{kcd}\vva{ka}{cd} t_{ki}^{cd}\end{aligned}$\\
$\bra{\Phi_i^a}\hh{H}_N\hh{T}_2\ket{\Phi_0}$ &$\hh{V}_{N,5}$ & $-1$  &$S_{3c}$  & $\begin{aligned}-\frac{1}{2}\sum_{klc}\vva{kl}{ci} t_{kl}^{ca}\end{aligned}$\\
\hline
$\bra{\Phi_i^a}\frac{1}{2}\hh{H}_N\hh{T}_1^2\ket{\Phi_0}$ &$\hh{F}_{N,3}$ & $-1$ & $S_{4a}$ & $\begin{aligned}-\sum_{kc} f_c^k t_i^c t_k^a \end{aligned}$\\
$\bra{\Phi_i^a}\frac{1}{2}\hh{H}_N\hh{T}_1^2\ket{\Phi_0}$ &$\hh{V}_{N,4}$ & $-1$ & $S_{4b}$ & $\begin{aligned}-\sum_{klc}\vva{kl}{ci} t_{k}^{c} t_{l}^{a}\end{aligned}$\\
$\bra{\Phi_i^a}\frac{1}{2}\hh{H}_N\hh{T}_1^2\ket{\Phi_0}$ &$\hh{V}_{N,5}$ & $-1$ & $S_{4c}$ & $\begin{aligned}\quad \sum_{kcd} \vva{ka}{cd} t_{k}^{c} t_{i}^{d}\end{aligned}$\\
\hline
$\bra{\Phi_i^a}\hh{H}_N\hh{T}_1\hh{T}_2\ket{\Phi_0}$ &$\hh{V}_{N,9}$ & $-2$ & $S_{5a}$ & $\begin{aligned} \quad \sum_{klcd} \vva{kl}{cd} t_{k}^{c} t_{li}^{da} \end{aligned}$\\
$\bra{\Phi_i^a}\hh{H}_N\hh{T}_1\hh{T}_2\ket{\Phi_0}$ &$\hh{V}_{N,9}$ & $-2$ & $S_{5b}$ & $\begin{aligned} -\frac{1}{2}\sum_{klcd} \vva{kl}{cd} t_{kl}^{ca} t_{d}^{i} \end{aligned}$\\
$\bra{\Phi_i^a}\hh{H}_N\hh{T}_1\hh{T}_2\ket{\Phi_0}$ &$\hh{V}_{N,9}$ & $-2$ & $S_{5c}$ & $\begin{aligned} -\frac{1}{2}\sum_{klcd} \vva{kl}{cd} t_{ki}^{cd} t_{l}^{a} \end{aligned}$\\
\hline
$\bra{\Phi_i^a}\frac{1}{3!}\hh{H}_N\hh{T}_1^3\ket{\Phi_0}$       &$\hh{V}_{N,9}$ & $-2$ &$S_{6}$   & $\begin{aligned} -\sum_{klcd} \vva{kl}{cd} t_{k}^{c} t_{i}^{d} t_{l}^{a} \end{aligned}$\\
\hline\hline
\end{tabular}
\label{table:T1 expressions}
\end{table}

\setlength{\fboxsep}{15pt}
\begin{figure}[h!]
  \centering
  \fbox{$\begin{matrix}\scalebox{0.35}{\input{Figurer/Tikz/D1}} 
& \scalebox{0.35}{\input{Figurer/Tikz/D2a}}
&\scalebox{0.35}{\input{Figurer/Tikz/D2b}}
&\scalebox{0.35}{\input{Figurer/Tikz/D2c}}
\\D_1 & D_{2a} & D_{2b} & D_{2c}\\
\scalebox{0.35}{\input{Figurer/Tikz/D2d}} 
&\scalebox{0.35}{\input{Figurer/Tikz/D2e}}
&\scalebox{0.35}{\input{Figurer/Tikz/D3a}}
&\scalebox{0.35}{\input{Figurer/Tikz/D3b}}
\\D_{2d} & D_{2e} & D_{3a} & D_{3b}\\
&\scalebox{0.35}{\input{Figurer/Tikz/D3c}} 
&\scalebox{0.35}{\input{Figurer/Tikz/D3d}}
&
\\& D_{3c} & D_{3d} & \\
\end{matrix}$}
  \caption{Diagrams of the $\hh{T}_2$ amplitude equation Eq. (\ref{def:H_c for T_2}), with $\hh{T}_2$ contractions only (CCD) }
  \label{fig:Diagrams of T2}
\end{figure}


\begin{table}[h!]
\caption{Algebraic expression of $\hh{T}_2$-amplitude diagrams from Fig. \ref{fig:Diagrams of T2}}
\centering
\begin{tabular}{c c c c l}
\hline\hline
Interaction & Contraction & $\epsilon$ & Diagram & \qquad Expression \\   
\hline
$\bra{\Phi_{ij}^{ab}}\hh{H}_N\ket{\Phi_0}$         &$\hh{V}_{N,8}$ & $+2$ & $D_1$     & $\quad \qquad \vva{ab}{ij}$\\
\hline
$\bra{\Phi_{ij}^{ab}}\hh{H}_N\hh{T}_2\ket{\Phi_0}$ &$\hh{F}_{N,1}$ & $0$  &$D_{2a}$   & $\begin{aligned}\quad P(ab) \sum_c f_c^b t_{ij}^{ac}\end{aligned}$\\
$\bra{\Phi_{ij}^{ab}}\hh{H}_N\hh{T}_2\ket{\Phi_0}$ &$\hh{F}_{N,2}$ & $0$  &$D_{2b}$   & $\begin{aligned} - P(ij) \sum_k f_j^k t_{ik}^{ab}\end{aligned}$\\
$\bra{\Phi_{ij}^{ab}}\hh{H}_N\hh{T}_2\ket{\Phi_0}$ &$\hh{V}_{N,1}$ & $0$  &$D_{2c}$   & $\begin{aligned}\quad \frac{1}{2} \sum_{cd} \vva{ab}{cd} t_{ij}^{cd}\end{aligned}$\\
$\bra{\Phi_{ij}^{ab}}\hh{H}_N\hh{T}_2\ket{\Phi_0}$ &$\hh{V}_{N,2}$ & $0$  &$D_{2d}$   & $\begin{aligned}\quad \frac{1}{2} \sum_{kl} \vva{kl}{ij} t_{kl}^{ab}\end{aligned}$\\
$\bra{\Phi_{ij}^{ab}}\hh{H}_N\hh{T}_2\ket{\Phi_0}$ &$\hh{V}_{N,3}$ & $0$  &$D_{2e}$   & $\begin{aligned} P(ij)P(ab) \sum_{kc} \vva{kb}{cj} t_{ik}^{ac}\end{aligned}$\\
\hline
$\bra{\Phi_{ij}^{ab}}\frac{1}{2}\hh{H}_N\hh{T}_2^2\ket{\Phi_0}$ &$\hh{V}_{N,9}$ & $-2$  &$D_{3a}$  & $\begin{aligned}\quad \frac{1}{4} \sum_{klcd} \vva{kl}{cd} t_{ij}^{ac} t_{kl}^{ab}\end{aligned}$\\
$\bra{\Phi_{ij}^{ab}}\frac{1}{2}\hh{H}_N\hh{T}_2^2\ket{\Phi_0}$ &$\hh{V}_{N,9}$ & $-2$  &$D_{3b}$  & $\begin{aligned}P(ij) \sum_{klcd}\vva{kl}{cd} t_{ik}^{ac}t_{jl}^{bd}\end{aligned}$\\
$\bra{\Phi_{ij}^{ab}}\frac{1}{2}\hh{H}_N\hh{T}_2^2\ket{\Phi_0}$ &$\hh{V}_{N,9}$ & $-2$  &$D_{3c}$  & $\begin{aligned}-\frac{1}{2}P(ij)\sum_{klcd}\vva{kl}{cd} t_{ik}^{dc}t_{lj}^{ab}\end{aligned}$\\
$\bra{\Phi_{ij}^{ab}}\frac{1}{2}\hh{H}_N\hh{T}_2^2\ket{\Phi_0}$ &$\hh{V}_{N,9}$ & $-2$  &$D_{3d}$  & $\begin{aligned}-\frac{1}{2}P(ab)\sum_{klcd}\vva{kl}{cd} t_{lk}^{ac}t_{ij}^{db}\end{aligned}$\\
\hline\hline
\end{tabular}
\label{table:T2 expressions}
\end{table}


\setlength{\fboxsep}{15pt}
\begin{figure}[h!]
  \centering
  \fbox{$\begin{matrix}\scalebox{0.35}{\input{Figurer/Tikz/D4a}} 
& \scalebox{0.35}{\input{Figurer/Tikz/D4b}}
&\scalebox{0.35}{\input{Figurer/Tikz/D5a}}
&\scalebox{0.35}{\input{Figurer/Tikz/D5b}}\\
%%
D_{4a} & D_{4b} & D_{5a} & D_{5b}\\
\scalebox{0.35}{\input{Figurer/Tikz/D5c}} 
& \scalebox{0.35}{\input{Figurer/Tikz/D6a}}
&\scalebox{0.35}{\input{Figurer/Tikz/D6b}}
&\scalebox{0.35}{\input{Figurer/Tikz/D6c}}\\
%%
D_{5c} & D_{6a} & D_{6b} & D_{6c}\\
\scalebox{0.35}{\input{Figurer/Tikz/D6d}} 
& \scalebox{0.35}{\input{Figurer/Tikz/D6e}}
&\scalebox{0.35}{\input{Figurer/Tikz/D6f}}
&\scalebox{0.35}{\input{Figurer/Tikz/D6g}}\\
%%
D_{6d} & D_{6e} & D_{6f} & D_{6g}\\
\scalebox{0.35}{\input{Figurer/Tikz/D6h}} 
& \scalebox{0.35}{\input{Figurer/Tikz/D7a}}
&\scalebox{0.35}{\input{Figurer/Tikz/D7b}}
&\scalebox{0.35}{\input{Figurer/Tikz/D7c}}\\
%%
D_{6h} & D_{7a} & D_{7b} & D_{7c}\\
\scalebox{0.35}{\input{Figurer/Tikz/D7d}} 
& \scalebox{0.35}{\input{Figurer/Tikz/D7e}}
&\scalebox{0.35}{\input{Figurer/Tikz/D8a}}
&\scalebox{0.35}{\input{Figurer/Tikz/D8b}}\\
%%
D_{7d} & D_{7e} & D_{8a} & D_{8b}\\
\scalebox{0.35}{\input{Figurer/Tikz/D9}}\\
D_9
\end{matrix}$}
  \caption{Diagrams of the $\hh{T}_2$ amplitude equation Eq. (\ref{def:H_c for T_2}), with $\hh{T}_1+\hh{T}_2$ contractions (CCSD) }
  \label{fig:Diagrams of T1 and T2}
\end{figure}


\begin{table}[h!]
\caption{Algebraic expression of $\hh{T}_1+\hh{T}_2$-amplitude diagrams from Fig. \ref{fig:Diagrams of T1 and T2}}
\centering
\begin{tabular}{c c c c l}
\hline\hline
Interaction & Contraction & $\epsilon$ & Diagram & \qquad Expression \\   
\hline
$\bra{\Phi_{ij}^{ab}}\hh{H}_N\hh{T}_1\ket{\Phi_0}$         &$\hh{V}_{N,6}$ & $+1$ & $D_{4a}$     & $\begin{aligned}P(ij)\sum_{c}  \vva{ab}{cj} t_i^c \end{aligned}$\\
$\bra{\Phi_{ij}^{ab}}\hh{H}_N\hh{T}_1\ket{\Phi_0}$         &$\hh{V}_{N,7}$ & $+1$ & $D_{4b}$     & $\begin{aligned}-P(ab)\sum_{k} \vva{kb}{ij  t_k^a}\end{aligned}$\\
\hline
$\bra{\Phi_{ij}^{ab}}\frac{1}{2}\hh{H}_N\hh{T}_1^2\ket{\Phi_0}$         &$\hh{V}_{N,1}$ & $0$ & $D_{5a}$     & $\begin{aligned}\frac{1}{2} P(ij)\sum_{cd} \vva{ab}{cd} t_i^c t_j^d\end{aligned}$\\
$\bra{\Phi_{ij}^{ab}}\frac{1}{2}\hh{H}_N\hh{T}_1^2\ket{\Phi_0}$         &$\hh{V}_{N,2}$ & $0$ & $D_{5b}$     & $\begin{aligned}\frac{1}{2} P(ab)\sum_{kl} \vva{kl}{ij} t_k^a t_l^b\end{aligned}$\\
$\bra{\Phi_{ij}^{ab}}\frac{1}{2}\hh{H}_N\hh{T}_1^2\ket{\Phi_0}$         &$\hh{V}_{N,3}$ & $0$ & $D_{5c}$     & $\begin{aligned}-P(ij|ab)\sum_{kc}  \vva{kb}{cj} t_i^c t_k^a\end{aligned}$\\
\hline
$\bra{\Phi_{ij}^{ab}}\hh{H}_N\hh{T}_1\hh{T}_2\ket{\Phi_0}$ &$\hh{F}_{N,1}$ & $-1$  &$D_{6a}$  & $\begin{aligned}-P(ij)    \sum_{kc} f_c^k t_i^c t_{kj}^{ab}\end{aligned}$\\
$\bra{\Phi_{ij}^{ab}}\hh{H}_N\hh{T}_1\hh{T}_2\ket{\Phi_0}$ &$\hh{F}_{N,1}$ & $-1$  &$D_{6b}$  & $\begin{aligned}-P(ab)    \sum_{kc} f_c^k t_a^k t_{ij}^{cb}\end{aligned}$\\
$\bra{\Phi_{ij}^{ab}}\hh{H}_N\hh{T}_1\hh{T}_2\ket{\Phi_0}$ &$\hh{V}_{N,4}$ & $-1$  &$D_{6c}$  & $\begin{aligned} P(ij|ab) \sum_{kcd}\vva{kd}{cd} t_k^a t_{ij}^{cd}\end{aligned}$\\
$\bra{\Phi_{ij}^{ab}}\hh{H}_N\hh{T}_1\hh{T}_2\ket{\Phi_0}$ &$\hh{V}_{N,5}$ & $-1$  &$D_{6d}$  & $\begin{aligned}-\frac{1}{2} P(ab) \sum_{kcd} \vva{kb}{cd} t_k^a t_{ij}^{cd}\end{aligned}$\\
$\bra{\Phi_{ij}^{ab}}\hh{H}_N\hh{T}_1\hh{T}_2\ket{\Phi_0}$ &$\hh{V}_{N,4}$ & $-1$  &$D_{6e}$  & $\begin{aligned} P(ab) \sum_{kcd} \vva{ka}{cd} t_k^c t_{ij}^{db}\end{aligned}$\\
$\bra{\Phi_{ij}^{ab}}\hh{H}_N\hh{T}_1\hh{T}_2\ket{\Phi_0}$ &$\hh{V}_{N,5}$ & $-1$  &$D_{6f}$  & $\begin{aligned} -P(ij|ab) \sum_{klc} \vva{kl}{ic} t_i^c t_{kl}^{ab}\end{aligned}$\\
$\bra{\Phi_{ij}^{ab}}\hh{H}_N\hh{T}_1\hh{T}_2\ket{\Phi_0}$ &$\hh{V}_{N,4}$ & $-1$  &$D_{6g}$  & $\begin{aligned}\frac{1}{2} P(ij)\sum_{klc} \vva{kl}{cj} t_i^c t_{kl}^{ab}\end{aligned}$\\
$\bra{\Phi_{ij}^{ab}}\hh{H}_N\hh{T}_1\hh{T}_2\ket{\Phi_0}$ &$\hh{V}_{N,5}$ & $-1$  &$D_{6h}$  & $\begin{aligned}- P(ij)\sum_{klc} \vva{kl}{ci} t_k^c t_{lj}^{ab}\end{aligned}$\\
\hline
$\bra{\Phi_{ij}^{ab}}\frac{1}{2}\hh{H}_N\hh{T}_1^2\hh{T}_2\ket{\Phi_0}$ &$\hh{V}_{N,9}$ & $-2$  &$D_{7a}$  & $\begin{aligned} \frac{1}{4}P(ij)\sum_{klcd} \vva{kl}{cd}t_i^c t_{kl}^{ab} t_j^d t_{lj}^{ab}\end{aligned}$\\
$\bra{\Phi_{ij}^{ab}}\frac{1}{2}\hh{H}_N\hh{T}_1^2\hh{T}_2\ket{\Phi_0}$ &$\hh{V}_{N,9}$ & $-2$  &$D_{7b}$  & $\begin{aligned} \frac{1}{4}P(ab)\sum_{klcd} \vva{kl}{cd}t_k^a t_{ij}^{cd} t_l^b t_{lj}^{ab}\end{aligned}$\\
$\bra{\Phi_{ij}^{ab}}\frac{1}{2}\hh{H}_N\hh{T}_1^2\hh{T}_2\ket{\Phi_0}$ &$\hh{V}_{N,9}$ & $-2$  &$D_{7c}$  & $\begin{aligned} -P(ij|ab)\sum_{klcd} \vva{kl}{cd} t_i^c t_a^k t_{lj}^{db} t_{lj}^{ab}\end{aligned}$\\
$\bra{\Phi_{ij}^{ab}}\frac{1}{2}\hh{H}_N\hh{T}_1^2\hh{T}_2\ket{\Phi_0}$ &$\hh{V}_{N,9}$ & $-2$  &$D_{7d}$  & $\begin{aligned} -P(ij)\sum_{klcd} \vva{kl}{cd} t_k^c t_i^d t_{lj}^{ab} t_{lj}^{ab}\end{aligned}$\\
$\bra{\Phi_{ij}^{ab}}\frac{1}{2}\hh{H}_N\hh{T}_1^2\hh{T}_2\ket{\Phi_0}$ &$\hh{V}_{N,9}$ & $-2$  &$D_{7e}$  & $\begin{aligned} -P(ab)\sum_{klcd} \vva{kl}{cd}t_k^c t_l^a t_{ij}^{db} t_{lj}^{ab}\end{aligned}$\\
\hline
$\bra{\Phi_{ij}^{ab}}\frac{1}{2}\hh{H}_N\hh{T}_1^2\hh{T}_2\ket{\Phi_0}$ &$\hh{V}_{N,9}$ & $-2$  &$D_{8a}$  & $\begin{aligned}\frac{1}{2}P(ij|ab)\sum_{klc} \vva{kl}{cd}t_i^c t_{kl}^{ab} t_j^d t_{lj}^{ab}\end{aligned}$\\
$\bra{\Phi_{ij}^{ab}}\frac{1}{2}\hh{H}_N\hh{T}_1^2\hh{T}_2\ket{\Phi_0}$ &$\hh{V}_{N,9}$ & $-2$  &$D_{8b}$  & $\begin{aligned}\frac{1}{2}P(ij|ab)\sum_{klc} \vva{kl}{cd}t_i^c t_{kl}^{ab} t_j^d t_{lj}^{ab}\end{aligned}$\\
\hline
$\bra{\Phi_{ij}^{ab}}\frac{1}{2}\hh{H}_N\hh{T}_1^2\hh{T}_2\ket{\Phi_0}$ &$\hh{V}_{N,9}$ & $-2$  &$D_{9}$  & $\begin{aligned}\frac{1}{4}P(ij|ab)\sum_{klcd} \vva{kl}{cd}t_i^c t_j^d t_k^a t_l^b t_{lj}^{ab}\end{aligned}$\\
\hline\hline
\end{tabular}
\label{table:T1 and T2 expressions}
\end{table}

\newpage

\section{Concluding Remarks}

\subsection{Issues with the Coupled Cluster Method}
Convergence etc. see \cite{kummel-ccsd}. The CC equations for the energy does not have any variational condition. Resulting in that we could in theory get lower energy than the \emph{exact} when we truncate $\hh{T}$ \cite{crawford}. 

%
One of the fundamental postulates in quantum mechanics states that observables are eigenvalues to an Hermitian operator, but the similartiy transformed Hamiltonian $e^{-\hh{T}}\hh{H}e^{\hh{T}}$ is not Hermitian for any $\hh{T}$. 
 
\begin{equation}
\ql (e^{-\hh{T}} \hh{H} e^{\hh{T}} \qr)^{\dagger} = e^{\hh{T}^{\dagger}} \hh{H} e^{-\hh{T}^{\dagger}}  \neq  e^{-\hh{T}} \hh{H} e^{\hh{T}}
\label{eq:not hermitian}
\end{equation}
%

The similarity transformed Hamiltonian $\hh{H}'$ and $\hh{H}$ same eigenvalues:

\small{
\begin{proof} 
\begin{equation*}
 \ket{\Psi} = e^{\hh{T}} \ket{\Phi_0}  
\end{equation*}
\begin{align*}
 \hh{H}\ket{\Psi} = \hh{H} (e^{\hh{T}} \ket{\Phi_0}) =
 e^{\hh{T}}e^{-\hh{T}}\hh{H}e^{\hh{T}}\ket{\Phi_0} &=e^{\hh{T}}\hh{H}'\ket{\Phi_0} = \lambda (e^{\hh{T}}\ket{\Phi_0}) = \lambda \ket{\Psi}
\end{align*}
 \label{proof:eigenvalues of a similarity transformed Hamiltonian}
\end{proof}
%
$\ket{\Phi_0}$ is the eigenfunctions of $\hh{H}'$ and $\ket{\Psi}$ is the eigenfunction of $\hh{H}$. Both with the same eigenvalues
%
The $\hh{T} = \hh{T}_1$ gives us the Hartree Fock approxiamtion,

\begin{equation}
E_{CCSD} - E_0 = \sum_{ia}f_a^i t_i^a + \frac{1}{2}\sum_{ijab} \vva{ij}{ab}t_i^a t_j^b
 \label{eq:ccs and Hartree Fock}
\end{equation}
%
and the expression for the $\hh{T}_1$-amplitude  
\begin{equation}
f_i^a + \sum_c f_c^a t_i^c + \sum_{kc}\vva{ka}{ci} t_k^c = 0
 \label{def:ccs and Hatree Fock amplitude equations}
\end{equation}


\subsection{Wavefunction Seperability and Size Consistency of the Energy}
In CC theory it is possible to $\Psi = e^{T_X + T_Y} \Phi_0$ and get $E = E_X + E_Y$ no can do in CI with linear terms in the ansatz



% 
% 
% %%%%%%%%%%%%%%%%%%%%%%%%%%%%%%%%%%%%%%%%%%%%%%%%%%%%%%%%%%%%%%%%%%%%%%%%%
% %%%%%%%%%%%%%%%%%%%%%%%%%%%%%% Implementation%%%%%%%%%%%%%%%%%%%%%%%%%%%%
% %%%%%%%%%%%%%%%%%%%%%%%%%%%%%%%%%%%%%%%%%%%%%%%%%%%%%%%%%%%%%%%%%%%%%%%%%
%\setlength{\parindent}{0pt}   % Set no indentation in the beginning of each paragraph.
%\setlength{\parskip}{2ex}     % Separate lines each paragraph.

\chapter{Software design and implementation}\label{implementation}
The use of compiled low level languages such as Fortran and C is a deeply-rooted tradition among computational scientists and physicists for doing numerical simulations requiring high performance. However, the increased demand for more flexibility has motivated the development of object oriented languages such as C++  and Fortran 95. The use of encapsulation, polymorphism and inheritance allows for setting up source codes into a public interface, and a private implementation of that interface. The style includes overloading of standard operators so that they have an appropriate behaviour according to the context and creating subclasses that are specializations of their parents. Big projects are split into several classes communicating with each other.  This has several advantages, in particular 
for debugging, reuse, maintenance and extension of the code. These ideas will be used in the rest of the chapter for designing and implementing a Quantum Variational Monte Carlo simulator. %Three alternatives are explored.

\section{Structuring a software for QVMC in close shell systems}\label{softwareStructure}
When design software one should take into account the efficient usage of memory, the possibility 
for future extensions (flexibility) and software structure (easy to understand for other programmers and the final users)\cite{Shapira2006,Langtangen2003}. \\
\\
In general, several approaches exist in the design of software, one of them is the use of prototypes. The high level programming language Python\footnote{Python is an interpreted object-oriented language that shares with Matlab many of its characteristics, but which is much more powerful and flexible when equipped with tools for numerical simulations and visualization. Because Python was designed to be extended with legacy code for efficiency, it is easier to interface it with  software written in C++, C and fortran than in other environments. A balance between computational efficienty, to get fast codes, and programming efficiency, is preferred. 

} has been used in this thesis with this in mind. In essence what a prototype does is to help the designer exploring alternatives, making performance tests, and modifying design strategies.

\subsection{Implementing a prototype in Python}\label{PythonPrototype}
We start implementing the Quantum Variational Monte Carlo algorithm from chapter \ref{QMC}. To do so, we implement a \citecode{Vmc} class which serves to administrate the simulation, setting the parameters of the simulation, creating objects related to the algorithm, and running it, as shown below. 
% \begin{lstlisting}[language=Python]
\begin{Python}
...
#Import some packages
...
class VMC():
  def __init__(self, _dim, _np, _charge, ...,_parameters):
    ...
    #Set parameters and create objects for the simulation
    ...
    particle = Particle(_dim, _np, _step) 
    psi      = Psi(_np, _dim, _parameters)
    energy   = Energy(_dim, _np, particle, psi, _nVar, _ncycles, _charge) 
    self.mc  = MonteCarlo(psi, _nVar, _ncycles, _dim, _np, particle, energy)

  
  def doVariationalLoop(self):
    if (self.mcMethod=='BF'):
      for var in xrange(nVar):
		    self.mc.doMonteCarloBruteForce()
		    self.mc.psi.updateVariationalParameters()
    else:
      for var in xrange(nVar):
	      self.mc.doMonteCarloImportanceSampling()
	      self.mc.psi.updateVariationalParameters()
  
  ...
\end{Python}
% \end{lstlisting}
The information regarding the configuration space can be encapsulated in a class called 
\citecode{Particle} with the straightforward implementation\\
%%%\begin{lstlisting}[language=Python]
\begin{Python}
...
class Particle(): 
  def __init__(self, _dim, _np, _step):
    # Initialize matrices for configuration space
    ...

  def acceptMove(self, i):
    for j in xrange(dim):
      self.r_old[i,j] = self.r_new[i,j]

  def resetPosition(self, i):
    for k in xrange(self.np):
      if(k != i):
				for j in xrange(self.dim):
					self.r_new[k,j] = self.r_old[k,j]

	def setTrialPositionsBF(self):
    self.r_old = self.step*random.uniform(-0.5,0.5,size=np*dim).reshape(np,dim)

  def setTrialPositionsIS(self):
    for i in xrange(self.np):
      for j in xrange(self.dim):
				self.r_old[i,j] = random.normal()*sqrt(self.step) # step=dt
\end{Python}

\begin{Python}

  def updatePosition(self, i):
    #Note use of vectorized loop
    self.r_new[i,0:dim]=self.r_old[i,0:dim] + self.step*(random.random()-0.5)
    self.resetPosition(i)
\end{Python}
%%%%\end{lstlisting}
\noindent
In the class \citecode{MonteCarlo} we implement a function \citecode{equilibrate()} meant to deal with the equlibration of the Markov chain, that is to reach the most likely state of the system before we start with the production part of the simulation. The kind of sampling with which to run the simulation can be choosen from the \citecode{Vmc} class.
% % \begin{lstlisting}[language=Python]
\begin{Python}

class MonteCarlo():
  def __init__(self, psi, ..., _ncycles, ..., particle, energy):
    #Set data and member functions
    ...
  
  def doMonteCarloImportanceSampling(self):
    ...  
    #Matrices storing the quantum force terms
    Fold = zeros(np*dim).reshape(np, dim)  
    Fnew = zeros(np*dim).reshape(np, dim)
    
    #Reset energies
    self.energy.resetEnergies()
    deltaEnergy = 0.0		
    ...
    #Set initial trial positions (space configuration)
    self.particle.setTrialPositionsIS()
    
    #Reach the most likely state
    self.equilibrate()

    #Evaluate the trial wave function for the current positions
    wfold = getPsiTrial(r_old)
			    
    #Evaluate the quantum force at the current positions
    Fold = self.psi.getQuantumForce(r_old, wfold)
    
    for c in xrange(1, self.ncycles+1, 1):	# Monte Carlo loop
      for i in xrange(np):									# Loop over particles
				for j in xrange(dim):								# Loop over coordinates
					#Suggest a trial move
					r_new[i,j] = r_old[i,j] + random.normal()*sqrt(dt) + Fold[i,j]*dt*D

				#Reset the position of the particles, but the ones containing i as its first index
				resetPostion(i)

				#Evaluate the trial wave function at the suggested position
				wfnew = self.psi.getPsiTrial(r_new)
						
				#Evaluate the quantum force for particles at the suggested position
				Fnew = self.psi.getQuantumForce(r_new, wfnew)
						
				#Compute the Green's function
				greensFunction = 0.0
						
				for j in xrange(dim):
					greensFunction 	+= 0.5*(Fold[i,j] + Fnew[i,j]) \
														*(D*dt*0.5*(Fold[i,j] - Fnew[i,j])- r_new[i,j] + r_old[i,j])
				
				greensFunction = exp(greensFunction)
\end{Python}

\begin{Python}

				acceptanceRatio = greensFunction*wfnew*wfnew/wfold/wfold

				#Do the metropolis/hasting test
				if(random.random() <= acceptanceRatio):
					for j in xrange(dim):
						r_old[i,j] = r_new[i,j] 
						Fold[i,j] = Fnew[i,j]   
					
					#Update the wave function
					wfold = wfnew

      #Compute local energy and its cumulants
      self.energy.getLocalEnergy(wfold)
    self.energy.computeEnergyExpectationValue()
    self.energy.resetEnergies()
\end{Python}
% % % % \end{lstlisting}
% % % % % % % % % The main difference with the brute force method is that it does not include terms related to the quantum force. Moreover, the new position is suggested according to:
% % % \begin{lstlisting}[language=Python]
% % % % % % % % % \begin{Python}
% % % % % % % % % 
% % % % % % % % % self.particle.r_new[i,j] = self.particle.r_old[i,j] + step*(random.random() - 0.5)
% % % % % % % % % 
% % % % % % % % % \end{Python}
% % % % % % % % % % % % \end{lstlisting}
\noindent
To encapsulate information concerning the trial wave function and energy we create the \citecode{Psi} and \citecode{Energy} classes, respectively.
% % % \begin{lstlisting}[language=Python]
\begin{Python}
...
class Psi:
  def __init__(self, _np, _dim, _parameters):
    ...
    self.cusp = zeros((_np*(_np-1)/2)) 		# Cusp factors
    self.setCusp()
  
  def setCusp(self):
    ...
    if(self.np==2):
      self.cusp[0] = 0.5
    else:
      for i in xrange(size):
				self.cusp[i] = 0.5
	
      self.cusp[0] = 0.5
      self.cusp[5] = 0.5

  #Define orbitals (single particle wave functions)
  def phi1s(self, rij):
    return exp(-self.parameters[0]*rij)
    
  def phi2s(self, rij):
    return (1.0 - self.parameters[0]*rij/2.0)*exp(-self.parameters[0]*rij/2.0)

  def getPsiTrial(self, r):
    return self.getModelWaveFunctionHe(r)*self.getCorrelationFactor(r)
	  
  def getCorrelationFactor(self, r):
    correlation = 1.0
    for i in xrange(self.np-1):
      for j in xrange(i+1, self.np):
				idx = i*(2*self.np - i-1)/2 - i + j - 1
				
		r_ij = 0.0
		for k in xrange(self.dim):
			r_ij += (r[i,k]-r[j,k])*(r[i,k]-r[j,k])
		r_ij=sqrt(r_ij)
		
		correlation *= exp(self.cusp[idx]*r_ij/(1.0 + self.parameters[1]*r_ij))
    return correlation
\end{Python}

\begin{Python}

  #Set the Slater determinant part of Be
  def getModelWaveFunctionBe(self, r):
    argument = zeros((self.np))
    wf = 0.0
    ...
    psi1s = self.phi1s #Shortcut, convenient in for-loops
    ...
    for i in xrange(self.np):
      argument[i] = 0.0
      r_singleParticle = 0.0
				for j in xrange(self.dim):
					r_singleParticle += 	r[i,j]*r[i,j]
	
      argument[i] = sqrt(r_singleParticle)
	
    wf = (psi1s(argument[0])*psi2s(argument[1])	\
				-psi1s(argument[1])*psi2s(argument[0]))	\
				*(psi1s(argument[2])*psi2s(argument[3])	\
				-psi1s(argument[3])*psi2s(argument[2]));
    
    return wf

  #Set the Slater determinant part of He
  def getModelWaveFunctionHe(self, r):
    argument = 0.0
    for i in xrange(self.np):
      r_singleParticle = 0.0
      for j in xrange(self.dim):
				r_singleParticle += r[i,j]*r[i,j]
      argument += sqrt(r_singleParticle)

    return exp(-argument*self.parameters[0])

	#Compute the quantum force numerically 
	def getQuantumForce(self, r, wf): 	 
    ...
    for i in xrange(np):
      for j in xrange(dim):
      				r_plus[i,j] = r[i,j] + h
				r_minus[i,j] = r[i,j] - h
				wfminus = self.getPsiTrial(r_minus) 
				wfplus = self.getPsiTrial(r_plus)
				
				quantumForce[i,j] = (wfplus - wfminus)/wf/h
				
				r_plus[i,j] = r[i,j]
				r_minus[i,j] = r[i,j]
	
    return quantumForce
...
\end{Python}
\noindent
Class \citecode{Energy} is also equipped with some functions to do the statistical analysis of uncorrelated data.

% % % % 	\begin{lstlisting}[language=Python]
\begin{Python}
...
class Energy:	
  def __init__(self, dim, np, particle, psi, maxVar, ncycles, charge):	
    self.cumEnergy = zeros(maxVar)			  #Cumulant for energy
    self.cumEnergy2= zeros(maxVar)			#Cumulant for energy squared
		...
\end{Python}

\begin{Python}

  def getLocalEnergy(self, wfold):
	  EL 	   = self.getKineticEnergy(wfold) + self.getPotentialEnergy()
	  self.E  += EL
	  self.E2 += EL*EL 
  ...

  def getPotentialEnergy(self):
    ...
		PE = 0.0
		
    # Contribution from electron-proton potential
    for i in xrange(np):
      r_single_particle = 0.0
      for j in xrange(dim): 
				r_single_particle += r_old[i,j]*r_old[i,j]
    PE -= charge/sqrt(r_single_particle) 

    # Contribution from electron-electron potential
    for i in xrange(0, np-1):
      for j in xrange(i+1, np):
			r_12 = 0.0
			for k in xrange(0, dim):
				r_12 += (r_old[i,k] - r_old[j,k])*(r_old[i,k] - r_old[j,k])
			PE += 1.0/sqrt(r_12) 
    return PE

  def getKineticEnergy(self, wfold):
		KE = 0.0  		 	 	#Initialize kinetic energy
		h = 0.001		 		 	#Step for the numerical derivative
    h2= 1000000.	 		#hbar squared
    ...
    for i in xrange(np):
      for j in xrange(dim):
				r_plus[i,j] = self.particle.r_old[i,j] + h
				r_minus[i,j]= self.particle.r_old[i,j] - h
				wfplus  = self.psi.getPsiTrial(r_plus)
				wfminus = self.psi.getPsiTrial(r_minus)

				# Get the laplacian_wave_function to wave_function ratio
				KE -= (wfminus + wfplus - 2.0*wfold)
				
				r_plus[i,j]=self.particle.r_old[i,j].copy()
				r_minus[i,j]  = self.particle.r_old[i,j].copy()

  return 0.5*h2*KE/wfold # include electron mass and hbar squared
\end{Python}
% % % % \end{lstlisting}
The calling code is just
% % % % \begin{lstlisting}[language=Python]
\begin{Python}
import sys
from VMC import *

#Set parameters of simulation
dim 	= 3	  						#Number of spatial dimensions
nVar 	= 10							#Number of variations for a quick optimization method
ncycles = 10000					#Number of monte Carlo cycles
np	=2						# Number of electrons
charge	=2.0		 	# Nuclear charge
...
vmc = VMC(dim, np, charge, mcMethod, ncycles, step, nVar, parameters)
vmc.doVariationalLoop()
vmc.mc.energy.printResults()
\end{Python}
% % % % \end{lstlisting}
The source code\footnote{For details, we refer to the CD included with this thesis under \citecode{QVMC/Python}.} above is relatively straightforward to implement, besides being short (600 lines). The syntax used is clear, compact and close to the way we formulate the expressions mathematically. The language is easy to learn and one can develop a program and modify it easily. The dual nature of Python of being both an object oriented and scripting language is convenient in structuring the problem such that we can split it into smaller parts, while testing quickly the results (output) against analytical or experimental data. A limitation, especially in problems managing a high number of degrees of freedom, as happens in quantum mechanics, is its speed as shown in figures \ref{executionTimeHePyCpp} to \ref{delayExecutionTimeHePyCpp}.\\
\\
Computing the ground energy of a He atom (2 electrons) using 10000 Monte Carlo cycles and five parameter variations with this Python simulator takes about  1.5 mi\-nu\-tes on a Pentium-4 laptop running at 2.4 GHz. On the other hand, attempting to do the same for a Be atom (4 electrons) with only $5 \times 10^4$ Monte Carlo cycles (not enough to get statistically acceptable results for Be) takes more than one hour. Several alternatives exists for being able to simulate bigger systems than He. Three of them are exposed here and explored in the following: 
\begin{enumerate}
 \item Keeping the software structure as above, but implement all the classes in C++ (or another low-level programming language).
 \item Integrate Python and C++ in a way that the portions of the software that requires high-performance computing considerations can be implemented in C++. The administrative parts are then written in Python.
 \item Exploting the object-oriented characteristics of C++ for developing an efficient, flexible and easy to use software (optimized algorithm).
\end{enumerate}


\subsection{Moving the prototype to a low level programming language}\label{cppPrototype}
The implementation of the simulator in C++ follows more or less the recipe above. Here, however, we separate the potential energy from the Hamiltonian by introducing a \citecode{Potential} class, which becomes an object member of the class \citecode{Energy}. The use of dynamic memory to manipulate data structures is usually a source of bugs, as C++ does not have automatic garbage collection as Python does. \\
\\
In the present implementation we use a ready made class for manipulating array data structures. {\citecode{MyArray} class stores all the array entries in contiguous blocks of me\-mo\-ry\cite{Langtangen2000}. This design aims at interfacing Python and C++, as the Python arrays are, also, stored in the same way. The class \citecode{MyArray} supports arrays in one, two and three dimensions.}\cite{HPL}. Because it is a template class, it can be reused easily with any type. Moreover, changing its behaviour is easily done in only one place and to free of me\-mo\-ry is made in its destructor\cite{Langtangen2000}.\\


\subsection{Mixing Python and C++}
There are several ways one can integrate Python with C/C++. Two of them are called extending and embedding, respectively. The first one, extending, involves creating a wrapper for C++ that Python imports, builds, and then can execute. We will focus on the wrapper extension. In the second alternative, embedding, the C++ part is given direct access to the Python interpreter. Python is extended for many different reasons. A developer may want to use an existing C++ library, or port work from an old project into a new Python development effort.\\
\\
\begin{lstlisting}[language=c++]
...
template< typename T > class MyArray
{
 public:
  T* A;                   // the data
  int ndim;               // no of dimensions (axis)
  int size[MAXDIM];       // size/length of each dimension
  int length;             // total no of array entries
  T* allocate(int n1);
  T* allocate(int n1, int n2);
  T* allocate(int n1, int n2, int n3);
  void deallocate();
  bool indexOk(int i) const;      
  bool indexOk(int i, int j) const;
  bool indexOk(int i, int j, int k) const;
  
 public:
  MyArray() { A = NULL; length = 0; ndim = 0; }
  MyArray(int n1) { A = allocate(n1); }
  MyArray(int n1, int n2) { A = allocate(n1, n2); }
  MyArray(int n1, int n2, int n3) { A = allocate(n1, n2, n3); }
  MyArray(T* a, int ndim_, int size_[]);
  MyArray(const MyArray<T>& array);
  ~MyArray() { deallocate(); }

  bool redim(int n1);           
  bool redim(int n1, int n2);   
  bool redim(int n1, int n2, int n3);   
  
  // return the size of the arrays dimensions:
  int shape(int dim) const { return size[dim-1]; }
  
  // indexing:
  const T& operator()(int i) const;
  T& operator()(int i);
  const T& operator()(int i, int j) const;
  T& operator()(int i, int j);
  const T& operator()(int i, int j, int k) const;
  T& operator()(int i, int j, int k);
  
  MyArray<T>& operator= (const MyArray<T>& v);
  
  // return pointers to the data:
  const T* getPtr() const { return A;}   
  T* getPtr() { return A; }
  
  ...
};
\end{lstlisting}
\noindent
Before attempting to extend Python, we should identify the parts to be moved to C++. The results of a profile of the  prototype discussed in section \ref{PythonPrototype} are summarized in tables \ref{profileHe} and \ref{profileBe}. It is not surprising that \citecode{Energy.py}, \citecode{MonteCarlo.py} and \citecode{Psi.py} are the most time consuming parts of the algorithm. It is there where the major number of operations and calls to functions in other classes are carried out. Fortunately, most of the work needed to rewrite these classes to their equivalent C++ part 
has been already carried out in the last subsection.\\
\\
The class \citecode{Vmc.py} is responsible of initializing the C++ objects of type \citecode{Psi}, \citecode{Particle}, \citecode{Potential}, \citecode{Energy} and \citecode{MonteCarlo}. Letting Python create the objects has the advantage of introducing automatic garbage collection, reducing the risk implied in managing memory. Memory handling is very often a source of errors in C++. It is also often very difficult to spot such errors.\\
\\
To access C++ via an extension we write a wrapper. The wrapper acts as a glue between the two languages, converting function arguments from Python into C++, and then returning results from C++ back to Python in a way that Python canunderstand and handle. Fortunately, SWIG (Simplified Wrapper and Interface Generator)\cite{Beazley} does much of the work automatically.\\
\\
Before trying to implement a wrapper for the present application, one should note that the \citecode{Psi} class takes a generic object of type \citecode{MyArray<double>} as argument in its constructor. It means that we should find the way of converting the \emph{variational parameters} (a numpy object) into a reference to an object of type \citecode{MyArray<double>}. This kind of conversion is not supported by SWIG and has to be done manually. Doing this for a lot of functions is a hard and error prone job. \\

\begin{lstlisting}[language=c++]
...
#include "MyArray.h"
...
class Psi{
  private:
    ...
    MyArray<double> cusp;			// Cusp factors in the Jastrow form
    MyArray<double> varPar;		// Variational parameters
    MyArray<double> quantumForce;	

  public:
    Psi(MyArray<double>& varParam, int _np, int dim);
    void setParameters(MyArray<double>& varPar);
    double getCorrelationFactor(const MyArray<double>& r);
    double getModelWaveFunctionHe(const MyArray<double>& r);
    MyArray<double>& getQuantumForce(const MyArray<double>& r, double wf);
    ...
}; 
\end{lstlisting}
\noindent
A conversion class \citecode{Convert} specially adapted for \citecode{MyArray} class has already been proposed in \cite{HPL}. The advantage with it is that we do not need to interface the whole \citecode{MyArray} class. Instead, the conversion class is equiped with static functions for converting a \citecode{MyArray} object to a \citecode{Numpy} array and vice versa. The conversion functions in this class can be called both manually or using SWIG to automatically generate wrapper code. The last option is preferible because the conversion functions has only pointers or reference as input and output data.

\begin{lstlisting}[language=c++]
...
#include <Python.h>
#include <pyport.h>
#include <numpy/arrayobject.h>
#include "MyArray.h"
...
class Convert
{
 public:
    Convert();
  ~Convert();

    // borrow data:
    PyObject*        my2py (MyArray<double>& a);
    MyArray<double>* py2my (PyObject* a);

    // copy data:
    PyObject*        my2py_copy (MyArray<double>& a);
    MyArray<double>* py2my_copy (PyObject* a);

    // npy_intp to/from int array for array size:
    npy_intp         npy_size[MAXDIM];
    int              int_size[MAXDIM];
    void             set_npy_size(int*      dims, int nd);
    void             set_int_size(npy_intp* dims, int nd);

    // print array:
    void             dump(MyArray<double>& a);

    ...
    #Code necessary to make callbacks
    ...
 
};
\end{lstlisting}
\noindent
With the information above we create the SWIG interface file \citecode{ext\_QVMC.i} having the same name as the module \cite{Beazley}
\begin{src}
%module ext_QVMC
%{
#include "Convert.h"
#include "Energy.h"
#include "MonteCarlo.h"
#include "Psi.h"
#include "Particle.h"
#include "Potential.h"
%}
%include "Convert.h"
%include "Energy.h"
%include "MonteCarlo.h"
%include "Particle.h"
%include "Potential.h"
%include "Psi.h"
\end{src}
\noindent
To build a Python module with extension to C++ we run SWIG with the options \citecode{-Python} and \citecode{-c++}. Running
\begin{src}
swig -c++ -Python ext_QVMC.i
\end{src}
generates a wrapper code in \citecode{ext\_QVMC\_wrp.cxx} and the Python module \citecode{\_ext\_QMVC.py}. Assuming that all the sources files are in the same directory, this file has to be compiled
\begin{src}
c++ -c -O3 *.cpp *.cxx -I/usr/include/Python2.6/
\end{src}
and linked to a shared library file with name \citecode{\_ext\_QVMC.so}
\begin{src}
c++ -shared -o _ext_QVMC.so *.o
\end{src}
The Python \citecode{VMC.py} class calling the C++ code has to be slightly adjusted such that it can use the new extension module \citecode{ext\_QVMC}.

\begin{Python}

from math import sqrt   # use sqrt for scalar case
from numpy import*
import string

# Set the path to the extensions
import sys
sys.path.insert(0, './extensions')
import ext_QVMC

class Vmc():
  def __init__(self, _Parameters):
    # Create an object of the 'conversion class' for 
    # convert Python-to-C++-to-Python data structure.
    self.convert = ext_QVMC.Convert()
    
    # Get the paramters of the currrent simulation
    simParameters 	= _Parameters.getParameters()
    
    self.nsd 	  	= simParameters[0]
    self.nel 		= simParameters[1]
    self.nVar 		= simParameters[2]
    self.nmcc		= simParameters[3]
    self.charge		= simParameters[4]
    self.step 		= simParameters[5]
    alpha		= simParameters[6]
    beta		= simParameters[7]
    
    # Set the Variational parameters
    self.varpar = array([alpha, beta])
    
    # Convert a Python array to a MyArray object
    self.v_p = self.convert.py2my_copy(self.varpar)
\end{Python}

\begin{Python}
    
    # Create an wave function object
    self.psi 		= ext_QVMC.Psi(self.v_p, self.nel, self.nsd)
    
    self.particle 	= ext_QVMC.Particle(self.nsd, self.nel, self.step)
    self.potential 	= ext_QVMC.Potential(self.nsd, self.nel, self.charge)
    self.energy 	= ext_QVMC.Energy(self.nsd, self.nel, self.potential, self.psi, self.nVar, self.nmcc)
    self.mc		= ext_QVMC.MonteCarlo(self.nmcc, self.nsd, self.nel, 0.001, self.particle, self.psi, self.energy) 
    
  def doVariationalLoop(self):
    nVar = self.nVar
    for var in xrange(nVar):
      self.mc.doMonteCarloLoop()
  
 ...
\end{Python}
\noindent
The \emph{caller}\footnote{The function that calls another (callee).} code in Python resembles the one in the simulator prototype from section \ref{PythonPrototype}, but with an additional class \citecode{SimParameters.py} that is able to read from a conveniently formatted file containing the parameters of the simulation. An example for the Be atom is:
\begin{src}
Number of spatial dimensions 	= 3
Number of electrons 		= 4
Number of variations		= 10
Number of Monte Carlo cycles  	= 100000
Nuclear charge  		= 4.0
Step				= 0.01
Alpha				= 2.765
Beta				= 0.6
\end{src}
\noindent
With this small change, the new caller looks more compact and user friendly.

\begin{lstlisting}[language=Python]
import sys

from SimParameters import * #Class in Python encapsulating the parameters
from Vmc import *

# Create an object containing the parameters of the current simulation
simpar = SimParameters('Be.data')

# Create a Variational Monte Carlo simulation
vmc = Vmc(simpar)

vmc.doVariationalLoop()
vmc.energy.doStatistics("resultsBe.data", 1.0)
\end{lstlisting}

\section{Developing a robust QVMC simulator in pure C++}
The rest of this chapter is dedicated to describe how one can take advantage of the object oriented characteristic of C++ to create a flexible and robust QVMC simulator. In order to reduce the computational cost of evaluating quantities associated with the trial wave function, we implement instead the algorithms discussed in chapter (\ref{TWF}).
This requires that we add some extra functionality. We discuss these additional functionalities 
in the rest of this chapter\footnote{The corresponding source codes can be found in the CD following this thesis in \citecode{QVMC/parallel/}. Also, an extensive documentation has been prepared and can be compiled by running \citecode{doxygen pyVMC} in a terminal under the directory of the sources codes. It will generate the documentation in html (\citecode{QVMC/parallel/doc}).}.\\

\subsubsection{Parameters of the system}
The \citecode{SimParameters} class encapsulates\footnote{This kind of encapsulation reduces the number of parametersthat are sent to the various functions. It reduces also the possibility of making mistakes when swapping them. Furthemore, the localization of bugs in the code can become easier, because the whole program is divided into small parts.} the parameters of the simulation. It creates an object of a class especially designed to read files in ascii format containing comments as well. In this way the user can easyly know what kind of information the software requires. The function \citecode{readSimParameters()} reads the parameters and loads them as data members. Moreover, functions to return each of them are provided.

\begin{src}
sys = 2DQDot2e            # Two dimensional quantum dot with two electrons.
nsd = 2                   # Number of spatial dimensions.
nel = 2                   # Number of electrons.
potpar = 2.0              # Potential parameter (omega).	
lambda = 1                # Confinament parameter in Coulomb potential.
nmc = 1.000000e+07        # Number of Monte Carlo cycles.
nth = 1.000000e+06        # Number of equilibration steps.
dt = 1.0000e-06           # Step size.
nvp = 2                   # Number of Variational parameters.
alpha = 0.75              # Variational parameter.
beta = 0.16               # Variational parameter.
enableBlocking = no       # Blocking option off.
enableCorrelation = yes   # Correlation part included.
\end{src}


\subsubsection{Initializing and setting in operation a simulation}
We declare a class \citecode{VmcSimulation.h} having a pointer to the \citecode{SimParameters} class. The parameters of the simulation are then used in setting a Variational Monte Carlo simulator in operation. A function \citecode{setInitialConditions()} initializes the space configuration and the trial wave function (includes the gradients and Laplacians of the single wave functions and the Jastrow correlation function). By declaring the virtual functions \citecode{setWaveFunction()} and \citecode{setHamiltonian} to be called by \citecode{setPhysicalSystem()}, the software is able to handle atomic systems and quantum dots, that is either non-interacting or interacting electrons in harmomic-oscillator ike traps. The particular implementation of these functions takes place in the subclass \citecode{VmcApp} depending of the physical system treated.
\begin{lstlisting}[language=c++]
void VmcApp::setHamiltonian(){
  //Set the potential form and add it to the kinetic energy term
  potential = new CoulombAtom(simParameters);
  
  energy = new Energy(simParameters, potential, psi, parallelizer);
}
\end{lstlisting}
By calling \citecode{setHamiltonian()}, the application creates a \citecode{Potential} object which is sent as a pointer to the class \citecode{Energy}. In the same function, we call \citecode{setWaveFunction()} to set the \citecode{Jastrow} and the \citecode{SlaterDeterminant} objects, which are used as argument to initialize the \citecode{WaveFnc} class.\\ 
\\
At the time the \citecode{VmcSimulation} is created, we construct a \citecode{RandomNumberGenerator()} object that deals with (pseudo) random numbers and uses them together with other parameters to create a \citecode{MonteCarlo} object implementing the equilibration and a sampling algorithm as the one shown in figure \ref{chartFlowMA}. The Monte Carlo for-loop is initiated by \citecode{VmcSimulation::runSimulation()}. The current code does not provide an interface for the Monte Carlo class, but it should be straightforward to create one  with the following functions declared as virtual: \citecode{doMonteCarloLoop()}, \citecode{suggestMove()} and \citecode{doMetropolisTest()}.\\
\begin{lstlisting}[language=c++]
...
class VmcSimulation{
  protected:
    ...
    Energy *energy;       // Hamiltonian
    MonteCarlo* mc;       // Monte Carlo method
    Optimizer *optimizer; // Optimization algorithm
    Parallelizer *para;   // Encapsulate MPI
		...
\end{lstlisting}
\begin{lstlisting}[language=c++]
    Potential *potential; 
    RandomGenerator *random;
    RelativeDistance *relativeDistance;   // Vectorial and scalar relative distances
    SimParameters *simpar;                // Parameters of simulation
    SingleStateWaveFuncs **singleStates;  // Single state wave functions
    Slater *slater; 
    WaveFnc *psi;                         // Trial wave function

    int numprocs;
    int myRank;
    int rankRoot;
    ...
  public:
    VmcSimulation(SimParameters *_simpa, int argc, char** argv);
    virtual ~VmcSimulation();
    void dumpResults(string resultsFile);
    void runSimulation();
    void setPhysicalSystem(){}
    void setInitialConditions(long& idum);
    void setPhysicalSystem(long& idum);
    
    virtual void optimizeWaveFunction(MyArray<double>& p, ...); 
    virtual void setRandomGenerator();
    virtual void setHamiltonian(){}
    virtual void setWaveFunction(){}
};

\end{lstlisting}
\noindent
To create a trial wave function object, we first instanciate \citecode{singleStateWaveFuncs} and \citecode{JastrowForm} and use them to create the \citecode{SlaterDeterminant} and the \citecode{Jastrow} objects. Function objects are used to encapsulate the forms of the Jastrow function\footnote{In this case in particular we use a function object, an instance of an class that behaves like a function by virtue of its \emph{function call operator ()} are those that can be called as if they were a function.}. For the single particle wave functions we create an array of pointers to objects of type \citecode{SingleSateWaveFnc}. 
%%%%%\begin{figure*}
\begin{lstlisting}[language=c++]
void VmcApp::setWaveFunction(){

  // Set single state wave functions (spin orbitals in atoms)
  singleStates 		= new SingleStateWaveFuncs*[2]; 
	singleStates[0] = new phi1s(nsd, varPar); 
	singleStates[1] = new phi2s(nsd, varPar);

	// set Slater determinant
	slater  = new Slater(simpar, varPar, singleStates); 

  if(enableCorrelation == true){
   // set correlation function
   setJastrow();
	 // set Slater Jastrow trial wave function 
	 psi = new SlaterJastrow(slater, jastrow, simpar, varPar, relativeDistance); 

  }else{
   // set Slater determinant as trial wave function
	 psi = new SlaterNOJastrow(slater, simpar, varPar, relativeDistance);
	}
}
\end{lstlisting}
%%%%\end{figure*}
% % % \end{src}
% % % \end{center}

\subsubsection{Class structure of the trial wave function}

The definition of the trial wave function, single particle wave functions, the Jastrow correlation function, etc, has to be flexible enough in order to easily be provided by the user for different physical systems and computational methods (numerical or analytical). This kind of behaviour is known as \emph{polymorphism} and it is achieved with the use of interfaces, that is classes that define a general behaviour without specifying it. Examples are \citecode{WaveFnc}, \citecode{Potential.h} and the interface for the single particle wave functions \citecode{SingleStateWaveFuncs.h}.
\begin{lstlisting}[language=c++]
...
class SingleStateWaveFuncs{
  ...
  public:
    ...
    virtual double evaluate(int p, const MyArray<double>& r)=0;
    virtual MyArray<double>& getStateGradient(int p, const MyArray<double>& r)=0;
    virtual double getSecondDerivative(int p, const MyArray<double>& r)=0;
    virtual double getParametricDerivative(int p, const MyArray<double>& r)=0;
    ...
  };
  ...
\end{lstlisting}
A convenient class derived from the interface above is \citecode{SlaterNOJastrow}. Because it does not include a Jastrow correlation factor, it can be used both for computing energies of harmonic oscillators and in the validitation of results derived from simulations with trial wave function including only the Slater determinant part. What one should expect in the last case is that experiments carried out in this way reproduce some of the analytical results derived in chapter \ref{cases}. Moreover, because a Slater determinant is an exact wave function when no correlation is included, it is expected that the variance resulting from the experiments is zero. 
\begin{lstlisting}[language=c++]
class WaveFnc{
  public:
    virtual ~WaveFnc(){} 
    virtual double getLapPsiRatio(int i)=0;
    virtual double getPsiPsiRatio(int p, const MyArray<double>& rnew)=0;
    virtual MyArray<double>& getVariationalParameters()=0;
    virtual MyArray<double>& getParametricDerivative(const MyArray<double>& r)=0;
    virtual void getQuantumForce(int p, MyArray<double>& F)=0;
    virtual void resetJastrow(int i)=0;
    virtual void getTrialRow(int p, const MyArray<double>& rnew)=0;
    virtual void getTrialQuantumForce(int p, MyArray<double>& Fnew, ...) = 0;
    virtual void initializePsi(const MyArray<double>& rold)=0;
    virtual void updatePsi(int k, const MyArray<double>& rnew)=0;	
    virtual void setVariationalParameters(MyArray<double>& varPar)=0;	
 };
\end{lstlisting}
On the other hand, the class \citecode{SlaterJastrow} has \citecode{SlaterDeterminant} and \citecode{Jastrow} as data members. Thus, the functionalities of \citecode{SlaterJastrow} can be carried out independently on each of the member classes and sometimes, as in the case of computing the total quantum force, summed up in the calling class, as suggested by table \ref{psiFunctions}. Furthermore, since the the Slater determinant equals the product of two determinants, we create a \citecode{SlaterDeterminant} class containing two pointers, pointing to a \citecode{SlaterMatrix} whose entries are single particle wave functions for electrons with spin up and down, respectively\footnote{The way the spin is set up is by placing half of electrons in each of the matrices.}. Besides the functionality shown in table \ref{psiFunctions}, this class has some arrays for storing $\phi_j(\bfv{x^{cur}_i})$,  $\phi_j(\bfv{x^{trial}_i})$, $\bfv{\nabla}_i \phi_j(\bfv{x^{cur}_i})$, $\bfv{\nabla}_i \phi_j(\bfv{x^{new}_i})$ and $\nabla^{2}_{i}(\bfv{x_i})$.
% % % \begin{src}
\begin{lstlisting}[language=c++]
class SlaterJastrow: public WaveFnc
{
  private:
    SlaterDeterminant *slater;  
    Jastrow           *jastrow;
    MyVector<double>   varPar;    // variational parametes
    ...
  public:
    ...
    SlaterJastrow(Slater *_slt, Jastrow *_jtr,...);
    void getPsiPsiRatio(...);
    double getLapPsiRatio(...);
    MyArray<double>& getGradPsiRatio(...);
    void setVariationalParameters(MyArray<double>& _varPar);
    ...
}
\end{lstlisting}
% % % % \end{src}
% \end{center}

\noindent Because the operations with Slater matrices are done by moving one particle (one row) at the time, the \citecode{SlaterDeterminant} class needs to know which matrix it has to operate on. Therefore, during the initialization process we define an array of pointers to \citecode{SlaterMatrix} objects called \citecode{curSM} (for current Slater matrix) whose size equals the number of electrons in the system. Half of its entries point to a Slater matrix with spin up and the rest to the one with spin down. This can be used to decide which matrix to compute. Doing so is more efficient than using an \citecode{if-test} each time an electron is moved.\\

% % % % The flow of information in the \citecode{SlaterMatrix} is schematized in figure\footnote{In the updating algorithm we need only to invert the Slater determinant matrix once. This can by calling standard LU decomposition methods.} ({\color{red}{CITAR DIAGRAMA QUE ME FALTA HACER}}).\\
\begin{table}\label{psiFunctions}
\centering
\begin{tabular}{llr}
\toprule[1pt]
\textbf{Class} & \textbf{Function} & \textbf{Implemented equation}\\
\midrule[1pt]
\citecode{SlaterMatrix} & \citecode{getDetDetRatio()}  & \ref{RSD}\\
                        & \citecode{getGradDetRatio()} & \ref{gradDetRatioO}-\ref{gradDetRatioN}\\
			& \citecode{getLapDetRatio()}  & \ref{lapDetRatio}\\
			& \citecode{updateInverse()}   & \ref{updatingInverse}\\
\hline
\citecode{Jastrow}	& \citecode{getJasJasRatio()}  & \ref{padepadeRatio}\\
			& \citecode{getGradJasRatio()} & \ref{padeJastrowGradJasRatio}\\
			& \citecode{getLapJasRatio()}  & \ref{lapJasRatio}\\
\hline
\citecode{SlaterJastrow}& \citecode{getPsiPsiRatio()}  & \ref{acceptanceRatio}\\
			& \citecode{getGradPsiRatio()} & \ref{grad_det_ratio_gen}\\
			& \citecode{getLapPsiRatio()}  & \ref{laplacian_psi_psi_ratio}\\
\bottomrule[1pt]
\end{tabular}\caption{Some functions implemented in relation with the trial wave function.}
\end{table}
\noindent
For the \citecode{Jastrow} we create upper-triangular matrices for the scalar quantities $g(r_{ij})$ and $\nabla^2 g(r_{ij})$ (two per term) , and two for the vectorial term $\bfv{\nabla} g(r_{ij})$, all of them evaluated at the current and new positions. A similar storage scheme is used for the matrix $a_{ij}$ containing the electron-electron cusp condition factors given in Eqs. (\ref{cusp2D}) and (\ref{cusp3D}). The access to the scalar and vectorial (interelectronic) distances is carried out by means of a pointer to a \citecode{RelativeDistance} object\footnote{Information on scalar and vectorial interelectronic relative distances are also stored in upper-triangular matrices by the \citecode{RelativeDistance} class.}. Each time a move is suggested, accepted or rejected, the matrices in \citecode{Jastrow} have to be updated following the algorithms described in sections \ref{optimizingCorrelation} to \ref{lapCorOpt}. Similar comments apply to the class \citecode{RelativeDistance}.\\
\\
Each scalar and vectorial matrix has a size $N(N-1)/2$ and $N(N-1)/2 \times nsd$, respectively, with $N$  the number of electrons in the system and $nsd$ being the number of spatial dimensions. In the implementation, the matrices are stored as one-dimensional arrays in contiguous blocks of memory, using the \emph{generic class} \citecode{MyArray}\footnote{Arrays will be read and written a large number of times during the simulation, so the access should be fast as possible. It is reached by storing all the array entries in contiguous blocks\cite{Langtangen2000}.}. \citecode{MyArray} supports arrays in one, two and three dimensions. The mapping between the entry $(i,j)$ in the upper triangular matrix and the index $idx$ in the 1D-array is given by the equation $idx = i(2 N - i-1)/2 - i + j - 1$, with $0\leq i < N-1$ and $i+1 \leq j < N$.

\subsubsection{Computing the energy, parallelizing with MPI and optimizing with a quasi-Newton method}
The collection of samples and evaluation of the energy happens in the \citecode{Energy} class, which has a pointer to a \citecode{Potential} class with some functions declared virtual. 
\begin{lstlisting}[language=c++]
class Potential{
  public:
    virtual ~Potential(){};
    virtual double getPotentialEnergy(const MyArray<double>& r){
      ...
    };
};
\end{lstlisting}
\noindent
In this way, what we need to do for implementing new potential forms is to inherit from it and overriding its functions in subclases. The creation of a new object and its use has the form
\begin{lstlisting}[language=c++]
  // Create a potential form for atoms
  Potential *coulombAt = new CoulombAtoms(...);
  
  // Get the potential energy
  coulombsAt->getPotentialEnergy(...); 
\end{lstlisting}
On the other hand, the \citecode{Energy} class  does not care about the kind of potential gotten in its constructor. Therefore we do not need to modify it when the potential form changes.
% % % % % For the kinetic energy we use the optimized expression 
% % % % % \begin{equation}\label{lap_det_ratio}
% % % % %  \boxed{KE = -\frac{1}{2}\left[\frac{\nabla^2 \Det{D}_{\uparrow}}{\Det{D}_{\uparrow}} + \frac{\nabla^2 \Det{D}_{\downarrow}}{\Det{D}_{\downarrow}} + \frac{\nabla^2 \Psi_C}{\Psi_C}\right] - \left[\frac{\Grad{\Det{D}_{\uparrow}}}{\Det{D}_{\uparrow}} +  \frac{\Grad{\Det{D}_{\downarrow}}}{\Det{D}_{\downarrow}}\right]\bullet \frac{\Grad{\Psi_C}}{\Psi_C}},
% % % % % \end{equation}
% % % % % where each of the components computed by the \citecode{SlaterJastrow} class are obtained by pointers acting on functions.\\
\begin{lstlisting}[language=c++]
....
class Parallelizer{
  private:
    int numprocs;       // Number of procesors
    int myRank;         // Number (label) of this procesor
    int rankRoot;       // Number (label) of the root procesor
    
    #if PARALLEL
      MPI_Status status;
    #endif
\end{lstlisting}

\begin{lstlisting}[language=c++]
  public: 
    Parallelizer(int argc, char** argv);
    ~Parallelizer(){mpi_final();}
    void mpi_final();	// Finalize MPI
    int read_MPI_numprocs() const;
    int read_MPI_myRank() const;
    int read_MPI_rankRoot() const;
};// end of Parallelizer class
\end{lstlisting}



\begin{lstlisting}[language=c++]
...
#include "Parallelizer.h"
...
class Energy{
  private:
    ...// 
    double eMean;	// Mean energy
    double eSum;	// Cumulative energy 
    double eSqSum;	// Cumulative energy squared
    
    #if OPTIMIZER_ON
      MyArray<double> daMean;	  // Mean value of dPhi/dAlpha
      MyArray<double> daSum;	  // Cumulative dPhi/dAlpha
      MyArray<double> ELdaMean;	  // Mean of local energy times dPhi/dAlpha
      MyArray<double> ELdaSum;	  // Cumulative local energy times dPhi/dAlpha
    #endif
                      
    ofstream ofile, blockofile; 
    int totalNMC;
  
    #if PARALLEL
      double totalESum;
      double totalESqSum;
      MyArray<double> allEnergies;	// Save all the local energies
      MPI_Status status;
    #endif
  
    Potential *potential;
    WaveFnc *psi;
    int numprocs, rankRoot, myRank;
    Parallelizer *para;
    
  public:  
    Energy(SimParameters *simpar, Potential *potential, WaveFnc *psi,Parallelizer * para);
    ~Energy();
    void computeExpectationValues(int mcCycles);
    void computeLocalEnergy(int cycle, const MyArray<double>& rnew);
    void dumpResults(const char* resultsFile);
    double getEnergy();
    MyArray<double>& getLocalParametricDerivative(const MyArray<double>& r);
    void getParametricDerivatives(MyArray<double>& dEda); 
    void resetCumulants();
    void sum_dPda(double E_Local, const MyArray<double>& r);
    void updateCumulants(double E_Local, const MyArray<double>& r);
};
...
\end{lstlisting}
Furthemore, the class defined by \citecode{Parallelizer.h}, appearing on the top of \citecode{Energy.h} and \citecode{VmcSimulation.h} files encapsulates information to handle MPI in a straighforward way.\\
\\
When working in parallel (by setting \citecode{PARALLEL=1} during the compilation time) each processor executes the QVMC algorithm separately. The energy data for each processor is accumulated in its own \citecode{eSum} member. To get the expectation value from all the computations we call \citecode{Energy::dumpResults()} with \citecode{PARALLEL=1} directive. Because \citecode{Energy.h} has access to \citecode{MPI} by means of \citecode{Parallelizer.h}, the only execution needed is
\\
\begin{lstlisting}[language=c++]
  // Collect data in total averages
  MPI_Reduce(&eSum, &totalESum, 1, MPI_DOUBLE, MPI_SUM, 0, MPI_COMM_WORLD);
  MPI_Reduce(&eSqSum, &totalESqSum, 1, MPI_DOUBLE, MPI_SUM, 0, MPI_COMM_WORLD);
\end{lstlisting}
inside this function, which can be translated as: "Collect the cumulative energies from all the processors, sum them and divide by the total number of processors to get the mean energy". When the statistical data are assumed to be uncorrelated, the mean energy, variance and error can be computed in
\\
\begin{lstlisting}[language=c++]
  if(myRank==rankRoot){
    ...
    energy    = totalESum/totalNMC;
    variance  = totalESqSum/totalNMC - energy*energy;
    error     = sqrt(variance/(totalNMC - 1)); 
    ...
  }
\end{lstlisting}

The way the energy is computed above gives only an estimate of the real variance, and it serves only as a reference to be compared with the real variance we obtain when using the blocking technique. The data necessary to do it are obtained from the same function by
\\
\begin{lstlisting}[language=c++]
...
// Set file name for this rank
  ostringstream ost;
  ost << "blocks_rank" << myRank << ".dat";
  
  // Open file for writting in binary format
  blockofile.open(ost.str().c_str(), ios::out | ios::binary); 
  
  // Write to file
  blockofile.write((char*)(allEnergies.A), ncycles*sizeof(double));
  
  blockofile.close(); // close file
\end{lstlisting}
\noindent
The \citecode{Energy} class is also equipped with functions for computing and returning the expectation value of the energy and its derivative with respect to the variational parameters. This information is used by the class \citecode{Optimizer} to optimize the parameters of the wave function that hopefully will lead to the computation of the minimum variational energy. The process of optimizing before attempting to run the production stage of a simulation is illustred below:

\begin{lstlisting}[language=c++]
#include "SimParameters.h"
#include "VmcSpecificApplication.h"
#include "VmcSimulation.h"
...
int main(int argc, char *argv[]){
  // Set some trial variational parameters
  ...
  // Declarations for the optimization
  int iter; double minEnergy=0.0;

  // Set the name of the files for the parameters 
  // of this simulation and the output
  ...
  // Encapsulate the parameters of this simulation in an object
  SimParameters *simpa = new SimParameters();
  ...
  // Create a "simulation" 
  VmcSimulation *vmcatoms = new VmcAtoms(simpa,varParam, argc, argv);

  // Set the Hamiltonian and the trial wave function corresponding
  // to atoms. Initialize space configuration
  vmcatoms->setPhysicalSystem(idum);
	
  // Run the optimizer
  vmcatoms->OptimizeWaveFunction(varParam, iter, minEnergy);
  
  // Output the results of the optimization
  ...
  return 0;
} // end main
\end{lstlisting}
And the class implementing the optimizer is:
%%%\begin{figure*}
\begin{lstlisting}[language=c++]
#include "Energy.h"
#include "MonteCarloIS.h"
#include "WaveFnc.h"
...
class Optimizer{
  private:
		int n;  		   				// Number of variational parameters
    double gtol;	   			// Tolerance for convergence
    Energy *pt2energy;
    MonteCarloIS *pt2mc;
    WaveFnc *pt2psi;
    ...
  public:
    Optimizer(int _n, double _gtol, Energy *_pt2energy, 
	      MonteCarloIS *_pt2mc, WaveFnc *_pt2psi):
	      n(_n), gtol(_gtol), pt2energy(_pt2energy), 
	      pt2mc(_pt2mc), pt2psi(_pt2psi){
      ...
    } 
		
		// Function to be called from VmcApp::OptimizeWaveFunction(...).
    void run(MyArray<double>& p, int& _iter, double& _fret){
      dfpmin(p, _iter, _fret); _iter = _iter+1;
    }
\end{lstlisting}

\begin{lstlisting}[language=c++]

    // Gets the expectation value of the energy for a set of variational 
    // parameters in the current Monte Carlo loop. 
    double func(MyArray<double>& p){
      pt2psi->setVariationalParameters(x);
      pt2mc->doMonteCarloLoop();
      return pt2energy->getEnergy();
    }

    // Sets the vector gradient g=df[1...n] of the function returned by 
    // func() evaluated using the input parameters p.
    void dfunc(MyArray<double>& p, MyArray<double>& g){
      pt2energy->getParametricDerivatives(x, g); 
    }
    
    // Given a starting set of parameters p[1...n] performs a Fletcher-
    // Reeves-Polak-Ribiere minimization on func(). 
    void dfpmin(MyArray<double>& p, int& _iter, double& _fret); 	
   ...   
};
\end{lstlisting}

\subsubsection{Running a simulation}
The body of the main method for the productive phase has almost the same form. One needs only to run a simulation with the optimal parameters found in the step above.
\begin{lstlisting}[language=c++]
...
int main(int argc, char** argv){
  ...
  // Set some trial variational parameters
  ...
  // Encapsulate the parameters of this simulation in an object
  SimParameters *simpa = new SimParameters();
  ...
  // Create a "simulation" 
  VmcSimulation *vmcatoms = new VmcAtoms(simpa,varParam, argc, argv);

  // Initialize the simulation
  vmcatoms->setPhysicalSystem(idum);
	
  // Start the simulation
  vmcatoms->runSimulation();

  // Print results to file
  vmcatoms->dumpResults(resultsFile);
  ...
  return 0;
}
\end{lstlisting}
% % % % \end{figure*}
% \end{src}
% \end{center}
\noindent
An overview of the classes and some of the functions to be implemented by the end user of the simulator are summarized in table \ref{endUserData}. Because of limitations of space and time, it is impossible to make a more detailed description of the simulator. The interested reader is, therefore, invited to consult the documentation of the source code following the appended CD of this thesis.


\begin{table}\label{userImplemented}
\centering
\begin{tabular}{llr}
\toprule[1pt]
\textbf{Class} & \textbf{Function}\\
\midrule[1pt]
\citecode{-} & \citecode{main()}  \\
\hline
\citecode{SingleStateWaveFncApp}  & \citecode{evaluate(...)}\\
				  & \citecode{getStateGradient(...)}\\
				  & \citecode{getLaplacian(...)}\\
				  & \citecode{...}\\
\hline
\citecode{PotentialApp}& \citecode{getPotentialEnergy(...)}\\
			& \citecode{getOneBodyPotential(...)}\\
			& \citecode{...}\\
\hline
\citecode{VmcApp}	&\citecode{setHamiltonian(...)}\\
			& \citecode{setWaveFunction(...)}\\
			& \citecode{...}\\
\bottomrule[1pt]
\end{tabular}\caption{Classes and functions to be implemented by the user of the software.}
\label{endUserData}
\end{table}

% % % % % % % \clearemptydoublepage


\section{Implementation of the double dot}
We will in this section outline the numerical methods and alogrithms related to finding the one-electron eigenvalues of double dot potential, see section \ref{section:the double dot potential}. The code can take a general potential in two dimentions, and find the eigenvalues and eigenvectors. But for our particular problem, we only diagonalize in $x$-direction, find the eigenvalues and eigenvectors. Then find the overlap coefficients with an harmonic oscillator basis
% 
\begin{equation}
  C_{n_x',n_X m_X} = \sum_{n_x}^N \braket{n_x'}{n_X m_X} \ket{n_X m_X} 
\end{equation}
%
Where $n_X$ and $m_X$ are the radial and angular quantum numbers from a polar basis solution, see Eq. \ref{total eigenstate}. Since there is no barrier in the $y$-direction, the hamiltonian \ref{rerescaled Hamiltonian} is separable and we get 
\begin{equation}
  C_{n_Y' m_Y', n_Y m_Y}  = \delta_{n_Y' m_Y', n_Y m_Y}  \ket{n_X m_X} 
\end{equation}
%
The relations between cartesian quantum numbers $(n_x,n_y)$ and $n,m$ are as follows 
\begin{equation}
 E = n_x + n_y + 1 = 2n + |m| + 1
\end{equation}
%
Then our new basis states can be expressed as linear combination of harmonic oscillator states 
%
\begin{equation}
 \ket{a} = \sum_{\alpha} C_\alpha \ket{\alpha} , \qquad C_\alpha =  C_{n_x',n_X m_X} C_{n_Ym_Y} 
 \label{new spinorbitals}
\end{equation}
%
Our goal is to find the two-particle interaction elements in the new basis
%
\begin{equation}
 \bra{ab}v\ket{cd} = \sum_{\alpha\beta\gamma\delta} C_\alpha^*C_\beta^*C_\gamma C_\delta \bra{\alpha\beta}v\ket{\gamma\delta}
 \label{new interactions}
\end{equation}
%
The interaction elements for the parabolic qunatum dot Eq. \ref{Hamiltonian for QD} $\bra{\alpha\beta}v\ket{\gamma\delta}$ can be obtained from Simen Kvaal's Open FCI code for quantum dots \cite{kvaalOpenFCI}. 
%
This is a transformation of the interaction elements from a polar basis to a cartesian basis. And these basis transformations are costly to calculate, we have tried optimalize as much as we could. 

These new interaction elements Eq. \ref{new interactions}, together with the new spinorbital quantum numbers Eq. \ref{new spinorbitals} $\ket{a}$ and single-particle energies $\braket{a}{b}$ are used as an input in our CCSD and Hartree-Fock. 

\subsection{Scaling the Hamiltonian}
One of the potentials we wish to study \cite{doubledot2}

\begin{equation}
V(x,y) = \frac{1}{2} m^* \omega_0^2 [x^2 + y^2 -2L_x|x| + L_x^2]
\end{equation}
%
We want our potential to dimentionless in our calculation. Introducing the scaled variables
\begin{align}
 L_x & = L_c \bar{L}_x \\
   x & = x_c \bar{x}   \\ 
   y & = y_c \bar{y}   \\
   \omega_0 &= \omega_c \bar{\omega}
\end{align}
Setting 
\begin{align}
 L_c & = x_c = y_c = l_0 \\
\end{align}
We get
\begin{equation}
  \bar{V} = \frac{1}{2} m^* \omega_0 l_0^2 \left[\bar{x}^2 + \bar{y}^2 - 2|\bar{x}| + \bar{L}_x \right]
\end{equation}
%
And setting
\begin{equation}
l_0 = \sqrt{\frac{\hbar}{m^* \omega_c \bar{\omega}}}
\end{equation}
%
Which gives
\begin{equation}
  \bar{V}_C  =  \frac{\hbar}{2} \omega_c \bar{\omega} \left[\bar{x}^2 + \bar{y}^2 - 2|\bar{x}| + \bar{L}_x \right]
    \label{Vabs}
\end{equation}
%
Defining $E_h$

\begin{equation}
 E_h = \frac{m^*}{\epsilon^2} \qquad \kappa = \frac{e^2}{4\pi \epsilon_0 \epsilon_r \hbar}
\end{equation}
%
Scaling the Hamiltonian $\hh{H} = E_h \bar{H}$, where $E_h$ is effective Hartrees.
 
\begin{equation}
  \bar{H} = -\frac{\omega_c \bar{\omega} \hbar \kappa^2}{2m^*} \bar{\nabla}^2 + \frac{\hbar\kappa^2}{2m^*}\omega_c\bar{\omega} \left[\bar{x}^2 + \bar{y}^2 - 2|\bar{x}| + \bar{L}_x \right]
\end{equation}
%
Setting 
\begin{equation}
 \frac{\hbar \kappa^2  \omega_c}{m^*} = 1
\end{equation}
%
We then get the one-body Hamiltonian
%
\begin{equation}
 \bar{H} = -\frac{\bar{\omega}}{2} \nabla + \frac{1}{2} \bar{\omega} \left[ \bar{x}^2 + \bar{y}^2 - 2|\bar{x}| + \bar{L}_x \right]
 \label{doublepotential abs}
\end{equation}
%
The two-body interaction is the same as in Eq. \ref{N-body Hamiltonian scaled}. And the total Hamiltonian for our model becomes
%
\begin{equation}
 \bar{H} = -\frac{\bar{\omega}}{2} \nabla + \frac{1}{2} \bar{\omega} \left[ \bar{x}^2 + \bar{y}^2 - 2|\bar{x}| + \bar{L}_x \right] + \sqrt{\bar{\omega}}\sum_{i<j}^N \frac{1}{\bar{r}_{ij}}
  \label{total hamiltonian double dot}
\end{equation}
%
In the article \cite{doubledot2}, they set $L_x = 50$ nm , then we can choose $L_c = 50$ nm and setting $\bar{L}_x = 1$, $\bar{\omega}=1$. 
%
\begin{figure}[H]
\centering
\scalebox{0.7}{\includegraphics{Doubledot3D.eps}}
\caption{Confinement potential for Eq. \ref{Vabs} $V(\bar{x},\bar{y})$, $\bar{L}_x = 10$ and $\bar{\omega} = 1$}
\end{figure}
%
\begin{figure}[H]
\centering
\scalebox{0.7}{\includegraphics{Doubledot2D.eps}}
\caption{Confinement potential for Eq. \ref{Vabs} $V(\bar{x},0)$, $\bar{L}_x = 10$ and $\bar{\omega} = 1$}
\end{figure}
%
The code that we is so general that we can change the potential, instead of the absolute value barrier $|\bar{x}|$, we can choose a gaussian curve barrier which is smoother. An example could be 
%
\begin{equation}
\bar{V}_G = \frac{1}{2}\bar{\omega}\left[\bar{x}^2 + \bar{y}^2 +V_0 \exp\left(\frac{\bar{x}^2}{2\sigma}\right)\right]
  \label{Vgauss}
\end{equation}
%
\begin{figure}[H]
\centering
\scalebox{0.7}{\includegraphics{Doubledot3D_gaussian.eps}}
\caption{Confinement potential for Eq. \ref{Vgauss} $V(\bar{x},\bar{y})$, $V_0 = 100$ and $\sigma=1, \omega = 1$}
\label{fig:Vgauss3D}
\end{figure}
%
\begin{figure}[H]
\centering
\scalebox{0.7}{\includegraphics{Doubledot2D_gaussian.eps}}
\caption{Confinement potential for Eq. \ref{Vgauss} $V(x,0)$, $V_0 = 20,\sigma=1$ and $\bar{\omega} = 1$}
\label{fig:Vgauss2D}
\end{figure}
%


\subsection{Finding the eigenvectors and eigenvalues}
%
We diagonalize the one-body Hamiltonian in $x$-direction
\begin{equation}
\hh{H}_X = -\frac{\bar{\omega}}{2}\frac{d}{d\bar{x}^2} + V(\bar{x}) 
  \label{we want to diagonalize}
\end{equation}
%
In order to find the eigenvalues and the eigenvectors we can use the technique of discretization described in section \ref{section:the double dot potential}. We use the algorithm (Listing \ref{list:fillingtridiagonal}) to create the tridiagonal matrix Eq. \ref{tridiagonal}

\begin{lstlisting}[label={list:fillingtridiagonal},caption={Filling the tridiagonal matrix}]
    double e = 0.5 * (-1.0) / (xstep * xstep);
    double f = 1.0 / (xstep * xstep);
    double V = 0;
    for (int i = 0; i < N - 1; i++) {
        x = xmin + i*xstep;
        //        r[i] = x;
        V = potential(x);

        //Filling the diagonal
        T[i][i] = f + V;

        //Filling the off diagonal;
        T[i][i + 1] = e;
        T[i + 1][i] = e;
    }
    //Filling the Last diagonal element
    x = xmin + (N - 1) * xstep;
    V = potential(x);
    T[N - 1][N - 1] = f + V;
\end{lstlisting}
 %
We use the \texttt{LAPACK} routine for solving a symmetric dense matrix, \texttt{dsyev()}, which basically use the accelerated QR-Algorithm, section \ref{section:The QR-Algorithm}. The eigenvalues $E_{n_x'}$ are then sorted in ascending order,

\begin{equation}
 n_{x}' = 0,1,2...N \qquad E_{0} \geq E_1 \geq E_2 \geq ... \geq E_{n_x' = N}
 \label{eigenvalues found by diag}
\end{equation}


\begin{equation}
  \psi_{n_x'} = \begin{bmatrix} \psi_{n_x'}(x_{\min}) \\ \psi_{n_x'}(x_1) \\ \vdots \\ \psi_{n_x}(x_N) \end{bmatrix}, \qquad x_j = x_{\min} + j\cdot h, \qquad j \in \{0,1,..,N\}                           
\end{equation}
%

\subsection{Finding the coefficients}
We want to find the overlap coefficients
\begin{equation}
C_{n_x',n} = \braket{\psi_{n_x'}}{n} = \sum_{n}^N \psi_{n_x'}(x_i) \psi_n(x_i) \qquad n = 0,1,2,...N
  \label{overlap}
\end{equation}
%
\begin{equation}
x_i = x_{\min} + i \cdot h \qquad h = \frac{x_{\max} - x_{\min}}{N} 
\end{equation}

Where $\psi_n(x)$ are the harmonic oscialltor eigenfuncions in one dimention:
%
\begin{align}
  \psi_n(x) & = \left(\frac{\beta}{\pi}\right)^{1/4} \frac{1}{\sqrt{2^n n!}} H_n(x)e^{\frac{\beta x^2}{2}}
\end{align}
%
\begin{equation}
 \beta = \frac{m \omega}{\hbar}
\end{equation}
%
$H_n(x)$ are the Hermite polynomials \cite{boas}. Using the recursive relation for the Hermite polynomials 
%
\begin{align}
H_{n+1}(x) = 2xH_n(x) - 2nH_{n-1}(x)
\end{align}
%
We get
\begin{align}
  \psi_{n+1} & = \left(\frac{1}{\pi}\right)^{1/4} \frac{1}{\sqrt{2^{n+1} (n+1)!}} H_{n+1}e^{\frac{1 x^2}{2}} \\
             & = \left(\frac{1}{\pi}\right)^{1/4} \frac{1}{\sqrt{2^{n+1} (n+1)!}} \left( 2xH_n - 2nH_{n-1} \right) \\
             & = x\sqrt{\frac{2}{n+1}}\psi_n - \sqrt{\frac{n}{n+1}}\psi_{n-1} 
\end{align}
%
Or 
%
\begin{align}
  \psi_{n} &= x\sqrt{\frac{2}{n}}\psi_{n-1} - \sqrt{\frac{n-1}{n}}\psi_{n-2} 
\end{align}
%
Where
%
\begin{align}
  \psi_0 & = \left(\frac{1}{\pi}\right)^{1/4} e^{-\frac{ x^2}{2}} \\
  \psi_1 & = \left(\frac{1}{\pi}\right)^{1/4} \frac{1}{\sqrt{2}} 2x e^{-\frac{ x^2}{2}}
\end{align}
%
The algorithm for generating the harmonic oscillator wavefunctions and evalute at $x$, $\beta = 1$
%
\begin{lstlisting}
double wf(int n, double x) {
    double *W = new double[n + 2];
    double a = (1.0 / pow(pi, 0.25));
    double b = (1.0 / sqrt(2));

    W[0] = 1.0 * a; //psi_0(x)

    if (n == 0) {
        return W[0] * exp(-0.5 * x * x);
    } else {
        W[1] = sqrt(2) * x * a; //psi_1(x)
        for (int i = 1; i <= n - 1; i++) {

            W[i + 1] = (sqrt(2.0 / (i + 1)) * x * W[i] - sqrt(double(i) / (i + 1)) * W[i - 1]);
        }
        return W[n] * exp(-0.5 * x * x);
    }

    delete [] W;
\end{lstlisting}
%
The coefficients are then calculated by the following algorithm
%
\begin{lstlisting}
    for (int nxprime = 0; nxprime < nxprime_max; nxprime++) {
        for (int n = 0; n < nmax; n++) {
            innerprod = 0;
            for (int i = 0; i < N; i++) {
                innerprod += xstep * Evecs[i][nxprime] * H[i][n];
            }
            Coeff[n][nxprime] = innerprod;
        }
    }
\end{lstlisting}

\subsubsection{Validation}
%
One way to validate the coefficients is to check if it preserve the probability.
\begin{equation}
  P(n') = \sum_{i}^N C_{i,n'}^*C_{i,n'} \approx 1
\end{equation}
%
Another check is to plot the diagonalized curve against the linear expansion and see if 
\begin{equation}
  \sum_i^N |\psi_{n'}(x_i) - \psi_n(x_i)|^2 < \epsilon
\end{equation}


\subsection{Transformation from polar to cartesian representation}
\label{sec:cartesian basis transform}

Harmonic oscillator wavefunctions in two dimentions for som selected values
%
\begin{align}
\psi^{\text{\tiny HO}}_{n_x,n_y}(x,y) & = \psi^{\text{\tiny HO}}_{n_x}(x)\psi^{\text{\tiny HO}}_{n_y}(y)   
  \label{Harmonic Oscillator Wavefunctions Cartesian}
\end{align}
%
\begin{align}
\psi^{\text{\tiny HO}}_{n_x,n_y}(x,y) = H_{n_x}(x)H_{n_y}(y) e^{-x^2} e^{-y^2}  
  \label{Harmonic Oscillator Wavefunctions Cartesian Explicit}
\end{align}
In cartesian representation (not normalized): 
\begin{align*}
\psi^{\text{\tiny HO}}_{0,0}(x,y) & =  e^{-x^2}e^{-y^2}   & (\epsilon = 1) \\
\psi^{\text{\tiny HO}}_{1,0}(x,y) & =  xe^{-x^2}e^{-y^2} & (\epsilon = 2) \\
\psi^{\text{\tiny HO}}_{0,1}(x,y) & =  ye^{-x^2}e^{-y^2} & (\epsilon = 2) \\
\psi^{\text{\tiny HO}}_{1,1}(x,y) & =  yxe^{-x^2/2}e^{-y^2} & (\epsilon = 3) \\ 
\end{align*}
%
Where $\epsilon = n_x + n_y + 1$.
%
In polar coordinate representation (not normalize):
\begin{align}
\psi^{\text{\tiny HO}}_{n,m}(r,\theta) & = R(r)\phi(\theta)  
  \label{Harmonic Oscillator Wavefunctions Polar}
\end{align}
%
\begin{align}
\psi^{\text{\tiny HO}}_{n,m}(r,\theta) & = r^{|m|}e^{-r^2}L_n^{|m|}e^{im\theta}  
  \label{Harmonic Oscillator Wavefunctions Polar Explicit}
\end{align}
\begin{align*}
\psi^{\text{\tiny HO}}_{0,0}(r,\theta)   & =  e^{-r^2}   & (\epsilon = 1) \\
\psi^{\text{\tiny HO}}_{0,1}(r,\theta)   & =  r e^{-r^2}e^{i\theta} & (\epsilon = 2) \\
\psi^{\text{\tiny HO}}_{0,-1}(r,\theta)  & =  r e^{-r^2}e^{-i\theta} & (\epsilon = 2) \\
\psi^{\text{\tiny HO}}_{1,0}(r,\theta)   & =  (-r^2+1)e^{-r^2} & (\epsilon = 3) \\ 
\end{align*}
%
Where $\epsilon = 2n + |m| + 1$. We transformation between the representation by the following
\begin{align*}
x & = r \cos(\theta) \\
y & = r \sin(\theta) \\
r^2 & = x^2 + y^2
  \label{def:CartesianPolarMapping}
\end{align*}
%
We have that
%
\begin{equation}
\cos(\theta) = \frac{e^{i\theta} + e^{i\theta}}{2} \qquad \cos(\theta) = \frac{e^{i\theta} - e^{i\theta}}{2i}
  \label{cosinus relation}
\end{equation}
%
We see that (normalized):
\begin{align}
\psi^{\text{\tiny HO}}_{0,0}(x,y) & = \psi^{\text{\tiny HO}}_{0,0}(r,\theta)\\
\psi^{\text{\tiny HO}}_{1,0}(x,y) & = \frac{1}{\sqrt{2}}\psi^{\text{\tiny HO}}_{1,1}(r,\theta) +   \frac{1}{\sqrt{2}}\psi^{\text{\tiny HO}}_{1,-1}(r,\theta)\\
\psi^{\text{\tiny HO}}_{0,1}(x,y) & =  \frac{i}{\sqrt{2}}\psi^{\text{\tiny HO}}_{1,-1}(r,\theta) -   \frac{i}{\sqrt{2}}\psi^{\text{\tiny HO}}_{1,1}(r,\theta)\\
  \label{we see mapping} 
\end{align}
%
The Transformation matrix $T^{N}$, $N = n_x + n_y$ :

\begin{align}
T^{0}  & = 1 \\
T^{1}  & = \begin{bmatrix*}[r] \frac{1}{\sqrt{2}} & \frac{1}{\sqrt{2}} \\ \frac{1}{\sqrt{2}} & -\frac{1}{\sqrt{2}}
 \end{bmatrix*} \\
T^{2}  & = \begin{bmatrix*}[r] \frac{1}{2} & \frac{1}{\sqrt{2}} & \frac{1}{2} \\ \frac{1}{\sqrt{2}} & 0 & -\frac{1}{\sqrt{2}} \\ \frac{1}{2} & -\frac{1}{\sqrt{2}} & \frac{1}{2} \end{bmatrix*} \\
T^{N} & = ...
\end{align}
%
These transformation matrices are obtained from Simen Kvaals PhD-thesis \cite{kvaalphd}
Our transformation from a polar basis to a cartesian one is defined
\begin{equation}
\ket{a} \equiv \ket{n_x'(a), n_y'(a)}^{CAR} = \sum_{n_x(a)} C_{n_x(a), n_x'(a)} \sum_{n_+(a)}^{N_a} i^{-n_y'(a)} T_{n_+(a),n_x(a)}^{N_a}
  \label{DoubleDot}
\end{equation}
%
Where $N_a = n_x(a) + n_y'(a)$. 
%
Then the interaction elements in the cartesian space represented by 
%
\begin{align*}
\bra{ab}v\ket{cd}^{CAR} &= \sum_{n_x(a)}\sum_{n_x(b)}\sum_{n_x(c)}\sum_{n_y(d)} C_{n_x(a), n_x'(a)} C_{n_x(b), n_x'(b)} C_{n_x(c), n_x'(c)} C_{n_x(d), n_x'(d)} \\
&\times \sum_{n_+(a)}^{N_a}\sum_{n_+(b)}^{N_b}\sum_{n_+(c)}^{N_c}\sum_{n_+(d)}^{N_d} T_{n_+(a),n_x(a)}^{N_a} T_{n_+(b),n_x(b)}^{N_b} T_{n_+(c),n_x(c)}^{N_c} T_{n_+(d),n_x(d)}^{N_d} \\
&\times \bra{\underbrace{n_+(a)n_-(a)}_{a}|\underbrace{n_+(b)n_-(b)}_b}v\ket{\underbrace{n_+(c)n_-(c)}_c|\underbrace{n_+(d)n_-(d)}_d}^{POL} \\
&\times i^{n_y'(a)} i^{n_y'(b)} i^{-n_y'(c)} i^{-n_y'(d)}
\label{cartesian interaction element representation}
\end{align*}
%
Where 
\begin{align}
  m = n_+ - n_- \qquad n = \frac{n_+ + n_- - |m|}{2} 
\end{align}
%
Think of it as a ladder, where number of $n_+$, defines how high to the right you move on the ladder, and $n_-$ defines how high to the left.
%
Therefore each of the subshells will be defined by a set of $\{n_+,n_-\}$. 
%
\begin{figure}
\centering
\input{simenmap}
\caption{Shell structure of the quantum dot, and the relations to the quantum number $n_+$, $n_-$. There is a 1-1 map between the radial quantum number $n$ and angular momentum quantum number $m_s$ to $\{n_+,n_-\}$ }
\label{fig:simenmap}
\end{figure}


We can do a simple example using the parabolic harmonic oscillator potential, i.e. $L_x=0$ and $V_0 = 0$. Our system is a harmonic oscillator with 2 shell level, and the basis states are:
\begin{align*}
\ket{0} \rightarrow &n_x'=0, n_y'=0, m_s' = -0.5, &E = 0.99988 \\  
\ket{1} \rightarrow &n_x'=0, n_y'=0, m_s' = 0.5,  &E = 0.99988 \\  
\ket{2} \rightarrow &n_x'=1, n_y'=0, m_s' = -0.5, &E = 1.99943 \\  
\ket{3} \rightarrow &n_x'=1, n_y'=0, m_s' =  0.5, &E = 1.99943 \\  
\ket{4} \rightarrow &n_x'=0, n_y'=1, m_s' = -0.5, &E = 1.99988 \\  
\ket{5} \rightarrow &n_x'=0, n_y'=1, m_s' =  0.5, &E = 1.99988  
\end{align*}
Polar basis:
\begin{align*}
  \ket{0} & = \{n=0, m=0, m_s = -0.5\} \\ 
  \ket{1} & = \{n=0, m=0, m_s = 0.5\} \\
  \ket{2} & = \{n=0, m=-1, m_s = -0.5\} \\
  \ket{3} & = \{n=0, m=-1, m_s = 0.5\} \\
  \ket{4} & = \{n=0, m=+1, m_s = -0.5\} \\
  \ket{5} & = \{n=0, m=+1, m_s = 0.5\}   
\end{align*}
Talmi matrix:
\begin{align}
  T^{(1)} &= 1 \\
  T^{(2)} &= \begin{bmatrix}\frac{1}{\sqrt{2}} & \frac{1}{\sqrt{2}} \\ \frac{1}{\sqrt{2}} &- \frac{1}{\sqrt{2}} \end{bmatrix}                             
\end{align}
%
Want to find the interaction element $\bra{45}v\ket{45}^{CAR}$:
%
\begin{equation}
  \ket{n_x' n_y'}^{CAR} = \sum_{n_x} C_{n_x',n_x} \sum_{n_+}^{n_x+n_y'} T_{n_+,n_x}^{n_x+n_y'} \ket{n_+,n_-}^{POL}
\end{equation}

\begin{align}
  \ket{4}^{CAR} & = T_{0,0}^{(1)} \ket{(n_+=0,n_-=1)}^{POL} + T_{1,0}^{(1)} \ket{(n_+=1,n_-=0)}^{POL} \\
  \ket{5}^{CAR} & = T_{0,0}^{(1)} \ket{(n_+=0,n_-=1)}^{POL} + T_{1,0}^{(1)} \ket{(n_+=1,n_-=0)}^{POL}  
\end{align}
%
If we translate these states into the Polar basis quantum numbers we get, and remember that $\ket{4}$ is a spin down states, so $\{n_+,n_-\}$ must map to a state which is spin down:
\begin{align}
  \ket{4}^{CAR} & = T_{0,0}^{(1)} \ket{4}^{POL} + T_{1,0}^{(1)} \ket{2}^{POL} \\
  \ket{5}^{CAR} & = T_{0,0}^{(1)} \ket{5}^{POL} + T_{1,0}^{(1)} \ket{3}^{POL}  
\end{align}
%
Then 
\begin{align}
  \ket{45}^{CAR} & = (T_{0,0}^{(1)})^2 \ket{4 5}^{POL} + T_{0,0}^{(1)}T_{1,0}^{(1)} \ket{43}^{POL} + T_{1,0}^{(1)}T_{0,0}^{(1)} \ket{25}^{POL} + (T_{1,0}^{(1)})^2 \ket{23}^{POL}  
\end{align}
%
Then 
\begin{align}
  \bra{45}v\ket{45}^{CAR} & = \frac{1}{2}\frac{1}{2} \bra{45}v\ket{45}^{POL} + \frac{1}{2}\frac{1}{2} \bra{43}v\ket{25}^{POL}  + \frac{1}{2}\frac{1}{2} \bra{43}v\ket{43}^{POL} \\
  &+ \frac{1}{2}\frac{1}{2} \bra{25}v\ket{25}^{POL} + \frac{1}{2}\frac{1}{2} \bra{25}v\ket{43}^{POL}+ \frac{1}{2}\frac{1}{2} \bra{23}v\ket{23}^{POL}  
\end{align}
%
List of interaction elements from \texttt{tabulate.x} with $\omega = 1.0$ and standard v.v.:
%
\begin{align}
\bra{45}v\ket{45}^{POL} & =  0.86165\\
\bra{43}v\ket{25}^{POL} & =  0.23499\\
\bra{43}v\ket{43}^{POL} & =  0.86165\\
\bra{25}v\ket{25}^{POL} & =  0.86165\\
\bra{25}v\ket{43}^{POL} & =  0.23499\\
\bra{23}v\ket{23}^{POL} & =  0.86165
\end{align}
%
The our interaction element becomes
\begin{align}
  \bra{45}v\ket{45}^{CAR} & = 0.86165 + \frac{1}{2}0.23499 = 0.97915
\end{align}
%
\begin{align}
  \bra{45}v\ket{45}^{POL} & = 0.86165
\end{align}

\subsection{Tabulating new quantum numbers}
In our cartesian basis, the two-particle interaction still does not change spin. Therefore the total spin is conserved.
 
\begin{equation}
  \bra{M_s}v\ket{M_s'} = \delta_{M_s,M_s'} \qquad M_s = m_{1s} + m_{2s} 
\end{equation}

\begin{table}[H]
\centering
\begin{minipage}[b]{0.3\linewidth}\centering 
\begin{tabular}{ccc}
\toprule
$M_s$ & $\alpha$ & State\\
\midrule
 -1 &  0  &$\ket{0,2}$ \\
 -1 &  0  &$\ket{0,4}$ \\
 -1 &  0  &$\ket{2,4}$ \\
\bottomrule
\end{tabular}
\caption{Tabulated two-particle basis for $M_s=-1$}
\end{minipage}
\hspace{0.2cm}
\begin{minipage}[b]{0.3\linewidth}
\begin{tabular}{ccc}
\toprule
$M_s$ & $\alpha$ & State\\
\midrule
  0 &  1  &$\ket{0,1}$ \\
  0 &  1  &$\ket{0,3}$ \\
  0 &  1  &$\ket{0,5}$ \\
  0 &  1  &$\ket{1,2}$ \\
  0 &  1  &$\ket{1,4}$ \\
  0 &  1  &$\ket{2,3}$ \\
  0 &  1  &$\ket{2,5}$ \\
  0 &  1  &$\ket{3,4}$ \\
  0 &  1  &$\ket{4,5}$ \\
  0 &  1  &$\ket{1,3}$ \\
\bottomrule
\end{tabular}
\caption{Tabulated two-particle basis for $M_s=0$}
\end{minipage}
\hspace{0.2cm}
\begin{minipage}[b]{0.3\linewidth}
\begin{tabular}{ccc}
\toprule
$M_s$ & $\alpha$ & State\\
\midrule
  1 &  2  &$\ket{3,4}$ \\
  1 &  2  &$\ket{1,5}$ \\
  1 &  2  &$\ket{3,5}$ \\
\bottomrule
\end{tabular}
\caption{Tabulated two-particle basis for $M_s=1$}
\end{minipage}
\end{table}
%
Every odd number is now a spin up state, and even number are spin down. The energy is thereby degenerate in spin.
%
\begin{equation}
\ket{a} = \left\{ \begin{array}{lcl} a = 2n & \mbox{if} & m_s=-0.5\\ 
 a = 2n+1 & \mbox{if} & m_s = 0.5
\end{array}\right.
\end{equation}
%
In section \ref{parabolic quantum dot} we discussed the shell structure of the parabolic quantum dot. We have the same situasjon now but with a different kind of subshells. We define a subshell with the set of quantum numbers $\{n_x',n_y'\}$, where each subshell have the energy

\begin{equation}
E_{sub} = E_{n_x'} + E_{n_y'}
  \label{subshell energy}
\end{equation}
%
And $E_{n_x'}$ are the eigenvalues found by diagonalization Eq. \ref{eigenvalues found by diag} and Fig. \ref{fig:Vgauss2D}, Fig. \ref{fig:Vgauss3D}. While $E_{n_y'} = (0.5 + n_y')$ are the one-dimentional solution to the harmonic oscillator potential. Example for the potential \ref{Vgauss}, $V=20,\sigma=1$, with $N=500$.

\begin{table}[H]
\centering
\begin{minipage}[b]{0.4\linewidth}\centering 
\begin{tabular}{cc}
\toprule
${n_x'}$ & $E_{n_x'}$\\
\midrule
  0 &  0  5.68161 \\
  1 &  0  5.68162 \\
  2 &  0  7.42767 \\
\bottomrule
\end{tabular}
\caption{Energyeigenvaleus for $n_x'$}
\end{minipage}
\hspace{1cm}
\begin{minipage}[b]{0.4\linewidth}
\begin{tabular}{cc}
\toprule
$n_y'$ & $E_{n_y'}$\\
\midrule
  0 & 0.5   \\
  1 &  1   \\
  2 &  2   \\
\bottomrule
\end{tabular}
\caption{Energyeigenvaleus for $n_y'$}
\end{minipage}
\end{table}
The first 6 spinorbitals are tabulated as

\begin{align*}
  \ket{0} & = \{n_x'=0, n_y'=0, m_s = \downarrow\} & E =  5.68161 \\ 
  \ket{1} & = \{n_x'=0, n_y'=0, m_s = \uparrow \}  & E =  5.68161 \\
  \ket{2} & = \{n_x'=1, n_y'=0, m_s = \downarrow\} & E = 5.68161 \\
  \ket{3} & = \{n_x'=1, n_y'=0, m_s = \uparrow \}  & E =  5.68161 \\
  \ket{4} & = \{n_x'=1, n_y'=1, m_s = \downarrow\} & E = 6.68162\\
  \ket{5} & = \{n_x'=1, n_y'=1, m_s = \uparrow \}  & E = 6.68162 
\end{align*}



\subsection{Validation of the Code}
One of the test we can do is to drop the barrier in the $x$-direction, i.e. $L_x = 0$ in Eq. \ref{doublepotential abs}. Then we get a harmonic oscillator potential. And we want to do a diagonalization for the 2-shell case. We get 6 single-particle states with eigenvalues in the cartesian basis:

\begin{align*}
\ket{0} \rightarrow &\{n_x'=0, n_y'=0, m_s' = \downarrow\} &E = 0.99988 \\  
\ket{1} \rightarrow &\{n_x'=0, n_y'=0, m_s' = \uparrow \} &E =  0.99988 \\  
\ket{2} \rightarrow &\{n_x'=1, n_y'=0, m_s' = \downarrow\} &E = 1.99943  \\  
\ket{3} \rightarrow &\{n_x'=1, n_y'=0, m_s' =  \uparrow\} &E =  1.99943  \\  
\ket{4} \rightarrow &\{n_x'=0, n_y'=1, m_s' = \downarrow\} &E = 1.99988 \\  
\ket{5} \rightarrow &\{n_x'=0, n_y'=1, m_s' =  \uparrow\} &E =  1.99988  
\end{align*}

If we start with the case $M_s = -1$
\begin{equation*}
\mathbf{H} = \begin{bmatrix}
\bra{\Phi_{02}}H_0\ket{\Phi_{02}} + \bra{\Phi_{02}}V\ket{\Phi_{02}} & 
\bra{\Phi_{02}}H_0\ket{\Phi_{04}} + \bra{\Phi_{02}}V\ket{\Phi_{04}} &  
\bra{\Phi_{02}}H_0\ket{\Phi_{24}} + \bra{\Phi_{02}}V\ket{\Phi_{24}} \\
\bra{\Phi_{04}}H_0\ket{\Phi_{02}} + \bra{\Phi_{04}}V\ket{\Phi_{02}} & 
\bra{\Phi_{04}}H_0\ket{\Phi_{04}} + \bra{\Phi_{04}}V\ket{\Phi_{04}} &  
\bra{\Phi_{04}}H_0\ket{\Phi_{24}} + \bra{\Phi_{04}}V\ket{\Phi_{24}} \\
\bra{\Phi_{24}}H_0\ket{\Phi_{02}} + \bra{\Phi_{24}}V\ket{\Phi_{02}} & 
\bra{\Phi_{24}}H_0\ket{\Phi_{04}} + \bra{\Phi_{24}}V\ket{\Phi_{04}} &  
\bra{\Phi_{24}}H_0\ket{\Phi_{24}} + \bra{\Phi_{34}}V\ket{\Phi_{24}} \\
\end{bmatrix}
  \label{validationDiagonalizationoftheHarmonicOscillator}
\end{equation*}
%
Where the Slater determinant is defined as
\begin{equation}
\ket{\Phi_{ab}} \equiv \frac{1}{\sqrt{2}}\left(\ket{ab}-\ket{ba}\right)
  \label{slater}
\end{equation}
%
Then we get
\begin{align*}
\mathbf{H} & \approx \begin{bmatrix}
2 + \bra{02}v\ket{02}_{\text{AS}} & 
    \bra{02}v\ket{04}_{\text{AS}} &  
    \bra{02}v\ket{24}_{\text{AS}} \\
    \bra{04}v\ket{02}_{\text{AS}} & 
4 + \bra{04}v\ket{04}_{\text{AS}} &  
    \bra{04}v\ket{24}_{\text{AS}} \\
    \bra{24}v\ket{02}_{\text{AS}} & 
    \bra{24}v\ket{04}_{\text{AS}} &  
4 + \bra{24}v\ket{24}_{\text{AS}} \\
\end{bmatrix}\\
&= \begin{bmatrix}
    3.6260   &      0    &     0 \\
         0  &  3.6264     &    0 \\
         0   &      0   & 4.6260 \\
\end{bmatrix}\\
  \label{validationDiagonalizationoftheHarmonicOscillator}
\end{align*}
%
Which is what we get when we diagonalize for $M_s=0$ in polar basis. The case $M_s=1$ is excatly the same, and for 
the $Ms = 0$ we get the eigvalues 

\begin{align*}
E_0 &= 3.1523 \qquad E_1 = 3.6260 \qquad E_2 = 3.6264 \\
E_3 &= 4.2533 \qquad E_4 = 4.2533 \qquad E_5 = 4.6267 \\
E_6 &= 4.8617 \qquad E_7 = 4.8617 \qquad E_8 = 5.1976 
\end{align*}
%
Which is the same as for the polar basis. 

\chapter{Results}

\begin{table}[H]
\centering 
\begin{tabular}{cccc|cccc}
\toprule
\multicolumn{8}{c}{$nh=2,\,\,\omega=1,\,\,L_x=0$} \\
\midrule
\multicolumn{4}{c|}{Standard Interaction} & \multicolumn{4}{c}{Effective Interaction} \\
$R_m$ & $R$ & $HF$ & $CCSD$ & $R_m$ & $R$ & $HF$ & $CCSD$\\
\midrule 
2 & 3	  & 3.253314	  &  3.152329 & 2 & 3  &  3.1307082 &  3.107322  \\
3 & 6	  & 3.162691	  &  3.038605 & 3 & 6  &  3.1307082 &  3.022718  \\
4 & 10	  & 3.162691	  &  3.025231 & 4 & 10 &  3.1387972 &  3.014679  \\
5 & 15	  & 3.161921	  &  3.017606 & 5 & 15 &  3.1428785 &  3.009848  \\
6 & 21	  & 3.161921	  &  3.013626 & 6 & 21 &  3.1460900 &  3.007503  \\
7 & 28	  & 3.161909	  &  3.011020 & 7 & 28 &  3.1483639 &  3.005983  \\ 
8 & 36	  & 3.161909	  &  3.009236 & 8 & 36 &  3.1500757 &  3.004966  \\
9 & 45	  & 3.161908	  &  3.007930 & 9 & 45 &  3.1514044 &  3.004230  \\
10 &55	  & 3.161908      &  3.006938 & 10& 55 &  3.1524655 &  3.003678  \\
\bottomrule
\end{tabular}
\caption{Ground-state energies for holes states $nh=2$ in a circular quantum dot using a standard coulomb interaction for $L_x=0$. The interaction elements are obtained from a cartesian basis transformation, section \ref{sec:cartesian basis transform}. HF is the Hartree Fock energy and CCSD is the coupled-cluster energy Eq. \ref{Vabs}. $R_m$ is the total number of subshells while $R_m$ stands for the number of major oscillator shells Fig. \ref{fig:shellnumber}}
\end{table}
%
We define new shell structure, each subshell consist of quantum numbers $n_x'$ and $n_y'$ and they can be occupied with at most 2 electrons. In the special case where we dont have any perturbation in the middle $L_x=0$ or $V_0=0$, some subshells are part of the major oscillator shells which we have discussed in section \ref{parabolic quantum dot}. The relations are

\begin{equation}
R = \frac{R_m(R_m+1)}{2}
\label{RmvsR}
\end{equation}

\begin{table}[H]
\small
\centering 
\begin{tabular}{cccrc|cccrc}
\hline
$p$ & $n_x'$ & $n_y'$ & $m_s'$ & $E$ & $p$ & $n_x'$ & $n_y'$ & $m_s'$ & $E$\\
\hline
0  & 0 & 0 & -1 & 1 & 55 & 0 & 6 &  1 & 7   \\
1  & 0 & 0 &  1 & 1 & 56 & 7 & 0 & -1 & 8   \\
2  & 1 & 0 & -1 & 2 & 57 & 7 & 0 &  1 & 8   \\
3  & 1 & 0 &  1 & 2 & 58 & 6 & 1 & -1 & 8   \\
4  & 0 & 1 & -1 & 2 & 59 & 6 & 1 &  1 & 8   \\
5  & 0 & 1 &  1 & 2 & 60 & 5 & 2 & -1 & 8   \\ 
6  & 2 & 0 & -1 & 3 & 61 & 5 & 2 &  1 & 8   \\
7  & 2 & 0 &  1 & 3 & 62 & 4 & 3 & -1 & 8   \\
8  & 1 & 1 & -1 & 3 & 63 & 4 & 3 &  1 & 8   \\
9  & 1 & 1 &  1 & 3 & 64 & 3 & 4 & -1 & 8   \\
10 & 0 & 2 & -1 & 3 & 65 & 3 & 4 &  1 & 8   \\
11 & 0 & 2 &  1 & 3 & 66 & 2 & 5 & -1 & 8   \\
12 & 3 & 0 & -1 & 4 & 67 & 2 & 5 &  1 & 8   \\
13 & 3 & 0 &  1 & 4 & 68 & 1 & 6 & -1 & 8   \\ 
14 & 2 & 1 & -1 & 4 & 69 & 1 & 6 &  1 & 8   \\
15 & 2 & 1 &  1 & 4 & 70 & 0 & 7 & -1 & 8   \\
16 & 1 & 2 & -1 & 4 & 71 & 0 & 7 &  1 & 8   \\
17 & 1 & 2 &  1 & 4 & 72 & 8 & 0 & -1 & 9   \\
18 & 0 & 3 & -1 & 4 & 73 & 8 & 0 &  1 & 9   \\
19 & 0 & 3 &  1 & 4 & 74 & 7 & 1 & -1 & 9   \\
20 & 4 & 0 & -1 & 5 & 75 & 7 & 1 &  1 & 9   \\
21 & 4 & 0 &  1 & 5 & 76 & 6 & 2 & -1 & 9   \\
22 & 3 & 1 & -1 & 5 & 77 & 6 & 2 &  1 & 9   \\
23 & 3 & 1 &  1 & 5 & 78 & 5 & 3 & -1 & 9   \\
24 & 2 & 2 & -1 & 5 & 79 & 5 & 3 &  1 & 9   \\
25 & 2 & 2 &  1 & 5 & 80 & 4 & 4 & -1 & 9   \\ 
26 & 1 & 3 & -1 & 5 & 81 & 4 & 4 &  1 & 9   \\
27 & 1 & 3 &  1 & 5 & 82 & 3 & 5 & -1 &	9   \\
28 & 0 & 4 & -1 & 5 & 83 & 3 & 5 &  1 & 9   \\
29 & 0 & 4 &  1 & 5 & 84 & 2 & 6 & -1 & 9   \\
30 & 5 & 0 & -1 & 6 & 85 & 2 & 6 &  1 & 9   \\
31 & 5 & 0 &  1 & 6 & 86 & 1 & 7 & -1 & 9   \\
32 & 4 & 1 & -1 & 6 & 87 & 1 & 7 &  1 & 9   \\
33 & 4 & 1 &  1 & 6 & 88 & 0 & 8 & -1 & 9   \\ 
34 & 3 & 2 & -1 & 6 & 89 & 0 & 8 &  1 & 9   \\
35 & 3 & 2 &  1 & 6 & 90 & 9 & 0 & -1 & 10  \\
36 & 2 & 3 & -1 & 6 & 91 & 9 & 0 &  1 & 10  \\
37 & 2 & 3 &  1 & 6 & 92 & 8 & 1 & -1 & 10  \\
38 & 1 & 4 & -1 & 6 & 93 & 8 & 1 &  1 & 10  \\
39 & 1 & 4 &  1 & 6 & 94 & 7 & 2 & -1 & 10  \\
40 & 0 & 5 & -1 & 6 & 95 & 7 & 2 &  1 & 10  \\
41 & 0 & 5 &  1 & 6 & 96 & 6 & 3 & -1 & 10  \\
42 & 6 & 0 & -1 & 7 & 97 & 6 & 3 &  1 & 10  \\
43 & 6 & 0 &  1 & 7 & 98 & 5 & 4 & -1 & 10  \\
44 & 5 & 1 & -1 & 7 & 99 & 5 & 4 &  1 & 10  \\
45 & 5 & 1 &  1 & 7 &100 & 4 & 5 & -1 & 10  \\
46 & 4 & 2 & -1 & 7 &101 & 4 & 5 &  1 & 10  \\
47 & 4 & 2 &  1 & 7 &102 & 3 & 6 & -1 & 10  \\
48 & 3 & 3 & -1 & 7 &103 & 3 & 6 &  1 & 10  \\
49 & 3 & 3 &  1 & 7 &104 & 2 & 7 & -1 & 10  \\
50 & 2 & 4 & -1 & 7 &105 & 2 & 7 &  1 & 10  \\
51 & 2 & 4 &  1 & 7 &106 & 1 & 8 & -1 & 10  \\
52 & 1 & 5 & -1 & 7 &107 & 1 & 8 &  1 & 10  \\
53 & 1 & 5 &  1 & 7 &108 & 0 & 9 & -1 & 10  \\
54 & 0 & 6 & -1 & 7 &109 & 0 & 9 &  1 & 10  \\
\hline
\end{tabular}
\caption{$L_x=0$,$\omega=1.0$, here we have defined $m_s' = 2m_s$, and $m_s=\pm1/2$}
\end{table}


\subsubsection{Standard Interaction for $L_x=2.5 \,,\, \omega=0.8$}

\begin{table}[H]
\small
\centering 
\begin{tabular}{cccrc}
\hline
$p$ & $n_x'$ & $n_y'$ & $m_s'$ & $E$\\
\hline
0  & 0 & 0 & -1 & 0.9977\\
1  & 0 & 0 &  1 & 0.9970\\
2  & 1 & 0 & -1 & 1.0025\\
3  & 1 & 0 &  1 & 1.0025\\
4  & 2 & 0 & -1 & 1.9662\\
5  & 2 & 0 &  1 & 1.9662\\ 
6  & 0 & 1 & -1 & 1.9970\\
7  & 0 & 1 &  1 & 1.9970\\
8  & 1 & 1 & -1 & 2.0025\\
9  & 1 & 1 &  1 & 2.0025\\
10 & 3 & 0 & -1 & 2.0245\\
11 & 3 & 0 &  1 & 2.0245\\
12 & 4 & 0 & -1 & 2.8722\\
13 & 4 & 0 &  1 & 2.8722\\ 
14 & 2 & 1 & -1 & 2.9662\\
15 & 2 & 1 &  1 & 2.9662\\
16 & 0 & 2 & -1 & 2.9970\\
17 & 0 & 2 &  1 & 2.9970\\
18 & 1 & 2 & -1 & 3.0025\\
19 & 1 & 2 &  1 & 3.0025\\
20 & 3 & 1 & -1 & 3.0245\\
21 & 3 & 1 &  1 & 3.0245\\
22 & 5 & 0 & -1 & 3.8722\\
23 & 5 & 0 &  1 & 3.8722\\
\hline
\end{tabular}
\caption{single-particle states for $L_x=2.5$}
\end{table}



\begin{table}[H]
\centering 
\begin{tabular}{cccl}
\toprule
\multicolumn{4}{c}{$nh=2,\,\,\omega=0.8,\,\,L_x=2.5$} \\
\midrule
\multicolumn{4}{c}{Standard Interaction} \\
$R$ & $nh+np$ & $HF$ & $CCSD$ \\
\midrule 
1 & 2	  & 2.8476	  &  2.8466   \\
2 & 4	  & 2.8466	  &  2.7185   \\
3 & 6	  & 2.8337	  &  2.6566   \\
4 & 8	  & 2.8337	  &  2.6608   \\
5 & 10	  & 2.8337	  &  2.6407   \\
6 & 12	  & 2.8337	  &  2.6376   \\ 
7 & 14	  & 2.8337	  &  2.6373   \\
8 & 16	  & 2.7477	  &  2.5870   \\
9 & 18	  & 2.6978        &  2.5199   \\
10 &20	  & 2.6978        &  not conv.  \\
11 &22	  & 2.6978        &  not conv.  \\
12 &24	  & 2.6978        &  not conv.  \\
\bottomrule
\end{tabular}
\caption{Ground-state energies for hole states $nh=2$ in a circular quantum dot using a standard coulomb interaction for $L_x=2.5$, $\omega=0.8$, where $nh+np$ are the number of holes plus particles in our system. The CCSD energy have converged very poorly, I had to reduce the tolerance to 1E-4 for it to converge, and i cased it didnt converge, I've extrapolated the minima.}
\end{table}

\begin{figure}[H]
\centering
\scalebox{0.7}{\includegraphics{nh=2_w=0_8_Lx=2_5_R=1-10.eps}}
\caption{Hartree Fock and Coupled Cluster energy for $w=0.8, \, L_x=2.5$}
\end{figure}

\begin{figure}[H]
\centering
\input{w=0_8_Lx=2_5_R=12_map}
\caption{Shellstructure for $L_x=2.5$}
\end{figure}

\begin{figure}[H]
\centering
\scalebox{0.5}{\includegraphics{Lx=2_5_w=0_8.eps}}
\caption{Potential for $V(x,0)$ with $w=1.0, \, V_0=10.66$}
\end{figure}



\subsubsection{Standard Interaction for $L_x=2.5 \,,\, \omega=1$}


\begin{table}[H]
\centering 
\begin{tabular}{cccl}
\toprule
\multicolumn{4}{c}{$nh=2,\,\,\omega=1.0,\,\,L_x=2.5$} \\
\midrule
\multicolumn{4}{c}{Standard Interaction} \\
$R$ & $nh+np$ & $HF$ & $CCSD$ \\
\midrule 
1 & 2	  & 3.3932	  &  3.3931   \\
2 & 4	  & 3.3931	  &  3.2508   \\
3 & 6	  & 3.3808	  &  3.1865*  \\
4 & 8	  & 3.3808	  &  3.1893*  \\
5 & 10	  & 3.3808	  &  3.1634*  \\
6 & 12	  & 3.3808	  &  3.1633*  \\ 
7 & 14	  & 3.3808	  &  3.1637*  \\
8 & 16	  & 3.2918	  &  3.1119*  \\
9 & 18	  & 3.2400        &  3.0716*  \\
10 &20	  & 3.2400        &  3.0715*  \\
11 &22	  & 3.2400        &  not conv.  \\
12 &24	  & 3.2400        &  not conv.  \\
\bottomrule
\end{tabular}
\caption{Ground-state energies for $nh=2$ in a circular quantum dot using a standard coulomb interaction for $L_x=2.5$, $\omega=1$. I have marked the CCSD energies with (*) where I have extrapolated due to poor convergence.}
\end{table}

\begin{figure}[H]
\centering
\scalebox{0.7}{\includegraphics{nh=2_w=1_Lx=2_5_R=2-12_std.eps}}
\caption{Hartree Fock and Coupled Cluster energy for $w=1.0, \, L_x=2.5$}
\end{figure}




\subsubsection{Standard Interaction for $V_0=8 \,,\, \omega=1$}

\begin{table}[H]
\small
\centering 
\begin{tabular}{cccrc}
\hline
$p$ & $n_x'$ & $n_y'$ & $m_s'$ & $E$\\
\hline
0  & 0 & 0 & -1 & 4.5510\\
1  & 0 & 0 &  1 & 4.5510\\
2  & 1 & 0 & -1 & 4.5536\\
3  & 1 & 0 &  1 & 4.5536\\
4  & 0 & 1 & -1 & 5.5510\\
5  & 0 & 1 &  1 & 5.5510\\ 
6  & 1 & 1 & -1 & 5.5536\\
7  & 1 & 1 &  1 & 5.5536\\
8  & 2 & 0 & -1 & 6.3731\\
9  & 2 & 0 &  1 & 6.3731\\
10 & 3 & 0 & -1 & 6.4135\\
11 & 3 & 0 &  1 & 6.4135\\
12 & 0 & 2 & -1 & 6.5510\\
13 & 0 & 2 &  1 & 6.5510\\ 
14 & 1 & 2 & -1 & 6.5536\\
15 & 1 & 2 &  1 & 6.5536\\
16 & 2 & 1 & -1 & 7.3731\\
17 & 2 & 1 &  1 & 7.3731\\
18 & 3 & 1 & -1 & 7.4135\\
19 & 3 & 1 &  1 & 7.4135\\
\hline
\end{tabular}
\caption{single-particle states for $V_0=8$}
\end{table}

\begin{figure}[H]
\centering
\input{V=8_map}
\caption{Shellstructure for $V_0=8$}
\end{figure}



\begin{table}[H]
\centering 
\begin{tabular}{cccl}
\toprule
\multicolumn{4}{c}{$nh=2,\,\,\omega=1.0,\,\,V_0=8$} \\
\midrule
\multicolumn{4}{c}{Standard Interaction} \\
$R$ & $nh+np$ & $HF$ & $CCSD$ \\
\midrule 
1 & 2	  & 10.2789 	  &  10.2789   \\
2 & 4	  & 10.2789	  &  10.1768   \\
3 & 6	  & 10.2789	  &  10.1601   \\
4 & 8	  & 10.2789	  &  10.1433   \\
5 & 10	  & 10.2783	  &  10.1232*  \\
6 & 12	  & 10.2783	  &  10.1192*  \\ 
7 & 14	  & 10.2342	  &  10.0985*  \\
8 & 16	  & 10.2342	  &  10.0829*  \\
9 & 18	  & 10.2342       &  10.0829*  \\
10 &20	  & 10.2342       &  10.0825*  \\
\bottomrule
\end{tabular}
\caption{Ground-state energies for $nh=2$ in a circular quantum dot using a standard coulomb interaction for $V_0=8$, $\omega=1$. I have marked the CCSD energies with (*) where I have extrapolated due to poor convergence.}
\end{table}

\begin{figure}[H]
\centering
\scalebox{0.7}{\includegraphics{nh=2_w=1_V=8_R=1-10_std.eps}}
\caption{Hartree Fock and Coupled Cluster energy for $w=1.0, \, V_0=8$}
\end{figure}



\subsubsection{Standard Interaction for $V_0=10.66 \,,\, \omega=1$}

\begin{table}[H]
\small
\centering 
\begin{tabular}{cccrc|cccrc}
\hline
$p$ & $n_x'$ & $n_y'$ & $m_s'$ & $E$ & $p$ & $n_x'$ & $n_y'$ & $m_s'$ & $E$\\
\hline
0  & 0 & 0 & -1 & 4.9123 & 20 & 0 & 3 &  1 & 7.9123   \\
1  & 0 & 0 &  1 & 4.9123 & 21 & 0 & 3 & -1 & 7.9123   \\
2  & 1 & 0 & -1 & 4.9127 & 22 & 1 & 3 &  1 & 7.9126   \\
3  & 1 & 0 &  1 & 4.9127 & 23 & 1 & 3 & -1 & 7.9126   \\
4  & 0 & 1 & -1 & 5.9123 & 24 & 4 & 0 &  1 & 8.7184   \\
5  & 0 & 1 &  1 & 5.9123 & 25 & 4 & 0 & -1 & 8.7184   \\ 
6  & 1 & 1 & -1 & 5.9127 & 26 & 5 & 0 &  1 & 8.7628   \\
7  & 1 & 1 &  1 & 5.9127 & 27 & 5 & 0 & -1 & 8.7628   \\
8  & 2 & 0 & -1 & 6.9018 & 28 & 2 & 2 &  1 & 8.9018   \\
9  & 2 & 0 &  1 & 6.9018 & 29 & 2 & 2 & -1 & 8.9018   \\
10 & 3 & 0 & -1 & 6.9071 & 30 & 3 & 2 &  1 & 8.9071   \\
11 & 3 & 0 &  1 & 6.9071 & 31 & 3 & 2 & -1 & 8.9071   \\
12 & 0 & 2 & -1 & 6.9123 & 32 & 0 & 4 &  1 & 8.9123   \\
13 & 0 & 2 &  1 & 6.9123 & 33 & 0 & 4 & -1 & 8.9123   \\ 
14 & 1 & 2 & -1 & 6.9127 & 34 & 1 & 4 &  1 & 8.9127   \\
15 & 1 & 2 &  1 & 6.9127 & 35 & 1 & 4 & -1 & 8.9127   \\
16 & 2 & 1 & -1 & 7.9018 & 36 & 4 & 1 &  1 & 9.7184   \\
17 & 2 & 1 &  1 & 7.9018 & 37 & 4 & 1 & -1 & 9.7184   \\
18 & 3 & 1 & -1 & 7.9071 & 38 & 5 & 1 &  1 & 9.7628   \\
19 & 3 & 1 &  1 & 7.9071 & 39 & 5 & 1 & -1 & 9.7628   \\
\hline
\end{tabular}
\caption{$L_x=0$,$\omega=1.0$, here we have defined $m_s' = 2m_s$, and $m_s=\pm1/2$}
\end{table}


\begin{figure}[H]
\centering
\input{V=10_66_map}
\caption{Shellstructure for $\omega=1,\,V_0=10.66$}
\end{figure}


\begin{table}[H]
\centering 
\begin{tabular}{cccl}
\toprule
\multicolumn{4}{c}{$nh=2,\,\,\omega=1.0,\,\,V_0=8$} \\
\midrule
\multicolumn{4}{c}{Standard Interaction} \\
$R$ & $nh+np$ & $HF$ & $CCSD$ \\
\midrule 
2 & 4	  & 10.9547 	  &  10.8737   \\
4 & 8	  & 10.9547	  &  10.8425   \\
6 & 12	  & 10.9544	  &  10.8239   \\
8 & 16	  & 10.9152	  &  10.7911*  \\
12 & 24	  & 10.9152	  &  10.7868*  \\
14 & 28	  & 10.8881	  &  10.7663*  \\ 
16 & 32	  & 10.8881       &  10.7661*  \\
18 & 36	  & 10.8878	  &  10.7650*  \\
20 & 40	  & 10.8878       &  10.7638*  \\
\bottomrule
\end{tabular}
\caption{Ground-state energies for $nh=2$ in a circular quantum dot using a standard coulomb interaction for $V_0=8$, $\omega=1$. I have marked the CCSD energies with (*) where I have extrapolated due to poor convergence.}
\end{table}

\begin{figure}[H]
\centering
\scalebox{0.7}{\includegraphics{nh=2_w=1_V=10_66_R=2-20_std.eps}}
\caption{Hartree Fock and Coupled Cluster energy for $w=1.0, \, V_0=8$}
\end{figure}

\subsubsection{Standard Interaction for $V_0=20 \,,\, \omega=1$}

\begin{figure}[H]
\centering
\scalebox{0.7}{\includegraphics{nh=2_w=1_V=20_R=1-16_std.eps}}
\caption{$w=1.0, \, V_0=20$}
\end{figure}


\begin{figure}[H]
\centering
\scalebox{0.7}{\includegraphics{V=8_vs_V=10_66_vs_V=20.eps}}
\caption{The potential for $w=1.0, \, V_0=8$ and $V_0 = 10.66$}
\end{figure}

\begin{figure}[H]
\centering
\scalebox{0.7}{\includegraphics{nh=2_w=1_V=0-0_5_R=10_std.eps}}
\caption{$w=1.0$,$V_0 = 0$ to $V_0 = 0.5$}
\end{figure}

\begin{figure}[H]
\centering
\scalebox{0.7}{\includegraphics{nh=2_w=1_V=8-20_R=12_std.eps}}
\caption{$w=1.0$,$V_0 = 8$ to $V_0 = 20$}
\end{figure}

\begin{figure}[H]
\centering
\scalebox{0.7}{\includegraphics{nh=2_w=0_25-1_V=15_R=12_std.eps}}
\caption{$w=1.0$,$V_0 = 0$ to $V_0 = 0.5$}
\end{figure}

\begin{figure}[H]
\centering
\scalebox{0.7}{\includegraphics{nh=4_vs_8_w=0_25-1_V=15_R=12_std.eps}}
\caption{$w=1.0$,$V_0 = 0$ to $V_0 = 0.5$}
\end{figure}

\begin{figure}[H]
\centering
\scalebox{0.7}{\includegraphics{nh=4_w=0_25-1_V=15_R=12_std.eps}}
\caption{$w=1.0$,$V_0 = 0$ to $V_0 = 0.5$}
\end{figure}




\chapter{Conclusion}



% 
% 
% \part{Appendix}
% \appendix
% \chapter{Derivation of the Baker-Campbell-Hausdorff formula }
% \label{dix:derivation of baker-campbell-hausdorff}
% 
% We will now derive the following relation:
% 
% \begin{align}
% e^{-\lambda \hh{A}}\hh{B}e^{-\lambda\hh{A}} &= \hh{B} + \lambda \ql[\hh{B}, \hh{A}  \qr] + \frac{\lambda^2}{2!} \ql[\ql[\hh{B}, \hh{A}  \qr],\hh{A} \qr] + \frac{\lambda^3}{3!} \ql[ \ql[\ql[\hh{B}, \hh{A}  \qr],\hh{A} \qr],\hh{A} \qr] +...
% \nonumber \\
% &= \sum_{n=0}^{\infty} \frac{1}{n!} (\text{ad}(\hh{B}))^n \hh{A}, \qquad (\text{ad}(\hh{B}))^n \hh{A} = \ql[ \ql[\ql[\hh{B}..., \hh{A}  \qr],\hh{A} \qr],\hh{A} \qr]\,\,\,(\text{n times})
% \label{app:show this relation} 
% \end{align}
% %
% We define 
% 
% \begin{equation}
% \hh{G}(\lambda) = e^{-\lambda \hh{A}}\hh{B}e^{\lambda \hh{A}}
%  \label{app:G(lambda)}
% \end{equation}
% %
% We want to do a Taylor series expansion with this function around $\lambda = 0$,
% 
% \begin{equation}
% \hh{G}(\lambda) = \hh{G}(0) + \ql .\frac{d\hh{G}}{d\lambda} \qr|_{\lambda = 0} (\lambda-0) + \frac{1}{2!} \ql .\frac{d^2\hh{G}}{d^2\lambda} \qr|_{\lambda = 0} (\lambda-0)^2 + \frac{1}{3!} \ql .\frac{d^3\hh{G}}{d^3\lambda} \qr|_{\lambda = 0} (\lambda-0)^3 + ...
%  \label{app:Taylor expansion}
% \end{equation}
% %
% {\bf{0th order}}:
% \begin{equation}
% \hh{G}(0) = \hh{B} 
%  \label{app:0th order}
% \end{equation}
% %
% {\bf{1th order}}:
% \begin{equation}
% \ql.\frac{d\hh{G}}{d\lambda} \qr|_{\lambda = 0} = \ql[ -\hh{A}e^{-\lambda \hh{A}}\hh{B}e^{\lambda \hh{A}} + e^{-\lambda \hh{A}}\hh{B}e^{\lambda \hh{A}}\hh{A} \qr]_{\lambda = 0} = \ql[\hh{G}(0),\hh{A}\qr] = \ql[\hh{B},\hh{A}\qr] 
%  \label{app:1th order}
% \end{equation}
% %
% {\bf{2nd order}}:
% \begin{equation}
% \frac{d^2\hh{G}}{d^2\lambda} = \frac{d\hh{G}}{d\lambda} \ql[\hh{G},\hh{A}\qr] = \ql[ \frac{d\hh{G}}{d\lambda},\hh{A}\qr] =  
% \ql[ \ql[ \hh{G},\hh{A}\qr] ,\hh{A}\qr] \label{app:2th order} 
% \end{equation}
% %
% \begin{equation}
% \Rightarrow \ql.\frac{d^2\hh{G}}{d^2\lambda}\qr|_{\lambda=0} = \ql[ \ql[ \hh{B},\hh{A}\qr] ,\hh{A}\qr] \label{defapp:2th order 2} 
% \end{equation}
% %
% {\bf{3rd order}}:
% \begin{equation}
% \frac{d^3\hh{G}}{d^3\lambda} = \frac{d\hh{G}}{d\lambda} \ql[\ql[\hh{G},\hh{A}\qr],\hh{A}\qr] = \ql[\ql[\frac{d\hh{G}}{d\lambda} ,\hh{A}\qr],\hh{A}\qr] =  
% \ql[ \ql[ \ql[ \hh{G},\hh{A}\qr],\hh{A}\qr] ,\hh{A}\qr] \label{app:3th order} 
% \end{equation}
% %
% \begin{equation}
% \Rightarrow \ql.\frac{d^3\hh{G}}{d^3\lambda}\qr|_{\lambda=0} = \ql[ \ql[\ql[ \hh{B},\hh{A}\qr],\hh{A}\qr] ,\hh{A}\qr] \label{defapp:3th order 2} 
% \end{equation}
% %
% {\bf{n-th order}}:
% \begin{equation}
% \frac{d^n\hh{G}}{d^n\lambda} = \frac{d\hh{G}}{d\lambda} \underbrace{\ql[ \ql[\ql[\hh{G}..., \hh{A}  \qr],\hh{A} \qr],\hh{A} \qr]}_{\text{$(n-1)$ nested commutators}} =  \underbrace{\ql[ \ql[\ql[\hh{G}..., \hh{A}  \qr],\hh{A} \qr],\hh{A} \qr]}_{\text{$n$ nested commutators}} \nonumber 
% \end{equation}
% 
% 
% 




\bibliographystyle{plain}% sorting of plainnat A = Author, C = Where in the text its sorted
\bibliography{minbib}


\end{document}

