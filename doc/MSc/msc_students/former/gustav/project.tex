\documentstyle[a4wide]{article}
\newcommand{\OP}[1]{{\bf\widehat{#1}}}

\newcommand{\be}{\begin{equation}}

\newcommand{\ee}{\end{equation}}
\newcommand{\bra}[1]{\left\langle #1 \right|}
\newcommand{\ket}[1]{\left| #1 \right\rangle}
\newcommand{\braket}[2]{\left\langle #1 \right| #2 \right\rangle}


\begin{document}

\pagestyle{plain}

\section*{Computational Physics thesis: Studies of Hypernuclei}

Hypernuclei are bound systems of neutrons, protons and one or more strange
baryons, such as the $\Lambda$ or $\Sigma$ hyperons. Understanding the
behavior of hypernuclei (how they are produced, their spectroscopy and
decay mechanisms) has been the subject of intense investigations
during the last decades.

One of the main goals of such studies is to explore how the presence
of the new degree of freedom (strangeness) alters and broadens the
knowledge achieved from
conventional nuclear physics. Several features of the
$\Lambda$ single-particle properties in the nucleus, being
essentially different from  those of the nucleon, have clearly
emerged from these efforts. It is well accepted nowadays that the
depth of the $\Lambda$-nucleus potential is around $-30$ MeV,
which is 20 MeV less attractive than the
corresponding nucleon-nucleus one. The spin-orbit splittings of single
particle levels in $\Lambda$ hypernuclei were found to be much smaller
than their nucleonic counterparts, typically more
than one order of magnitude . Moreover,
the $\Lambda$, contrary to the nucleon, maintains its single-particle
character even for states well below the Fermi surface indicating
a weaker interaction with other nucleons. Studies of the mesonic weak
decay of light $\Lambda$ hypernuclei
have shown that the data clearly favour $\Lambda$-nucleus potentials which
show a repulsion at short distances.

Attempts to derive the hyperon properties in a nucleus have followed
several approaches. A $\Lambda$-nucleus potential of Woods-Saxon type
reproduces reasonably well the measured $\Lambda$ single-particle
energies of medium to heavy hypernuclei.
Non localities and density dependent effects, included in
non-relativistic Hartree-Fock calculations using Skyrme
hyperon-nucleon ($YN$), improve the
overall fit to the single-particle binding energies.  The properties
of hypernuclei have also been studied in a relativistic framework,
such as Dirac phenomenology or
relativistic mean field theory.

Microscopic calculations, which aim at relating the hypernuclear
observables to the bare $YN$ interaction, are also available, but still 
of an approximative character. 
In studies of hypernuclei, many-body calculations are still in  a
very preliminar stage, in particular when it comes to studies
of weakly bound systems, of current experimental interest both at GSI in Darmstadt
and at KEK in Japan. To provide reliable ab initio calculations for weakly bound
hypernuclei is a great challenge and this thesis is placed in such a 
context.

The aims of this thesis are     

\begin{enumerate} 
\item Derive new effective interactions for hyperons interacting with 
nucleons in many single-particle orbits.This defines the space for 
a self-consistent Hartree-Fock calculations of hyperons and nucleons.
To achieve the candidate has to set up the interaction, based on existing
models, between hyperons and nucleons. 
The numerical methods used are based on diagonalization techniques and
similarity transformations. 

\item The previous step is limited to the treatment of single-particle wave functions
in an oscillator basis and thereby bound single-particle states.
The next step is to extend the developed code to treat systems where
some of the states are weakly bound or in the scattering continuum.
This development will use as starting point earlier developed codes
for the nucleon-nucleon interaction.
It represents the starting point for studies  of hypernuclei where 
the hyperons are weakly bound.  Both steps may lead to a publication
in for example the Physical Review C. 
\item The next step,and this is optional,   is to use the interactions developed in the previous 
steps and perform many-body calculations with our recently developed
Coupled-Cluster codes.  It will be the first ab initio calculation of weakly
bound hypernuclei and could be published in  Physical Review Letters.
Steps 1 and 2 contain already enough material for publication in top physics
journals, while step 3 paves the road for an eventual PhD thesis. 

The numerical methods for the coupled-cluster codes are 
based on the iterative solution of non-linear equations
which are parallelized.
\end{enumerate}



The activity will be based within the 
Computational Quantum Mechanics project at the Department of Physics and 
CMA. The main location is the Nuclear Physics group. 
The Computational Quantum Mechanics group consists  
presently of two full professors (Hjorth-Jensen and Osnes), one professor Emeritus (Engeland), 
one Adjunct Professor (Dean, Oak Ridge National Lab and CMA), one post-doc (Gaute Hagen, CMA) five PhD
students (CMA(3), Dept ot Physics (2)) and six Master students (Dept of Physics).  

\section*{Progress plan and milestones}
The thesis is expected to be finished towards the end  of the spring 
semester of 2008
\begin{itemize}
\item Fall 2007: Develop effective interactions for nucleons and hyperons
using exisiting models for the hyeron-nucleon interaction.
Include a basis suitable for weakly bound systems. (steps 1 and 2 above).
\item Spring 2003:  If possible, include the derived effective interactions 
in the coupled-cluster machinery and perform ab initio calculations
of light hypernuclei.
Writeup of thesis and final thesis exam
\end{itemize}

\end{document}





































