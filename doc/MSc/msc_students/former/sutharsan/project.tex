\documentstyle[a4wide]{article}

\begin{document}

\pagestyle{plain}

\section*{N\o ytrinoemisjon i n\o ytronstjerner}

Her f\o lger en prosjektbeskrivelse samt litteraturliste til
Sutharsan  Arumugam's master oppgave ved Fysisk Institutt UiO.


\subsection*{Introduksjon til n\o ytronstjernefysikk}







N\o ytronstjerner utviser en rik og komplisert struktur.
I de ytre lag best\aa r materien 
av kjerner i likevekt med en elektrongass, 
kjerner i en tilstand slik en kan finne her p\aa\ jorda, 
dvs.\ atomkjernene har proton og n\o ytron tall som ikke
avviker vesentlig fra hverandre. 
Etterhvert som en g\aa r innover i stjerna, og dermed
ogs\aa\ \o ker tettheten til materien, blir kjernene mere
og mere n\o ytronrike pga.\ bla.\ $\beta$-henfall.
\O kes tettheten enda mere  
vil kjernene eksistere sammen med en n\o ytronv\ae ske for til
slutt \aa\ oppl\o ses helt. Ved enda h\o yere tettheter har
en muligheten for forekomsten av mere eksotiske former
for materie, slik som pion-kondensat, kaon-kondensat,
mere massive baryoner slik som $\Lambda$, $\Sigma$ eller
en fase av rein kvarkmaterie.

Viktige emner i slike n\o ytronstjernestudier blir da
\aa\ beskrive teoretisk egenskapene
til slik materie over tetthetsomr\aa der som kan variere
med flere st\o rrelsesordener. I tillegg kommer studier
av stjernenes dynamiske egenskaper.

For \aa\ studere egenskapene til en n\o ytronstjerne
slik som f.eks.\ total masse, radius eller treghetsmoment,
trenger en tilstandslikningen for tett materie.
Med tett materie meiner vi materie som er opptil 
10 ganger tettere enn materien i atomkjerner 
slik vi kjenner dem fra eksperimenter her p\aa\ jorda.
For \aa\ nevne et eksempel, s\aa\ er tettheten
innerst inne i blyatomet m\aa lt til ca.\
$0.16$ fm$^{-3}$ hvor fm er femtometer, dvs.\ $10^{-15}$
meter. Dette svarer til en massetetthet p\aa\ ca.\
$2.8\times 10^{14}$ gcm$^{-3}$. Ved tettheter opp til
2-3 ganger $0.16$ fm$^{-3}$, kan en anta at materien
best\aa r hovedsakelig av n\o ytroner i 
$\beta$-likevekt med protoner og elektroner. 
Siden forholdet mellom antall n\o ytroner og protoner
er p\aa\ st\o rrelsesorden med 10, er det i all hovedsak
trykket som blir satt opp av denne n\o ytrongassen
som motvirker gravitasjonskreftene (derav ogs\aa\ navnet
n\o ytronstjerner). Elektronene har en masse som er ca.\
1000 ganger mindre enn n\o ytron- og protonmassen,
og bidrar dermed lite til totaltrykket.
Totaltrykket gir tilstandslikningen for n\o ytronstjerne-materie.
Tar en ogs\aa\ med at man ved
enda st\o rre tettheter har mulighet for at 
materien vil underg\aa\ en faseovergang til rein
kvarkmaterie, s\aa\ kompliseres bildet dramatisk.

I mastergradsprosjektet skal vi ta som gitt (fra mange-legeme
beregninger og modeller for det indre av ei stjerne) ulike
tilstandslikninger og sammensetninger av partikler. 

Med gitte partikkelkonsentrasjoner
for ulike tettheter i det indre av stjernen, kan en beregne
emisjonsraten for n\o ytrinoer. Disse prosessene er beskrevet nedenfor.




\subsection*{Nedkj\o ling av n\o ytronstjerner og emisjon av n\o ytrinoer}

I en n\o ytronstjerne har alle fusjonsprosesser som 
forsyner en vanlig stjerne med energi opph\o rt, og 
den vil derfor kj\o les raskt ned.  
Med utgangspunkt i satellitt-observasjoner er det mulig 
\aa\ bestemme overflatetemperaturen for enkelte n\o ytronstjerner.  
Hvis man samtidig kan bestemme alderen, som ogs\aa \  er mulig 
i enkelte tilfeller, har man funnet et punkt p\aa \ stjernens 
nedkj\o lingskurve.  Dette kan gi viktig informasjon om 
stjernens indre.  
I store deler av n\o ytronstjernens liv er nemlig reaksjoner 
der n\o ytrinoer frigis det viktigste bidraget til nedkj\o lingen.  
Hva slags typer reaksjoner som foreg\aa r og hvor raskt de 
forl\o per avhenger blant annet av hvordan materien i stjernen er 
sammensatt og om man har superflytende faser til stede.  
For eksempel vil en stjerne med en indre kjerne av ren kvarkmaterie 
kj\o les hurtigere ned enn en stjerne som utelukkende best\aa r 
av n\o ytroner, protoner og elektroner.   
Samspillet mellom observasjoner og teoretiske beregninger av 
nedkj\o lingskurver kan kanskje fortelle oss om man f\aa r 
dannet en kvarkfase i det indre av en n\o ytronstjerne.  


Den termiske utviklinga av ei n\o ytronstjerne kan dermed
gi oss en viss
informasjon om materiens sammensetting i det indre av 
stjerna. I de siste \aa r har det v\ae rt stor aktivitet
for \aa\ m\aa le overflatetemperaturer, spesielt via
Einstein-observatoriet og ROSAT.

Den viktigste kj\o lningsmekanismen, dvs.\ hvordan energi
avgis fra en n\o ytronstjerne, i tidlige stadier av stjernas liv  
antas \aa\ foreg\aa\ via n\o ytrinoemisjon fra det indre av
stjerna.
Den viktigste mekanismen for energitap er den s\aa kalte
direkte URCA prosessen (navnet er fra fysikeren Gamow,
og har sin bakgrunn i navnet p\aa\ et casino i Rio de Janeiro,
et veritabelt pengesluk for den uinnvidde.)
\begin{equation}
    n\rightarrow p +e +\overline{\nu}_e, \hspace{1cm} p+e \rightarrow
    n+\nu_e .
    \label{eq:directU}
\end{equation}
Dette er ikke annet enn reaksjonslikningene for $\beta$-henfall og
elektroninnfangning.
Men i  de ytre lag av en stjerne, 
er ikke disse prosessene tillatt fordi tettheten er for lav 
til at bevegelsesmengden kan v\ae re bevart i prosessen.  

I lang tid har man derfor ment at en av de dominerende prosessen for 
n\o ytrinoemisjon er klassen av s\aa kalte
modifiserte URCA prosesser, med f.eks.\ f\o lgende prosesser
\begin{equation}
    n+n\rightarrow p+n +e +\overline{\nu}_e,
    \hspace{0.5cm} p+n+e \rightarrow
    n+n+\nu_e .
    \label{eq:ind_neutr}
\end{equation}
Dette svarer til prosessene for  
$\beta$-henfall og elektron innfangning  med  protoner 
fra likning (\ref{eq:directU}) men med den forskjell at vi har 
et ekstra n\o ytron for \aa\ \o ke faserommet.
Jon Thonstad jobber med denne type prosesser i sin masteroppgave
og blir ferdig H-2006.
I tillegg finnes flere andre n\o ytrino emitterende 
prosesser, slik som f.eks.\
bremsestr\aa lingsreaksjoner.

Denne siste type reknes ogs\aa\ \aa\ v\ae re like viktig som prosessene
i likning (\ref{eq:ind_neutr}) og sammen danner disse prosessene 
det som kalles 'The standard
cooling scenario for a neutron star'. 
Prosessen er da gitt ved
\begin{equation}
    B_1+B_2\rightarrow B_1+B_2 +\nu_i +\overline{\nu}_i,
    \label{eq:vv}
\end{equation}
hvor $B_{1,2}$ kan v\ae re n\o ytroner eller protoner og $\nu_i$
er enten elektron eller myon n\o ytrinoer. 


\subsection*{M\aa l}
I prosjektet skal vi ta for oss ei stjerne som best\aa r kun an n\o ytroner
og protoner samt elektroner og myoner
og fokusere i f\o rste omgang p\aa\ modifiserte URCA
prosesser som innvolverer n\o ytroner og n\o ytrino annihilasjonsprosesser.

Prosjektet her tar sikte p\aa\ rekne ut analytiske uttrykk for likning
(\ref{eq:vv}) ved f\o rst \aa\ sette opp Feynman matrise og deretter
rekne ut emissiviteten av n\o ytrinoer. Til det siste vil det m\aa tte skrives et numerisk program.
Vekselvikrningen mellom nukleoner tas fra en parametrisert form
fra en Dirac-Brueckner-Hartree-Fock berekning av Malfliet og Boersma
\cite{malfliet}. 
Litteratur om likning (\ref{eq:vv}) finnes bla i Ref.~\cite{armen1,armen2}.

Oppgaven vil lett kunne utvides til en PhD. Dette er et sv\ae rt aktuelt tema 
i n\o ytronstjerne fysikk og en skikkelig evaluering av disse prosessene 
basert p\aa\ mer realistiske effektive vekselvirkninger i kjernematerie
har ikke blitt gjort siden Friman og Maxwell sitt klassiske arbeid fra
1979, se Ref.~\cite{fm79}.
I tillegg, vil Sutharsan  Arumugam jobbe sammen med Jon Thonstad, som studerer
prosessene i likning (\ref{eq:ind_neutr}).




\section*{Framdriftsplan og delm\aa l}
M\aa lsettinga er \aa\ bli ferdig V-2007.
\begin{itemize}
\item Ut H\o st 2006 : Sette opp alle uttrykk, Feynman diagram
og uttrykk for emissiviteten for prosessen gitt i likning 
(\ref{eq:ind_neutr}). Start \aa\ bygge et program som kan sette opp
integralet med gitt effektiv vekselvirkning.  
\item V\aa r 2007, Ferdigstillelse av kode, produksjon av resultat samt
skriving av oppgave og avslutning.

\end{itemize}
%\section*{Litteraturliste}
\begin{thebibliography}{99}
\bibitem{st83} S.L.\ Shapiro and S.A.\ Teukolsky,
               Black Holes, White Dwarfs and Neutron Stars,
               (Wiley, New York, 1983), Kap 2-4, 5, 8-11, appendiks F

\bibitem{pplp92} M.\ Prakash, M.\ Prakash, J.M.\ Lattimer and C.J.\
Pethick, Astrophys.\ J.\  390 (1992) L77.

\bibitem{pethick92} C.J.\ Pethick, Rev.\ Mod.\ Phys.\ 64 (1992) 1133.

\bibitem{prakash94} M.\ Prakash, Phys.\ Rep.\ 242 (1994) 191.


\bibitem{iwamoto82} N.\ Iwamoto,\ Ann.\ Phys.\  141 (1982) 1.

\bibitem{cs64} H.-Y.\ Chiu and E.E.\ Salpeter, Phys.\ Rev.\ Lett.\
12 (1964) 413.
\bibitem{it72} I.\ Itoh and T.\ Tsuneto, Prog.\ Theor.\ Phys.\ 48 (1972) 149.

\bibitem{fm79} B.L.\ Friman and O.V.\ Maxwell, 
               Astrophys.\ J.\ 232 (1979) 541.


\bibitem{glendenning91} N.K.\ Glendenning, Compact Stars, (Springer, Berlin, 1997).
\bibitem{pbpelk97} M.\ Prakash, I.\ Bombaci, M.\ Prakash, P.J.\ Ellis, J.M.\ 
Lattimer and R.\
Knorren, Phys.\ Rep.\ 280 (1997) 1. 
\bibitem{tsuruta98} S.\ Tsuruta, Phys.\ Rep.\ 292 (1998) 1.

\bibitem{pal97} L.B.\ Leinson and A.\ P\'erez, JHEP 9 (1998) 20;
S.\ Chakrabarty, D.\ Bandyopadahay and S.\ Pal, Phys.\ Rev.\ Lett.\
78 (1997) 2898; ibid.\ 79 (1997) 2176.
\bibitem{HH99} H.\ Heiselberg and M.\ Hjorth-Jensen, 
                      Phys. Rep. 328 (2000) 237.
\bibitem{page94} D.\ Page, Astrophys.\ J.\ 428 (1994) 250.
\bibitem{schaab97} Ch.\ Schaab, D.\ Voskresensky, A.D.\ Sedrakian,
                   F.\ Weber and M.K.\ Weigel, Astron.\ Astrophys.\
                   321 (1997) 591; Ch.\ Schaab, 
                   F.\ Weber, M.K.\ Weigel and N.K.\ Glendenning, 
                   Nucl.\ Phys.\ A 605 (1996) 531.
\bibitem{schaab98} Ch.\ Schaab, F.\ Weber and M.K.\ Weigel, 
                   Astron.\ Astrophys.\ 335 (1998) 596.
\bibitem{malfliet} H.F.\ Boersma and R.~Malfliet, Phys.~Rev.~C 49 (1994) 633.

\bibitem{armen1} Armen Sedrakian and Alex Dieperink, 
Phys.~Lett.~B463 (1999) 145.
\bibitem{armen2}   Armen Sedrakian, preprint nucl-th/0601086, 
http://arxiv.org/abs/nucl-th/0601086.
\end{thebibliography}

\end{document}












