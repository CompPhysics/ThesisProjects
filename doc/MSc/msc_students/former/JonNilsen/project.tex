\documentstyle[a4wide,11pt]{article}

\begin{document}

\pagestyle{plain}

\begin{center} \huge \bf Cand. Scient. project for Jon Nilsen \end{center}

\section*{Bose-Einstein condensation in trapped bosons: A quantum Monte Carlo analysis}

The spectacular demonstration of Bose-Einstein condensation (BEC) in gases of
alkali atoms $^{87}$Rb, $^{23}$Na, $^7$Li confined in magnetic
traps\cite{anderson95,davis95,bradley95} has led to an explosion of interest in
confined Bose systems. Of interest is the fraction of condensed atoms, the
nature of the condensate, the excitations above the condensate, the atomic
density in the trap as a function of Temperature and the critical temperature of BEC,
$T_c$. The extensive progress made up to early 1999 is reviewed by Dalfovo et
al.\cite{dalfovo99}.

A key feature of the trapped alkali and atomic hydrogen systems is that they are
dilute. The characteristic dimensions of a typical trap for $^{87}$Rb is
$a_{h0}=\left( {\hbar}/{m\omega_\perp}\right)^\frac{1}{2}=1-2 \times 10^4$
\AA\ (Ref. 1). The interaction between $^{87}$Rb atoms can be well represented
by its s-wave scattering length, $a_{Rb}$. This scattering length lies in the
range $85 < a_{Rb} < 140 a_0$ where $a_0 = 0.5292$ \AA\ is the Bohr radius.
The definite value $a_{Rb} = 100 a_0$ is usually selected and
for calculations the definite ratio of atom size to trap size 
$a_{Rb}/a_{h0} = 4.33 \times 10^{-3}$ 
is usually chosen \cite{dalfovo99}. A typical $^{87}$Rb atom
density in the trap is $n \simeq 10^{12}- 10^{14}$ atoms/cm$^3$ giving an
inter-atom spacing $\ell \simeq 10^4$ \AA. Thus the effective atom size is small
compared to both the trap size and the inter-atom spacing, the condition
for diluteness (i.e., $na^3_{Rb} \simeq 10^{-6}$ where $n = N/V$ is the number
density). In this limit,
although the interaction is important, dilute gas approximations such as the
Bogoliubov theory\cite{bogoliubov47}, valid for small $na^3$ and large
condensate fraction $n_0 = N_0/N$, describe the system well. Also, since most
of the atoms are in the condensate (except near $T_c$), the Gross-Pitaevskii
equation\cite{gross61,pitaevskii61} for the condensate describes the whole gas
well. Effects of atoms excited above the condensate have been incorporated
within the Popov approximation\cite{hutchinson97}. 


The purpose of this Cand.~Scient thesis
is to go beyond the dilute limit, to test the limits of the above
approximations and to explore the properties of the trapped Bose gas as $na^3$
increases between the dilute limit and the dense limit.  
The tools to be used are from Variational Monte
Carlo (VMC) and Diffusion Monte Carlo methods (DMC) methods. 

The aim is to use these methods and evaluate 
the ground state properties of
a trapped, hard sphere Bose gas over a wide range of densities
using VMC and DMC  methods with several 
trial wave functions. These wave functions are used 
to study the sensitivity of condensate and 
non-condensate properties to the hard sphere radius and the number 
of particles.
The traps we will use are both a spherical symmetric (S) harmonic and 
an elliptical (E) harmonic trap in three dimensions given by 
 \begin{equation}
V_{ext}({\bf r}) = 
\Bigg\{
\begin{array}{ll}
        \frac{1}{2}m\omega_{ho}^2r^2 & (S)\\
\strut
	\frac{1}{2}m[\omega_{ho}^2(x^2+y^2) + \omega_z^2z^2] & (E)
\label{trap_eqn}
\end{array}
\end{equation}
with 
\begin{equation}
    H = \sum_i^N \left(
        \frac{-\hbar^2}{2m}
        { \bigtriangledown }_{i}^2 +
        V_{ext}({\bf{r}}_i)\right)  +
        \sum_{i<j}^{N} V_{int}({\bf{r}}_i,{\bf{r}}_j),
\end{equation}
as the two-body Hamiltonian of the system.
Here $\omega_{ho}^2$ defines the trap potential strength.  In the case of the
elliptical trap, $V_{ext}(x,y,z)$, $\omega_{ho}=\omega_{\perp}$ is the trap frequency
in the perpendicular or $xy$ plane and $\omega_z$ the frequency in the $z$
direction.
The mean square vibrational amplitude of a single boson at $T=0K$ in the 
trap (\ref{trap_eqn}) is $<x^2>=(\hbar/2m\omega_{ho})$ so that 
$a_{ho} \equiv (\hbar/m\omega_{ho})^{\frac{1}{2}}$ defines the 
characteristic length
of the trap.  The ratio of the frequencies is denoted 
$\lambda=\omega_z/\omega_{\perp}$ leading to a ratio of the
trap lengths
$(a_{\perp}/a_z)=(\omega_z/\omega_{\perp})^{\frac{1}{2}} = \sqrt{\lambda}$.

We represent the inter boson interaction by a pairwise, hard core potential
\begin{equation}
V_{int}(r) =  \Bigg\{
\begin{array}{ll}
        \infty & {r} \leq {a}\\
        0 & {r} > {a}
\end{array}
\end{equation}
where ${a}$ is the hard core diameter of the bosons.  Clearly, $V_{int}(r)$
is zero if the bosons are separated by a distance $r$ greater than $a$ but
infinite if they attempt to come within a distance $r \leq a$.


\begin{thebibliography}{999}%\parskip=0pt\itemsep 0pt%
\footnotesize
\bibitem {anderson95}%1
M.H. Anderson, J.R. Ensher, M.R. Matthews, C.E. Wieman, and E.A. Cornell, {\em Science} {\bf
269}, 198 (1995).

\bibitem{davis95}%2
K.B. Davis, M.-O. Mewes, M.R. Andrews, N.J. van Druten, D.S. Durfee, D.M. Kurn, and W. Ketterle,
{\em Phys. Rev. Lett.} {\bf 75}, 3969 (1995).

\bibitem{bradley95}%3
C.C. Bradley, C.A. Sackett, J.J. Tolett, and R.G. Hulet, {\em Phys. Rev. Lett.} {\bf 75}, 1687
(1995); C.C. Bradley, C.A. Sackett, and R.G. Hulet, {\em ibid.} {\bf 78}, 985 (1997).

\bibitem{dalfovo99}%4
F. Dalfovo, S. Giorgini, L. Pitaevskii, and S. Stringari, {\em Rev. Mod. Phys.} {\bf 71}, 463
(1999).

\bibitem{bogoliubov47}%5
N.N. Bogoliubov, {\em J. Phys. (Moscow)} {\bf 11}, 23 (1947).

\bibitem{gross61}%6
E.P. Gross, {\em Nuovo Cimento} {\bf 20}, 454 (1961).

\bibitem{pitaevskii61}%7
L.P. Pitaevskii, {\em Ah. Eksp. Teor. Fiz.} {\bf 40}, 646 (1961) [{\em Sov. Phys. JETP} {\bf 13},
451 (1961)].

\bibitem{hutchinson97}%8
D.A.W. Hutchinson, E. Zaremba, and A. Griffin, {\em Phys. Rev. Lett.} {\bf 78}, 1842 (1997).
\end{thebibliography}


\end{document}












