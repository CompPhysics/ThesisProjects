\documentclass[xcolor=pdftex,hyperref={pdfpagelabels=false},table]{beamer}
%%\documentclass[handout,xcolor=pdftex,dvipsnames,table,hyperref={pdfpagelabels=false},notes,handout]{beamer} %%% for printed version of the slides

%\usepackage{beamerthemeBerkeley}
% Use either the one above or the one below

%\usetheme{Berkeley}
\usepackage{beamerthemeshadow}
%\usepackage{beamerthemesplit}

\setbeamertemplate{caption}[numbered]
\setbeamerfont{caption}{size=\tiny}

\usepackage{booktabs} %%%% for using toprule in tables
\usepackage{amsmath, amssymb,amsfonts}
\usepackage{etex}
\usepackage{colortbl}
\usepackage{multirow}
\usepackage{graphicx}
\usepackage{color}
\usepackage{array}
\usepackage{../mystyles}    %%%% my commands



\usepackage{tikz}
\usepackage{pgfplots}
\usetikzlibrary{shapes,arrows,calc,plotmarks}
%%%% Define block styles for Flowchart in TIKZ
\tikzstyle{decision} = [diamond, draw, fill=blue!20, text width=4.5em, text badly centered, node distance=2.5cm, inner sep=0pt]
\tikzstyle{block} = [rectangle, draw, fill=blue!20, text width=8em, text centered, rounded corners, minimum height=4em]
\tikzstyle{longblock} = [rectangle, draw, fill=blue!20, text width=13em, text centered, rounded corners, minimum height=3em]
\tikzstyle{line} = [draw, very thick, color=black!50, -latex']
\tikzstyle{cloud} = [draw, ellipse,fill=red!20, node distance=2.5cm, minimum height=2em]
\def\radius{.7mm}
\tikzstyle{branch}=[fill,shape=circle,minimum size=3pt,inner sep=0pt]

% For every picture that defines or uses external nodes, you'll have to
% apply the 'remember picture' style. To avoid some typing, we'll apply
% the style to all pictures.
\tikzstyle{every picture}+=[remember picture]
% By default all math in TikZ nodes are set in inline mode. Change this to
% displaystyle so that we don't get small fractions.
\everymath{\displaystyle}

%%% TIKZ-LUA PACKAGE REQUIRED TO PLOT RESULTS !!!!!
\usepackage{gnuplot-lua-tikz} %% PACKAGE for calling TIKZ code generating by an external script with Gnuplot, see http://www.texample.net/tikz/examples/gnuplot-tikz-terminal/

\usepackage{shadow}
\usepackage{bezier}



%%%% Color wavelengths
\newcount\WL \unitlength.75pt

\title{Many-Body Appproaches to Quantum Dots}
\author[MSc. in Computational Physics]{Patrick Merlot}
\institute[University of Oslo]{Department of Computational Physics}
\date[UiO]{\today}
\subject{Many-Body Appproaches to Quantum Dots}


\begin{document}
%%%%%%%%%%%%%%%%%%%%%%%%%%%%%%%%%%%%%%%%%%%%%%%%%%%%%%%%%%%%%%%%%%%%%%%%%%%%%%%%%%%%%%%%%%%%%%
\frame{\titlepage}

%%%%%%%%%%%%%%%%%%%%%%%%%%%%%%%%%%%%%%%%%%%%%%%%%%%%%%%%%%%%%%%%%%%%%%%%%%%%%%%%%%%%%%%%%%%%%%
\section[Outline]{}
\frame{\tableofcontents}

%%%%%%%%%%%%%%%%%%%%%%%%%%%%%%%%%%%%%%%%%%%%%%%%%%%%%%%%%%%%%%%%%%%%%%%%%%%%%%%%%%%%%%%%%%%%%%
% % % % \frame{
% % % % \frametitle{Computational Study}
% % % % \setbeamercovered{invisible}
% % % % %%%CaptionFIG 1:Quantum dot defined by 5 metallic gates fabricated on the surface of a GaAs based heterostructure, in which a two-dimensional electron gas recides.
% % % % %%%CaptionFIG 2: Micrograph of pyramid-shaped quantum dots grown from indium, gallium, and arsenic. Each dot is about 20 nanometers wide and 8 nanometers in height. 
% % % % \begin{table}
% % % % \centering
% % % % \scalebox{0.9}{
% % % % {\footnotesize
% % % %   \begin{tabular}{m{3cm}m{5cm}m{3cm}}
% % % %   \multirow{2}{3cm}{Real system} &  \includegraphics[width=0.2\textheight]{IMAGES/QDcavity}   \tiny{Courtesy of C. Sch\"onenberger, Univ. of Basel} & \\ 
% % % % 			&  \includegraphics[width=0.2\textheight]{IMAGES/PyramideQD_GaAs}  \tiny{Courtesy of F. McGehan, NIST} & \\ %%%%%%%&  \\ \hline%%& \onslide<8->{X} & \onslide<2->{X} \\ \hline
% % % % 
% % % %  \multirow{2}{3cm}{Model}&  & \\ 
% % % % 	& \multirow{-2}{5cm}{$\Rightarrow \quad$ Schr\"odinger equation  $\hat{H} |\Psi \rangle= E |\Psi \rangle$} & \multirow{-2}{*}{with $\hat{H}=\hat{T}+\hat{V}$} \\ 
% % % % \multirow{1}{3cm}{Computational technique} &  &\\
% % % % 			& \multirow{-1}{5cm}{$\Rightarrow \quad$ Approximations/Algorithms} & \scalebox{0.5}{\input{IMAGES/diagramHF.tex}}  \\ 
% % % % \multirow{2}{3cm}{Implementation into computer language} &   &\\ 
% % % % 			& \multirow{-2}{5cm}{$\Rightarrow \quad$ \citecodeBeamer{C++}, \citecodeBeamer{FORTRAN}} &
% % % %   \end{tabular}
% % % % 
% % % % } %%% end small font
% % % % } %% end scalebox
% % % % \end{table} 
% % % % } %% end Frame
% % % % 


\section{Quantum Dots}
%%%%%%%%%%%%%%%%%%%%%%%%%%%%%%%%%%%%%%%%%%%%%%%%%%%%%%%%%%%%%%%%%%%%%%%%%%%%%%%%%%%%%%%%%%%%%%
\subsection{What is a quantum dot?}
\begin{frame}
\frametitle{What is a quantum dot?}
%%%%\begin{center}
\begin{columns}[T,c]
\column{5cm}
\scriptsize
\begin{block}{Definition}
\begin{itemize}
\item Semiconductor whose charge-carriers are confined in space.
\item $\neq$ types/shapes/fabrication \newline $\Rightarrow$ $\neq$ QDs
\item Applications (Qubits for QCA, nanobiological sensors)
\end{itemize}
  \end{block}

\begin{block}{Quantum dot properties}
\begin{itemize}
\item Semiconductor band gap increased by size quantization
\item Tunable optical/electrical properties
\item Perfect system for computational studies
\end{itemize}
  \end{block}
\column{6cm}

\begin{figure}
	\begin{center}
\scalebox{0.3}{
	\begin{tabular}{rl}
		\huge{(a)} & \scalebox{1.2}{\includegraphics{IMAGES/QDshapes/verticaldots}} \\ 
		\huge{(b)} & \scalebox{3.6}{\includegraphics{IMAGES/QDshapes/evifluorModified}} \\ 
		\huge{(c)} & \scalebox{0.3}{\includegraphics{IMAGES/QDcavity}} \\ 
		%%%\huge{(c)} & \scalebox{0.3}{\includegraphics{IMAGES/QDshapes/cap_178486}}
	\end{tabular}
}%%end Scalebox
	\vspace{-5pt}
	\caption{Possible types/shapes of QDs.
	\newline (a) Various shapes of QDs pillars$\sim 0.5 \mu m$,
	\newline (b) Colloidal QD (InGaP+ZnS+lipid)$\sim 10 nm$,
	\newline (c) QD defined by 5 metallic gates on GaAs where 2-DEG is trapped.
	%%%\newline (c) InAs islands grown on GaAs.
	}%%end caption
	\vspace{-5pt}
	\end{center}
\end{figure}
\end{columns}
\end{frame}


%%%%%%%%%%%%%%%%%%%%%%%%%%%%%%%%%%%%%%%%%%%%%%%%%%%%%%%%%%%%%%%%%%%%%%%%%%%%%%%%%%%%%%%%%%%%%%
\subsection{Applications of QDs}
\begin{frame}
\frametitle{Applications of QDs}
%%%%\begin{center}
\begin{columns}[T,c]
\column{5cm}
\scriptsize
\begin{block}{Definition}
\begin{itemize}
\item Semiconductor whose charge-carriers are confined in space.
\item $\neq$ types/shapes/fabrication \newline $\Rightarrow$ $\neq$ QDs
\item Applications (Qubits for QCA, nanobiological sensors)
\end{itemize}
  \end{block}

\begin{block}{Quantum dot properties}
\begin{itemize}
\item Semiconductor band gap increased by size quantization
\item Tunable optical/electrical properties
\item Perfect system for computational studies
\end{itemize}
  \end{block}
\column{6cm}
\TwofigTopDownScaled{0.17}{../IMAGES/InVivoTargetting01}{../IMAGES/InVivoTargetting02}{QDs imaging in live animals compared to classical organic dyes.\newline (Courtesy of X.~Gao)}{}
\end{columns}
\end{frame}

%%%%%%%%%%%%%%%%%%%%%%%%%%%%%%%%%%%%%%%%%%%%%%%%%%%%%%%%%%%%%%%%%%%%%%%%%%%%%%%%%%%%%%%%%%%%%%
\subsection{Physics of QDs}
\begin{frame}
\frametitle{Physics of QDs}
%%%%\begin{center}
\begin{columns}[T,c]
\column{5cm}
\scriptsize
\begin{block}{Definition}
\begin{itemize}
\item Semiconductor whose charge-carriers are confined in space.
\item $\neq$ types/shapes/fabrication \newline $\Rightarrow$ $\neq$ QDs
\item Applications (Qubits for QCA, nanobiological sensors)
\end{itemize}
  \end{block}

\begin{block}{Quantum dot properties}
\begin{itemize}
\item Semiconductor band gap increased by size quantization
\item Tunable optical/electrical properties
\item Perfect system for computational studies
\end{itemize}
  \end{block}
\column{6cm}
\begin{figure}
	\begin{center}
\scalebox{0.3}{
	\begin{tabular}{l}
		\scalebox{0.78}{\includegraphics{../IMAGES/Isolator-metal}} \\ 
		\scalebox{0.78}{\includegraphics{../IMAGES/CdS_winter}} \\ 
		%%%\huge{(c)} & \scalebox{0.3}{\includegraphics{IMAGES/QDshapes/cap_178486}}
	\end{tabular}
}%%end Scalebox
	\vspace{-5pt}
	\caption{Electronic band structure of semiconductor and quantum dots (Courtesy of J.Winter\cite{Winter2004}).}%%end caption
	\vspace{-5pt}
	\end{center}
\end{figure}
\end{columns}
\end{frame}

%%%%%%%%%%%%%%%%%%%%%%%%%%%%%%%%%%%%%%%%%%%%%%%%%%%%%%%%%%%%%%%%%%%%%%%%%%%%%%%%%%%%%%%%%%%%%%
\begin{frame}
\frametitle{Physics of QDs}
%%%%\begin{center}
\begin{columns}[T,c]
\column{5cm}
\scriptsize
\begin{block}{Definition}
\begin{itemize}
\item Semiconductor whose charge-carriers are confined in space.
\item $\neq$ types/shapes/fabrication \newline $\Rightarrow$ $\neq$ QDs
\item Applications (Qubits for QCA, nanobiological sensors)
\end{itemize}
  \end{block}

\begin{block}{Quantum dot properties}
\begin{itemize}
\item Semiconductor band gap increased by size quantization
\item Tunable optical/electrical properties
\item Perfect system for computational studies
\end{itemize}
  \end{block}
\column{6cm}
\begin{figure}
	\begin{center}
\scalebox{0.3}{
	\begin{tabular}{l}
		\scalebox{0.78}{\includegraphics{../IMAGES/fluo02}} \\ 
		\scalebox{0.65}{\includegraphics{../IMAGES/QDwavelength}} \\ 
		%%%\huge{(c)} & \scalebox{0.3}{\includegraphics{IMAGES/QDshapes/cap_178486}}
	\end{tabular}
}%%end Scalebox
	\vspace{-5pt}
	\caption{Fluorescent emission (Courtesy of J.Winter\cite{Winter2004}) and emission spectra of various QDs.}%%end caption
	\vspace{-5pt}
	\end{center}
\end{figure}
\end{columns}
\end{frame}



%%%%%%%%%%%%%%%%%%%%%%%%%%%%%%%%%%%%%%%%%%%%%%%%%%%%%%%%%%%%%%%%%%%%%%%%%%%%%%%%%%%%%%%%%%%%%%
%%%%%%%%%%%%%%%%%%%%%%%%%%%%%%%%%%%%%%%%%%%%%%%%%%%%%%%%%%%%%%%%%%%%%%%%%%%%%%%%%%%%%%%%%%%%%%
\section{Model, Methods and Implementation}
\subsection{Model of a quantum dot}
%%%%%%%%%%%%%%%%%%%%%%%%%%%%%%%%%%%%%%%%%%%%%%%%%%%%%%%%%%%%%%%%%%%%%%%%%%%%%%%%%%%%%%%%%%%%%%
\begin{frame}
\frametitle{Model of a quantum dot}
\begin{block}{Simple model of a quantum dot}
\scriptsize
\begin{itemize}
\item \textbf{Atomic scale} problem $\Rightarrow$ \textbf{Quantum mechanics} for an accurate description of the system.\\ $\Rightarrow$ at rest, solve the time-independent Schr\"odinger equation $\hat{H} |\Psi \rangle=E |\Psi \rangle$.
 \item Not modelling all nuclei/electrons %of a QD and their interactions would just ask for too much resources. \\
$\Rightarrow$ just model the few quasiparticles confined by the semiconductor. %The Schr\"odinger equation becomes
% % %  \begin{equation}
% % %  %%\hat{H}(\bf{r_1},\bf{{r_2},\dots,\bf{{r_N}) |\Psi_\kappa(\bf{{r_1},\bf{{r_2},\dots,\bf{{r_N}) \rangle=E_\kappa |\Psi_\kappa(\bf{{r_1},\bf{{r_2},\dots,\bf{{r_N}) \rangle
% % %  \end{equation} 
% % %  \\ where $\bf{r_i}$ reprensents the (spatial/spin) coordinates of quasiparticle $i$, $\kappa$ stands for all quantum numbers need to classify a given $N$-particle state and $ | \Psi_\kappa \rangle$, $E_\kappa$ the respective eigenfunctions and eigenenergies of the system.
\end{itemize}
  \end{block}

\begin{columns}[T,l]
\column{5.5cm}
\begin{figure}
	\begin{center}
	\scalebox{0.3}{	\scalebox{0.4}{\includegraphics{IMAGES/schemaQD}}}%%end Scalebox
	\end{center}
\end{figure}
\column{5.5cm}
\begin{figure}
	\begin{center}
	\scalebox{0.3}{\scalebox{0.65}{\includegraphics{IMAGES/SimenDoubleDot}}}%%end Scalebox
	\end{center}
	\caption{Illustration of a quantum dot model \newline (Courtesy of S.Kvaal\cite{SimenThesis}).}
\end{figure}

% % %   \end{alertblock}
\end{columns}
\end{frame}


%%%%%%%%%%%%%%%%%%%%%%%%%%%%%%%%%%%%%%%%%%%%%%%%%%%%%%%%%%%%%%%%%%%%%%%%%%%%%%%%%%%%%%%%%%%%%%
\begin{frame}
\frametitle{Model of a $N$-particle system}
%\scriptsize
\small
\begin{block}{The Schr\"odinger equation}
\begin{itemize}
\item The Schr\"odinger equation of a $N$-particle system
\begin{equation}
\hat{H}(\mathbf{r_1},\mathbf{r_2},\dots,\mathbf{r_N}) | \Psi_\kappa (\mathbf{r_1},\mathbf{r_2},\dots,\mathbf{r_N}) \rangle=E_\kappa | \Psi_\kappa (\mathbf{r_1},\mathbf{r_2},\dots,\mathbf{r_N}) \rangle
\end{equation} 
 \\ where $\bf{r_i}$ reprensents the (spatial/spin) coordinates of quasiparticle $i$, $\kappa$ stands for all quantum numbers needed to classify a given $N$-particle state, \newline $ | \Psi_\kappa \rangle$ and $E_\kappa$ are the respective eigenstates and eigenenergies of the system.
\end{itemize}
  \end{block}
\begin{alertblock}{How to define our Hamiltonian?}
\begin{itemize}
\item $\hat{H}= \sum_{i=1}^{N_e} \frac{\mathbf{p_i}^2}{2m^*} +\dots$
% % % % % \item $\hat{H} = \hat{T} + \hat{V}$
% % % % %  \newline where $\hat{T}$: kinetic energy operator,\newline $\hat{V}$: potential energy operator (including confining potential and particle interaction potential).
\end{itemize}
  \end{alertblock}
\end{frame}


%%%%%%%%%%%%%%%%%%%%%%%%%%%%%%%%%%%%%%%%%%%%%%%%%%%%%%%%%%%%%%%%%%%%%%%%%%%%%%%%%%%%%%%%%%%%%%
\begin{frame}
\frametitle{Definitions of the interactions/potentials}
\begin{small}
%\scriptsize
Forces/Fields acting on the quasiparticles:
\begin{itemize}
 \item Forces confining the particles $\Rightarrow$ \alert{Confining potential}
\item Interactions between the particles $\Rightarrow$ \alert{Interaction potential}
\end{itemize}

The Hamiltonian of our two-electron quantum dot model \\
$\equiv$ the Hamiltonian of 2 particles trapped in a harmonic oscillator potential repealing each other by a Coulomb interaction:
\tikzstyle{na} = [baseline=-.5ex]
\begin{itemize}%[<+-| alert@+>]
 \item Confining potential $\Rightarrow$ \alert{the harmonic}
	\tikz[na] \node[coordinate] (n1) {}; 
 \alert{oscillator (parabolic) potential}

\end{itemize}

\begin{equation}
  \hat{H}= \sum_{i=1}^{N_e=2} \frac{\mathbf{p_i}^2}{2m^*} +
        \tikz[baseline]{
            \node[fill=blue!20, anchor=base] (t1)
            {$\sum_{i=1}^{N_e=2} \frac{1}{2} m^* \omega_0^2 \norm{\mathbf{r_i}}^2$};
        } +
        \tikz[baseline]{
            \node[fill=green!20,anchor=base] (t2)
            {$\frac{e^2}{4 \pi \epsilon_0 \epsilon_r} \frac{1}{\norm{\mathbf{r_1}-\mathbf{r_2}}},$};
	}
\end{equation}
\begin{itemize}%[<+-| alert@+>]
\item Interaction potential  $\Rightarrow$  \alert{the two-body Coulomb interaction}
	\tikz[na] \node[coordinate] (n2) {};
\end{itemize}

where $\mathbf{p_i}$ is the canonical momentum of the particle $i$.

% Now it's time to draw some edges between the global nodes. Note that we
% have to apply the 'overlay' style.
\begin{tikzpicture}[overlay]
% % % %         \path[->]<1-> (n1) edge [bend left] (t1);%to [out=270,in=90] (t1);
% % % %         \path[->]<2-> (n2) to [out=90,in=270] (t2);
        \path[->]<1-> (n1) edge [bend left] (t1);%to [out=270,in=90] (t1);
        \path[->]<1-> (n2) edge [bend right] (t2);
\end{tikzpicture}

\end{small}
\end{frame}

%%%%%%%%%%%%%%%%%%%%%%%%%%%%%%%%%%%%%%%%%%%%%%%%%%%%%%%%%%%%%%%%%%%%%%%%%%%%%%%%%%%%%%%%%%%%%%
\begin{frame}
\frametitle{Applying an external magnetic field}
\begin{scriptsize}
%\scriptsize
%%%%Within a magnetic field:
\begin{enumerate}
 \item 
	$\mathbf{p_i} \longrightarrow \mathbf{p_i}+e\mathbf{A}$, %%%and the Hamiltonian of a $N$-particle QD becomes:
% % % \tikzstyle{na} = [baseline=-.5ex]
% % % \begin{equation}
% % %   \hat{H}= \sum_{i=1}^{N_e}\left(  \frac{(\mathbf{p_i}+e\mathbf{A})^2}{2m^*}  + \frac{1}{2} m^* \omega_0^2 \norm{\mathbf{r_i}}^2  \right) + \frac{e^2}{4 \pi \epsilon_0 \epsilon_r} \sum_{i<j}^{N_e}\frac{1}{\norm{\mathbf{r_i}-\mathbf{r_j}}},
% % % \end{equation}
where $\mathbf{A}$ is the vector potential defined by $\mathbf{B}=\nabla \times \mathbf{A}$.
\end{enumerate}
\begin{itemize}
 \item
	in coordinate space $\mathbf{p_i} \rightarrow -i \hbar \nabla_i$,
\item
	using a Coulomb gauge $\nabla \cdot \mathbf{A} = 0$ (by choosing $\mathbf{A}(\mathbf{r_i}) = \frac{\mathbf{B} \times \mathbf{r_i}}{2}$),
\begin{equation}
(\mathbf{p_i}+e\mathbf{A})^2 \rightarrow  \left( -\frac{ \hbar^2}{2m^*} \nabla_i^2- i \hbar \frac{e}{m^*} \mathbf{A}\cdot \nabla_i + \frac{e^2}{2m^*}\mathbf{A}^2  \right).
\end{equation}
In terms of  $\mathbf{B}$, the linear and quadratic terms in  $\mathbf{A}$ have the form
$\frac{-i \hbar e}{m^*} \mathbf{A} \cdot \nabla_i = \frac{ e}{2m^*} \mathbf{B} \cdot \mathbf{L}$, and $\frac{e^2}{2m^*} \mathbf{A}^2 = \frac{e^2}{8m^*} B^2 r_i^2$.
where $\mathbf{L}=-i \hbar (\mathbf{r_i} \times \nabla_i)$ is the orbital angular momentum operator of the electron $i$.
\end{itemize}
\begin{enumerate}
\setcounter{enumi}{1}
\item 
	$\mathbf{B}$ also act on spin with the additional energy term: $\hat{H_s}= g^*_s \frac{\omega_c}{2} \hat{S_z}$, where $\hat{S}$ is the spin operator of the electron and $g^*_s$ its effective spin gyromagnetic ratio and $\omega_c=e B/m^*$ is known as the cyclotron frequency.
\end{enumerate}
\end{scriptsize}
\end{frame}


%%%%%%%%%%%%%%%%%%%%%%%%%%%%%%%%%%%%%%%%%%%%%%%%%%%%%%%%%%%%%%%%%%%%%%%%%%%%%%%%%%%%%%%%%%%%%%
\begin{frame}
\frametitle{Final Hamiltonian}
\begin{small}
The final Hamiltonian reads:
\begin{align}
  \hat{H}&=\sum_{i=1}^{N_e} \bigg(  \frac{- \hbar^2}{2m^*} \nabla_i^2 + \overbrace{\frac{1}{2} m^* \omega_0^2 \norm{\mathbf{r_i}}^2}^{\begin{smallmatrix}
  \text{Harmonic ocscillator} \\
  \text{potential}
\end{smallmatrix}} \bigg) + \overbrace{\frac{e^2}{4 \pi \epsilon_0 \epsilon_r} \sum_{i<j}\frac{1}{\abs{\mathbf{r_i}-\mathbf{r_j}}}}^{\begin{smallmatrix}
  \text{Coulomb} \\
  \text{interactions}
\end{smallmatrix}}  \nonumber \\
& + \underbrace{\sum_{i=1}^{N_e} \left( \frac{1}{2} m^* \left( \frac{\omega_c}{2} \right)^2 \norm{\mathbf{r_i}}^2 + \frac{1}{2}  \omega_c \hat{L}_z^{(i)}+ \frac{1  }{2} g_s^*  \omega_c \hat{S}_z^{(i)}\right)}_{\begin{smallmatrix}
  \text{single particle interactions} \\
  \text{with the magnetic field}
\end{smallmatrix}},
\end{align}
\end{small}
\end{frame}


%%%%%%%%%%%%%%%%%%%%%%%%%%%%%%%%%%%%%%%%%%%%%%%%%%%%%%%%%%%%%%%%%%%%%%%%%%%%%%%%%%%%%%%%%%%%%%
\begin{frame}
\frametitle{Scaling the problem: dimensionless form of $\hat{H}$}
\begin{footnotesize}
New constant, the oscillator frequency $\omega = \sqrt{\omega_0+\left( \frac{\omega_c}{2} \right)^2}$, \newline
New units:
\begin{itemize}
\item the energy unit $\hbar \omega$,
\item 	the length unit, the oscillator length defined by $l=\sqrt{\hbar /(m^* \omega)}$.
\end{itemize}
The dimensionless Hamiltonian gives
\begin{equation}
  \hat{H}=\sum_{i=1}^{N_e} \left[  -\frac{1}{2} \nabla_i^2 + \frac{1}{2} r_i^2  \right]+ \lambda \sum_{i<j}\frac{1}{r_{ij}} +  \sum_{i=1}^{N_e} \left(  \frac{1}{2}  \frac{\omega_c}{\hbar \omega} m_l^{(i)}+ \frac{1  }{2} g_s^* \frac{\omega_c}{\hbar \omega} m_s^{(i)}\right).
\end{equation}
where new dimensionless parameter $\lambda=l / a_0^*$ describes the strength of the electron-electron interaction ($a_0^*= 4 \pi \epsilon_0 \epsilon_r \hbar^2 / (e^2 m^*)$ being the effective Bohr radius).
\end{footnotesize}
\end{frame}


%%%%%%%%%%%%%%%%%%%%%%%%%%%%%%%%%%%%%%%%%%%%%%%%%%%%%%%%%%%%%%%%%%%%%%%%%%%%%%%%%%%%%%%%%%%%%%
\begin{frame}
\frametitle{The Hamiltonian solved in this project}
\begin{footnotesize}

\begin{columns}[T,l]
\column{6cm}
$\lambda(B)=\frac{1}{a_0^*} \left( \frac{4 \hbar^2}{4 \omega_0^2 m^{*2}+e^2 B^2} \right)^{\frac{1}{4}}$

\begin{alertblock}{Does this model make sense including $\mathbf{B}?$}%%%How a strong $\mathbf{B}$ field will affect the model?}
\begin{itemize}
\item
	Squeezing the particles should increase the strength of the electron-electron interaction.
\item
	$\lambda$ only decreases as the magnetic field increases in this model.
\end{itemize}
\end{alertblock}
\column{5cm}
\begin{figure}
	\begin{center}
		\scalebox{0.7}{\input{../IMAGES/lambdaB.tex}}
	\end{center}
	\caption{Dimensionless confinement strength $\lambda$ as a function of the magnetic field strength in GaAs.}
\end{figure}
\end{columns}
In the rest of the project, we will solve the following Hamiltonian:
\begin{equation}
  \hat{H}=\sum_{i=1}^{N_e} \left[  -\frac{1}{2} \nabla_i^2 + \frac{1}{2} r_i^2  \right]+ \lambda \sum_{i<j}\frac{1}{r_{ij}}.
\end{equation}

\end{footnotesize}
\end{frame}



\subsection{The method of Variational calculus}
%%%%%%%%%%%%%%%%%%%%%%%%%%%%%%%%%%%%%%%%%%%%%%%%%%%%%%%%%%%%%%%%%%%%%%%%%%%%%%%%%%%%%%%%%%%%%%
\begin{frame}
\frametitle{The method of variational calculus}
\begin{scriptsize}
\begin{definition}
  Method to solve the Schr\"odinger eq. much more efficiently than using numerical integration.
\end{definition}
\begin{itemize}
\item Based on the method of Lagrange multipliers, where the functional to minimize (the energy functional) is an integral over the unknown wave function $| \Phi \rangle$
\begin{equation} 
  E[\Phi]= \frac{\langle \Phi |H | \Phi \rangle}{\langle \Phi | \Phi \rangle}= \frac{ \int \Phi^* H  \Phi d \tau}{\int  \Phi^*  \Phi d \tau},
\end{equation}
while subject to a normalization constraint $\langle \Phi | \Phi \rangle=1$.
\item The method introduces new variables for each of the constraints (the Lagrange multipliers $\epsilon$) and defines  the Lagrangian ($\Lambda$) with respect to $| \Phi \rangle$ as
\begin{equation} 
  \Lambda(\Phi,\epsilon)= E[\Phi] - \epsilon \left( \langle \Phi | \Phi \rangle -1 \right),
\end{equation}
\item Find stationnary solutions by solving the set of equations by writing $\frac{\partial \Lambda}{\partial\Phi}=0$.
\end{itemize}

\end{scriptsize}
\end{frame}

\subsection{The Hartree-Fock method}
%%%%%%%%%%%%%%%%%%%%%%%%%%%%%%%%%%%%%%%%%%%%%%%%%%%%%%%%%%%%%%%%%%%%%%%%%%%%%%%%%%%%%%%%%%%%%%
\begin{frame}
\frametitle{The Hartree-Fock method}
\begin{footnotesize}
\begin{definition}
The HF method is a particular case of variational method in accordance with
\begin{itemize}
 \item the independent particle approximation,
\item the Pauli exclusion principle.
\end{itemize}
\end{definition}
\begin{alertblock}{The approximated wave function}
To fullfil these criteria, the wave-function must be antisymmetric with respect to an interchange of any two particles: \begin{equation}
\Phi(\mathbf{r_1},\mathbf{r_2},\dots,\mathbf{r_i},\dots, \mathbf{r_j},\dots,\mathbf{r_N})=-\Phi(\mathbf{r_1},\mathbf{r_2},\dots,\mathbf{r_j},\dots, \mathbf{r_i},\dots,\mathbf{r_N}).
\end{equation}

The Slater determinant is an elegant way of writing the antisymmetric product of single particle orbitals.
\end{alertblock}
\end{footnotesize}
\end{frame}


%%%%%%%%%%%%%%%%%%%%%%%%%%%%%%%%%%%%%%%%%%%%%%%%%%%%%%%%%%%%%%%%%%%%%%%%%%%%%%%%%%%%%%%%%%%%%%
\begin{frame}
\frametitle{The Hartree-Fock wave function}
\begin{scriptsize}
% \begin{block}{The Slater determinant}
The \alert{Slater determinant} is an antisymmetric product of the single particle orbitals:
\begin{equation}
\Phi(\mathbf{r_1},\mathbf{r_2},\dots,\mathbf{r_N},\alpha,\beta,\dots,\sigma)= \frac{1}{\sqrt{N!}}\left|
 \begin{array}{cccc}
  \psi_{\alpha}(\mathbf{r}_1)&\psi_{\alpha}(\mathbf{r}_2)&\dots&\psi_{\alpha}(\mathbf{r}_N) \\ [4pt]
  \psi_{\beta}(\mathbf{r}_1)&\psi_{\beta}(\mathbf{r}_2)&\dots&\psi_{\beta}(\mathbf{r}_N) \\[4pt]
  \vdots              & \vdots            &\ddots&\vdots\\[4pt]
  \psi_{\sigma}(\mathbf{r}_1)&\psi_{\sigma}(\mathbf{r}_2)&\dots&\psi_{\sigma}(\mathbf{r}_N)
 \end{array}
 \right|,
\end{equation}
% where the variables $\mathbf{r_i}$ include the coordinates of spin and space of particle $i$, and $\alpha,\beta,\dots,\sigma$ encompass all possible quantum numbers needed to specify a particular system.
It can be rewritten as 
\begin{align}
\Phi_T(\mathbf{r_1},\mathbf{r_2},\dots,\mathbf{r_N}, \alpha, \beta, \dots, \sigma)&=\frac{1}{\sqrt{N!}}\sum_p (-)^p P \Psi_\alpha(\mathbf{r_1})\Psi_\beta(\mathbf{r_2}) \dots \Psi_\sigma(\mathbf{r_N})\\
&=\sqrt{N!} \mathcal{A} \Phi_H,
\end{align}
by introducing the \alert{antisymmetrization operator $\mathcal{A}$} defined by the summation over all possible permutations of 2 particles $\mathcal{A}=\frac{1}{N!} \sum_P (-)^p \hat{P}$.
% \end{block}
\end{scriptsize}
\end{frame}


%%%%%%%%%%%%%%%%%%%%%%%%%%%%%%%%%%%%%%%%%%%%%%%%%%%%%%%%%%%%%%%%%%%%%%%%%%%%%%%%%%%%%%%%%%%%%%
\begin{frame}
\frametitle{Matrix elements calculations}
\begin{scriptsize}
\begin{definition}
We write the Hamiltonian for $N$ electrons as $\hat{H}=\hat{H}_0+\hat{H}_1=\sum_{i=1}^{N}\hat{h}_i+ \sum_{i<j}^{N}v(\mathbf{r_i},\mathbf{r_j})$,
where $r_{ij}=\norm{\vec{r_i}-\vec{r_j}}$, $\hat{h}_i$ and $v(\mathbf{r_i},\mathbf{r_j})$  are respectively the one-body and the two-body Hamiltonian.
\end{definition}
Using properties of $\mathcal{A}$ and commutation rule with $\hat{H}_0$ and $\hat{H}_1$, one can write:

\begin{align}
&\int \Phi_T^* \hat{H}_0 \Phi_T d\tau = \sum_{\mu=1}^N  \int \Psi_\mu^*(\mathbf{r}) \hat{h} \Psi_\mu(\mathbf{r}) d\mathbf{r} = \sum_{\mu=1}^N \langle \mu |h| \mu \rangle.\\
&\int \Phi_T^*\hat{H}_1 \Phi_T d\tau = \frac{1}{2}  \sum_{\mu=1}^N \sum_{\nu=1}^N \langle \mu \nu |V| \mu \nu \rangle_{AS}.
\end{align}
where we define the antisymmetrized matrix element as $\langle \mu \nu |V| \mu \nu \rangle_{AS} = \langle \mu \nu |V| \mu \nu \rangle - \langle \mu \nu |V| \nu \mu \rangle$, with the following shorthands $\langle \mu \nu |V| \mu \nu \rangle = \int \Psi_\mu^*(\mathbf{r_i}) \Psi_\nu^*(\mathbf{r_j}) V(r_{ij}) \Psi_\mu(\mathbf{r_i}) \Psi_\nu(\mathbf{r_j}) d\mathbf{r_i} d\mathbf{r_j}$.
\end{scriptsize}
\end{frame}


%%%%%%%%%%%%%%%%%%%%%%%%%%%%%%%%%%%%%%%%%%%%%%%%%%%%%%%%%%%%%%%%%%%%%%%%%%%%%%%%%%%%%%%%%%%%%
\begin{frame}
\frametitle{The Hartree-Fock energy in the harmonic oscillator basis}
\begin{scriptsize}
\begin{block}{The energy functional}
The energy functional is our starting point for the Hartree-Fock calculations.
\begin{align}
E [ \Phi_T ] &= \langle \Phi_T | \hat{H}_0 | \Phi_T  \rangle + \langle \Phi_T | \hat{H}_1 | \Phi_T  \rangle \\
 &= \sum_{\mu=1}^N \langle \mu |h| \mu \rangle + \frac{1}{2}  \sum_{\mu=1}^N \sum_{\nu=1}^N \langle \mu \nu |V| \mu \nu \rangle_{AS}.
\end{align}
\end{block}
We expand each single-particle eigenvector $\Psi_i$ in terms of a convenient complete set of single-particle states $|\alpha \rangle$ (the harmonic oscillator eigenstates in our case),
\begin{equation}
\Psi_i= |i \rangle = \sum_{\alpha} c_i^\alpha |\alpha \rangle.
\end{equation}
The energy functional now reads
\tikzstyle{na} = [baseline=-.5ex]
\begin{equation}
         \tikz[baseline]{
            \node[fill=blue!20, anchor=base] (a1)
            {$E[\Phi] = \sum_{i=1}^N \sum_{\alpha \gamma} C_i^{\alpha *} C_i^{\gamma} \langle \alpha |h| \gamma \rangle + \frac{1}{2}   \sum_{i,j=1}^N \sum_{\alpha \beta \gamma \delta} C_i^{\alpha *} C_j^{\beta *} C_i^{\gamma} C_j^{\delta}  \langle \alpha \beta |V| \gamma \delta \rangle_{AS}.$};
        }
\end{equation}
\end{scriptsize}
\end{frame}


%%%%%%%%%%%%%%%%%%%%%%%%%%%%%%%%%%%%%%%%%%%%%%%%%%%%%%%%%%%%%%%%%%%%%%%%%%%%%%%%%%%%%%%%%%%%%
\begin{frame}
\frametitle{The Hartree-Fock equations (1)}
\begin{scriptsize}
\begin{alertblock}{Remember the method of the Lagrange multipliers}
\begin{enumerate}
 \item Define a functional $E [ \Phi_T ]$,
\item  Identify the constraints: $\langle \Psi_i |\Psi_j \rangle= \delta_{ij}$ which implies $\langle \Phi_T | \Phi_T  \rangle=1$, \newline with $\langle \Psi_i |\Psi_j \rangle = \sum_{\alpha \beta} C_i^{\alpha *} C_j^\beta \underbrace{\langle \alpha | \beta \rangle}_{\delta_{ \alpha \beta } } = \sum_{\alpha} C_i^{\alpha *} C_j^\alpha $
\item Define the Lagrangian $\Lambda$
\begin{equation}
\Lambda(C_1^{\alpha},C_2^{\alpha},\dots,C_N^{\alpha},\epsilon_1,\epsilon_2,\dots,\epsilon_N)= E[\Phi_T] - \sum_{i=1}^N \epsilon_i \left( \sum_\alpha C_i^{\alpha *} C_i^{\alpha} - \delta_{ij} \right).
\end{equation}
where $\epsilon_i$ are the Lagrange multipliers for each of the normalization constraints.
\item Get the system of equations to solve by setting $\Lambda$
\begin{equation}
\frac{d \Lambda}{d\Phi_T}  \equiv \frac{d}{dC_i^{\alpha *}} \left[ \Lambda(C_1^{\alpha},C_2^{\alpha},\dots,C_N^{\alpha},\epsilon_1,\epsilon_2,\dots,\epsilon_N) \right] = 0, \quad \forall \; i \in \mathbb{N}^*.
\end{equation}
\end{enumerate}
\end{alertblock}
\end{scriptsize}
\end{frame}



%%%%%%%%%%%%%%%%%%%%%%%%%%%%%%%%%%%%%%%%%%%%%%%%%%%%%%%%%%%%%%%%%%%%%%%%%%%%%%%%%%%%%%%%%%%%%
\begin{frame}
\frametitle{The Hartree-Fock equations (2)}
\begin{scriptsize}
Treating $C_i^\alpha$ and  $C_i^{\alpha *}$ as independent, we arrive at the Hartree-Fock equations \newline (one equation for each basis state $|\alpha \rangle$)


\tikzstyle{na} = [baseline=-.5ex]
\begin{equation}
         \tikz[baseline]{
            \node[fill=blue!20, anchor=base] (a1)
            {$\sum_\gamma \langle \alpha | h| \gamma \rangle \; C_i^\gamma +   \sum_\gamma \overbrace{\Bigg[ \sum_{j=1}^N \sum_{\beta \delta} C_j^{\beta *} \underbrace{\langle \alpha \beta | V | \gamma  \delta \rangle_{AS}}_{\begin{smallmatrix}
  \text{Two-body interaction} \\
  \text{matrix element} V_{\alpha \beta \gamma \delta}
\end{smallmatrix}}  \; C_j^\delta \Bigg] }^{\begin{smallmatrix}
  \text{Effective potential} \\
  \langle \alpha |U| \gamma \rangle
\end{smallmatrix}} C_i^\gamma = \epsilon_i \ C_i^\alpha,$};}
\end{equation}
which we can rewrite as $\sum_\gamma \mathcal{O}_{\alpha \gamma} \; C_i^\gamma = \epsilon_i \ C_i^\alpha, \quad \forall \; \alpha \in \mathcal{H}$.

$\Rightarrow$ \alert{System of non-linear equations} in the $C_i^{\alpha *}$, since $\mathcal{O}_{\alpha \gamma}$ depends itself on the unknowns.

$\Rightarrow$ \alert{To be solved by an iterative procedure}.
\end{scriptsize}
\end{frame}


%%%%%%%%%%%%%%%%%%%%%%%%%%%%%%%%%%%%%%%%%%%%%%%%%%%%%%%%%%%%%%%%%%%%%%%%%%%%%%%%%%%%%%%%%%%%%
\begin{frame}
\frametitle{The Hartree-Fock (self-consistent) iterative procedure}
\begin{scriptsize}
\begin{columns}[T,l]
\column{8cm}
\begin{enumerate}
 \item Compute the effective Coulomb interaction potential $\langle \alpha |U^{(0)}| \gamma \rangle$ with an initial guess of the $C_i^{\alpha(0)}$.
\item Build the resulting Fock matrix $\mathcal{O}$.
\item Solve the linearised system given by the equations \newline (Fock matrix diagonalization)
 \begin{equation}
\nonumber
\sum_\gamma \left[ \langle \alpha | h| \gamma \rangle +  \langle \alpha | U| \gamma \rangle \right] C_i^\gamma = \epsilon_i \ C_i^\alpha.
\end{equation}
at iteration $(k)$, store the output eigenenergies $\epsilon_i^{(k)}$  \newline and the coefficients of the new eigenvectors $C_i^{\alpha(k)}$.
\item Substitute back the new coefficients to compute a new Coulomb interaction potential.
\item $\dots$
\item Continue the process until self-consistency is reached  \newline (i.e.\ convergence of the $\epsilon_i$ for an arbitrary precision.)
\end{enumerate}
\column{3cm}
\begin{figure}
\centering
\scalebox{0.5}{\input{IMAGES/diagramHF.tex}}
\caption{Flowchart of Hartree-Fock algorithm.}
\end{figure}
\end{columns}
\end{scriptsize}
\end{frame}

\subsection{The many-body perturbation theory}
%%%%%%%%%%%%%%%%%%%%%%%%%%%%%%%%%%%%%%%%%%%%%%%%%%%%%%%%%%%%%%%%%%%%%%%%%%%%%%%%%%%%%%%%%%%%%
\begin{frame}
\frametitle{Many-body perturbation theory}
\begin{scriptsize}
Take the Hamiltonian  $\hat{H}=\hat{H}_0+\hat{H}'$, and treat $\hat{H}'$ as a perturbation, such as the Coulomb interaction.

Suppose $\Phi_n$ eigenfunctions of $\hat{H}_0$ corresponding to the eigenvalues $E_n$: $\hat{H}_0 \Phi_n = E_n \Phi_n$.
Consider the effect of the perturbation on a particular state $\Phi_0$.

We denote by $\Psi_0$ the state into which $\Phi_0$ changes under the action of the perturbation, so that $\Psi_0$ is an eigenfunction of $\hat{H}$, corresponding to the eigenvalue $E$.
\begin{align}
 \hat{H}_0 \Phi_0 &= E_0 \Phi_0. \\
\hat{H}_0 \Psi_0 &= E \Psi_0.
\end{align}
Therefore $\Phi_0$ and $\Psi_0$ denote the ground states of the unperturbed and perturbed systems respectively.

Since $\hat{H}_0$ is Hermitian, one can show that:
\begin{equation}
E-E_0 = \frac{\langle \Phi_0 |\hat{H}'| \Psi_0 \rangle}{\langle \Phi_0 |\Psi_0 \rangle}.
\end{equation}
which is an exact expression and independent of any particular perturbation method.

Since $\Psi_0$ is unknown, using a \textit{projection operator} $\mathbf{R}$ for the state $\Phi_0$ defined by the equation
\begin{equation}
\mathbf{R} \Psi = \Psi - \Phi_0 \langle \Phi_0 | \Psi\rangle,
\end{equation}
\end{scriptsize}
\end{frame}


%%%%%%%%%%%%%%%%%%%%%%%%%%%%%%%%%%%%%%%%%%%%%%%%%%%%%%%%%%%%%%%%%%%%%%%%%%%%%%%%%%%%%%%%%%%%%
\begin{frame}
\frametitle{The perturbed energy}
\begin{scriptsize}
The perturbed energy can be derived from the iterated $\Psi_0$ which gives
\begin{equation}
\label{eq:176}
E-E_0 = \sum_{n=0}^{\infty} \big\langle   \Phi_0  | \hat{H}' \left(  \frac{\mathbf{R}}{E_0 - \hat{H}_0} (E_0 - E +\hat{H}') \right)^n   |   \Phi_0 \big\rangle.
\end{equation}
It will be observed that the right-hand side of this equation also contains $E$, but this is eliminated when the terms are expanded.
We shall write
\begin{equation}
\nonumber
\Delta E = E -E_0 = \Delta E^{(1)} + \Delta E^{(2)} +\Delta E^{(3)} + \dots
\end{equation}
where the $m^{th}$-order energy correction $\Delta E^{(m)}$ contains the $m^{th}$-order power of the perturbation $\hat{H}'$.
\end{scriptsize}
\end{frame}


%%%%%%%%%%%%%%%%%%%%%%%%%%%%%%%%%%%%%%%%%%%%%%%%%%%%%%%%%%%%%%%%%%%%%%%%%%%%%%%%%%%%%%%%%%%%%
\begin{frame}
\frametitle{The many-body perturbation corrections}
\begin{scriptsize}

\begin{itemize}
 \item The $1^{st}$-order correction is 
\begin{equation}
\Delta E^{(1)} = \langle   \Phi_0  | \hat{H}' | \Phi_0 \rangle.
\end{equation}
\item The $2^{nd}$-order correction is 
\begin{equation}
\Delta E^{(2)} = \langle   \Phi_0  | \hat{H}'  \frac{\mathbf{R}}{E_0 - \hat{H}_0} (E_0 - E +\hat{H}')   |   \Phi_0 \rangle.
\end{equation}
\item The $3^{rd}$-order energy correction reads
\begin{align}
\label{eq:3rdOrderMBPT}
\Delta E^{(3)} &= \sum_{n=0}^{\infty} \sum_{n=0}^{\infty}  \frac{  \langle \Phi_0 |\hat{H}'| \Phi_m  \rangle  \langle \Phi_m |\hat{H}'| \Phi_n  \rangle \langle \Phi_n |\hat{H}'| \Phi_0  \rangle }{(E_0 - E_m)(E_0 - E_n)} \\ \nonumber
&  - \langle \Phi_0 |\hat{H}'| \Phi_0  \rangle \sum_{n=0}^{\infty} \frac{ \langle \Phi_0 |\hat{H}'| \Phi_n  \rangle \langle \Phi_n |\hat{H}'| \Phi_0  \rangle }{(E_0 - E_n)^2}.
\end{align}
\end{itemize}

\end{scriptsize}
\end{frame}



%%%%%%%%%%%%%%%%%%%%%%%%%%%%%%%%%%%%%%%%%%%%%%%%%%%%%%%%%%%%%%%%%%%%%%%%%%%%%%%%%%%%%%%%%%%%%
\begin{frame}
\frametitle{The MBPT corrections expanded in a basis set}
\begin{scriptsize}
It is possible to rewrite the many-body energy corrections in particle and hole state formalism by using the expression of $\hat{H}'$ as expressed in terms of anihilation ($c_k$) and creation ($c_k^{\dagger}$) operators 
\begin{equation}
\nonumber
\hat{H}'= \frac{1}{2} \sum_{ijkl} \langle ij | v | kl \rangle c_i^{\dagger} c_j^{\dagger}  c_l c_k,
\end{equation}

The previous many-body perturbation corrections now read
\begin{align}
\Delta E^{(1)} &= \langle   \Phi_0  | \hat{H}' | \Phi_0 \rangle.= \frac{1}{2} \sum_{h_1 h_2} \langle h_1 h_2 |v| h_1 h_2\rangle_{as}, \\ \nonumber
\Delta E^{(2)} &= \sum_{n=0}^{\infty} \frac{\langle \Phi_0 |\hat{H}'| \Phi_n  \rangle \langle \Phi_n |\hat{H}'| \Phi_0 \rangle}{E_0 - E_n} = \frac{1}{4} \sum_{h_1 h_2 p_1 p_2} \frac{\abs{\langle h_1 h_2 |v|p_1 p_2 \rangle}_{as}^2}{\epsilon_{h_1}+\epsilon_{h_2}-\epsilon_{p_1}-\epsilon_{p_2}},
\end{align}
where $h_i$ and $p_i$ are respectively hole states and particles states, and $\epsilon_i$ are the single particle energies of the basis set.
\end{scriptsize}
\end{frame}


%%%%%%%%%%%%%%%%%%%%%%%%%%%%%%%%%%%%%%%%%%%%%%%%%%%%%%%%%%%%%%%%%%%%%%%%%%%%%%%%%%%%%%%%%%%%%
\begin{frame}
\frametitle{The MBPT corrections expanded in a basis set}
\begin{scriptsize}
The $3^{rd}$-order many-body perturbation correction reads
\begin{align}
\nonumber
\Delta E^{(3)} &= \sum_{n=0}^{\infty} \sum_{n=0}^{\infty}  \frac{  \langle \Phi_0 |\hat{H}'| \Phi_m  \rangle  \langle \Phi_m |\hat{H}'| \Phi_n  \rangle \langle \Phi_n |\hat{H}'| \Phi_0  \rangle }{(E_0 - E_m)(E_0 - E_n)} \\ \nonumber
&  - \langle \Phi_0 |\hat{H}'| \Phi_0  \rangle \sum_{n=0}^{\infty} \frac{ \langle \Phi_0 |\hat{H}'| \Phi_n  \rangle \langle \Phi_n |\hat{H}'| \Phi_0  \rangle }{(E_0 - E_n)^2}\\
&= \Delta E^{(3)}_{4p-2h} + \Delta E^{(3)}_{2p-4h} + \Delta E^{(3)}_{3p-3h},
\end{align}
where 
\begin{itemize}
 \item $\Delta E^{(3)}_{4p-2h}$ is the contribution to the third-order energy correction due to the 4-particle/2-hole excitations,
\item  $\Delta E^{(3)}_{2p-4h}$ is the contribution to the third-order energy correction due to the 2-particle/4-hole excitations,
\item  $\Delta E^{(3)}_{3p-3h}$ is the contribution to the third-order energy correction due to the 3-particle/3-hole excitations.
\end{itemize}
\end{scriptsize}
\end{frame}


%%%%%%%%%%%%%%%%%%%%%%%%%%%%%%%%%%%%%%%%%%%%%%%%%%%%%%%%%%%%%%%%%%%%%%%%%%%%%%%%%%%%%%%%%%%%%
\begin{frame}
\frametitle{The MBPT corrections expanded in a basis set}
\begin{scriptsize}
The contributions to the third-order energy correction can be written as
\begin{align*}
\Delta E^{(3)}_{4p-2h} &= \frac{1}{8} \sum_{h_1 h_2 p_1 p_2}  \left(  \frac{\langle h_1 h_2 |v|p_1 p_2 \rangle_{as}}{\epsilon_{h_1}+\epsilon_{h_2}-\epsilon_{p_1}-\epsilon_{p_2}}  \sum_{p_3 p_4} \frac{\langle p_1 p_2 |v|p_3 p_4 \rangle_{as} \langle p_3 p_4 |v|h_1 h_2 \rangle_{as}}{\epsilon_{h_1}+\epsilon_{h_2}-\epsilon_{p_3}-\epsilon_{p_4}}    \right),\\ \nonumber
\Delta E^{(3)}_{2p-4h} &= \frac{1}{8} \sum_{h_1 h_2 p_1 p_2}  \left(  \frac{\langle h_1 h_2 |v|p_1 p_2 \rangle_{as}}{\epsilon_{h_1}+\epsilon_{h_2}-\epsilon_{p_1}-\epsilon_{p_2}}  \sum_{h_3 h_4} \frac{\langle h_1 h_2 |v|h_3 h_4 \rangle_{as} \langle h_3 h_4 |v|h_1 h_2 \rangle_{as}}{\epsilon_{h_3}+\epsilon_{h_4}-\epsilon_{p_1}-\epsilon_{p_2}}    \right),\\ \nonumber
\Delta E^{(3)}_{3p-3h} &=  \sum_{h_1 h_2 p_1 p_2}  \left(  \frac{\langle h_1 h_2 |v|p_1 p_2 \rangle_{as}}{\epsilon_{h_1}+\epsilon_{h_2}-\epsilon_{p_1}-\epsilon_{p_2}} \left( \sum_{h_3}\sum_{p_3} \frac{\langle h_1 h_3 |v|p_1 p_3 \rangle_{as} \langle p_3 h_2 |v|h_3 h_2 \rangle_{as}}{\epsilon_{h_1}+\epsilon_{h_3}-\epsilon_{p_1}-\epsilon_{p_3}}  \right)  \right),\nonumber
\end{align*}
where the $p_i$ denote the  particle states, $h_i$ the hole states, and $\epsilon_{i}$ the single particle energies of the corresponfing state.
\end{scriptsize}
\end{frame}

\subsection{Implementation}
% %%%%%%%%%%%%%%%%%%%%%%%%%%%%%%%%%%%%%%%%%%%%%%%%%%%%%%%%%%%%%%%%%%%%%%%%%%%%%%%%%%%%%%%%%%%%%
\begin{frame}
\frametitle{Code implementation}
\begin{scriptsize}
\begin{block}{Tools}
 \begin{itemize}
 \item \citecodeBeamer{C++} language: for flexibility using classes and efficiency.
\item \textsc{Blitz++} library: managing dense arrays.
\item \textsc{Lpp / Lapack} library: (\citecodeBeamer{Fortran}) routines for solving linear algebra.
\item  \textsc{Message-Passing Interface}: to build a cluster for parallel computing 
\end{itemize}
\end{block}
\begin{block}{Functionality}
 \begin{itemize}
 \item Read parameters from a unique textual input file or command line.
\item The \citecodeBeamer{simulator} class performs the initialization and calls other objects.
\item The \citecodeBeamer{orbitalsQuantumNumbers} class: generates the 	harmonic oscillator states.
\item The \citecodeBeamer{CoulombMatrix} class generates the Coulomb interaction matrix outside Hartree-Fock.
\item The \citecodeBeamer{HartreeFock} class computes the HF energy and generates the interaction matrix in the HF basis.
\item The \citecodeBeamer{PerturbationTheory} class computes many-body perturbation corrections from $1^{st}$-	 to $3^{rd}$-order, either in the harmonic oscillator or in the HF basis set.
\end{itemize}
\end{block}
\end{scriptsize}
\end{frame}

% %%%%%%%%%%%%%%%%%%%%%%%%%%%%%%%%%%%%%%%%%%%%%%%%%%%%%%%%%%%%%%%%%%%%%%%%%%%%%%%%%%%%%%%%%%%%%
\begin{frame}
\frametitle{Implementation issues}
\begin{scriptsize}
\begin{alertblock}{Difficulties encountered}
 \begin{itemize}
\item Huge Fock matrix to diagonalize (grows exponentially with $R^b$).
\item Huge Coulomb interaction matrix to store $V_{\alpha \beta \gamma \delta}=\langle (n_1,m_{l1})(n_2,m_{l2})|V|(n_3,m_{l3})(n_4,m_{l4})\rangle$ is a 8-dimensional array.
\end{itemize}
\end{alertblock}
\begin{block}{Solutions implemented}
By using the symmetry and invariance of the antisymmetrized Coulomb interaction:
 \begin{itemize}
\item $V_{\alpha \beta \gamma \delta}$ does not act on spin: $m_{s1}=m_{s3}$ \& $m_{s2}=m_{s4}$.
\item $V_{\alpha \beta \gamma \delta}$ conserves the total spin and angular momentum: $m_{l1}+m_{l2}=m_{l3}+m_{l4}$.
\end{itemize}
  By sorting the states of the basis by blocks of identical angular ($m_l$) and spin ($m_s$) quantum numbers:
 \begin{itemize}
\item It allows to reduce the storage of the Coulomb interactions per blocks of couple of states by avoiding to store zeros's elements.
\item The Fock matrix appears as block diagonal, allowing much smaller eigenvalue problems to solve.
\end{itemize}
\end{block}
\end{scriptsize}
\end{frame}

\section{Results and Discussions}
\subsection{Validation of the simulator}
%%%%%%%%%%%%%%%%%%%%%%%%%%%%%%%%%%%%%%%%%%%%%%%%%%%%%%%%%%%%%%%%%%%%%%%%%%%%%%%%%%%%%%%%%%%%%
\begin{frame}
\frametitle{Validation of the simulator}
\begin{scriptsize}
\begin{columns}[T,l]
\column{5cm}%%%%%%%%%%%%%%%%	\begin{figure}
\begin{figure}
\begin{center}
\scalebox{1}{
	\begin{tabular}{c}
		\scalebox{0.23}{\includegraphics{../IMAGES/WaltMBPT}} \\ 
		\scalebox{0.35}{\input{../IMAGES/waltMBPT.tex}}
	\end{tabular}
}%%end Scalebox
	\vspace{-5pt}
	\caption{$2^{nd}$-order perturbation theory correction for the $2e^-$ QD. Comparison between results of Waltersson (top)~\cite{Waltersson2007} and our results (down).}
	\vspace{-5pt}
\end{center}
\end{figure}
\column{6cm}
\begin{block}{Simple checks}
 \begin{itemize}
\item  Reproduce the non-interacting ground state energies.
\item  Reproduce the two-body interaction matrix elements of \textsc{OpenFCI} almost with machine precision
\end{itemize}
\end{block}
\begin{block}{Comparison of MBPT results with similar experiments}
 \begin{itemize}
\item  Reproduce the non-interacting ground state energies.
\item  Reproduce the two-body interaction matrix of \textsc{OpenFCI} almost with machine precision
\end{itemize}
\end{block}
\end{columns}
\end{scriptsize}
\end{frame}
 
\subsection{Limit of the closed-shell model as a function of $\lambda$}
%%%%%%%%%%%%%%%%%%%%%%%%%%%%%%%%%%%%%%%%%%%%%%%%%%%%%%%%%%%%%%%%%%%%%%%%%%%%%%%%%%%%%%%%%%%%%%
\begin{frame}
\frametitle{Level crossing as a function of $B$ without interactions (1/2)}
\begin{scriptsize}
\begin{block}{Fock-Darwin orbitals}
When neglecting the repulsions between the particles, the eigenenergies $\epsilon_{n \, m_l}$ as a function of the magnetic field $B$ can be solved analytically for a parabolic confining potential $V(r)=1/(2m^*\omega_0^2 r^2)$ leading to a spectrum known as the Fock-Darwin states

Rewriting the eigenenergies in units of $\hbar \omega_0$, $\epsilon_{n \, m_l}$ becomes dimensionless and we obtain
\begin{align}
 \epsilon_{n \, m_l} &= (2n+|m_l|+1)  \sqrt{1+\frac{(\omega_c/ \omega_0)^2}{4}} -\frac{1}{2}(\omega_c /\omega_0) \, m_l\\
&= (2n+|m_l|+1)  \sqrt{1+(\frac{e B}{2m^* \omega_0})^2} -\frac{e B}{2m^* \omega_0} \, m_l.
\end{align}
 \end{block}
\end{scriptsize}
\end{frame}

%%%%%%%%%%%%%%%%%%%%%%%%%%%%%%%%%%%%%%%%%%%%%%%%%%%%%%%%%%%%%%%%%%%%%%%%%%%%%%%%%%%%%%%%%%%%%%
\begin{frame}
\frametitle{Level crossing as a function of $B$ without interactions (2/2)}
\begin{scriptsize}
\begin{columns}[T,l]
\column{5.5cm}
\begin{figure}
\centering
\scalebox{0.45}{\input{../IMAGES/FockFDarwin06.tex}}
\caption{Spectrum of Fock-Darwin orbitals for 6 non-interacting particles (GaAs:$\hbar \omega_0=5meV$,$\epsilon_r=12$).}
\end{figure}
\column{5.5cm}%%%%%%%%%%%%%%%%	\begin{figure}
\begin{figure}
\centering
\scalebox{0.45}{\input{../IMAGES/FockFDarwin12.tex}}
\caption{Spectrum of Fock-Darwin orbitals for 12 non-interacting particles (GaAs:$\hbar \omega_0=5meV$,$\epsilon_r=12$).}
\end{figure}
\end{columns}
\end{scriptsize}
\end{frame}


%%%%%%%%%%%%%%%%%%%%%%%%%%%%%%%%%%%%%%%%%%%%%%%%%%%%%%%%%%%%%%%%%%%%%%%%%%%%%%%%%%%%%%%%%%%%%%
\begin{frame}
\frametitle{Level crossing as a function of $\lambda$ with interactions (1/2)}
\begin{scriptsize}
\begin{figure}
	\begin{center}
\begin{tabular}{cc}
	\scalebox{0.25}{\input{../../qdot/OpenFCI_RESULTS/E_matR4L0p1.tikz}} & \scalebox{0.25}{\input{../../qdot/OpenFCI_RESULTS/E_matR5L1.tikz}} \\
	\scalebox{0.25}{\input{../../qdot/OpenFCI_RESULTS/particles06QD/E_matR5L15.tikz}} & \scalebox{0.25}{\input{../../qdot/OpenFCI_RESULTS/E_matR5L50.tikz}} \\
	%\hline
\end{tabular}
	\end{center}
	\vspace{-5pt}
	\caption{FCI ground state energies for a 6-electron QD with $\lambda=\{0.1,1\}$ (top) and $\lambda=\{15,50\}$ (bottom).}
\end{figure}
\end{scriptsize}
\end{frame}


%%%%%%%%%%%%%%%%%%%%%%%%%%%%%%%%%%%%%%%%%%%%%%%%%%%%%%%%%%%%%%%%%%%%%%%%%%%%%%%%%%%%%%%%%%%%%%
\begin{frame}
\frametitle{Level crossing as a function of $\lambda$ with interactions (2/2)}
\begin{scriptsize}
\begin{alertblock}{Summary of FCI results using \textsc{OpenFCI}}
\begin{columns}[T,l]
\column{4.5cm}
\begin{center}
\rowcolors[]{1}{blue!20}{blue!10}
\begin{tabular}{l!{\vrule}||p{1.5cm}|c}
  $\lambda$ & FCI ground state energy (in unit of $\hbar \omega$) & (M,S)  \\\hline
0.1 & 11.197 & (0,0)\\
0.5 & 15.561&  (0,0)\\
1 & 20.257&  (0,0)\\
2 & 28.032&  (0,0)\\
5 & 46.482&  (0,0)\\
10 & 73.067&  (0,0)\\
11 & 78.143&  (0,0)\\
12 & 83.168&  (0,0)\\
13 & 88.152&  (0,0)\\
15 & 98.027&  \textbf{(1,2)}\\
20 & 122.325&  \textbf{(1,2)}\\
50 & 266.157&  \textbf{(1,4)}
\end{tabular}
\end{center}
\column{5.5cm}
\begin{center}
\begin{itemize}
 \item For a 6-particle QD, break of the model of a single Slater determinant  from $\lambda \simeq 13$.
\item A similar study performed on a 2-particle QD indicates a break of the closed-shell model from $\lambda \simeq 150$.

\end{itemize}

\end{center}
\end{columns}
\end{alertblock}
\end{scriptsize}
\end{frame}



\subsection{Convergence/Stability/Accuracy of HF}
%%%%%%%%%%%%%%%%%%%%%%%%%%%%%%%%%%%%%%%%%%%%%%%%%%%%%%%%%%%%%%%%%%%%%%%%%%%%%%%%%%%%%%%%%%%%%%
\begin{frame}
\frametitle{Exponential convergence of HF as a function of $R^b$ (1/3)}
\begin{scriptsize}
\begin{definition}
The size of the basis set characterized by $R^b$ ($R^b\in\mathbb{N} \; R^b\geq R^f$)

It defines the maximum shell number in the model space for our Hartree-Fock computation. It implies the number of orbitals in which each single particle wavefunction will be expanded, with spin degeneracy the number of states $N_S$ is
\begin{equation}
 N_S=(R^b+1)(R^b+2)
\end{equation}

The bigger the basis set, the more accurate the single particle wavefunction is expected. In mathematical notation, $R^b$ and the size of the basis set $\mathcal{B}$ are defined by
\begin{equation}
 \mathcal{B} = \mathcal{B}(R^b) = \left\{ | \phi_{n m_l}(\textbf{r})\rangle   \quad : 2n+ |m_l| \leq R^b  \right\},
\end{equation} 
where $| \phi_{n m_l}(\textbf{r})\rangle$ are the single orbital in the Harmonic oscillator basis with quantum numbers $n$, $m_l$ such that the single orbital energy reads: $\epsilon_{n m_l}=2n+\abs{m_l}+1$ in two-dimensions.
\end{definition}
\end{scriptsize}
\end{frame}


%%%%%%%%%%%%%%%%%%%%%%%%%%%%%%%%%%%%%%%%%%%%%%%%%%%%%%%%%%%%%%%%%%%%%%%%%%%%%%%%%%%%%%%%%%%%%%
\begin{frame}
\frametitle{Exponential convergence of HF as a function of $R^b$ (2/3)}
\begin{scriptsize}
\begin{figure}
	\begin{center}
\begin{tabular}{cc}
	\scalebox{0.35}{\input{../IMAGES/HF_Rb02new}} & 	\scalebox{0.35}{\input{../IMAGES/HF_Rb06new}}  \\
	\scalebox{0.35}{\input{../IMAGES/HF_Rb12new}}  & 	\scalebox{0.35}{\input{../IMAGES/HF_Rb20new}}  \\
	%\hline
\end{tabular}
	\end{center}
	\vspace{-5pt}
	\caption{Hartree-Fock ground state $(E^{HF}$ as a function of $R^b$ for 2-,6-electron QD (top) and 12-,20-electron QD (bottom).}
\end{figure}
\end{scriptsize}
\end{frame}


%%%%%%%%%%%%%%%%%%%%%%%%%%%%%%%%%%%%%%%%%%%%%%%%%%%%%%%%%%%%%%%%%%%%%%%%%%%%%%%%%%%%%%%%%%%%%%
\begin{frame}
\frametitle{Exponential convergence of HF as a function of $R^b$ (3/3)}
\begin{scriptsize}
\begin{figure}
	\begin{center}
\begin{tabular}{cc}
	\scalebox{0.35}{\input{../IMAGES/HF_RelErr02new.tex}} & 	\scalebox{0.35}{\input{../IMAGES/HF_RelErr06new.tex}}  \\
	\scalebox{0.35}{\input{../IMAGES/HF_RelErr12new.tex}}  & 	\scalebox{0.35}{\input{../IMAGES/HF_RelErr20new.tex}}  \\
	%\hline
\end{tabular}
	\end{center}
	\vspace{-5pt}
	\caption{Hartree-Fock relative error $(E^{HF}(R^b)-E^{HF}_{min})/E^{HF}_{min}$ as a function of $R^b$ for 2-,6-electron QD (top) and 12-,20-electron QD (bottom).}
\end{figure}
\end{scriptsize}
\end{frame}


%%%%%%%%%%%%%%%%%%%%%%%%%%%%%%%%%%%%%%%%%%%%%%%%%%%%%%%%%%%%%%%%%%%%%%%%%%%%%%%%%%%%%%%%%%%%%%
\begin{frame}
\frametitle{``Convergence history'' as a function of $\lambda$ (1/3)}
\begin{scriptsize}
\begin{definition}
The convergence history of a simulation shows how the convergence ``improve'' over iterations.
We could plot
\begin{itemize}
 \item the energy difference from one iteration to the next
\begin{equation*}
\delta (iter)= \abs{E^{HF}(iter)-E^{HF}(iter-1)}.
\end{equation*}
\item  the difference between eigenvalues from one iteration to the next (more adapted to our algorithm).
\begin{equation*}
\delta (iter)= \frac{1}{nbStates} \left( \sum_{n\,m_l} \abs{\epsilon_{n\,m_l}(iter)-\sum_{n\,m_l} \epsilon_{n\,m_l}(iter-1)} \right),
\end{equation*}
where $\epsilon_{n\,m_l}$ is the eigenenergy of the system of the single orbital $|n\,m_l \rangle$.
\item more intuitive: on the form:  $\delta (iter) \simeq 10^{- \beta \, iter}$, where $\beta$ indicates how many digits are gained from one iteration to the next.
\end{itemize}

\end{definition}
\end{scriptsize}
\end{frame}


%%%%%%%%%%%%%%%%%%%%%%%%%%%%%%%%%%%%%%%%%%%%%%%%%%%%%%%%%%%%%%%%%%%%%%%%%%%%%%%%%%%%%%%%%%%%%%
\begin{frame}
\frametitle{``Convergence history'' as a function of $\lambda$ (2/3)}
\begin{scriptsize}

\begin{columns}[T,l]
\column{7.5cm}
\begin{figure}
	\begin{flushleft}
\begin{tabular}{cc}
	\scalebox{0.25}{\input{../../qdot/RESULTS/processedData/iterb0.tikz}} & 	\scalebox{0.25}{\input{../../qdot/RESULTS/processedData/iterb1.tikz}}  \\
	\scalebox{0.25}{\input{../../qdot/RESULTS/processedData/iterb2.tikz}}  & 	\scalebox{0.25}{\input{../../qdot/RESULTS/processedData/iterb3.tikz}}  \\
	%\hline
\end{tabular}
	\end{flushleft}
	\vspace{-5pt}
	\caption{Convergence history of the Hartree-Fock iterative process  for 2-,6-electron QD (top) and 12-,20-electron QD (bottom).}
\end{figure}
\column{3.5cm}
\begin{itemize}
 \item slower convergence as $\lambda$ increases.
\item much less impact due to $R^b$ or to the nb.of particles, except when leading to unstability.
\end{itemize}
\end{columns}
\end{scriptsize}
\end{frame}


%%%%%%%%%%%%%%%%%%%%%%%%%%%%%%%%%%%%%%%%%%%%%%%%%%%%%%%%%%%%%%%%%%%%%%%%%%%%%%%%%%%%%%%%%%%%%%
\begin{frame}
\frametitle{``Convergence history'' as a function of $\lambda$ (3/3)}
\begin{scriptsize}
\begin{columns}[T,l]
\column{6cm}
\begin{flushleft}
\begin{figure}
\centering
\scalebox{0.4}{\input{../../qdot/RESULTS/processedData/iterb3zoom.tikz}}
\caption{Zoom over the limit of convergence of the Hartree-Fock iterative process for a 20-particle QD.}
\end{figure}
\end{flushleft}
\column{5cm}
\begin{itemize}
 \item  expected ``plateau'' when reaching machine precision.
\end{itemize}

However it seems that increasing the interaction strength:
\begin{itemize}
 \item  slows down the convergence process, as adding error at each iteration,
\item induces a lower accuracy, as if it could ``decrease the machine precision'',
\newline $\Rightarrow$ Phenomena maybe due to round-off error, proportional to $\lambda$, and entering the eigenvalue solver in a non-trivial way.
\end{itemize}
\end{columns}
\end{scriptsize}
\end{frame}

\subsection{Comparison on HF/MBPT/FCI calculations}
% %%%%%%%%%%%%%%%%%%%%%%%%%%%%%%%%%%%%%%%%%%%%%%%%%%%%%%%%%%%%%%%%%%%%%%%%%%%%%%%%%%%%%%%%%%%%%
\begin{frame}
\frametitle{Quadratic error growth of HF/MBPT wrt FCI ground state}
\begin{figure}
\begin{tabular}{cc}
	\scalebox{0.32}{\input{../IMAGES/comparisonMethods/allFixedAxis_02e_R8.tex}} & 	\scalebox{0.32}{\input{../IMAGES/comparisonMethods/allFixedAxis_06e_R8.tex}}  \\
	\scalebox{0.32}{\input{../IMAGES/comparisonMethods/allFixedAxis_12e_R8.tex}}  & \\
	%\hline
\end{tabular}
	\vspace{-5pt}
	\caption{Comparison of HF/MBPT and HF corrected by MBPT up to $3^{rd}$-order wrt to the FCI ground state taken as reference for 2-,6-electron QD (top) and 12-electron QD (bottom).}
\end{figure}
\end{frame}

\begin{frame}
\frametitle{Respective accuracy of HF and MBPT (1/2)}
\begin{figure}
\begin{tabular}{cc}
	\scalebox{0.32}{\input{../IMAGES/comparisonMethods/zoomQuadratic_02e_R8.tex}} & 	\scalebox{0.32}{\input{../IMAGES/comparisonMethods/zoomQuadratic_06e_R8.tex}}  \\
	\scalebox{0.32}{\input{../IMAGES/comparisonMethods/zoomQuadratic_12e_R8.tex}}  & \\
	%\hline
\end{tabular}
	\vspace{-5pt}
	\caption{Zoom on the quadratic growth of the error when $\lambda$ is relatively small ($\lambda < 0.05$), showing different accuracies with respect to the method, the number of particles and the size of the basis. for 2-,6-electron QD (top) and 12-electron QD (bottom).}
\end{figure}
\end{frame}

\begin{frame}
%%\frametitle{Respective accuracy of HF and MBPT (2/2)}
\begin{table}
\centering      % used for centering table
{\tiny
\begin{tabular}[c]{c|c|r l} 
\toprule[1pt]
\multicolumn{1}{c|}{$\sharp \; e^{-}$}  & \multicolumn{1}{c|}{Basis size ($R^b$)} & \multicolumn{2}{c}{Relative error shift between each method}\\
\hline
\multirow{8}{*}{2}&  	& MBPT(HF)-$2^{nd}$order and MBPT(HF)-$3^{rd}$order  & $\rightarrow \epsilon_{min}$ \\
		& 	&    MBPT(H0)-$2^{nd}$order and MBPT(H0)-$3^{rd}$order  & $\rightarrow 6.6 \, \epsilon_{min}$ \\ 
		& 	&    HF							& $\rightarrow 12.6 \, \epsilon_{min}$ \\ 
	& \multirow{-4}{*}{4}	&    MBPT(HO)-$1^{st}$ order		& $\rightarrow 20.7 \, \epsilon_{min}$ \\ \cline{2-4}
		&	& MBPT(HF)-$2^{nd}$order and MBPT(HF)-$3^{rd}$order  & $\rightarrow \epsilon_{min}$ \\ 
		&	&    MBPT(H0)-$2^{nd}$order and MBPT(H0)-$3^{rd}$order  & $\rightarrow 2.5 \, \epsilon_{min}$ \\ 
		&	&    HF							& $\rightarrow 8.5 \, \epsilon_{min}$ \\ 
	& \multirow{-4}{*}{8}	&    MBPT(HO)-$1^{st}$ order		& $\rightarrow 12.7 \, \epsilon_{min}$ \\ \cline{2-4}
\hline
\multirow{8}{*}{6} &	& MBPT(HF)-$2^{nd}$order and MBPT(HF)-$3^{rd}$order  & $\rightarrow \epsilon_{min}$ \\ 
		&	&    HF& $\rightarrow 9.8 \, \epsilon_{min}$ \\ 
		&	&    MBPT(H0)-$2^{nd}$order and MBPT(H0)-$3^{rd}$order  & $\rightarrow 40.9\, \epsilon_{min}$ \\ 
	& \multirow{-4}{*}{4}	&    MBPT(HO)-$1^{st}$ order		& $\rightarrow 54.1 \, \epsilon_{min}$ \\ \cline{2-4}
		&	& MBPT(HF)-$2^{nd}$order and MBPT(HF)-$3^{rd}$order  & $\rightarrow \epsilon_{min}$ \\ 
		&	&    HF& $\rightarrow 1.3 \, \epsilon_{min}$ \\ 
		&	&   MBPT(H0)-$2^{nd}$order and MBPT(H0)-$3^{rd}$order  & $\rightarrow 5.3\, \epsilon_{min}$ \\ 
	& \multirow{-4}{*}{8}	&    MBPT(HO)-$1^{st}$ order		& $\rightarrow 7.9 \, \epsilon_{min}$ \\ \cline{2-4}
\hline
\multirow{8}{*}{12}& 	& MBPT(HO)-$2^{nd}$order and MBPT(HO)-$3^{rd}$order  & $\rightarrow \epsilon_{min}$ \\ 
		&	&    MBPT(HO)-$1^{st}$ order  & $\rightarrow 1.3\, \epsilon_{min}$ \\ 
		&	&    HF							& $\rightarrow 1.8 \, \epsilon_{min}$ \\ 
	& \multirow{-4}{*}{4}	&   MBPT(HF)-$2^{nd}$order and MBPT(HF)-$3^{rd}$order & $\rightarrow 2 \, \epsilon_{min}$ \\ \cline{2-4}
		&	& MBPT(HO)-$2^{nd}$order and MBPT(HO)-$3^{rd}$order   & $\rightarrow \epsilon_{min}$ \\ 
		&	&  MBPT(HO)-$1^{st}$ order	   & $\rightarrow 1.6 \, \epsilon_{min}$ \\
		&	&    HF							& $\rightarrow 2.3 \, \epsilon_{min}$ \\
	& \multirow{-4}{*}{8}	&    MBPT(HF)-$2^{nd}$order and MBPT(HF)-$3^{rd}$order & $\rightarrow 2.8\, \epsilon_{min}$ \\ 
\toprule[1pt]
\end{tabular}
}
\vspace{-7pt}
 \caption{Classification of the methods with respect to their relative accuracy in the range of $\lambda$ that exhibits a quadractic error growth.}
\end{table} 
\end{frame}

\begin{frame}
\frametitle{Break of the methods}
\begin{figure}
\begin{tabular}{cc}
	\scalebox{0.32}{\input{../IMAGES/comparisonMethods/zoomGarbage_02e_R8.tex}} & 	\scalebox{0.32}{\input{../IMAGES/comparisonMethods/zoomGarbage_06e_R8.tex}}  \\
	\scalebox{0.32}{\input{../IMAGES/comparisonMethods/zoomGarbage_12e_R8.tex}}  & \\
	%\hline
\end{tabular}
	\vspace{-5pt}
	\caption{The plots display a zoom for $\lambda$ approaching the limit of the closed-shell model for 2-,6-electron QD (top) and 12-electron QD (bottom).}
\end{figure}
\end{frame}


\begin{frame}
\frametitle{Summary of the results}
\begin{scriptsize}
\begin{itemize}
 \item Exponential convergence of HF as a function of $R^b$.
\item Increasing $\lambda$ slows down the convergence of HF, and decreases its accuracy.
\item Compared to FCI, HF and MBPT have a quadratic error growth wrt $\lambda$.
\item Unstability of the $2^{nd}$- $3^{rd}$-order MBPT corrections before $1^{st}$-order MBPT and HF.
\end{itemize}
\begin{alertblock}{Break of the method before the closed-shell model}
\begin{center}
\rowcolors[]{1}{blue!20}{blue!10}
\begin{tabular}{p{1.5cm}|p{1.5cm}|p{1.5cm}}
  \# particles & Break of the method & Break of the model  \\\hline
2 & $\lambda \simeq$5 & $\lambda \simeq$150 \\\hline
6 & $\lambda \simeq$2 & $\lambda \simeq$14 \\\hline
12 & $\lambda \simeq$1 & $\lambda \simeq$??? \\\hline
\end{tabular}
\end{center}
\end{alertblock}
\end{scriptsize}
\end{frame}


\begin{frame}
\begin{center}
\textbf{Thank you for your attention ;)}
\end{center}
\end{frame}

%%%%%%%%%%%%%%%%%%%%%%%%%%%%%%%%%%%%%%%%%%%%%%%%%%%%%%%%%%%%%%%%%%%%%%%%%%%%%%%%%%%%%%%%%%%%%%%%%%%%%%
%%%%%%%%%%%%%%%%%%%%%%%%%%%%%%%%%%%%%%%%%%%%%%%%%%%%%%%%%%%%%%%%%%%%%%%%%%%%%%%%%%%%%%%%%%%%%%%%%%%%%%
%%%%%%%%%%%%%%%%%%%%%%%%%%%%%%%%%%%    BIBLIOGRAPHY      %%%%%%%%%%%%%%%%%%%%%%%%%%%%%%%%%%%%%%%%%%%%%
%%%%%%%%%%%%%%%%%%%%%%%%%%%%%%%%%%%%%%%%%%%%%%%%%%%%%%%%%%%%%%%%%%%%%%%%%%%%%%%%%%%%%%%%%%%%%%%%%%%%%%
%%%%%%%%%%%%%%%%%%%%%%%%%%%%%%%%%%%%%%%%%%%%%%%%%%%%%%%%%%%%%%%%%%%%%%%%%%%%%%%%%%%%%%%%%%%%%%%%%%%%%%
\begin{frame}
\begin{small}
 {\scriptsize
\beamertemplatebookbibitems
\begin{thebibliography}{10}
\bibitem{Kouwenhoven1997}
L.~P. Kouwenhoven.
{\em Electron transport in quantum dots}, Proceedings of the Advanced Study
  Institute, 1997.
\bibitem{Gao2004} X.~Gao, Y.~Cui, R.~M. Levenson, L.~W. Chung, and S.~Nie.
 In vivo cancer targeting and imaging with semiconductor quantum dots.
 {\em Nat Biotech}, 22:969--976, 2004.
\bibitem{fewElectronQDExperiment}
L.~P. Kouwenhoven, D.~G. Austing, and S.~Tarucha.
 Few-electron quantum dots.
 {\em Reports on Progress in Physics}, 64:701--736, 2001.
\bibitem{Kvaal2008}
S.~Kvaal.
 Open source fci code for quantum dots and effective interactions.
 {\em arXiv.org}, 2008.
\bibitem{SimenThesis}
S.~Kvaal.
{\em Analysis of many-body methods for quantum dots}.
PhD thesis, University of Oslo, 2009.
\bibitem{Moshinsky}
M.~Moshinsky and Y.~F. Smirnov.
 {\em The Harmonic Oscillator in Modern Physics: From Atoms to
  Quarks}.
 Taylor \& Francis, 1996.
\end{thebibliography}
%\documentclass[11pt,a4paper]{book}
\usepackage[inner=1.5in,outer=1in,bottom=1.5in]{geometry}
\usepackage{verbatim}
\usepackage{listings}
\usepackage{graphicx}
\usepackage{float}
%\usepackage{a4wide}
\usepackage{color}
\usepackage{amsmath}
\usepackage{amssymb}
%\usepackage[dvips]{epsfig}
\usepackage[T1]{fontenc}
\usepackage{cite} % [2,3,4] --> [2--4]
\usepackage{shadow}
\usepackage[hyphens]{url}
\usepackage[breaklinks]{hyperref}
\usepackage{tcolorbox}
\usepackage{caption}
\usepackage{subcaption}
\usepackage{tabularx}
%\usepackage[margin=1in]{geometry}
\usepackage{pdfpages}
\usepackage{setspace}
\usepackage[toc,page]{appendix}
\usepackage[nottoc,notlot,notlof]{tocbibind}
\usepackage{titlesec}
\newcolumntype{Y}{>{\centering\arraybackslash}X}
%\usepackage[cm]{fullpage}		% Smalere marger.
%\captionsetup[subfigure]{labelformat = parens, labelsep = space, font = small}
%\usepackage{fancyhdr}

\setcounter{tocdepth}{3}
\setcounter{secnumdepth}{3}
\setlength{\footskip}{1 in}%100pt}

\definecolor{light-gray}{gray}{0.95}
\definecolor{Ubuntu-purple}{RGB}{48, 10, 36}

\lstset{language=c++}
\lstset{alsolanguage=Python}
\lstset{basicstyle=\scriptsize}
\lstset{backgroundcolor=\color{light-gray}}
\lstset{frame=single}
\lstset{stringstyle=\ttfamily}
\lstset{keywordstyle=\color{red}\bfseries}
\lstset{commentstyle=\itshape\color{blue}}
\lstset{showspaces=false}
\lstset{showstringspaces=false}
\lstset{showtabs=false}
\lstset{breaklines}
\lstset{showlines=false}
\lstset{numbers=left}
\lstset{captionpos=b}
\usepackage[utf8]{inputenc}
%\usepackage[utf8]{inputenc}
\usepackage[english]{babel}
%\usepackage{graphicx}
\graphicspath{{images/}{../images/}}

\usepackage{sectsty}
\partfont{\centering}
\numberwithin{equation}{section}

\def\changemargin#1#2{\list{}{\rightmargin#2\leftmargin#1}\item[]}
\let\endchangemargin=\endlist 

\lstdefinestyle{Bash}
{
    language={},
    backgroundcolor=\color{Ubuntu-purple},
    basicstyle=\small\color{white}\ttfamily
}

\let\origdoublepage\cleardoublepage
\newcommand{\clearemptydoublepage}{%
  \clearpage
  {\pagestyle{empty}\origdoublepage}%
}

\let\cleardoublepage\clearemptydoublepage

\usepackage[usestackEOL]{stackengine}
 
\usepackage{subfiles}
 
\usepackage{blindtext}

%\usepackage[parfill]{parskip}
 
%\title{Thesis}
%\author{Christian Fleischer}
%\date{ }
 
\begin{document}
 
\includepdf{sections/front-page.pdf}
\clearemptydoublepage
%\maketitle

%\chapter*{Abstract}
\frontmatter
\vspace*{\fill}
\thispagestyle{plain}
\begin{center}
\textbf{Abstract}
\end{center}
\subfile{sections/abstract}
\vspace*{\fill}

\chapter*{Preface}
\subfile{sections/preface}

\doublespacing
\tableofcontents
\singlespacing
 
\mainmatter
\chapter{Introduction}
\subfile{sections/introduction}
 

\part{Theory} 
\subfile{sections/theory}
 
\part{Implementation}
\subfile{sections/implementation}

\part{Results}
\subfile{sections/results}

\chapter{Conclusions}\label{sec: Conclusions}
\subfile{sections/conclusion}

\subfile{sections/appendix}

%\chapter{Bibliography}
\newpage\thispagestyle{empty}
\begin{changemargin}{0 cm}{4 cm}
\subfile{sections/bibliography}
\end{changemargin}

 
\end{document}

}
 \end{small}
\end{frame}

\begin{frame}
\begin{small}
 {\scriptsize
\beamertemplatebookbibitems
\begin{thebibliography}{10}
\bibitem{pdgvmp1993}
D.~Pfannkuche, V.~Gudmundsson, and P.~A. Maksym.
 Comparison of a hartree, a hartree-fock, and an exact treatment of
  quantum-dot helium.
 {\em Phys. Rev. B}, 47:2244--2250, 1993.
\bibitem{Raimes1972}
S.~Raimes.
 {\em Many-electron Theory}.
 North-Holland Publishing, 1972.
\bibitem{Waltersson2007}
E.~Waltersson and E.~Lindroth.
 Many-body perturbation theory calculations on circular quantum dots.
 {\em Phys. Rev. B}, 76:045314, 2007.
\bibitem{Winter2004}
J.~O. Winter.
 {\em Development and optimization of quantum dot-neuron interfaces}.
 PhD thesis, The University of Texas at Austin, 2004.
\end{thebibliography}
%\documentclass[11pt,a4paper]{book}
\usepackage[inner=1.5in,outer=1in,bottom=1.5in]{geometry}
\usepackage{verbatim}
\usepackage{listings}
\usepackage{graphicx}
\usepackage{float}
%\usepackage{a4wide}
\usepackage{color}
\usepackage{amsmath}
\usepackage{amssymb}
%\usepackage[dvips]{epsfig}
\usepackage[T1]{fontenc}
\usepackage{cite} % [2,3,4] --> [2--4]
\usepackage{shadow}
\usepackage[hyphens]{url}
\usepackage[breaklinks]{hyperref}
\usepackage{tcolorbox}
\usepackage{caption}
\usepackage{subcaption}
\usepackage{tabularx}
%\usepackage[margin=1in]{geometry}
\usepackage{pdfpages}
\usepackage{setspace}
\usepackage[toc,page]{appendix}
\usepackage[nottoc,notlot,notlof]{tocbibind}
\usepackage{titlesec}
\newcolumntype{Y}{>{\centering\arraybackslash}X}
%\usepackage[cm]{fullpage}		% Smalere marger.
%\captionsetup[subfigure]{labelformat = parens, labelsep = space, font = small}
%\usepackage{fancyhdr}

\setcounter{tocdepth}{3}
\setcounter{secnumdepth}{3}
\setlength{\footskip}{1 in}%100pt}

\definecolor{light-gray}{gray}{0.95}
\definecolor{Ubuntu-purple}{RGB}{48, 10, 36}

\lstset{language=c++}
\lstset{alsolanguage=Python}
\lstset{basicstyle=\scriptsize}
\lstset{backgroundcolor=\color{light-gray}}
\lstset{frame=single}
\lstset{stringstyle=\ttfamily}
\lstset{keywordstyle=\color{red}\bfseries}
\lstset{commentstyle=\itshape\color{blue}}
\lstset{showspaces=false}
\lstset{showstringspaces=false}
\lstset{showtabs=false}
\lstset{breaklines}
\lstset{showlines=false}
\lstset{numbers=left}
\lstset{captionpos=b}
\usepackage[utf8]{inputenc}
%\usepackage[utf8]{inputenc}
\usepackage[english]{babel}
%\usepackage{graphicx}
\graphicspath{{images/}{../images/}}

\usepackage{sectsty}
\partfont{\centering}
\numberwithin{equation}{section}

\def\changemargin#1#2{\list{}{\rightmargin#2\leftmargin#1}\item[]}
\let\endchangemargin=\endlist 

\lstdefinestyle{Bash}
{
    language={},
    backgroundcolor=\color{Ubuntu-purple},
    basicstyle=\small\color{white}\ttfamily
}

\let\origdoublepage\cleardoublepage
\newcommand{\clearemptydoublepage}{%
  \clearpage
  {\pagestyle{empty}\origdoublepage}%
}

\let\cleardoublepage\clearemptydoublepage

\usepackage[usestackEOL]{stackengine}
 
\usepackage{subfiles}
 
\usepackage{blindtext}

%\usepackage[parfill]{parskip}
 
%\title{Thesis}
%\author{Christian Fleischer}
%\date{ }
 
\begin{document}
 
\includepdf{sections/front-page.pdf}
\clearemptydoublepage
%\maketitle

%\chapter*{Abstract}
\frontmatter
\vspace*{\fill}
\thispagestyle{plain}
\begin{center}
\textbf{Abstract}
\end{center}
\subfile{sections/abstract}
\vspace*{\fill}

\chapter*{Preface}
\subfile{sections/preface}

\doublespacing
\tableofcontents
\singlespacing
 
\mainmatter
\chapter{Introduction}
\subfile{sections/introduction}
 

\part{Theory} 
\subfile{sections/theory}
 
\part{Implementation}
\subfile{sections/implementation}

\part{Results}
\subfile{sections/results}

\chapter{Conclusions}\label{sec: Conclusions}
\subfile{sections/conclusion}

\subfile{sections/appendix}

%\chapter{Bibliography}
\newpage\thispagestyle{empty}
\begin{changemargin}{0 cm}{4 cm}
\subfile{sections/bibliography}
\end{changemargin}

 
\end{document}

}
 \end{small}
\end{frame}
\end{document}
