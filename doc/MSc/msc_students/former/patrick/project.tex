\documentstyle[a4wide]{article}
\newcommand{\OP}[1]{{\bf\widehat{#1}}}

\newcommand{\be}{\begin{equation}}

\newcommand{\ee}{\end{equation}}

\begin{document}

\pagestyle{plain}

\section*{Thesis title: Hartree-Fock studies of quantum dots}

{\bf The aim of this thesis is to study numerically systems consisting of several
interacting electrons in two dimensions and three dimensions}, confined to small regions
between layers of semiconductors. 
These electron systems
are dubbed quantum dots in the literature. 
Semiconductor quantum dots are structures where
charge carriers are confined in all three spatial dimensions, 
the dot size being of the order of the Fermi wavelength 
in the host material, typically between  10 nm and  1 $\mu$m.
The confinement is usually achieved by electrical gating of a 
two-dimensional electron gas (2DEG), 
possibly combined with etching techniques. Precise control of the
number of electrons in the conduction band of a quantum dot 
(starting from zero) has been achieved in GaAs heterostructures. 
The electronic spectrum of typical quantum dots
can vary strongly when an external magnetic field is applied, 
since the magnetic length corresponding to typical 
laboratory fields  is comparable to typical dot sizes.
In coupled quantum dots Coulomb blockade effects, 
tunneling between neighboring dots, and magnetization 
have been observed as well as the formation of a
delocalized single-particle state. 

Quantum dots have been used to fabricate  quantum gates and are also used in the emerging field
of quantum nano medicine.  

More specifically, this thesis aims at studying the reliability of the Hartree-Fock method for studies of quantum
dots in two and three-dimensions. The latter entails the development of a non-spherical Hartree-Fock code.
The emphasis of this thesis rests however on studies of two-dimensional systems.
The reliability of the Hartree-Fock method as function of a the externally applied magnetic field
will be compared with ab initio Monte Carlo and large-scale diagonalization techniques. The latter
two calculations will be performed by other Master of science students.

With the Hartree-Fock results, one can in turn study the time-development of electrons in a quantum dot  under the influence
of a time-dependent external field.  The final aim  is to link these studies with possible implementations in nano medicine.


The aims of this thesis are as follows

\begin{itemize}
\item Develop a Hartree-Fock code for electrons trapped in a single harmonic oscillator trap in two dimensions.
\item Compare the results with those obtained using variational Monte Carlo methods and large scale diagonalization techniques.
\item Extend the two-dimensional calculations to two coupled harmonic oscillator wells.
 \item Study the three-dimensional quantum dot (one single well) 
with cylindrical symmetry with the $x$ and $y$ axis kept as before
but letting the $z$-axis vary. These results are in turn compared with variational Monte Carlo
calculations.
\item Perform the Hartree-Fock calculations with a time-dependent external field and study the temporal development of
the ground state.
\end{itemize}
 


The thesis is expected to be finished towards the end  of the spring
semester of 2009.



\end{document}
