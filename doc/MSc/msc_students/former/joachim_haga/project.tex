\documentstyle[a4wide,11pt]{article}

\begin{document}

\pagestyle{plain}

\begin{center} \huge \bf Master of Science Project for Carl Joachim Berdal Haga 
\end{center}

\section*{Bose-Einstein condensation in trapped bosons: A mathematical analysis of the 
non-linear Schr\"odinger equation.}




The spectacular demonstration of Bose-Einstein condensation (BEC) in gases of
alkali atoms $^{87}$Rb, $^{23}$Na, $^7$Li confined in magnetic
traps\cite{anderson95,davis95,bradley95} has led to an explosion of interest in
confined Bose systems. Of interest is the fraction of condensed atoms, the
nature of the condensate, the excitations above the condensate, the atomic
density in the trap as a function of Temperature and the critical temperature of BEC,
$T_c$. The extensive progress made up to early 1999 is reviewed by Dalfovo et
al.\cite{dalfovo99}. 

A key feature of the trapped alkali and atomic hydrogen systems is that they are
dilute. The characteristic dimensions of a typical trap for $^{87}$Rb is
$a_{h0}=\left( {\hbar}/{m\omega_\perp}\right)^\frac{1}{2}=1-2 \times 10^4$
\AA\ (Ref. 1). The interaction between $^{87}$Rb atoms can be well represented
by its s-wave scattering length, $a_{Rb}$. This scattering length lies in the
range $85 < a_{Rb} < 140 a_0$ where $a_0 = 0.5292$ \AA\ is the Bohr radius.
The definite value $a_{Rb} = 100 a_0$ is usually selected and
for calculations the definite ratio of atom size to trap size 
$a_{Rb}/a_{h0} = 4.33 \times 10^{-3}$ 
is usually chosen \cite{dalfovo99}. A typical $^{87}$Rb atom
density in the trap is $n \simeq 10^{12}- 10^{14}$ atoms/cm$^3$ giving an
inter-atom spacing $\ell \simeq 10^4$ \AA. Thus the effective atom size is small
compared to both the trap size and the inter-atom spacing, the condition
for diluteness (i.e., $na^3_{Rb} \simeq 10^{-6}$ where $n = N/V$ is the number
density). In this limit,
although the interaction is important, dilute gas approximations such as the
Bogoliubov theory\cite{bogoliubov47}, valid for small $na^3$ and large
condensate fraction $n_0 = N_0/N$, describe the system well. Also, since most
of the atoms are in the condensate (except near $T_c$), the Gross-Pitaevskii 
equation\cite{gross61,pitaevskii61} for the condensate describes the whole gas
well. 
The latter approach leads to a Schr\"odinger-like equation with non-linear terms.
The time-independent  Gross-Pitaevskii (GP) equation reads 
\begin{equation}
  \left [ - \frac {\hbar^2}{2m} \nabla^2 + V_{\mathrm{trap}}({\bf r}) + 
   \frac{4\pi\hbar^2 a}{m} \mid \Psi \mid^2 \right ]\Psi=\mu\Psi , 
\label{gp1} 
\end{equation}
where $\mu$ is the chemical potential and 
\begin{equation}
  V_{\mathrm{trap}}({\bf r}) = \frac {1}{2} m (\omega_{\bot}^2 x^2 
  + \omega_{\bot}^2 y^2 +\omega_{z}^2 z^2 ) 
\label{trap}
\end{equation}
is the confining potential defined by the two angular frequencies
$\omega_{\bot}$ and $\omega_{z}$ associated with the external
potential of the anisotropic trap. The wave function $\Psi$ is
normalized to the total number of particles. All particles are assumed to be
in the condensate. The constant $a$ is the so-called $s$-wave scattering length.
A modified version of this equation which can be used for higher densities,
see for example \cite{jon2005,fabro99}, is the so-called 
modified GP equation (MGP), 
\begin{equation}
  \left [ - \frac {\hbar^2}{2m} \nabla^2 + V_{\mathrm{trap}}({\bf r})+\frac {4 \pi \hbar^2 a}{m} \mid \Psi \mid^2 
    \left (1 + \frac {32 a^{3/2}}{3 \pi^{1/2}} \mid \Psi\mid \right)
    \right ] \Psi =  \mu \Psi .
\label{gp2}
\end{equation}
The modified GP equation has been used in Refs.~\cite{jon2005,fabro99} to estimate the 
corrections to the GP equation. For the case of a
deformed cylindrical trap, the  scattering lengths used 
are those from the first JILA experiments. These experiments took 
advantage of the  presence of a Feshbach resonance in the collision of two
$^{85}$Rb atoms to tune their scattering length \cite{dalfovo99}.
Fully microscopic calculations using  a hard-spheres interaction have
also been performed in the framework of Variational and Diffusion Monte
Carlo methods \cite{jon2005,glyde1,glyde2,glyde3,blume1}. 

The purpose of this Master of Science thesis
is to investigate various numerical methods for solving
the time-independent partial differential Schr\"odinger equation with non-linear terms, 
study their stability and applicability with tools like Diffpack \cite{hpl}.
If time allows, one can eventually move to the time-dependent version of the GP equation.



\section*{Progress Plan}
\begin{itemize}
\item Spring 2005: Follow the course INF5610, Partial Differential Equations. Reformulate
the GP and modified GP equations as non-linear eigenvalue problems. 
Study the steepest descent method as exposed in Ref.~\cite{flo80} 
and try to implement it.
Write a one-dimensional version of the GP equation using Diffpack. 
\item Fall 2005: Extend the calculations to the full GP and modified GP equations
using Diffpack. Study Krylov methods for non-linear eigenvalue problems.
If possible study also the feasibility of Jacobi-Davidson projection methods for
non-linear eigenvalue problems. 
\item Spring 2006: Finalize and write up thesis. 
\end{itemize}



  \begin{thebibliography}{999}%\parskip=0pt\itemsep 0pt%
  \footnotesize
  \bibitem {anderson95}%1
  M.H. Anderson, J.R. Ensher, M.R. Matthews, C.E. Wieman, and E.A. Cornell, 
   Science {\bf 269}, 198 (1995).

  \bibitem{davis95}%2
  K.B. Davis, M.-O. Mewes, M.R. Andrews, N.J. van Druten, D.S. Durfee, D.M. Kurn, and W. Ketterle,
  Phys. Rev. Lett. {\bf 75}, 3969 (1995).

  \bibitem{bradley95}%3
  C.C. Bradley, C.A. Sackett, J.J. Tolett, and R.G. Hulet, Phys. Rev. Lett. {\bf 75}, 1687
  (1995); C.C. Bradley, C.A. Sackett, and R.G. Hulet, {\em ibid.} {\bf 78}, 985 (1997).

  \bibitem{dalfovo99}%4
  F. Dalfovo, S. Giorgini, L. Pitaevskii, and S. Stringari, Rev. Mod. Phys. {\bf 71}, 463
  (1999).

  \bibitem{bogoliubov47}%5
  N.N. Bogoliubov, J. Phys. (Moscow) {\bf 11}, 23 (1947).

  \bibitem{gross61}%6
  E.P. Gross, Nuovo Cimento {\bf 20}, 454 (1961).

  \bibitem{pitaevskii61}%7
  L.P. Pitaevskii, Ah. Eksp. Teor. Fiz. {\bf 40}, 646 (1961) [Sov. Phys. JETP {\bf 13},
  451 (1961)].

\bibitem{jon2005} J.~K.~Nilsen, J.~Mur-Petit,
  M.~Guilleumas, M.~Hjorth-Jensen, 
  and A.~Polls, submitted to Phys.~Rev.~A.

\bibitem{fabro99} A.~Fabrocini and A.~Polls, Phys.~Rev.~A {\bf 60},
  2319 (1999). 


\bibitem{glyde1} J.~L.~Dubois and H.~R.~Glyde, Phys.~Rev.~A {\bf 63},
  023602 (2001). 

\bibitem{glyde2} A.~R.~Sakhel, J.~L.~Dubois, and H.~R.~Glyde,
  Phys.~Rev.~A {\bf 66}, 063610 (2002). 

\bibitem{glyde3} J.~L.~Dubois and H.~R.~Glyde, Phys.~Rev.~A {\bf 68},
  033602 (2003).

\bibitem{blume1} D.~Blume and C.~H.~Greene, Phys.~Rev.~A {\bf 63},
  063601 (2001).

\bibitem{hpl} H.~P.~Langtangen, {\em Computational Partial Differential Equations},
(Springer, Berlin, 1999).

\bibitem{flo80} K.~T.~R.~Davies, H.~Flocard, S.~Krieger, and M.~S.~Weis,
 Nucl.~Phys.~A {\bf 342}, 111 (1980).

\end{thebibliography}


\end{document}










