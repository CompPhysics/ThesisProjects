\documentclass{article}
\usepackage{graphicx} % Required for inserting images
\usepackage{hyperref}

\title{Master Thesis}
\author{Odin Johansen}
\date{January 2024}

\begin{document}

\maketitle  

Master Thesis - Project Description
Tentative title: "Bridging Machine Learning and Quantum Physics: A Comparative Study of PINNs and Finite Element Methods in Multi-Particle Quantum Systems and Classical Fluid Dynamics"

\section{Background}

This thesis explores the pioneering intersection of machine learning,
quantum physics, and classical fluid dynamics, focusing on the
application of Physics-Informed Neural Networks (PINNs) in the
modeling of quantum systems and classical fluid dynamics governed by
the Navier-Stokes and diffusion equations, see Refs.~[1,2] below.  Traditional approaches,
primarily using partial differential equations (PDEs) solved by finite
element methods and finite difference methods, face significant
challenges in terms of computational complexity and scalability when
dealing with complex systems.  PINNs represent an approach that combines
deep learning with physics-based modeling and proposes a solution to these
challenges. This research aims to embark on a comprehensive
comparative study of PINNs, finite element methods, and finite
difference methods, starting with classical systems and progressively
advancing to complex quantum mechanical models.

\section{Main objectives}

The primary goal of this research is to conduct an in-depth
comparative analysis of the PINN approach against classical finite
element and finite difference models in both simple and more complex
quantum mechanical systems, with the quantum mechanical system of quantum dots as the
our final test case.  This includes evaluating the effectiveness, computational
efficiency, and scalability of PINNs in accurately modeling these
systems, thereby contributing new insights into the potential fusion
of machine learning techniques with traditional computational physics
methods.

Finite element and boundary element methods have a wide range of
applicability in quantum mechanical systems, in particular in
applications to two-dimensional and three-dimensional systems in
condended matter physics, see for example Ref.~[3] below here.  The
final goal here is to apply PINNs and finite element methods to
systems of two to three electrons trapped in harmonic oscilator-like.
potentials, so-called quantum dots. Such systems have been extensively studied by our research
group in Computational Physics, spanning from standard many-body
approaches like configuration interaction theory to various Monte
Carlo methods and machine learning algorithms, see for example Refs.~[4-7] below.

Quantum dot systems play also a central role in actually implementing
quantum operations and are seen as potential candidate systems for
quantum technologies. Studies of the time-evolution of entangled
quantum dot systems plays a central role in our basic understanding
for making various quantum gates.
The aim here is thus to develop a time-dependent machine learning
approach using PINNs for quantum mechanical systems, with the main focus on quantum dots systems.
The results from this thesis will be compared with our more traditional time-dependent many-body studies developed by members of the Computational Physics research group.

The candidate has already a good background in machine learning and
the solution of partial differential equations with both the finite
element and the finite difference methods. Similalry, her has already
made himself familiar with the solution of the Navier-Stokes equations
with PINNs. The solution of the Navier-Stokes equations will serve
mainly as the entry point for these studies, which span from classical systems to quantum mechanical systems.


\section{Project Tasks and Tentative Timeline}

\begin{enumerate}
\item Spring 2024: Study the application of PINNs to solve the Navier-Stokes and diffusion equations, comparing the results with those obtained using finite difference and finite element methods.
\item Fall 2024: Implement and validate a PINN-based model for Schrodinger's equation in a simple harmonic oscillator. The implementation will be benchmarked against existing finite element solutions to:
  \begin{enumerate}
   \item Verify the accuracy and reliability of the PINN model.
   \item Assess the computational efficiency and accuracy trade-offs compared to traditional methods.
  \end{enumerate}
\item Fall 2024: Systematically increase the complexity of the modeled systems, applying the PINNs methodology to increasingly complex quantum scenarios and classical fluid dynamics problems.
\item Spring 2025: Perform a detailed comparative analysis of inverse PINN solutions, finite difference, and conventional finite element methods, focusing on performance metrics in  quantum dot systems and compare these to results from other many-body theories.
\item Spring 2025: Finalize thesis with deadline medio May 2025.
\end{enumerate}

\section{References}
\begin{enumerate}
    \item Brunton, S. L., Kutz, J. N. (2023). Data-Driven Science and Engineering: Machine Learning, Dynamical Systems, and Control. University of Washington. Available at: \url{https://faculty.washington.edu/sbrunton/DataBookV2.pdf}
    \item Kutz, J. N., Brunton, S. L. (2023). Data-Driven Discovery: A New Era of Exploiting Data in the Sciences. University of Washington. Available at: \url{https://databookuw.com/databook.pdf}
    \item L. Ramdas Ram-Mohan, Finite Element and Boundary Element Applications in Quantum Mechanics, Oxfor, 2002
    \item Even Nordhagen et al.,Efficient solutions of fermionic systems using artificial neural networks, Frontiers in Physics, Volume 11, 2023, \url{https://doi.org/10.3389/fphy.2023.1061580}
    \item Bryce Force et al.,  Dilute neutron star matter from neural-network quantum states, Physical Review Research 5, 033062 (2023)
    \item Jane Kim et al., Neural-network quantum states for ultra-cold Fermi gases, Nature Physics, in press
    \item Niyaz Beysengulov et al., Coulomb interaction-driven entanglement of electrons on helium, PRX Quantum, under review
\end{enumerate}

\end{document}



