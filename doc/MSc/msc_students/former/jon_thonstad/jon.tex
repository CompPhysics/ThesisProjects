\documentstyle[a4wide]{article}
\newcommand{\OP}[1]{{\bf\widehat{#1}}}

\newcommand{\be}{\begin{equation}}

\newcommand{\ee}{\end{equation}}
\newcommand{\bra}[1]{\left\langle #1 \right|}
\newcommand{\ket}[1]{\left| #1 \right\rangle}
\newcommand{\braket}[2]{\left\langle #1 \right| #2 \right\rangle}


\begin{document}

\pagestyle{plain}

\section*{Thesis title:  Quantum Monte Carlo Approaches to Dense Nuclear and Neutron Star Matter }


{\bf The aim of this thesis is to study numerically systems such as 
dense nuclear matter and neutron star matter  using  different models for the nucleon-nucleon interaction.
Variational Monte Carlo (VMC) and 
Diffusion Monte Carlo (DMC) methods will be used for solving 
Schr\"odinger's equation. }

A microscopic treatment of 
problems in nuclear physics involves a highly complex and state dependent
interaction. For systems of few nucleons (protons and neutrons) in a light nucleus or in a lattice 
which defines the density, ab initio
calculations can be performed. In this thesis the aim is to use 
VMC and DMC approaches. These methods are briefly described below.
A general reference to neutron star and dense matter studies can be found in Ref.~[1].
Monte Carlo applications to dense matter are in Ref.~[2], while Refs.~[3,4] discuss implementations and 
algorithms for atoms and quantum liquids.
\subsection*{Quantum Monte Carlo}
The variational quantum Monte Carlo (VMC) has been widely applied 
to studies of quantal systems. 
The recipe consists in choosing 
a trial wave function
$\psi_T({\bf R})$ which we assume to be as realistic as possible. 
The variable ${\bf R}$ stands for the spatial coordinates, in total 
$dN$ if we have $N$ particles present. The variable $d$ is the dimension
of the system. 
The trial wave function serves then as
a mean to define the quantal probability distribution 
\be
   P({\bf R})= \frac{\left|\psi_T({\bf R})\right|^2}{\int \left|\psi_T({\bf R})\right|^2d{\bf R}}.
\ee
This is our new probability distribution function  (PDF). 

The expectation value of the energy $E$
is given by
\be
   \langle E \rangle =
   \frac{\int d{\bf R}\Psi^{\ast}({\bf R})H({\bf R})\Psi({\bf R})}
        {\int d{\bf R}\Psi^{\ast}({\bf R})\Psi({\bf R})},
\ee
where $\Psi$ is the exact eigenfunction. Using our trial
wave function we define a new operator, 
the so-called  
local energy, 
\be
   E_L({\bf R})=\frac{1}{\psi_T({\bf R})}H\psi_T({\bf R}),
   \label{eq:locale1}
\ee
which, together with our trial PDF allows us to rewrite the 
expression for the energy as
\be
  \langle H \rangle =\int P({\bf R})E_L({\bf R}) d{\bf R}.
  \label{eq:vmc1}
\ee
This equation expresses the variational Monte Carlo approach.
For most hamiltonians, $H$ is a sum of kinetic energy, involving 
a second derivative, and a momentum independent potential. 
The contribution from the potential term is hence just the 
numerical value of the potential.
A good discussion of the VMC approach for nuclear systems can be found 
in Ref.~[1].
\subsection*{Diffusion Monte Carlo}
The DMC method is based on rewriting the 
Schr\"odinger equation in imaginary time, by defining
$\tau=it$. The imaginary time Schr\"odinger equation is then
\be
   \frac{\partial \psi}{\partial \tau}=-\OP{H}\psi,
\ee
where we have omitted the dependence on $\tau$ and the spatial variables
in $\psi$.
The wave function $\psi$  is again expanded in eigenstates of the Hamiltonian 
\be
    \psi = \sum_i^{\infty}c_i\phi_i,
    \label{eq:wexpansion_dmc}
\ee 
where 
\be 
   \OP{H}\phi_i=\epsilon_i\phi_i, 
\ee 
$\epsilon_i$ being an eigenstate of $\OP{H}$. 

In order to solve the above equations, 
importance sampling is essential for DMC methods, 
if the simulation is to be efficient. 
A trial or guiding wave function $\psi_T({\bf R})$, which closely
approximates the ground state wave function is introduced.
This is where typically the VMC result enters.
A new distribution is defined as 
\be
   f({\bf R}, \tau)=\psi_T({\bf R})\psi({\bf R}, \tau),
\ee
which is also a solution of the Schr\"odinger equation when  
$\psi({\bf R}, \tau)$ 
is a solution. 
Schr\"odinger's equation is consequently modified to
\be
   \frac{\partial f({\bf R}, \tau)}{\partial \tau}=
    \frac{1}{2}\nabla\left[\nabla -F({\bf R})\right]f({\bf R}, \tau)
    +(E_L({\bf R})-E_T)f({\bf R}, \tau).
    \label{eq:dmcequation2}
\ee
In this equation we have introduced the so-called force-term $F$,
given by
\be
   F({\bf R})=\frac{2\nabla \psi_T({\bf R})}{ \psi_T({\bf R})},
\ee
and is commonly referred to as the ``quantum force''. 
The local energy $E_L$ is defined as previously
\be
    E_L{\bf R})=-\frac{1}{\psi_T({\bf R})}
                \frac{\nabla^2 \psi_T({\bf R})}{2}+V({\bf R})\psi_T({\bf R}),
\ee
and is computed, as in the VMC method, with respect to the trial wave function
while $E_T$ is the trial energy.

We can give the following interpretation to Eq.~(\ref{eq:dmcequation2}).
The right hand side of the importance sampled DMC equation
consists, from left to right, of diffusion, drift and rate terms. The
problematic potential dependent rate term of the non-importance 
sampled method is replaced by a term dependent on the difference 
between the local
energy of the guiding wave function and the trial energy. 
The trial energy is initially chosen to be the VMC energy of 
the trial  wave function, and is
updated as the simulation progresses. Use of an optimised 
trial function minimises the difference between the local 
and trial energies, and hence
minimises fluctuations in the distribution $f$ . 
A wave function optimised using VMC is ideal for this purpose, 
and in practice VMC provides the best
method for obtaining wave functions that accurately 
approximate ground state wave functions locally. 


A large program for VMC for both fermions and bosons has been developed by Mateusz R\o stad,
Simen Reine, Jon Nilsen and Victoria Popsueva, all PhD students at present.
Similarly, we have at present a DMC program for bosons planned extended to fermion systems.
The fixed node approach will be used for studying fermionic systems or excited states
of bosonic systems.
This forms part of the present thesis with applications to dense matter of 
quantum liquids, with emphasis on
neutron star studies. The DMC work of this thesis can easily be extended to other fermionic 
systems such as quantum dots, atoms and nuclei. Similarly,  excited states 

\section*{Preliminary Progress Plan}
The thesis is expected to be finished towards the end  of the spring
semester of 2006.
\begin{itemize}
\item Fall 2004: 
      Exams in FYS4520 (Subatomic Many-Body Physics I), FYS4170 (Relativistic Quantum Field Theory) and  INF-MAT4350	Numerical Linear Algebra.
Thesis work: Write  a code for VMC and DMC Monte Carlo for the alpha particle and possibly $^{16}$O. 
This means partly reproducing 
results from Mateusz R\o stad's thesis, see Ref.~[5]. The same simplified nucleon-nucleon interaction
is to be used. 
The program developed by Mateusz can be used and extended upon. This work can also be done together
with Andreas S\ae bj\o rnsen.
\item Spring 2005:  Exams in FYS4530	Subatomic Many-Body Physics II, FYS5120
Advanced Quantum Field theory and INF5620 Numerical methods for partial differential equations.
Thesis work: Write a code for VMC and DMC Monte Carlo for liquid Helium following the lecture notes
of Guardiola, see Ref.~[4]. The same interaction as that used in Ref.~[4] is to be used.
The thesis of Jon Nilsen and his codes for Bosonic DMC can be used, see 
Ref.~[6].

\item Fall 2005: The code developed during the spring 2005 semester is now to be extended to dense
nuclear systems, following much of the same recipe with periodic lattices defining the density of
the system. The main aim is to compute the binding energy of the system. Pairing correlations, as done in
Ref.~[7] can also be extracted if time allows for. Ref.~[7] is a very useful work.
\item Spring 2006: Finalize thesis and final exam.
\end{itemize}

\section*{Literature}
\begin{enumerate}
\item H.~Heiselberg and M.~Hjorth-Jensen, Phys.~Rep.~{\bf 328} (2000) 237
\item A.~Sarsa, S.~Fantoni, K.~E.~Schmidt, F.~Pederiva, Phys.~Rev.~{\bf C68} (2003) 024308
\item B. L. Reynolds, W. A. Lester Jr., and P. J.~Reynolds, {\em Monte Carlo Methods in Ab Initio Quantum Cheistry}, (World Scientific, Singapore, 1994). 
\item R. Guardiola, Bolivian Summer School and Valencia Summer School Lecture notes.
\item M.~M.~R\o stad, Master Thesis, UiO august 2004.
\item J.~K.~Nilsen, Master Thesis, UiO june 2004.
\item  S.~Y.~Chang, J.~Morales, Jr., V.~R.~Pandharipande, D.~G.~Ravenhall, J.~Carlson, Steven C.~Pieper, R. B. Wiringa, K.~E.~Schmidt, preprint nucl-th/0401016, see xxx.lanl.gov and look at nuclear theory, nucl-th
\end{enumerate}

\end{document}





