\newcommand{\missingref}{\textcolor{red}{[ref]}}
\newcommand{\fakeref}[1]{\textcolor{blue}{[#1]}}
\newcommand{\comm}[1]{\textcolor{teal}{#1}}
\chapter{Introduction}

As the name suggests, many-body physics is the study of systems involving more than one particle, which describes most systems in nature. While the solution of the Schrödinger equation for one particle might not be difficult, the many-body problem often cannot be decomposed into smaller sub-problems. Hence, the reductionism approach does not help physicists solve the Schrödinger equation for a many-body problem. Notoriously, for a system consisting of  $N$  particles, analytical solutions to the Schrödinger equation generally do not exist for $ N \geq 2 $, with a few exceptions such as the Calogero-Sutherland model~\cite{Pasquier}. Numerically solving the ground state of a Hamiltonian is QMA-hard, which is believed to be harder than NP-complete problems~\cite{landau2015polynomial}. Almost a century ago, the Hartree-Fock theory was developed~\cite{BLINDER20191} to reduce the complexity of these problems, and it still remains a benchmark for other many-body methods today. Methods developed later are regarded as “Post-Hartree-Fock” methods, including the Coupled-Cluster method~\cite{Shavitt2009} and Configuration-Interaction theory~\cite{TOWNSEND201963}. Full Configuration-Interaction theory is considered the “exact” solution within the given basis functions, but it is computationally expensive and can only be performed on a limited set of basis states~\cite{TOWNSEND201963}.

Physicist Richard Feynman envisioned the possibility of simulating a quantum system with a new kind of computer—quantum computers~\cite{preskill2023}. A few years later, David Deutsch showed that a “Universal Quantum Computer” can perfectly simulate every finitely realizable physical system~\cite{deutsch1985}. The Quantum Phase Estimation (QPE) algorithm was first proposed to estimate the eigenvalue of unitary operators~\cite{Nielsen_Chuang_2010}. However, the state-of-the-art technology today is still far from the universal quantum computer, and we are in the so-called Noisy Intermediate-Scale Quantum (NISQ) era~\cite{Bharti_2022}. The lengthy circuit that the QPE requires would not be realistic to run on today's quantum computers.

Fast forward forty years from when Feynman gave the talk, and today the field of quantum computation provides researchers with a different approach to many-body problems and other fields of science~\cite{Bharti_2022}. One solution that harnesses the power of both classical and quantum computers is the Variational Quantum Eigensolver (VQE)~\cite{peruzzo2014}. This hybrid algorithm ensures that the quantum circuits being executed are of reasonable length and has shown promising results in calculations of electronic structures in quantum chemistry. The parallels in electronic structure and nuclear structure have led to the application of the VQE algorithm in the latter.

The ansatz in VQE is flexible, but it is key to whether the correct ground state energy can be found efficiently. It is an active area of research to find the best ansatz for a given problem~\cite{park2024}. In the past few years, a variant of the VQE, the Adaptive, Problem-Tailored VQE (ADAPT-VQE), was invented with the goal of constructing ansatzes that contain information about the Hamiltonian of interest and are insensitive to the famous barren plateau problem~\cite{grimsley2019, Grimsley_2023}. It has been shown that ADAPT-VQE produces better results than the VQE algorithm with the Unitary Coupled-Cluster ansatz for molecular calculations~\cite{grimsley2019} and the nuclear shell model~\cite{P_rez_Obiol_2023}, inspiring the application of ADAPT-VQE to other models in nuclear physics.

The thesis aims to develop a quantum computing library named Quanthon that is suitable for physicists without a background in quantum chemistry. It is lightweight, easily adaptable, and has a shallow learning curve. It contains basic functionalities for quantum computation and focuses on modules that facilitate quantum simulations, specifically those involving solving the ground state energy of a Hamiltonian using VQE and ADAPT-VQE. In the second part of the thesis, we will utilize the developed library to perform quantum simulations on multiple simple physical systems and observe the behaviour of different algorithms under various circumstances.

In this thesis, we will first introduce the basic definitions and theories of quantum mechanics and quantum computing in Part~\ref{part:theory}. After that, we will explain the detailed implementation of our quantum computing library, Quanthon, in Chapter~\ref{ch:quanthon}. This will be followed by an introduction to the physical systems we will be running quantum simulations on in Chapter~\ref{chap:physical_systems}. Then, we will discuss the application of VQE in many-body physics for finding the ground state energy of a system in models of nuclear physics and simple electronic structures. We will constrain ourselves to running classical simulations of an ideal quantum computer for most of the discussion while keeping in mind that “In theory, there is no difference between theory and practice, but in practice there is.”\footnote{I will not put a citation to this as it is unclear who said it.}