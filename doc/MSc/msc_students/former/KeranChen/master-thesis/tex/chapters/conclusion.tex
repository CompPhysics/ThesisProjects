
\chapter{Conclusion}

\section{Summary of Results}
The aim of this thesis project is twofold: to create a quantum computing library that is oriented towards physicists and to study the performance of the Variational Quantum Eigensolver (VQE) and the Adaptive, Problem Tailor (ADAPT) VQE to obtain some insights for the best practices in using these algorithms. 

We built Quanthon as a result, which contains the basic elements to be able to perform any quantum operation. The \texttt{Hamiltonian} class was made to allow the Hamiltonians to be entered in a low-level way by specifying the one- and two-body coefficients. Other functionalities include the \texttt{VQE} and \texttt{ADAPT-VQE} classes which are the main focus of this project, as well as different modules that can assist the quantum simulations, including the mappers to convert fermionic Hamiltonians or matrix Hamiltonians to qubit Hamiltonians, the staircase and inverted staircase algorithms for converting exponentials of Pauli strings to quantum circuits, and the expectation value estimator which simulates the measurements of the quantum circuits and computes the expectation values of the Hamiltonians. 

With the library built, we then proceeded to study the performance of the VQE with hardware efficient ansatzes and ADAPT-VQE with two different minimal complete pools, the $ V $ and the $ G $ pool defined in Subsection~\ref{sub:qubit_adapt_vqe}, through both exact energy and ideal simulations of the multiple models. With the exact energy simulation, we observed that the hardware efficient ansatz in many cases does not converge to the ground state. When it does, the relative error is usually of order $ 10^{-2} $. The ADAPT-VQE with minimal pools, however, can indeed converge to the ground state with relative error lower than $ 10^{-6} $ given enough ADAPT iterations. The number of iterations it takes for the qubit-ADAPT-VQE to converge scales roughly linearly with the number of qubits in the model and the choice of the optimiser is independent of the number of iterations required. The best optimiser for the VQE was found to be the Powell method, and the best optimiser for the qubit-ADAPT-VQE was the BFGS method. 

In the presence of shot noise, the convergence of the qubit-ADAPT-VQE slows down but is still more stable and consistently outperforms the hardware efficient ansatz. We compared the error with the expected error from the number of shots to see if the algorithms have converged. The best optimiser, in this case, is the COBYLA method for the qubit-ADAPT-VQE and the Powell method for the VQE. We noticed that the ADAPT-VQE rarely exits due to the gradient being below the tolerance level, which is likely due to the noise.

No significant differences were observed in the performances of the $ V $ and $ G $ pool, except in the case of large pairing strength the $ G $ pool converges much faster than the $ V $ pool.

Additionally, we found that the gradient for all the operators in the minimal complete pool can be $ 0 $ for many Hamiltonians if the state is initialised in the $ \ket{0} $ state. This causes the ADAPT-VQE to exit immediately without any iterations. Initialising randomly avoids this problem but causes slow convergence. First, we used the maximally superposed state as an initial state. Later, we found that by initialising the ADAPT-VQE with the optimised state from hardware efficient ansatz boosts performances by reducing the number of iterations it takes for the ADAPT-VQE to converge. 

\section{Future Work}

We will group future work into two categories: improvements to the Quanthon library and improvements to the VQE and ADAPT-VQE algorithms.

\subsection{Improvements to Quanthon}
Many functionalities can be added to the Quanthon Library, such as adding more predefined gates or other popular algorithms. We could also explore other mappers, such as the Bravyi-Kitaev transformation~\cite{Seeley2012}. To increase the speed of simulations we could parallelise the measurements and add an option to group commuting terms of the Hamiltonian to be measured simultaneously. Integration with quantum hardware could also be extremely useful to allow the library to run on real quantum devices. More ansatzes could be implemented to allow for flexibility and performance comparison, specifically the unitary coupled cluster ansatz~\cite{Romero2019} and the fermionic-ADAPT ansatz~\cite{grimsley2019}. This would allow for a more comprehensive comparison of the ansatz. Another interesting direction would be to implement noise models to allow for noisy simulations.
\subsection{Improvements to Simulation Results}
The ideal simulation takes a long time to run, therefore the maximum number of iterations was all set to below $ 30 $, the performance of the ADAPT-VQE could potentially be improved by simply allowing the algorithm to run for more iterations. The $ 0 $ gradient problem is also concerning as the initial state could cause the algorithm to exit prematurely. More research is needed to find out under which circumstance would the gradient vanish and come up with either a different operator selection criteria for choosing an operator to be appended to the ansatz, or a systematic way to select the initial state to avoid the difficulty in convergence.

