\chapter{Quanthon: Quantum Computing in Python}
\label{ch:quanthon}

This chapter details the construction of a quantum computing library in python, namely \texttt{Quanthon}, with focus of simulation of real quantum devices for small physical systems. The library is can be found at \url{https://pypi.org/project/Quanthon/} and the source code at\url{https://github.com/moyasui/Quanthon}.

While existing quantum computing libraries such as Qiskit, Circ and OpenFermion \missingref are usually better integrated with hardware, more optimised and contain more functionalities, writing an entire library from scratch can be beneficial pedagogically by forbidding the existence of black boxes as well as allowing more freedom in structurs and conventions. One example of such is the ordering of qubits. Let $ A,B,C,D $ be single qubit operations, most literature uses $ ABCD $ to mean $ A_1B_2C_3D_4 $ i.e. applying the gate $ A $ on the first qubit, $ B $ on the second etc. This, however, is inconsistent with the computer science convention of representing numbers using bit strings, where the most significant bit is to the left. For example, the decimal $ 4 $ is usually written as $ 100 $ in binary representation. Most quantum computing libraries adheres to the binary representation convention by having the most significant bit to the left, therefore logically the aforementioned 4-qubit gate should have been written as $ DCBA $, which sadly is not. You will find my answer to this in two words' time, that is NO. In Quanthon, the most significant bit is to the right, and the order of operators/gates are written normally, in the same order. 

Quanthon will continue to be maintained and expanded after the completion of this master's project. 

\newpage

\section{Base}
\label{sec:base}
This section explains the structure for \texttt{base.py}, including the \texttt{Gate} class and \texttt{Qubits} class.
The state is represented by 
\begin{listing}[language=python]
	
\end{listing}

\section{Algorithms}
\label{sec:algo}
This section explains the structure for \texttt{algorithms.py}, including \texttt{VQE} and \texttt{ADAPT-VQE}.

