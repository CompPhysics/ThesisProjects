%In this thesis, we study quantum dots, which are electrons confined in semiconducting heterostructures. In those semiconductor structures, a few up to several thousand electrons are confined in regions on the nanometer scale. In the last two decades, quantum dots have become of particular interest since they exhibit, due to their small size, discrete quantum levels.
% Since quantum dots can be  manufactured and designed
%artificially  in the laboratory, those levels can be tuned to one's needs by changing, for example, the external field or the size and shape of the system \cite{REI02}.
% Hence quantum dots are perfectly suited to study quantum effects. Since the ground state of circular dots shows similar shell structures and magic numbers as seen for atoms and nuclei \cite{PhysRevLett.77.3613} , they give the oppurtunity to study electronic systems without the presence of the nucleus that affects the electrons.\\
% 
% 
%Quantum dots, which are electrons confined in semiconducting heterostructures. Due to their small size, they exhibit discrete quantum levels, which makes them perfect to study quantum effects \cite{REI02}.

% Sentence to nuclear theory in introduction
% Especially in nuclear theory it has been successfully applied to study systems  with different underlying potentials, their binding energy and other observables (ref!). \\