\section{In-medium SRG}
Instead of performing SRG in free space, the evolution can be done at finite density, i.e. directly in the A-body system \cite{kehrein2006flow}, which has recently been successfully applied in nuclear physics \cite{IMSRG,Koshi}. 
This approach is called in-medium SRG (IM-SRG) and allows the evolution of 
 $3,...,A$-body operators using only two-body machinery. These simplifications arise from the use of normal-ordering with respect to a reference state with finite density.

To explain the concept, let us consider a second-quantized Hamiltonian with two- and thee-body interactions,
\begin{align}
\hat{H} & = \sum\limits_{pq}\left\langle p \right| \hat{h}^{(0)} \left| q \right\rangle a_p\da a_q + \frac{1}{4}\sum\limits_{pqrs}\lla pq \right|\left| r s \rra a_p\da a_q\da a_s a_r \notag \\
 & \qquad + \frac{1}{36}\sum_{pqrstu} \lla pqr \right|\hat{v}^{(3)}\left| stu \rra a_p\da a_q\da a_r\da a_u a_t a_s . \notag
\end{align} 
As demonstrated in chapter \ref{chap:mb}, normal-ordering with respect to a reference state $|\Psi_0\rangle$ yields 
\begin{align*}
\hat{H_N} &= \hat{F}_N + \hat{V}_N + \hat{H}_N^{(3)}\\
&= \sum\limits_{pq}f_{pq}\llb a_p\da a_q\rrb + \frac{1}{4}\sum\limits_{pqrs}\Gamma_{pqrs} \llb a_p\da a_q\da a_s a_r \rrb + \frac{1}{36}\sum\limits_{pqrstu} W_{pqrstu} \llb a_p\da a_q\da a_r\da a_u a_t a_s\rrb ,
\end{align*}
with the amplitudes for the one-body operator given by
\[
f_{pq} = \lla p \right|\hat{h}^{(0)}\left| q \rra + \sum\limits_i \lla pi \right| \left| qi\rra + \frac{1}{2}\sum\limits_{ij}\lla pij\right|\hat{v}^{(3)}\left| pij\rra,
\]
the ones for the two-body operator
\[
\Gamma_{pqrs} = \lla pq \right| \left| r s \rra + \sum\limits_i \lla p q i \right| \hat{v}^{(3)}\left| rsi \rra,
\]
and the ones for the three-body operator
\[
W_{pqrstu} =  \lla pqr\right|\hat{v}^{(3)}\left|stu\rra.
\]
As in chapter \ref{chap:mb}, we use the notation that indices $\lbrace a,b,c,...\rbrace$ denote particles states, $\lbrace i,j,k,... \rbrace$ denote hole states, and $\lbrace p,q,r,...\rbrace$ can be used for both particle and hole states. 

In relation to the full Hamiltonian, $\hat{H}_N$ is obtained as follows (see chapter \ref{chap:mb}),
\[
\hat{H}_N = \hat{H} - E_0,
\]
where $E_0$, the energy expectation value between reference states, is 

\begin{align*}
E_0 &= \lla \Phi_0 \right| \hat{H} \left| \Phi_0 \rra \\
&= \sum\limits_i \lla i \right| \hat{h}^{(0)} \left| i \rra + \frac{1}{2}\sum\limits_{ij}\lla ij\right|  \left| ij\rra + \frac{1}{6}\sum\limits_{ijk}\lla ijk \right| \hat{v}^{(3)} \left| ijk \rra.
\end{align*}
			
Exactly as in free space, we want to compute the flow equations 
\[
\frac{d \hat{H}_s}{ds} = \left[\hat{\eta}_s, \hat{H}_s \right].
\]
The difference is that we now formulate the derivatives and the generator $\hat{\eta}$ in the language of second quantization and normal-ordering, too. \\
Using this approach, one faces one of the major challenges of the SRG method, which is the generation of higher and higher order interaction terms during the flow. Each time the derivative is computed, the generator $\hat{\eta}$ and interaction $V$ gain terms of higher order, and this continues in principle to infinity.\\
To make the method computationally possible, one is therefore forced to truncate the flow equations after a certain order. This affects of course the accuracy of the result, and the fewer orders one includes, the higher the truncation error is. On the other hand, improving the result is quite straightforward by including higher order interaction terms. If one included all generated terms until the result has converged within the desired accuracy, one would obtain the same results as with SRG in free space.

Truncating to normal-ordered three-body operators, the generator $\hat{\eta}$ can be written as
\[
\hat{\eta} = \sum_{pq} \eta_{pq}^{(1)} \llb a_p\da a_q\rrb + \frac{1}{4}\sum\limits_{pqrs}\eta_{pqrs}^{(2)} \llb a_p\da a_q\da a_s a_r \rrb + \frac{1}{36}\sum\limits_{pqrstu} \eta_{pqrstu}^{(3)} \llb a_p\da a_q\da a_r\da a_u a_t a_s\rrb,
\]
where $\eta_{pq}^{(1)}, \eta_{pqrs}^{(2)}$ and $\eta_{pqrstu}^{(3)}$ are the one-, two- and three-body operator of the generator, respectively. For different choices of $\hat{\eta}$, those amplitudes look differently and we will demonstrate in section \ref{sub:imsrg2}, how explicitly to compute them for a Hamiltonian containing maximal two-body interactions.\\
Including also three-body interactions, the flow equations are given by
\begin{align*}
\frac{d \hat{H}_s}{ds} &= \left[\hat{\eta}_s, \hat{H}_s \right] \\
&= \left [  \sum_{pq} \eta_{pq}^{(1)} \llb a_p\da a_q\rrb + \frac{1}{4}\sum\limits_{pqrs}\eta_{pqrs}^{(2)} \llb a_p\da a_q\da a_s a_r \rrb + \frac{1}{36}\sum\limits_{pqrstu} \eta_{pqrstu}^{(3)} \llb a_p\da a_q\da a_r\da a_u a_t a_s\rrb, \right.\\
& \left.\sum\limits_{pq}f_{pq}\llb a_p\da a_q\rrb + \frac{1}{4}\sum\limits_{pqrs}\Gamma_{pqrs} \llb a_p\da a_q\da a_s a_r \rrb + \frac{1}{36}\sum\limits_{pqrstu} W_{pqrstu} \llb a_p\da a_q\da a_r\da a_u a_t a_s\rrb
\right],
\end{align*}
where we suppressed the $s$-dependence for simplicity.

 Using the commutation relations presented in Appendix \ref{App:AppendixB} and collecting the constants in $E_0$, the one-body terms in $f$, the two-body terms in $\Gamma$ and the three-body terms in $W$, we obtain Eq. (\ref{eq:flow0}-\ref{eq:flow3}), where we make use of the permutation operators
\be
\hat{P}_{pq}f(p,q) = f(q,p), \quad \hat{P}(pq/r) = 1-\hat{P}_{pq}-\hat{P}_{qr}, \quad \hat{P}(p/qr) = 1-\hat{P}_{pq}-\hat{P}_{pr}.
\ee
The derivative of $E_0$ is given by
\begin{align}
\frac{d}{ds} E_0(s) &= \sum_{ia} (1-\hat{P}_{ia}) \eta_{ia}^{(1)} f_{ai} + \frac{1}{2}\sum_{ijab}\eta_{ijab}^{(2)}\Gamma_{abij}+ \frac{1}{18}\sum_{ijkabc} \eta_{ijkabc}^{(3)} W_{abcijk}, 
\label{eq:flow0}
\end{align}
the one of the one-body operator by
\begin{align}
\frac{d}{ds}f_{pq}(s) &= \sum_r (1+\hat{P}_{pq}) \eta_{pr}^{(1)}f_{rq} + \sum_{ia}\lb 1-\hat{P}_{ia}\rb\lb \eta_{ia}^{(1)}\Gamma_{apiq} - f_{ia}\eta_{apiq}^{(2)}\rb \notag \\
& + \frac{1}{2}\sum_{ija}\lb 1+\hat{P}_{pq} \rb \eta_{apij}^{(2)}\Gamma_{ijaq} + \frac{1}{2}\sum_{abi}\lb 1+\hat{P}_{pq}  \rb \eta_{ipab}^{(2)}\Gamma_{abiq} \notag \\
& + \frac{1}{4}\sum_{ijab}\lb \eta_{ijpqab}^{(3)}\Gamma_{abij} - W_{ijpqab}\eta_{abij}^{(2)}\rb + \frac{1}{12}\sum_{ijabc}\lb \eta_{ijpabc}^{(3)} W_{abcijq} - W_{ijpabc} \eta_{abcijq}^{(3)} \rb \notag \\ & + 
\frac{1}{12}\sum_{abijk}\lb \eta_{abpijk}^{(3)} W_{ijkabq} - W_{abpijk} \eta_{ijkabq}^{(3)} \rb,
\label{eq:flow1} 
\end{align}
the derivative of the two-body operator by
\begin{align}
\frac{d}{ds}\Gamma_{pqrs}(s) &= \sum_t \left[ \lb 1-\hat{P}_{pq} \rb \lb \eta_{pt}^{(1)}\Gamma_{tqrs} - f_{pt}\eta_{tqrs}^{(2)} \rb - \lb 1-\hat{P}_{rs}\rb \lb \eta_{tr}^{(1)} \Gamma_{pqts} - f_{tr} \eta_{pqts}^{(2)} \rb \right] \notag \\
& + \frac{1}{2}\sum_{ab}\lb \eta_{pqab}^{(2)}\Gamma_{abrs} - \Gamma_{pqab}\eta_{abrs}^{(2)}\rb - 
\frac{1}{2}\sum_{ij}\lb \eta_{pqij}^{(2)}\Gamma_{ijrs} - \Gamma_{pqij}\eta_{ijrs}^{(2)}\rb \notag \\
& -\sum_{ia} \lb 1- \hat{P}_{ia} \rb \lb 1-\hat{P}_{pq}\rb \lb 1-\hat{P}_{rs} \rb \eta_{aqis}^{(2)}\Gamma_{ipar} + \sum_{ia} \lb 1- \hat{P}_{ia} \rb \lb \eta_{apqirs}^{(3)} f_{ai} - W_{apqirs} \eta_{ai}^{(1)} \rb \notag \\
& + \frac{1}{2}\sum_{abi} \lb 1 - \hat{P}_{pr}\hat{P}_{qs}\hat{P}_{pq} - \hat{P}_{rs} + \hat{P}_{pr}\hat{P}_{qs} \rb \lb \eta_{ipqabs}^{(3)}\Gamma_{abir} - W_{ipqabs} \eta_{abir}^{(2)} \rb \notag \\
& + \frac{1}{2}\sum_{aij} \lb 1 - \hat{P}_{pr}\hat{P}_{qs}\hat{P}_{pq} - \hat{P}_{rs} + \hat{P}_{pr}\hat{P}_{qs} \rb \lb \eta_{apqijs}^{(3)}\Gamma_{ijar} - W_{apqijs} \eta_{ijar}^{(2)} \rb \notag \\
& + \frac{1}{6} \sum_{iabc}\lb \eta_{ipqars}^{(3)} W_{abcirs} - \eta_{abcirs}^{(3)} W_{ipqabc} \rb
+ \frac{1}{6} \sum_{iabc}\lb \eta_{ipqars}^{(3)} W_{abcirs} - \eta_{abcirs}^{(3)} W_{ipqabc} \rb \notag \\
& + \frac{1}{4}\sum_{abij}\lb 1 - \hat{P}_{pq} \rb \lb 1 - \hat{P}_{rs} \rb \lb \eta_{abpijs}^{(3)} W_{ijqabr} - \eta_{ijpabs}^{(3)} W_{abqijr} \rb, 
\label{eq:flow2}
\end{align}
and finally the one of the three-body operator by
\begin{align}
\frac{d}{ds} W_{pqrstu}(s) &= \sum_v \left[ \hat{P}(p/qr)\eta_{pv}^{(1)} W_{vqrstu} - \hat{P}(s/tu) \eta_{vs}^{(1)} W_{pqrvtu} \right] \notag \\
& - \sum_v \left[ \hat{P}(p/qr)f_{pv} \eta_{vqrstu}^{(3)} - \hat{P}(s/tu) f_{vs} \eta_{pqrvtu}^{(3)} \right] \notag \\
& + \sum_v \hat{P}(pq/r) \hat{P}(s/tu) \lb \eta_{pqsv}^{(2)} \Gamma_{vrtu} - \Gamma_{pqsv} \eta_{vrtu}^{(2)} \rb\notag \\
& + \frac{1}{2} \sum_{ab} \hat{P}(p/qr) \lb \eta_{pqab}^{(2)} W_{abrstu} - \Gamma_{pqab} \eta_{abrstu}^{(3)}\rb
-  \frac{1}{2} \sum_{ij} \hat{P}(p/qr) \lb \eta_{pqij}^{(2)} W_{ijrstu} - \Gamma_{pqij} \eta_{ijrstu}^{(3)}\rb \notag \\
& - \frac{1}{2} \sum_{ab} \hat{P}(s/tu) \lb \eta_{abtu}^{(2)} W_{pqrsab} - \Gamma_{abtu} \eta_{pqrsab}^{(3)}\rb
+  \frac{1}{2} \sum_{ij} \hat{P}(s/tu) \lb \eta_{ijtu}^{(2)} W_{pqrsij} - \Gamma_{ijtu} \eta_{pqrsij}^{(3)}\rb \notag \\
& - \sum_{ia} \lb 1-\hat{P}_{ia}\rb \hat{P}(p/qr)\hat{P}(s/tu)  \lb \eta_{apis}^{(2)} W_{iqratu} - \Gamma_{apis}^{(2)} \eta^{(3)}_{iqratu}\rb \notag \\
& + \frac{1}{6}\sum_{ijk} \lb \eta_{pqrijk}^{(3)} W_{ijkstu} - W_{pqrijk} \eta_{ijkstu}^{(3)} \rb + 
\frac{1}{6}\sum_{abc} \lb \eta_{pqrabc}^{(3)} W_{abcstu} - W_{pqrabc} \eta_{abcstu}^{(3)} \rb \notag \\
& + \frac{1}{2}\sum_{aij} \hat{P}(pq/r) \hat{P}(s/tu) \lb \eta_{ijratu}^{(3)} W_{apqijs} - \eta_{aqriju}^{(3)} W_{pijsta} \rb \notag \\
& + \frac{1}{2}\sum_{abi} \hat{P}(pq/r) \hat{P}(s/tu) \lb \eta_{abritu}^{(3)} W_{ipqabs} - \eta_{iqrabu}^{(3)} W_{pabsti} \rb
\label{eq:flow3}
	\end{align}

Note that on the right-hand side terms the $s$-dependence of all the amplitudes has been skipped for better readability. 
The large number of summations makes the equations computationally quite expensive, as they have to be computed for each integration step. Table \ref{tab:numeqs} gives some examples of how the number of equations is quickly increasing with number of shells $R$.

\begin{table}
\begin{center}
\begin{tabular}{cr}
\hline\hline
 $R$ & \# eqs. \\
 \hline
 2 & 5 \\
 5  & 6531   \\
10& 624149 \\
15& 9600152 \\
18& 33187537 \\
20& 68138690 \\
\hline\hline
\end{tabular}
\end{center}
\caption{Number of ordinary differential equations to be solved dependent on the number of shells $R$.}
\label{tab:numeqs}
\end{table}

Especially the computational cost connected to the three-body operators is very high and time-consuming, Therefore, a possible alternative is to truncate the flow equations (\ref{eq:flow0})-(\ref{eq:flow3}) up to normal-ordered two-body operators, an approach which is called IM-SRG(2).




