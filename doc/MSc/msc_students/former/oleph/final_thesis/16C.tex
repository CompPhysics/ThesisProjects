\begin{figure}[htbp]
\setlength{\unitlength}{3.0cm}
\begin{center}
\begin{picture}(6.9,6.5)(0,-1)
\psset{xunit=1.00cm, yunit=6.0cm}
\newcommand{\drawlevel}[5]{\psline[origin={#1,#2}, linewidth=0.3pt](0,0)(1.5,0.0)\rput(#3,#4){\scriptsize \makebox(0,0){$#5$}}}
\newcommand{\connect}[4]{\psline[linewidth=0.3pt, linestyle=dotted, dotsep=1.2pt](#1,#2)(#3,#4)}
\thicklines
\rput(0.1,-0.2){\makebox(0,0){{\large 0p3-0d3}}}
\drawlevel{0.7}{-0}{2}{0}{\ 0^+ \ (0)}
\drawlevel{0.7}{-1.03973}{2}{1.03973}{\ 2^+ \ (3.36^)}
\drawlevel{0.7}{-1.70674}{2}{1.70674}{\ 0^+ \ (5.51^)}
\drawlevel{0.7}{-2}{2}{2}{\ 3^+ \ (6.46^)}
\connect{0.8}{2}{1.1}{2}

\rput(4.3,-0.2){\makebox(0,0){{\large 0p3-0f7}}}
\drawlevel{-3.5}{-0}{6.2}{0}{\ 0^+ \ (0)}
\drawlevel{-3.5}{-0.515253}{6.2}{0.515253}{\ 2^+ \ (1.66^)}
\drawlevel{-3.5}{-0.938028}{6.2}{0.938028}{\ 0^+ \ (3.03^)}
\drawlevel{-3.5}{-1.2158}{6.2}{1.2158}{\ 2^+ \ (3.93^)}
\connect{5}{1.2158}{5.3}{1.2158}

\rput(8.5,-0.2){\makebox(0,0){{\large exp. data}}}
\drawlevel{-7.7}{-0}{10.4}{0}{\ 0^+ \ (0)}
\drawlevel{-7.7}{-0.546935}{10.4}{0.546935}{\ 2^+ \ (1.77^)}
\drawlevel{-7.7}{-0.937471}{10.4}{0.937471}{\ (0^+) \ (3.03^)}
\drawlevel{-7.7}{-1.23448}{10.4}{1.23448}{\ 2 \ (3.99^)}
\connect{9.2}{1.23448}{9.5}{1.23448}
\drawlevel{-7.7}{-1.26607}{10.4}{1.30448}{\ 3(^+) \ (4.09^)}
\connect{9.2}{1.26607}{9.5}{1.30448}
\drawlevel{-7.7}{-1.28279}{10.4}{1.37448}{\ 4^+ \ (4.14^)}
\connect{9.2}{1.28279}{9.5}{1.37448}
\drawlevel{-7.7}{-1.89198}{10.4}{1.89198}{\ (2^+,3^-,4^+) \ (6.11^)}

\end{picture}
\end{center}
\caption{Energy spectra for 16C, in the full 0p3-0d3 and reduced 0p3-0f7 model spaces. By reduced 0p3-0f7 model space we mean that we have reduced our 0p3-0f7 model space so that we have a maximum of 4 particles in the 0d5 orbital, a maximum of 2 particles in the 0d3 orbital and maximum of 2 neutrons and 0 protons in the 0f7 orbital. The energies are given in MeV.}
\label{fig:16C}
\end{figure}



