\newcommand{\h}{\hat{H}}
\newcommand{\he}{\hat{H}_{\textrm{eff}}}
\renewcommand{\H}{\hat{\mathscr{H}}}
\renewcommand{\v}{\hat{V}}
\newcommand{\vv}{\hat{\tilde{V}}}
\renewcommand{\t}{\hat{T}}
\newcommand{\ve}{\hat{V}_{\textrm{eff}}}
\renewcommand{\u}{\hat{U}}
\newcommand{\p}{\hat{P}}
\newcommand{\q}{\hat{Q}}
\newcommand{\qq}{\hat{\tilde{Q}}}
\newcommand{\g}{\hat{G}}
\renewcommand{\gg}{\hat{\tilde{G}}}
\newcommand{\e}{\hat{e}}
\renewcommand{\o}{\hat{\Omega}}
\newcommand{\x}{\hat{X}}
\renewcommand{\u}{\hat{U}}
\newcommand{\w}{\hat{w}}
\renewcommand{\a}{\hat{A}}
\renewcommand{\b}{\hat{B}}
\chapter{Renormalizing the Hamiltonian}

The calculations done in this thesis are based on the framework of the nuclear
shell model, with neutrons and protons as degrees of freedom. The effective
interactions are based on a renormalized nucleon-nucleon (NN) interaction
including the Coulomb force.

The renormalization of the Hamiltonian is done in a harmonic oscillator basis.
This basis is also used for the Hartree-Fock calculations.

\section{Renormalizing the Hamiltonian}

We want to solve the non-relativistic Shcr\"{o}dinger equation for the nuclei
$^{14}$C, $^{16}$C, $^{15}$B and $^{15}$C,
\begin{align}
	\h\ket{\Psi(1\dots A)} = E\ket{\Psi(1\dots A)}.
	\label{Eq:SE}
\end{align}

Equation \ref{Eq:SE} is a many-particle equation that must be solved
approximately. Here we will approximate it by using two-particle interactions.
This is because the three-particle and higher interactions are more difficult
to calculate, and give smaller contributions to the total interaction. 
Note that three-particle interactions can have a significant impact on the
binding energy, providing up to $10\%$ of the total energy, and sometimes
influencing the energy spectra.

The two-body Hamiltonian is given by
\begin{align}
	\h = \t + \v = \sum_{i=1}^A\hat t_i + \sum_{i<j}\v_{ij},
	\label{eq:Ham}
\end{align}
where $\hat t_i$ is the kinetic energy of nucleon $i$ and $\v_{ij}$ is
the nucleon nucleon (NN) interaction between particle $i$ and $j$. In nuclear
physics the NN interaction V is not well known, and there exists many methods
to derive it.

This Hamiltonian is often rewritten using an auxiliary potential $\u$,
\begin{align}
	\h = (\t+\u) + (\v-\u) = \h_0+\h_1,
\end{align}
where we want to choose $\u$ so that $\h_1$ is small. 

Our two-body Hamiltonian consists of A nucleons interacting with each other in
our infinite Hilbert space, and is too large for practical calculations. We
will use $^4$He as a closed core to reduce this space. This means that the four
first nucleons of our nuclei cannot be excited to higher-lying states. $^4$He
fills the $0s$ shell, and is very stable. Excitations from the $0s$ shell is
therefore highly unlikely, and using it as a closed core will be a good
approximation. The closed core will be our new reference vacuum.

The nucleus $^4$He is
\begin{align}
	\ket{c} = \prod_{i=1}^4a_i^\dagger\ket0.
	\label{eq:core}
\end{align}

We will also map our Hamiltonian down on a reduced Hilbert space, our so-called
model space.

The wave function $\ket{\Psi_\lambda}$ in the full Hilbert space is
\begin{align}
	\ket{\Psi_\lambda} = \sum_{\lambda=1}^\infty b_\lambda\ket{\phi_\lambda}.
\end{align}

We divide the Hilbert space into a model space and an excluded state with the
projection operators $\p$ and $\q$, that map the full Hilbert space to
respectively the model space and the excluded space,
\begin{align}
	\p = \sum_{i=1}^d\ket{\phi_i}\bra{\phi_i},
\end{align}
\begin{align}
	\q = \sum_{i=d+1}^\infty\ket{\phi_i}\bra{\phi_i},
	\label{eq:Q}
\end{align}
$d$ is here the dimensionality of the model space.

The projection operators $\p$ and $\q$ are hermitian, and fulfill the
relations $\p\q=0$, $\p^2=\p$, $\q^2=\q$ and $\p+\q=1$.

In our calculations we will work with three different model spaces. These model spaces are listed
below.
\begin{itemize}

\item I now define the shorthand notation $0p\frac32-1s\frac12$. This notation
comprises the following single particle orbitals, $0p\frac32$, $0p\frac12$,
$0d\frac52$ and $1s\frac12$ for neutrons and protons. The notation
$0p\frac32-1s\frac12$ will hereafter be used for this model space.

\item I now define the shorthand notation $0p\frac32-0d\frac32$. This notation
comprises the following single particle orbitals, $0p\frac32$, $0p\frac12$,
$0d\frac52$, $1s\frac12$ and $0d\frac32$ for neutrons and protons. The notation
$0p\frac32-0d\frac32$ will hereafter be used for this model space.

\item I now define the shorthand notation $0p\frac32-0f\frac72$. This notation
comprises the following single particle orbitals, $0p\frac32$, $0p\frac12$,
$0d\frac52$ $1s\frac12$, $0d\frac32$ and $0f\frac72$ for neutrons and protons.
The notation $0p\frac32-0f\frac72$ will hereafter be used for this model space.

\end{itemize}

Using the $\p$ and $\q$ projection operators one can derive a new two-body
interaction that is effective only in the model space. This $\ve$ is also known
as the $\qq$-box,

\begin{align}
	\ve(\omega) = \qq(\omega) &= \p(\h_1 + \h_1\frac{\q}{\omega-\h_0-\q\h_1\q}\q\h_1)\p\\
	&= \p(\h_1 + \h_1\frac{\q}{\omega-\h_0}\h_1 + \h_1\frac{\q}{\omega-\h_0}\h_1\frac{\q}{\omega-\h_0}\h_1 + \dots)\p,
	\label{eq:Qbox}
\end{align}
where $\omega$ is a general energy variable, the so-called starting energy. See
\citep{kuo} for more details. This equation is also known as the
Rayleigh-Schr\"{o}dinger perturbative expansion.

For simplicity we will write $\h_1$ as $\v$ in our equations.

The $\qq$-box can thus be expanded in a diagrammatic expansion of the NN
interaction, as shown in figure \ref{fig:goldstone}. Note that in this case the
model space is defined by the hole states. Figure \ref{fig:goldstone} i) shows
a closed, first order Goldstone diagram, while diagram ii) shows a closed,
second order Goldstone diagram and diagram iii) shows a closed, third order
Goldstone diagram.

Diagram i) can be expressed as
\begin{align}
	(i)=\frac{(-)^{n_h+n_l}}{2^{n_{ep}}}\sum_{k_i,k_j\leq k_F}
	\bra{k_ik_j}\vv\ket{k_ik_j}_{AS},
\end{align}
where $n_h$ denotes the number of hole lines, $n_l$ the number of closed
fermion loops and $n_{ep}$ is the number of equivalent pairs. The subscript
$AS$ denotes the antisymmetrized and normalized matrix element
\begin{align}
	\bra{k_ik_j}\vv\ket{k_ik_j}_{AS}=\bra{k_ik_j}\v\ket{k_ik_j}-
	\bra{k_jk_i}\v\ket{k_ik_j}=\vv_{ijij}.
\end{align}
\begin{figure}
	\setlength{\unitlength}{1mm}
	\begin{align}
	\put(-70,10){\epsfxsize=12cm \epsfbox{goldstone.eps}}
	\end{align}
	\caption{Closed Goldstone diagrams. Diagrams (i), (ii) and (iii) are,
	respectively, the first, second and third order contributions to
	the interaction $\v$.}
	\label{fig:goldstone}
\end{figure}
Diagram (i) then becomes
\begin{align}
	(i)=\frac{(-)^{2+2}}{2^1}\sum_{ij\leq k_F}\vv_{ijij}.
\end{align}
Diagrams (ii) and (iii) gives
\begin{align}
	(ii)=\frac{(-)^{2+2}}{2^2}\sum_{ij\leq k_F}\sum_{mn>k_F}
	\frac{\vv_{ijmn}\tilde{V}_{mnij}}
	{\varepsilon_i+\varepsilon_j-\varepsilon_m-\varepsilon_n},
\end{align}
and
\begin{align}
	(iii)=\frac{(-)^{2+2}}{2^3}\sum_{ij\leq k_F}\sum_{mn>k_F}
	\sum_{pq>k_F}
	\frac{\vv_{ijmn}\vv_{mnpq}\vv_{pqij}}
	{(\varepsilon_i+\varepsilon_j-\varepsilon_m-\varepsilon_n)
	(\varepsilon_i+\varepsilon_j-\varepsilon_p-\varepsilon_q)}.
	\label{eq:goldDia3}
\end{align}

\begin{figure}
	\setlength{\unitlength}{1mm}
	\begin{align}
	\put(-40,10){\epsfxsize=9cm \epsfbox{nn-potential.eps}}
	\end{align}
	\caption{Schematic model for the NN potential.}
	\label{fig:potential}
\end{figure}

Figure \ref{fig:potential} shows that the interaction $\v$ is large for small
values of the interparticle distance ${\bf r}$. We see readily from this and
equation (\ref{eq:goldDia3}) that the Goldstone linked-diagram theory is
unsuited for perturbation theory as the terms will only get larger when ${\bf
r}$ is small.

We need to use another method to solve the Schr\"{o}dinger equation. Two such
methods that the program uses for finding the effective interaction are the
$G$-matrix and the Vlowk methods.

\section{$G$-matrix method}

%Se side 130 for referanser for linjen under.
The $G$-matrix method was originally developed by Brueckner \citep{brueck}, and
was further developed by Goldstone \citep{gold} and Bethe, Brandow and
Petschek \citep{bbp}, see Ref.~\cite{g261} for a historical overview. It involves the summation of all particle-particle ladder type of
diagrams to inifnite order and serves therefore to renormalize the short-range part of the nucleon-nucleon interaction. 
See for Ref.~\citep{g261} for more details. It can be written as

\begin{align}
	\gg_{ijij} =& \frac{1}{2}\vv_{ijij} + \frac{1}{2}\sum_{mn>k_F}
	\frac{\vv_{ijmn}}
	{\varepsilon_i+\varepsilon_j-\varepsilon_m-\varepsilon_n}\notag
	\\&\times \left(\frac{1}{2}\vv_{mnij}+\frac{1}{2}\sum_{mn>k_F}
	\sum_{pq>k_F}\frac{\vv_{mnpq}}
	{\varepsilon_i+\varepsilon_j-\varepsilon_p-\varepsilon_q}\right)\notag
	\\&\times\left(\frac{1}{2}\vv_{pqij}+\dots\right).
\end{align}
This is a recursive equation, and can thus be written as
\begin{align}
	\vv_{ijij} = \frac{1}{2}\vv_{ijij} + \frac{1}{2}\sum_{mn>k_F}
	\vv_{ijmn}\frac{1}
	{\varepsilon_i+\varepsilon_j-\varepsilon_m-\varepsilon_n}\gg_{mnij}.
	\label{eq:G_rec}
\end{align}
Although the interaction diverges for small values of $\bf r$ and although we
have an infinite sum, we know that the energy of the system is still finite.
This means that the sum in equation \ref{eq:G_rec} is convergent and can
therefore be calculated accurately.

The single particle (sp) energies $\varepsilon$ are given by the unperturbed
(non-interacting) Hamiltonian $\h_0$
\begin{align}
	\h_0\ket{\psi_m\psi_n}=(\varepsilon_m+\varepsilon_n)\ket{\psi_m\psi_n}.
\end{align}

We can now, using the projection operator $\q$ defined in equation \ref{eq:Q},
rewrite the $G$-matrix into a more general form
\begin{align}
	\g(\omega) = \v+\v\frac{\q}{\omega -\h_0}\g(\omega),
	\label{eq:G}
\end{align}
where $\omega$ is a general energy variable, that is, the unperturbed sp
energies for a general two-particle state.

Equation \ref{eq:G} assumes that $\h_0$ commutes with $\q$. If this is not the
case, then the $G$-matrix assumes the form

\begin{align}
	\g(\omega) = \v+\v\q\frac{1}{\omega -\q\h_0\q}\q\g(\omega ).
	\label{eq:G2}
\end{align}

We will in our coming calculations assume that $\h_0$ commutes with $\q$. This
is safe to do because we can always choose $\u$ so that $\h_0$ will commute
with $\q$.

Equation \ref{eq:G} is still problematic though, as we have the infinitely
large excluded space in it. To solve this we introduce a free scattering
interaction matrix ($\g_F$, which is easy to calculate), in order to get rid of
the excluded space dependency. In order to do this we need to define two
matrices $\g$ and $\g_F$, where $\g$ is the effective two-body interaction for
bound states, and $\g_F$ is the free $G$-matrix, that is, the free scattering
two-body interaction. I will, for simplicity, from now on write $\g(\omega)$ as
$\g$.

Before I define the $G_F$-matrix, I will show how we can express one $G$-matrix by another $G$-matrix, see Ref.~\cite{g261}
for more details.

We define two $G$-matrices,
\begin{align}
	\g_1 &= \v_1 + \v_1\frac{\q_1}{\e_1}\g_1,\\
	\g_2 &= \v_2 + \v_2\frac{\q_2}{\e_2}\g_2,
\end{align}
where
\begin{align}
	\e_i &= \omega_i - \q_i\h_0\q_i,& i &= 1,2. \notag
\end{align}
We define the wave operators
\begin{align}
	\o_i &= 1 + \frac{\q_i}{\e_i}\g_i,& i &= 1,2
\end{align}
This leads to the following relation
\begin{align}
	\g_i &= \v_i\o_i,& i &= 1,2.
	\label{eq:GO}
\end{align}
We can now write $\g_1$ as
\begin{align}
	\g_1 = &\g_1 - \g_2^{\dagger}\left(\o_1-1-\frac{\q_1}{\e_1}\g_1\right)
	+ \left(\o_2^{\dagger}-1-\g_2^{\dagger}\frac{\q_2}{\e_2}\right)\g_1,
\end{align}
and using equation \ref{eq:GO} we get
\begin{align}
	\g_1 = \g_2^{\dagger} + \g_2^{\dagger}
	\left(\frac{\q_1}{\e_1}-\frac{\q_2}{\e_2}\right)\g_1
	+ \o_2^{\dagger}(\v_1-\v_2)\o_1.
	\label{eq:G12}
\end{align}

When $\v_1 = \v_2$ equation \ref{eq:G12} reduces to
\begin{align}
	\g_1 = \g_2^{\dagger} + \g_2^{\dagger}
	\left(\frac{\q_1}{\e_1}-\frac{\q_2}{\e_2}\right)\g_1.
	\label{eq:G12V}
\end{align}

Now we can go back to the free $G$-matrix. The $G_F$-matrix is basis-independent, and is defined as
\begin{align}
	\g_F = \v + \v\frac{1}{\tilde\omega-T}\g_F,
	\label{eq:GF}
\end{align}
where $\v$ is the usual two-body interaction, $T$ is the kinetic term and
$\omega$ is the unperturbed energy.

Since $\g$ and $\g_F$ have the same NN interaction, we can use equation
\ref{eq:G12V} to define $\g$,
\begin{align}
	\g = \g_F^{\dagger} + \g_F^{\dagger}
	\left(\frac{\q}{\e}-\frac{1}{\e_F}\right)\g,
	\label{eq:GGF}
\end{align}
where we have used the fact that $\q_F = 1$.

Using the matrix relation
\begin{align}
	\q\frac{1}{\q\e\q}\q = \frac1\e-\frac1\e\p\frac{1}{\p\frac1\e\p}\p\frac1\e,
\end{align}
we can rewrite equation \ref{eq:G2} as
\begin{align}
	\g = \g_F+\Delta\g,
	\label{eq:G-matrix}
\end{align}
where $\Delta\g$ is a correction term, defined fully in the model space $P$,
that gives us the bound energies. It is given as
\begin{align}
	\Delta\g = -\g_F\frac1\e\p\frac{1}{\p\left(\frac1\e+\frac1\e\g_f\frac1\e\right)\p}\p\frac1\e\g_F.
\end{align}
The $\Delta\g$ matrix is expressed in terms of the free interaction matrix
$G_F$, and is not dependent on the high-lying states in the excluded space $Q$.
The $G_F$ matrix can be calculated numerically exactly for the full
two-particle Hilbert space, and this makes the $\Delta\g$ matrix easy to
calculate.

Note that the $G_F$ matrix is still a recursive equation that needs to be
solved.
%The $G_F$ matrix is still a recursive equation that needs to be solved. We
%will in the next chapter show how this is solved with the
%Rayleigh-Schr\"{o}dinger equation.

%In the $G$-matrix for finite nuclei method, the two $G$-matrices mentioned
%above are defined as following.
%\begin{align}
%	\g=\gg + \gg \left(\frac{q_l}{\omega-\h_0}\right)\g\\
%	\label{eq:Gfinite}
%	\gg=\v+\v\frac{\q_h}{\omega -\t}\gg
%\end{align}
%and
%\begin{align}
%	\q = \q_l + \q_h
%\end{align}
%where $\q_l$ stands for the lower-lying states in $\q$, and $\q_h$ stands for
%the higher-lying states in $\q$. Equation \ref{eq:Gfinite} is easily seen to be
%true by using equation \ref{eq:G12} and that $\v_1 = \v_2 = \v$ in equation
%\ref{eq:Gfinite}.
%
%I will not go into how to define $\q_l$ and $\q_h$ or the further calculations
%needed to calculate $\gg$. I will instead refer to the works of \citet{g261}. I
%will only say that $\q_l$ is commonly? chosen as the part of the $Q$-space
%where the interaction has negative potential.
%
%The $G$-matrix works in a single particle basis !Stemmer dette?!.
%
%!Dette avsnittet boer skrives om! Since $\g$ is a function of the unperturbed
%energy !Usikker paa dette!, we need to describe it by perturbation theory.
%Furthermore, the $G$-matrix only calculates the short range part of the
%interaction. Unlike the Goldstone linked-diagram theory, we can describe it by
%perturbation theory since the G-matrix contains all the diagrams in the
%Goldstone linked-diagram theory.
%
%One of the more common selected values for $\chi$ is $\chi$ = P*$\chi$*Q,
% as this choice gives eigenvalues that belong to the P-space.

\section{Vlowk method}

The Vlowk method renormalizes the Hamiltonian in the momentum space. First we
use a similarity transformation of the full Hilbert space to the momentum
space. This transformation is numerically exact. In the momentum space basis
we introduce a cutoff, were we exclude the states with high momenta. This
cutoff is necessary because we do not have a one to one correspondence when
going from the full momentum space. The low-momentum space will then be part
of our model space $P$, while the high momentum space will be part of our
excluded state $Q$. The NN interaction is strongly dependent upon this cutoff,
so care must be taken when choosing where to set the cutoff. Due to the choice
of cutoff, this introduces a stronger dependency on many-body forces than the
$G$-matrix. We need at least three-body forces computed with the same cutoff.

We have chosen a cutoff of $2.2$ fm$^{-1}$ in our calculations.

We start by using the Lee and Suzuki similarity transformation of the full two-particle Hamiltonian,
\begin{align}
	\H = \x^{-1}\h\x.
\end{align}

We divide the Hilbert space into the $P$ and $Q$ spaces. For the Vlowk method
these $P$ and $Q$ spaces are not the same as the models space given in figures
\ref{modellrom1}-\ref{modellrom4}. They are defined by the cutoff in momentum
space mentioned above.
\begin{equation}
	\H =
	\left(
	\begin{array}{cc}
		\p\H\p & \p\H\q \\
		\q\H\p & \q\H\q \\
	\end{array}
	\right)
	\label{eq:matr1}
\end{equation}

The Lee and Suzuki iterative method involves choosing $\x$ as
\begin{align}
	\x = e^{\w},
\end{align}
where $\w$, not to be confused with $\omega$ used in the previous section, is
defined to have the relation
\begin{align}
	\w = \q\w\p.
\end{align}
Using this definition of $\w$, $\x$ has the properties $\x = 1+\w$ and
$\x^{-1} = 1-\w$.

When we require $\H$ (\ref{eq:matr1}) to be block diagonal and set in for $\x$,
we get
\begin{align}
	\q\H\p = \q\h\p-\w\h\p+\q\h\w-\w\h\w = 0.
	\label{eq:QHP}
\end{align}
We then need to solve this non-linear equation for $\w$.

When $\w$ is found, the Hamiltonian projected down on the model space $\p\H\p$
becomes
\begin{align}
	\he = \p\h\p+\p\h\w = \p\h\p+\p\v\w.
	\label{eq:Heff}
\end{align}

We now do another similarity transformation $\u$ on $\he$ \citep{VlowkU}.
\begin{align}
	\u = (1+\w-\w^\dagger)(1+\w\w^\dagger+\w^\dagger\w)^{-\frac12}
	\label{eq:U}
\end{align}
We want the transformed Hamiltonian to be diagonal so that we can easily separate the interaction term from the kinetic term.
\begin{align}
	\vv = \u^{-1}(\h_0+\v)\u-\h_0
	\label{eq:V1}
\end{align}

Using equations \ref{eq:Heff}, \ref{eq:U} and \ref{eq:V1} the effective
interaction in the model space becomes
\begin{align}
	\ve = (\p+\w^\dagger\w)^{\frac12}(\p\h\p+\p\v\w)(\p+\w^\dagger\w)^{-\frac12}-\p\h_0\p.
\end{align}
To solve this equation we need to calculate the matrix
$(\p+\w^\dagger\w)^{\frac12}$, where $\w$ is found through equation
\ref{eq:QHP}.

I will now very shortly mention how to transform to the momentum space. For
further details, see \citep{G-matrix}.

We start by solving the Schr\"{o}dinger equation in relative momentum space
\begin{align}
	\int dk'k'^2\bra kT+V\ket{k'}\braket{k'}{\psi_\alpha} = E_\alpha\braket{k}{\psi_\alpha},
\end{align}
where we have used the completeness relation.

We discretize the Schr\"{o}dinger equation to get a matrix equation,
\begin{align}
	\sum_\gamma\varpi_\gamma k_\gamma^2\bra{k_\delta}T+V\ket{k_\gamma}\braket{k_\gamma}{\psi_\alpha} = E_\alpha\braket{k_\delta}{\psi_\alpha},
\end{align}
where $k_\gamma$ are the integration points and $\varpi_\gamma$ are the
corresponding quadrature weights.

To get a hermitian matrix we introduce $\ket{\bar k_\delta} =
k_\delta\sqrt{\varpi_\delta}\ket{k_\delta}$. After some calculation one gets
the effective interaction in the original basis $\ket{k_\delta}$,
\begin{align}
	\bra{k_\delta}\ve\ket{k_\gamma} = \frac{\bra{\bar k_\delta}\ve\ket{\bar k_\gamma}}{\sqrt{\omega_\delta\omega_\gamma}k_\delta k_\gamma}.
\end{align}

Thereafter, we transform this interaction to a harmonic oscillator basis in the
laboratory frame. This interaction is used in our calculations.

%Many of the formulas used for the $G$-matrix method also hold for the Vlowk
%method, and vice versa. But the calculations are done in different basis's

\section{Hartree-Fock}

The program has a Hartree-Fock (HF) option. If this option is chosen, then it
will renormalize the wavefunction with the Hartree-Fock method and use the new
HF wavefunction in the perturbation theory. !Usikker paa siste setningen her.!

The purpose of the Hartree-Fock method is to replace the interaction $V$ with
an auxiliary potential $U$ where we replace the interaction one particle feels
from all the other A-1 particles by one potential for the A-1 particles.

We can ease our calculations by introducing an auxiliary potential $U$ and
rewriting the Hamiltonian
\begin{align}
	H = T + U + V - U = H_0 + H_1,
\end{align}
where we want to choose a $U$ so that $H_1$ is small.

The Hartree-Fock method is about finding one such $U$. The $U$ chosen in the
Hartree-Fock method is the average interaction between one nucleon and all the
other nucleons. The energy eigenvalue equation can then be rewritten as a
one-particle equation,
\begin{align}
	[T(x) + U_1^{HF}(x)]\phi_i(x) + \int dx'U_2^{HF}(x,x')\phi_i(x') =
	\varepsilon_i^{HF}\phi_i(x),
	\label{eq:HF}
\end{align}
where
\begin{align}
U_1^{HF}(x) = \sum_{j=1}^n\int dx'\phi_j^*(x')V_{ij}(x,x')\phi_j(x'),\\
U_2^{HF}(x,x') = -\sum_{j=1}^n\phi_j^*(x')V_{ij}(x,x')\phi_j(x),
\end{align}
and $\varepsilon_i^{HF}$ is the Hartree-Fock energy eigenvalue for particle
$i$.

To calculate equation \ref{eq:HF} we need to know $\phi_i(x)$. In the
Hartree-Fock method we choose an initial trial wavefunction $\phi_i(x)$ and
initial position $x$ of the nucleons. We then find the energy eigenvalues and
eigenstates. The next step is to compare the old wavefunction $\phi_i(x)$ with
the new one. If they're not sufficiently equal, then we repeat the calculations
of equation $\ref{eq:HF}$ with the new $\phi_i(x)$. Then we compare that and so
on until $\phi_i(x)_{old} \approx \phi_i(x)_{new}$.

For a more detailed discussion of the Hartree-Fock method, see \citet{HF}.

After these calculations, we employ many-body perturbation theory to third
order. 


\section{Perturbative many--body approaches}

Finally, we briefly sketch how to calculate an effective interaction
in terms of the $G$--matrix.
The first step here is to define the so--called $\hat{Q}$--box given by
\begin{equation}
   P\hat{Q}P=PH_1P+
   P\left(H_1 \frac{Q}{\omega-H_{0}}H_1+H_1
   \frac{Q}{\omega-H_{0}}H_1 \frac{Q}{\omega-H_{0}}H_1 +\dots\right)P,
   \label{eq:qbox}
\end{equation}
where we will replace $H_1$ with $G-U$ ($G$ replaces the free NN interaction
$V$) or the Vlowk renormalized effectige interaction.
The $\hat{Q}$--box\footnote{The $\hat{Q}$--box should not be confused
with the exclusion operator $Q$.}
is made up of non--folded diagrams which are irreducible
and valence linked. A diagram is said to be irreducible if between each pair
of vertices there is at least one hole state or a particle state outside
the model space. In a valence--linked diagram the interactions are linked
(via fermion lines) to at least one valence line. Note that a valence--linked
diagram can be either connected (consisting of a single piece) or
disconnected. In the final expansion including folded diagrams as well, the
disconnected diagrams are found to cancel out \cite{g261}.
This corresponds to the cancellation of unlinked diagrams
of the Goldstone expansion \cite{g261}.
We illustrate
these definitions by the diagrams shown in fig.~\ref{fig:diagsexam},
where an arrow pointing upwards
(downwards) is a particle (hole) state.
\begin{figure}[hbtp]
   \setlength{\unitlength}{1cm}
 \begin{center}
   \begin{picture}(7,3.5)
%%%\put(0,0){\framebox(7,3.5){}}
  \put(0,0){\epsfxsize=7cm,\epsfbox{qbox1.eps}}
   \end{picture}
  \end{center}
\caption{Different types of valence--linked diagrams. Diagram (a)
is irreducible and connected, (b) is reducible, while (c) is irreducible
and disconnected.}
\label{fig:diagsexam}
\end{figure}
Particle states outside the model space are given by railed lines.
Diagram (a) is irreducible, valence linked and connected,
while (b) is reducible since
the intermediate particle states belong to the model space.
Diagram (c) is irreducible, valence linked and disconnected.

We can then obtain an effective interaction
$H_{eff}$ in terms of the $\hat{Q}$--box,
with \cite{g261}
\begin{equation}
    H_{eff}^{(n)} = \omega + \hat{Q}+{\displaystyle\sum_{m=1}^{\infty}}
    \frac{1}{m!}\frac{d^m\hat{Q}}{d\omega^m}\left\{
    H_{eff}^{(n-1)} - \omega \right\}^m .
    \label{eq:fd}
\end{equation}
Observe also that the
effective interaction $H_{eff}^{(n)}$
is evaluated at a given model space energy
$\omega$, as is the case for the $G$--matrix as well. We choose this
starting energy to be $-20$ MeV.
Moreover, although $\hat{Q}$ and its derivatives contain disconnected
diagrams, such diagrams cancel exactly in each order \cite{g261}, thus
yielding a fully connected expansion in eq.\ (\ref{eq:fd}).
The first iteration is then given by
\begin{equation}
   H_{eff}^{(0)} =  \omega + \hat{Q}.
\end{equation}
We define the $\hat{Q}$--box to consist of all two--body
diagrams\footnote{With two--body we also mean one--body diagrams like
diagram (a) in fig.~\ref{fig:qbox} with a spectator valence line.}
through third order in the $G$--matrix, as shown in
ref.~\cite{g261}. Typical examples of diagrams which are included
in the $\hat{Q}$--box are shown in fig.~\ref{fig:qbox}. The summations
over intermediate states is restricted to excitations
of $2\hbar\Omega$ ($\hbar\Omega$ is the oscillator energy)
in oscillator energy, an approximation which has
been found to be appropriate if one uses a potential with a weak
tensor force like the Bonn A potential. The reader should note that
diagram (c) in fig.~\ref{fig:qbox} is not included in
the $\hat{Q}$--box. Three--body
and other many--body effective contributions may be of importance
in the study of spectra of nuclei with more than two valence
nucleons. These contributions will be studied by us in future works.
%
\begin{figure}[hbtp]
   \setlength{\unitlength}{1cm}
  \begin{center}
   \begin{picture}(7,3)
%%     \put(0,0){\framebox(7,3){}}
      \put(0,0){\epsfxsize= 7cm,\epsfbox{qbox2.eps}}
\end{picture}
\end{center}
\caption{Examples of diagrams included in the $\hat{Q}$--box.
Diagram (a) is a one--body diagram, whereas diagram (b) is a two--body
diagram. Diagram (c) is an effective three--body diagram which is not
included in our definition of the $\hat{Q}$--box.}
\label{fig:qbox}
\end{figure}
%

Another iterative scheme which has been much favored in the literature
is a method proposed by Lee and Suzuki (LS). The effective interaction
we will employ in this work has been obtained using the LS method, which
gives the following expression for the effective interaction
%
\begin{equation}
H_{eff}^{(n)} =  \omega + \left[1-\hat{Q}_{1}-\sum_{m=2}^{n-1}\hat{Q}_{m}
\prod_{k=n-m+1}^{n-1} \left ( H_{eff}^{(k)} - \omega \right )
              \right]^{-1}\hat{Q},
\end{equation}
where
\begin{equation}
\hat{Q}_{m}=\frac{1}{m!}\frac{d^m\hat{Q}}{d\omega^m}.
\end{equation}
%



In the next chapter we will discuss the method we use to diagonalize the
Hamiltonian.
