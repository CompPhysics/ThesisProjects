\begin{figure}[htbp]
\setlength{\unitlength}{3.0cm}
\begin{center}
\begin{picture}(6.9,6.5)(0,-1)
\psset{xunit=1.00cm, yunit=6.0cm}
\newcommand{\drawlevel}[5]{\psline[origin={#1,#2}, linewidth=0.3pt](0,0)(1.5,0.0)\rput(#3,#4){\scriptsize \makebox(0,0){$#5$}}}
\newcommand{\connect}[4]{\psline[linewidth=0.3pt, linestyle=dotted, dotsep=1.2pt](#1,#2)(#3,#4)}
\thicklines
\rput(0.1,-0.2){\makebox(0,0){{\large Full 0p3-0d3}}}
\drawlevel{0.7}{-0}{2}{0}{\ \frac{3}{2}^- \ (0)}
\drawlevel{0.7}{-0.986992}{2}{0.986992}{\ \frac{5}{2}^- \ (2.22^)}
\drawlevel{0.7}{-1.76563}{2}{1.76563}{\ \frac{7}{2}^- \ (3.97^)}
\drawlevel{0.7}{-2}{2}{2}{\ \frac{3}{2}^- \ (4.50^)}

\rput(4.3,-0.2){\makebox(0,0){{\large Reduced 0p3-0d3}}}
\drawlevel{-3.5}{-0}{6.2}{0}{\ \frac{3}{2}^- \ (0)}
\drawlevel{-3.5}{-0.991617}{6.2}{0.991617}{\ \frac{5}{2}^- \ (2.23^)}
\drawlevel{-3.5}{-1.77248}{6.2}{1.77248}{\ \frac{7}{2}^- \ (3.99^)}
\drawlevel{-3.5}{-1.9928}{6.2}{1.9928}{\ \frac{3}{2}^- \ (4.48^)}

\rput(8.5,-0.2){\makebox(0,0){{\large 0p3-0f7}}}
\drawlevel{-7.7}{-0}{10.4}{0}{\ \frac{3}{2}^- \ (0)}
\drawlevel{-7.7}{-0.654422}{10.4}{0.654422}{\ \frac{5}{2}^- \ (1.47^)}
\drawlevel{-7.7}{-1.42696}{10.4}{1.42696}{\ \frac{7}{2}^- \ (3.21^)}
\drawlevel{-7.7}{-1.92084}{10.4}{1.92084}{\ \frac{1}{2}^- \ (4.32^)}

\rput(12.7,-0.2){\makebox(0,0){{\large exp. data}}}
\drawlevel{-11.9}{-0}{14.6}{0}{\ \frac{3}{2}^- \ (0)}
\drawlevel{-11.9}{-0.594161}{14.6}{0.594161}{\ \frac{5}{2}^- \ (1.34^)}
\drawlevel{-11.9}{-1.2199}{14.6}{1.2199}{\ \frac{7}{2}^- \ (2.74^)}
\connect{13.4}{1.2199}{13.7}{1.2199}

\end{picture}
\end{center}
\caption{Energy spectra for 15B, in the full 0p3-0d3 and reduced 0p3-0f7 model spaces. Reduced 0p3-0d3 means that we have reduced our 0p3-0d3 model space so that we have a maximum of 4 particles in the 0d5 orbital and a maximum of 2 particles in the 0d3 orbital. Reduced 0p3-0f7 model space we mean that we have reduced our 0p3-0f7 model space so that we have a maximum of 4 particles in the 0d5 orbital, a maximum of 2 particles in the 0d3 orbital and a maximum of 2 particles in the 0f7 orbital. The energies are given in MeV.}
\label{fig:15B}
\end{figure}



