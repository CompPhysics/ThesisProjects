\documentclass{article}
\usepackage{graphicx} % Required for inserting images
\usepackage{hyperref}

\title{Time-dependent many-body methods for quantum technologies}
\author{H\aa kon Kvernmoen}
\date{January 2024}

\begin{document}

\maketitle  

\section{Scientific aims}


The aim of this PhD project is to develop many-body theories for
studies of the time evolution of many-body systems, with an emphasis
on fermionic systems.


For interacting many-particle systems
where the degrees of freedom increase exponentially, quantum
mechanical many-body methods like Coupled Cluster theory, Green's
function theories, density functional theories and other, play a
central role in understanding experiments in a wide range of fields,
spanning from atomic and molecular physics and thereby quantum chemistry to
condensed matter physics, materials science, nano-technologies and
quantum technologies and finally processes like fusion and fission in
nuclear physics. This list is definitely not exhaustive as the are many other
areas of applications for quantum mechanical many-body theories.

This project focuses in particular on Coupled Cluster (CC) theory
which is a computationally efficient approximation to the physics of
the quantum many-body systems typically studied in many of the
above-mentioned systems.  At the university of Oslo there are two main
groups who have studied and developed CC theory, the Center for
Computing in Science Education at the department of Physics, with an
emphasis on nuclear physics and quantum technologies, and the
Hylleraas Centre for Quantum Molecular Sciences at UiO, with an
emphasis on quantum chemistry. Both centers have strong collaborations
through the exchange and supervision of students and cross-collaborations.
Researchers at both centers collaborate with many other theorists
worldwide.

However, CC theory (as well as most other many-boy theories) for the
time evolution (or dynamics) of systems has been computationally
out-of-reach until recently and is hence less developed. Furthermore
dynamical CC theory is general to all quantum many-body systems
(within its valid regime), and so it is of current interest to apply
the theory to fusion processes within nuclear physics as well as
entangled qubit systems relevant to quantum computing which are
currently under research at the Center for Materials Science and
Nanotechnologies at the University of Oslo, as well as in many other
places.  Because nuclear systems involve three-body interactions which
do not appear in other areas of physics, the CC theory of triples
correlations (CCSDT theory) is of particular relevance. The CCSDT
simulations are computationally more demanding than conventional
quantum chemical (CCSD) simulations, so good approximations and
implementations of this theory will have to be developed. These
correlations are also important for other many-body systems.

This PhD project aims to contribute to the development of dynamical CC
theory in its standard form as well as to CCSDT theory by developing
suitable optimalisations. The developed theory can be used to study
systems of relevance for quantum technologies, nuclear physics
reactions as well as in quantum chemistry.

The main focus of the applications will be directed towards the use of
quantum many-body theory applied to candidate systems for realizing
quantum circuits and gates, that is towards quantum technologies and
the realization and studies of systems of relevance for such
technologies. In particular we will analyze and study properties like
the time evolution of entanglement and how to realize quantum gates
for systems of electrons confined in one, two and three dimensions,
so-called quantum dot systems \cite{Reimann2002} as well as point
defects in semi-conductors.

These systems are studied experimentally
at the university of Oslo at the Center for Materials Science and
Nanotechnologies (SMN). The theoretical activity at the Center for
Computing in Science Education and the Computational Physics research
group have through the last years developed a strong collaboration
with several researchers at the SMN. An example of such studies can be found in Ref.~\cite{us2024}.

The present research proposals, in addition to the developments in
many-body theories, aims at strengthening the collaboration between
many-body theories for quantum technologies and the experimental
activity at the SMN.


Secondly, we expect to further advance our collaboration with the
Hylleraas center, with potential applications to systems in quantum
chemistry.

Finally, the methods to be developed can also be applied to systems in
nuclear physics, and a possible first target is the study of fusion of
two alpha-particles. 


\begin{thebibliography}{99}

\bibitem{Reimann2002} Stephanie M. Reimann and Matti Manninen, Reviews of  Modern Physics {\bf 74}, 1283 (2002), \url{https://doi.org/10.1103/RevModPhys.74.1283}
\bibitem{us2024} Niyaz R. Beysengulov, Johannes Pollanen, Øyvind S. Schøyen, Stian D. Bilek, Jonas B. Flaten, Oskar Leinonen, Håkon Emil Kristiansen, Zachary J. Stewart, Jared D. Weidman, Angela K. Wilson, Morten Hjorth-Jensen, Coulomb interaction-driven entanglement of electrons on helium, PRX Quantum, in press and \url{https://arxiv.org/abs/2310.04927}
\end{thebibliography}  


\end{document}




