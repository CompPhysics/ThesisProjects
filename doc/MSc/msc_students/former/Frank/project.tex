\documentstyle[a4wide]{article}
\newcommand{\OP}[1]{{\bf\widehat{#1}}}

\newcommand{\be}{\begin{equation}}

\newcommand{\ee}{\end{equation}}

\begin{document}

\pagestyle{plain}

\section*{Thesis title: Full Configuration Interaction Theory for Many-body Systems}

The aim of this thesis is to develop, based on our existing codes \cite{torgeir,simen},
a fully parallel Configuration Interaction (CI) code. The code will be 
flexible enough that it can handle
systems spanning from quantum dots in two and three dimensions, atoms and molecules and nuclei. 

A parallel version written in MPI of our nuclear physics CI code 
exists already, see Ref.~\cite{torgeir}.
The parallelization has been done in MPI and we plan to extend this to OpenMP and possibly also
OpenCL and GPU programming.  Similarly, a full CI code tailored to quantum dots has been developed
by Simen Kvaal \cite{simen}. This code is not parallelized yet.

The first step is thus to merge these two codes, 
based on the iterative Lanczos method \cite{glasgow,massimo}, 
in order to have a more general code that can tackle inputs from several physical systems. 
The nuclear physics CI code is written in C and needs to be upgraded to C++ in order to allow for more
general structures. Furthermore, the nuclear physics code 
has a limited word length for Slater determinants
of 64 bits. The number of single-particle states that can then be included is limited to small 
numbers, typically between 12-30. This needs also to be extended to arbitrarily 
long single-particle bases. 

The final aim is to obtain a flexible CI solver which can be used to tackle 
different physical systems
and be used on various architectures, from homogenous clusters to heterogenous ones.

The applications of the code are many. We will in particular select the chain of carbon isotopes
from $^{14}$C to $^{22}$C using two major oscillator shells. The aim is to test how these isotopes behave as we move toward the neutron drip line, studying both ground and excited states properties and
various transition operators as well.
Another application is to run quantum dots calculations for systems containing up to 8-10 electrons in up to 8-10 major shells. These results will be compared with our Monte Carlo codes and coupled-cluster
codes. The results from these calculations will most likely be published in the Physical Review B and C.

\section*{Progress plan and milestones}
The aims and progress plan of this thesis are as follows
\begin{itemize}
\item Spring 2012: Rewrite the nuclear physics CI code in C++ using both OpenMP and MPI. Extend the 
length of a word needed to characterize a Slater determinant to arbitrary length.  
Make the code flexible enough so that it can handle several physical systems. 
Consider eventually (if time) an extension to OpenCL and GPUs.
 \item Fall 2012:  Write up of thesis and analysis and discussion of results. 
\end{itemize}
 


The thesis is expected to be handed in November 1 2012.

\begin{thebibliography}{999}
\bibitem{torgeir} Torgeir Engeland, code Peer Gynt, University of Oslo, unpublished.
\bibitem {simen}  Simen Kvaal, PhD Thesis, University of Oslo (2009); preprint arXiv:0810.2644 (2008).
\bibitem{glasgow} R.~R.~Whitehead {\em et al.}, Adv.~Nucl.~Phys.~{\bf 9}, 123 (1977)
\bibitem{massimo} Massimo Rontani {\em et al}, J.~Chem.~Phys.~{\bf 124}, 124102 (2006).

\end{thebibliography}



\end{document}



