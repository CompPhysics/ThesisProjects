% Appendix A

\chapter{Diagram names} % Main appendix title

\label{AppendixA} % For referencing this appendix elsewhere, use \ref{AppendixA}

\lhead{Appendix A. \emph{Diagram names}} % This is for the header on each page - perhaps a shortened title

Because of the way we generated the diagrams in the CCDT equations, there are some inconsistencies concerning how the diagrams are named between chapters \ref{Chapter5} and \ref{Chapter7}. While we use the same naming convention as Shavitt and Bartlett in Ref.~\cite{ShavittBartlett2009} for chapter \ref{Chapter5}, we move to our own naming in chapter \ref{Chapter7}. 

All codes written for this thesis use the latter naming convention, as
these codes at an early stage inherited such names from the
calculation with CCAlgebra. It should also be noted that even in
Ref. \cite{ShavittBartlett2009}, breach of naming convention occurs
when the diagrammatic approach is introduced for the CCD
equation.\footnote{They change from $L_{1a} \rightarrow D_{2a}$ for
  all linear terms, as well as $Q_n \rightarrow D_{xn}$ for the
  quadratic contributions to the CCD equation.}

For this reason we provide this table \ref{tab:diagramtranslation},
that will make it easier to translate between the two naming
conventions.

\begin{table}[h]
\caption{Translation of diagram names}
\begin{center}
\begin{threeparttable}
\begin{tabular}{l l l}
    \toprule
Name in Ref.\cite{ShavittBartlett2009} & Names used in chapter \ref{Chapter7} \\ \hline 
$T_{1a}$ & $(t_2)_a$ \\
$T_{1b}$ & $(t_2)_b$ \\ \hline
$T_{2c}$ & $(t_3)_a$\\
$T_{2d}$ & $(t_3)_b$\\
$T_{2e}$ & $(t_3)_c$\\ \hline
$T_{3b}$ & $(t_2t_2)_b$\\
$T_{3c}$ & $(t_2t_2)_c$\\
$T_{3d}$ & $(t_2t_2)_d$\\ \hline
$T_{5a}$ & $(t_2t_3)_a$\\
$T_{5b}$ &$(t_2t_3)_b$ \\
$T_{5c}$ & $(t_2t_3)_c$\\
$T_{5d}$ &$(t_2t_3)_d$ \\
$T_{5e}$ & $(t_2t_3)_e$\\
$T_{5f}$ &$(t_2t_3)_f$ \\
$T_{5g}$ &$(t_2t_3)_g$ \\
\bottomrule
\end{tabular}
\begin{tablenotes}
\end{tablenotes}
\end{threeparttable}
\end{center}
\label{tab:diagramtranslation}
\end{table}
