% Chapter Template

\stepcounter{section}

\chapter{Introduction} % Main chapter title

\label{Preface} % Change X to a consecutive number; for referencing this chapter elsewhere, use \ref{ChapterX}

\lhead{\emph{Introduction}} % Change X to a consecutive number; this is for the header on each page - perhaps a shortened title

%----------------------------------------------------------------------------------------
%	Theory
%----------------------------------------------------------------------------------------

In this thesis we will use the coupled cluster method, also called coupled cluster \emph{theory}, to investigate the homogeneous electron gas. More specifically, we will investigate how inclusion of triple excitations in the coupled cluster equations will affect calculations of the ground state energy.  Numerous (see for example Refs.~\cite{Baardsen2014, Shepherd2012, Roggero2013}) calculations of the system in question has been made using the coupled cluster doubles (CCD) method, doubles with perturbative triples (CCD(T)), and various other many-body methods, but to the author's best knowledge this thesis presents for the first time results containing triple excitations for the electron gas.

We shall systematically investigate how each diagram present in the equation contributes to the ground state energy for smaller basis sets, incrementally leading up the the full inclusion of triple excitations (CCDT), and we will estimate the energy in the thermodynamical limit using a CCDT-1 \cite{ShavittBartlett2009} approach and compare these results with existing results from full configuration interaction quantum monte carlo (FCIQMC) present in the literature.

In the second chapter of this thesis \ref{Chapter1}, we review the context from which the many-body problem arise, and we motivate the need for the formalism developed in Chapters \ref{Chapter2} and \ref{Chapter3}. In Chapter \ref{Chapter4}, we give a broad overview of the methodology involved, and we give some special attention to the methods that closely relates to the coupled cluster method, such as Hartree-Fock theory, configuration interaction theory and many-body perturbation theory.

In Chapter \ref{Chapter5}, we derive the coupled cluster method, and we introduce both diagrammatic techniques and computational implementations of these that we use to derive the actual equations for various truncations of the coupled cluster method.

The homogeneous electron gas (HEG) is reviewed in Chapter \ref{Chapter6}, where we derive and present expressions needed to evaluate the coupled cluster equations, such as the the plane wave basis, Fock matrix elements \cite{Thijssen}, the two-body interaction \cite{ShavittBartlett2009} and the general structure of the model. We also place this thesis into context with newly published material on the system.

In Chapter \ref{Chapter7} we present implementational details, and we describe two conceptually different schemes for solving the coupled cluster equations for our system. We also discuss topics such as performance and memory usage.

In the final Chapter \ref{Chapter8}, we validate the implementations by comparing with results published in other studies and by comparing results from the two independent solvers. We then perform a series of calculations for smaller basis sets, ending up with the full inclusion of triple amplitudes. Estimates in the thermodynamical limit is performed by extrapolation from a data set obtained by running our CCDT-1 code on the Abel cluster (see Ref.~\cite{abel}).

We discuss these findings in light of calculations performed by others, and we finally give some concluding remarks, recommendations and future prospects.












