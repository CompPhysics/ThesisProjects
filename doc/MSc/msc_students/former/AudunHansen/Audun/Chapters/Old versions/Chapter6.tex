% Chapter Template

\chapter{The Homogenous Electron Gas} % Main chapter title

\label{Chapter6} % Change X to a consecutive number; for referencing this chapter elsewhere, use \ref{ChapterX}

\lhead{Chapter 7. \emph{The Homogenous Electron Gas}} % Change X to a consecutive number; this is for the header on each page - perhaps a shortened title

%----------------------------------------------------------------------------------------
%	SECTION 1
%----------------------------------------------------------------------------------------

\section{The Homogenous Electron Gas}

The homogenous electron gas (HEG) is a system where free electrons
interact with each other and a uniformly distributed background
charge \cite{GrossRungeHeinonen}. The model is also known as the
\emph{Jellium Model} or the \emph{Free Electron  Gas} \cite{GrossRungeHeinonen}, and is currently a frequently
studied system within many-body physics \cite{Shepherd2014,Baardsen2014, Roggero2013, Shepherd2012}.

Since the background charge is uniformly distributed, the model mainly
focuses on effects due to interactions between the electrons. The
model will in some sense be valid for systems where the electrons are
weakly bound to the nuclei, such as periodic lattices with closed
shells and weakly bound valence electrons \cite{GrossRungeHeinonen}.

A very similar treatment as we apply to the HEG system by expanding it
in a plane wave basis will also be applicable to infinite, homogenous
nuclear matter with minor alterations \cite{Baardsen2014}, and
may provide insights into properties of supernova explosions
\cite{burrows2013} and neutron stars \cite{weber1999,hh2000}.

Some of the earliest treatments of the HEG using CC were performed in
the 1970's by Singal and Das
\cite{Singal1973}, Freeman \cite{Freeman1977} and Bishop
together with Luhrmann \cite{Bishop1978,Bishop1982}.



\section{The Hamiltonian}

The Hamiltonian for the HEG is, see for example Ref.~\cite{GrossRungeHeinonen},
\begin{equation}
\hat{H} = \hat{H}_e + \hat{H}_{eb} + \hat{H}_{bb},
\label{eqn:h_HEG}
\end{equation}
where $\hat{H}_e$ relates to electron part only, $\hat{H}_{eb}$ is the
interaction with the positive background and $\hat{H}_{bb}$ is the
interaction between the background charges. This last term is
analogous to the term that would be "frozen out" by the
Born-Oppenheimer approximation if we were working with molecules or atoms, but in
our case it will simply turn out to be a constant due to the
uniformity of th density $\rho$ \cite{GrossRungeHeinonen}, that is
\begin{equation}
\hat{H}_{bb} = \frac{e^2}{2}  \int_{\Omega} d^3R \int_{\Omega} d^3R' \frac{\rho_i(\bold{R})\rho_i(\bold{R}')}{\vert \bold{R}-\bold{R}'\vert} = 
\hat{H}_{bb} = \frac{e^2}{2}  \int_{\Omega} d^3R \int_{\Omega} d^3R' \frac{(\frac{\Omega}{N})^2}{\vert \bold{R}-\bold{R}'\vert},
\label{eqn:h_b}
\end{equation}
where $e$ is the  charge of the electron, $\omega$ is the volume and $N$ is the number of charged ions in the background. 

The Hamiltonian associated with the electrons consists of the kinetic energy of the electrons and their interactions is given as
\begin{equation}
\hat{H}_e = \sum_{i} \frac{\hat{p_i}}{2n} + \frac{1}{2} \sum_{i \neq j} \frac{e^2}{\vert \bold{r}_i - \bold{r}_j \vert }.
\label{eqn:h_electron}
\end{equation}
The interaction between the electrons and the background charge may be written as
\begin{equation}
\hat{H}_{eb} = - \sum_{i} \int_\Omega d^3R \frac{N}{\Omega} \frac{e^2}{\vert \bold{r}_i - \bold{R} \vert }.
\label{eqn:h_eb}
\end{equation}

\section{Ewald's summation technique}

The electrons repel each other through the coulomb force, which has the general form
\begin{equation}
 \frac{1}{2} \sum_{i \neq j} \frac{1}{\vert \bold{r}_i - \bold{r}_j \vert },
\label{eqn:repulsion}
\end{equation}
where we have employed atomic units, that is $\hbar = c = e = 1$.

This expression is not convergent for an infinite number of particles, but in cases where the net charge of the system is neutral we may use \emph{Ewald's summation technique} to make this energy convergent.

The error function is defined \cite{rottmann}
\begin{equation}
erf(x) \equiv \frac{2}{\sqrt{\pi}} \int_{0}^x dt e^{-t^2},
\end{equation}
while the complementary error function is defined \cite{rottmann} as
\begin{equation}
erfc(x) \equiv 1 - erf(x) = \frac{2}{\sqrt{\pi}} \int_{x}^\infty dt e^{-t^2} .
\end{equation}

Ewald \cite{Ewald1921} found that 
\begin{equation}
\frac{1}{r} = \frac{erf( \frac{1}{2} \sqrt{\eta} r) }{r} + \frac{ erfc(\frac{1}{2} \sqrt{\eta} r) }{r},
\end{equation}
which allows us to rewrite the electronic repulsion as
\begin{equation}
 \frac{1}{2} \sum_{i \neq j} ( \frac{erf( \frac{1}{2} \sqrt{\eta} r_{ij}) }{r_{ij}} + \frac{ erfc(\frac{1}{2} \sqrt{\eta} r_{ij}) }{r_{ij}}),
\label{eqn:ewaldsum}
\end{equation}
where we have  defined
\begin{equation}
r_{ij} \equiv \vert \bold{r}_i - \bold{r}_j \vert .
\end{equation}

\section{The Ewald interaction}

It may be shown that the interaction between the electrons and the
background, as well as interactions among the background charges
vanish when using Ewald's summation technique \cite{Fraser1996}, and
we will end up with the interaction energy for the three dimensional
HEG \cite{Drummond2008} as
\begin{equation}
v_{E}(\bold{r}) = \sum_{k \neq 0} \frac{4 \pi}{L^3 k^2 } e^{i \bold{k} \bullet \bold{r}}  e^{\frac{-\eta^2 k^2}{4} } + \sum_\bold{R} \frac{1}{\bold{r}- \bold{R}} erfc(\frac{\vert \bold{r}- \bold{R} \vert }{\eta}) - \frac{\pi \eta^2}{L^3 } ,
\label{eqn:HEG_interaction}
\end{equation}
where $L$ is the length of one side in the simulation cell and the vector

\begin{equation}
\bold{R} = L (n_z \bold{u}_z + n_y \bold{u}_y + n_z \bold{u}_z),
\end{equation}

is used to refer to all simulation cells in real space. The quantity $\bold{k}$ represents the momentum vector, while $\bold{r}$ is the position vector for each electron \cite{Baardsen2014}.

The parameter $\eta$ allows us to gradually vary the amount of the 
different terms. We shall take it to be infinitesimally small and
positive, as in \cite[p.97]{Baardsen2014}, computing thereby the 
interaction in momentum space. This results in the
following expression for the interaction
\begin{equation}
v_{E}(\bold{r}) = \sum_{k \neq 0} \frac{4 \pi}{L^3 k^2} e^{i \bold{k} \bullet \bold{r}}  .
\label{eqn:HEG_interaction2}
\end{equation}

\section{The antisymmetric matrix elements}

The antisymmetric matrix elements for the three dimensional HEG are \cite{Baardsen2014}
\begin{multline}
\langle pq \vert \vert rs \rangle = \frac{4\pi}{L^3} \delta_{\bold{k}_p + \bold{k}_q , \bold{k}_r + \bold{k}_s} \big{(}  \delta_{m_{s_p}, m_{s_r}}  \delta_{m_{s_q}, m_{s_s}} (1-\delta_{k_p, k_r}) \frac{1}{\vert \bold{k}_r -\bold{k}_p \vert^2 } \\
-\delta_{m_{s_p}, m_{s_s}}  \delta_{m_{s_q}, m_{s_r}} (1-\delta_{k_p, k_s}) \frac{1}{\vert \bold{k}_s -\bold{k}_p \vert^2 } 
 \big{)}.
\end{multline}

The antisymmetric matrix elements may also be defined for the two dimensional case \cite{Baardsen2014}, but the focus of this thesis is on the three-dimensional electron gas.

\section{The Hartree-Fock energy}

For the electron gas the reference energy is  \cite{Baardsen2014}
\begin{equation}
E_{ref} = \sum_i \langle i \vert \hat{h}_0 \vert i \rangle + \frac{1}{2} \sum_{ij} \langle ij \vert  \vert ij \rangle + \frac{1}{2} Av_0 .
\end{equation}

The number $A$ is the number of particles, and the quantity
$v_0$ is the so called Madelung constant. This term describes so-called 
\emph{finite size} effects \cite{Baardsen2014} that are
stronger for small systems. As we increase the number of particle
states towards the \emph{thermodynamical limit}, it will vanish.

\section{The Fock Matrix}

The Fock matrix elements are \cite{Baardsen2014}
\begin{equation}
\langle p \vert f \vert q \rangle = \frac{k_p^2}{2m} \delta_{\bold{k}_p \bold{k}_q} \delta_{m_{s_p} m_{s_q}} + \sum_i  \langle pi \vert \vert qi \rangle 
\end{equation}

\FloatBarrier

\section{The Wigner Seitz radius}

We will not directly use the volume $\Omega = L^3$ in the
implementation. We will instead follow the same procedure as
\cite[p.105]{Baardsen2014}, and calculate it using the dimensionless
quantity $r_s$, implying that
\begin{equation}
\Omega(r_s) = \frac{4 \pi}{3} r_B^3 r_s^3,
\end{equation}
where
\begin{equation}
r_s = \frac{r_1}{r_B},
\end{equation}
and
\begin{equation}
\frac{\Omega}{N} = \frac{4\pi}{3} r_1^3,
\end{equation}
Where $N$ is the number of electrons. The quantity $r_s$ is called the \emph{Wigner Seitz} radius, and is interpreted as an effective 
or mean distance between the electrons. The density increase as $r_s$ becomes smaller, and $r_s \leq 1.5$ is sometimes referred to as \emph{the high-density regime} (see for example Ref.~\cite{Shepherd2013}).

\section{The plane wave basis}

We will expand the system in a plane wave basis\footnote{We assume the reader to be familiar with the notion of a plane wave basis as this 
forms the basis for the simplest possible quantum mechanical system taught in introductory quantum physics courses.} for the finite
volume $\Omega = L^3$, 
with single-particle basis functions defined as
\begin{equation}
\psi_{{\bf k}m_s}({\bf r})= \frac{1}{\sqrt{\Omega}}\exp{(i{\bf kr})}\xi_{m_s}.
\end{equation}
The spin orientation is either up or down, and represented by $\xi$.
Each such basis function has an associated single particle energy
\begin{equation}
\epsilon(x,y,z) = \frac{1}{2m} (\frac{2\pi}{L})^2 (n_x^2 +n_y^2 + n_z^2).
\end{equation}
The quantum numbers $n_x$, $n_y$ and $n_z$ allow us to define
so-called magic numbers, as evidenced in table \ref{tab:3shell}.  The
first three shells (resulting in 38 states) of the three dimensional
plane wave basis are shown in table \ref{tab:3shell}. We see that we
have magic numbers corresponding to 2, 14, 38, 54, 66, 114, 162, 186
and so on.


\begin{table}[h]
\caption{The first three shells in the plane wave basis for the three-dimensional homogeneous electron gas}
\label{tab:3shell}
\begin{center}
\begin{threeparttable}
\begin{tabular}{l l l l l l}
    \toprule
Shell & $n_x$ & $n_y$ & $n_z$ & $m_s$ \\ \hline
0 & 0 & 0 & 0 & 1 & \\
 & 0 & 0 & 0 & -1 & \\ \hline
1 & -1 & 0 & 0 & 1 & \\
 & -1 & 0 & 0 & -1 & \\
 & 0 & -1 & 0 & 1 & \\
 & 0 & -1 & 0 & -1 & \\
 & 0 & 0 & -1 & 1 & \\
 & 0 & 0 & -1 & -1 & \\
 & 0 & 0 & 1 & 1 & \\
 & 0 & 0 & 1 & -1 & \\
 & 0 & 1 & 0 & 1 & \\
 & 0 & 1 & 0 & -1 & \\
 & 1 & 0 & 0 & 1 & \\
 & 1 & 0 & 0 & -1 & \\ \hline
2 & -1 & -1 & 0 & 1 & \\
 & -1 & -1 & 0 & -1 & \\
 & -1 & 0 & -1 & 1 & \\
 & -1 & 0 & -1 & -1 & \\
 & -1 & 0 & 1 & 1 & \\
 & -1 & 0 & 1 & -1 & \\
 & -1 & 1 & 0 & 1 & \\
 & -1 & 1 & 0 & -1 & \\
 & 0 & -1 & -1 & 1 & \\
 & 0 & -1 & -1 & -1 & \\
 & 0 & -1 & 1 & 1 & \\
 & 0 & -1 & 1 & -1 & \\
 & 0 & 1 & -1 & 1 & \\
 & 0 & 1 & -1 & -1 & \\
 & 0 & 1 & 1 & 1 & \\
 & 0 & 1 & 1 & -1 & \\
 & 1 & -1 & 0 & 1 & \\
 & 1 & -1 & 0 & -1 & \\
 & 1 & 0 & -1 & 1 & \\
 & 1 & 0 & -1 & -1 & \\
 & 1 & 0 & 1 & 1 & \\
 & 1 & 0 & 1 & -1 & \\
 & 1 & 1 & 0 & 1 & \\
 & 1 & 1 & 0 & -1 & \\
\bottomrule
\end{tabular}
\begin{tablenotes}
\end{tablenotes}
\end{threeparttable}
\end{center}
\end{table}

\FloatBarrier

\section{Recent progress on the Electron gas}

A number of recent publications has sparked an interest in the
homogeneous electron gas. Calculations using Coupled Cluster theory at the level of doubles excitations (CCD)  
have been performed by
Roggero {\em et al.} \cite{Roggero2013} and Shepherd {\em et al.} 
\cite{Shepherd2012}, and to very high precision by Gustav Baardsen
in his PhD thesis \cite{Baardsen2014}. Shepherd {\em et al.} have
frequently published on this topic since 2012, see for example 
Refs.~\cite{Shepherd2012,Shepherd2012a, Shepherd2013, Shepherd2013c, Shepherd2014}, and in 
Ref.~\cite{Shepherd2014}, the authors suggest
a formal connection between the random phase approximation and the 
CCD approach.

Highly accurate calculations using Variational Monte Carlo (VMC) have
also been performed as early as 1980 by Ceperley and Alder
\cite{Ceperley1980}, but also in the recent years 
by Drummond {\em et al.} \cite{Drummond2006}, Shepherd {\em et al.} \cite{Shepherd2012} and
Gurtubay {\em et al.} \cite{Gurtubay2010}.

The system has also been a "hot topic" in the Computational Physics
group at the University of Oslo, and several master theses have touched
upon this subject. From this group, Sarah Reimann has produced IM-SRG(2)
results \cite{Reimann2013}, while Karl Leikanger has produced FCIQMC
\cite{Leikanger2013} results for the tree-dimensional HEG. Similar results as those from
Leikanger are available also from Shepherd {\em et al.} \cite{Shepherd2012}.

Comparisons between CCD and FCIQMC have shown that CCD fails to
account for important correlations in the system \cite{Baardsen2014}.

Shepherd and Grüneis have also preformed CCD(T) calculations on the
system \cite{Shepherd2013}, but they found that the perturbative
treatment of the triples (using Möller-Plesset perturbation theory) resulted in
divergent energies for the HEG. They propose in the same article a
modification that lifts the divergent behavior.

To the present author's knowledge, we have not been able to find any
results beyond the CCD(T) results of Ref.~\cite{Shepherd2013}. This
defines the rationale for this thesis, since a proper treatment of
triples correlations are expected to have a non-negligible effect on
the results for the ground state energy and the equation of state for
the electron gas. The results to be presented here are thus the first
ever studies of triples correlations for the homogeneous electron
gas. Although we will focus on the three-dimensional electron gas, our
codes can easily be applied to the two-dimensional electron
gas. Furthermore, a proper assessment of such correlations has
important consequences for studies of dense nuclear matter, expected
to form the bulk matter of compact objects like neutron stars and
proto-neutron stars. Since the theoretical description of infinite
nuclear matter and/or neutron matter is similar to the electron gas,
the developed formalism can be extended to such studies as well. Our
codes are fully object oriented, allowing thereby for an easy
extension to systems like dense nuclear matter.

