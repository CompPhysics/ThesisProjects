% Chapter Template

\stepcounter{section}

\chapter{Many Body Quantum Theory} % Main chapter title

\label{Chapter1} % Change X to a consecutive number; for referencing this chapter elsewhere, use \ref{ChapterX}

\lhead{Chapter 1. \emph{Many Body Quantum Theory}} % Change X to a consecutive number; this is for the header on each page - perhaps a shortened title

%----------------------------------------------------------------------------------------
%	Theory
%----------------------------------------------------------------------------------------

\paragraph{A brief review} In this chapter we will review the historical context for this thesis in the broad sense, get familiar with some fundamental physical concepts relevant to the thesis, and describe the physical sub-discipline known as "Many Body Quantum Theory".


\section{Many Body Quantum Theory}

Many Body Quantum Theory is the framework which to date best describes and predicts phenomenas relating to interacting quantum system. The main body of theory was developed by physicists such as Fermi, Pauli and Dirac in the late 1920s. As the Davisson-Germer electron diffraction experiment confirmed the particle wave duality of matter in 1927 \cite{Giuliani2005} and the discovery of half integer spin was made by Goudsmit and Uhlenbeck in 1925 \cite{Giuliani2005}, these results found their rationale in the theoretical work of Pauli, Fermi and Dirac \cite{Giuliani2005}. Dirac introduced the \emph{Second Quantization Formalism} in 1927 \cite{ShavittBartlett2009}.

Important contributions has been made continually over the years by others. Feynman introduced the diagrammatic formalism in 1949 \cite{ShavittBartlett2009}, just prior to the advent of many-body perturbation theory as introduced by Brueckner and Levinson (1955) \cite{ShavittBartlett2009}. In the mid 1950's, groundbreaking work on the coupled cluster method was performed by nuclear physicists Coester and Kümmel. \cite{Kummel}

The theory builds on fundamental quantum mechanics with the addition of some further assumptions, a number of more or less overlapping formalisms, and a number of methods for solving - or approximating a solution to - the many-body Schrödinger equation.

Because of the theory`s ability to make highly accurate predictions for quantum systems using only first principles (ab initio), it has also become a central aspect of modern chemistry. The theory covers a broad array of phenomenas on the quantum scale, ranging from hadrons to molecules, solids and periodic systems, material science and life science. 

Although the fundamental physics of the theory has remained mostly unchanged since the 1920s, it is a field in continual development. Computers evolves gradually to enable more comprehensive calculations, and methodological advancements yields new insights and perspectives on the underlying physics.

The theory offers a range of methods for approximating energies and properties of systems. When choosing which method to utilize, one has to consider the trade off between performance and accuracy. While some methods may in principle provide you with the exact solution, they may have computational requirements beyond what is currently achievable. By these considerations, many currently consider the CCSD(T) (Coupled Cluster Singles and Doubles with Perturbative Triples) to be the "gold standard" of ab initio techniques, as it is both efficient and highly accurate.\footnote{The author was not able to pinpoint exactly where this concept of "gold-standard" originated, but a search on Google Scholar with the keywords "coupled cluster gold standard" clearly shows that this terminology is commonly used within the quantum chemistry community.}

In the following sections we will briefly review the theory and notation used in the rest of this thesis. A number of in-depth and excellent modern textbooks are written on this topic by scholars with loads of insights and practical experience, so the aim of this review is merely to introduce concepts and theory that will be utilized in later parts of this thesis. For an extensive introduction to this subject, the reader is referred to one of the many books on the subject. \cite{ShavittBartlett2009, Szabo, Harris, GrossRungeHeinonen, Thijssen}.


%-----------------------------------
%	The Fundamental postulates of Quantum Mechanics
%-----------------------------------
\section{The Postulates of Quantum Mechanics}

At the heart of quantum mechanics lies some fundamental postulates (or assumptions). We will here briefly state these in no particular order.

\paragraph{(1) The Wave Function}
The state of a quantum mechanical system is fully specified in time and space by a wave function $\vert \Psi(x,t) \rangle$. Born's Statistical Interpretation \cite{Griffiths2005} suggests that the probability of finding the system in the volume element $d\tau$ at time $t$ is defined by $\Psi(x,t)^* \Psi(x,t) d\tau$. Another important property is that the wave function should be normalized to 1 in the full occupied space \cite{Griffiths2005}, so that the probability of finding the particle \emph{anywhere} is 1:

$$ \int_\Omega \Psi(x,t)^* \Psi(x,t) d\tau = 1 $$

\paragraph{(2) Observables} For any measurable quantity, such as energy, momentum, or spin, there exists an corresponding linear, hermitean operator. Such operators are commonly denoted with a hat; $\hat{O}$

\paragraph{(3) Measurement} A measurement of any observables $\hat{A}$ on the system, will result in a value $a$, corresponding to the eigenvalues of the equation

$$ \hat{A} \vert \Psi \rangle = a \vert \Psi \rangle $$

\paragraph{(4) Average measurement} For a system in the state $\vert \Psi \rangle$, we may find the average measurement of $\hat{A}$ by

 $$ \int_{\Omega} \Psi(x,t)^* \hat{A} \Psi(x,t) d\tau \equiv \langle  \Psi \vert  \hat{A} \vert \Psi \rangle  =  \langle{ A} \rangle$$

The average measurement is \emph{not the most likely result}, merely the average of a multitude of measurements on identical systems. 

\paragraph{(5) Time evolution} The system will evolve in time in accordance with the time independent Schrödinger equation

\begin{equation}
\hat{H} \vert \Psi(x,t) \rangle = i\hbar \frac{\partial}{\partial t} \vert \Psi(x,t) \rangle
\label{eqn:tisl}
\end{equation}

\paragraph{(6) The Pauli Exclusion Principle} No two indistinguishable \emph{fermions} may occupy the same quantum state. 

While not all textbooks lists The Pauli Principle as a separate postulate, it is a well known experimental fact from which no exempt has been found.\footnote{In "Foundations of Physics" (1957) by Lindsay and Margenau \cite{lindsay} it is even claimed that \emph{"There is no way of deducing Pauli’s principle; its validity has to be inferred from its results [...]"}} Most of this thesis does rely on the Pauli Principle being true, so for all intents and purposes we may as well take it to be a fundamental postulate.



%-----------------------------------
%	The many body wave function
%-----------------------------------

\section{The Many Body Wave Function}
A single particle may in isolation be completely described by a wave function in Hilbert space. We will refer to this SP (single particle) state by

$$ \phi(\bold{x}) $$

where $\bold{x}$ now contains all the relevant quantum numbers (including spin).

In the presence of other particles it will make sense to define a wave function that describes the system as a whole $\Psi$, and it is reasonable to assume that this function relies on each constituent SP state. For a system of $N$ particles, we then have an N-body wave function on the form

\begin{equation}
\Phi \equiv \Phi(\psi_0(\bold{x}_0), \phi_1(\bold{x}_1), \phi_2(\bold{x}_2) ... \phi_N(\bold{x}_N)) 
\label{eqn:tisldef}
\end{equation}

Since each SP state have an associated Hilbert space, the systems state space will be a tensor product of each SP state space

\begin{equation}
\mathcal{H}_0 \otimes \mathcal{H}_1 \otimes \mathcal{H}_2 \otimes ...  \otimes \mathcal{H}_N
\label{eqn:systemspace}
\end{equation}

It is however possible for a subspace of the above to be sufficient. We may also refer to the totality of these spaces as the \emph{Fock space}. \cite{ShavittBartlett2009}

This may lead us to guess that the systems wave function is something like

\begin{equation}
\Phi_h  = \phi_0 \otimes \phi_1 \otimes ...  \otimes \phi_N = \phi_0(\bold{x}_0) \phi_1(\bold{x}_1) ... \phi_N(\bold{x}_N)
\label{eqn:hartreeprod}
\end{equation}

The subscript $h$ denotes that the above product is the so called \emph{Hartree product} or \emph{-function}. It may also be written

\begin{equation}
\prod _{i} \phi _j(\mathbf{x_i})
\label{eqn:hartree3}
\end{equation}

\section{Antisymmetry}

The Hartree product lacks one important feature that is needed to properly describe fermionic systems, namely the antisymmetrization described in postulate 6 in the previous subsection. The Hartree product is \emph{completely uncorrelated}, meaning that the probability of finding fermions simultaneously at locations $x_0,x_1,...$ is simply

$$|\phi_i(x_0)\phi_j(x_1)...|^2 dx_0 dx_1 ... = |\phi_i(x_0)|^2 dx_0 |\phi_j(x_1)|^2 dx_2 ...$$

This is just the product of each constituent particle wave function squared. The motion of these particles is in effect independent of each other.

From experiment, we know that fermions are identical particles with half integer spin. \cite{Griffiths2005} Being \emph{identical} is equivalent to being experimentally \emph{indistinguishable}. This holds true unless some special preparation of the fermions is made, such as giving them opposite momentum. 

Although it doesn't immediately solve the antisymmetrization issues, we may therefore assume that each way of permuting the Hartree function is an equally valid representation of the system, so that also a linear combination of such permuted Hartree products is a valid representation of the system:

\begin{equation}
 \Phi_p = \frac{1}{\sqrt{N!}} \sum_{\pi}^{N!} \hat{P}_{\pi} \Psi_h
\label{eqn:hartreeperm}
\end{equation}

The subscript $p$ refers to the permutations, $N$ is here a normalizing constant, and the operator $\hat{P}_{\pi}$ is the permutation operator, performing all $N!$ possible permutations of the Hartree product. 

The Pauli Exclusion Principle is an interpretation of experimental facts, such as the pairing tendency of electrons, and the relation between stability and particle count in a variety of systems. It is commonly stated as \emph{no two indistinguishable fermions may occupy the same quantum state}. When applied on the permuted Hartree function, we see that this principle does not apply in its current form.

To mend this shortcoming of the permuted Hartree function we require that interchanging two particles should also change the sign of the resulting function. Thus, an odd number of permutations should result in a sign change, while an even number of permutations should not. We may express this by

\begin{equation}
 \Phi_{SD} = \frac{1}{\sqrt{N!}} \sum_{\pi}^{N!} \hat{P}_{\pi} (-)^{n(\pi)}\Psi_h \equiv \sqrt{N!} \mathcal{A} \Psi_h
\label{eqn:slaterdet}
\end{equation}

The subscript $SD$ now denotes the Slater determinant, which from here on will be referred to as simply the SD. The antisymmetrizer $\mathcal{A}$ is introduced to ease upcoming manipulations. One important property of the antisymmetrizer is that it commutes with the hamiltonian \cite{hh4480}

\begin{equation}
 \big[ \mathcal{A}, \hat{H} \big] = \mathcal{A}\hat{H} - \hat{H}\mathcal{A} = 0
\label{eqn:antisymm_commute}
\end{equation}

Another one is that its square is simply itself \cite{hh4480}

\begin{equation}
 \mathcal{A}^2 =  \mathcal{A} 
\label{eqn:antisymm_square}
\end{equation}

And yet another one is that its conjugate is itself \cite{hh4480}

\begin{equation}
 \mathcal{A}^\dagger =  \mathcal{A} 
\label{eqn:antisymm_conjugate}
\end{equation}

Another common representation of the SD is \cite{ShavittBartlett2009}

\begin{equation}
\Phi_{SD} (\mathbf{x_1},\mathbf{x_2},...,\mathbf{x_N}) =  \
\frac{1}{\sqrt{N!}}
 \begin{vmatrix}
  \psi_1(\mathbf{x_1}) & \psi_2(\mathbf{x_1}) & \cdots & \psi_N(\mathbf{x_1}) \\
  \psi_1(\mathbf{x_2}) & \psi_2(\mathbf{x_2}) & \cdots & \psi_N(\mathbf{x_2}) \\
  \vdots               & \vdots               & \ddots & \vdots               \\
  \psi_1(\mathbf{x_N}) & \psi_2(\mathbf{x_N}) & \cdots & \psi_N(\mathbf{x_N}) \\
 \end{vmatrix} 
\label{eqn:slaterlinalg}
\end{equation}


\section{The Hamiltonian}

In classical mechanics, the total energy of a particle is called the \emph{Hamiltonian}, and is written \cite{Griffiths2005}

\begin{equation}
H(x,p) = \frac{p^2}{2m} + V(x)
\label{eqn:classical_hamiltonian}
\end{equation}

Where $p$ is the momentum, $x$ is the position, $m$ is the mass and $V(x)$ is the potential. By substituting $p \rightarrow \frac{\hbar}{i} \frac{\partial}{\partial x}$, we find the corresponding quantum mechanical hamiltonian to be \cite{Griffiths2005}

\begin{equation}
\hat{H} = -\frac{\hbar^2}{2m} \frac{\partial^2}{\partial x^2} + V(x)
\label{eqn:quantum_hamiltonian}
\end{equation}

Using the hamiltonian, we may write the time independent Schrödinger equation as

\begin{equation}
\hat{H} \vert \Phi_{SD} \rangle = E \vert \Phi_{SD} \rangle
\label{eqn:TUSL}
\end{equation}
 
In cases where we know $\vert \Phi_{SD} \rangle$, it is possible to solve the eigenvalue problem by multiplying both sides of the equation with $\langle \Phi_{SD} \vert$ from the left, so that

\begin{equation}
\langle \Phi_{SD} \vert \hat{H} \vert \Phi_{SD} \rangle = \langle \Phi_{SD} \vert  E \vert \Phi_{SD} \rangle = E
\label{eqn:TUSL_sol}
\end{equation}

Or, by writing out the inner products as integrals

\begin{equation}
E = \int_{\Omega} \Phi_{SD}^*(\bold{x})  \hat{H} \Phi_{SD}(\bold{x}) d\Omega 
\label{eqn:TUSL_sol_integral}
\end{equation}




\section{Operators and matrix elements}

The form of the potential $V(x)$ in \ref{eqn:quantum_hamiltonian} will be of special interest to us when working with many body systems. For interacting systems, it is not sufficient for this operator to have a dependency on the coordinates of one particle at the time, since some of the potential energy is attributed to forces between the particles. Such forces normally depend on the distance between the particles, in other words two sets of coordinates at a time. In this context, it makes sense to separate terms that relate to a common potential from the terms that relates to multiple particles that interact. For a particle present in the system we may therefore write

\begin{equation}
V(x_i) = v(x_i) + \sum_j v(x_i, x_j) + \sum_{j<k, jk \neq i} v(x_i, x_j, x_k) + ... \equiv v_i + v_{ij} + v_{ijk} + ...
\label{eqn:potential_1}
\end{equation}

The first term now relates to the common potential, the second term relates to forces that act on two particles at a time, and the third relates to forces that involves three particles at a time. We could extend this to include four body forces and further, but we will not encounter any interactions involving more than two particles at a time in this thesis. We will define these interactions in more detail in \ref{Chapter2}.

For simplicity, we may include the kinetic energy in the one body force, so that

\begin{equation}
v(x_i) =  -\frac{\hbar^2}{2m} \frac{\partial^2}{\partial x_i^2} + v(x_i)
\label{eqn:onebodyforce}
\end{equation}

We assume to work with identical particles such as fermions, so there is no need to assign any index to the mass. We may now write our general hamiltonian for particle $i$ as

\begin{equation}
H = \sum_i v(x_i) + \sum_{i<j} v(x_i, x_j)
\label{eqn:general_hamiltonian}
\end{equation}

Knowing our hamiltonian as well as the SD enables us to solve the Schrödinger equation \ref{eqn:TUSL_sol}

\begin{multline}
\langle \Phi_{SD}  \vert [v\sum_i v(x_i) + \sum_{i<j} v(x_i, x_j)] \vert \Phi_{SD} \rangle = \\
 \langle \Phi_{SD} \vert   \sum_i v(x_i) \vert \Phi_{SD} \rangle + \langle \Phi_{SD} \vert  \sum_{i<j} v(x_i, x_j) \vert \Phi_{SD} \rangle= E
\label{eqn:TUSL_sol2}
\end{multline}

If we consider the form of the SD defined in \ref{eqn:slaterdet} and the properties of the antisymmetrizer, we will find that

\begin{multline}
    \epsilon _0 = \langle \sqrt{N!} \mathcal{A} \phi _h | \hat{H} | \sqrt{N!} \mathcal{A} \phi _h \rangle = \\
    N!\langle \phi _h | \mathcal{A}^{\dagger} \hat{H} \mathcal{A} \phi _h \rangle = N!\langle \phi _h | \hat{H}\mathcal{A}\phi _h \rangle = \\ 
    N!\langle \phi _h |\hat{H} \Psi _0 \rangle  = N!\langle \phi _h |\hat{H} | \Phi _{SD} \rangle 
 \label{eqn:groundstate}
\end{multline}

Inserting our hamiltonian we find that

\begin{equation}
\epsilon _0 = \langle \psi _h | \sum _i^N \hat{f}(\mathbf{x_i}) | \Psi _0\rangle + \frac{N!}{2}\langle \psi _h | \sum _{i,j\neq i}^N \hat{v}(\mathbf{x_i}, \mathbf{x_j}) | \Psi _0 \rangle
 \label{eqn:convinient_groundstate}
\end{equation}

so the problem is naturally separated in terms relating to the one body part and the two body part. The contributions to each part will depend on the the hamiltonian operators ability to bring permutations aside from the Hartree product into alignment with the Hartree product.

\subsection{The one body problem}

Since the one body operator only acts on one particle at a time, we will find that

\begin{equation}
\hat{f}(\mathbf{x_i})\prod _{j=1}^N \psi _j (\mathbf{x_j}) = \Bigg( \prod _{j=1}^{N-1} \psi _j(\mathbf{x_j}) \Bigg) \hat{f}(\mathbf{x_i}) \psi _i(\mathbf{x_i})
 \label{eqn:onebody_2}
\end{equation}

So that it in a way "picks out" the target state and leaves the rest unaltered. We may write out the inner product as an integral

\begin{multline}
N! \int d\tau \Bigg( \prod _{j=1}^N \psi _i^{*} (\mathbf{x_i}) \Bigg) \hat{f}(\mathbf{x_j}) \Bigg( \prod _{k=1}^N \psi _k (\mathbf{x_k}) \Bigg) = \\
\prod _{i\neq j}^{N-1} \Big( \int d\mathbf{x_i} |\psi _i (\mathbf{x_i}) | ^2 \Big) \int d\mathbf{x_j} \big( \psi _j^{*}(\mathbf{x_j}) \hat{f}(\mathbf{x_j}) \psi _j(\mathbf{x_j}) \big)
\label{eqn:onebody_inner}
\end{multline}

In the case of an orthonormal basis, it is apparent that the outcome of this integral is dependent exclusively on how the one body operator acts on the targeted state, since

\begin{equation}
\prod _{i\neq j}^{N-1} \Big( \int d\mathbf{x_i} |\psi _i (\mathbf{x_i}) | ^2 \Big) = 1 
\label{eqn:orthogonal_onebody}
\end{equation}

For the terms beyond the unpermuted hartree product we will either find that

\begin{equation}
  \int d\mathbf{x_j} \big( \psi _j^{*}(\mathbf{x_j})\hat{f}(\mathbf{x_j})\psi _i(\mathbf{x_j})\big) = 0
\end{equation}

or that

\begin{equation}
  \int d\mathbf{x_j} \big( \psi _j^{*}(\mathbf{x_j})\psi _i(\mathbf{x_j})\big) = 0  
\end{equation}

so the contribution either way will be zero. This means that the one body contribution to the energy should be simply

\begin{equation}
\epsilon_f  = \sum_i \langle \phi_i \vert \hat{f} \vert \phi_i \rangle
\label{eqn:onebody_energy}
\end{equation}

\subsection{The two body problem}

For the two body problem, we now seek a solution to

\begin{equation}
\epsilon _v = \frac{N!}{2}\sum _{i,j\neq i}^N \langle \psi _h | v_{ij} | \mathcal{A}\psi _h \rangle
 \label{eqn:two_body_inner}
\end{equation}

If we first consider only the unpermuted hartree product to the right we will find that

\begin{multline}\label{two_body_integral_2}
  \langle \psi _h | v_{ij} | \psi _h \rangle = \\
  \prod _{k\neq (i,j)}^N \Big( \int d\mathbf{x_k}|\psi _k(\mathbf{x_k})|^2 \Big) 
  \int d\mathbf{x_i}d\mathbf{x_j} \Big( \psi _i^{*}(\mathbf{x_i})\psi _j^{*}(\mathbf{x_j}) \hat{v} (\mathbf{x_i},\mathbf{x_j}) \psi _i(\mathbf{x_i})\psi _j(\mathbf{x_j}) \Big) \\
\end{multline}

The factor in front will vanish if our basis is properly normalized. For one singly permuted Hartree products we find instead

\begin{multline}\label{two_body_integral_permut}
  \langle \psi _h | v_{ij} | \hat{P} _{ij} \psi _h \rangle = \\ 
  \int d\mathbf{x_i}d\mathbf{x_j} \Big( \psi _i^{*}(\mathbf{x_i})\psi _j^{*}(\mathbf{x_j}) \hat{v} (\mathbf{x_i},\mathbf{x_j})\psi _i(\mathbf{x_j})\psi _j(\mathbf{x_i}) \Big) \\
\end{multline}

The two body operators ability to bring the permuted states into alignment with the unpermuted states results in the above not necessarily being zero, so we will need to include it in the final energy evaluation. We may rewrite it into

\begin{equation}
 \epsilon _v = \frac{1}{2}\sum _{i,j\neq i} \langle \psi _h | \hat{v}_{ij} |(1 - \hat{P} _{ij})\psi _h\rangle = \frac{1}{2}\sum _{i,j\neq i}^N \Big( \langle ij | \hat{v} | ij \rangle  - \langle ij | \hat{v} | ji \rangle \Big)
 \label{two_body_en_eq}
\end{equation}

So that for our SD energy we have

\begin{equation}
E = \epsilon _f + \epsilon _v = \sum _i \langle i | \hat{f}_i | i \rangle + \frac{1}{2}\sum _{i,j\neq i}^N \Big( \langle ij | \hat{v} | ij \rangle  - \langle ij | \hat{v} | ji \rangle \Big)
\label{eqn:many_body_energy}
\end{equation}




%-----------------------------------
%	The Aim of Many Body Quantum Theory
%-----------------------------------

\section{The Aim of Many Body Quantum Theory}

At this point, we should note that while the SD fulfills the criterions laid out so far, we have still not defined the SP states. For the SD to fulfill the time independent Schrödinger equation, so must its constituent SP states. 

Depending on the form of the Hamiltonian in the Schrödinger equation, we may or may not have some idea of the form of the SP states. In many systems, it is possible to separate the hamiltonian into terms describing the interaction between the particles and terms associated with the constituent particles. 

\begin{equation}
 \hat{H} = \hat{H}_{onebody} + \hat{H}_{interaction}
\label{eqn:sephamilt}
\end{equation}

By ignoring the interaction terms, we may then try to solve the Schrödinger equation for the one body problem.

Consider for example a number of interacting fermions in a common potential. When solving the corresponding one body problem, one typically obtains a set of wave functions that fulfill the Schrödinger equation, where each constituent function corresponds to energy states. The number of states may be infinite. By letting this set of states populate the SD we may construct an infinite number of SDs, in effect spanning the Fock space defined in \ref{eqn:systemspace}.

While the Fock space completely spans the space for the systems wave function, it is also possible for a subset of SDs to do the same. Another possibility is that most of the system's wave function is contained in such a subset, so that a truncation of the Fock space may be made while retaining a decent approximation to the systems wave function.

For example, in cases where

\begin{equation}
 \hat{H}_{onebody} \gg \hat{H}_{interaction}
\label{eqn:lesserinteraction}
\end{equation}

we reasonably expect to be able to represent most of the systems ground state wave function with a subset of SDs containing the least exited one body states.

This is in essence the aim of Many Body Quantum Theory: we seek the set of SDs that gives the most accurate representation of the system's wave function. 

To obtain such a set we may choose a variety of paths, but common to all is the fact that the mathematical framework used so far would prove very tedious in deriving the upcoming expressions. We will therefore need to utilize the formalism commonly called \emph{second quantization} or \emph{the occupation number representation}, and for even more simplicity we will extend this to diagrammatic formalism.










