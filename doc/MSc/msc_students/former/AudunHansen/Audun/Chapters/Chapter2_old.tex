% Chapter Template

\chapter{Second Quantization} % Main chapter title

\label{Chapter2} % Change X to a consecutive number; for referencing this chapter elsewhere, use \ref{ChapterX}

\lhead{Chapter 2. \emph{Second Quantization}} % Change X to a consecutive number; this is for the header on each page - perhaps a shortened title

%-----------------------------------
%	Second Quantization
%-----------------------------------

\paragraph{A brief review} In this chapter we introduce the Second Quantization (or Number Representation) formalism, and we use it to derive the hamiltonian for many body problems with at most two body interactions.

\section{Second Quantization}

When choosing the SP states to populate the SD, we will normally choose from a set of states that solves the corresponding one body problem. Each of these states will have an energy eigenvalue in the one body problem. Because of the Pauli Exclusions Principle, any state may only occur once in the Hartree product, since multiple identical functions will cause the SD to cancel. 

When setting up the SD in this way, we create a model for our system that is commonly referred to as an \emph{independent particle model}, since it is constructed from uncorrelated states.

For a collection of $N$ completely non interacting identical fermions in a common potential, it then makes sense to define an energy level corresponding to the distribution of states where the $N$ lowest energy states is occupied. The energy of the most excited state is called the Fermi Energy, while the SD where all states below the Fermi Energy is occupied is called the Fermi Vacuum. In Dirac notation, we refer to this state as \cite{ShavittBartlett2009}

\begin{equation}
 \vert \Phi_0 \rangle 
\label{eqn:fermivacuum}
\end{equation}

In the same way, we may consider the "true" vacuum, where no states are occupied as

\begin{equation}
 \vert 0 \rangle
\label{eqn:truevacuum}
\end{equation}

It is in this context that second quantization will come in handy. We will first define the framework for the true vacuum, and thereafter for the Fermi vacuum. This is similar to the order in which these topics are introduced in Shavitt and Bartlett \cite{ShavittBartlett2009}, and in the following sections we will basically just restate the formalism as it appears in this informative work.

\section{Creation and annihilation operators}

In the following, we will for the sake of simplicity assume all SDs to be normalized and that the constituent SP states are orthogonal. 

To be \emph{occupied}, means that the SP state occurs in the SD, and to be \emph{unoccupied} then naturally means the opposite. Beginning with the true vacuum state, we may then define operators that creates or annihilates (removes) occupied states in the SD. We call these operators \emph{creation-} and {annihilation} operators. 

In the second quantization formalism, we denote creation operators with the letter "a" with a hat, followed by a subindex denoting the state, and a dagger. When acting on the true vacuum state, it will produce a SD where the given state is occupied:

\begin{equation}
\hat{a}_p^\dagger \vert 0 \rangle = \vert p \rangle
\label{eqn:creation}
\end{equation}

The annihilation operator is the same, only missing the dagger:

\begin{equation}
\hat{a}_p \vert p \rangle = \vert 0 \rangle
\label{eqn:annihilation}
\end{equation}



Creation operators acting on already occupied states does violate the Pauli Exclusion Principle, and should yield zero:

\begin{equation}
\hat{a}_p^\dagger \vert pqr \rangle = 0
\label{eqn:createExclusion}
\end{equation}

Annihilating states that are not present in the SD should also yield zero

\begin{equation}
\hat{a}_p \vert qrs \rangle = 0
\label{eqn:annihilateExclusion}
\end{equation}

We may combine these operators to define the so called \emph{number} operator:

\begin{equation}
\hat{a}_p^\dagger \hat{a}_p \vert \Phi \rangle \equiv  \hat{n}_p(\Phi) \vert \Phi \rangle =
\begin{cases}
\vert \Phi \rangle,& p \in \vert \Phi \rangle \\
0,& p \not \in \vert \Phi \rangle
\end{cases}
\label{eqn:numberop}
\end{equation}

\section{Strings of operators}

Constructing a slater determinant from a set of orbitals may now be performed by 

\begin{equation}
\hat{a}_i^\dagger \hat{a}_j^\dagger  ... \hat{a}_q^\dagger \hat{a}_r^\dagger \vert 0 \rangle = \vert ij ... qr \rangle
\label{eqn:createSD}
\end{equation}

While the subindices refer to the states present in the SD, the order in which these appear tell us the particle occupying each state. If we need to explicitly express which particle occupies which state, we may give each particle a number and write

\begin{equation}
\hat{a}_{i,1}^\dagger \hat{a}_{j,2}^\dagger  ... \hat{a}_{q,N-1}^\dagger \hat{a}_{r,N}^\dagger \vert 0 \rangle = \vert i(1)j(2) ... q(N-1)r(N) \rangle
\label{eqn:createSD2}
\end{equation}

Since the permutation of occupied states changes the sign of the SD \ref{eqn:slaterdet}, so should the permutation of operators. Permuting operators acting on a SD is basically the same operation as permuting particles in the SD. We have 

\begin{equation}
\hat{P} \vert \Phi \rangle = (-1)^{\sigma(P)} \vert \Phi \rangle
\label{eqn:Permute2nquant}
\end{equation}

Which is equivalent to

\begin{equation}
\hat{a}_i^\dagger \hat{a}_j^\dagger  ... \hat{a}_m^\dagger \hat{a}_n^\dagger ... \hat{a}_q^\dagger \hat{a}_r^\dagger \vert 0 \rangle = - \hat{a}_i^\dagger \hat{a}_j^\dagger  ... \hat{a}_n^\dagger \hat{a}_m^\dagger ... \hat{a}_q^\dagger \hat{a}_r^\dagger \vert 0 \rangle
\label{eqn:permuteSD2}
\end{equation}

Because of orthonormality, we will find that the the expectation value of a annihilation operator is

\begin{equation}
\langle \Phi' \vert \hat{a}_p \vert \Phi \rangle = 
\begin{cases}
\pm 1,& \hat{n}_p(\Phi) = 1,  \hat{n}_p(\Phi') = 0,  \hat{n}_i(\phi) = \hat{n}_i{\Phi'}  (i \neq p) \\
0,& else
\end{cases}
\label{eqn:expvalue}
\end{equation}

This means that the expectation value will be nonzero if $\Phi'$ and $\Phi$ have the same occupied states, except for that the state $p$ occurs in $\Phi$ but not in $\Phi'$. From this, we may deduce that

\begin{equation}
\langle \Phi' \vert \hat{a}_p \vert \Phi \rangle = \langle \hat{a}_p^\dagger \Phi' \vert \vert \Phi \rangle
\label{eqn:expvalue2}
\end{equation}

which shows that $\hat{a}_p$ is the adjoint to $ \hat{a}_p^\dagger $.

\section{Anticommutation relations}

To enable us to evaluate expectation values for strings of operators, we will need to be able to manipulate these strings. Such manipulations will at some abstract level be involved at a later stage in this thesis, for example when deriving the Coupled Cluster equations. 

While we have already defined the interchange or permutations of two or more operators in strings of exclusively creation or annihilation operators (\ref{eqn:Permute2nquant}), we will run into complications when the strings contains a mix of these operators. To this end, we will utilize the anticommutator, defined as

\begin{equation}
[\hat{A}, \hat{B}]_+ \equiv \hat{A}\hat{B} + \hat{B}\hat{A}
\label{eqn:anticommutator}
\end{equation}

From what we already have discussed \ref{permuteSD2}, it is apparent that 

\begin{equation}
[\hat{a}_p, \hat{a}_q]_+ = \hat{a}_p\hat{a}_q + \hat{a}_q\hat{a}_p = \hat{a}_p\hat{a}_q - \hat{a}_p\hat{a}_q = 0
\label{eqn:anticommutator_aa}
\end{equation}

and

\begin{equation}
[\hat{a}_p^\dagger, \hat{a}_q^\dagger]_+ = \hat{a}_p^\dagger\hat{a}_q^\dagger + \hat{a}_q^\dagger\hat{a}_p^\dagger = \hat{a}_p^\dagger\hat{a}_q^\dagger - \hat{a}_p^\dagger\hat{a}_q^\dagger = 0
\label{eqn:anticommutator_cc}
\end{equation}

By evaluating how the anticommutator for mixed operators acts on certain SDs, we find that

\begin{equation}
[\hat{a}_p^\dagger, \hat{a}_q]_+ = [\hat{a}_p, \hat{a}_q^\dagger]_+ = \delta_{p,q}
\label{eqb:anticommutator_ca_ac}
\end{equation}

This will allow us to rewrite certain strings of operators, since it means that for example

\begin{equation}
\hat{a}_p^\dagger \hat{a}_q = \delta_{p,q} -  \hat{a}_q \hat{a}_p^\dagger
\label{eqn:anticommutator_ca_ac2}
\end{equation}

\section{Inner products and operators}

The vacuum state is assumed to be normalized:

\begin{equation}
\langle 0 \vert 0 \rangle = 1
\label{eqn:vacuuminner}
\end{equation}

With the framework laid out so far, we are now able to evaluate inner products of multiple SDs. Considering two SDs:

\begin{equation}
\hat{a}_p^\dagger \hat{a}_q^\dagger  ... \hat{a}_N^\dagger \vert 0 \rangle, \: \: \: \:
\hat{a}_r^\dagger \hat{a}_s^\dagger  ... \hat{a}_M^\dagger \vert 0 \rangle
\label{eqn:sd12}
\end{equation}

We may use the the fact that the creation operator is the adjoint of the annihilation operator, so that

\begin{equation}
\langle 0\vert  \hat{a}_M ...  \hat{a}_s  \hat{a}_r   \hat{a}_p^\dagger \hat{a}_q^\dagger  ... \hat{a}_N^\dagger \vert 0 \rangle
\label{eqn:sd12inner}
\end{equation}

 

In the case in \ref{eqn:sd12inner}, we may find that the operators line up perfectly, so that

\begin{equation}
(r=p), (s=q), ... ,(M=N) \rightarrow
\langle 0\vert  \hat{a}_M ...  \hat{a}_s  \hat{a}_r   \hat{a}_p^\dagger \hat{a}_q^\dagger  ... \hat{a}_N^\dagger \vert 0 \rangle = 1
\label{eqn:sd12inner_prog}
\end{equation}

In other cases, the calculation of this inner product (also commonly called \emph{matrix product} in the literature \cite{ShavittBartlett2009}), becomes a matter of reorganizing the creation and annihilation operators in such a manner that we end up with something we are able to evaluate. As will become apparent, a good strategy is to move all creation operators to the left of the annihilation operators, since this inner product will be zero.

One very important aim is to be able to evaluate the expectation value of the hamiltonian, and as such we will also need to extend the formalism to include operators. The way in which they occur in the hamiltonian, we may have operators acting on one particle at a time, such as the term associated with the common potential, or we may have interactions between two or more particles. We will refer to these operators as \emph{one body operators}, \emph{two body operators}, and so on. Since we are only dealing with interactions at the electronic scale, we will at most encounter two body operators. 

\paragraph{The One Body Operator}

A one body operator may be defined as \cite{ShavittBartlett2009}

\begin{equation}
\hat{F} = \sum_{k,l} f_{k,l}  \hat{a}_k^{\dagger} \hat{a}_l
\label{eqn:onebody}
\end{equation}

It is informative to calculate the expectation value of this operator, so we consider the inner product

\begin{equation}
 \sum_{k,l} f_{k,l}  \langle 0\vert  \hat{a}_M ...  \hat{a}_s  \hat{a}_r ( \hat{a}_k^{\dagger} \hat{a}_l)  \hat{a}_p^\dagger \hat{a}_q^\dagger  ... \hat{a}_N^\dagger \vert 0 \rangle
\label{eqn:sd12inner_op}
\end{equation}

Depending on the SDs, we may have four different outcomes of the above:

- If we have perfect line up as in \ref{eqn:sd12inner_prog}, we find that $\langle \hat{F} \rangle = \sum_i^N f_{i,i}$

- If all SP states involved occur in both SDs but in no particular order, we find that $\langle \hat{F} \rangle =(-1)^{\sigma(\hat{P})} \sum_i^N f_{i,i}$

- If all except one SP state (one \emph{non-coincidence} \cite{ShavittBartlett2009}) occur in both SDs, the one body operator may act on the SDs in such a way that when encountering the non-coincidence it brings it to coincide:

%\begin{equation}
\begin{multline}
(r=p), (s=q), ...,(m \neq n) , ... ,(M=N) \rightarrow \\ \sum_{k,l} f_{k,l}  \langle 0\vert  \hat{a}_M ...  \hat{a}_m ... \hat{a}_s  \hat{a}_r ( \hat{a}_k^{\dagger} \hat{a}_l)  \hat{a}_p^\dagger \hat{a}_q^\dagger  ... \hat{a}_n...\hat{a}_N^\dagger \vert 0 \rangle =(-1)^{\sigma(\hat{P})} f_{m,n}
\end{multline}
\label{eqn:sd12inner_noncoincide}
%\end{equation}

- If there are more than one non-coincidence, the resulting SDs will be orthogonal so that $\langle \hat{F} \rangle = 0$.




\paragraph{The Two Body Operator}

We may also define the two body operator within the second quantization formalism \cite{ShavittBartlett2009}:

\begin{equation}
\hat{G} = \frac{1}{2} \sum_{i,j,k,l} \langle i(1) j(2) \vert g_{12} \vert k(1) l(2) \rangle  \hat{a}_{i}^{\dagger}\hat{a}_{j}^{\dagger} \hat{a}_{l} \hat{a}_k
\label{eqn:twobody}
\end{equation}

Similarly as for the one body operator, we will need to know the expectation value of this operator in a general Fock space, so also for this operator we consider the inner product

\begin{equation}
 \frac{1}{2} \sum_{i,j,k,l}    \langle i(1) j(2) \vert g_{12} \vert k(1) l(2) \rangle \langle 0\vert  \hat{a}_M ...  \hat{a}_s  \hat{a}_r (\hat{a}_{i}^{\dagger}\hat{a}_{j}^{\dagger} \hat{a}_{l} \hat{a}_k)  \hat{a}_p^\dagger \hat{a}_q^\dagger  ... \hat{a}_N^\dagger \vert 0 \rangle
\label{eqn:sd12inner_op2}
\end{equation}

If we have perfectly aligned states in the two SDs, we will now find that

\begin{equation}
\langle \Phi \vert \hat{G} \vert \Phi \rangle = \sum_{i < j, ij \in \Phi} (\langle ij \vert \hat{g} \vert ij \rangle -   \langle ij \vert \hat{g} \vert ij \rangle) \equiv \sum_{i < j, ij \in \Phi}  \langle ij \vert \vert ij \rangle
\label{eqn:antisymmetric_element}
\end{equation}

This is the \emph{antisymmetric matrix element}, and the two terms appearing in this element is by convention named the \emph{direct}- and \emph{exhange} term, respectively. 

For cases where the is one non-coincident in particle number $p$, we find that

\begin{equation}
\langle \Phi' \vert \hat{G} \vert \Phi \rangle = \sum_{j \in \Phi} \langle i'(p) j \vert \vert i(p)j \rangle
\label{eqn:antisymmetric_element_2}
\end{equation}

And for cases where we have two non-coincidences, we now find that

\begin{equation}
\langle \Phi' \vert \hat{G} \vert \Phi \rangle = \langle i'(p) j'(q) \vert \vert i(p) j(q) \rangle
\label{eqn:antisymmetric_element_3}
\end{equation}

\paragraph{The hamiltonian}

We now have the means to write down a hamiltonian in second quantized form containing both one body and two body operators. This is the form of the hamiltonian that will be the topic of this thesis, and we will at a later stage be more specific on what the different parts resemble. For now, we will utilize the fact that

$$ \langle ij \vert \vert kl \rangle =  -\langle ij \vert \vert lk \rangle$$

to rewrite the hamiltonian into the following form:

\begin{equation}
\hat{H} = \sum_{ij} h_{ij} \hat{a}_i^{\dagger} \hat{a}_j + \frac{1}{4} \sum_{i,j,k,l} \langle ij \vert \vert kl \rangle \hat{a}_{i}^{\dagger}\hat{a}_{j}^{\dagger} \hat{a}_{l} \hat{a}_k
\label{eqn:2nq_hamiltonian}
\end{equation}

\section{Normal ordering}

As previously mentioned, reorganizing strings of operators so that all annihilation operators are to the right of the creation operators will be a good strategy when evaluating inner products. The reason for this is that this sequence of operators must yield zero when evaluated as the expectation value in the vacuum state, and in the process of this reorganization we will produce all nonzero contributions as kroenecker deltas in accordance with \ref{eqb:anticommutator_ca_ac}. 

For this reason, the process of reorganizing strings of operators into this so called \emph{normal ordered} sequence is of special interest when doing calculations on many body wave functions. While the diagrammatic approach introduced by Feynman \cite{ShavittBartlett2009} will be our main weapon of choice when dealing with such problems at a later stage in this thesis, we will first treat this using \emph{Wick's Theorem} \cite{Wick1950}. This straightforward approach has the advantage of being easily translated into computer algebra, as utilized in for example the \emph{Secondquant} package for Python \cite{secondquant}. Although we have already introduced the basic operations needed in this process, we will in this section see that it may be greatly simplified.

The normal ordered product (or simply \emph{normal product}) is commonly denoted by either an "n" followed by square brackets, or curly brackets \cite{ShavittBartlett2009}:

\begin{equation}
n[\hat{A} \hat{B} ...] = \{\hat{A} \hat{B}...  \}
\label{eqn:2nq_hamiltonian}
\end{equation}

\section{Contractions}

We define the process of \emph{contracting} two (creation or annihilation) operators by

\begin{equation}
\contraction{}{A}{}{B}
AB \equiv AB - n[AB] 
\label{eqn:contractiondef}
\end{equation}

For the operators discussed so far, we will only encounter four different situations when performing such contractions. Either we have the three cases where the contracted operators are already basically normal ordered

\begin{equation}
\contraction{}{\hat{a}_p^\dagger}{}{\hat{a}_q^\dagger}
\hat{a}_p^\dagger \hat{a}_q^\dagger 
\contraction{=}{\hat{a}_p}{}{\hat{a}_q}
= \hat{a}_p \hat{a}_q
\contraction{=}{\hat{a}_p^\dagger}{}{\hat{a}_q}
 = \hat{a}_p^\dagger \hat{a}_q = 0
\label{eqn:contractionbank}
\end{equation}

or we have the singular nonzero case, where

\begin{equation}
\contraction{}{\hat{a}_p}{}{\hat{a}_q^\dagger}
\hat{a}_p \hat{a}_q^\dagger = [\hat{a}_p, \hat{a}_q^\dagger]_+ = \delta_{p,q}
\label{eqn:contractionbank2}
\end{equation}

\section{Wick's theorem}

Wick's theorem was introduced by Gian-Carlo Wick in 1950 \cite{Wick1950}, and states that 

\begin{theorem}[The time independent Wick's theorem]
A product of a string of creation and annihilation operators is equal to their normal product plus the sum of all possible normal products with contractions.
\end{theorem}

This is how it occurs in Shavitt and Bartlett's  \emph{"Many Body Methods in Physics and Chemistry"} \cite{ShavittBartlett2009}).

We have already seen that the expectation value of any normal product on the vacuum state will be zero, so this basically means that only the possible fully contracted normal products will contribute to the expectation value. As stated in \ref{eqn:contractionbank}, many of these contraction will also be zero, so we need only consider the possible non-zero contractions. 

This is a great simplification of the tedious reorganization of operators we have previously encountered, and it is an important tool when working with many body wave functions.

\section{Particles and holes}

We have previously briefly mentioned the \emph{fermivacuum} (also commonly called the \emph{reference state}) \ref{eqn:fermivacuum}, although we up to this point have mainly considered operators acting on the true vacuum state \ref{eqn:truevacuum}. When we at a later stage will perform actual calculations, we will typically have a fixed number (N) of particles that may be represented by a number of fixed sized SDs. In this context, it will make much more sense to define the second quantization formalism in relation to the fermi vacuum, namely the SD where the N particles occupies the N lowest energy eigenstates of the SP basis.  

We will then need to redefine normal ordering, creation and annihilation operators, and Wick's theorem with respect to this new reference state. 

By convention, we will now refer to unoccupied states up to the Fermi level as holes (or hole states), and label them with the letters $i,j,k,...$. Occupied states above the Fermi level will be referred to as \emph{particles} (or particle states), and will be labeled $a,b,c,...$. Creation and annihilation operators will now behave differently depending on whether they target states above or below the Fermi level. 

A \emph{pseudo particle} creation operator may act on the reference state to either remove an occupied state below the Fermi level, thus \emph{creating a hole}, or it may create a particle in an unoccupied particle state above the fermi level. Operators acting on particle states will then behave the same as before, while operators acting on holes  will have the opposite effect as previously discussed.

For example, we may excite the reference state by annihilating a state $i$ below the Fermi level (thus creating a hole), and thereafter create a particle state above the Fermi level. By convention, such a process may be written

\begin{equation}
\hat{a}_a^\dagger \hat{a}_i^\dagger \vert \Phi_0 \rangle = \vert \Phi_i^a \rangle 
\label{eqn:excitation}
\end{equation}

Note that both operators above are assigned the dagger, denoting that they are pseudo particle creation operators \cite{ShavittBartlett2009}.  This inversion in notation for operators acting on states below the Fermi level is motivated by the fact that the way we previously defined the normal product no longer will will yield zero when evaluated on the reference state. If we instead define our normal product to have all pseudo creation operators to the left of all pseudo annihilation operators, we achieve the same behavior as for the true vacuum.

In this new context, Wick's theorem will remain basically unchanged, apart from that the only possible nonzero contractions will be

\begin{equation}
\contraction{}{\hat{a}_a}{}{\hat{a}_b^\dagger}
\hat{a}_a \hat{a}_b^\dagger = \delta_{a,b}
\label{eqn:nnz_contractions_1}
\end{equation}

and

\begin{equation}
\contraction{}{\hat{a}_i^\dagger}{}{\hat{a}_j}
\hat{a}_i^\dagger \hat{a}_j = \delta_{i,j}
\label{eqn:nnz_contractions_2}
\end{equation}


\section{The normal ordered hamiltionian and Wick's generalized theorem}

By normal ordering the hamiltonian we will achieve two things; we will derive expressions that are especially suitable for dealing with wave functions in Fock space (post Hartree Fock calculations), and we will be able to utilize the so-called generalized Wick's theorem, basically stating that

\begin{theorem}[The generalized time independent Wick's theorem]
When considering products of strings of normal ordered creation and/or annihilation operators, we need only consider fully contracted contributions between the normal ordered strings. Internal contractions in the products will yield zero.
\end{theorem}\footnote{Note that this is not the actual formulation, see \cite[p.86]{ShavittBartlett2009} for the full theorem. For our purpose this formulation is sufficient, since it means that when we encounter expectation values involving normal ordered strings, it will efficiently reduce the number of evaluations needed.}

To this end, we will benefit from rewriting the hamiltonian into normal ordered form.

\subsection{Normal ordered one body operator}

Acting on all states (particles and holes), the one body operator may be written

\begin{equation}
\hat{F} = \sum_{p,q} f_{p,q}  \hat{a}_p^{\dagger} \hat{a}_q
\label{eqn:onebody_n}
\end{equation}

By Wick's theorem, we find that

\begin{equation}
\hat{F} = \sum_{p,q} f_{p,q}  (\{\hat{a}_p^{\dagger} \hat{a}_q \} + \delta_{p,q \leq Fermilimit}) \equiv \hat{F}_N + \sum_{i} f_{i,i} 
\label{eqn:onebody_n2}
\end{equation}

If we treat particle and hole states separately in the normal ordered one body operator, we will find that it is constituted by

\begin{equation}
\hat{F}_N = \sum_{i,j} f_{i,j}   \hat{a}_j \hat{a}_i^{\dagger} + \sum_{a,b} f_{a,b}  \hat{a}_a^{\dagger} \hat{a}_b + \sum_{a,i} f_{i,a}  \hat{a}_i^{\dagger} \hat{a}_a + \sum_{a,i} f_{a,i}   \hat{a}_i \hat{a}_a^{\dagger}
\label{eqn:onebody_N}
\end{equation}

The remaining term is now simply the expectation value associated with each SP state below the Fermi limit.

\subsection{Two body operator}

Analogous for the two body operator, we find that

\begin{equation}
\hat{G} = \frac{1}{2} \sum_{p,q,r,s} \langle pq \vert g \vert rs \rangle  \hat{a}_{p}^{\dagger}\hat{a}_{q}^{\dagger} \hat{a}_{s} \hat{a}_r
\label{eqn:twobody_n}
\end{equation}

Again, by Wick's theorem, we find \cite[p.82]{ShavittBartlett2009}:

\begin{equation}
\hat{G} = \hat{G}_N + \frac{1}{2} \sum_{p,q} (\sum_i \langle pi \vert \hat{g} \vert qi \rangle_{AS} \{ \hat{a}_p^{\dagger} \hat{a}_q \}) + \frac{1}{2} \sum_{ij} \rangle ij \vert \hat{g} \vert ij \rangle_{AS}
\label{eqn:twobody_N}
\end{equation}

The middle term is a one body term associated with a two body operator.  \cite{ShavittBartlett2009}

The normal ordered term above may also be written in a way that specifies particles and holes, so we have

\begin{multline}
\hat{G}_N = \sum_{abcd} \langle ab\vert \hat{g} \vert cd \rangle \{ \Cr{a} \Cr{b} \An{d} \An{c} \}
+  \sum_{ijkl} \langle ij\vert \hat{g} \vert kl \rangle \{ \Cr{i} \Cr{j} \An{l} \An{k} \}
+  \sum_{aibj} \langle ij\vert \hat{g} \vert bj \rangle \{ \Cr{a} \Cr{i} \An{j} \An{b} \} \\
+  \sum_{abci} \langle ab \vert \hat{g} \vert ci \rangle \{ \Cr{a} \Cr{b} \An{i} \An{c} \}
+  \sum_{iajk} \langle ia\vert \hat{g} \vert jk \rangle \{ \Cr{i} \Cr{a} \An{k} \An{j} \}
+  \sum_{aibc} \langle ai\vert \hat{g} \vert bc \rangle \{ \Cr{a} \Cr{i} \An{b} \An{c} \} \\
+  \sum_{ijka} \langle ij \vert \hat{g} \vert ka \rangle \{ \Cr{i} \Cr{j} \An{a} \An{k} \}
+  \sum_{abij} \langle ab\vert \hat{g} \vert ij \rangle \{ \Cr{a} \Cr{b} \An{j} \An{i} \}
+  \sum_{ijab} \langle ij\vert \hat{g} \vert ab \rangle \{ \Cr{i} \Cr{j} \An{b} \An{a} \}
\label{eqn:twobody_NN}
\end{multline}

\subsection{The normal ordered hamiltonian}

From the above derivations, we may now express the full hamiltonian as

\begin{multline}
\hat{H} = hat{F}_N + \frac{1}{2} \sum_{p,q}  \sum_i \langle pi \vert \hat{v} \vert qi \rangle_{AS} \{\Cr{p} \An{q} \} + \hat{V}_N + \frac{1}{2} \sum_{ij} \rangle ij \vert \hat{v} \vert ij \rangle_{AS} +  \sum_{i} f_{i,i} 
\label{eqn:hamiltonian_full}
\end{multline}

Where we conventionally renamed $\hat{G} \rightarrow \hat{V}$ to identify the interaction. 

We have previously derived the expectation value of one- and two body operators one reference state \ref{eqn:antisymmetric_element} \ref{eqn:sd12inner_op}, where we found that

\begin{equation}
\langle \Phi_0 \vert  \hat{H} \vert \Phi_0 \rangle = \sum_{i \in \Phi} f_{ii} + \sum_{i < j, ij \in \Phi}  \langle ij \vert \vert ij \rangle
\label{eqn:hamiltonian1}
\end{equation}

Comparing \ref{eqn:hamiltonian_full} to \ref{eqn:hamiltonian1}, we see that the full hamiltonian may be rewritten

\begin{equation}
\hat{H} = 
(\hat{F}_N + 
\hat{V}_N) +
\langle \Phi_0 \vert \hat{H} \vert \Phi_0 \rangle \equiv 
\hat{H}_N +
\langle \Phi_0 \vert \hat{H} \vert \Phi_0 \rangle
\label{eqn:hamiltonian_full_N}
\end{equation}

And we find that the normal ordered hamiltonian must be

\begin{equation}
\hat{H}_N = 
\hat{H}  -
\langle \Phi_0 \vert \hat{H} \vert \Phi_0 \rangle
\label{eqn:hamiltonian_N}
\end{equation}

\section{The correlation energy}

We may interpret the result in \ref{eqn:hamiltonian_N} as follows: if we seek the expectation value of the normal ordered hamiltonian on the true vacuum, we would find that 

\begin{equation}
\langle 0 \vert \hat{H}_N  \vert 0 \rangle = 
\langle 0 \vert \hat{H}  \vert 0 \rangle -
\langle \Phi_0 \vert \hat{H} \vert \Phi_0 \rangle \equiv
\langle 0 \vert \hat{H}  \vert 0 \rangle -
E_{ref}
\label{eqn:hamiltonian_N_vacuum}
\end{equation}

so that the energy found from calculating the above equals the actual energy of the system minus the part of the energy associated with the ground state SD $\Phi_0$. We may express this as

\begin{equation}
\langle 0 \vert \hat{H}_N  \vert 0 \rangle \equiv \Delta E = E - E_{ref}
\label{eqn:correlation_energy}
\end{equation}

In cases where $f_{pq}$ is diagonal, we will refer to the energy associated with the normal ordered hamiltonian as the \emph{correlation energy}\footnote{See also figure \ref{correlation}.}, while the energy associated with the reference state will be referred to as the \emph{reference energy}. The rationale of this ordering is, as well as the way they naturally occur in the equations, that the reference energy is normally found by seeking a decent reference state for the system, while the correlation energy may be sought in a variety of ways building upon this first approximation. 

In cases where treating the interaction as a small perturbation is justified (\ref{eqn:lesserinteraction}), we may expect the reference energy to be a major contributor to the systems energy.

