% Chapter Template

\chapter{Conclusions and future prospects} % Main chapter title

\label{Chapter9} % Change X to a consecutive number; for referencing this chapter elsewhere, use \ref{ChapterX}

\lhead{Chapter 10. \emph{Conclusions and perspectives}} % Change X to a consecutive number; this is for the header on each page - perhaps a shortened title

%----------------------------------------------------------------------------------------
%	SECTION 1
%----------------------------------------------------------------------------------------

\section{Conclusions}

The electron gas has been studied with various {\em ab initio}
techniques, and in recent years these techniques also include coupled
cluster theory at the level of doubles excitations (CCD), see Refs.~ \cite{Baardsen2014,
  Shepherd2012}, many-body perturbation theory \cite{Shepherd2013}, stochastic methods \cite{Shepherd2012, Roggero2013, Leikanger2013}, and even CCD
with perturbative treatment of triples, see for example
Ref.~\cite{Shepherd2013}. From these studies, it has been clear that
the CCD aproach fails to account for important correlations in the homogeneous
electron gas in two and three dimensions. In the concluding remarks of
his doctoral thesis (see Ref.~\cite{Baardsen2014}), Baardsen suggests
that inclusion of triple correlations may be necessary to obtain
accurate results for the two-dimensional electron gas, and he also
proposes doing CCDT calculations on the three dimensional case for
small basis sets to explore how these correlations play in.

The aim of this thesis was to study the role of triples
correlations. These correlations have never been explored properly
before this work, mainly due to large computational obstacles.  This
thesis is thus likely to be the first time such results are presented
for the three-dimensional homogeneous electron gas. With our
results, we have demonstrated that it is computationally feasible to do
large-scale coupled cluster calculations on this system even with the
inclusion of triples. The formalism and techniques discussed are
intended to be general, so that the same procedure may easily be
extended to similar systems for infinite matter, like the electron gas in two dimensions or infinite nuclear matter and 
neutron star matter.

With our results, we have been able to demonstrate that the triple
amplitudes indeed do account for significant correlations in the
homogeneous electron gas in three dimensions. While the correlation
energy is generally overestimated by the various CCDT truncations, it
tends to reduce the deviation with the FCIQMC results (see
Refs. \cite{Shepherd2012, Leikanger2013}) by a factor of roughly 10
compared to the CCD energies, although this ratio depends  on
the density (and thereby Fermi momentum) under study.  The FCIQMC provides most likely the best possible {\em ab initio} benchmark for such systems.

In the same way as for the CCD results, the correlation energy
associated with most diagrams in the $\hat{T}_3$ amplitude equation
tend to diverge as we increase the Wigner-Zeiss radius $r_s$. This suggests
that more complicated correlations become important in
when we increase $r_s$.  
These features have also been noted for systems like quantum dots in two or three dimensions.
For larger values of $r_s$, most likely approaches like FCIQMC will provide more reliable estimates for various observables.

We have also performed qualitative studies of how the various
diagrams contributes to the correlation energy, and we have identified
some few diagrams that contribute more than others. These results
could possibly be utilized to do truncations in the CCDT approach for the sake
of performance and memory usage, and still retain accurate results.

Some reservation in these conclusions may arise due to the fact that
we have neglected both finite size effects and incomplete basis
errors. From the comparison with Shepherd \emph{et al.} in
Ref.~\cite{Shepherd2013}, it is clear that the single point
extrapolation technique will impact the thermodynamic limit estimates
considerably. It would also be interesting to obtain third-party
CCDT results in order to better validate our results. 

Finally, the softare CCAlgebra that we have developed, allows one to
produce automatically all relevant equations for a given level of
truncation of coupled cluster theory. This is an extremely useful tool
which provides important theory benchmarks when developing complicated
many-body methods like coupled cluster theory.
The software can easily be extended to accomodate other many-body methods. 
\section{Perspectives and recommendations}

With the code we have developed, we expect to be able to produce a lot
more results than what is presented in this thesis.

First, the main results produced in this thesis is in the authors
opinion the data set used in the estimations for the thermodynamical
limit. This set could be extended by correcting a minor coding issue
in the amplitude initialization function. We suspect that basis sets
above 1598 states resulted in a segmentation fault due to a static
allocation of memory. At this point, we set up Abel's "hugemem" nodes
to prepare 320GB of memory to our calculation, while the maximum limit
on these nodes is 1TB. It should be possible to do CCDT-1 calculations
beyond 2000 basis states.

Another way of improving our data set is to make it more accurate by
lowering the convergence threshold. This will lead to prolonged
iteration time, but since we have not implemented any methods for
convergence optimization, such as \emph{DIIS} (see
Ref.~\cite{scuseria1986}), we may be able to compensate for this
easily.

Another issue with these results is the insufficient treatment of
finite-size effects and incomplete basis error. The single point
extrapolation technique described in Refs.~\cite{Shepherd2012} and
methods discussed in \cite{Drummond2007} should be used on our results
to obtain a more comparable estimate to those of
Ref.~\cite{Shepherd2013}.

While the sparse implementation is mainly used for smaller basis sets,
it could be optimized considerably by choosing a more suitable
algorithm and container for the sparse tensors. Sparse matrix-matrix
multiplication is basically performed by sorting and traversing two
lists only once. While our algorithm relies on the Armadillo library
for this operation, it does not utilize any
parallelization. Functionality such as that provided by CUDA
\cite{cuda} or OpenMP \cite{openmp} could possibly improve performance
to such a degree that also larger basis calculations could be
feasible. The syntax used in the sparse implementation allows for fast
prototyping, so this could possibly provide us with means to include
even more correlations from the CCDTQ equations for smaller basis
sets.

Further, it seems to be great potential for optimizations of both
implementations. Comparison of our performance with the CCD
performance benchmarks form Baardsen, see Ref.~\cite{Baardsen2015},
suggests that our CCD code could be improved considerably, and in doing
this we could as well get some ideas as to how the CCDT
implementation may be optimized using similar techniques. It would be
wise to write such an implementation using the Intel MKL library and
either Fortran or plain C to achieve the best performance
possible. Avoiding libraries such as Armadillo will also simplify
compilation on computing clusters as they are not always supported,
while the Intel compiler generally is.

The simplicity of the sparse implementation syntax is also easy to
interface with the software CCAlgebra developed here, see for example
the description of \cite{CCAlgebra}) discussed in Chapter
\ref{Chapter5}. This process would allow the user to perform
complicated calculations of any coupled cluster truncation, even with
complicated many-body forces like three-body forces central in nuclear
physics using directly IPython notebook. IPython supports compilation
of C++ code directly from the cells, and it would even be possible to
set up the basis class externally, thus, changing the system would only
be a matter of rewriting a few lines in Python. This could be a
powerful tool, especially in obtaining preliminary results or fast
prototyping of code.

It is at the present unknown to the author exactly how the inclusion
of more diagrams (or even the full CCDT) in the block implementation
would affect the memory usage. Each diagrams may have values that add
into the full element vector, and we have observed quite a large
intersection between these elements. It could very well be that a full
CCDT implementation is not that much more computationally expensive
than the CCDT-1. As pointed out in
Ref~\cite[p.351]{ShavittBartlett2009}, one of the advantages of
conventional CCDT-1 implementations is that they avoid the need for
storing any $\hat{T}_3$ amplitudes. This is not possible in our
implementation, since we actually need to now in advance how the
$\hat{T}_3$ amplitudes align, so it may very well be that our
implementation already accommodates efficient implementation of the
full CCDT. This should be explored further.

With the formalism, implementations and results presented in this
thesis, it should be possible to carry out extensive experiments on
the system in question\footnote{Other similar systems could also be
  studied with these means.}. It is the authors hope that the work
presented in this thesis may enable future numerical experiments to focus more
on the physical aspects of the system in question, since a great deal
of our work has been to develop a working and efficient implementation
of the triple amplitudes for the homogeneous electron gas.

