


% Header, overrides base

    % Make sure that the sphinx doc style knows who it inherits from.
    \def\sphinxdocclass{article}

    % Declare the document class
    \documentclass[letterpaper,10pt,english]{/Users/kinealicegulbrandsen/anaconda/lib/python2.7/site-packages/sphinx/texinputs/sphinxhowto}

    % Imports
    \usepackage[utf8]{inputenc}
    \DeclareUnicodeCharacter{00A0}{\\nobreakspace}
    \usepackage[T1]{fontenc}
    \usepackage{babel}
    \usepackage{times}
    \usepackage{import}
    \usepackage[Bjarne]{/Users/kinealicegulbrandsen/anaconda/lib/python2.7/site-packages/sphinx/texinputs/fncychap}
    \usepackage{longtable}
    \usepackage{/Users/kinealicegulbrandsen/anaconda/lib/python2.7/site-packages/sphinx/texinputs/sphinx}
    \usepackage{multirow}

    \usepackage{amsmath}
    \usepackage{amssymb}
    \usepackage{ucs}
    \usepackage{enumerate}

    % Used to make the Input/Output rules follow around the contents.
    \usepackage{needspace}

    % Pygments requirements
    \usepackage{fancyvrb}
    \usepackage{color}
    % ansi colors additions
    \definecolor{darkgreen}{rgb}{.12,.54,.11}
    \definecolor{lightgray}{gray}{.95}
    \definecolor{brown}{rgb}{0.54,0.27,0.07}
    \definecolor{purple}{rgb}{0.5,0.0,0.5}
    \definecolor{darkgray}{gray}{0.25}
    \definecolor{lightred}{rgb}{1.0,0.39,0.28}
    \definecolor{lightgreen}{rgb}{0.48,0.99,0.0}
    \definecolor{lightblue}{rgb}{0.53,0.81,0.92}
    \definecolor{lightpurple}{rgb}{0.87,0.63,0.87}
    \definecolor{lightcyan}{rgb}{0.5,1.0,0.83}

    % Needed to box output/input
    \usepackage{tikz}
        \usetikzlibrary{calc,arrows,shadows}
    \usepackage[framemethod=tikz]{mdframed}

    \usepackage{alltt}

    % Used to load and display graphics
    \usepackage{graphicx}
    \graphicspath{ {figs/} }
    \usepackage[Export]{adjustbox} % To resize

    % used so that images for notebooks which have spaces in the name can still be included
    \usepackage{grffile}


    % For formatting output while also word wrapping.
    \usepackage{listings}
    \lstset{breaklines=true}
    \lstset{basicstyle=\small\ttfamily}
    \def\smaller{\fontsize{9.5pt}{9.5pt}\selectfont}

    %Pygments definitions
    
\makeatletter
\def\PY@reset{\let\PY@it=\relax \let\PY@bf=\relax%
    \let\PY@ul=\relax \let\PY@tc=\relax%
    \let\PY@bc=\relax \let\PY@ff=\relax}
\def\PY@tok#1{\csname PY@tok@#1\endcsname}
\def\PY@toks#1+{\ifx\relax#1\empty\else%
    \PY@tok{#1}\expandafter\PY@toks\fi}
\def\PY@do#1{\PY@bc{\PY@tc{\PY@ul{%
    \PY@it{\PY@bf{\PY@ff{#1}}}}}}}
\def\PY#1#2{\PY@reset\PY@toks#1+\relax+\PY@do{#2}}

\expandafter\def\csname PY@tok@gd\endcsname{\def\PY@tc##1{\textcolor[rgb]{0.63,0.00,0.00}{##1}}}
\expandafter\def\csname PY@tok@gu\endcsname{\let\PY@bf=\textbf\def\PY@tc##1{\textcolor[rgb]{0.50,0.00,0.50}{##1}}}
\expandafter\def\csname PY@tok@gt\endcsname{\def\PY@tc##1{\textcolor[rgb]{0.00,0.27,0.87}{##1}}}
\expandafter\def\csname PY@tok@gs\endcsname{\let\PY@bf=\textbf}
\expandafter\def\csname PY@tok@gr\endcsname{\def\PY@tc##1{\textcolor[rgb]{1.00,0.00,0.00}{##1}}}
\expandafter\def\csname PY@tok@cm\endcsname{\let\PY@it=\textit\def\PY@tc##1{\textcolor[rgb]{0.25,0.50,0.50}{##1}}}
\expandafter\def\csname PY@tok@vg\endcsname{\def\PY@tc##1{\textcolor[rgb]{0.10,0.09,0.49}{##1}}}
\expandafter\def\csname PY@tok@m\endcsname{\def\PY@tc##1{\textcolor[rgb]{0.40,0.40,0.40}{##1}}}
\expandafter\def\csname PY@tok@mh\endcsname{\def\PY@tc##1{\textcolor[rgb]{0.40,0.40,0.40}{##1}}}
\expandafter\def\csname PY@tok@go\endcsname{\def\PY@tc##1{\textcolor[rgb]{0.53,0.53,0.53}{##1}}}
\expandafter\def\csname PY@tok@ge\endcsname{\let\PY@it=\textit}
\expandafter\def\csname PY@tok@vc\endcsname{\def\PY@tc##1{\textcolor[rgb]{0.10,0.09,0.49}{##1}}}
\expandafter\def\csname PY@tok@il\endcsname{\def\PY@tc##1{\textcolor[rgb]{0.40,0.40,0.40}{##1}}}
\expandafter\def\csname PY@tok@cs\endcsname{\let\PY@it=\textit\def\PY@tc##1{\textcolor[rgb]{0.25,0.50,0.50}{##1}}}
\expandafter\def\csname PY@tok@cp\endcsname{\def\PY@tc##1{\textcolor[rgb]{0.74,0.48,0.00}{##1}}}
\expandafter\def\csname PY@tok@gi\endcsname{\def\PY@tc##1{\textcolor[rgb]{0.00,0.63,0.00}{##1}}}
\expandafter\def\csname PY@tok@gh\endcsname{\let\PY@bf=\textbf\def\PY@tc##1{\textcolor[rgb]{0.00,0.00,0.50}{##1}}}
\expandafter\def\csname PY@tok@ni\endcsname{\let\PY@bf=\textbf\def\PY@tc##1{\textcolor[rgb]{0.60,0.60,0.60}{##1}}}
\expandafter\def\csname PY@tok@nl\endcsname{\def\PY@tc##1{\textcolor[rgb]{0.63,0.63,0.00}{##1}}}
\expandafter\def\csname PY@tok@nn\endcsname{\let\PY@bf=\textbf\def\PY@tc##1{\textcolor[rgb]{0.00,0.00,1.00}{##1}}}
\expandafter\def\csname PY@tok@no\endcsname{\def\PY@tc##1{\textcolor[rgb]{0.53,0.00,0.00}{##1}}}
\expandafter\def\csname PY@tok@na\endcsname{\def\PY@tc##1{\textcolor[rgb]{0.49,0.56,0.16}{##1}}}
\expandafter\def\csname PY@tok@nb\endcsname{\def\PY@tc##1{\textcolor[rgb]{0.00,0.50,0.00}{##1}}}
\expandafter\def\csname PY@tok@nc\endcsname{\let\PY@bf=\textbf\def\PY@tc##1{\textcolor[rgb]{0.00,0.00,1.00}{##1}}}
\expandafter\def\csname PY@tok@nd\endcsname{\def\PY@tc##1{\textcolor[rgb]{0.67,0.13,1.00}{##1}}}
\expandafter\def\csname PY@tok@ne\endcsname{\let\PY@bf=\textbf\def\PY@tc##1{\textcolor[rgb]{0.82,0.25,0.23}{##1}}}
\expandafter\def\csname PY@tok@nf\endcsname{\def\PY@tc##1{\textcolor[rgb]{0.00,0.00,1.00}{##1}}}
\expandafter\def\csname PY@tok@si\endcsname{\let\PY@bf=\textbf\def\PY@tc##1{\textcolor[rgb]{0.73,0.40,0.53}{##1}}}
\expandafter\def\csname PY@tok@s2\endcsname{\def\PY@tc##1{\textcolor[rgb]{0.73,0.13,0.13}{##1}}}
\expandafter\def\csname PY@tok@vi\endcsname{\def\PY@tc##1{\textcolor[rgb]{0.10,0.09,0.49}{##1}}}
\expandafter\def\csname PY@tok@nt\endcsname{\let\PY@bf=\textbf\def\PY@tc##1{\textcolor[rgb]{0.00,0.50,0.00}{##1}}}
\expandafter\def\csname PY@tok@nv\endcsname{\def\PY@tc##1{\textcolor[rgb]{0.10,0.09,0.49}{##1}}}
\expandafter\def\csname PY@tok@s1\endcsname{\def\PY@tc##1{\textcolor[rgb]{0.73,0.13,0.13}{##1}}}
\expandafter\def\csname PY@tok@sh\endcsname{\def\PY@tc##1{\textcolor[rgb]{0.73,0.13,0.13}{##1}}}
\expandafter\def\csname PY@tok@sc\endcsname{\def\PY@tc##1{\textcolor[rgb]{0.73,0.13,0.13}{##1}}}
\expandafter\def\csname PY@tok@sx\endcsname{\def\PY@tc##1{\textcolor[rgb]{0.00,0.50,0.00}{##1}}}
\expandafter\def\csname PY@tok@bp\endcsname{\def\PY@tc##1{\textcolor[rgb]{0.00,0.50,0.00}{##1}}}
\expandafter\def\csname PY@tok@c1\endcsname{\let\PY@it=\textit\def\PY@tc##1{\textcolor[rgb]{0.25,0.50,0.50}{##1}}}
\expandafter\def\csname PY@tok@kc\endcsname{\let\PY@bf=\textbf\def\PY@tc##1{\textcolor[rgb]{0.00,0.50,0.00}{##1}}}
\expandafter\def\csname PY@tok@c\endcsname{\let\PY@it=\textit\def\PY@tc##1{\textcolor[rgb]{0.25,0.50,0.50}{##1}}}
\expandafter\def\csname PY@tok@mf\endcsname{\def\PY@tc##1{\textcolor[rgb]{0.40,0.40,0.40}{##1}}}
\expandafter\def\csname PY@tok@err\endcsname{\def\PY@bc##1{\setlength{\fboxsep}{0pt}\fcolorbox[rgb]{1.00,0.00,0.00}{1,1,1}{\strut ##1}}}
\expandafter\def\csname PY@tok@kd\endcsname{\let\PY@bf=\textbf\def\PY@tc##1{\textcolor[rgb]{0.00,0.50,0.00}{##1}}}
\expandafter\def\csname PY@tok@ss\endcsname{\def\PY@tc##1{\textcolor[rgb]{0.10,0.09,0.49}{##1}}}
\expandafter\def\csname PY@tok@sr\endcsname{\def\PY@tc##1{\textcolor[rgb]{0.73,0.40,0.53}{##1}}}
\expandafter\def\csname PY@tok@mo\endcsname{\def\PY@tc##1{\textcolor[rgb]{0.40,0.40,0.40}{##1}}}
\expandafter\def\csname PY@tok@kn\endcsname{\let\PY@bf=\textbf\def\PY@tc##1{\textcolor[rgb]{0.00,0.50,0.00}{##1}}}
\expandafter\def\csname PY@tok@mi\endcsname{\def\PY@tc##1{\textcolor[rgb]{0.40,0.40,0.40}{##1}}}
\expandafter\def\csname PY@tok@gp\endcsname{\let\PY@bf=\textbf\def\PY@tc##1{\textcolor[rgb]{0.00,0.00,0.50}{##1}}}
\expandafter\def\csname PY@tok@o\endcsname{\def\PY@tc##1{\textcolor[rgb]{0.40,0.40,0.40}{##1}}}
\expandafter\def\csname PY@tok@kr\endcsname{\let\PY@bf=\textbf\def\PY@tc##1{\textcolor[rgb]{0.00,0.50,0.00}{##1}}}
\expandafter\def\csname PY@tok@s\endcsname{\def\PY@tc##1{\textcolor[rgb]{0.73,0.13,0.13}{##1}}}
\expandafter\def\csname PY@tok@kp\endcsname{\def\PY@tc##1{\textcolor[rgb]{0.00,0.50,0.00}{##1}}}
\expandafter\def\csname PY@tok@w\endcsname{\def\PY@tc##1{\textcolor[rgb]{0.73,0.73,0.73}{##1}}}
\expandafter\def\csname PY@tok@kt\endcsname{\def\PY@tc##1{\textcolor[rgb]{0.69,0.00,0.25}{##1}}}
\expandafter\def\csname PY@tok@ow\endcsname{\let\PY@bf=\textbf\def\PY@tc##1{\textcolor[rgb]{0.67,0.13,1.00}{##1}}}
\expandafter\def\csname PY@tok@sb\endcsname{\def\PY@tc##1{\textcolor[rgb]{0.73,0.13,0.13}{##1}}}
\expandafter\def\csname PY@tok@k\endcsname{\let\PY@bf=\textbf\def\PY@tc##1{\textcolor[rgb]{0.00,0.50,0.00}{##1}}}
\expandafter\def\csname PY@tok@se\endcsname{\let\PY@bf=\textbf\def\PY@tc##1{\textcolor[rgb]{0.73,0.40,0.13}{##1}}}
\expandafter\def\csname PY@tok@sd\endcsname{\let\PY@it=\textit\def\PY@tc##1{\textcolor[rgb]{0.73,0.13,0.13}{##1}}}

\def\PYZbs{\char`\\}
\def\PYZus{\char`\_}
\def\PYZob{\char`\{}
\def\PYZcb{\char`\}}
\def\PYZca{\char`\^}
\def\PYZam{\char`\&}
\def\PYZlt{\char`\<}
\def\PYZgt{\char`\>}
\def\PYZsh{\char`\#}
\def\PYZpc{\char`\%}
\def\PYZdl{\char`\$}
\def\PYZhy{\char`\-}
\def\PYZsq{\char`\'}
\def\PYZdq{\char`\"}
\def\PYZti{\char`\~}
% for compatibility with earlier versions
\def\PYZat{@}
\def\PYZlb{[}
\def\PYZrb{]}
\makeatother


    %Set pygments styles if needed...
    
        \definecolor{nbframe-border}{rgb}{0.867,0.867,0.867}
        \definecolor{nbframe-bg}{rgb}{0.969,0.969,0.969}
        \definecolor{nbframe-in-prompt}{rgb}{0.0,0.0,0.502}
        \definecolor{nbframe-out-prompt}{rgb}{0.545,0.0,0.0}

        \newenvironment{ColorVerbatim}
        {\begin{mdframed}[%
            roundcorner=1.0pt, %
            backgroundcolor=nbframe-bg, %
            userdefinedwidth=1\linewidth, %
            leftmargin=0.1\linewidth, %
            innerleftmargin=0pt, %
            innerrightmargin=0pt, %
            linecolor=nbframe-border, %
            linewidth=1pt, %
            usetwoside=false, %
            everyline=true, %
            innerlinewidth=3pt, %
            innerlinecolor=nbframe-bg, %
            middlelinewidth=1pt, %
            middlelinecolor=nbframe-bg, %
            outerlinewidth=0.5pt, %
            outerlinecolor=nbframe-border, %
            needspace=0pt
        ]}
        {\end{mdframed}}
        
        \newenvironment{InvisibleVerbatim}
        {\begin{mdframed}[leftmargin=0.1\linewidth,innerleftmargin=3pt,innerrightmargin=3pt, userdefinedwidth=1\linewidth, linewidth=0pt, linecolor=white, usetwoside=false]}
        {\end{mdframed}}

        \renewenvironment{Verbatim}[1][\unskip]
        {\begin{alltt}\smaller}
        {\end{alltt}}
    

    % Help prevent overflowing lines due to urls and other hard-to-break 
    % entities.  This doesn't catch everything...
    \sloppy

    % Document level variables
    \title{t3\_alignments}
    \date{May 02, 2015}
    \release{}
    \author{Unknown Author}
    \renewcommand{\releasename}{}

    % TODO: Add option for the user to specify a logo for his/her export.
    \newcommand{\sphinxlogo}{}

    % Make the index page of the document.
    \makeindex

    % Import sphinx document type specifics.
     


% Body

    % Start of the document
    \begin{document}

        
            \maketitle
        

        


        
        \section{CCDT: Diagrams in the \(t_3\)
amplitude}\label{ccdt-diagrams-in-the-tux5f3-amplitude}This document lists all diagrams entering the \(t_3\) equation, as well
as the index realignment to perform matrix multiplications.

Notationwise, the operation

\[ t^{ab}_{ij} \rightarrow t^{ai}_{bj} \]

indicates a index transformation where we simply align the matrix
representation of this tensor in a fashion corresponding to the element
order above. The purpose of this operation is to align tensors so that
contractions may be performed as matrix multiplications. Upper indices
is mapped to a row index, while lower indices are mapped to columns.

Technically, this means to recalculate the row and column indices of the
matrix elements in the COO format (flexmat class). The corresponding
code to generate the two different representations of the flexmat object
\(t_2\) above is

\(t^{ab}_{ij}\) = t2.pq\_rs()

\(t^{ai}_{bj}\) = t2.pr\_qs()

From a theoretical point of view, this operation may be interpreted as a
generalized transpose for tensors of \(rank > 2\).\subsection{The \((t_2 t_3)\) terms}\label{the-tux5f2-tux5f3-terms}\begin{longtable}[c]{@{}lllll@{}}
\toprule
Diagram Label & Factor & Permutation & Index Transform & Code
translation\tabularnewline
\midrule
\endhead
\((t_2 t_3)_a\) & \(+1\) & \(\hat{P}(i/jk \vert a/bc)\) &
\[  \sum_{ldme} \langle l m \vert \vert d e \rangle t^{a d}_{i l}t^{e b c}_{m j k}  \rightarrow  \sum_{me} \sum_{ld} t^{bjck}_{me} \langle me\vert \vert ld\rangle t^{ld}_{ai} \]
& update\_as\_qtru\_ps(t3.qtru\_sp() \(*\) vhhpp.qs\_pr() \(*\)
t2.sq\_pr())\tabularnewline
\((t_2 t_3)_b\) & \(-\frac{1}{2}\) & \(\hat{P}(i/jk)\) &
\[  \sum_{ldme} \langle l m \vert \vert d e \rangle t^{d e}_{l i}t^{a b c}_{m j k}  \rightarrow  \sum_{m} \sum_{lde} t^{abjck}_{m} \langle m\vert \vert lde\rangle t^{lde}_{i} \]
&
update\_as\_pqtru\_s(t3.pqtru\_s()\(*\)vhhpp.q\_prs()\(*\)t2.rpq\_s())\tabularnewline
\((t_2 t_3)_c\) & \(-\frac{1}{2}\) & \(\hat{P}(a/bc)\) &
\[  \sum_{ldme} \langle l m \vert \vert d e \rangle t^{d a}_{l m}t^{e b c}_{i j k}  \rightarrow  \sum_{e} \sum_{lmd} t^{ibjck}_{e} \langle e\vert \vert lmd\rangle t^{lmd}_{a} \]
&
update\_as\_sqtru\_p(t3.sqtru\_p()\(*\)vhhpp.s\_pqr()\(*\)t2.rsp\_q())\tabularnewline
\((t_2 t_3)_d\) & \(-\frac{1}{2}\) & \(\hat{P}(k/ij \vert a/bc)\) &
\[  \sum_{ldme} \langle l m \vert \vert d e \rangle t^{a d}_{i j}t^{b e c}_{l m k}  \rightarrow  \sum_{lme} \sum_{d} t^{bck}_{lme} \langle lme\vert \vert d\rangle t^{d}_{aij} \]
&
update\_as\_qru\_pst(t3.pru\_stq()\(*\)vhhpp.pqs\_r()\(*\)t2.q\_prs())\tabularnewline
\((t_2 t_3)_e\) & \(-\frac{1}{2}\) & \(\hat{P}(i/jk \vert c/ab)\) &
\[  \sum_{ldme} \langle l m \vert \vert d e \rangle t^{a b}_{i l}t^{d e c}_{j m k}  \rightarrow  \sum_{mde} \sum_{l} t^{jck}_{mde} \langle mde\vert \vert l\rangle t^{l}_{abi} \]
&
update\_as\_tru\_pqs(t3.sru\_tpq()\(*\)vhhpp.qrs\_p()\(*\)t2.s\_pqr())\tabularnewline
\((t_2 t_3)_f\) & \(+\frac{1}{4}\) & \(\hat{P}(k/ij)\) &
\[  \sum_{ldme} \langle l m \vert \vert d e \rangle t^{d e}_{i j}t^{a b c}_{l m k}  \rightarrow  \sum_{lm} \sum_{de} t^{abck}_{lm} \langle lm\vert \vert de\rangle t^{de}_{ij} \]
&
update\_as\_pqru\_st(t3.pqru\_st()\(*\)vhhpp.pq\_rs()\(*\)t2.pq\_rs())\tabularnewline
\((t_2 t_3)_q\) & \(+\frac{1}{4}\) & \(\hat{P}(c/ab)\) &
\[  \sum_{ldme} \langle l m \vert \vert d e \rangle t^{a b}_{l m}t^{d e c}_{i j k}  \rightarrow  \sum_{de} \sum_{lm} t^{ijck}_{de} \langle de\vert \vert lm\rangle t^{lm}_{ab} \]
&
update\_as\_stru\_pq(t3.stru\_pq()\(*\)vhhpp.rs\_pq()\(*\)t2.rs\_pq())\tabularnewline
\bottomrule
\end{longtable}\subsection{The \((t_2 t_2)\) terms}\label{the-tux5f2-tux5f2-terms}These are incorrectly generated due to unconnected lines in the
interaction, so they are not yet ready for implementation.

Special attention will need to be given to the antisymmetric elements in
the multiplication.\begin{longtable}[c]{@{}lllll@{}}
\toprule
Diagram Label & Factor & Permutation & Index Transform & Code
translation\tabularnewline
\midrule
\endhead
\((t_2 t_2)_a\) & \(-1\) & \(\hat{P}(k/ij \vert a/bc)\) &
\[  \sum_{ld} \langle l \vert f \vert d \rangle t^{a d}_{i j}t^{b c}_{l k}  \rightarrow  \sum_{l} \sum_{d} t^{bck}_{l} \langle l\vert f \vert d\rangle t^{d}_{aij} = 0\]
& (canonical HF basis) \(\rightarrow\) no contribution\tabularnewline
\((t_2 t_2)_b\) & \(+1\) & \(\hat{P}(i/jk \vert abc)\) &
\[  \sum_{lde} \langle l b \vert \vert d e \rangle t^{a d}_{i l}t^{e c}_{j k}  \rightarrow  \sum_{ld} \sum_{e} (t^{ai}_{ld} \langle ld\vert \vert be\rangle)^{aib}_e t^{e}_{cjk} \]
&
update\_as\_psq\_rtu(t2.pr\_sq()\(*\)vhppp.pr\_qs()\(*\)t2.p\_qrs())\tabularnewline
\((t_2 t_2)_c\) & \(-\frac{1}{2}\) & \(\hat{P}(i/jk \vert c/ab)\) &
\[  \sum_{ldce} \langle l c \vert \vert d e \rangle t^{a b}_{i l}t^{d e}_{j k}  \rightarrow  \sum_{de} \sum_{l} (t^{jk}_{de} \langle de\vert \vert lc\rangle)^{jkc}_l t^{l}_{abi} \]
&
update\_as\_tur\_pqs(t2.rs\_pq()\(*\)vhppp.rs\_pq()\(*\)t2.s\_pqr())\tabularnewline
\((t_2 t_2)_d\) & \(+\frac{1}{2}\) & \(\hat{P}(k/ij \vert a/bc)\) &
\[  \sum_{ldmk} \langle l m \vert \vert d k \rangle t^{a d}_{i j}t^{b c}_{l m}  \rightarrow  \sum_{lm} \sum_{d} (t^{bc}_{lm} \langle lm\vert \vert dk\rangle)^{bck}_d t^{d}_{aij} \]
&
update\_as\_qru\_pst(t2.pq\_rs()\(*\)vhhph.pq\_rs()\(*\)t2.q\_prs())\tabularnewline
\bottomrule
\end{longtable}The problem of unconnected lines leaving the interaction may be solved
by performing the multiplication and alignment in three steps:

\begin{enumerate}
\def\labelenumi{\arabic{enumi}.}
\itemsep1pt\parskip0pt\parsep0pt
\item
  Align and multiply inside paranthesis.
\item
  Align the resulting product to the final amplitude and multiply.
\item
  Align the resulting product to the amplitudes.
\end{enumerate}\subsection{The linear \(t_3\) terms}\label{the-linear-tux5f3-terms}The following terms are linear in the \(t_3\) amplitude.

\begin{longtable}[c]{@{}lllll@{}}
\toprule
Diagram Label & Factor & Permutation & Index Transform & Code
translation\tabularnewline
\midrule
\endhead
\((t_3)_a\) & \(+\frac{1}{2}\) & \(\hat{P}(c/ab)\) &
\[\sum_{de} \langle ab \vert \vert de \rangle t^{dec}_{ijk} \rightarrow \sum_{de} \langle ab \vert \vert de \rangle t^{de}_{cijk}\]
& update\_as\_pq\_rstu(vpppp.pq\_rs() \(*\)
t3.pq\_rstu())\tabularnewline
\((t_3)_b\) & \(+\frac{1}{2}\) & \(\hat{P}(k/ij)\) &
\[\sum_{lm} \langle lm \vert \vert ij \rangle t^{abc}_{lmk} \rightarrow \sum_{lm} t^{abck}_{lm} \langle lm \vert \vert ij \rangle \]
& update\_as\_pqru\_st(t3.pqrs\_tu()\(*\) vphhp.pq\_rs())\tabularnewline
\((t_3)_c\) & \(+1\) & \(\hat{P}(i/jk \vert a/bc)\) &
\[\sum_{ld} \langle al \vert \vert id \rangle t^{dbc}_{ljk} \rightarrow \sum_{ld} \langle ai \vert \vert ld \rangle t^{ld}_{bcjk}\]
& update\_as\_ps\_qrtu(vphhp.pr\_qs() \(*\)
t3.sp\_qrtu())\tabularnewline
\bottomrule
\end{longtable}The diagram \((t_3)_a\) includes the ladder operator from \(\hat{V}\),
so it will have to be calulcated using some block scheme.\subsection{The linear \(t_2\) terms}\label{the-linear-tux5f2-terms}\begin{longtable}[c]{@{}lllll@{}}
\toprule
Diagram Label & Factor & Permutation & Index Transform & Code
translation\tabularnewline
\midrule
\endhead
\((t_2)_a\) & \(+1\) & \(\hat{P}(k/ij \vert a/bc)\) &
\[\sum_{d} \langle bc \vert \vert dk \rangle t^{ad}_{ij} \rightarrow \sum_d \langle bck \vert \vert d \rangle t^d_{aij} \]
& update\_as\_qru\_pst(vppph.pqs\_r() \(*\) t2.q\_prs())\tabularnewline
\((t_2)_b\) & \(-1\) & \(\hat{P}(i/jk \vert c/ab)\) &
\[\sum_{l} \langle lc \vert \vert jk \rangle t^{ab}_{il} \rightarrow \sum_l t^{abi}_{l} \langle l \vert \vert cjk \rangle  \]
& update\_as\_pqs\_rtu(t2.pqr\_s() \(*\) vhphh.p\_qrs() )\tabularnewline
\bottomrule
\end{longtable}\section{Implementation}\label{implementation}

Finally we state the basic implementation of the t3 amplitude equation.
Tables is separated by horizontal lines.

Note that permutations are not yet included.

\begin{center}\rule{0.5\linewidth}{\linethickness}\end{center}

t2t3a.update\_as\_qtru\_ps(t3.qtru\_sp()\(*\)vhhpp.qs\_pr()\(*\)t2.sq\_pr())

t2t3b.update\_as\_pqtru\_s(t3.pqtru\_s()\(*\)vhhpp.q\_prs()\(*\)t2.rpq\_s())

t2t3c.update\_as\_sqtru\_p(t3.sqtru\_p()\(*\)vhhpp.s\_pqr()\(*\)t2.rsp\_q())

t2t3d.update\_as\_qru\_pst(t3.pru\_stq()\(*\)vhhpp.pqs\_r()\(*\)t2.q\_prs())

t2t3e.update\_as\_tru\_pqs(t3.sru\_tpq()\(*\)vhhpp.qrs\_p()\(*\)t2.s\_pqr())

t2t3f.update\_as\_pqru\_st(t3.pqru\_st()\(*\)vhhpp.pq\_rs()\(*\)t2.pq\_rs())

t2t3g.update\_as\_stru\_pq(t3.stru\_pq()\(*\)vhhpp.rs\_pq()\(*\)t2.rs\_pq())

\begin{center}\rule{0.5\linewidth}{\linethickness}\end{center}

t2t2b.update\_as\_psq\_rtu(t2.pr\_sq()\(*\)vhppp.pr\_qs()\(*\)t2.p\_qrs())

t2t2c.update\_as\_tur\_pqs(t2.rs\_pq()\(*\)vhppp.rs\_pq()\(*\)t2.s\_pqr())

t2t2d.update\_as\_qru\_pst(t2.pq\_rs()\(*\)vhhph.pq\_rs()\(*\)t2.q\_prs())

\begin{center}\rule{0.5\linewidth}{\linethickness}\end{center}

t3a.update\_as\_pq\_rstu(vpppp.pq\_rs() \(*\) t3.pq\_rstu()) //Note that
this will probably be replaced by a block implementation.

t3b.update\_as\_pqru\_st(t3.pqrs\_tu()\(*\) vphhp.pq\_rs())

t3c.update\_as\_ps\_qrtu(vphhp.pr\_qs() \(*\) t3.sp\_qrtu())

\begin{center}\rule{0.5\linewidth}{\linethickness}\end{center}

t2a.update\_as\_qru\_pst(vppph.pqs\_r() \(*\) t2.q\_prs())

t2b.update\_as\_pqs\_rtu(t2.pqr\_s() \(*\) vhphh.p\_qrs() )\section{CCDT: Diagrams in the \(t_2\)
amplitude}\label{ccdt-diagrams-in-the-tux5f2-amplitude}Actually, the inclusion of tripes will only result in three extra terms
in the doubles equation, and we may even remove the first due to the
fact that we use a canonical HF basis where \(f\) is diagonal.\[ D_{CCD} + \sum_{me} f^m_e t^{abe}_{ijm} + \frac{1}{2} \hat{P}(ab) \sum_{mef} \langle bm \vert \vert ef \rangle t^{aef}_{ijm}  - \frac{1}{2}\hat{P}(ij) \sum_{mne} \langle mn \vert \vert je \rangle t^{abe}_{imn} = 0 \]The terms we need to include in the \(t_2\) equation is then (labelling
as in Shavitt and Bartlett):\begin{longtable}[c]{@{}lllll@{}}
\toprule
Diagram Label & Factor & Permutation & Index Transform & Code
translation\tabularnewline
\midrule
\endhead
\(D_{10b}\) & \(+\frac{1}{2}\) & \(\hat{P}(ab)\) &
\[\sum_{mef} \langle bm \vert \vert ef \rangle t^{aef}_{ijm} \rightarrow (\sum_{mef} \langle b \vert \vert mef \rangle t^{mef}_{ija})^{ab}_{ij} \]
& update\_as\_q\_rsp(vphpp.p\_qrs() \(*\) t3.uqr\_stp() )\tabularnewline
\(D_{10c}\) & \(-\frac{1}{2}\) & \(\hat{P}(ij)\) &
\[\sum_{mne} \langle mn \vert \vert je \rangle t^{abe}_{imn} \rightarrow (\sum_{mne} t^{abi}_{mne} \langle mne \vert \vert j \rangle)^{ab}_{ij}  \]
& update\_as\_pqr\_s(t3.pqs\_tur() \(*\) vhhhp.pqs\_r())\tabularnewline
\bottomrule
\end{longtable}This means we have to add in the following when computing the doubles
contribution

D10b.update\_as\_q\_rsp(vphpp.p\_qrs() ∗ t3.uqr\_stp())

D10c.update\_as\_pqr\_s(t3.pqs\_tur() ∗ vhhhp.pqs\_r())

    % Make sure that atleast 4 lines are below the HR
    \needspace{4\baselineskip}

    
        \vspace{6pt}
        \makebox[0.1\linewidth]{\smaller\hfill\tt\color{nbframe-in-prompt}In\hspace{4pt}{[}{]}:\hspace{4pt}}\\*
        \vspace{-2.65\baselineskip}
        \begin{ColorVerbatim}
            \vspace{-0.7\baselineskip}
            \begin{Verbatim}[commandchars=\\\{\}]

\end{Verbatim}

            
                \vspace{0.3\baselineskip}
            
        \end{ColorVerbatim}
    

        

        \renewcommand{\indexname}{Index}
        \printindex

    % End of document
    \end{document}


