\documentstyle[a4wide,11pt]{article}

\begin{document}

\pagestyle{plain}

\begin{center} \huge \bf Cand. Scient. project for Simen Kvaal \end{center}

\section*{Time-dependent Schr\"odinger equation for systems of trapped bosons and fermions}
This thesis project has three main aims:
\begin{enumerate} 
\item \underline{Numerical axis}
The first part of this thesis project deals with  
a critical evaluation of the finite difference and the finite element methods 
applied to the solution of Schr\"odinger equation for systems of 
one and two particles. 
Simen is familiar with several aspects of partial differential
equations through the course IN-NMFPD I, but since the set of equations
are complex, this entails several open problems to be explored in his thesis
work, among these large systems of equations for implicit schemes.
The particle(s) is(are) 
confined in one, two and three-dimesional 
traps (e.g., a symmetric harmonic oscillator)
with possible complicating geometries. 
In addition, the effect of time-dependent fields will be studied, such as
a Laser pulse acting on the particle(s) in the trap. 
For two particles, this entails the solution of 
a set of partial-differential-integral equations. 
This has never been done properly and will form an important contribution
to our understanding of systems of two interacting particles.
\item \underline{Visualization axis} 
An important aspect of this thesis project is the visualization of the quantal
probability distribution, 
both of its spatial dependence and time-dependence, and especially
after the action of e.g., a time-dependent electromagnetic field. 
Simen has already worked on the visualization aspect, see his 
movies at  
\begin{verbatim}
http://www.fys.uio.no/~simenkva/kvekk/index.shtml.
\end{verbatim}
This piece has important pedagogical features since the capability of 
portraying a quantal probability distribution may improve  our understanding 
of matter waves.
\item \underline{Physics axis}   
The time-dependent equation to be solved takes the following form 
for one particle
\begin{equation}
 -\frac{\hbar^2\nabla^2}{2m}\psi({\bf r},t)
  +V_{\mathrm{ext}}({\bf r})\psi({\bf r},t)+ V_t({\bf r},t)\psi({\bf r},t)= 
  i\hbar \frac{\partial \psi({\bf r},t)}{\partial t}.
\end{equation}
The spatial 
dimensionality is defined by the specific system under consideration, see 
below. The one-body time-independent potential $V_{\mathrm{ext}}$ 
could e.g., represent 
the features of a Paul trap. This can be approximated with e.g., a symmetric
(S) harmonic potential or an elliptic (E) harmonic potential in three dimensions
 \begin{equation}
V_{\mathrm{ext}}({\bf r}) = 
\Bigg\{
\begin{array}{ll}
        \frac{1}{2}m\omega_{ho}^2r^2 & (S)\\
\strut
	\frac{1}{2}m[\omega_{ho}^2(x^2+y^2) + \omega_z^2z^2] & (E)
\label{trap_eqn}
\end{array}
\end{equation}
Here $\omega_{ho}^2$ defines the trap potential strength.  In the case of the
elliptical trap, $V_{ext}(x,y,z)$, $\omega_{ho}=\omega_{\perp}$ is the trap frequency
in the perpendicular or $xy$ plane and $\omega_z$ the frequency in the $z$
direction.

The solution of this problem entails writing a program which can accomodate 
several such external potentials, with differing dimensionalities.
There is also the possibility of adding an external time-dependent potential
$V_t$, e.g., through a temporal electric or magnetic field 
applied to the system.

With two particles in the trap we have the following equation
\begin{eqnarray}
& \left[-\frac{\hbar^2\nabla_1^2}{2m}-\frac{\hbar^2\nabla_2^2}{2m}
  +V({\bf r}_1, {\bf r}_2) + V_{\mathrm{ext}}({\bf r}_1)+V_{\mathrm{ext}}({\bf r}_2)+V_{t}({\bf r}_1,t)+V_{t}({\bf r}_2,t)\right]
   \psi({\bf r}_1, {\bf r}_2,t) \nonumber \\
&=i\hbar \frac{\partial \psi({\bf r}_1, {\bf r}_2,t)}{\partial t},
\end{eqnarray}
where the two-body interaction represents the system under study.
For a quantum dot, with localized electrons in two-dimensions, we have the
repulsive Coulomb potential 
\begin{equation}
 V({\bf r}_1, {\bf r}_2)=\frac{e^2}{4\pi\epsilon_0r_{12}},
\end{equation}
where $r_{12}=|{\bf r}_1-{\bf r}_2|$.
The potential could obviously also represent other physical 
systems, such as the potential between two bosons. 
Note well that the external potential could be replaced with the Coulomb
attraction from say a nucleus. The above two-body equation could then 
represent the Helium-atom.
The program which solves these equations allows for a variety of two-particle
interactions in various dimensions. As such, it can be applied to many 
physical systems of current interest.
The development of techniques for solving the above equations for two 
interacting particles is another important part of this thesis.


Of special interest for this thesis is the study of quantum dots.
Semiconductor quantum dots are structures where
charge carriers are confined in all three spatial dimensions,
the dot size being of the order of the Fermi wavelength
in the host material, typically between  10 nm and  1 $\mu$m.
The confinement is usually achieved by electrical gating of a
two-dimensional electron gas (2DEG),
possibly combined with etching techniques. Precise control of the
number of electrons in the conduction band of a quantum dot
(starting from zero) has been achieved in GaAs heterostructures.
The electronic spectrum of typical quantum dots
can vary strongly when an external magnetic field is applied,
since the magnetic length corresponding to typical
laboratory fields  is comparable to typical dot sizes.
The role played by such an external field magnetic or electric field
will be studied here. 

Quantum dots form  an interesting quantum mechanical topic per se and 
there is quite a lot of 
research going on in this field due to the possibility of making
quantum circuits based on quantum dots. 
There are thus several ramifications which can studied within this thesis.
Another possibility is to study 
coupled quantum dots. These provide a powerful source of
deterministic entanglement between localized
but also of delocalized electrons. It is for example 
possible to create a singlet state out of two electrons
and subsequently separate (by electronic transport)
the two electrons spatially with the spins of the two electrons still being
entangled--the prototype of an EPR pair.
This opens up the possibility to study a new class
of quantum phenomena in electronic nanostructures
such as the entanglement and
non-locality of electronic EPR pairs.




\end{enumerate}

\section*{Summary}
In summary, this thesis deals with the study and developments of stable
numerical approaches to Schr\"odinger's equation for systems of one and
two particles confined in space, under the action of external time-independent and/or time-dependent one-body interactions, and/or
two-particle interactions. Applications are to systems of locally
confined electrons, but the program to be developed opens up for studies of
other quantal systems of spatially confined particles.
The computational aspect involves the study of large systems of 
complex equations. This is an important part of the thesis work. 
The thesis aims also at a study of ways to
visualize the quantal probability distribution.
 

\end{document}












