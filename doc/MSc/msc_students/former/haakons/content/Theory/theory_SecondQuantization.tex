Let us look at a different approach to the quantum many-body systems, namely a formalism called second quantization. Introduced by Dirac in 1927 \cite{DiracSQ}, and further developed by among others Vladimir Fock \cite{FockSQ} and Pascual Jordan \cite{JordanSQ}, the techniques used in second-quantization met a need to solve problems where the number of particles is not fixed or known with a background of independent-particle models, such as the Hartree-Fock model. 

Second quantization provides a way to write independent-particle-model wave functions, that is the Slater determinant, as previously discussed in this thesis. It also gives a method of representing operators in a efficient way, and easily manipulating these functions and operators.

To describe second quantization we introduce the time-independent operators $a^{\dagger}_{\alpha}$ and $a_{\alpha}$, which respectively create and annihilate a particle in the single-particle state $ \phi_{\alpha}$. Also, we define the fermion creation operator $a^{\dagger}_{\alpha}$
\[
	a_{\alpha}^{\dagger}\left|0\right\rangle \equiv\left|\alpha\right\rangle 
\]
where the operator a acts on the vacuum state 0. The vacuum state does not contain any particles. We also define 
\[
	a_{\alpha}^{\dagger}\left|\alpha_{1}\dots\alpha_{n}\right\rangle_{AS} \equiv\left|\alpha\alpha_{1}\dots\alpha_{n}\right\rangle_{AS}
\]
where the operator $a_{\alpha}^{\dagger}$ acts on an antisymmetric n-particle state, creating an antisymmetric $(n+1)$-particle state. Here the one-body state $\phi_{\alpha}$ is occupied, under the condition $\alpha\neq\alpha_{1}\dots\alpha_{n}$. We can thus express an antisymmetric state as the product of the creation operators acting on the vacuum state
\begin{equation}\label{eq:ASstateProduct}
	\left|\alpha_{1}\dots\alpha_{n}\right\rangle_{AS}=a_{\alpha_{1}}^{\dagger}a_{\alpha_{2}}^{\dagger}\dots a_{\alpha_{n}}^{\dagger}\left|0\right\rangle
\end{equation}

To find the commutation and anticommutation rules for the fermionic creation operators $a_{\alpha}^{\dagger}$, we use the antisymmetry
\[
	\left|\alpha_{1}\dots\alpha_{i}\dots\alpha_{k}\dots\alpha_{n}\right\rangle_{AS}=-\left|\alpha_{1}\dots\alpha_{k}\dots\alpha_{i}\dots\alpha_{n}\right\rangle_{AS}
\]
and get
\begin{equation}\label{eq:ASofStates}
	a_{\alpha_{i}}^{\dagger}a_{\alpha_{k}}^{\dagger}=-a_{\alpha_{k}}^{\dagger}a_{\alpha_{i}}^{\dagger}
\end{equation}

From the Pauli principle we know that
\[
	\left|\alpha_{1}\dots\alpha_{i}\dots\alpha_{i}\dots\alpha_{n}\right\rangle_{AS}=0
\]
thus
\begin{equation}\label{eq:ASPauli}
	a_{\alpha_{i}}^{\dagger}a_{\alpha_{i}}^{\dagger}=0
\end{equation}

Combining equations \ref{eq:ASofStates} and \ref{eq:ASPauli} we get the anti-commutation rule
\begin{equation}\label{eq:AntiCommutation}
	a_{\alpha}^{\dagger}a_{\beta}^{\dagger}+a_{\beta}^{\dagger}a_{\alpha}^{\dagger}\equiv\left\{ a_{\alpha}^{\dagger},\, a_{\beta}^{\dagger}\right\} =0
\end{equation}

Furthermore the hermitian conjugate of $a_{\alpha}^{\dagger}$ is
\[
	a_{\alpha}=\left(a_{\alpha}^{\dagger}\right)^{\dagger}
\]
Taking the hermitian conjugate of equation \ref{eq:AntiCommutation}, we finally get
\[
	\left\{ a_{\alpha},\, a_{\beta}\right\} = 0
\]


Let us look at the effect of $a_{\alpha}$ on a given state $\left|\alpha_{1}\alpha_{2}\dots\alpha_{n}\right\rangle_{AS}$, by considering the matrix element
\begin{equation}\label{eq:matrixElement}
	\left\langle\alpha_{1}\alpha_{2}\dots\alpha_{n}\right | a_{\alpha} \left | \alpha_{1}'\alpha_{2}'\dots\alpha_{m}'\right\rangle
\end{equation}

If $\alpha \in \left\{\alpha_{i}\right\}$, by using the Pauli principle we get
\[
	\left\langle\alpha_{1}\alpha_{2}\dots\alpha_{n}\right | a_{\alpha} = 0
\]

If on the other hand $\alpha \notin \left\{\alpha_{i}\right\}$, we get an hermitian conjugation
\[
	\left\langle\alpha_{1}\alpha_{2}\dots\alpha_{n}\right | a_{\alpha} = \left\langle\alpha\alpha_{1}\alpha_{2}\dots\alpha_{n}\right |
\]

We can rewrite equation \ref{eq:matrixElement} as 
\begin{equation}\label{eq:matrixElementRewritten}
	\left\langle\alpha_{1}\alpha_{2}\dots\alpha_{n}\right | a_{\alpha} \left | \alpha_{1}'\alpha_{2}'\dots\alpha_{m}'\right\rangle = \left\langle\alpha_{1}\alpha_{2}\dots\alpha_{n}\right | \alpha\alpha_{1}'\alpha_{2}'\dots\alpha_{m}'
\end{equation}
For it to be nonzero we must have $m=n+1$. We thus get
\begin{equation}\label{eq:matrixElementCases}
	\langle\alpha_{1}\alpha_{2}\dots\alpha_{n}|a_{\alpha}|\alpha_{1}'\alpha_{2}'\dots\alpha_{m}'\rangle=\begin{cases}
	0 & \alpha\in\left\{ \alpha_{i}\right\} \lor\left\{ \alpha\alpha_{i}\right\} \neq\left\{ \alpha_{i}'\right\} \\
	\pm1 & \alpha\notin\left\{ \alpha_{i}\right\} \cup\left\{ \alpha\alpha_{i}\right\} =\left\{ \alpha_{i}'\right\} 
	\end{cases}
\end{equation}

For the second of the two cases the plus minus sign applies when the sequences $\alpha_{1}\alpha_{2}\dots\alpha_{n}$ and $\alpha_{1}'\alpha_{2}'\dots\alpha_{m}'$ are related via even and odd permutations. Assuming that $\alpha \notin \left\{\alpha_{i}\right\}$ and $\alpha \in \left\{\alpha_{i}'\right\}$ we have

\[
	\left\langle\alpha_{1}\alpha_{2}\dots\alpha_{n}\right | a_{\alpha} \left | \alpha_{1}'\alpha_{2}'\dots\alpha_{n+1}'\right\rangle = 0
\]
And if we assume that $\alpha \notin \left\{\alpha_{i}'\right\}$
\[
	a_{\alpha}\underset{\neq\alpha}{\underbrace{\left|\alpha_{1}'\alpha_{2}'\dots\alpha_{n+1}'\right\rangle }}=0
\]
and 
\[
	a_{\alpha}\left | 0 \right\rangle=0
\]

The sequence $\alpha_{1}\alpha_{2}\dots\alpha_{n}$ is identical to $\alpha_{1}'\alpha_{2}'\dots\alpha_{n+1}'$ if $\left\{\alpha\alpha_{i}\right\}=\left\{\alpha_{i}'\right\}$ and we perform the right permutations, yielding
\[
	\left\langle \alpha_{1}\alpha_{2}\dots\alpha_{n} \right | a_{\alpha} \left | \alpha \alpha_{1}\alpha_{2}\dots\alpha_{n} \right \rangle = 1
\]
and thus 
\[
	a_{\alpha} \left | \alpha \alpha_{1}\alpha_{2}\dots\alpha_{n} \rangle = | \alpha_{1}\alpha_{2}\dots\alpha_{n} \right \rangle
\]

The operator $a_{\alpha}$, as we can see, works on a state vector on its right side and removes one particle in the state $\alpha$. If there is no single-particle state $\alpha$ in the state vector, the outcome of the operation is zero. 

It can be shown that the commutator algebra for the creation and annihilation operators boils down to 
\[
\left\{ a_{\alpha}^{\dagger},\, a_{\beta}^{\dagger}\right\} =\left\{ a_{\alpha},\, a_{\beta}\right\} =0\qquad\mbox{and}\qquad\left\{ a_{\alpha}^{\dagger},\, a_{\beta}\right\} =\delta_{\alpha\beta}
\]
where $\delta_{\alpha\beta}$ is the Kronecker $\delta$ symbol, giving $1$ if $\alpha=\beta$ and $0$ otherwise.

We are now familiar with two new operators: a creation operator $a_{\alpha}^{\dagger}$, which adds one particle to a single-particle state $\alpha$ of a given many-body state vector, and an annihilation operator $a_{\alpha}$, which removes a particle from the single-particle state $\alpha$.

Next we want to find an expression for the one-body operator for the kinetic energy. In coordinate space this operator is
\[
	\hat{H}_{0}=\sum_{i}\hat{h}_{0}(x_{i})
\]
Furthermore we define
\[
	\hat{h}_{0}\psi_{\alpha_{i}}(x_{i})=\sum_{\alpha_{k}'}\psi_{\alpha_{i}'}(x_{i}) \braVket{\alpha_{k}'}{\hat{h}_{0}}{\alpha_{k}}
\]
We have the anti-symmetric n-particle Slater determinant, as defined by
\[
	\phi\left(x_{1},\, x_{2},\,\dots,\, x_{n},\,\alpha_{1},\,\alpha_{2},\,\dots,\,\alpha_{n}\right)=\frac{1}{\sqrt{n!}}\sum_{p}\left(-1\right)^{p}\psi_{\alpha_{1}}\left(x_{1}\right)\psi_{\alpha_{2}}\left(x_{2}\right)\dots\psi_{\alpha_{n}}\left(x_{n}\right)
\]
Using this we obtain
\footnotesize
\begin{eqnarray*}
	\left(\sum_{i}\hat{h}_{0}\left(x_{i}\right)\right)\psi_{\alpha_{1}}\left(x_{1}\right)\psi_{\alpha_{2}}\left(x_{2}\right)\dots\psi_{\alpha_{n}}\left(x_{n}\right) & = & \sum_{\alpha_{1}'}\braVket{\alpha_{1}'}{\hat{h}_{0}}{\alpha_{1}}\psi_{\alpha_{1}'}\left(x_{1}\right)\psi_{\alpha_{2}}\left(x_{2}\right)\dots\psi_{\alpha_{n}}\left(x_{n}\right)\\
	 & + & \sum_{\alpha_{2}'}\braVket{\alpha_{2}'}{\hat{h}_{0}}{\alpha_{2}}\psi_{\alpha_{1}}\left(x_{1}\right)\psi_{\alpha_{2}'}\left(x_{2}\right)\dots\psi_{\alpha_{n}}\left(x_{n}\right)\\
	 &  & \vdots\\
	 & + & \sum_{\alpha_{n}'}\braVket{\alpha_{n}'}{\hat{h}_{0}}{\alpha_{n}}\psi_{\alpha_{1}}\left(x_{1}\right)\psi_{\alpha_{2}}\left(x_{2}\right)\dots\psi_{\alpha_{n}'}\left(x_{n}\right)
\end{eqnarray*}
\normalsize
We can interchange the positions of particle 1 and 2 and get
\footnotesize
\begin{eqnarray*}
	\left(\sum_{i}\hat{h}_{0}\left(x_{i}\right)\right)\psi_{\alpha_{1}}\left(x_{2}\right)\psi_{\alpha_{1}}\left(x_{2}\right)\dots\psi_{\alpha_{n}}\left(x_{n}\right) & = & \sum_{\alpha_{2}'}\braVket{\alpha_{2}'}{\hat{h}_{0}}{\alpha_{2}}\psi_{\alpha_{1}}\left(x_{2}\right)\psi_{\alpha_{2}'}\left(x_{1}\right)\dots\psi_{\alpha_{n}}\left(x_{n}\right)\\
	 & + & \sum_{\alpha_{1}'}\braVket{\alpha_{1}'}{\hat{h}_{0}}{\alpha_{1}}\psi_{\alpha_{1}'}\left(x_{2}\right)\psi_{\alpha_{2}}\left(x_{1}\right)\dots\psi_{\alpha_{n}}\left(x_{n}\right)\\
	 &  & \vdots\\
	 & + & \sum_{\alpha_{n}'}\braVket{\alpha_{n}'}{\hat{h}_{0}}{\alpha_{n}}\psi_{\alpha_{1}}\left(x_{2}\right)\psi_{\alpha_{1}}\left(x_{2}\right)\dots\psi_{\alpha_{n}'}\left(x_{n}\right)
\end{eqnarray*}
\normalsize
The other permutations continue in a similar fashion. Now we rewrite our Slater determinant in its second quantized form, skipping the dependence on the quantum numbers $x_{i}$. We sum up all contributions and get
\begin{eqnarray*}
	\hat{H}_{0}\left|\alpha_{1},\,\alpha_{2},\,\dots,\,\alpha_{n}\right\rangle  & = & \sum_{\alpha_{1}'}\braVket{\alpha_{1}'}{\hat{h}_{0}}{\alpha_{1}}\left|\alpha_{1}',\,\alpha_{2},\,\dots,\,\alpha_{n}\right\rangle \\
	 & + & \sum_{\alpha_{2}'}\braVket{\alpha_{2}'}{\hat{h}_{0}}{\alpha_{2}}\left|\alpha_{1},\,\alpha_{2}',\,\dots,\,\alpha_{n}\right\rangle \\
	 &  & \vdots\\
	 & + & \sum_{\alpha_{n}'}\braVket{\alpha_{n}'}{\hat{h}_{0}}{\alpha_{n}}\left|\alpha_{1},\,\alpha_{2},\,\dots,\,\alpha_{n}'\right\rangle 
\end{eqnarray*}
Now we use the properties of the creation and annihilation operators to further rewrite our one-body equation. This means inserting
\[
	\left|\alpha_{1}\alpha_{2}\dots\alpha_{k}'\dots\alpha_{n}\right\rangle = a_{\alpha_{k}'}^{\dagger}a_{\alpha_{k}} \left|\alpha_{1}\alpha_{2}\dots\alpha_{k}\dots\alpha_{n}\right\rangle 
\]
in our equation, and we finally get
\begin{eqnarray*}
	\hat{H}_{0}\left|\alpha_{1}\alpha_{2}\dots\alpha_{n}\right\rangle  & = & \sum_{\alpha_{1}'}\braVket{\alpha_{1}'}{\hat{h}_{0}}{\alpha_{1}}a_{\alpha_{1}'}^{\dagger}a_{\alpha_{1}}\left|\alpha_{1}\alpha_{2}\dots\alpha_{n}\right\rangle \\
	 & + & \sum_{\alpha_{2}'}\braVket{\alpha_{2}'}{\hat{h}_{0}}{\alpha_{2}}a_{\alpha_{2}'}^{\dagger}a_{\alpha_{2}}\left|\alpha_{1}\alpha_{2}\dots\alpha_{n}\right\rangle \\
	 &  & \vdots\\
	 & + & \sum_{\alpha_{n}'}\braVket{\alpha_{n}'}{\hat{h}_{0}}{\alpha_{n}}a_{\alpha_{n}'}^{\dagger}a_{\alpha_{n}}\left|\alpha_{1}\alpha_{2}\dots\alpha_{n}\right\rangle \\
	 & = & \sum_{\alpha\beta}\braVket{\alpha}{\hat{h}_{0}}{\beta}a_{\alpha'}^{\dagger}a_{\beta}\left|\alpha_{1}\alpha_{2}\dots\alpha_{n}\right\rangle 
\end{eqnarray*}

In a similar fashion one can derive the two-body interaction part, and get
\[
	\hat{H}_{I}=\frac{1}{2}\sum_{\alpha\beta\gamma\delta}\braVket{\alpha\beta}{V}{\gamma\delta}a_{\alpha}^{\dagger}a_{\beta}^{\dagger}a_{\delta}a_{\gamma}
\]
Or in terms of anti-symmetrized matrix elements
\[
	\hat{H}_{I}=\frac{1}{4}\sum_{\alpha\beta\gamma\delta}\braVket{\alpha\beta}{V}{\gamma\delta}a_{\alpha}^{\dagger}a_{\beta}^{\dagger}a_{\delta}a_{\gamma}
\]
In summary we can express the Hamiltonian operator for a many-body system consisting of fermions as 
\[
	H=\sum_{\alpha\beta}\braVket{\alpha}{t+u}{\beta}a_{\alpha}^{\dagger}a_{\beta}+\frac{1}{4}\sum_{\alpha\beta\gamma\delta}\braVket{\alpha\beta}{V}{\gamma\delta}a_{\alpha}^{\dagger}a_{\beta}^{\dagger}a_{\delta}a_{\gamma}
\]

\section{Particle-hole formalism}
	Although second quantization gives an elegant formalism for many-body states and quantum mechanical operators, there is no notable gain in the practicalities of solving the Schrödinger equation. The many-body equation remains just as hard to solve regardless of representation. It does, however make it easy to introduce another reference state than the pure vacuum state, $\left | 0 \right \rangle$. This is often useful with many particles present. We label this new reference state $\left | c \right \rangle$, where $c$ stands for ``core''. This new reference will reduce the complexity and dimensionality of the problem. To use this new reference state we introduce particle-hole formalism.

	We define new creation and annihilation operators, which can act on a state $\alpha$, creating a new quasiparticle state
	\[
		b_{\alpha}^{\dagger}=\begin{cases}
		a_{\alpha}^{\dagger} & \alpha>F\\
		a_{\alpha} & \alpha\leq F
		\end{cases}\qquad\mbox{and}\qquad b_{\alpha}=\begin{cases}
		a_{\alpha} & \alpha>F\\
		a_{\alpha}^{\dagger} & \alpha\leq F
		\end{cases}
	\]
	where $F$ is called the Fermi level and represents the last occupied single-particle orbit of the new reference state $\left | c \right \rangle$.

	The one-body operator 
	\[
		\hat{H}_{0} = \sum_{pq} \braVket{p}{h_{0}}{q}\left\{a_{p}^{\dagger}a_{q}\right\}+\sum_{i}\braVket{i}{h_{0}}{i}
	\]
	can be written in particle-hole notation as 
	\begin{eqnarray*}
		\hat{H}_{0} & = & \sum_{ab}\braVket{a}{h_{0}}{b}b_{a}^{\dagger}b_{b}+\sum_{ai}\left[\braVket{a}{h_{0}}{i}b_{a}^{\dagger}b_{i}^{\dagger}+\braVket{i}{h_{0}}{a}b_{i}b_{a}\right]\\
		 &  & +\sum_{i}\braVket{i}{h_{0}}{i}-\sum_{ij}\braVket{j}{h_{0}}{i}b_{i}^{\dagger}b_{j}
	\end{eqnarray*}

	In a similar fashion, the two-body operator can be written as a two-body part
	\[
		\hat{V}_{N}=\frac{1}{4}\sum_{pqrs}\sum_{ab}\braVket{pq}{\hat{v}}{rs}\left\{ a_{p}^{\dagger}a_{q}^{\dagger}a_{s}a_{r}\right\} 
	\]
	a one-body part
	\[
		\hat{F}_{N}=\sum_{pqi}\sum_{ab}\braVket{pi}{\hat{v}}{qi}\left\{ a_{p}^{\dagger}a_{q}\right\} 
	\]
	and a scalar part
	\[
	E_{0}=\frac{1}{2}\sum_{ij}\sum_{ab}\braVket{ij}{\hat{v}}{ij}
	\]

	In particle-hole formalism we split the two-body operator into five terms
	\[
		\hat{H}_{I}=\hat{H}_{I}^{\left(a\right)}+\hat{H}_{I}^{\left(b\right)}+\hat{H}_{I}^{\left(c\right)}+\hat{H}_{I}^{\left(d\right)}+\hat{H}_{I}^{\left(e\right)}
	\]
	We use anti-symmetrized matrix elements, and the first term is
	\[
		\hat{H}_{I}^{\left(a\right)}=\frac{1}{4}\sum_{abcd}\braVket{ab}{V}{cd}b_{a}^{\dagger}b_{b}^{\dagger}b_{d}b_{c}
	\]
	The next term, $\hat{H}_{I}^{(b)}$, is
	\[
		\hat{H}_{I}^{\left(b\right)}=\frac{1}{2}\sum_{abci}\left(\braVket{ab}{V}{ci}b_{a}^{\dagger}b_{b}^{\dagger}b_{i}^{\dagger}b_{c}+\braVket{ai}{V}{cb}b_{a}^{\dagger}b_{i}b_{b}b_{c}\right)
	\]
	Here we conserve number of quasiparticles, but create or remove a three-particle-hole-state. The next term is $\hat{H}_{I}^{(c)}$ and reads
	\begin{eqnarray*}
		\hat{H}_{I}^{\left(c\right)} & = & \frac{1}{4}\sum_{abij}\left(\braVket{ab}{V}{ij}b_{a}^{\dagger}b_{b}^{\dagger}b_{j}^{\dagger}b_{i}^{\dagger}+\braVket{ij}{V}{ab}b_{a}b_{b}b_{j}b_{i}\right)\\
		 &  & +\sum_{abij}\braVket{ai}{V}{bj}b_{a}^{\dagger}b_{j}^{\dagger}b_{b}b_{i}+\sum_{abi}\braVket{ai}{V}{bi}b_{a}^{\dagger}b_{b}
	\end{eqnarray*}
	This term will in its first term create a two-particle-two-hole state, and in its second term create two one-particle-one-hole pairs. In its last term it gives a contribution to the single-particle energy from the hole states, meaning an interaction between the particle states and the hole states within the new vacuum state. Next we have the fourth term
	\begin{eqnarray*}
		\hat{H}_{I}^{\left(d\right)} & = & \frac{1}{2}\sum_{aijk}\left(\braVket{ai}{V}{jk}b_{a}^{\dagger}b_{k}^{\dagger}b_{j}^{\dagger}b_{i}+\braVket{ji}{V}{ak}b_{k}^{\dagger}b_{j}b_{i}b_{a}\right)\\
		 &  & +\sum_{aij}\left(\braVket{ai}{V}{ji}b_{a}^{\dagger}b_{j}^{\dagger}+\braVket{ji}{V}{ai}b_{j}b_{a}\right)
	\end{eqnarray*}
	The first two terms give a creation of a particle-hole state interacting with hole states, which we can label as a two-hole-one-particle contribution. The last two terms is a particle-hole state interacting with the holes in the vacuum state. The final term in our two-body operator reads
	\[
		\hat{H}_{I}^{\left(e\right)}=\frac{1}{4}\sum_{ijkl}\braVket{kl}{V}{ij}b_{i}^{\dagger}b_{j}^{\dagger}b_{l}b_{k}-\sum_{ijk}\braVket{ij}{V}{kj}b_{k}b_{i}+\frac{1}{2}\sum_{ij}\braVket{ij}{V}{ij}
	\]
	Here the first term describes the interaction between two holes. The second term describes interaction between a hole and the remaining holes in the vacuum state, representing a contribution to a single-hole energy to first order. The last term gathers all contributions of the ground state of a closed-shell system arising from hole-hole correlations. 

\section{Hartree-Fock with second quantization}
	Let us now derive the Hartree-Fock equations using second-quantized formalism. For the ground state of the system the Slater determinant ansatz is approximated as
	\[
		\ket{\phi_0} = \ket{c} = a_{i}^{\dagger}a_{j}^{\dagger}\dots a_{l}^{\dagger}\ket{0}
	\]
	We want to find a $\hat{u}^{HF}$ so that $E_{0}^{HF}=\braVket{c}{\hat{H}}{c}$ becomes a local minimum. Next we use Thouless' theorem \cite{thouless60} to define a new Slater determinant $\ket{c'}$, which is not orthogonal to determinant 
	\[
		\ket{c}=\prod_{i=1}^{n}a_{i}^{\dagger}\ket{0}
	\]
	and the new determinant is defined as
	\[
		\ket{c'}=\exp \left\{ \sum_{a>F}^{\infty}\sum_{i\leq F} C_{ai}a_{a}^{\dagger}a_{i} \right\}\ket{c}
	\]

	When we derive the Hartree-Fock equations the variational condition states only that there exists an extreme value to the expectation value $\braVket{c}{\hat{H}}{c}$, but it is not guaranteed to be a minimum. By using second quantization we can find a criterion for the expectation value to be a local minimum. We will show that
	\[
		\frac{\braVket{c'}{\hat{H}}{c'}}{\bra{c'}\ket{c'}} \geq \braVket{c}{\hat{H}}{c}=E_0
	\]
	where
	\[
		\ket{c'}=\ket{c}+\ket{\delta c}
	\]

	From Thouless' theorem we know we can write out the new Slater determinant $\ket{c'}$ as
	\begin{eqnarray*}
		\ket{c'} & = & \exp\left\{ \sum_{a>F}\sum_{i\leq F}\delta C_{ai}a_{a}^{\dagger}a_{i}\right\} \ket{c}\\
	 	& = & \left\{ 1+\sum_{a>F}\sum_{i\leq F}\delta C_{ai}a_{a}^{\dagger}a_{i}+\frac{1}{2!}\sum_{ab>F}\sum_{ij\leq F}\delta C_{ai}\delta C_{bj}a_{a}^{\dagger}a_{i}a_{b}^{\dagger}a_{j}+\dots\right\} 
	\end{eqnarray*}

	Normalizing $\ket{c'}$ by using a intermediate normalization condition $\bra{c'}\ket{c}=1$, we get
	\[
		\bra{c'}\ket{c'}=1+\sum_{a>F}\sum_{i\leq F}\left | \delta C_{ai} \right |^{2} + O\left(\delta C_{ai}^{3}\right)
	\]

	Now we can use the Hartree-Fock condition and get the expectation value for the energy
	\begin{eqnarray} \label{eq:SQ_HF_EnergyExpVal}
		\braVket{c'}{\hat{H}}{c'} & = & \braVket{c}{\hat{H}}{c}+\sum_{ab>F}\sum_{ij\leq F}\delta C_{ai}^{*}\delta C_{bj}\braVket{c}{a_{i}^{\dagger}a_{a}\hat{H}a_{b}^{\dagger}a_{j}}{c}\\
		 &  & +\frac{1}{2!}\sum_{ab>F}\sum_{ij\leq F}\delta C_{ai}\delta C_{bj}\braVket{c}{\hat{H}a_{a}^{\dagger}a_{i}a_{b}^{\dagger}a_{j}}{c} \nonumber\\
		 &  & +\frac{1}{2!}\sum_{ab>F}\sum_{ij\leq F}\delta C_{ai}^{*}\delta C_{bj}^{*}\braVket{c}{a_{j}^{\dagger}a_{b}a_{i}^{\dagger}a_{a}\hat{H}}{c}+\dots \nonumber
	\end{eqnarray}
	where we skip higher order terms.

	Let us look at the second term of the right hand side of equation \ref{eq:SQ_HF_EnergyExpVal}
	\begin{multline*}
		\braVket{c}{\left\{ a_{a}^{\dagger}a_{i}\right\} \hat{H}\left\{ a_{b}^{\dagger}a_{j}\right\} }{c}  =  \sum_{pq}\sum_{ijab}\delta C_{ai}^{*}\delta C_{bj}\braVket{p}{\hat{h}_{0}}{q}\braVket{c}{\left(\left\{ a_{i}^{\dagger}a_{a}\right\} \left\{ a_{p}^{\dagger}a_{q}\right\} \left\{ a_{b}^{\dagger}a_{j}\right\} \right)}{c}\\
		  +  \frac{1}{4}\sum_{pqrs}\sum_{ijab}\delta C_{ai}^{*}\delta C_{bj}\braVket{pq}{\hat{v}}{rs}\braVket{c}{\left(\left\{ a_{i}^{\dagger}a_{a}\right\} \left\{ a_{p}^{\dagger}a_{q}^{\dagger}a_{s}a_{r}\right\} \left\{ a_{b}^{\dagger}a_{j}\right\} \right)}{c}
	\end{multline*}
	giving us the result
	\[
	E_{0}\sum_{ai}\left|\delta C_{ai}\right|^{2}+\sum_{ai}\left|\delta C_{ai}\right|^{2}\left(\epsilon_{a}-\epsilon_{i}\right)-\sum_{ijab}\braVket{aj}{\hat{v}}{bi}\delta C_{ai}^{*}\delta C_{bj}
	\]

	Next we look at the third term of equation \ref{eq:SQ_HF_EnergyExpVal}
	\begin{multline*}
		\frac{1}{2!}\braVket{c}{\left(\hat{V}_{N}\left\{ a_{a}^{\dagger}a_{i}\right\} \left\{ a_{b}^{\dagger}a_{j}\right\} \right)}{c} = \\
		\frac{1}{8}\sum_{pqrs}\sum_{ijab}\delta C_{ai}\delta C_{bj}\braVket{pq}{\hat{v}}{rs}\braVket{c}{\left(\left\{ a_{p}^{\dagger}a_{q}^{\dagger}a_{s}a_{r}\right\} \left\{ a_{a}^{\dagger}a_{i}\right\} \left\{ a_{b}^{\dagger}a_{j}\right\} \right)}{c}\\
	 	=  \frac{1}{8}\sum_{pqrs}\sum_{ijab}\delta C_{ai}\delta C_{bj}\bra{\mbox{c}} \left(\left\{
		\contraction[2.5ex]{}{a}{_{p}^{\dagger}a_{q}^{\dagger}a_{s}a_{r}a_{a}^{\dagger}a_{i}a_{b}^{\dagger}}{a}
		\contraction[1.5ex]{a_{p}^{\dagger}}{a}{_{q}^{\dagger}a_{s}a_{r}a_{a}^{\dagger}}{a}
		\contraction[2ex]{a_{p}^{\dagger}a_{q}^{\dagger}}{a}{_{s}a_{r}a_{a}^{\dagger}a_{i}}{a}
		\contraction[1ex]{a_{p}^{\dagger}a_{q}^{\dagger}a_{s}}{a}{_{r}}{a}
	 	a_{p}^{\dagger}a_{q}^{\dagger}a_{s}a_{r}a_{a}^{\dagger}a_{i}a_{b}^{\dagger}a_{j}\right\}  +\left\{ 
	 	\contraction[2.5ex]{}{a}{_{p}^{\dagger}a_{q}^{\dagger}a_{s}a_{r}a_{a}^{\dagger}a_{i}a_{b}^{\dagger}}{a}
	 	\contraction[1.5ex]{a_{p}^{\dagger}}{a}{_{q}^{\dagger}a_{s}a_{r}a_{a}^{\dagger}}{a}
	 	\contraction[1ex]{a_{p}^{\dagger}a_{q}^{\dagger}}{a}{_{s}a_{r}}{a}
	 	\contraction[2ex]{a_{p}^{\dagger}a_{q}^{\dagger}a_{s}}{a}{_{r}a_{a}^{\dagger}a_{i}}{a}
	 	a_{p}^{\dagger}a_{q}^{\dagger}a_{s}a_{r}a_{a}^{\dagger}a_{i}a_{b}^{\dagger}a_{j}\right\} \right. \\
	 	\left. +\left\{ 
	 	\contraction[1.5ex]{}{a}{_{p}^{\dagger}a_{q}^{\dagger}a_{s}a_{r}a_{a}^{\dagger}}{a}
	 	\contraction[2.5ex]{a_{p}^{\dagger}}{a}{_{q}^{\dagger}a_{s}a_{r}a_{a}^{\dagger}a_{i}a_{b}^{\dagger}}{a}
	 	\contraction[2ex]{a_{p}^{\dagger}a_{q}^{\dagger}}{a}{_{s}a_{r}a_{a}^{\dagger}a_{i}}{a}
	 	\contraction[1ex]{a_{p}^{\dagger}a_{q}^{\dagger}a_{s}}{a}{_{r}}{a}
	 	a_{p}^{\dagger}a_{q}^{\dagger}a_{s}a_{r}a_{a}^{\dagger}a_{i}a_{b}^{\dagger}a_{j}\right\} +\left\{ 
	 	\contraction[1.5ex]{}{a}{_{p}^{\dagger}a_{q}^{\dagger}a_{s}a_{r}a_{a}^{\dagger}}{a}
	 	\contraction[2.5ex]{a_{p}^{\dagger}}{a}{_{q}^{\dagger}a_{s}a_{r}a_{a}^{\dagger}a_{i}a_{b}^{\dagger}}{a}
	 	\contraction[1ex]{a_{p}^{\dagger}a_{q}^{\dagger}}{a}{_{s}a_{r}}{a}
	 	\contraction[2ex]{a_{p}^{\dagger}a_{q}^{\dagger}a_{s}}{a}{_{r}a_{a}^{\dagger}a_{i}}{a}
	 	a_{p}^{\dagger}a_{q}^{\dagger}a_{s}a_{r}a_{a}^{\dagger}a_{i}a_{b}^{\dagger}a_{j}\right\} \right)\ket{c}
	\end{multline*}
	carrying out the contractions this results in 
	\[
		\frac{1}{2}\sum_{ijab}\braVket{ij}{\hat{v}}{ab}\delta C_{ai}\delta C_{bj}
	\]

	From the hermiticity of $\hat{H}$ and $\hat{V}$ we have the relation
	\[
		\braVket{a}{\hat{A}}{b}=\left( \braVket{b}{\hat{A}^{\dagger}}{a} \right)^{*}
	\]
	Using this, we  we can write the last term of equation \ref{eq:SQ_HF_EnergyExpVal} as
	\[
		\frac{1}{2!}\braVket{c}{\left(\left\{ a_{j}^{\dagger}a_{b}\right\} \left\{ a_{i}^{\dagger}a_{a}\right\} \hat{V}_{N}\right)}{c} = \frac{1}{2!}\braVket{c}{\left(\hat{V}_{N}\left\{ a_{a}^{\dagger}a_{i}\right\} \left\{ a_{b}^{\dagger}a_{j}\right\} \right)^{\dagger}}{c}^{*}
	\]
	which boils down to simply
	\begin{eqnarray*}
		\frac{1}{2!}\braVket{c}{\left(\hat{V}_{N}\left\{ a_{a}^{\dagger}a_{i}\right\} \left\{ a_{b}^{\dagger}a_{j}\right\} \right)^{\dagger}}{c}^{*} & = & \frac{1}{2}\sum_{ijab}\left(\braVket{ij}{\hat{v}}{ab}\right)^{*}\delta C_{ai}^{*}\delta C_{bj}^{*}\\
		& = & \frac{1}{2}\sum_{ijab}\left(\braVket{ab}{\hat{v}}{ij}\right)^{*}\delta C_{ai}^{*}\delta C_{bj}^{*}
	\end{eqnarray*}

	Now we define two anti-symmetrized matrix elements
	\begin{eqnarray*}
		A_{ai,\,bj} & = & -\braVket{aj}{\hat{v}}{bi}\\
		B_{ai,\,bj} & = & \braVket{ab}{\hat{v}}{ij}
	\end{eqnarray*}
	which lets us write the energy as
	\begin{multline*}
		\braVket{c'}{\hat{H}}{c'} = \left(1 + \sum_{ai}\left|\delta C_{ai}\right|^{2}\right)\braVket{c}{H}{c}+
		\sum_{ai}\left|\delta C_{ai}\right|^{2}\left(\epsilon_{a}^{HF}-\epsilon_{i}^{HF}\right)+\\
		\sum_{ijab}A_{ai,\,bj}\delta C_{ai}^{*}\delta C_{bj}+\frac{1}{2}\sum_{ijab}B_{ai,\,bj}^{*}\delta C_{ai}\delta C_{bj} + \frac{1}{2}\sum_{ijab}B_{ai,\,bj}\delta C_{ai}^{*}\delta C_{bj}^{*} + O\left( \delta C_{ai}^{3} \right)
	\end{multline*}
	This we can rewrite as 
	\[
		\braVket{c'}{\hat{H}}{c'}=\left(1 + \sum_{ai}\left|\delta C_{ai}\right|^{2}\right)\braVket{c}{H}{c} + \Delta E + O\left( \delta C_{ai}^{3} \right)
	\]
	Now, skipping higher order terms, we have
	\[
		\frac{\braVket{c'}{\hat{H}}{c'}}{\left\langle c' \right. |  \left. c' \right\rangle}
		= E_{0} + \frac{\Delta E}{1+\sum_{ai}\left | \delta C_{ai} \right |^{2}}
	\]

	To go further with the analyzation of the Hartree-Fock equations, we can write $\Delta E$ as
	\[
		\Delta E = \frac{1}{2}\braVket{\chi}{\hat{M}}{\chi}
	\]
	We have now introduced the vectors
	\begin{equation} \label{eq:chiVectors}
		\chi = \left [\delta C \quad \delta C^{*} \right]^{T}
	\end{equation}
	and the matrix
	\[
		\hat{M}=\left(\begin{array}{cc}
		\Delta+A & B\\
		B^{*} & \Delta+A^{*}
		\end{array}\right)
	\]
	where $\Delta_{ai\,bj}=(\epsilon_{a}-\epsilon{i})\delta_{ab}\delta{ij}$. This makes us able to write a condition for $\Delta E$,
	\[
		\Delta E = \frac{1}{2}\braVket{\chi}{\hat{M}}{\chi}\geq 0
	\]
	for a vector $\chi$ as defined by \ref{eq:chiVectors}. In other words, all eigenvalues of the matrix $\hat{M}$ must be larger or equal to zero. A condition that arises from this, for all matrix elements, is
	\[
		\left(\epsilon_{a} - \epsilon{i}\right)\delta_{ab}\delta_{ij}+A_{ai,\,bj}\geq 0
	\]
	which can be used as a test for the stability of the Hartree-Fock equation.

