%Final conclusions and perspectives.

%% what was the aim in the thesis, what has been done
In this thesis the main aim has been to develop a variational Monte Carlo
solver. Furthermore, emphasis has been put on making the solver as
general as possible, thus making it possible to study a wide variety
of systems. To achieve this the solver was written in C++ using
object-oriented programming. To ease the handling of vectors and
matrices being handled by the solver it utilizes the linear algebra
library Armadillo \cite{sanderson2010armadillo}, which provides
efficient methods for this.

As presented in section \ref{sec:hydrogenic_wavefunctions}, modelling
atoms is fairly simple and only a handfull of orbitals are used to
create the trialfunctions. With quantum dots however, harmonic
oscillator orbitals are used, and with the aim of using a great number
of particles in the calculations, in both two and three dimensions, a
way to automate the creation of the orbitals was needed. This was done
using the python library SymPy, with which the orbitals could easily
be created, their gradient and laplacian functions could be calculated
automatically, and source files used by the variational Monte Carlo
solver could seamlessly be generated.

A way to handle the immense ammounts of data produced by the solver
program was also needed. For this purpouse a script was created, which
converted the datafiles with the size of noumerous gigabytes to
something visualizable and with a more managable size.

%% put what has been done in perspective with previous works
Over the years many Master projects have had focus on solving quantum
mechanical many-body systems, using different methods, like the
Multi-Configuration time-dependent Hartree-Fock method \cite{skattum13}, post Hartree-Fock
methods liek coupled cluster theory \cite{hirthCC}, full configuration interaction theory  \cite{olsenFCI}, and quantum Monte Carlo
methods \cite{hogbergetDMC}. The focus in this thesis has thus not
been something entirely new, but nonetheless gave interesting insight
to quantum mechanical many-body systems. Creating a flexible solver
allowed the study of many systems: atoms with a size up to neon,
consisting of 10 particles, molecules with a size up to diberyllium,
consisting of 12 particles, and quantum dots with even greater size,
in both two and three dimensions.

By successfully implementing and studying a wide variety of quantum
mechanical systems the generality and flexibility of the VMC solver is
demonstrated. This shows that the solver that has been developed has
reached the goal which was set, by performing calculations on atoms,
molecules, and quantum dots.

%% comments on the main findings and interpretations
The variational Monte Carlo method gives fairly accurate results, both
for atoms and molecules, and also for quantum dots. With atoms it is
demonstrated how the so-called Jastrow factor efficiently deals with
the correlation factors, thus yielding energies comparable with those
computed using more exact methods. In the case of helium the relative
error is less than 0.5 percent compared to the optimized theoretical 
reference energies. For neon the relative error is just under 1
percent, and for the largest molecule used, the relative error is
about 6.5 percent, which is fairly good. Due to limited resources the
larger atoms and molecules had to be run with fewer Monte Carlo
cycles, so it is probable that the results could have been
better. There is however another limitation, which is the simple trial
function used. But considering this with the reduced number of Monte
Carlo cycles, the results are very good.

For all of the atoms the radial distribution of the wavefunctions
showed traces of the hydrogenic wavefunctions they were constructed
from. As the binding energy of the atoms during the VMC computations
are close to the experimental values, it is likely that the atomic
structure of the real atoms resemble the atomic structures of the VMC
simulations.

Results for quantum dots are for a comparable number of particles even
more accurate than for atoms and molecules. For the two particle case
the relative error is less than 0.04 percent and less than 0.2 percent
for two and three dimensions, respectively. For the quantum dots
consisting of the largest number of electrons, 30, the relative error
is also low, at about 1.5 percent. Unfortunately the number of quantum
dot systems had to be limited due to insufficient time. Despite this
the characteristics of the quantum dot system could still be
demonstrated.

In comparing the radial distribution of quantum dots in two and three
dimensions which have the same number of closed shells the similarity
is striking, as long as the frequency is high. For very low
frequencies there is a break in symmetry, where the two dimensional
case shows electrons that become strongly localized , while the three dimensional case does not exhibit the same degree of localization. 

%% comments on the method (pros and cons, improvements)
From the low error in the ground state energies it is clear that the
results are very good considering the amount of resources that went
into calculating them. Many of the calculations for a smaller number
of particles could be run on a single computer in a handful of
hours. For larger systems however a cluster computer is
used. Considering that a relatively simple and very well tested method
was used to obtain these results it is very easy to see why the VMC method is so
popular for solving quite complicated problems like calculating the
ground state energies of non-trivial quantum mechanical systems.

From the results presented in this thesis it is clear that the VMC
method computes fairly accurate results by utilizing the Jastrow
factor in computing correlations. There is however a tradeoff in that
the trial function used is a limiting factor on the accuracy. Choosing
better trialfunctions would thus result in more accurate results,
especially for larger molecules.

%% comments on the efficiency of the method
Because the variational Monte Carlo method uses a statistical approach
to the quantum mechanical many-body problem it is not as efficient as
for example the Hartree-Fock method or the Density functional theory
method \cite{kohn1965self}. It does however take into account
correlations, unlike the Hartree-Fock method. Other, more efficient
methods exist, like the diffusion Monte Carlo method.

\subsubsection{Perspectives for future work}
%% perspectives for future work
Looking ahead, to improve the Variational Monte Carlo program for
atoms, more analytical functions for the Slater Determinant and for
the derivatives of the GTO functions could be calculated. These parts
are used a lot during the program and switching from numerical to
analytical solution would speed things up. Further, as the Gaussian
Type Orbitals gave slightly dissapointing energies, we could implement
another, larger basis set than the 3-21G basis, such as the 6-311G
basis. This would give the GTO functions a shape that more closely
resembles Slater Type Orbitals and therefore give more accurate
energies.

%\todo{more future work}
The solver could also be made more efficient, and given more time,
systems of greater size could be studied. A possibility is to study a
higher scale, that is studying molecules. The break in symmetry of
quantum dots in two and three dimensions could also be studied in
greater detail. Our code is also flexible enough to accomodate double-well 
quantum dots, of great interest for computing computing. The implementation of a
diffusion Monte Carlo solver is also something which can be added to the code, providing
thereby almost exact ground state energies. 
