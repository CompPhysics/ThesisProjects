\documentstyle[a4wide]{article}
\newcommand{\OP}[1]{{\bf\widehat{#1}}}

\newcommand{\be}{\begin{equation}}

\newcommand{\ee}{\end{equation}}

\begin{document}

\pagestyle{plain}

\section*{Thesis title: Monte-Carlo simulations of quantum dots}

{\bf The aim of this thesis is to study numerically systems consisting of several
interacting electrons in two dimensions and three dimensions}, confined to small regions
between layers of semiconductors. 
These electron systems
are dubbed quantum dots in the literature. 

In this thesis,
the study of such a system of two-dimensional and three-dimensional electrons 
entails more specifically the use of many-body techniques through the 
development of variational Monte Carlo (VMC) to solve Schr\"odinger's equation, in order to obtain various expectation values of interest, such as the energy
of the ground state, magnetization etc. 

In the next section we give a 
brief description of the physics behind quantum dots and their
potential for constructing quantum gates. Thereafter, 
we sketch the ideas behind the VMC approach to be used.



\subsection*{Introduction to quantum dots}

Quantum computing has attracted much interest 
recently as it opens up the possibility of outperforming 
classical computation through new and more powerful quantum algorithms
such as the ones discovered by Shor and by Grover. 
There is now a growing list of quantum tasks 
such as cryptography, error correcting schemes, quantum
teleportation, etc. that have indicated even more the desirability 
of experimental implementations of quantum computing. 
In a quantum computer each quantum bit (qubit) is allowed
to be in any state of a quantum two-level system. 
All quantum algorithms can be implemented by 
concatenating one- and two-qubit gates. There is a growing number of proposed
physical implementations of qubits and quantum gates. 
A few examples are: Trapped ions, cavity QED, nuclear spins, 
superconducting devices, and
our qubit proposal 
based on the spin of the electron in quantum-confined nanostructures. 

Coupled quantum dots provide a powerful source of 
deterministic entanglement between qubits of localized 
but also of delocalized electrons. E.g., with such quantum gates it is
possible to create a singlet state out of two electrons 
and subsequently separate (by electronic transport) 
the two electrons spatially with the spins of the two electrons still being
entangled--the prototype of an EPR pair. 
This opens up the possibility to study a new class 
of quantum phenomena in electronic nanostructures 
such as the entanglement and
non-locality of electronic EPR pairs, tests of Bell inequalities, 
quantum teleportation, and quantum cryptography 
which promises secure information transmission. 


Semiconductor quantum dots are structures where
charge carriers are confined in all three spatial dimensions, 
the dot size being of the order of the Fermi wavelength 
in the host material, typically between  10 nm and  1 $\mu$m.
The confinement is usually achieved by electrical gating of a 
two-dimensional electron gas (2DEG), 
possibly combined with etching techniques. Precise control of the
number of electrons in the conduction band of a quantum dot 
(starting from zero) has been achieved in GaAs heterostructures. 
The electronic spectrum of typical quantum dots
can vary strongly when an external magnetic field is applied, 
since the magnetic length corresponding to typical 
laboratory fields  is comparable to typical dot sizes.
In coupled quantum dots Coulomb blockade effects, 
tunneling between neighboring dots, and magnetization 
have been observed as well as the formation of a
delocalized single-particle state. 


Quantum mechanical studies of such many-body systems are also
interesting per se. 
This thesis deals with a numerical {\em ab initio}  solution of  
Schr\"odinger's equation for quantum dots, from few electrons to many.
A critical understanding of our ability 
to solve such many-body systems through Monte Carlo methods is one 
of the aims
of this thesis project. Presently, Monte Carlo methods are
the only ones which allow us to solve, in principle exactly, 
systems with many interacting particles. These techniques are briefly sketched 
in the next section.


\subsection*{Aims of this thesis with Variational Monte Carlo}
The variational quantum Monte Carlo (VMC) has been widely applied 
to studies of quantal systems. 
The recipe consists in choosing 
a trial wave function
$\psi_T({\bf R})$ which we assume to be as realistic as possible. 
The variable ${\bf R}$ stands for the spatial coordinates, in total 
$dN$ if we have $N$ particles present. The variable $d$ is the dimension
of the system. 
The trial wave function serves then as
a mean to define the quantal probability distribution 
\be
   P({\bf R})= \frac{\left|\psi_T({\bf R})\right|^2}{\int \left|\psi_T({\bf R})\right|^2d{\bf R}}.
\ee
This is our new probability distribution function  (PDF). 

The expectation value of the energy $E$
is given by
\be
   \langle E \rangle =
   \frac{\int d{\bf R}\Psi^{\ast}({\bf R})H({\bf R})\Psi({\bf R})}
        {\int d{\bf R}\Psi^{\ast}({\bf R})\Psi({\bf R})},
\ee
where $\Psi$ is the exact eigenfunction. Using our trial
wave function we define a new operator, 
the so-called  
local energy, 
\be
   E_L({\bf R})=\frac{1}{\psi_T({\bf R})}H\psi_T({\bf R}),
   \label{eq:locale1}
\ee
which, together with our trial PDF allows us to rewrite the 
expression for the energy as
\be
  \langle H \rangle =\int P({\bf R})E_L({\bf R}) d{\bf R}.
  \label{eq:vmc1}
\ee
This equation expresses the variational Monte Carlo approach.
For most hamiltonians, $H$ is a sum of kinetic energy, involving 
a second derivative, and a momentum independent potential. 
The contribution from the potential term is hence just the 
numerical value of the potential.

Using the Metropolis algorithm and the Monte Carlo 
evaluation of Eq.~(\ref{eq:vmc1}), a detailed algorithm is   
as follows
       \begin{itemize}
          \item Initialisation: Fix the number of Monte Carlo steps and 
                thermalization steps. Choose an initial ${\bf R}$ and
                variational parameters $\alpha$ and 
                calculate
                $\left|\psi_T^{\alpha}({\bf R})\right|^2$. 
                Define also the value 
                of the stepsize to be used when moving from one value of 
                ${\bf R}$ to a new one.
          \item Initialise the energy and the variance.
          \item Start the Monte Carlo calculation 
                \begin{enumerate}
                  \item Thermalise first.
                  \item Thereafter start your Monte carlo sampling.
                  \item Calculate  a trial position  ${\bf R}_p={\bf R}+r*step$
                        where $r$ is a random variable $r \in [0,1]$.
                  \item Use then the Metropolis algorithm to accept
                        or reject this move by calculating the ratio
                        \[
                           w = P({\bf R}_p)/P({\bf R}).
                        \]
                        If $w \ge s$, where $s$ is a random number
                          $s \in [0,1]$, 
                          the new position is accepted, else we 
                          stay at the same place.
                  \item If the step is accepted, then we set 
                        ${\bf R}={\bf R}_p$. 
                  \item Update the local energy and the variance.
                 \end{enumerate}
          \item When the Monte Carlo sampling is finished, 
we calculate the mean energy and the standard deviation.
      \end{itemize}



The aims of this thesis are as follows

\begin{itemize}
\item Implement importance sampling in the VMC codes and parallelize the codes.
\item Study the three-dimensional quantum dots  with broken spherical symmetry.
Compare these results with those obtained with a pure two-dimensional quantum dot in order to
test the validity of the two-dimensional approximation.
\item Provide benchmarks and input for Hartree-Fock (HF) calculations and density-functional (DFT) calculations  of quantum dots
(the DFT and HF calculations are done by another Master of Science student, Patrick Merlot).
\item Perform VMC calculations with double oscillator wells.
\end{itemize}



The thesis is expected to be finished towards the end  of the fall 
semester of 2008.


\end{document}
