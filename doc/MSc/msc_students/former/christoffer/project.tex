\documentstyle[a4wide,11pt]{article}

\begin{document}

\pagestyle{plain}


\section*{Thesis project for Christoffer Hirth: Studies of quantum dots. {\em Ab initio} coupled-cluster analysis using OpenCL and GPU programming}

The aim of this thesis is to develop  a code for doing {\em ab initio} calculations of quantum-mechanical many-particle systems using coupled cluster theory.
The code will be 
written in OpenCL, in order to test the potential
of Graphical processing units (GPUs) for systems with large memory requirements.

The thesis will start with existing coupled-cluster  
codes for fermions,
tailored to run on large supercomputing cluster with thousands of cores.

These codes will then be written in OpenCL and tested on different GPUs,
in order to see if a considerable speedup can be gained compared with our exisiting C++ and Fortran codes.
Such a speedup, if at least an order of magnitude is gained compared 
with existing in C++/Fortran codes that run on present  supercomputers, will have trememdous consenquences for ab initio studies of quantum mechanical systems.  The prices of the GPUs are of the order of few thousands of NOK, compared
with large supercomputers which can cost several millions of NOK. 

The Physics aims of this thesis is to study the structure of quantum dots using
Coupled cluster theory, combining results
from Hartree-Fock calculations in order to achieve a as good as possible
starting point for the coupled-cluster calculations. 
The thesis will explore various Coupled cluster
approaches and use these to define the best possible density functional.
Semiconductor quantum dots are structures where
charge carriers are confined in all three spatial dimensions, 
the dot size being of the order of the Fermi wavelength 
in the host material, typically between  10 nm and  1 $\mu$m.
The confinement is usually achieved by electrical gating of a 
two-dimensional electron gas (2DEG), 
possibly combined with etching techniques. Precise control of the
number of electrons in the conduction band of a quantum dot 
(starting from zero) has been achieved in GaAs heterostructures. 
The electronic spectrum of typical quantum dots
can vary strongly when an external magnetic field is applied, 
since the magnetic length corresponding to typical 
laboratory fields  is comparable to typical dot sizes.
In coupled quantum dots Coulomb blockade effects, 
tunneling between neighboring dots, and magnetization 
have been observed as well as the formation of a
delocalized single-particle state. 

This thesis 
entails the development of a coupled-cluster code tailored to GPU programming, including if time allows so-called triples corrections, based on our existing codes in order
programs to solve Schr\"odingers equation
and obtain various expectation values of interest, such as the energy
of the ground state and excited states. 
Being an {\em ab initio} method, these calculations will provide a as best as possible determination of the 
ground state wave function. This wave function will in turn be used to define the quantum
mechanical density.  The density will be used to construct a density functional for quantum dots
using the adiabatic-connection method as described by Teale {\em et al} in J.~Chem.~Phys.~{\bf 130},
104111 (2009).  The results will be compared with existing density functional for quantum dots.


\section*{Progress plan and milestones}
The aims and progress plan of this thesis are as follows
\begin{itemize}
\item Fall 2011: Develop a coupled-cluster code for fermions in OpenCL
Parallel with this, one needs also a corresponding efficient 
C++ code which can run
in parallel on existing supercomputers.  Parts of the thesis will also be devoted to run such calculations.
\item If possible, add the possibility to include triples correlations.
\item Spring 2012: Extensive benchmarks of codes, production runs 
and writeup of master
thesis.

\end{itemize}
 
If successful, this thesis will end up in a scientific publication in a high-level Journal.
The thesis is expected to be handed in June 1 2012.


\end{document}


 






\end{document}



