\chapter{Conclusions}
In this master's thesis we aimed towards a more flexible and faster coupled-cluster code.
The main goal was to study the use of alternative methods for matrix-multiplication, especially accelerated by the use of graphics cards, GPUs.
Initially a previous C++ code~(\cite{marte,ymwang}) was considered to be extended, but the decision fell on a complete redesign and rewrite of the program.
This was partially to fit a different coding style, but also to redesign parts that were bottlenecks.
In fact, most of the speedup acquired here was due to a simplification of the problem, rewriting all large matrices on a block-diagonal form, not by the accelerated multiplications.

Over now three master's projects, the boundary has been pushed from 20 particles in 110 basis functions~\cite{mplohne}, first increasing the basis size to up to 420 functions~\cite{marte}, and now up to 56 particles in 930 basis functions.
The current limit is now the memory requirements of up to $100$ gigabytes and not the long run times anymore.
A consequence of the current state is that the next area of focus should no longer be aimed at a pure speed up.
Further work on the program could include triples, leading to either perturbative triples correction, CCSD(T), or full inclusion of $\hat{T}_3$, CCSDT.
In order to move to larger systems it is also needed to either distribute matrix elements across different nodes, or calculate them on the fly, in order to reduce memory usage.


\paragraph{}
When it comes to results, we have managed to expand the range of frequencies, $\omega$, toward less bound systems, using a step-by-step approach where the initial guess is now based on a previous calculation.
Although this technique opens up a new area of convergence, it is not sufficient for larger systems.
We are for instance not able to break the barrier of $\omega = 1.0$ for 56 particles.
From the more physical point of view, it is of great interest to study more difficult systems such as a double well quantum dot, along the lines of Wang's studies~\cite{ymwang}.
In our opinion it should not be hard to extend the developed library to include such a system.
A reduction in the wall time used for simulations of a double dot would then most likely be seen as well, if the symmetries of the new Hamiltonian is successfully described in the `Basis' class.

Another interesting topic to investigate is simulating the time-dependent Schrödinger equation.
A time-dependent coupled-cluster framework is already tried to some extent.
See for example~\cite{kvaal:194109}.
Including a variable magnetic field in the calculations, if possible, one could simulate spin-flipping and other phenomena.
Such simulations are not done previously, and would probably require a rethinking of the approach.


\paragraph{}
We are well satisfied with the work done on implementing the coupled cluster equations.
To the best of our knowledge, coupled-cluster calculations using GPU programming have not been studied before.
Although only a speedup of 2-4 times was seen compared to a quad core for the matrix-multiplications itself, the impact on the full program was not dramatic for such a small system.
We predict that GPUs could be more effective if we could circumvent the memory transfer between the host and the GPU.
Perhaps could GPUs play a more important role if elements were calculated on the fly, and these calculations were parallelized.
Seen from a cost-effective perspective, the use of multiple GPUs is worth a try, as multiple GPUs can be attached to the same node.

Developing this library was time consuming as the project rapidly grew in size.
The current code base consists of more than 6000 lines of code along with 2000 lines of comments.
We therefore hope that sometime someone will continue along the path we have partly paved here, by creating creating a library that breaks the barriers we met.


\paragraph{}
\begin{itemize}
\item Efficiency gain for HFsys
\item Efficiency due to channel/conf
\item Thin vs. square matrices
\item Strassen not optimal (no or little gain)
\item Scaling is now better than previous program.
\item Mappings are too slow
\item Effective 24 shells conv $< 5\cdot 10^{-5}$.
\item Correlations dominate for many particles and low frequencies.
\item Beyond HF important for few particles and low frequencies.
\item Compares well to DMC.
\item FCI excitations vs. shells?
\end{itemize}


