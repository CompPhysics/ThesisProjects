\documentstyle[a4wide]{article}
\newcommand{\OP}[1]{{\bf\widehat{#1}}}

\newcommand{\be}{\begin{equation}}

\newcommand{\ee}{\end{equation}}

\begin{document}

\pagestyle{plain}

\section*{Thesis title: Monte-Carlo simulations of quantum dots}

{\bf The aim of this thesis is to study numerically systems consisting of several
interacting electrons in two dimensions}, confined to small regions
between layers of semiconductors. 
These electron systems
are dubbed quantum dots in the literature. 

In this thesis,
the study of such a system of two-dimensional electrons 
entails more specifically the use of many-body techniques through the 
development of variational Monte Carlo (VMC)
and diffusion Monte Carlo  (DMC) programs to solve Schr\"odinger's equation, in order to obtain various expectation values of interest, such as the energy
of the ground state, magnetization etc. 
The DMC approach allows, in principle, for a numerically
exact solution of Schr\"odinger's equation. However, it needs a reasonable
starting point. It is here where a variational Monte Carlo calculation
of the same system provides a variationally optimal trial wave function
of a many-body system.

In the next section we give a 
brief description of the physics behind quantum dots and their
potential for constructing quantum gates. Thereafter, 
we sketch the ideas behind both the VMC and DMC approaches to be used.

A progress plan for this thesis project is given at the end.


\subsection*{Introduction to quantum dots}

Quantum computing has attracted much interest 
recently as it opens up the possibility of outperforming 
classical computation through new and more powerful quantum algorithms
such as the ones discovered by Shor and by Grover. 
There is now a growing list of quantum tasks 
such as cryptography, error correcting schemes, quantum
teleportation, etc. that have indicated even more the desirability 
of experimental implementations of quantum computing. 
In a quantum computer each quantum bit (qubit) is allowed
to be in any state of a quantum two-level system. 
All quantum algorithms can be implemented by 
concatenating one- and two-qubit gates. There is a growing number of proposed
physical implementations of qubits and quantum gates. 
A few examples are: Trapped ions, cavity QED, nuclear spins, 
superconducting devices, and
our qubit proposal 
based on the spin of the electron in quantum-confined nanostructures. 

Coupled quantum dots provide a powerful source of 
deterministic entanglement between qubits of localized 
but also of delocalized electrons. E.g., with such quantum gates it is
possible to create a singlet state out of two electrons 
and subsequently separate (by electronic transport) 
the two electrons spatially with the spins of the two electrons still being
entangled--the prototype of an EPR pair. 
This opens up the possibility to study a new class 
of quantum phenomena in electronic nanostructures 
such as the entanglement and
non-locality of electronic EPR pairs, tests of Bell inequalities, 
quantum teleportation, and quantum cryptography 
which promises secure information transmission. 


Semiconductor quantum dots are structures where
charge carriers are confined in all three spatial dimensions, 
the dot size being of the order of the Fermi wavelength 
in the host material, typically between  10 nm and  1 $\mu$m.
The confinement is usually achieved by electrical gating of a 
two-dimensional electron gas (2DEG), 
possibly combined with etching techniques. Precise control of the
number of electrons in the conduction band of a quantum dot 
(starting from zero) has been achieved in GaAs heterostructures. 
The electronic spectrum of typical quantum dots
can vary strongly when an external magnetic field is applied, 
since the magnetic length corresponding to typical 
laboratory fields  is comparable to typical dot sizes.
In coupled quantum dots Coulomb blockade effects, 
tunneling between neighboring dots, and magnetization 
have been observed as well as the formation of a
delocalized single-particle state. 


Quantum mechanical studies of such many-body systems are also
interesting per se. 
This thesis deals with a numerical {\em ab initio}  solution of  
Schr\"odinger's equation for quantum dots, from few electrons to many.
A critical understanding of our ability 
to solve such many-body systems through Monte Carlo methods is one 
of the aims
of this thesis project. Presently, Monte Carlo methods are
the only ones which allow us to solve, in principle exactly, 
systems with many interacting particles. These techniques are briefly sketched 
in the next two sections.


\subsection*{Variational Monte Carlo}
The variational quantum Monte Carlo (VMC) has been widely applied 
to studies of quantal systems. 
The recipe consists in choosing 
a trial wave function
$\psi_T({\bf R})$ which we assume to be as realistic as possible. 
The variable ${\bf R}$ stands for the spatial coordinates, in total 
$dN$ if we have $N$ particles present. The variable $d$ is the dimension
of the system. 
The trial wave function serves then as
a mean to define the quantal probability distribution 
\be
   P({\bf R})= \frac{\left|\psi_T({\bf R})\right|^2}{\int \left|\psi_T({\bf R})\right|^2d{\bf R}}.
\ee
This is our new probability distribution function  (PDF). 

The expectation value of the energy $E$
is given by
\be
   \langle E \rangle =
   \frac{\int d{\bf R}\Psi^{\ast}({\bf R})H({\bf R})\Psi({\bf R})}
        {\int d{\bf R}\Psi^{\ast}({\bf R})\Psi({\bf R})},
\ee
where $\Psi$ is the exact eigenfunction. Using our trial
wave function we define a new operator, 
the so-called  
local energy, 
\be
   E_L({\bf R})=\frac{1}{\psi_T({\bf R})}H\psi_T({\bf R}),
   \label{eq:locale1}
\ee
which, together with our trial PDF allows us to rewrite the 
expression for the energy as
\be
  \langle H \rangle =\int P({\bf R})E_L({\bf R}) d{\bf R}.
  \label{eq:vmc1}
\ee
This equation expresses the variational Monte Carlo approach.
For most hamiltonians, $H$ is a sum of kinetic energy, involving 
a second derivative, and a momentum independent potential. 
The contribution from the potential term is hence just the 
numerical value of the potential.

Using the Metropolis algorithm and the Monte Carlo 
evaluation of Eq.~(\ref{eq:vmc1}), a detailed algorithm is   
as follows
       \begin{itemize}
          \item Initialisation: Fix the number of Monte Carlo steps and 
                thermalization steps. Choose an initial ${\bf R}$ and
                variational parameters $\alpha$ and 
                calculate
                $\left|\psi_T^{\alpha}({\bf R})\right|^2$. 
                Define also the value 
                of the stepsize to be used when moving from one value of 
                ${\bf R}$ to a new one.
          \item Initialise the energy and the variance.
          \item Start the Monte Carlo calculation 
                \begin{enumerate}
                  \item Thermalise first.
                  \item Thereafter start your Monte carlo sampling.
                  \item Calculate  a trial position  ${\bf R}_p={\bf R}+r*step$
                        where $r$ is a random variable $r \in [0,1]$.
                  \item Use then the Metropolis algorithm to accept
                        or reject this move by calculating the ratio
                        \[
                           w = P({\bf R}_p)/P({\bf R}).
                        \]
                        If $w \ge s$, where $s$ is a random number
                          $s \in [0,1]$, 
                          the new position is accepted, else we 
                          stay at the same place.
                  \item If the step is accepted, then we set 
                        ${\bf R}={\bf R}_p$. 
                  \item Update the local energy and the variance.
                 \end{enumerate}
          \item When the Monte Carlo sampling is finished, 
we calculate the mean energy and the standard deviation.
      \end{itemize}

\subsection*{Diffusion Monte Carlo}
The DMC method is based on rewriting the 
Schr\"odinger equation in imaginary time, by defining
$\tau=it$. The imaginary time Schr\"odinger equation is then
\be
   \frac{\partial \psi}{\partial \tau}=-\OP{H}\psi,
\ee
where we have omitted the dependence on $\tau$ and the spatial variables
in $\psi$.
The wave function $\psi$  is again expanded in eigenstates of the Hamiltonian 
\be
    \psi = \sum_i^{\infty}c_i\phi_i,
    \label{eq:wexpansion_dmc}
\ee 
where 
\be 
   \OP{H}\phi_i=\epsilon_i\phi_i, 
\ee 
$\epsilon_i$ being an eigenstate of $\OP{H}$. 
A formal solution of the imaginary time Schr\"odinger equation is  
\be 
   \psi(\tau_1+\delta\tau)=e^{-\OP{H}\delta\tau}\psi(\tau_1) 
\ee 
where the state  $\psi(\tau_1)$ 
evolves from an imaginary time $\tau_1$ to a later time $\tau_1+\delta$.  
If the initial state $\psi(\tau_1)$ is 
expanded in energy ordered eigenstates,
following Eq.~(\ref{eq:wexpansion_dmc}), then we obtain
\be
    \psi(\delta\tau)=\sum_i^{\infty}c_ie^{-\epsilon_i\delta\tau}\phi_i.
\ee

Hence any initial state, $\psi$, 
that is not orthogonal to the ground state $\phi_0$ 
will evolve to the ground state in the long time limit, that is
\be
  \lim_{\tau\rightarrow\infty}\psi(\delta\tau)=c_0e^{-\epsilon_0\tau}\phi_0.
\ee
This derivation shares many formal similarities with that given 
for the variational principle discussed in the previous
section. However in the DMC method the imaginary
time evolution results in excited states decaying exponentially fast, 
whereas in the VMC method any excited state contributions 
remain and contribute
to the VMC energy. 

The DMC method is a realisation of the above derivation in position space.
Including the spatial variables as well, the above equation reads
\be
  \lim_{\tau\rightarrow\infty}\psi({\bf R}, \delta\tau)=
   c_0e^{-\epsilon_0\tau}\phi_0({\bf R}).
   \label{eq:wexpansion_dmc2}
\ee

By introducing a constant offset to the energy, 
$E_T=\epsilon_0$, 
the long-time limit of Eq.~(\ref{eq:wexpansion_dmc2}) 
can be kept finite. If the Hamiltonian is separated into the
kinetic energy and potential terms, the imaginary time Schr\"odinger equation,
takes on a form similar to a diffusion equation, namely
\be
   -\frac{\partial \psi({\bf R}, \tau)}{\partial \tau}=
    \left[\sum_i^{N}-\frac{1}{2}\nabla^2_i\psi({\bf R}, \tau)\right]
    +(V({\bf R})-E_T)\psi({\bf R}, \tau).
    \label{eq:dmcequation1}
\ee
This equation is a diffusion equation where  the wave function $\psi$
may be interpreted as the density of diffusing particles (or ``walkers''), 
and the term  $V({\bf R})-E_T$ 
is a rate term describing a potential-dependent increase 
or decrease in the particle density. 
The above equation may be transformed into a form suitable for Monte Carlo methods, but this leads to a very inefficient algorithm. The
potential $V({\bf R})$  
is unbounded in coulombic systems and hence the rate term
$V({\bf R})-E_T$  can diverge. 
Large fluctuations in the particle density then
result and give impractically large statistical errors. 

These fluctuations may be substantially reduced by the incorporation 
of importance sampling in the
algorithm. 
Importance sampling is essential for DMC methods, 
if the simulation is to be efficient. 
A trial or guiding wave function $\psi_T({\bf R})$, which closely
approximates the ground state wave function is introduced.
This is where typically the VMC result enters.
A new distribution is defined as 
\be
   f({\bf R}, \tau)=\psi_T({\bf R})\psi({\bf R}, \tau),
\ee
which is also a solution of the Schr\"odinger equation when  
$\psi({\bf R}, \tau)$ 
is a solution. 
Eq.~(\ref{eq:dmcequation1}) consequently modified to
\be
   \frac{\partial f({\bf R}, \tau)}{\partial \tau}=
    \frac{1}{2}\nabla\left[\nabla -F({\bf R})\right]f({\bf R}, \tau)
    +(E_L({\bf R})-E_T)f({\bf R}, \tau).
    \label{eq:dmcequation2}
\ee
In this equation we have introduced the so-called force-term $F$,
given by
\be
   F({\bf R})=\frac{2\nabla \psi_T({\bf R})}{ \psi_T({\bf R})},
\ee
and is commonly referred to as the ``quantum force''. 
The local energy $E_L$ is defined as previously
\be
    E_L{\bf R})=-\frac{1}{\psi_T({\bf R})}
                \frac{\nabla^2 \psi_T({\bf R})}{2}+V({\bf R})\psi_T({\bf R}),
\ee
and is computed, as in the VMC method, with respect to the trial wave function.

We can give the following interpretation to Eq.~(\ref{eq:dmcequation2}).
The right hand side of the importance sampled DMC equation
consists, from left to right, of diffusion, drift and rate terms. The
problematic potential dependent rate term of the non-importance 
sampled method is replaced by a term dependent on the difference 
between the local
energy of the guiding wave function and the trial energy. 
The trial energy is initially chosen to be the VMC energy of 
the trial  wave function, and is
updated as the simulation progresses. Use of an optimised 
trial function minimises the difference between the local 
and trial energies, and hence
minimises fluctuations in the distribution $f$ . 
A wave function optimised using VMC is ideal for this purpose, 
and in practice VMC provides the best
method for obtaining wave functions that accurately 
approximate ground state wave functions locally. 


\section*{Progress plan}
The thesis is expected to be finished towards the end  of the spring
semester of 2003.
\begin{itemize}
\item Spring 2002: Exam in Fys 372 (Nuclear structure, 3vt).
      Write a code which solves the variational Monte-Carlo (VMC) problem
      for two interacting electrons in a harmonic oscillator potential
      in two dimensions. The role of an external magnetic field 
      is also to be studied. Possible connections to the quantum Hall
      effect.
      This code is then expanded in order to account for more electrons
      (tipycally 10-20 or more) which are localized in the quantum dot.  
\item Fall 2002:  Exam in Fys 303 (Relativistic quantum physics, 4vt).
                  Follow lectures in Fys 305 (many-body physics, 3vt).
      The next step is to write a Diffusion Monte Carlo (DMC) programme 
      for the 
      same system. 
     The DMC code receives as input the optimal 
     variational energy and wave function from the VMC calculation and solves
     in principle the Schr\"odinger equation exactly.
\item Spring 2003: Writeup of thesis, thesis exam and exam in Fys 305.

\end{itemize}

\end{document}
