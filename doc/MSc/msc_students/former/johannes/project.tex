\documentstyle[a4wide]{article}
\newcommand{\OP}[1]{{\bf\widehat{#1}}}

\newcommand{\be}{\begin{equation}}

\newcommand{\ee}{\end{equation}}
\newcommand{\bra}[1]{\left\langle #1 \right|}
\newcommand{\ket}[1]{\left| #1 \right\rangle}
\newcommand{\braket}[2]{\left\langle #1 \right| #2 \right\rangle}


\begin{document}

\pagestyle{plain}

\section*{Computational Physics thesis project for Johannes Rekkedal: Ab initio studies of infinite matter}



The aim of this thesis is to perform coupled-cluster calculations of infinite nuclear matter    
with the inclusion of so-called singles and doubles correlations to infinite order. 
Such studies have crucial consequences for our understanding of dense matter such as the interiors of stars
and neutron stars.  Proper calculations of the ground state of infinite matter in neutrons stars,
are crucial ingredients in studies of superfluidity in such stars and their cooling history

The summation to infinite order of important  correlations  remains an unsolved problem in nuclear many-body physics. 
By utilizing our recently developed coupled cluster codes for nuclei, where we can now perform ab initio
calculations with unrenomarlized nuclear forces for nuclei up to $^{100}$Sn, the aim of this thesis is to extend the application
of the codes to infinite matter.

The goal is to use
a representation of the nuclear interaction in a plane wave basis  given by two-body matrix elements 
in the laboratory system but coupled to final angular momenta (partial wave expansion). The technicalities  are given below.

These matrix elements will in turn be used as input to two many-body codes, the coupled-cluster code and the code for many-body perturbation 
to third order in the interaction. 

Since the amount of such two-body matrix elements will be huge, the code for the generating them has to be parallelized.
To develop such a code and interfacing it  with the existing many-body codes is the first task of this thesis.

With these codes working, the last part is to compute the equation of state for infinite nuclear matter  and neutron-star
matter in $\beta$-equilibrium. If successful, these calculations can most likely be published in Physical Review Letters.
With the addition of triples correlations, the work done at the master level can easily be extended to a PhD degree
with topics such as studies of superfluidity and neutrino emission in neutron stars.

\subsection*{Technicalities}
\begin{enumerate} 
\item 
The first task is to compute the matrix elements of the nucleon-nucleon interaction in the laboratory frame,
with two particles coupled to final angular momentum $J$, 
starting from the momentum-space version of the 
nucleon-nucleon (NN) interaction $V$ in the relative and center of mass system.
The NN interaction is normally provided with quantum numbers of the relative and center of mass frame.
In this reference frame, 
the NN interaction $V$ 
is defined in terms of various quantum numbers as follows
\begin{equation}
\left\langle klKL({\cal J})S T_z\right |
      V\left | k'l'KL({\cal J})S T_z \right\rangle,
\end{equation}
where the variables $k$, $k'$ and $l$, $l'$
denote respectively relative and angular momenta,
while
$K$ and $L$ are the quantum numbers of the center of mass
motion. ${\cal J}$, $S$ and $T_z$ represent the total angular
momentum in the relative and center of mass system, spin and isospin projections,
respectively. 

One can obtain the corresponding matrix elements in the laboratory system
through appropriate transformation coefficients \cite{balian69,wc72,kkr79}. This transformation proceeds through the definition
of a two-particle state in the laboratory system.
With these coefficients,
the expression for a two-body wave function in momentum space
using the laboratory coordinates can be written as
\begin{equation}
   \begin{array}{ll}
     &\\
     \left | (k_al_aj_at_{z_a})(k_bl_bj_bt_{z_b})JT_z\right \rangle =&
      {\displaystyle \sum_{lL\lambda S{\cal J}}}\int k^{2}dk\int K^{2}dK
      \left\{\begin{array}{ccc}
      l_a&l_b&\lambda\\\frac{1}{2}&\frac{1}{2}&S\\
      j_a&j_b&J\end{array}
      \right\}\\&\\
      &\times (-1)^{\lambda +{\cal J}-L-S}
      F\hat{{\cal J}}\hat{\lambda}^{2}
      \hat{j_{a}}\hat{j_{b}}\hat{S}
      \left\{\begin{array}{ccc}L&l&\lambda\\S&J&{\cal J}
      \end{array}\right\}\\&\\
      &\times \left\langle klKL| k_al_ak_bl_b\lambda\right\rangle
      \left | klKL({\cal J})SJT_z\right \rangle ,
   \end{array}
   \label{eq:relcm-lab}
\end{equation}
where $\left\langle klKL| k_al_ak_bl_b\lambda\right\rangle$
is the transformation coefficient (vector bracket) from the relative and center of mass system 
to the laboratory system  defined in Refs.~\cite{wc72,kkr79}.
The factor $F$ is defined as $F=(1-(-1)^{l+S+T_z})/\sqrt{2}$ if
we have identical particles only ($T_z=\pm 1$) and $F=1$ for 
different particles (protons and neutrons here, $T_z=0$). The other factors are $6j$ and $9j$ symbols.


The transformation coefficient $\left\langle klKL| k_al_ak_bl_b\right\rangle$ is 
given by \cite{balian69,wc72,kkr79}
\begin{equation}
\left\langle klKL| k_al_ak_bl_b\lambda\right\rangle=\frac{4\pi^2}{kKk_ak_b}\delta(w)\theta(1-x^2)A(x),
\label{eq:vectorbras}
\end{equation}
with
\begin{equation}
w= k^2+\frac{1}{4}K^2-\frac{1}{2}(k_a^2+k_b^2),
\end{equation}
\begin{equation}
x=(k_a^2-k^2-\frac{1}{4}K^2)/kK,
\end{equation}
and
\begin{equation}
A(x) = \frac{1}{2\lambda+1}\sum_{\mu}[Y^l(\hat{k})\times Y^L(\hat{K})]_{\mu}^{\lambda *}\times
[Y^{l_a}(\hat{k}_a)\times Y^{l_b}(\hat{k}_b)]_{\mu}^{\lambda}.
\end{equation}
The functions $Y^l$ are the standard spherical harmonics, $x$ is the cosine of the angle 
between $\vec{k}$ and $\vec{K}$ so that the step function takes input values from
0 to 1. In the codes, the coordinate system of Kuo {\em et al.}  \cite{kkr79} gives a practical algorithm.
 
To compute the NN interaction in the laboratory frame we need to perform two such transformation, one on the
bra-side and one on the ket-side, resulting in a final matrix element
\begin{equation}
   \left\langle (k_al_aj_at_{z_a})(k_bl_bj_bt_{z_b})JT_z\right |
    V\left | (k_cl_cj_ct_{z_c})(k_dl_dj_dt_{z_d})JT_Z \right\rangle,
\end{equation}
where the labels $a,b,c$ and $d$ represent plane waves
All matrix elements discussed here are assumed to be 
antisymmetrized (AS).

The computation of these matrix elements, with typically 20-30 mesh points in $k$-space and cutoff  $l\le 6-10$,
has to be parallelized.



\item  The next task is to interface the code for computing these matrix elements with the computation of
the equation of state (binding energy per nucleon) for infinite matter using many-body perturbation theory to
third-order.  

\item The final stage is to perform coupled-cluster calculations with these interactions 
for infinite nuclear matter  and neutron-star
matter in $\beta$-equilibrium.

\end{enumerate}

\subsection*{Local organization}

The activity will be based within the 
Computational Quantum Mechanics project at the Department of Physics and 
CMA. The main location is the Computational Physics group. 
The Computational Physics  group consists  
presently of two full professors (Hjorth-Jensen and Malthe-S\o renssen) and two professor Emeriti (Engeland and Osnes), 
one Adjunct Professor (Dean, Oak Ridge National Lab and CMA), four PhD
students (CMA(2), Dept ot Physics (2)) and fourteen Master students (Dept of Physics) with five new to come from fall 2008, totalling
19 master students, 4 PhD students and two permanent researchers.
More inofrmation at the webpage of the group at http://www.fys.uio.no/compphys.

The computational quantum mechanics part of the computational physics group
has extensive collaborations on ab initio quantum mechanics with Oak Ridge national laboratory (ORNL), Michigan State University
and Tokyo University.
\section*{Progress plan and milestones}
The thesis is expected to be finished June 1 2009.
\begin{itemize}
\item Fall 2008: Develop a parallel code for setting up the nucleon-nucleon
interaction in  a plane wave basis and perform many-perturbation theory to third
order in infinite matter.
A stay in november 2008 (duration 2-3 weeks) at Oak Ridge national Laboratory (Oak Ridge, Tennessee) is included.
Here Johannes Rekkedal will perform the coupled cluster calculation at the level of singles and doubles excitations.
\item Spring 2009:  
Conclusion of coupled cluster calculations and 
writeup of thesis and final thesis exam
\end{itemize}


\begin{thebibliography}{53}
\expandafter\ifx\csname natexlab\endcsname\relax\def\natexlab#1{#1}\fi
\expandafter\ifx\csname bibnamefont\endcsname\relax
  \def\bibnamefont#1{#1}\fi
\expandafter\ifx\csname bibfnamefont\endcsname\relax
  \def\bibfnamefont#1{#1}\fi
\expandafter\ifx\csname citenamefont\endcsname\relax
  \def\citenamefont#1{#1}\fi
\expandafter\ifx\csname url\endcsname\relax
  \def\url#1{\texttt{#1}}\fi
\expandafter\ifx\csname urlprefix\endcsname\relax\def\urlprefix{URL }\fi
\providecommand{\bibinfo}[2]{#2}
%\providecommand{\eprint}[2][]{\url{#2}}



\bibitem[{\citenamefont{Balian and Brezin}(1969)}]{balian69}
\bibinfo{author}{\bibfnamefont{R.}~\bibnamefont{Balian}} \bibnamefont{and}
  \bibinfo{author}{\bibfnamefont{E.}~\bibnamefont{Brezin}},
  \bibinfo{journal}{Nuovo Cim.} \textbf{\bibinfo{volume}{61}},
  \bibinfo{pages}{403} (\bibinfo{year}{1969}).

\bibitem[{\citenamefont{Wong and Clement}(1972)}]{wc72}
\bibinfo{author}{\bibfnamefont{C.~W.} \bibnamefont{Wong}} \bibnamefont{and}
  \bibinfo{author}{\bibfnamefont{D.~M.} \bibnamefont{Clement}},
  \bibinfo{journal}{Nucl. Phys. A} \textbf{\bibinfo{volume}{183}},
  \bibinfo{pages}{210} (\bibinfo{year}{1972}).

\bibitem[{\citenamefont{Kung et~al.}(1979)\citenamefont{Kung, Kuo, and
  Ratcliff}}]{kkr79}
\bibinfo{author}{\bibfnamefont{C.~L.} \bibnamefont{Kung}},
  \bibinfo{author}{\bibfnamefont{T.~T.~S.} \bibnamefont{Kuo}},
  \bibnamefont{and} \bibinfo{author}{\bibfnamefont{K.~F.}
  \bibnamefont{Ratcliff}}, \bibinfo{journal}{Phys. Rev. C}
  \textbf{\bibinfo{volume}{19}}, \bibinfo{pages}{1063} (\bibinfo{year}{1979}).



\end{thebibliography}

\end{document}
