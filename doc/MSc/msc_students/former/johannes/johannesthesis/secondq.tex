\chapter{The Second Quantization}
\label{chapsecondq}

Nuclear physics is about many-particle systems, the need for interparticle
potentials has to be accounted for when trying to describe 
 such systems. The interparticle potentials have to be implemented in the
many-particle Schr\"odinger equation.
A direct solution of the schr\"odinger equation in configuration space is impractical \cite{fetter}. 
It is necessary to resort to other techniques such as the second 
quantization.\\

The system studied in this text consists of nucleons which belong to the 
types of particles called fermions. Fermions are particles with half 
integer spin. 
A system of fermions is described by an antisymmetric wave function, these particles obeys the Pauli principle which states that two identical 
fermions can not occupy the same single particle state.\\

In most of the cases of interest, as is also the case in this text, the Hamiltonian 
takes the form 

\be
H=\sum_{k=1}^NT(x_k)+\frac{1}{2}\sum_{k \neq l =1}^N V(x_k,x_l).
\label{hamfirst}
\ee  

Where $T$ is the kinetic energy and V is the potential energy of interaction
between the particles, while $x_k$ denotes the coordinates of particle $k$.
The potential energy term represents the interaction between every pair of 
particles, counted once which account for the factor of $\frac{1}{2}$ 
\cite{fetter}.

\section{Creation and annihilation operators}

The interpretation of occupation of the antisymmetric many-body fermion
states allows to introduce the two operators $a^\dagger_\alpha$ and 
$a_\alpha$, which creates and annihilates a particle in the single particle 
state $\alpha$.

\be
\begin{split}
& a^\dagger_\alpha \ket{0}=\ket{\alpha}\\
& a_\alpha \ket{\alpha}=\ket{0}
\end{split}
\ee 

The algebra of these operators depends on whether the system under 
consideration is a system of bosons or a system of fermions. If it consists of bosons the operators obey the commutation relations

\be
\begin{split}
& [a_k,a^\dagger_{k'}]=\delta_{k,k'}\\
& [a_k,a_{k'}]=[a^\dagger_k,a^\dagger_{k'}]=0.
\end{split}
\ee

While in the fermion case they obey the anti commutation relations

\be
\begin{split}
& \{a_k,a^\dagger_{k'}\}=\delta_{k,k'}\\
& \{a_k,a_{k'}\}=\{a^\dagger_k,a^\dagger_{k'}\}=0
\end{split}
\ee

With these expressions for the commutations and anti commutations 
between the creation and annihilation operators the Hamiltonian can be 
reshaped to the form in Eq. \eqref{hamiltoniansec}.

\be
H=\sum_{ik} T_{ki}a^\dagger_ka_i + \frac{1}{2}\sum_{ijkl}V_{ijkl}a^\dagger_i
a^\dagger_ja_la_k.
\label{hamiltoniansec}
\ee

When the operators in the second quantization are non-relativistic and 
conserve the particle number, there should be an equal amount of creation
and destruction operators in the Hamiltonian. A second quantized one particle 
operator, an operator that acts on one particle a time is written as in 
Eq. \eqref{onesec}

\be
F=\sum_{\alpha,\beta}\bra{\alpha}f\ket{\beta}a^\dagger_\alpha a_\beta,
\label{onesec}
\ee 

and a two particle operator is written in the form as in Eq. \eqref{twosec}

\be
V=\frac{1}{2}\sum_{\alpha\beta\gamma\delta}\bra{\alpha\beta}v\ket{\gamma\delta}
a^\dagger_\alpha a^\dagger_\beta a_\delta a_\gamma.
\label{twosec}
\ee





\section{Wicks Theorem}
\label{wicksteorem}
A normal ordered second quantized operator is defined as an operator whose 
all annihilation operators stands to right of all the creation operators. 
It is in some manner easier to calculate when the annihilation operators are
placed to the right.
Wicks theorem describes a fast method to put the annihilation operators to
the right of the creation operators, by using the anti commutation rules
for these operators. Before introducing Wicks theorem some definitions should
be introduced like the normal product of operators and contractions of 
operators. 

Given a product 

\beq
XYZ\cdots W
\eeq

of creation and annihilation operators, the normal product is defined as 

\beq
N(XYZ \cdots W),
\eeq
 
where all the destruction operators stand to the right of the creation 
operators. Thus as an example let us study the cases

\be
N(a^\dagger_\alpha a_\beta)=a^\dagger_\alpha a_\beta
\ee

and

\be
N(a_\alpha a^\dagger_\beta)=\pm a^\dagger_\beta a_\alpha,
\ee

where the minus sign yields for operators acting only on fermions , and the plus 
sign yields for operators acting on  bosons. 

One of the properties of a normalized product of operators is that the
ground state expectation value of the product is zero since the destruction 
operator annihilates the ground state. 

A contraction of two operators $XY$ is defined as it's  expectation value 
regarding the ground state.

\be
\wick{1}{<1a_\alpha >1a^\dagger_\beta}=\bra{0}a_\alpha a^\dagger_\beta\ket{0}=\bra{0}\delta_{\alpha\beta}-a^\dagger_\beta a_\alpha \ket{0}=\delta_{\alpha\beta}
\label{contraction}
\ee  

By having defined the normal product and the contraction we are now ready to 
state Wicks theorem which says that a product of randomly oriented 
creation and annihilation operators can be written as the normal product of
these operators plus the normal product of all possible contractions.

\be
XYZ\cdots W=N(XYZ\cdots W) + \sum^{all \, possible}_{contractions} N(XYZ \cdots W)
\label{wicks}
\ee

As a remark, in this theorem only fermions has been considered.

The proof of this theorem can be found in almost all books that treat 
quantum field theory or quantum theory of many particles such as \cite{heinonen}.



\section{The Particle-Hole Formalism}
\label{particlehole}
In a theory of many particles, there is often more convenient to use another
state as reference state rather than the vacuum state. This reference state
should be a stable state. The normal ordering will then be altered from the one given above for the true vacuum state. That is our new vacuum state
$\ket{\Phi_0}=a^\dagger_ia^\dagger_j \cdots$. A somehow new definition of the
creation and destruction operators is needed. The operators will now create and 
annihilate holes and particles. The definition of a hole is a one particle
state that is occupied in the reference state $\ket{\Phi_0}$, while a 
particle state is one particle state that is not occupied in $\ket{\Phi_0}.$
This new nomenclature is easily understood when considering that a "hole" is
created when an originally occupied state is acted upon by an annihilation 
operator such as $a_i.$ A "particle" is created when an unoccupied state is
acted upon by a creation operator. These operators that destroy and create 
holes and particles are called quasiparticle 
operators. A q-annihilation operator annihilates holes and particles, 
while a q-creation operator creates holes and particles. 

A normal ordered product of quasiparticle operators would then be defined as
a product where all the quasiparticle destruction operators stand to the right
of all the quasiparticle creation operators. This definition of the normal 
ordered product changes the analysis of Wick's theorem a bit. The only
contractions that contribute are the ones where a destruction operator 
stands to the left of a creation  operator, there are two ways this can 
happen

\be
\begin{split}
& \wick{1}{<1a_i^\dagger>1a_j}=a^\dagger_i a_j-N(a^\dagger_ia_j)=a^\dagger_i
a_j+a_ja^\dagger_i=\delta_{ij}\\
& \wick{1}{<1a_i>1a^\dagger_j}=a_ia^\dagger_j-N(a_ia^\dagger_j)=a_ia^\dagger_j+a^\dagger_ja_i=\delta_{ij}
\label{normalhole}
\end{split}
\ee
 
That is if $i$ defines a hole state in Eqs. \eqref{normalhole}.

As an example, consider normal ordering of a two particle Hamiltonian, as 
the one in Eq. \eqref{twoham}.

\be
\hat H = \sum_{pq}\bra{p}h\ket{q}a^\dagger_p a_q + \frac{1}{4}\sum_{pqrs}
\bra{pq}V\ket{rs}a^\dagger_p a^\dagger_qa_sa_r
\label{twoham}
\ee


The one particle part can be written as 

\be
\sum_{pq}\bra{p}h\ket{q}N(a^\dagger_p a_q)+\sum_{i\in hole}\bra{i}h\ket{i}
\ee

While the two particle part would be rewritten as 

\be
\begin{split}
& \frac{1}{4}\sum_{pqrs} \bra{pq}V\ket{rs}a^\dagger_p a^\dagger_pa_sa_r=\\
& \frac{1}{4}\sum_{pqrs} \bra{pq}V\ket{rs}N(a^\dagger_p a^\dagger_q a_s a_r)
+\sum_{ipq}\bra{pi}V\ket{q i}N(a^\dagger_pa_r)+\frac{1}{2}\sum_{ij}\bra{ij}V\ket{ij}
\label{normH}.
\end{split}
\ee

After some tedious work, for the entire calculation see \cite{sjefer}.
After the equal sign in Eq. \eqref{normH} the letters $p,q,r,$ and $s$ indicate 
both hole and particle states, while the letters $i$ and $j$   indicate hole states.
The entire Hamiltonian is then written as

\be
\begin{split}
& \sum_{pq}\bra{p}h\ket{q}N(a^\dagger_p a_q)+\sum_i\bra{i}h\ket{i}+
\frac{1}{2}\sum_{ij}\bra{ij}V\ket{ij}+\\
& \frac{1}{4}\sum_{pqrs} \bra{pq}V\ket{rs}N(a^\dagger_p a^\dagger_q a_s a_r)
+\sum_{ipq}\bra{pi}V\ket{qi}N(a^\dagger_p a_q)
\label{normalham}
\end{split}
\ee

Where $p,q,r,$ and $s$ still run over all states, $i$ and $j$ over hole states only.
