\chapter{Special functions}
\section{Legendre polynomials}
\label{app:leg}
The Legendre functions are solutions of the differential equation 
\begin{align}
	&	{d \over dx} \left[ (1-x^2) {d \over dx} P(x) \right]+l(l+1)P(x) = 0,
		\label{eq:legdiff}
\end{align}
and by using Rodrigues' formula expressed as
\begin{align}
		&	P_l(x) = {1 \over2^l l!}{D^l \over dx^l}(x^2-1)^l.
		\label{eq:soluleg}
\end{align}
The Legendre polynomials satisfy an orthogonality property on the interval\\ $-1\leq x \leq 1$,
\begin{align}
		&\int^{1}_{-1}dxP_l(x)P_k(x)=\frac{2}{2l+1}\delta_{lk}.\notag
\end{align}
The Legendre polynomials for $l=0,\cdots, 5$ are\\
\\
%\begin{table}
%		\centering
		\begin{tabular}{ll}
$l$ &$	P_l(x)$,\\
$0$ &$	1$, \\
$1$ &$	x$, \\
$2$ &$	\frac{1}{2}(3x^2-1)$, \\
$3$ &$	\frac{1}{2} (5x^3-3x)$, \\
$4$ &$	\frac18 (35x^4-30x^2+3)$,\\
$5$ &$	\frac18 (63x^5-70x^3+15x)$.
%$6$ &$	\frac{1}{16} (231x^6-315x^4+105x^2-5)$\\
%$7$ &$	\frac{1}{16} (429x^7-693x^5+315x^3-35x)$\\
%$8$ &$	\frac{1}{128} (6435x^8-12012x^6+6930x^4-1260x^2+35)$\\
%$9$ &$	\frac{1}{128} (12155x^9-25740x^7+18018x^5-4620x^3+315x)$\\
%$10$ &$ \frac{1}{256} (46189x^{10}-109395x^8+90090x^6-30030x^4+3465x^2-63)$\\
\end{tabular}
%\end{table}

\section{Spherical Bessel functions}
\label{app:bess}
The spherical Bessel functions are solutions of the radial part of the differential equation
\begin{align}
		  &  x^2 \frac{d^2 y}{dx^2} + x \frac{dy}{dx} + (x^2 - \alpha^2)y = 0\notag
\end{align}
using spherical coordinates by separation of variables.
There are two linearly independent sets of solution to this equation $j_l(x)$ and $y_l(x)$, they are related to the ordinary Bessel functions $J_l$ and $Y_l$ by
\begin{align}
		&     j_{n}(x) = \sqrt{\frac{\pi}{2x}} J_{n+1/2}(x),\notag\\
\end{align}
and
\begin{align}
		&    y_{n}(x) = \sqrt{\frac{\pi}{2x}} Y_{n+1/2}(x) = (-1)^{n+1} \sqrt{\frac{\pi}{2x}} J_{-n-1/2}(x).\notag
\end{align}
In our calculation we have only used the spherical Bessel functions of first kind $j_l$ and they can be expressed as
\begin{align}
		&   j_n(x) = (-x)^n \left(\frac{1}{x}\frac{d}{dx}\right)^n\,\frac{\sin x}{x}.
\end{align}
While the spherical Bessel function of second kind can be expressed as 
\begin{align}
		&    y_n(x) = -(-x)^n \left(\frac{1}{x}\frac{d}{dx}\right)^n\,\frac{\cos x}{x}.\notag
\end{align}
The first four spherical Bessel functions of first kind are 
\begin{align}
&	j_0(x)=\frac{\sin x} {x}, \notag \\
&	j_1(x)=\frac{\sin x} {x^2}- \frac{\cos x} {x},\notag\\
&    j_2(x)=\left(\frac{3} {x^2} - 1 \right)\frac{\sin x}{x} - \frac{3\cos x} {x^2},\notag\\
&	j_3(x)=\left(\frac{15}{x^3} - \frac{6}{x} \right)\frac{\sin x}{x} -\left(\frac{15}{x^2} - 1\right) \frac{\cos x} {x}. \notag
\end{align}
