\chapter{Nuclear matter} 

Nuclear matter is an idealized theoretical system of nucleons, it is a nucleus
composed of infinitely many nucleons. This chapter will review some of the
properties of  nuclei. It is unfortunately impossible to cover all the physics
concerning nuclear physics, and the author humbly have to admit that much of it 
is still unclear. We will first go through the nuclear forces and nuclear
structure before we finish the chapter with the shell model. 



\section{Nuclear structure} 
\label{sec:semiemp}
%Different nuclei are
%composed of different amounts of nucleons, there exists some 
%similarities between different nuclei. By study the binding energies, the difference of it rest energy from the nucleons in terms of the nucleus' mass,
The binding energy, $B(N,Z)$, is given by
\be
B(N,Z)=\big( Nm_n+Zm_p -M(N,Z) \big),
\ee
where $N$ is the neutron number and $Z$ is the proton number. The binding energy
is almost proportional to the number of particles (both protons and neutrons), 
$A$, composing the nucleus, see Ref.\cite{siemenselementsnuclei}.\\
%\be
%B(A) \approx bA.
%\ee
It is also found by experiments that the radius, $R$, of a nucleus increases with the number of particles, $R=r_0A^{1/3}.$ The value of $r_0$ is estimated by experiments to be approximately  $1.2$ fm, see for example \cite{kraneintro}. When the nucleus is considered to be spherical, the volume, 
$\Omega=4 \pi R^3/3,$ is linearly dependent on the number of particles, as a consequence the particle density in a nucleus is independent of the number of constituents. By dividing the volume by the number of constituents in a nucleus results in a particle density on the form
\be
\frac{A}{\Omega}=\frac{3}{4\pi r_0^3}\approx 1.95 \times 10^{38} \,\mbox{particles/cm}^3\\
\ee
\\
Since the volume of a nucleus depends linearly on the number of particles (as a droplet does), we can 
model the nucleus with a liquid drop. With the liquid drop model we are able to find a formula
for the mass of a nucleus.  Since this formula is obtained with both empirical
data and theoretical assumptions this formula is called the semi-empirical mass
formula. The binding energy has (by using scattering data on nucleon-nucleon interaction) been parametrized as
\be
    B = a_{v} A - a_{S} A^{2/3} - a_{C} \frac{Z(Z-1)}{A^{1/3}} - a_{A} \frac{(A - 2Z)^{2}}{A} + \delta(A,Z).
\label{eq:semibinding}
\ee
 The first term is called the volume term, this term satisfy the almost linearly dependence on the nucleon number $A$.
 This linearly dependence of $A$ indicates that each nucleon attracts only it closest neighbors and not all the other
nucleons. Since we from experiments, such as electron scattering, have concluded
that the nucleus density is constant, every nucleon has the same amount of
closest neighbors. The exception are those nucleons that lie on the surface of
the nucleus, thus we have to subtract this term, since the term $a_vA$ is an
overestimate. The surface nucleons contribute with a negative term
$-a_sA^{2/3}.$ The repulsive coulomb term of the protons should also be taken
into consideration, assuming a uniformly charged sphere, we obtain that the
contribution is 
\beq
-a_{C} \frac{Z(Z-1)}{A^{1/3}}. 
\eeq
There are two more terms left,
which are mainly extracted from experiments.
The first one is called the asymmetry term, it accounts for the effect that
the most stable nuclei are symmetric, we include the term 
\beq
- a_{A} \frac{(A -2Z)^{2}}{A}.
\eeq
The last term is the pairing term, which accounts for the fact
that the most stable configuration is when the number of nucleons of the same kind  with spin up
is equal to the number of nucleons with spin down, which is a property because
of the Pauli principle.  The pairing term does not contribute if we have an odd
number of nucleons, but affect the mass or binding energy differently in the
two cases of even-even nuclei and odd-odd nuclei, we give it the symbol
$\delta(A,Z)$, its contribution to the formula is
\beq
    \delta(A,Z) = \begin{cases} +\delta_{0} & Z,N \mbox{ even } (A \mbox{ even}) \\ 0 & A \mbox{ odd} \\ -\delta_{0} & Z,N \mbox{ odd } (A \mbox{ even})\end{cases} \\
\eeq
To find the mass of a given nucleus we subtract the binding energy from the sum of the nucleon masses as done here;
\be
 m = Z m_{p} + N m_{n} - \frac{B}{c^{2}}.
\ee



\section{A review of nuclear forces}

The mere existence of the deuteron is an evidence of the nuclear force, or the
strong force, and that the force between protons and neutrons have to be
attractive at least in the $J^\pi = 1^+$ state, or the $^3S_1$ partial wave state, which carries the quantum numbers  of the deuteron.\\
\\
Interference between Coulomb and
nuclear scattering for the proton-proton partial wave $^1S_0$
shows that the nucleon-nucleon force is attractive at least for the $^1S_0$
partial wave, and that it has to be greater than the coulomb force at small distances.
If not the protons would have been repelled by the repulsive
electromagnetic forces the protons mediates with each other. However
for interparticle distances of atomic scale, the nucleon-nucleon interaction (of the strong force) is negligible.
The cross section for neutron-proton scattering is isotropic for
energies up to 10 MeV in the center of mass frame, it is then concluded that
the scattering occurs in the relative $S$ states.\\
\\
The nuclear force which is the same as the strong force, the one of quarks
and gluons.  The same force that holds the nucleus together is the same that
keeps together the quarks that combine to make hadrons. Hadrons are particles
that feel the strong force and are composite of quarks. The nucleons consist of
the up, $u$ and down, $d$ quarks.  There are three generations of quarks, six quarks
in total. Beside the already mentioned $u$ and $d$ quarks, we have the charm quark, $c$, strange quark $s$, the top quark $t$ and the bottom quark $b$. They were not all discovered at the same time, the most heavy,
the top quark was not discovered until 1995 by the CDF and D0 experiments at Fermilab, \cite{PhysRevLett.74.2422,PhysRevLett.74.2626}. We state the 
three generations of quarks as
\be
\begin{pmatrix}
u \\
d
\end{pmatrix},
\begin{pmatrix}
c\\
s
\end{pmatrix}
\mbox{and}
\begin{pmatrix}
t\\
b
\end{pmatrix}.
\ee
All quarks have both electric charge
and color charge. The electric charges differ by $e$ in each generation, where
the upper ones have $2/3e$ and the lower have charges $-1/3e,$ where $e$ is the
electron charge.  There are three types of color charges red blue and
green. The quarks are spin half particles which have to obey the Pauli principle.\\
\\
The nucleons, the proton and neutron consist of three quarks. The proton
consists of two $u$ quarks and one $d$ quark which combine to the electric
charge of one $e$. The neutron consists of two $d$ quarks and one $u$ quark
which make the neutron electrically neutral.\\
%Since the quarks are fermions and
%there exists hadrons where two of the same type of quarks are found to be in the
%same state, radial and spin state it was deduced that there must be an 
%extra quantum number identifying the quarks. This quantum number was named 
%color.\\ 
%!!!I have to check this
%In the last section I argued that the density is constant in the nuclei.  It
%does not matter how many nucleons you put into the system, the density remains
%the   same. This indicates that there is a repulsive core at very short
%distances, experimentally it is found to be at $\approx 0.5$ fm.\\
\\
Much of what is known about nucleons is by combinations of experiment and 
theoretical predictions. Much is known by nucleon-nucleon scattering. 
By calculating the differential cross section for nucleon-nucleon 
scattering it is understood that the nuclear potential is not just dependent on 
the coordinates but also on the spin of the particles.
The definition of the differential cross section $d\sigma/d\Omega,$ is the probability per unit solid angle that an incident particle is scattered into the
solid angle $d\Omega.$ The standard unit for measuring a cross section is the barn, b, and it is equal to $10^{-28}$m$^2$. The probability $d\sigma$ that an incident particle is 
scattered into $d\Omega$ is the ratio of the scattered current through $d\Omega$
to the incident current, see Ref. \cite{kraneintro}
\be
d\sigma = \frac{(j_{scattered})r^2d\Omega}{j_{incident}}
\label{Eq:crossec}
\ee
If we use that the current of the particles is
%%SHOW IT
\be
\bold{j}= \frac{1}{2mi}(\psi^*\nabla \psi- (\nabla\psi^*)\psi),
\ee
which is found by multiplying the Schr\"odinger equation with $\psi^*$,
\begin{equation*}
		\begin{split}
		i\psi^*\frac{\partial\psi}{\partial t}+\frac{1}{2m}
		\psi^*\nabla^2 \psi= i\frac{\partial}{\partial t}(\psi^*\psi)
		-i\frac{\partial \psi^*}{\partial t}\psi+\frac{1}{2m}\psi^*\psi\\
		=i\frac{\partial}{\partial t}(\psi^*\psi)+\frac{1}{2m}\nabla\left(\psi^*\nabla\psi-(\nabla\psi^*)\psi\right)=0\Rightarrow \frac{\partial \rho}{\partial t}+\bold{\nabla}\cdot \bold{j}=0,
\end{split}
\end{equation*}
where $\rho=\psi^*\psi$ is interpreted as the probability density.\\ 
It is just a matter of finding the wave function to calculate the cross section. We let the incoming wave have the form
\be
\psi_{incident}=\frac{A}{2ik}\left[\frac{e^{ikr}}{r}-\frac{e^{-ikr}}{r} \right].
\label{incident}
\ee
This form keeps the incident wave finite as $r \rightarrow 0.$ By assuming that the scattering cannot create or destroy particles, but only
change the phase of the outgoing wave, the total wavefunction can be written as
\be
\psi(r)=\frac{A}{2i}e^{-i\delta}\left[ \frac{e^{i(kr+i2\delta)}}{r} - \frac{e^{-ikr}}{r} \right].
\label{totalwf}
\ee
To find the scattered wave function we subtract the incident wave function Eq. \eqref{incident} from Eq. \eqref{totalwf},
where $\delta$ is the phase shift. 
The nodes of the wave function will be pushed away from the potential it sees if
the phase shift is negative and pulled inwards if the phase shift is 
positive.\\
\\
By using the formula for the differential cross section, Eq.\eqref{Eq:crossec}, we find it to be
\be
\frac{d\sigma}{d\Omega}=\frac{sin^2(\delta)}{k^2}.
\ee
The total cross section, which is interpreted as the probability to be scattered
in any direction is in the special case for $l=0$
\be
\sigma=\frac{4\pi sin^2(\delta)}{k^2}.
\ee
In order to get a better estimate to the cross section we need to consider the
spins.
Nucleons are fermions with spin 1/2, in the scattering process they
combine to either total spin 1, the triplet state or total spin 0, the singlet state. The
total cross section should then be the sum of the cross sections for all of the
possible states they can be in. There are in total four possible spin states, three spin 1 states and one spin 0 state. The probability for being in one
of the triplet states is 3/4 and in the singlet state 1/4. We can now write down the total cross section as
\be
\sigma = \frac{3}{4}\sigma_t + \frac{1}{4}\sigma_s,
\ee 
where $\sigma_t$ indicates the cross section for spin 1 states, and $\sigma_s$ for the
spin 0 state. By using parameters from deuteron scattering it is found that 
there is a significant difference between the cross sections for the triplet
state and singlet state, $\sigma_t=4.6$ b and $\sigma_s=67.8$ b.  This
difference can only be explained by a spin dependency in the nuclear force.\\
\\
If we assume that the charge is invariant under charge symmetry breaking and isospin symmetry breaking then the different nucleon-nucleon interaction channels, the proton-proton, neutron-neutron and neutron-proton, are all  identical. However in reality this symmetry is broken.\\
%The nucleon nucleon force is charge independent, where we here talk but the isospin charge. Two nucleons in a given two-body state will always feel
%the same force. The three nuclear forces pp, nn and np are identical.
\\
Observations of quadrupole moment of the ground state of the deuteron is an evidence for the tensor force. The ground state is a mixed orbital momentum, $l$, state, which must 
result from a non-central  potential. This force is on the form $V_T(r)S_{12}$
\be
S_{12}=3(\bold \sigma_1 \cdot \bold r)(\bold \sigma_2 \cdot \bold r)/r^2 -\bold \sigma_1 \cdot \sigma_2.
\ee
There is also a non-local part remaining, the spin orbit coupling $V_{LS}=V_{LS}(r)\bold{L\cdot S}.$ 


\section{The shell model}

The shell model description of the nucleus is in some senses similar to the
description of atomic shell structures. It is actually the atomic shell model that is the starting point since it has been so effective in describing the
atoms. Nuclear physicists attempted to describe nuclear theory in a similar way.\\ 
\\
However there are several important differences. In the atomic case the
electrons are orbiting the nucleus which acts as an external potential. In the nucleus
there are no external potential, the nucleons make their own field. In the atomic
case there are just one sort of particles to solve for, the electrons, at least 
in the Born-Oppenheimer approximation. In
the nuclear case we have two types of particles both protons and neutrons.
Evidence of a shell structure is increased stability of the nuclei when they
have a certain number $Z$  of protons and $N$ of neutrons. We
call those nuclei for magic nuclei. Magic nuclei are
determined to have $Z$  or $N =2,\, 8,\, 20,\,28, \, 50,\,82,\,126.
$ These numbers are more or less explained by introducing a one-body
attractive average field in the Hamiltonian,
\beq
\begin{split}
& H = T+V(r_1,r_2)= H=T+U(r) + V(r_1,r_2)-U(r)= T + U + V_I  \\
& =H_0 + V_I
\end{split}
\eeq
here $H_0$ denotes an attractive, or bounded one-body potential all nucleons feel. The smaller $ V_I $  is, the better is the
assumption of an independent field.\\
\\
The question that arises is what form the potential should have, to give the correct magic nuclei. In Ref. \cite{kraneintro} they are using a 
potential on an intermediate form between an infinite well and a harmonic potential
\beq
U(r)= \frac{-U_0}{1 + e^{\frac{r-R}{a}}}, 
\eeq
where  $R$ is the mean nuclear radius and $a$ is the skin thickness. The skin thickness is related to the charge density of a nucleus. It is the distance onver which the charge density falls from $90$\% of its central value to $10$\%. The skin thickness value $a$ is approximately $2.3$ fm.
However in order to get all the magic numbers they had to add a spin orbit term to the potential, a factor $ U_{sl}l\cdot s$. The form of $ U_{sl}$ is not so important as that the factor $ l\cdot s $ splits the energy levels.
By using the angular momentum relations 
\beq
\begin{split}
&j^2=(l+s)^2=l^2 + s^2 + 2l\cdot s \\
& l\cdot s= \frac{1}{2}(j^2 - l^2-s^2)
\end{split}
\eeq
and inserting for the eigenvalues for $j$, $l$ and $s$ we find the factor to be
\beq
\langle l\cdot s\rangle = \frac{1}{2}\left(j(j+1)+l(l+1)+ \frac{3}{4}\right).
\eeq
With the additional $ls$ term one could explain the magic numbers.

%With this form the physicists have managed to reproduce the same magic numbers as found experimentally, and they have even predicted a new one, at 184. 

\section{Energy per particle}

The main goal of physics is understanding the world and forces surrounding us,
when we have a model we need it to predict some properties which we can
measure, such as the force or energy. In the case of nuclear matter it is the
energy per particle which is the predictive quantity we wish to compare with
the experimentally known value. This quantity is called the
binding energy, which is defined as the energy needed to take a particle out of
the system it is interacting in. By dividing the binding energy with the nucleon number $A$,  
\be
B=\frac{E}{A}= \frac{E_{kin}}{A} + \frac{E_{interaction}}{A}.
\ee 
we can find an ''experimental'' value of the binding energy per nucleon
for symmetric nuclear matter, ie.~when the nucleon number  goes to infinity,
with an equal amount of protons and neutrons.  
From Eq. \eqref{eq:semibinding} we see that the only surviving term is the volume term $a_v$ which is approximately 16 MeV.\\
\\
As physicists we are not satisfied with just empirical and experimental values.
We want  to understand why it is so. We want to derive it with the theoretical
tools available, but this task is not formidable.\\
\\
If we approximate the wave functions as plane waves and assume that the 
nucleons form a non-interacting Fermi gas, we can estimate the saturation density which corresponds to the Fermi momentum $k_f$.
\\
The number of particles in a non-interacting Fermi gas is given by the equation
\be
N=\nu \int_0^{k_f}\Omega \frac{d^3k}{(2\pi)^3} =\Omega  \nu \frac{k_f^3}{3\cdot 2\pi^2},
\label{Eq:nr_partic}
\ee 
where $\nu$ is the degeneracy factor and $\Omega$ indicates the volume. The degeneracy factor $\nu$ is in the nuclear case equal to four. We have two isospin states and two spin states. From the quantum mechanical solution to 
the infinite well, with sides $L$, we can show that the principal number $n$ is related to the wave number $ k $ by 
\be
k=\frac{2\pi n}{L}.
\ee
When we operate in a three dimensional world $ d^3n=d^3k L^3/(2\pi)^3$ where $ L^3=\Omega$.\\
Since we let the volume go to infinity the amount of particles gets undefined, however the particle density is a well defined
quantity by dividing Eq. \eqref{Eq:nr_partic} by $\Omega $ and performing the integral over $k$ we get the density 
\be
\rho = \nu \frac{k_f^3}{3\cdot 2\pi^2}.
\label{eq:density}
\ee
With these relations it is possible to calculate the Fermi level of nuclear matter, in section  \eqref{eq:semibinding} we found a value for 
\beq
\frac{A}{\Omega}=\frac{2k_f^3}{3\cdot \pi^2}= 1.95\times10^{38}.
\eeq
Here we have used that the degeneracy factor $\nu$ is equal to four, we find 
that the Fermi level corresponds to $k_f \approx 1.42~\mbox{fm}^{-1}.$\\ 
The kinetic energy density is calculated by the formula
\be
\int_0^{k_f} dk \frac{3k^4}{4mk_f^3}=\frac{3k_f^2}{2\cdot 5m}. 
\ee
The interaction part is at least a two-body interaction. It is convenient to work in the
momentum picture and we write our two-body interaction as 
\be
\begin{split}
&\sum_{\substack{j_al_a,tz_a,j_b,l_b,tz_b\\j_c,l_c,tz_c,j_d,l_d,tz_d}}\int \frac{d^3k_a}{(2\pi)^3}\int \frac{d^3k_b}{(2\pi)^3}\int \frac{d^3k_c}{(2\pi)^3}\int \frac{d^3k_d}{(2\pi)^3} \\
&\times \bra{k_a j_al_a tz_a k_bj_bl_btz_b J Tz}V(k_a,k_b,k_c,k_d)\ket{k_c j_cl_c tz_c k_dj_dl_dtz_d J Tz}
\end{split}
\ee
In section \ref{sec:calc_int} we show how this is computed in the space of relative and center of mass coordinates. The form of
the potential may be the Bonn potential or N$^3$LO as is the case in our project.  
