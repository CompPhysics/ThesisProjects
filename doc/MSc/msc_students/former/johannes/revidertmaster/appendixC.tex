\chapter{Brueckner $G$-matrix}
\label{bgm}

The Brueckner $G$-matrix is one of the most important ingredients in many-body
calculations. The $G$-matrix was developed for microscopic nuclear matter
calculations, \cite{bruecknerG,bbp1963}. It is a method to overcome the non perturbative character of the
nuclear force, caused by the short range repulsive core in the NN
interaction.\\ 
\\
We want to calculate the nuclear matter ground state energy by using the non-relativistic Hamiltonian
\begin{equation*}
		H\Psi_0(A)=E_0\Psi_0(A),
\end{equation*}
where $H=T+V$ and $A$ denotes the particle number, $T$ is the kinetic energy and
$V$ is the nucleon-nucleon potential.
The unperturbed problem, is 
\begin{equation*}
		H_0\psi_0(A)=W_0\psi_0(A).
\end{equation*}
In this case, $H_0$ consists just of the kinetic energy, and $\psi_0$ is a Slater determinant representing the Fermi sea. The full ground state energy, $E_0$ is
\begin{equation*}
		E_0=W_0+\Delta E_0,
\end{equation*}
where $\Delta E_0$ is the ground state energy shift and is the value we need to find,
since $W_0$ is easily obtained. The energy shift is normally found with
perturbation theory. When the the potential $V(r)$ contains a strong
short-range repulsive core, the matrix elements containing $V$ will become very
large and contribute repulsive to the ground state energy. Thus it is
meaningless to treat the problem with perturbation theory.\\
\\
The resolution to this problem, was provided by Brueckner by introducing a
matrix, the so-called $G$-matrix.  It can be compared with the function\\ $f(x)=x/(1-x)$,
this function may be expanded in the series $f(x)=x+x^2+x^3+\cdots$ when x is
small, and it is not necessary to compute all terms if we want an
approximation. If $x$ is large, the power series become meaningless, but the
exact function $x/(1-x)$ is still well defined. Brueckner suggested that one
should sum up all terms in the perturbative approach, this sum is denoted by
$G_{ijij}$, where $G_{ijij}=\bra{ij}G\ket{ij}$.  The expression for $G$ is
\begin{equation*}
		\begin{split}
				&	G_{ijij}=V_{ijij}+\sum_{mn>k_f}V_{ijmn}\frac{1}{\varepsilon_i+\varepsilon_j-\varepsilon_m-\varepsilon_n}\\
		&		\times \bigg(V_{mnij}+\sum_{pq>k_f}V_{mnpq}\frac{1}{\varepsilon_i+\varepsilon_j-\varepsilon_p-\varepsilon_q}V_{pqij}\bigg).
		\end{split}
\end{equation*}
Which we again can write as
\begin{equation*}
		G_{ijij}=V_{ijij}+\sum_{mn>k_f}V_{ijmn}\frac{1}{\varepsilon_i+\varepsilon_j-\varepsilon_m-\varepsilon_n}G_{mnij}
\end{equation*}
The matrix elements become $\bra{\psi}G\ket{\psi}=\bra{\psi}V\ket{\Psi}$. Where $\Psi$ is the correlated wave function. When
it is not possible to solve for the $G$-matrix with matrix inversion it is done
by an iterative approach.\\
\\
It is useful to write the $G$-matrix in a more general form 
\begin{equation*}
		G_{ijij}=V_{ijij}+\sum_{mn>0}V_{ijmn}\frac{Q(mn)}{\omega-\varepsilon_m-\varepsilon_n}G_{mnij}.
\end{equation*}
The factor $Q(mn)$ corresponds to
\begin{equation*}
		Q(k_m,k_n)= \begin{cases} 1 & \mbox{min}(k_m,k_n)>k_f\\ 0 & \mbox{else}  \end{cases} 
\end{equation*}
The role of $Q$ is to enforce the Pauli principle by preventing scattering to occupied states. 
The $G$-matrix can be written on a more compact form, by noticing 
\begin{equation*}
		H_0\ket{\psi_m\psi_n}=(\varepsilon_m+\varepsilon_n)\ket{\psi_m\psi_n},
\end{equation*}
to 
\begin{equation*}
		G(\omega)=V+V\frac{Q}{\omega-H_0}G(\omega),~ Q=\sum_m\ket{\psi_m\psi_n}Q(mn)\bra{\psi_m\psi_n}.		
\end{equation*}
If the Pauli exclusion operator, Q, does not commute with the Hamiltionian $H_0$ we have to do the replacement 
\begin{equation*}
		\frac{Q}{\omega-H_0}\rightarrow Q\frac{1}{\omega-H_0}Q.
\end{equation*}
There are a number of complexities with the calculation of the $G$-matrix, we have already mentioned one, when the Q does not commute with $H_0$, the determination of the starting energy $\omega$ may also be a problem.

