\chapter{The calculations} 
%!!!!!!!!!!!!!!!!!!Finn et bedre navn p\aa \, kapitlet!!!!!!!!!!!!!!!!!!!!!!!!!!!  
%\\ 
\section{The matrix elements}
The matrix elements used in the coupled cluster approximation was ordinarily given in an oscillator basis, but
transformed into a planewave basis. The general method to transform from one basis to another is by projecting the basis we want on an orthogonal and 
complete basis

\be
\ket{\alpha}=\sum_a\ket{a}\braket{a}{\alpha}.
\label{transform}
\ee

What we need is the projection of $\bra{a}$ on $\ket{\alpha}$. If we know the matrix elements of the basis $a$ and the operator $O$, then we can find the
matrix elements of the basis $\alpha$ and $O$ by projecting the complete orthonormal basis $a$ on it. 

\be
\bra{\alpha}O\ket{\alpha'}=\sum_{a,a'}\braket{\alpha}{a}\bra{a}O\ket{a'}\braket{a'}{\alpha'}
\ee


If our set of orthonormal basis states is a continuous and infinite one, the sum gets replaced by an integral. See for example Ref. Ref. \cite{gamow}.
If our operator $O$ is a twoparticle operator, our twoparticle overlap integrals read 

\be
\braket{ab}{\alpha \beta}= \frac{\braket{a}{\alpha}\braket{b}{\beta}-(-1)^{J-j_\alpha-j_\beta}\braket{a}{\beta}\braket{b}{\alpha} }{\sqrt{(1+\delta_{ab})(1+\delta_{\alpha\beta})} }
\ee

If the two particles are identical, that is proton-proton or neutron-neutron. In the proton-neutron case it reads

\be
\braket{ab}{\alpha\beta}=\braket{a}{\alpha}\braket{b}{\beta}.
\label{twopartoverlap}
\ee

With the Eqs. (\ref{transform} - \ref{twopartoverlap}), it is straightforward to make the transformations from an oscillator basis to a planewave basis. 

\be
\ket{k_a l_a j_a k_b l_b j_b J T_z} = \sum_{n_a n_b} \braket{n_a l_a j_a n_b l_b j_b J T_z}{k_a l_a j_a k_b l_b j_b J T_z}\ket{n_a l_a j_a n_b l_b j_b J T_z}
\ee
 
The task gets down to compute the twoparticle overlap $\braket{n_a l_a j_a n_b l_b j_b J T_z}{k_a l_a j_a k_b l_b j_b J T_z}$. The way to compute this relation is to look at the reverse transformation,

\be
\ket{n_a l_a j_a n_b l_b j_b J T_z}= \int k_a^2 \int k_b^2 \braket{k_a l_a j_a k_b l_b j_b J T_z}{n_a l_a j_a n_b l_b j_b J T_z}\ket{k_a l_a j_a k_b l_b j_b J T_z}dk_adk_b.
\label{transformation}
\ee
Which is nothing more than

\be
\int k_a^2 \int k_b^2 R_{n_al_a}(k_a)R_{n_bl_b}(k_b)\ket{k_a l_a j_a k_b l_b j_b J T_z}dk_adk_b.
\ee

Now, if we project the state $\bra{k'_a l'_a j'_a k'_b l'_b j'_b J'T'_z}$ on Eq. \eqref{transformation} 
we will find the expression for the transformation bracket.

\be
\begin{split}
& \braket{k'_a l'_a j'_a k'_b l'_b j'_b J'T'_z}{n_a l_a j_a n_b l_b j_b J T_z}= \\
& \int k_a^2 \int k_b^2 R_{n_al_a}(k_a)R_{n_bl_b}(k_b)\braket{k'_a l'_a j'_a k'_b l'_b j'_b J'T'_z}{k_a l_a j_a k_b l_b j_b J T_z}dk_adk_b\\
& \int k_a^2 \int k_b^2 R_{n_al_a}(k_a)R_{n_bl_b}(k_b)\delta(k_a-k'_a)\delta(k_b-k'_b)\delta_{j_aj'_a}\delta_{l_al'_a}\delta_{j_bj'_b}\delta_{l_bl'_b}dk_adk_b\\
& = k'^2_{a}k'^2_{b}R_{n_al'_a}(k'_a)R_{n_bl'_b}(k'_b)\delta_{j_aj'_a}\delta_{j_bj'_b}\\
\end{split}
\ee

Luckily it is easily computed. The oscillator function is extended in Laguerre polynomials. 

















