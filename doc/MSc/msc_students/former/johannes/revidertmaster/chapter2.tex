\chapter{Some historical aspects regarding quantum mechanics}

It is not easy to give a short presentation of quantum mechanics, it is a rather huge and strange subject. When speaking about quantum mechanics we should always keep in mind what Feynman said, "I think it is safe to say that no one understands Quantum Mechanics". 
The world quantum mechanics treats, is a small world, the world of very small objects, such as electrons, atoms and nuclei. \\
\\
The most natural point to start when reviewing quantum mechanics is maybe how
it started. It started with light, the feature of light has long been an
important part in physics. The explanation of light has long been alternating
between the definition of light as a wave picture and a corpuscular picture.  
Light has
interested man in maybe all of time. The Iraqi born scientist Ibn
al-Haytham (965-1040), which in the west goes under the name Alhazen, in his
Book of optics, treats light as energy particles that travel in straight
lines at a high but finite speed \cite{alhazen}. Issac Newton followed the
particle interpretation of light, however he understood that he had to
associate light with waves in order to explain the diffraction properties of
light.
Robert Hooke and Christian Huygens believed light to be waves and worked out their own and separate theories of light. \\
\\
In our everyday life we can see a clear distinction between waves and
particles, waves exhibit a phenomena called interference which particles do
not. Interference occurs when two waves traveling in the same medium meet. As
an example we can look at two sine waves traveling in opposite directions and
with the same amplitude. If these two waves meet when both are on their
maxima the net result of the waves will be a peak with twice the amplitude of
the waves, which is an example of constructive interference. If the waves are
completely out of phase when they meet, one of the waves is phasing upwards
and the other downwards, the net result will be a zero peak, this type of
interference is called destructive. There will  be constructive interference
when the displacement of the two waves are in the same direction and
destructive when the displacement of the waves are in opposite directions.\\ 
\\
There are two experiments which are rather crucial in quantum mechanics, and
revolutionized physics.  The photoelectric effect, explained by Einstein,
for which he got the Nobel prize and the double-slit experiment.  In the
photoelectric effect light is scattered on metal and collides with the
electrons. The collisions can be registered by measuring the current. If light
were to be a wave the average energy measured of a single electron should
increase with the intensity, the phenomena observed was a surprise. The energy
of the ejected electrons did not at all depend on the intensity. It was found
that it depends on the frequency of the light waves, and that below a
certain frequency there were no ejected electrons. Einstein resolved this
paradox by proposing that light consists of individual quanta, which now are
called photons. These photons carry energies which come in discrete quanta. The
energy can just come in amounts of \sd \hbar \omega\sd, where \sd \hbar\,\sd
goes under the name of Planck's constant, and \sd\omega\,\sd is the frequency
of the light. By varying the frequency of light it was also discovered that
the momentum $p$ is proportional to the wavenumber $k$ and a multiple of planck's constant, $p=\hbar k$. With these expressions of energy and momentum it was
deduced from Einstein's famous equation for energy \sd E=\sqrt{p^2c^2
+m^2c^4}\sd\, that the photon is massless. \\
The double slit experiment with light shows the opposite behavior. 
In the double slit experiment light waves are send in a way such that
they are incident normally on a screen with two slits \sd S_1\,\sd and \sd
S_2\sd, which are a distance \sd a\,\sd apart. If only slit \sd S_1\,\sd is
open an intensity pattern \sd I_1\sd, is observed. And likewise if only \sd
S_2\,\sd is open an intensity pattern \sd I_2\,\sd is observed. When both of
the slits are left open an interference pattern is observed, what is crucial is
that the intensity \sd I_{1+2}\,\sd is not \sd I_1 +I_2\sd \,  which would be
the case if light were to be particles.\\
\\
This seemingly contradictory properties of light was then interpreted as the
particle wave duality of light. The Copenhagen interpretation, which states
that particles such as photons, but also electrons and other small particles
have both wave and particle properties. The particles obey a complementary principle
which states that an experiment can show particle like properties and another
wave like properties, but none can show them both at the same time.  This is
the most accepted interpretation of quantum mechanics, however Einstein has
always questioned this interpretation and together with Podolsky and Rosen
proposed a paradox later called the EPR paradox. We will not go through this
paradox, it can be read in any book treating quantum theory. When Aspect did
experiments on Bell's inequalities, he showed the consistency of the Copenhagen interpretation. However Afshar claims that he
in a recent experiment has showed both particle and wave properties at the same time,
Refs. \cite{afshar-2005-5866,afshar-2006-810}.\\ 
\\
Now the time has come to say something about the postulates and mathematics of
quantum mechanics.  The first postulate is that the state of a particle is
represented by a vector, or ket \sd\ket{\Psi(t)}\sd\, in the Hilbert space, \sd
\mathcal H\sd. All properties of the particle are contained in this wave
function. Properties of the particles which can be measured, such as position,
energy and velocity are in quantum mechanics called observables and are
represented by operators.  If a particle is in a state \sd \ket{\Psi}\sd, the
measurement of a variable \sd O\sd, will yield us one of the eigenvalues
\sd o\sd. The probability that the eigenvalue \sd o\sd \, is measured is
\sd|\braket{o}{\Psi}|^2\sd. After the measurement, the state of the system
changes from the state \sd \ket{\Psi}\sd\, to the state \sd\ket{o}\sd. This
effect is called the collapse of the state.  Complications caused by the
collapse of the wave function arise when measuring different observables. If we
measure an observable \sd \lambda\sd, just after the observable \sd \omega\sd\,
is measured we are not generally expected to get an accurate value of \sd \lambda\sd. When
we measure \sd \omega\sd\, the wave function collapses to the eigenfunction corresponding to the 
eigenvalue we get for its corresponding operator $\Omega$. The condition for getting an accurate value for both of the observables is that theirs corresponding operators commute
\beq
[\Omega,\Lambda]=\Omega\Lambda-\Lambda\Omega=0.
\eeq  
If two operators do not commute they form a different set of eigenfunctions, and we cannot measure both 
eigenvalues without
an uncertainty. The least uncertainty is the value \sd[\Omega,\Lambda] \sd. As an example of two operators
that do not commute are the two
operators of position and momentum, \sd[X,P]=i\hbar\sd.\\ 
\\
The last postulate treats the state's evolution with time. All states obey the Schr\"odinger equation
\be
i\hbar\frac{d}{dt}\ket{\Psi(t)}=H\ket{\Psi(t)},
\ee
where $H$ is the Hamiltonian operator whose eigenvalue denotes the energy of the
system. When we
are considering a system we use the classical Hamiltonian, but change all the
observables to operators. 
For instance the Hamiltonian describing a classical harmonic oscillator is
\beq
H=\frac{p^2}{2m}+\frac{1}{2}m\omega^2x^2,
\eeq
while in quantum mechanics it is on the form 
\beq
H=\frac{P^2}{2m}+\frac{1}{2}m\omega^2X^2,
\eeq
where \sd P\sd\, is the momentum operator and \sd X\sd \, the position
operator. When we work in coordinate space the momentum operator becomes a differential operator \sd P =
-i\hbar \nabla\sd.
Since \sd H\sd\, is an operator it should have an eigenvalue and an eigenstate. This has to be used
in order to find the state of a particle. We have to solve the equation 
\beq
H\ket{\Psi(t)}=E\ket{\Psi(t)},
\eeq
where the energy $E$ is the eigenvalue corresponding to the eigenket $\ket{\Psi(t)}$.
It is not always easy to solve the Schr\"odinger equation since it is a differential equation and  when 
we have to solve a many body problem it may seem impossible.  
%Two incompatible operators or observables do not share the same
%eigenfunctions, this gives us some complications when we want to measure
%simultaneously. 
%
%
%In the world of quantum mechanics the observables are quantized to operators
%that act on a state.  The momentum which is regarded as the generator for
%infinitesimal translations, and becomes a differential operator, \sd
%-i\hbar\nabla \sd. 

