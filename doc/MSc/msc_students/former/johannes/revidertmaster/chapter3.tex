\chapter{Second Quantization}
\label{chapsecondq}

When we are doing nuclear physics, we are actually studying many-particle systems. The difficult task in a many-body system 
is the interparticle potential. The nuclear case is not an exception.  
It turns out that a direct solution of the Schr\"odinger equation in configuration space is impractical, see for example Ref. \cite{fetter}.\\
\\
As mentioned in the last chapter a many-body problem may
seem impossible to solve. The problem complicates because of the interparticle potential. However the second quantization method has turned out to be a helpful and  practical tool when treating many-body physics.\\
\\
In second quantization one define the so-called creation operator,
$a^\dagger_\alpha$, which create a particle  and the annihilation operator
$a_\alpha$, which annihilate a particle. The subscript $\alpha$ indicates the
set of quantum number the particle have. A set of quantum numbers a particle
can have defines what is usually called a single-particle state.\\
\\
%In quantum mechanics the observables were quantized and described by operators, this is considered to
%be the first quantization. In second quantization  we are not only quantizing the observables, but also the particle states. 
%Each particle state is described by operators, we call them creation and annihilation operators, since they create
%and annihilate particles respectively.\\ 
The system studied in this thesis consists of nucleons which belong to the types
of particles called fermions. Fermions are particles with half integer spin. We
describe a system of fermions by an antisymmetric wave function, which
obey the Pauli principle. The Pauli principle states that two identical fermions
cannot have the same set of quantum numbers, ie.~they cannot be in the same
single particle state.\\
\\
As the case in this thesis, the
Hamiltonian takes the form \be H=\sum_{k=1}^Nt(x_k)+\frac{1}{2}\sum_{k \neq l
=1}^N v(x_k,x_l), \label{hamfirst} \ee  where $t$ is the kinetic energy and $v$
is the potential energy, the interactions between the particles. The letter
$x_k$ denotes the coordinates of particle $k$.  Since the potential energy term
represents the interaction between every pair of particles, counted once, we
need to multiply with the factor  $1/2$, see for example Ref.~\cite{fetter}, in
order to not double count.

\section{Creation and annihilation operators}

The interpretation of occupation of the antisymmetric many-body fermion
states allow us to introduce the two operators $a^\dagger_\alpha$ and 
$a_\alpha$, which create and annihilate a particle in the single particle 
state $\alpha$, which can be expressed as
\be
\begin{split}
& a^\dagger_\alpha \ket{0}=\ket{\alpha} ~ \mbox{and}~  a_\alpha \ket{\alpha}=\ket{0}
\end{split}
\ee 
respectively. The state vector $\ket{0}$ indicates the true vacuum.
The algebra of these operators depends on whether the system under 
consideration is a system of bosons or a system of fermions. If it consists of bosons the operators obey the commutation relations
\be
\begin{split}
& [a_k,a^\dagger_{k'}]=\delta_{k,k'}~\mbox{and}~ [a_k,a_{k'}]=[a^\dagger_k,a^\dagger_{k'}]=0,
\end{split}
\ee
while in the fermion case they obey the anti commutation relations
\be
\begin{split}
		& \{a_k,a^\dagger_{k'}\}=\delta_{k,k'}~ \mbox{and}~ \{a_k,a_{k'}\}=\{a^\dagger_k,a^\dagger_{k'}\}=0.
\end{split}
\ee
The Kronecker delta, \sd \delta_{k',k}\sd\,is equal to one if \sd k'=k\sd\, and zero else.\\
\\
With the above expressions for the commutators and anti-commutators
regarding the creation and annihilation operators the many-body Hamiltonian can 
be written as
\be
H=\sum_{ik} t_{ki}a^\dagger_ka_i + \frac{1}{2}\sum_{ijkl}v_{ijkl}a^\dagger_i
a^\dagger_ja_la_k.
\label{hamiltoniansec}
\ee
Many-body operators are denoted by capital letters in this text.\\
When the operators in the second quantization are non-relativistic and 
conserve the particle number, there should be an equal amount of creation
and destruction operators in the Hamiltonian. A second quantized 
many-body operator is written as a sum of one-particle operators, an operator 
that acts on one particle at a time as in Eq. \eqref{onesec}
\be
F=\sum_{\alpha,\beta}\bra{\alpha}f\ket{\beta}a^\dagger_\alpha a_\beta,
\label{onesec}
\ee 
and as a sum of two-particle operators in the form
\be
V=\frac{1}{2}\sum_{\alpha\beta\gamma\delta}\bra{\alpha\beta}v\ket{\gamma\delta}
a^\dagger_\alpha a^\dagger_\beta a_\delta a_\gamma.
\label{twosec}\\
\ee
\\
By using creation and annihilation operators we are able to write down a
many-body wave-function which we denote by capital Greek letters in contrast to 
single-particle states which are denoted by small Greek letters. A wave function
consisting of $N$ particles is written
as a product of $N$ creation operators,\\ $\ket{\Phi}
=a^\dagger_{1}(x_1)a^\dagger_2(x_2)a^\dagger_3(x_3)\cdots
a^\dagger_N(x_N)\ket{0}$. Here $x_1$ refers to the coordinates of particle 1,
the ket vector $\ket{0}$ still indicates the true vacuum and the subscript of
the creation operator refers to the single particle state the particle
occupies. The problem with this definition of the many-body wave function is
that it is not a symmetry eigenstate. By quantum mechanics every particle
should be able to occupy every single-particle state, $\varphi_i$, with a probability $p$. Since we now
are dealing only with fermions whose wave-function should be antisymmetric by
the interchange of two particles, we write the wave function as a determinant,
\be
\frac{1}{\sqrt{N!}}
\left|
\begin{array}{cccccc}
\varphi_i(x_1) & \varphi_j(x_1) & \varphi_k(x_1) & \varphi_l(x_1)& \cdots& \varphi_N(x_1) \\
\varphi_i(x_2) & \varphi_j(x_2) & \varphi_k(x_2) & \varphi_l(x_2)& \cdots& \varphi_N(x_2) \\
\varphi_i(x_3) & \varphi_j(x_3) & \varphi_k(x_3) & \varphi_l(x_3)& \cdots& \varphi_N(x_3)  \\
\varphi_i(x_4) & \varphi_j(x_4) & \varphi_k(x_4) & \varphi_l(x_4)& \cdots& \varphi_N(x_4) \\
\vdots & \vdots & \vdots & \vdots & \vdots& \vdots \\
\varphi_i(x_N) & \varphi_j(x_N) & \varphi_k(x_N) & \varphi_l(x_N)& \cdots& \varphi_N(x_N). \\
\end{array}
\right|
\label{eq:slaterdet}
\ee
In the many-body jargon this determinant is called a Slater determinant.\\
\\
%Interchanging two of the particles would be the same as interchanging two rows 
%in the Slater determinant, since this also changes the overall sign of the 
%determinant, obeying the anti-symmetry condition, this form of writing a many-body wave-function is 
%justified.
A more handy form to write the wave function, is as a permutation of every 
possible single-particle states $\varphi_i$. We rewrite the many-body wave-function as
\begin{equation*}
		\ket{\Phi}=\prod_{\substack{ij=1,\\i\neq j}}^NP(ij)a^\dagger_1(x_1)a^\dagger_2(x_2)a^\dagger_3(x_3)\cdots a^\dagger_N(x_N)\ket{0}.
\end{equation*}
The permutation operator $P(ij)$ is defined as when acting on
$a^\dagger_i(x_i)a^\dagger_j(x_j)$ gives $-a^\dagger_i(x_j)a^\dagger_j(x_i)$. 



\section{Wick's Theorem}
\label{wicksteorem}
A normal ordered second quantized operator is defined as an operator whose 
 annihilation operators stands to right of all creation operators. 
It is in some manner easier to calculate when the annihilation operators are
placed to the right.
Wick's theorem describes a fast method to put the annihilation operators to
the right of the creation operators, by using the anti commutation rules
for these operators. Before introducing Wick's theorem we present some
definitions like the normal product of operators and contractions of 
operators.\\ 
Given a product of creation and annihilation operators 
\beq
XYZ\cdots W,
\eeq
the normal product is defined as 
\beq
N(XYZ \cdots W),
\eeq
where all the destruction operators are moved to the right of the creation 
operators. As an example let us study the cases
\be
N(a^\dagger_\alpha a_\beta)=a^\dagger_\alpha a_\beta
\ee
and
\be
N(a_\alpha a^\dagger_\beta)=\pm a^\dagger_\beta a_\alpha,
\ee
where the minus sign applies for fermions only, and the
plus sign for bosons.\\ 
\\
One of the properties of a normal ordered product of operators is that the
vacuum expectation value of the product is zero, the destruction 
operator annihilates the vacuum state.\\
\\
A contraction of two operators $XY$ is defined as its  expectation value 
regarding the vacuum, $\ket{0}$,
\be
\wick{1}{<1a_\alpha >1a^\dagger_\beta}=\bra{0}a_\alpha a^\dagger_\beta\ket{0}=\bra{0}\delta_{\alpha\beta}-a^\dagger_\beta a_\alpha \ket{0}=\delta_{\alpha\beta}.
\label{contraction}
\ee  
By having defined the normal product and the contraction in Eq.~\eqref{contraction} we are now ready to 
state Wick's theorem which says that a product of randomly oriented 
creation and annihilation operators can be written as the normal product of
these operators plus the normal product of all possible contractions.
\be
XYZ\cdots W=N(XYZ\cdots W) + \sum^{\mbox{all possible}}_{\mbox{contractions}} N(XYZ \cdots W).
\label{wicks}
\ee
As a remark, in this theorem only fermions have been considered.
The proof of this theorem can be found in almost all books that treat 
quantum field theory or quantum theory of many-particles, see for example Ref.~\cite{heinonen}.



\section{The Particle-Hole Formalism}
\label{particlehole}
In a theory of many-particles, it is often more convenient to use another
state as reference state rather than the vacuum state. This reference state
should be a stable state. The normal ordering will then be altered from the one 
given above for the true vacuum state, it is written 
$\ket{\Phi_0}=a^\dagger_ia^\dagger_j \cdots\ket{0}$. A new definition of the
creation and destruction operators is needed. The operators will now create and 
annihilate holes and particles. The definition of a hole is a one-particle
state that is occupied in the reference state $\ket{\Phi_0}$, while a 
particle state is a one-particle state that is not occupied in $\ket{\Phi_0}.$
This new nomenclature is easily understood when considering that a "hole" is
created when an originally occupied state is acted upon by an annihilation 
operator such as $a_i.$ A "particle" is created when an unoccupied state is
acted upon by a creation operator. These operators that destroy and create 
holes and particles are called quasiparticle 
operators. A q-annihilation operator annihilates holes and particles, 
while a q-creation operator creates holes and particles.\\ 
\\
A normal ordered product of quasiparticle operators would then be defined as
a product where all the quasiparticle destruction operators stand to the right
of all the quasiparticle creation operators. This definition of the normal 
ordered product changes the analysis of Wick's theorem. The only
contractions that contribute are the ones where a destruction operator 
stands to the left of a creation  operator, there are two ways this can 
happen
\be
\begin{split}
& \wick{1}{<1a_i^\dagger>1a_j}=a^\dagger_i a_j-N(a^\dagger_ia_j)=a^\dagger_i
a_j+a_ja^\dagger_i=\delta_{ij}\\
& \wick{1}{<1a_i>1a^\dagger_j}=a_ia^\dagger_j-N(a_ia^\dagger_j)=a_ia^\dagger_j+a^\dagger_ja_i=\delta_{ij}.
\label{normalhole}
\end{split}
\ee
That is if $i$ defines a hole state in Eqs. \eqref{normalhole}.
As an example, consider normal ordering of a two particle Hamiltonian, as 
in the following equation
\be
\hat H = \sum_{pq}\bra{p}h\ket{q}a^\dagger_p a_q + \frac{1}{4}\sum_{pqrs}
\bra{pq}V\ket{rs}a^\dagger_p a^\dagger_qa_sa_r
\label{twoham}
\ee
The one-particle part  is written as 
\be
\sum_{pq}\bra{p}h\ket{q}N(a^\dagger_p a_q)+\sum_{i\in hole}\bra{i}h\ket{i}.
\label{eq:normen}
\ee
The two-particle part is rewritten as 
\be
\begin{split}
& \frac{1}{4}\sum_{pqrs} \bra{pq}V\ket{rs}a^\dagger_p a^\dagger_pa_sa_r=
 \frac{1}{4}\sum_{pqrs} \bra{pq}V\ket{rs}N(a^\dagger_p a^\dagger_q a_s a_r)
+\\ &\sum_{ipq}\bra{pi}V\ket{q i}N(a^\dagger_pa_r)+\frac{1}{2}\sum_{ij}\bra{ij}V\ket{ij}
\label{normH}.
\end{split}
\ee
For the entire calculation see Ref. \cite{sjefer}.
After the equal sign in Eq. \eqref{normH} the letters $p,q,r,$ and $s$ indicate 
both hole and particle states, while the letters $i$ and $j$   indicate hole states.
By combining the terms in equations \eqref{eq:normen} and \eqref{normH} we write
the entire Hamiltonian as
\be
\begin{split}
&H=\sum_{pq}\bra{p}h\ket{q}N(a^\dagger_p a_q)+\sum_i\bra{i}h\ket{i}+
\frac{1}{2}\sum_{ij}\bra{ij}V\ket{ij}+\\
& \frac{1}{4}\sum_{pqrs} \bra{pq}V\ket{rs}N(a^\dagger_p a^\dagger_q a_s a_r)
+\sum_{ipq}\bra{pi}V\ket{qi}N(a^\dagger_p a_q),
\label{normalham}
\end{split}
\ee
where $p,q,r,$ and $s$ still run over all states, $i$ and $j$ over hole states only.
