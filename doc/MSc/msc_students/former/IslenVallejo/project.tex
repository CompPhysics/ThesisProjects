\documentstyle[a4wide]{article}
\newcommand{\OP}[1]{{\bf\widehat{#1}}}

\newcommand{\be}{\begin{equation}}

\newcommand{\ee}{\end{equation}}

\begin{document}

\pagestyle{plain}

\section*{Thesis title: Effective parallelization of monte Carlo codes for quantum dots using Python}


The use of compiled low level languages such as fortran and C is a 
deeply-rooted tradition among computational scientists for doing numerical simulations. The extensive use of them is due, mainly, to their high performance. However, the increased demand for more flexibility has motived the deveploment of object oriented languages such as C++  and Fortran2003.  


In recent years, many scientifics and engineers 
have migrated to interpreted high level languages like matlab and maple. 
The abundant disponibility of documentation, clear and compact syntax, 
interactive execution of commands and the integration of simulation 
and visualizations make interpreted languages highly attractive. Python 
is an interpreted object-oriented language that shares with matlab 
many of its characteristics, but which is much more powerfull 
and flexible when equipped with tools for numerical simulations 
and visualization. Because python was designed to be extended with 
legacy code for efficiency, it is easier to interface 
it with  software written in C++, C and fortran than other environments. 

The numerical solution of quantum mechanical problems containing 
a large number of degrees of freedom requires the use of parallelization. 
A balance between computational efficienty, to get fast codes, 
and programming efficiency, to concentrate more in doing physics, 
is preferred. The Python package Pypar provides a wrapper to the message 
passing library MPI through a simple syntax and it has shown to 
give sufficiently good performance in other applications.

In this thesis we attempt to parallelize quantum mechanical 
codes using pypar. Concrete applications are the parallelization 
of  variational Monte Carlo sequential codes for quantum dots and 
the extension to diffusion monte Carlo for fermions. 
In the next section we give a brief description of the 
physics behind quantum dots and their potential for constructing quantum gates. Thereafter, we sketch the ideas behind the VMC approach to be used.


\subsection*{Introduction to quantum dots}

Semiconductor quantum dots are structures where
charge carriers are confined in all three spatial dimensions, 
the dot size being of the order of the Fermi wavelength 
in the host material, typically between  10 nm and  1 $\mu$m.
The confinement is usually achieved by electrical gating of a 
two-dimensional electron gas (2DEG), 
possibly combined with etching techniques. Precise control of the
number of electrons in the conduction band of a quantum dot 
(starting from zero) has been achieved in GaAs heterostructures. 
The electronic spectrum of typical quantum dots
can vary strongly when an external magnetic field is applied, 
since the magnetic length corresponding to typical 
laboratory fields  is comparable to typical dot sizes.
In coupled quantum dots Coulomb blockade effects, 
tunneling between neighboring dots, and magnetization 
have been observed as well as the formation of a
delocalized single-particle state. 


Quantum mechanical studies of such many-body systems are also
interesting per se. 
This thesis deals with the parallelization of serial variational and diffusion 
Monte Carlo code for quantum dots, from few electrons to many.  The diffusion 
Monte Carlo needs to be developed by the candidate.
Presently, Monte Carlo methods are
the only ones which allow us to solve, in principle exactly, 
systems with many interacting particles. 




The aims of this thesis are as follows

\begin{itemize}
\item Implement the parallelization of VMC codes in plain python and plain C++.
(the VMC calculations are done by another Master of Science student, Rune Albrigtsen).
\item Provide benchmarks of the above parallelization.
\item Move the slow part in python to C++.
\item Develop a diffusion Monte Carlo for quantum dots and 
extend the Python implementation of VMC to diffusion Monte Carlo. 
\item Provide benchmarks of the DMC codes.


\end{itemize}



The thesis is expected to be finished towards the end  of september 2009.


\end{document}
