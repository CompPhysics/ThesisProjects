\section{Implementation}
\subsection{General framework}
\begin{frame}{Implementing a QVMC simulator}
	\begin{scriptsize}
		\begin{alertblock}{How and why [later]?}
			\begin{enumerate}
				\item Programming style?: Object-orientation.
			
				\item Methodology?: Prototyping-test-extension/migration.
	% 		
				\item Programming languages?: Python/C++.
				
				\item Structure of the program?: Define classes [next slide], methods and flux of information (algorithm).
			\end{enumerate}
		\end{alertblock}
	\end{scriptsize}
	
  \begin{scriptsize}
    \begin{alertblock}{Basic class structure of a QVMC simulator}
      \begin{itemize}
        \item An administration class: {\color{blue}{VMC}}.
        \item A class computing energies: {\color{blue}{Energy}}.
        \item A class containing the trial wave function: {\color{blue}{PsiTrial}}.
        \item A class for administrate the configuration space: {\color{blue}{Particle}}.   
      \end{itemize}
    \end{alertblock}
  \end{scriptsize}
\end{frame}

\subsection{Implementation in Python}
\begin{frame}[fragile]{Quick design of a QVMC simulator in Python}
  \begin{Python}
    ...
    #Import some packages
    ...
    class VMC():
      def __init__(self, _dim, _np, _charge, ...,_parameters):
        ...
        particle = Particle(_dim, _np, _step) 
        psi      = Psi(_np, _dim, _parameters)
        energy   = Energy(..., particle, psi, _charge) 
        self.mc  = MonteCarlo(psi, _ncycles, particle, energy,...)

      def doVariationalLoop(self):
        ...
        for var in xrange(nVar):
          self.mc.doMonteCarloImportanceSampling()
          self.mc.psi.updateVariationalParameters()
  \end{Python}
\end{frame}


\begin{frame}[fragile]{Easy creation/manipulation of matrices in Python}
\begin{Python}

    ...
    class Particle(): 
      def __init__(self, _dim, _np, _step):
        # Initialize matrices for configuration space
        r_old = zeros((_np, _dim))
        ...

      def acceptMove(self, i):
        self.r_old[i,0:dim] = self.r_new[i,0:dim]

      ...
      def setTrialPositionsBF(self):
        dt = self.step
        r_old = dt*random.uniform(-0.5,0.5,size=np*dim)
        r_old.reshape(np,dim)
      ...
  \end{Python}
\end{frame}

% % % % 
% % % % \begin{frame}[fragile]
% % % %   \begin{Python}
% % % %     ...
% % % %     class Psi:
% % % %       def __init__(self, _np, _dim, _parameters):
% % % %         ...
% % % %         self.cusp = zeros((_np*(_np-1)/2))    # Cusp factors
% % % %         self.setCusp()
% % % % 
% % % %       # Define single particle wave functions
% % % %       def phi1s(self, rij):
% % % %         return exp(-self.parameters[0]*rij)
% % % %         
% % % %       def phi2s(self, rij):
% % % %         return (1.0 -self.parameters[0]*rij/2.0) \
% % % %               *exp(-self.parameters[0]*rij/2.0)
% % % %               
% % % %       def getPsiTrial(self, r):
% % % %         return self.getModelWaveFunctionHe(r) \
% % % %               *self.getCorrelationFactor(r)
% % % %       ...
% % % %   \end{Python}
% % % % \end{frame}


% % % 
% % % \begin{frame}[fragile]
% % %   \begin{Python}
% % %     ...
% % %     class Energy: 
% % %       def __init__(self, dim, np, particle, psi,...):
% % %         self.cumEnergy = zeros(maxVar)  #Cumulate local energy
% % %         self.cumEnergy2= zeros(maxVar)  #Cumulant local energy squared
% % %         ...
% % % 
% % %       def getLocalEnergy(self, wfold):
% % %         EL     = self.getKineticEnergy(wfold) \
% % %               + self.getPotentialEnergy()
% % %         self.E  += EL
% % %         self.E2 += EL*EL 
% % %       ...
% % %   \end{Python}
% % % \end{frame}


\begin{frame}[fragile]{Calling the code}
  \begin{Python}
    import sys
    from VMC import *

    # Set parameters of simulation
    nsd = 3        # Number of spatial dimensions
    nVar= 10       # Number of variations (optimization method)
    nmc = 10000    # Number of monte Carlo cycles
    nel = 2        # Number of electrons
    Z   = 2.0      # Nuclear charge
    ...
    vmc = VMC(nsd, nel, Z,..., nmc, dt, nVar, varPar)
    vmc.doVariationalLoop()
    vmc.mc.energy.printResults()
  \end{Python}
\end{frame}


\begin{frame}{What do you gain from using Python?}
   \begin{scriptsize}
% % %     \begin{columns}
% % %       \column{5cm}
% % %       \begin{scriptsize}
% % %       \begin{alertblock}{Object-oriented programming?}
% % %         \begin{itemize}
% % %           \item More flexibility.
% % %             \begin{itemize}
% % %               \scriptsize
% % %               \item Encapsulation (Public interface and private implementation.).
% % %               \item Polymorfims (Objects acts according to the context.).
% % %               \item Inheritance (specialization).
% % %             \end{itemize}
% % %           
% % %           \item Divide and konqueror approach.
% % %         \end{itemize}
% % %       \end{alertblock}
% % %       \end{scriptsize}
% % %     

% %       \column{5cm}
      \begin{alertblock}{High level language?}
        \begin{itemize}
          \item Clear and compact syntax.
          \item Support all the major program styles. 
          \item Runs on all major platforms.
          \item Free, open source.
          \item Comprehensive standard library.
          \item Huge collection of free modules on the web.
          \item Good support for scientific computing.
        \end{itemize}
      \end{alertblock}
% %     \end{columns}
  \end{scriptsize}
  
  \begin{alertblock}{However...}
%     \begin{enumerate}
     How well performs Python with respect to C++?
%      \item How quick is Python with respect to C++?
%     \end{enumerate}
  \end{alertblock}
\end{frame}


% % % % % 
% % % % % \begin{frame}{Comparing Python to C++}
% % % % %   \begin{figure}
% % % % %     \begin{tabular}{cc}
% % % % %       \scalebox{0.4}{\begin{tikzpicture}[gnuplot]
%% generated with GNUPLOT 4.2p5  (Lua 5.1.4; terminal rev. 81, script rev. 88)
%% Mon Sep  7 13:53:42 2009
\color{gp_lt_color_b}
\gpsetlinetype{gp_lt_border}
\gpsetlinewidth{1.00}
\draw[gp path] (2.056,0.985)--(2.236,0.985);
\draw[gp path] (12.040,0.985)--(11.860,0.985);
\node[gp node right] at (1.872,0.985) {-2.88};
\draw[gp path] (2.056,1.108)--(2.146,1.108);
\draw[gp path] (12.040,1.108)--(11.950,1.108);
\draw[gp path] (2.056,1.232)--(2.146,1.232);
\draw[gp path] (12.040,1.232)--(11.950,1.232);
\draw[gp path] (2.056,1.355)--(2.146,1.355);
\draw[gp path] (12.040,1.355)--(11.950,1.355);
\draw[gp path] (2.056,1.478)--(2.146,1.478);
\draw[gp path] (12.040,1.478)--(11.950,1.478);
\draw[gp path] (2.056,1.601)--(2.146,1.601);
\draw[gp path] (12.040,1.601)--(11.950,1.601);
\draw[gp path] (2.056,1.725)--(2.146,1.725);
\draw[gp path] (12.040,1.725)--(11.950,1.725);
\draw[gp path] (2.056,1.848)--(2.146,1.848);
\draw[gp path] (12.040,1.848)--(11.950,1.848);
\draw[gp path] (2.056,1.971)--(2.146,1.971);
\draw[gp path] (12.040,1.971)--(11.950,1.971);
\draw[gp path] (2.056,2.095)--(2.146,2.095);
\draw[gp path] (12.040,2.095)--(11.950,2.095);
\draw[gp path] (2.056,2.218)--(2.236,2.218);
\draw[gp path] (12.040,2.218)--(11.860,2.218);
\node[gp node right] at (1.872,2.218) {-2.87};
\draw[gp path] (2.056,2.341)--(2.146,2.341);
\draw[gp path] (12.040,2.341)--(11.950,2.341);
\draw[gp path] (2.056,2.464)--(2.146,2.464);
\draw[gp path] (12.040,2.464)--(11.950,2.464);
\draw[gp path] (2.056,2.588)--(2.146,2.588);
\draw[gp path] (12.040,2.588)--(11.950,2.588);
\draw[gp path] (2.056,2.711)--(2.146,2.711);
\draw[gp path] (12.040,2.711)--(11.950,2.711);
\draw[gp path] (2.056,2.834)--(2.146,2.834);
\draw[gp path] (12.040,2.834)--(11.950,2.834);
\draw[gp path] (2.056,2.958)--(2.146,2.958);
\draw[gp path] (12.040,2.958)--(11.950,2.958);
\draw[gp path] (2.056,3.081)--(2.146,3.081);
\draw[gp path] (12.040,3.081)--(11.950,3.081);
\draw[gp path] (2.056,3.204)--(2.146,3.204);
\draw[gp path] (12.040,3.204)--(11.950,3.204);
\draw[gp path] (2.056,3.327)--(2.146,3.327);
\draw[gp path] (12.040,3.327)--(11.950,3.327);
\draw[gp path] (2.056,3.451)--(2.236,3.451);
\draw[gp path] (12.040,3.451)--(11.860,3.451);
\node[gp node right] at (1.872,3.451) {-2.86};
\draw[gp path] (2.056,3.574)--(2.146,3.574);
\draw[gp path] (12.040,3.574)--(11.950,3.574);
\draw[gp path] (2.056,3.697)--(2.146,3.697);
\draw[gp path] (12.040,3.697)--(11.950,3.697);
\draw[gp path] (2.056,3.821)--(2.146,3.821);
\draw[gp path] (12.040,3.821)--(11.950,3.821);
\draw[gp path] (2.056,3.944)--(2.146,3.944);
\draw[gp path] (12.040,3.944)--(11.950,3.944);
\draw[gp path] (2.056,4.067)--(2.146,4.067);
\draw[gp path] (12.040,4.067)--(11.950,4.067);
\draw[gp path] (2.056,4.190)--(2.146,4.190);
\draw[gp path] (12.040,4.190)--(11.950,4.190);
\draw[gp path] (2.056,4.314)--(2.146,4.314);
\draw[gp path] (12.040,4.314)--(11.950,4.314);
\draw[gp path] (2.056,4.437)--(2.146,4.437);
\draw[gp path] (12.040,4.437)--(11.950,4.437);
\draw[gp path] (2.056,4.560)--(2.146,4.560);
\draw[gp path] (12.040,4.560)--(11.950,4.560);
\draw[gp path] (2.056,4.683)--(2.236,4.683);
\draw[gp path] (12.040,4.683)--(11.860,4.683);
\node[gp node right] at (1.872,4.683) {-2.85};
\draw[gp path] (2.056,4.807)--(2.146,4.807);
\draw[gp path] (12.040,4.807)--(11.950,4.807);
\draw[gp path] (2.056,4.930)--(2.146,4.930);
\draw[gp path] (12.040,4.930)--(11.950,4.930);
\draw[gp path] (2.056,5.053)--(2.146,5.053);
\draw[gp path] (12.040,5.053)--(11.950,5.053);
\draw[gp path] (2.056,5.177)--(2.146,5.177);
\draw[gp path] (12.040,5.177)--(11.950,5.177);
\draw[gp path] (2.056,5.300)--(2.146,5.300);
\draw[gp path] (12.040,5.300)--(11.950,5.300);
\draw[gp path] (2.056,5.423)--(2.146,5.423);
\draw[gp path] (12.040,5.423)--(11.950,5.423);
\draw[gp path] (2.056,5.546)--(2.146,5.546);
\draw[gp path] (12.040,5.546)--(11.950,5.546);
\draw[gp path] (2.056,5.670)--(2.146,5.670);
\draw[gp path] (12.040,5.670)--(11.950,5.670);
\draw[gp path] (2.056,5.793)--(2.146,5.793);
\draw[gp path] (12.040,5.793)--(11.950,5.793);
\draw[gp path] (2.056,5.916)--(2.236,5.916);
\draw[gp path] (12.040,5.916)--(11.860,5.916);
\node[gp node right] at (1.872,5.916) {-2.84};
\draw[gp path] (2.056,6.040)--(2.146,6.040);
\draw[gp path] (12.040,6.040)--(11.950,6.040);
\draw[gp path] (2.056,6.163)--(2.146,6.163);
\draw[gp path] (12.040,6.163)--(11.950,6.163);
\draw[gp path] (2.056,6.286)--(2.146,6.286);
\draw[gp path] (12.040,6.286)--(11.950,6.286);
\draw[gp path] (2.056,6.409)--(2.146,6.409);
\draw[gp path] (12.040,6.409)--(11.950,6.409);
\draw[gp path] (2.056,6.533)--(2.146,6.533);
\draw[gp path] (12.040,6.533)--(11.950,6.533);
\draw[gp path] (2.056,6.656)--(2.146,6.656);
\draw[gp path] (12.040,6.656)--(11.950,6.656);
\draw[gp path] (2.056,6.779)--(2.146,6.779);
\draw[gp path] (12.040,6.779)--(11.950,6.779);
\draw[gp path] (2.056,6.903)--(2.146,6.903);
\draw[gp path] (12.040,6.903)--(11.950,6.903);
\draw[gp path] (2.056,7.026)--(2.146,7.026);
\draw[gp path] (12.040,7.026)--(11.950,7.026);
\draw[gp path] (2.056,7.149)--(2.236,7.149);
\draw[gp path] (12.040,7.149)--(11.860,7.149);
\node[gp node right] at (1.872,7.149) {-2.83};
\draw[gp path] (2.056,7.272)--(2.146,7.272);
\draw[gp path] (12.040,7.272)--(11.950,7.272);
\draw[gp path] (2.056,7.396)--(2.146,7.396);
\draw[gp path] (12.040,7.396)--(11.950,7.396);
\draw[gp path] (2.056,7.519)--(2.146,7.519);
\draw[gp path] (12.040,7.519)--(11.950,7.519);
\draw[gp path] (2.056,7.642)--(2.146,7.642);
\draw[gp path] (12.040,7.642)--(11.950,7.642);
\draw[gp path] (2.056,7.766)--(2.146,7.766);
\draw[gp path] (12.040,7.766)--(11.950,7.766);
\draw[gp path] (2.056,7.889)--(2.146,7.889);
\draw[gp path] (12.040,7.889)--(11.950,7.889);
\draw[gp path] (2.056,8.012)--(2.146,8.012);
\draw[gp path] (12.040,8.012)--(11.950,8.012);
\draw[gp path] (2.056,8.135)--(2.146,8.135);
\draw[gp path] (12.040,8.135)--(11.950,8.135);
\draw[gp path] (2.056,8.259)--(2.146,8.259);
\draw[gp path] (12.040,8.259)--(11.950,8.259);
\draw[gp path] (2.056,8.382)--(2.236,8.382);
\draw[gp path] (12.040,8.382)--(11.860,8.382);
\node[gp node right] at (1.872,8.382) {-2.82};
\draw[gp path] (2.056,0.985)--(2.056,1.165);
\draw[gp path] (2.056,8.382)--(2.056,8.202);
\node[gp node center] at (2.056,0.677) {0};
\draw[gp path] (2.222,0.985)--(2.222,1.075);
\draw[gp path] (2.222,8.382)--(2.222,8.292);
\draw[gp path] (2.389,0.985)--(2.389,1.075);
\draw[gp path] (2.389,8.382)--(2.389,8.292);
\draw[gp path] (2.555,0.985)--(2.555,1.075);
\draw[gp path] (2.555,8.382)--(2.555,8.292);
\draw[gp path] (2.722,0.985)--(2.722,1.075);
\draw[gp path] (2.722,8.382)--(2.722,8.292);
\draw[gp path] (2.888,0.985)--(2.888,1.075);
\draw[gp path] (2.888,8.382)--(2.888,8.292);
\draw[gp path] (3.054,0.985)--(3.054,1.075);
\draw[gp path] (3.054,8.382)--(3.054,8.292);
\draw[gp path] (3.221,0.985)--(3.221,1.075);
\draw[gp path] (3.221,8.382)--(3.221,8.292);
\draw[gp path] (3.387,0.985)--(3.387,1.075);
\draw[gp path] (3.387,8.382)--(3.387,8.292);
\draw[gp path] (3.554,0.985)--(3.554,1.075);
\draw[gp path] (3.554,8.382)--(3.554,8.292);
\draw[gp path] (3.720,0.985)--(3.720,1.075);
\draw[gp path] (3.720,8.382)--(3.720,8.292);
\draw[gp path] (3.720,0.985)--(3.720,1.165);
\draw[gp path] (3.720,8.382)--(3.720,8.202);
\node[gp node center] at (3.720,0.677) {0.2};
\draw[gp path] (3.886,0.985)--(3.886,1.075);
\draw[gp path] (3.886,8.382)--(3.886,8.292);
\draw[gp path] (4.053,0.985)--(4.053,1.075);
\draw[gp path] (4.053,8.382)--(4.053,8.292);
\draw[gp path] (4.219,0.985)--(4.219,1.075);
\draw[gp path] (4.219,8.382)--(4.219,8.292);
\draw[gp path] (4.386,0.985)--(4.386,1.075);
\draw[gp path] (4.386,8.382)--(4.386,8.292);
\draw[gp path] (4.552,0.985)--(4.552,1.075);
\draw[gp path] (4.552,8.382)--(4.552,8.292);
\draw[gp path] (4.718,0.985)--(4.718,1.075);
\draw[gp path] (4.718,8.382)--(4.718,8.292);
\draw[gp path] (4.885,0.985)--(4.885,1.075);
\draw[gp path] (4.885,8.382)--(4.885,8.292);
\draw[gp path] (5.051,0.985)--(5.051,1.075);
\draw[gp path] (5.051,8.382)--(5.051,8.292);
\draw[gp path] (5.218,0.985)--(5.218,1.075);
\draw[gp path] (5.218,8.382)--(5.218,8.292);
\draw[gp path] (5.384,0.985)--(5.384,1.075);
\draw[gp path] (5.384,8.382)--(5.384,8.292);
\draw[gp path] (5.384,0.985)--(5.384,1.165);
\draw[gp path] (5.384,8.382)--(5.384,8.202);
\node[gp node center] at (5.384,0.677) {0.4};
\draw[gp path] (5.550,0.985)--(5.550,1.075);
\draw[gp path] (5.550,8.382)--(5.550,8.292);
\draw[gp path] (5.717,0.985)--(5.717,1.075);
\draw[gp path] (5.717,8.382)--(5.717,8.292);
\draw[gp path] (5.883,0.985)--(5.883,1.075);
\draw[gp path] (5.883,8.382)--(5.883,8.292);
\draw[gp path] (6.050,0.985)--(6.050,1.075);
\draw[gp path] (6.050,8.382)--(6.050,8.292);
\draw[gp path] (6.216,0.985)--(6.216,1.075);
\draw[gp path] (6.216,8.382)--(6.216,8.292);
\draw[gp path] (6.382,0.985)--(6.382,1.075);
\draw[gp path] (6.382,8.382)--(6.382,8.292);
\draw[gp path] (6.549,0.985)--(6.549,1.075);
\draw[gp path] (6.549,8.382)--(6.549,8.292);
\draw[gp path] (6.715,0.985)--(6.715,1.075);
\draw[gp path] (6.715,8.382)--(6.715,8.292);
\draw[gp path] (6.882,0.985)--(6.882,1.075);
\draw[gp path] (6.882,8.382)--(6.882,8.292);
\draw[gp path] (7.048,0.985)--(7.048,1.075);
\draw[gp path] (7.048,8.382)--(7.048,8.292);
\draw[gp path] (7.048,0.985)--(7.048,1.165);
\draw[gp path] (7.048,8.382)--(7.048,8.202);
\node[gp node center] at (7.048,0.677) {0.6};
\draw[gp path] (7.214,0.985)--(7.214,1.075);
\draw[gp path] (7.214,8.382)--(7.214,8.292);
\draw[gp path] (7.381,0.985)--(7.381,1.075);
\draw[gp path] (7.381,8.382)--(7.381,8.292);
\draw[gp path] (7.547,0.985)--(7.547,1.075);
\draw[gp path] (7.547,8.382)--(7.547,8.292);
\draw[gp path] (7.714,0.985)--(7.714,1.075);
\draw[gp path] (7.714,8.382)--(7.714,8.292);
\draw[gp path] (7.880,0.985)--(7.880,1.075);
\draw[gp path] (7.880,8.382)--(7.880,8.292);
\draw[gp path] (8.046,0.985)--(8.046,1.075);
\draw[gp path] (8.046,8.382)--(8.046,8.292);
\draw[gp path] (8.213,0.985)--(8.213,1.075);
\draw[gp path] (8.213,8.382)--(8.213,8.292);
\draw[gp path] (8.379,0.985)--(8.379,1.075);
\draw[gp path] (8.379,8.382)--(8.379,8.292);
\draw[gp path] (8.546,0.985)--(8.546,1.075);
\draw[gp path] (8.546,8.382)--(8.546,8.292);
\draw[gp path] (8.712,0.985)--(8.712,1.075);
\draw[gp path] (8.712,8.382)--(8.712,8.292);
\draw[gp path] (8.712,0.985)--(8.712,1.165);
\draw[gp path] (8.712,8.382)--(8.712,8.202);
\node[gp node center] at (8.712,0.677) {0.8};
\draw[gp path] (8.878,0.985)--(8.878,1.075);
\draw[gp path] (8.878,8.382)--(8.878,8.292);
\draw[gp path] (9.045,0.985)--(9.045,1.075);
\draw[gp path] (9.045,8.382)--(9.045,8.292);
\draw[gp path] (9.211,0.985)--(9.211,1.075);
\draw[gp path] (9.211,8.382)--(9.211,8.292);
\draw[gp path] (9.378,0.985)--(9.378,1.075);
\draw[gp path] (9.378,8.382)--(9.378,8.292);
\draw[gp path] (9.544,0.985)--(9.544,1.075);
\draw[gp path] (9.544,8.382)--(9.544,8.292);
\draw[gp path] (9.710,0.985)--(9.710,1.075);
\draw[gp path] (9.710,8.382)--(9.710,8.292);
\draw[gp path] (9.877,0.985)--(9.877,1.075);
\draw[gp path] (9.877,8.382)--(9.877,8.292);
\draw[gp path] (10.043,0.985)--(10.043,1.075);
\draw[gp path] (10.043,8.382)--(10.043,8.292);
\draw[gp path] (10.210,0.985)--(10.210,1.075);
\draw[gp path] (10.210,8.382)--(10.210,8.292);
\draw[gp path] (10.376,0.985)--(10.376,1.075);
\draw[gp path] (10.376,8.382)--(10.376,8.292);
\draw[gp path] (10.376,0.985)--(10.376,1.165);
\draw[gp path] (10.376,8.382)--(10.376,8.202);
\node[gp node center] at (10.376,0.677) {1};
\draw[gp path] (10.542,0.985)--(10.542,1.075);
\draw[gp path] (10.542,8.382)--(10.542,8.292);
\draw[gp path] (10.709,0.985)--(10.709,1.075);
\draw[gp path] (10.709,8.382)--(10.709,8.292);
\draw[gp path] (10.875,0.985)--(10.875,1.075);
\draw[gp path] (10.875,8.382)--(10.875,8.292);
\draw[gp path] (11.042,0.985)--(11.042,1.075);
\draw[gp path] (11.042,8.382)--(11.042,8.292);
\draw[gp path] (11.208,0.985)--(11.208,1.075);
\draw[gp path] (11.208,8.382)--(11.208,8.292);
\draw[gp path] (11.374,0.985)--(11.374,1.075);
\draw[gp path] (11.374,8.382)--(11.374,8.292);
\draw[gp path] (11.541,0.985)--(11.541,1.075);
\draw[gp path] (11.541,8.382)--(11.541,8.292);
\draw[gp path] (11.707,0.985)--(11.707,1.075);
\draw[gp path] (11.707,8.382)--(11.707,8.292);
\draw[gp path] (11.874,0.985)--(11.874,1.075);
\draw[gp path] (11.874,8.382)--(11.874,8.292);
\draw[gp path] (12.040,0.985)--(12.040,1.075);
\draw[gp path] (12.040,8.382)--(12.040,8.292);
\draw[gp path] (12.040,0.985)--(12.040,1.165);
\draw[gp path] (12.040,8.382)--(12.040,8.202);
\node[gp node center] at (12.040,0.677) {1.2};
\draw[gp path] (2.056,8.382)--(2.056,0.985)--(12.040,0.985)--(12.040,8.382)--cycle;
\node[gp node center,rotate=90] at (0.614,4.683) {Energy $\langle E \rangle$, au};
\node[gp node center] at (7.048,0.215) {Variantional parameter, $\beta$};
\color{gp_lt_color0}
\gpsetlinetype{gp_lt_plot0}
\draw[gp path] (2.306,8.325)--(2.306,8.381);
\draw[gp path] (2.216,8.325)--(2.396,8.325);
\draw[gp path] (2.216,8.381)--(2.396,8.381);
\draw[gp path] (2.555,6.830)--(2.555,6.884);
\draw[gp path] (2.465,6.830)--(2.645,6.830);
\draw[gp path] (2.465,6.884)--(2.645,6.884);
\draw[gp path] (2.805,5.728)--(2.805,5.782);
\draw[gp path] (2.715,5.728)--(2.895,5.728);
\draw[gp path] (2.715,5.782)--(2.895,5.782);
\draw[gp path] (3.054,4.680)--(3.054,4.732);
\draw[gp path] (2.964,4.680)--(3.144,4.680);
\draw[gp path] (2.964,4.732)--(3.144,4.732);
\draw[gp path] (3.304,4.193)--(3.304,4.249);
\draw[gp path] (3.214,4.193)--(3.394,4.193);
\draw[gp path] (3.214,4.249)--(3.394,4.249);
\draw[gp path] (3.554,3.685)--(3.554,3.736);
\draw[gp path] (3.464,3.685)--(3.644,3.685);
\draw[gp path] (3.464,3.736)--(3.644,3.736);
\draw[gp path] (3.803,3.305)--(3.803,3.354);
\draw[gp path] (3.713,3.305)--(3.893,3.305);
\draw[gp path] (3.713,3.354)--(3.893,3.354);
\draw[gp path] (4.053,2.885)--(4.053,2.937);
\draw[gp path] (3.963,2.885)--(4.143,2.885);
\draw[gp path] (3.963,2.937)--(4.143,2.937);
\draw[gp path] (4.302,2.565)--(4.302,2.616);
\draw[gp path] (4.212,2.565)--(4.392,2.565);
\draw[gp path] (4.212,2.616)--(4.392,2.616);
\draw[gp path] (4.552,2.396)--(4.552,2.446);
\draw[gp path] (4.462,2.396)--(4.642,2.396);
\draw[gp path] (4.462,2.446)--(4.642,2.446);
\draw[gp path] (4.802,2.219)--(4.802,2.268);
\draw[gp path] (4.712,2.219)--(4.892,2.219);
\draw[gp path] (4.712,2.268)--(4.892,2.268);
\draw[gp path] (5.051,2.025)--(5.051,2.074);
\draw[gp path] (4.961,2.025)--(5.141,2.025);
\draw[gp path] (4.961,2.074)--(5.141,2.074);
\draw[gp path] (5.301,1.927)--(5.301,1.976);
\draw[gp path] (5.211,1.927)--(5.391,1.927);
\draw[gp path] (5.211,1.976)--(5.391,1.976);
\draw[gp path] (5.550,1.893)--(5.550,1.944);
\draw[gp path] (5.460,1.893)--(5.640,1.893);
\draw[gp path] (5.460,1.944)--(5.640,1.944);
\draw[gp path] (5.800,1.810)--(5.800,1.859);
\draw[gp path] (5.710,1.810)--(5.890,1.810);
\draw[gp path] (5.710,1.859)--(5.890,1.859);
\draw[gp path] (6.050,1.695)--(6.050,1.747);
\draw[gp path] (5.960,1.695)--(6.140,1.695);
\draw[gp path] (5.960,1.747)--(6.140,1.747);
\draw[gp path] (6.299,1.743)--(6.299,1.793);
\draw[gp path] (6.209,1.743)--(6.389,1.743);
\draw[gp path] (6.209,1.793)--(6.389,1.793);
\draw[gp path] (6.549,1.613)--(6.549,1.664);
\draw[gp path] (6.459,1.613)--(6.639,1.613);
\draw[gp path] (6.459,1.664)--(6.639,1.664);
\draw[gp path] (6.798,1.691)--(6.798,1.740);
\draw[gp path] (6.708,1.691)--(6.888,1.691);
\draw[gp path] (6.708,1.740)--(6.888,1.740);
\draw[gp path] (7.048,1.617)--(7.048,1.666);
\draw[gp path] (6.958,1.617)--(7.138,1.617);
\draw[gp path] (6.958,1.666)--(7.138,1.666);
\draw[gp path] (7.298,1.346)--(7.298,1.489);
\draw[gp path] (7.208,1.346)--(7.388,1.346);
\draw[gp path] (7.208,1.489)--(7.388,1.489);
\draw[gp path] (7.547,1.695)--(7.547,1.753);
\draw[gp path] (7.457,1.695)--(7.637,1.695);
\draw[gp path] (7.457,1.753)--(7.637,1.753);
\draw[gp path] (7.797,1.686)--(7.797,1.736);
\draw[gp path] (7.707,1.686)--(7.887,1.686);
\draw[gp path] (7.707,1.736)--(7.887,1.736);
\draw[gp path] (8.046,1.665)--(8.046,1.715);
\draw[gp path] (7.956,1.665)--(8.136,1.665);
\draw[gp path] (7.956,1.715)--(8.136,1.715);
\draw[gp path] (8.296,1.646)--(8.296,1.695);
\draw[gp path] (8.206,1.646)--(8.386,1.646);
\draw[gp path] (8.206,1.695)--(8.386,1.695);
\draw[gp path] (8.546,1.669)--(8.546,1.720);
\draw[gp path] (8.456,1.669)--(8.636,1.669);
\draw[gp path] (8.456,1.720)--(8.636,1.720);
\draw[gp path] (8.795,1.717)--(8.795,1.767);
\draw[gp path] (8.705,1.717)--(8.885,1.717);
\draw[gp path] (8.705,1.767)--(8.885,1.767);
\draw[gp path] (9.045,1.790)--(9.045,1.839);
\draw[gp path] (8.955,1.790)--(9.135,1.790);
\draw[gp path] (8.955,1.839)--(9.135,1.839);
\draw[gp path] (9.294,1.846)--(9.294,1.895);
\draw[gp path] (9.204,1.846)--(9.384,1.846);
\draw[gp path] (9.204,1.895)--(9.384,1.895);
\draw[gp path] (9.544,1.840)--(9.544,1.890);
\draw[gp path] (9.454,1.840)--(9.634,1.840);
\draw[gp path] (9.454,1.890)--(9.634,1.890);
\draw[gp path] (9.794,1.859)--(9.794,1.911);
\draw[gp path] (9.704,1.859)--(9.884,1.859);
\draw[gp path] (9.704,1.911)--(9.884,1.911);
\draw[gp path] (10.043,2.001)--(10.043,2.051);
\draw[gp path] (9.953,2.001)--(10.133,2.001);
\draw[gp path] (9.953,2.051)--(10.133,2.051);
\draw[gp path] (10.293,1.944)--(10.293,1.995);
\draw[gp path] (10.203,1.944)--(10.383,1.944);
\draw[gp path] (10.203,1.995)--(10.383,1.995);
\draw[gp path] (10.542,1.934)--(10.542,1.984);
\draw[gp path] (10.452,1.934)--(10.632,1.934);
\draw[gp path] (10.452,1.984)--(10.632,1.984);
\draw[gp path] (10.792,2.004)--(10.792,2.054);
\draw[gp path] (10.702,2.004)--(10.882,2.004);
\draw[gp path] (10.702,2.054)--(10.882,2.054);
\draw[gp path] (11.042,2.001)--(11.042,2.053);
\draw[gp path] (10.952,2.001)--(11.132,2.001);
\draw[gp path] (10.952,2.053)--(11.132,2.053);
\draw[gp path] (11.291,2.000)--(11.291,2.050);
\draw[gp path] (11.201,2.000)--(11.381,2.000);
\draw[gp path] (11.201,2.050)--(11.381,2.050);
\draw[gp path] (11.541,2.166)--(11.541,2.216);
\draw[gp path] (11.451,2.166)--(11.631,2.166);
\draw[gp path] (11.451,2.216)--(11.631,2.216);
\draw[gp path] (11.790,2.163)--(11.790,2.214);
\draw[gp path] (11.700,2.163)--(11.880,2.163);
\draw[gp path] (11.700,2.214)--(11.880,2.214);
\draw[gp path] (12.040,2.239)--(12.040,2.291);
\draw[gp path] (11.950,2.239)--(12.130,2.239);
\draw[gp path] (11.950,2.291)--(12.130,2.291);
\gpsetpointsize{4.00}
\gppoint{gp_mark1}{(2.306,8.353)}
\gppoint{gp_mark1}{(2.555,6.857)}
\gppoint{gp_mark1}{(2.805,5.755)}
\gppoint{gp_mark1}{(3.054,4.706)}
\gppoint{gp_mark1}{(3.304,4.221)}
\gppoint{gp_mark1}{(3.554,3.711)}
\gppoint{gp_mark1}{(3.803,3.330)}
\gppoint{gp_mark1}{(4.053,2.911)}
\gppoint{gp_mark1}{(4.302,2.591)}
\gppoint{gp_mark1}{(4.552,2.421)}
\gppoint{gp_mark1}{(4.802,2.244)}
\gppoint{gp_mark1}{(5.051,2.049)}
\gppoint{gp_mark1}{(5.301,1.952)}
\gppoint{gp_mark1}{(5.550,1.918)}
\gppoint{gp_mark1}{(5.800,1.834)}
\gppoint{gp_mark1}{(6.050,1.721)}
\gppoint{gp_mark1}{(6.299,1.768)}
\gppoint{gp_mark1}{(6.549,1.639)}
\gppoint{gp_mark1}{(6.798,1.715)}
\gppoint{gp_mark1}{(7.048,1.641)}
\gppoint{gp_mark1}{(7.298,1.418)}
\gppoint{gp_mark1}{(7.547,1.724)}
\gppoint{gp_mark1}{(7.797,1.711)}
\gppoint{gp_mark1}{(8.046,1.690)}
\gppoint{gp_mark1}{(8.296,1.671)}
\gppoint{gp_mark1}{(8.546,1.694)}
\gppoint{gp_mark1}{(8.795,1.742)}
\gppoint{gp_mark1}{(9.045,1.814)}
\gppoint{gp_mark1}{(9.294,1.870)}
\gppoint{gp_mark1}{(9.544,1.865)}
\gppoint{gp_mark1}{(9.794,1.885)}
\gppoint{gp_mark1}{(10.043,2.026)}
\gppoint{gp_mark1}{(10.293,1.970)}
\gppoint{gp_mark1}{(10.542,1.959)}
\gppoint{gp_mark1}{(10.792,2.029)}
\gppoint{gp_mark1}{(11.042,2.027)}
\gppoint{gp_mark1}{(11.291,2.025)}
\gppoint{gp_mark1}{(11.541,2.191)}
\gppoint{gp_mark1}{(11.790,2.189)}
\gppoint{gp_mark1}{(12.040,2.265)}
\color{gp_lt_color3}
\gpsetlinetype{gp_lt_plot3}
\gpsetlinewidth{4.00}
\draw[gp path] (2.306,8.353)--(2.555,6.857)--(2.805,5.755)--(3.054,4.706)--(3.304,4.221)%
  --(3.554,3.711)--(3.803,3.330)--(4.053,2.911)--(4.302,2.591)--(4.552,2.421)--(4.802,2.244)%
  --(5.051,2.049)--(5.301,1.952)--(5.550,1.918)--(5.800,1.834)--(6.050,1.721)--(6.299,1.768)%
  --(6.549,1.639)--(6.798,1.715)--(7.048,1.641)--(7.298,1.418)--(7.547,1.724)--(7.797,1.711)%
  --(8.046,1.690)--(8.296,1.671)--(8.546,1.694)--(8.795,1.742)--(9.045,1.814)--(9.294,1.870)%
  --(9.544,1.865)--(9.794,1.885)--(10.043,2.026)--(10.293,1.970)--(10.542,1.959)--(10.792,2.029)%
  --(11.042,2.027)--(11.291,2.025)--(11.541,2.191)--(11.790,2.189)--(12.040,2.265);
\color{gp_lt_color_b}
\gpsetlinetype{gp_lt_border}
\gpsetlinewidth{1.00}
\draw[gp path] (2.056,8.382)--(2.056,0.985)--(12.040,0.985)--(12.040,8.382)--cycle;
%% coordinates of the plot area
\coordinate (gpbb south west 1) at (2.056,0.985);
\coordinate (gpbb south east 1) at (12.040,0.985);
\coordinate (gpbb north east 1) at (12.040,8.382);
\coordinate (gpbb north west 1) at (2.056,8.382);
\coordinate (gpbb north 1) at (7.048,8.382);
\coordinate (gpbb south 1) at (7.048,0.985);
\coordinate (gpbb west 1) at (2.056,4.684);
\coordinate (gpbb east 1) at (12.040,4.684);
\draw[gp path] (6.748,4.116)--(6.838,4.116);
\draw[gp path] (11.415,4.116)--(11.325,4.116);
\draw[gp path] (6.748,4.540)--(6.838,4.540);
\draw[gp path] (11.415,4.540)--(11.325,4.540);
\draw[gp path] (6.748,4.964)--(6.838,4.964);
\draw[gp path] (11.415,4.964)--(11.325,4.964);
\draw[gp path] (6.748,5.388)--(6.838,5.388);
\draw[gp path] (11.415,5.388)--(11.325,5.388);
\draw[gp path] (6.748,5.812)--(6.838,5.812);
\draw[gp path] (11.415,5.812)--(11.325,5.812);
\draw[gp path] (6.748,6.235)--(6.838,6.235);
\draw[gp path] (11.415,6.235)--(11.325,6.235);
\draw[gp path] (6.748,6.659)--(6.838,6.659);
\draw[gp path] (11.415,6.659)--(11.325,6.659);
\draw[gp path] (6.748,7.083)--(6.838,7.083);
\draw[gp path] (11.415,7.083)--(11.325,7.083);
\draw[gp path] (6.748,7.507)--(6.928,7.507);
\draw[gp path] (11.415,7.507)--(11.235,7.507);
\node[gp node right] at (6.564,7.507) {-2.87};
\draw[gp path] (6.833,4.116)--(6.833,4.206);
\draw[gp path] (6.833,7.507)--(6.833,7.417);
\draw[gp path] (7.172,4.116)--(7.172,4.206);
\draw[gp path] (7.172,7.507)--(7.172,7.417);
\draw[gp path] (7.172,4.116)--(7.172,4.296);
\draw[gp path] (7.172,7.507)--(7.172,7.327);
\node[gp node center,font=,20] at (7.172,3.808) {0.4};
\draw[gp path] (7.512,4.116)--(7.512,4.206);
\draw[gp path] (7.512,7.507)--(7.512,7.417);
\draw[gp path] (7.851,4.116)--(7.851,4.206);
\draw[gp path] (7.851,7.507)--(7.851,7.417);
\draw[gp path] (8.191,4.116)--(8.191,4.206);
\draw[gp path] (8.191,7.507)--(8.191,7.417);
\draw[gp path] (8.530,4.116)--(8.530,4.206);
\draw[gp path] (8.530,7.507)--(8.530,7.417);
\draw[gp path] (8.869,4.116)--(8.869,4.206);
\draw[gp path] (8.869,7.507)--(8.869,7.417);
\draw[gp path] (9.209,4.116)--(9.209,4.206);
\draw[gp path] (9.209,7.507)--(9.209,7.417);
\draw[gp path] (9.548,4.116)--(9.548,4.206);
\draw[gp path] (9.548,7.507)--(9.548,7.417);
\draw[gp path] (9.888,4.116)--(9.888,4.206);
\draw[gp path] (9.888,7.507)--(9.888,7.417);
\draw[gp path] (10.227,4.116)--(10.227,4.206);
\draw[gp path] (10.227,7.507)--(10.227,7.417);
\draw[gp path] (10.566,4.116)--(10.566,4.206);
\draw[gp path] (10.566,7.507)--(10.566,7.417);
\draw[gp path] (10.566,4.116)--(10.566,4.296);
\draw[gp path] (10.566,7.507)--(10.566,7.327);
\node[gp node center,font=,20] at (10.566,3.808) {0.8};
\draw[gp path] (10.906,4.116)--(10.906,4.206);
\draw[gp path] (10.906,7.507)--(10.906,7.417);
\draw[gp path] (11.245,4.116)--(11.245,4.206);
\draw[gp path] (11.245,7.507)--(11.245,7.417);
\draw[gp path] (6.748,7.507)--(6.748,4.116)--(11.415,4.116)--(11.415,7.507)--cycle;
\color{gp_lt_color0}
\gpsetlinetype{gp_lt_plot0}
\draw[gp path] (6.833,6.843)--(6.833,7.013);
\draw[gp path] (6.743,6.843)--(6.923,6.843);
\draw[gp path] (6.743,7.013)--(6.923,7.013);
\draw[gp path] (7.087,6.508)--(7.087,6.676);
\draw[gp path] (6.997,6.508)--(7.177,6.508);
\draw[gp path] (6.997,6.676)--(7.177,6.676);
\draw[gp path] (7.342,6.389)--(7.342,6.565);
\draw[gp path] (7.252,6.389)--(7.432,6.389);
\draw[gp path] (7.252,6.565)--(7.432,6.565);
\draw[gp path] (7.597,6.105)--(7.597,6.272);
\draw[gp path] (7.507,6.105)--(7.687,6.105);
\draw[gp path] (7.507,6.272)--(7.687,6.272);
\draw[gp path] (7.851,5.710)--(7.851,5.889);
\draw[gp path] (7.761,5.710)--(7.941,5.710);
\draw[gp path] (7.761,5.889)--(7.941,5.889);
\draw[gp path] (8.106,5.875)--(8.106,6.045);
\draw[gp path] (8.016,5.875)--(8.196,5.875);
\draw[gp path] (8.016,6.045)--(8.196,6.045);
\draw[gp path] (8.360,5.429)--(8.360,5.603);
\draw[gp path] (8.270,5.429)--(8.450,5.429);
\draw[gp path] (8.270,5.603)--(8.450,5.603);
\draw[gp path] (8.615,5.695)--(8.615,5.864);
\draw[gp path] (8.525,5.695)--(8.705,5.695);
\draw[gp path] (8.525,5.864)--(8.705,5.864);
\draw[gp path] (8.869,5.441)--(8.869,5.608);
\draw[gp path] (8.779,5.441)--(8.959,5.441);
\draw[gp path] (8.779,5.608)--(8.959,5.608);
\draw[gp path] (9.124,4.511)--(9.124,5.000);
\draw[gp path] (9.034,4.511)--(9.214,4.511);
\draw[gp path] (9.034,5.000)--(9.214,5.000);
\draw[gp path] (9.378,5.710)--(9.378,5.910);
\draw[gp path] (9.288,5.710)--(9.468,5.710);
\draw[gp path] (9.288,5.910)--(9.468,5.910);
\draw[gp path] (9.633,5.679)--(9.633,5.851);
\draw[gp path] (9.543,5.679)--(9.723,5.679);
\draw[gp path] (9.543,5.851)--(9.723,5.851);
\draw[gp path] (9.888,5.608)--(9.888,5.779);
\draw[gp path] (9.798,5.608)--(9.978,5.608);
\draw[gp path] (9.798,5.779)--(9.978,5.779);
\draw[gp path] (10.142,5.542)--(10.142,5.709);
\draw[gp path] (10.052,5.542)--(10.232,5.542);
\draw[gp path] (10.052,5.709)--(10.232,5.709);
\draw[gp path] (10.397,5.620)--(10.397,5.794);
\draw[gp path] (10.307,5.620)--(10.487,5.620);
\draw[gp path] (10.307,5.794)--(10.487,5.794);
\draw[gp path] (10.651,5.786)--(10.651,5.957);
\draw[gp path] (10.561,5.786)--(10.741,5.786);
\draw[gp path] (10.561,5.957)--(10.741,5.957);
\draw[gp path] (10.906,6.035)--(10.906,6.204);
\draw[gp path] (10.816,6.035)--(10.996,6.035);
\draw[gp path] (10.816,6.204)--(10.996,6.204);
\draw[gp path] (11.160,6.227)--(11.160,6.397);
\draw[gp path] (11.070,6.227)--(11.250,6.227);
\draw[gp path] (11.070,6.397)--(11.250,6.397);
\draw[gp path] (11.415,6.207)--(11.415,6.379);
\draw[gp path] (11.325,6.207)--(11.505,6.207);
\draw[gp path] (11.325,6.379)--(11.505,6.379);
\gppoint{gp_mark1}{(6.833,6.928)}
\gppoint{gp_mark1}{(7.087,6.592)}
\gppoint{gp_mark1}{(7.342,6.477)}
\gppoint{gp_mark1}{(7.597,6.189)}
\gppoint{gp_mark1}{(7.851,5.799)}
\gppoint{gp_mark1}{(8.106,5.960)}
\gppoint{gp_mark1}{(8.360,5.516)}
\gppoint{gp_mark1}{(8.615,5.779)}
\gppoint{gp_mark1}{(8.869,5.525)}
\gppoint{gp_mark1}{(9.124,4.756)}
\gppoint{gp_mark1}{(9.378,5.810)}
\gppoint{gp_mark1}{(9.633,5.765)}
\gppoint{gp_mark1}{(9.888,5.693)}
\gppoint{gp_mark1}{(10.142,5.625)}
\gppoint{gp_mark1}{(10.397,5.707)}
\gppoint{gp_mark1}{(10.651,5.872)}
\gppoint{gp_mark1}{(10.906,6.120)}
\gppoint{gp_mark1}{(11.160,6.312)}
\gppoint{gp_mark1}{(11.415,6.293)}
\color{gp_lt_color3}
\gpsetlinetype{gp_lt_plot3}
\gpsetlinewidth{4.00}
\draw[gp path] (6.748,7.151)--(6.833,6.928)--(7.087,6.592)--(7.342,6.477)--(7.597,6.189)%
  --(7.851,5.799)--(8.106,5.960)--(8.360,5.516)--(8.615,5.779)--(8.869,5.525)--(9.124,4.756)%
  --(9.378,5.810)--(9.633,5.765)--(9.888,5.693)--(10.142,5.625)--(10.397,5.707)--(10.651,5.872)%
  --(10.906,6.120)--(11.160,6.312)--(11.415,6.293);
\color{gp_lt_color_b}
\gpsetlinetype{gp_lt_border}
\gpsetlinewidth{1.00}
\draw[gp path] (6.748,7.507)--(6.748,4.116)--(11.415,4.116)--(11.415,7.507)--cycle;
%% coordinates of the plot area
\coordinate (gpbb south west 2) at (6.748,4.116);
\coordinate (gpbb south east 2) at (11.415,4.116);
\coordinate (gpbb north east 2) at (11.415,7.507);
\coordinate (gpbb north west 2) at (6.748,7.507);
\coordinate (gpbb north 2) at (9.082,7.507);
\coordinate (gpbb south 2) at (9.082,4.116);
\coordinate (gpbb west 2) at (6.748,5.812);
\coordinate (gpbb east 2) at (11.415,5.812);
\end{tikzpicture}
%% gnuplot variables
}&
% % % % %       \scalebox{0.4}{\begin{tikzpicture}[gnuplot]
%% generated with GNUPLOT 4.2p5  (Lua 5.1.4; terminal rev. 81, script rev. 88)
%% Wed Sep  9 19:37:54 2009
\color{gp_lt_color_b}
\gpsetlinetype{gp_lt_border}
\gpsetlinewidth{1.00}
\draw[gp path] (2.240,0.985)--(2.420,0.985);
\draw[gp path] (12.040,0.985)--(11.860,0.985);
\node[gp node right] at (2.056,0.985) {-2.88};
\draw[gp path] (2.240,1.133)--(2.330,1.133);
\draw[gp path] (12.040,1.133)--(11.950,1.133);
\draw[gp path] (2.240,1.281)--(2.330,1.281);
\draw[gp path] (12.040,1.281)--(11.950,1.281);
\draw[gp path] (2.240,1.429)--(2.330,1.429);
\draw[gp path] (12.040,1.429)--(11.950,1.429);
\draw[gp path] (2.240,1.577)--(2.330,1.577);
\draw[gp path] (12.040,1.577)--(11.950,1.577);
\draw[gp path] (2.240,1.725)--(2.420,1.725);
\draw[gp path] (12.040,1.725)--(11.860,1.725);
\node[gp node right] at (2.056,1.725) {-2.875};
\draw[gp path] (2.240,1.873)--(2.330,1.873);
\draw[gp path] (12.040,1.873)--(11.950,1.873);
\draw[gp path] (2.240,2.021)--(2.330,2.021);
\draw[gp path] (12.040,2.021)--(11.950,2.021);
\draw[gp path] (2.240,2.169)--(2.330,2.169);
\draw[gp path] (12.040,2.169)--(11.950,2.169);
\draw[gp path] (2.240,2.316)--(2.330,2.316);
\draw[gp path] (12.040,2.316)--(11.950,2.316);
\draw[gp path] (2.240,2.464)--(2.420,2.464);
\draw[gp path] (12.040,2.464)--(11.860,2.464);
\node[gp node right] at (2.056,2.464) {-2.87};
\draw[gp path] (2.240,2.612)--(2.330,2.612);
\draw[gp path] (12.040,2.612)--(11.950,2.612);
\draw[gp path] (2.240,2.760)--(2.330,2.760);
\draw[gp path] (12.040,2.760)--(11.950,2.760);
\draw[gp path] (2.240,2.908)--(2.330,2.908);
\draw[gp path] (12.040,2.908)--(11.950,2.908);
\draw[gp path] (2.240,3.056)--(2.330,3.056);
\draw[gp path] (12.040,3.056)--(11.950,3.056);
\draw[gp path] (2.240,3.204)--(2.420,3.204);
\draw[gp path] (12.040,3.204)--(11.860,3.204);
\node[gp node right] at (2.056,3.204) {-2.865};
\draw[gp path] (2.240,3.352)--(2.330,3.352);
\draw[gp path] (12.040,3.352)--(11.950,3.352);
\draw[gp path] (2.240,3.500)--(2.330,3.500);
\draw[gp path] (12.040,3.500)--(11.950,3.500);
\draw[gp path] (2.240,3.648)--(2.330,3.648);
\draw[gp path] (12.040,3.648)--(11.950,3.648);
\draw[gp path] (2.240,3.796)--(2.330,3.796);
\draw[gp path] (12.040,3.796)--(11.950,3.796);
\draw[gp path] (2.240,3.944)--(2.420,3.944);
\draw[gp path] (12.040,3.944)--(11.860,3.944);
\node[gp node right] at (2.056,3.944) {-2.86};
\draw[gp path] (2.240,4.092)--(2.330,4.092);
\draw[gp path] (12.040,4.092)--(11.950,4.092);
\draw[gp path] (2.240,4.240)--(2.330,4.240);
\draw[gp path] (12.040,4.240)--(11.950,4.240);
\draw[gp path] (2.240,4.388)--(2.330,4.388);
\draw[gp path] (12.040,4.388)--(11.950,4.388);
\draw[gp path] (2.240,4.536)--(2.330,4.536);
\draw[gp path] (12.040,4.536)--(11.950,4.536);
\draw[gp path] (2.240,4.683)--(2.420,4.683);
\draw[gp path] (12.040,4.683)--(11.860,4.683);
\node[gp node right] at (2.056,4.683) {-2.855};
\draw[gp path] (2.240,4.831)--(2.330,4.831);
\draw[gp path] (12.040,4.831)--(11.950,4.831);
\draw[gp path] (2.240,4.979)--(2.330,4.979);
\draw[gp path] (12.040,4.979)--(11.950,4.979);
\draw[gp path] (2.240,5.127)--(2.330,5.127);
\draw[gp path] (12.040,5.127)--(11.950,5.127);
\draw[gp path] (2.240,5.275)--(2.330,5.275);
\draw[gp path] (12.040,5.275)--(11.950,5.275);
\draw[gp path] (2.240,5.423)--(2.420,5.423);
\draw[gp path] (12.040,5.423)--(11.860,5.423);
\node[gp node right] at (2.056,5.423) {-2.85};
\draw[gp path] (2.240,5.571)--(2.330,5.571);
\draw[gp path] (12.040,5.571)--(11.950,5.571);
\draw[gp path] (2.240,5.719)--(2.330,5.719);
\draw[gp path] (12.040,5.719)--(11.950,5.719);
\draw[gp path] (2.240,5.867)--(2.330,5.867);
\draw[gp path] (12.040,5.867)--(11.950,5.867);
\draw[gp path] (2.240,6.015)--(2.330,6.015);
\draw[gp path] (12.040,6.015)--(11.950,6.015);
\draw[gp path] (2.240,6.163)--(2.420,6.163);
\draw[gp path] (12.040,6.163)--(11.860,6.163);
\node[gp node right] at (2.056,6.163) {-2.845};
\draw[gp path] (2.240,6.311)--(2.330,6.311);
\draw[gp path] (12.040,6.311)--(11.950,6.311);
\draw[gp path] (2.240,6.459)--(2.330,6.459);
\draw[gp path] (12.040,6.459)--(11.950,6.459);
\draw[gp path] (2.240,6.607)--(2.330,6.607);
\draw[gp path] (12.040,6.607)--(11.950,6.607);
\draw[gp path] (2.240,6.755)--(2.330,6.755);
\draw[gp path] (12.040,6.755)--(11.950,6.755);
\draw[gp path] (2.240,6.903)--(2.420,6.903);
\draw[gp path] (12.040,6.903)--(11.860,6.903);
\node[gp node right] at (2.056,6.903) {-2.84};
\draw[gp path] (2.240,7.051)--(2.330,7.051);
\draw[gp path] (12.040,7.051)--(11.950,7.051);
\draw[gp path] (2.240,7.198)--(2.330,7.198);
\draw[gp path] (12.040,7.198)--(11.950,7.198);
\draw[gp path] (2.240,7.346)--(2.330,7.346);
\draw[gp path] (12.040,7.346)--(11.950,7.346);
\draw[gp path] (2.240,7.494)--(2.330,7.494);
\draw[gp path] (12.040,7.494)--(11.950,7.494);
\draw[gp path] (2.240,7.642)--(2.420,7.642);
\draw[gp path] (12.040,7.642)--(11.860,7.642);
\node[gp node right] at (2.056,7.642) {-2.835};
\draw[gp path] (2.240,7.790)--(2.330,7.790);
\draw[gp path] (12.040,7.790)--(11.950,7.790);
\draw[gp path] (2.240,7.938)--(2.330,7.938);
\draw[gp path] (12.040,7.938)--(11.950,7.938);
\draw[gp path] (2.240,8.086)--(2.330,8.086);
\draw[gp path] (12.040,8.086)--(11.950,8.086);
\draw[gp path] (2.240,8.234)--(2.330,8.234);
\draw[gp path] (12.040,8.234)--(11.950,8.234);
\draw[gp path] (2.240,8.382)--(2.420,8.382);
\draw[gp path] (12.040,8.382)--(11.860,8.382);
\node[gp node right] at (2.056,8.382) {-2.83};
\draw[gp path] (2.240,0.985)--(2.240,1.165);
\draw[gp path] (2.240,8.382)--(2.240,8.202);
\node[gp node center] at (2.240,0.677) {0.1};
\draw[gp path] (2.436,0.985)--(2.436,1.075);
\draw[gp path] (2.436,8.382)--(2.436,8.292);
\draw[gp path] (2.632,0.985)--(2.632,1.075);
\draw[gp path] (2.632,8.382)--(2.632,8.292);
\draw[gp path] (2.828,0.985)--(2.828,1.075);
\draw[gp path] (2.828,8.382)--(2.828,8.292);
\draw[gp path] (3.024,0.985)--(3.024,1.075);
\draw[gp path] (3.024,8.382)--(3.024,8.292);
\draw[gp path] (3.220,0.985)--(3.220,1.075);
\draw[gp path] (3.220,8.382)--(3.220,8.292);
\draw[gp path] (3.416,0.985)--(3.416,1.075);
\draw[gp path] (3.416,8.382)--(3.416,8.292);
\draw[gp path] (3.612,0.985)--(3.612,1.075);
\draw[gp path] (3.612,8.382)--(3.612,8.292);
\draw[gp path] (3.808,0.985)--(3.808,1.075);
\draw[gp path] (3.808,8.382)--(3.808,8.292);
\draw[gp path] (4.004,0.985)--(4.004,1.075);
\draw[gp path] (4.004,8.382)--(4.004,8.292);
\draw[gp path] (4.200,0.985)--(4.200,1.075);
\draw[gp path] (4.200,8.382)--(4.200,8.292);
\draw[gp path] (4.200,0.985)--(4.200,1.165);
\draw[gp path] (4.200,8.382)--(4.200,8.202);
\node[gp node center] at (4.200,0.677) {0.2};
\draw[gp path] (4.396,0.985)--(4.396,1.075);
\draw[gp path] (4.396,8.382)--(4.396,8.292);
\draw[gp path] (4.592,0.985)--(4.592,1.075);
\draw[gp path] (4.592,8.382)--(4.592,8.292);
\draw[gp path] (4.788,0.985)--(4.788,1.075);
\draw[gp path] (4.788,8.382)--(4.788,8.292);
\draw[gp path] (4.984,0.985)--(4.984,1.075);
\draw[gp path] (4.984,8.382)--(4.984,8.292);
\draw[gp path] (5.180,0.985)--(5.180,1.075);
\draw[gp path] (5.180,8.382)--(5.180,8.292);
\draw[gp path] (5.376,0.985)--(5.376,1.075);
\draw[gp path] (5.376,8.382)--(5.376,8.292);
\draw[gp path] (5.572,0.985)--(5.572,1.075);
\draw[gp path] (5.572,8.382)--(5.572,8.292);
\draw[gp path] (5.768,0.985)--(5.768,1.075);
\draw[gp path] (5.768,8.382)--(5.768,8.292);
\draw[gp path] (5.964,0.985)--(5.964,1.075);
\draw[gp path] (5.964,8.382)--(5.964,8.292);
\draw[gp path] (6.160,0.985)--(6.160,1.075);
\draw[gp path] (6.160,8.382)--(6.160,8.292);
\draw[gp path] (6.160,0.985)--(6.160,1.165);
\draw[gp path] (6.160,8.382)--(6.160,8.202);
\node[gp node center] at (6.160,0.677) {0.3};
\draw[gp path] (6.356,0.985)--(6.356,1.075);
\draw[gp path] (6.356,8.382)--(6.356,8.292);
\draw[gp path] (6.552,0.985)--(6.552,1.075);
\draw[gp path] (6.552,8.382)--(6.552,8.292);
\draw[gp path] (6.748,0.985)--(6.748,1.075);
\draw[gp path] (6.748,8.382)--(6.748,8.292);
\draw[gp path] (6.944,0.985)--(6.944,1.075);
\draw[gp path] (6.944,8.382)--(6.944,8.292);
\draw[gp path] (7.140,0.985)--(7.140,1.075);
\draw[gp path] (7.140,8.382)--(7.140,8.292);
\draw[gp path] (7.336,0.985)--(7.336,1.075);
\draw[gp path] (7.336,8.382)--(7.336,8.292);
\draw[gp path] (7.532,0.985)--(7.532,1.075);
\draw[gp path] (7.532,8.382)--(7.532,8.292);
\draw[gp path] (7.728,0.985)--(7.728,1.075);
\draw[gp path] (7.728,8.382)--(7.728,8.292);
\draw[gp path] (7.924,0.985)--(7.924,1.075);
\draw[gp path] (7.924,8.382)--(7.924,8.292);
\draw[gp path] (8.120,0.985)--(8.120,1.075);
\draw[gp path] (8.120,8.382)--(8.120,8.292);
\draw[gp path] (8.120,0.985)--(8.120,1.165);
\draw[gp path] (8.120,8.382)--(8.120,8.202);
\node[gp node center] at (8.120,0.677) {0.4};
\draw[gp path] (8.316,0.985)--(8.316,1.075);
\draw[gp path] (8.316,8.382)--(8.316,8.292);
\draw[gp path] (8.512,0.985)--(8.512,1.075);
\draw[gp path] (8.512,8.382)--(8.512,8.292);
\draw[gp path] (8.708,0.985)--(8.708,1.075);
\draw[gp path] (8.708,8.382)--(8.708,8.292);
\draw[gp path] (8.904,0.985)--(8.904,1.075);
\draw[gp path] (8.904,8.382)--(8.904,8.292);
\draw[gp path] (9.100,0.985)--(9.100,1.075);
\draw[gp path] (9.100,8.382)--(9.100,8.292);
\draw[gp path] (9.296,0.985)--(9.296,1.075);
\draw[gp path] (9.296,8.382)--(9.296,8.292);
\draw[gp path] (9.492,0.985)--(9.492,1.075);
\draw[gp path] (9.492,8.382)--(9.492,8.292);
\draw[gp path] (9.688,0.985)--(9.688,1.075);
\draw[gp path] (9.688,8.382)--(9.688,8.292);
\draw[gp path] (9.884,0.985)--(9.884,1.075);
\draw[gp path] (9.884,8.382)--(9.884,8.292);
\draw[gp path] (10.080,0.985)--(10.080,1.075);
\draw[gp path] (10.080,8.382)--(10.080,8.292);
\draw[gp path] (10.080,0.985)--(10.080,1.165);
\draw[gp path] (10.080,8.382)--(10.080,8.202);
\node[gp node center] at (10.080,0.677) {0.5};
\draw[gp path] (10.276,0.985)--(10.276,1.075);
\draw[gp path] (10.276,8.382)--(10.276,8.292);
\draw[gp path] (10.472,0.985)--(10.472,1.075);
\draw[gp path] (10.472,8.382)--(10.472,8.292);
\draw[gp path] (10.668,0.985)--(10.668,1.075);
\draw[gp path] (10.668,8.382)--(10.668,8.292);
\draw[gp path] (10.864,0.985)--(10.864,1.075);
\draw[gp path] (10.864,8.382)--(10.864,8.292);
\draw[gp path] (11.060,0.985)--(11.060,1.075);
\draw[gp path] (11.060,8.382)--(11.060,8.292);
\draw[gp path] (11.256,0.985)--(11.256,1.075);
\draw[gp path] (11.256,8.382)--(11.256,8.292);
\draw[gp path] (11.452,0.985)--(11.452,1.075);
\draw[gp path] (11.452,8.382)--(11.452,8.292);
\draw[gp path] (11.648,0.985)--(11.648,1.075);
\draw[gp path] (11.648,8.382)--(11.648,8.292);
\draw[gp path] (11.844,0.985)--(11.844,1.075);
\draw[gp path] (11.844,8.382)--(11.844,8.292);
\draw[gp path] (12.040,0.985)--(12.040,1.075);
\draw[gp path] (12.040,8.382)--(12.040,8.292);
\draw[gp path] (12.040,0.985)--(12.040,1.165);
\draw[gp path] (12.040,8.382)--(12.040,8.202);
\node[gp node center] at (12.040,0.677) {0.6};
\draw[gp path] (2.240,8.382)--(2.240,0.985)--(12.040,0.985)--(12.040,8.382)--cycle;
\node[gp node center,rotate=90] at (0.614,4.683) {Energy $\langle E \rangle$, au};
\node[gp node center] at (7.140,0.215) {Variantional parameter, $\beta$};
\color{gp_lt_color0}
\gpsetlinetype{gp_lt_plot0}
\draw[gp path] (2.632,7.559)--(2.632,7.762);
\draw[gp path] (2.542,7.559)--(2.722,7.559);
\draw[gp path] (2.542,7.762)--(2.722,7.762);
\draw[gp path] (3.220,6.363)--(3.220,6.563);
\draw[gp path] (3.130,6.363)--(3.310,6.363);
\draw[gp path] (3.130,6.563)--(3.310,6.563);
\draw[gp path] (3.808,5.393)--(3.808,5.590);
\draw[gp path] (3.718,5.393)--(3.898,5.393);
\draw[gp path] (3.718,5.590)--(3.898,5.590);
\draw[gp path] (4.396,4.289)--(4.396,4.485);
\draw[gp path] (4.306,4.289)--(4.486,4.289);
\draw[gp path] (4.306,4.485)--(4.486,4.485);
\draw[gp path] (4.984,3.972)--(4.984,4.167);
\draw[gp path] (4.894,3.972)--(5.074,3.972);
\draw[gp path] (4.894,4.167)--(5.074,4.167);
\draw[gp path] (5.572,3.593)--(5.572,3.785);
\draw[gp path] (5.482,3.593)--(5.662,3.593);
\draw[gp path] (5.482,3.785)--(5.662,3.785);
\draw[gp path] (6.160,3.302)--(6.160,3.488);
\draw[gp path] (6.070,3.302)--(6.250,3.302);
\draw[gp path] (6.070,3.488)--(6.250,3.488);
\draw[gp path] (6.748,2.542)--(6.748,2.730);
\draw[gp path] (6.658,2.542)--(6.838,2.542);
\draw[gp path] (6.658,2.730)--(6.838,2.730);
\draw[gp path] (7.336,2.779)--(7.336,2.962);
\draw[gp path] (7.246,2.779)--(7.426,2.779);
\draw[gp path] (7.246,2.962)--(7.426,2.962);
\draw[gp path] (7.924,2.458)--(7.924,2.645);
\draw[gp path] (7.834,2.458)--(8.014,2.458);
\draw[gp path] (7.834,2.645)--(8.014,2.645);
\draw[gp path] (8.512,2.246)--(8.512,2.430);
\draw[gp path] (8.422,2.246)--(8.602,2.246);
\draw[gp path] (8.422,2.430)--(8.602,2.430);
\draw[gp path] (9.100,1.285)--(9.100,1.487);
\draw[gp path] (9.010,1.285)--(9.190,1.285);
\draw[gp path] (9.010,1.487)--(9.190,1.487);
\draw[gp path] (9.688,1.682)--(9.688,1.866);
\draw[gp path] (9.598,1.682)--(9.778,1.682);
\draw[gp path] (9.598,1.866)--(9.778,1.866);
\draw[gp path] (10.276,1.574)--(10.276,1.758);
\draw[gp path] (10.186,1.574)--(10.366,1.574);
\draw[gp path] (10.186,1.758)--(10.366,1.758);
\draw[gp path] (10.864,1.686)--(10.864,1.888);
\draw[gp path] (10.774,1.686)--(10.954,1.686);
\draw[gp path] (10.774,1.888)--(10.954,1.888);
\draw[gp path] (11.452,1.780)--(11.452,1.965);
\draw[gp path] (11.362,1.780)--(11.542,1.780);
\draw[gp path] (11.362,1.965)--(11.542,1.965);
\draw[gp path] (12.040,1.670)--(12.040,1.851);
\draw[gp path] (11.950,1.670)--(12.130,1.670);
\draw[gp path] (11.950,1.851)--(12.130,1.851);
\gpsetpointsize{4.00}
\gppoint{gp_mark1}{(2.632,7.661)}
\gppoint{gp_mark1}{(3.220,6.463)}
\gppoint{gp_mark1}{(3.808,5.492)}
\gppoint{gp_mark1}{(4.396,4.387)}
\gppoint{gp_mark1}{(4.984,4.069)}
\gppoint{gp_mark1}{(5.572,3.689)}
\gppoint{gp_mark1}{(6.160,3.395)}
\gppoint{gp_mark1}{(6.748,2.636)}
\gppoint{gp_mark1}{(7.336,2.871)}
\gppoint{gp_mark1}{(7.924,2.551)}
\gppoint{gp_mark1}{(8.512,2.338)}
\gppoint{gp_mark1}{(9.100,1.386)}
\gppoint{gp_mark1}{(9.688,1.774)}
\gppoint{gp_mark1}{(10.276,1.666)}
\gppoint{gp_mark1}{(10.864,1.787)}
\gppoint{gp_mark1}{(11.452,1.873)}
\gppoint{gp_mark1}{(12.040,1.761)}
\color{gp_lt_color3}
\gpsetlinetype{gp_lt_plot3}
\gpsetlinewidth{4.00}
\draw[gp path] (2.394,8.382)--(2.632,7.661)--(3.220,6.463)--(3.808,5.492)--(4.396,4.387)%
  --(4.984,4.069)--(5.572,3.689)--(6.160,3.395)--(6.748,2.636)--(7.336,2.871)--(7.924,2.551)%
  --(8.512,2.338)--(9.100,1.386)--(9.688,1.774)--(10.276,1.666)--(10.864,1.787)--(11.452,1.873)%
  --(12.040,1.761);
\color{gp_lt_color_b}
\gpsetlinetype{gp_lt_border}
\gpsetlinewidth{1.00}
\draw[gp path] (2.240,8.382)--(2.240,0.985)--(12.040,0.985)--(12.040,8.382)--cycle;
%% coordinates of the plot area
\coordinate (gpbb south west 1) at (2.240,0.985);
\coordinate (gpbb south east 1) at (12.040,0.985);
\coordinate (gpbb north east 1) at (12.040,8.382);
\coordinate (gpbb north west 1) at (2.240,8.382);
\coordinate (gpbb north 1) at (7.140,8.382);
\coordinate (gpbb south 1) at (7.140,0.985);
\coordinate (gpbb west 1) at (2.240,4.684);
\coordinate (gpbb east 1) at (12.040,4.684);
\draw[gp path] (6.748,4.116)--(6.838,4.116);
\draw[gp path] (11.415,4.116)--(11.325,4.116);
\draw[gp path] (6.748,4.964)--(6.838,4.964);
\draw[gp path] (11.415,4.964)--(11.325,4.964);
\draw[gp path] (6.748,5.812)--(6.838,5.812);
\draw[gp path] (11.415,5.812)--(11.325,5.812);
\draw[gp path] (6.748,6.659)--(6.838,6.659);
\draw[gp path] (11.415,6.659)--(11.325,6.659);
\draw[gp path] (6.748,7.507)--(6.928,7.507);
\draw[gp path] (11.415,7.507)--(11.235,7.507);
\node[gp node right] at (6.564,7.507) {-2.87};
\draw[gp path] (6.748,4.116)--(6.748,4.296);
\draw[gp path] (6.748,7.507)--(6.748,7.327);
\node[gp node center,font=,20] at (6.748,3.808) {0.4};
\draw[gp path] (7.681,4.116)--(7.681,4.206);
\draw[gp path] (7.681,7.507)--(7.681,7.417);
\draw[gp path] (8.615,4.116)--(8.615,4.206);
\draw[gp path] (8.615,7.507)--(8.615,7.417);
\draw[gp path] (9.548,4.116)--(9.548,4.206);
\draw[gp path] (9.548,7.507)--(9.548,7.417);
\draw[gp path] (10.482,4.116)--(10.482,4.206);
\draw[gp path] (10.482,7.507)--(10.482,7.417);
\draw[gp path] (11.415,4.116)--(11.415,4.206);
\draw[gp path] (11.415,7.507)--(11.415,7.417);
\draw[gp path] (6.748,7.507)--(6.748,4.116)--(11.415,4.116)--(11.415,7.507)--cycle;
\color{gp_lt_color0}
\gpsetlinetype{gp_lt_plot0}
\draw[gp path] (7.215,6.882)--(7.215,7.409);
\draw[gp path] (7.125,6.882)--(7.305,6.882);
\draw[gp path] (7.125,7.409)--(7.305,7.409);
\draw[gp path] (7.915,4.128)--(7.915,4.706);
\draw[gp path] (7.825,4.128)--(8.005,4.128);
\draw[gp path] (7.825,4.706)--(8.005,4.706);
\draw[gp path] (8.615,5.266)--(8.615,5.792);
\draw[gp path] (8.525,5.266)--(8.705,5.266);
\draw[gp path] (8.525,5.792)--(8.705,5.792);
\draw[gp path] (9.315,4.955)--(9.315,5.482);
\draw[gp path] (9.225,4.955)--(9.405,4.955);
\draw[gp path] (9.225,5.482)--(9.405,5.482);
\draw[gp path] (10.015,5.278)--(10.015,5.856);
\draw[gp path] (9.925,5.278)--(10.105,5.278);
\draw[gp path] (9.925,5.856)--(10.105,5.856);
\draw[gp path] (10.715,5.547)--(10.715,6.077);
\draw[gp path] (10.625,5.547)--(10.805,5.547);
\draw[gp path] (10.625,6.077)--(10.805,6.077);
\draw[gp path] (11.415,5.232)--(11.415,5.748);
\draw[gp path] (11.325,5.232)--(11.505,5.232);
\draw[gp path] (11.325,5.748)--(11.505,5.748);
\gppoint{gp_mark1}{(7.215,7.145)}
\gppoint{gp_mark1}{(7.915,4.417)}
\gppoint{gp_mark1}{(8.615,5.529)}
\gppoint{gp_mark1}{(9.315,5.218)}
\gppoint{gp_mark1}{(10.015,5.567)}
\gppoint{gp_mark1}{(10.715,5.812)}
\gppoint{gp_mark1}{(11.415,5.490)}
\color{gp_lt_color3}
\gpsetlinetype{gp_lt_plot3}
\gpsetlinewidth{4.00}
\draw[gp path] (6.800,7.507)--(7.215,7.145)--(7.915,4.417)--(8.615,5.529)--(9.315,5.218)%
  --(10.015,5.567)--(10.715,5.812)--(11.415,5.490);
\color{gp_lt_color_b}
\gpsetlinetype{gp_lt_border}
\gpsetlinewidth{1.00}
\draw[gp path] (6.748,7.507)--(6.748,4.116)--(11.415,4.116)--(11.415,7.507)--cycle;
%% coordinates of the plot area
\coordinate (gpbb south west 2) at (6.748,4.116);
\coordinate (gpbb south east 2) at (11.415,4.116);
\coordinate (gpbb north east 2) at (11.415,7.507);
\coordinate (gpbb north west 2) at (6.748,7.507);
\coordinate (gpbb north 2) at (9.082,7.507);
\coordinate (gpbb south 2) at (9.082,4.116);
\coordinate (gpbb west 2) at (6.748,5.812);
\coordinate (gpbb east 2) at (11.415,5.812);
\end{tikzpicture}
%% gnuplot variables
}
% % % % %     \end{tabular}
% % % % %     \caption{\scriptsize Energy for the He atom as a function of the variational parameter $\beta$ computed with plain C++ (left) and Python (right) simulators. Set up: $dt=0.1$ and $\alpha = 1.6785$. Monte Carlo cycles: $1\times 10^{7}$ and $1\times 10^{6}$ cycles for C++ and Python, respectively.}
% % % % %   \end{figure}
% % % % %    
% % % % %   \begin{table}
% % % % %     \centering
% % % % %      \begin{scriptsize}
% % % % %     \begin{tabular}{rl}
% % % % %       \toprule[0.5pt]
% % % % %       \textbf{Method} & \textbf{$\langle E \rangle$}, au \\
% % % % %       \midrule[0.5pt]
% % % % %         Hartree-Fock   & -2.8617 \\
% % % % %         DFT            & -2.83 \\
% % % % %         Exact          & -2.9037\\
% % % % %       \bottomrule[1pt]
% % % % %     \end{tabular}\caption{Ground state energies for He atom obtained from reference \cite{Thijssen}.}
% % % % %     \end{scriptsize}
% % % % %   \end{table}
% % % % % \end{frame}



\begin{frame}{Comparing Python to C++ computing VMC}
       \begin{figure}
        \begin{tabular}{cc}
        \centering
        \scalebox{0.42}{\begin{tikzpicture}[gnuplot]
%% generated with GNUPLOT 4.2p5  (Lua 5.1.4; terminal rev. 81, script rev. 88)
%% Fri Sep  4 17:27:28 2009
\color{gp_lt_color_b}
\gpsetlinetype{gp_lt_border}
\gpsetlinewidth{1.00}
\draw[gp path] (1.688,0.985)--(1.868,0.985);
\draw[gp path] (12.040,0.985)--(11.860,0.985);
\node[gp node right] at (1.504,0.985) {0};
\draw[gp path] (1.688,1.141)--(1.778,1.141);
\draw[gp path] (12.040,1.141)--(11.950,1.141);
\draw[gp path] (1.688,1.296)--(1.778,1.296);
\draw[gp path] (12.040,1.296)--(11.950,1.296);
\draw[gp path] (1.688,1.452)--(1.778,1.452);
\draw[gp path] (12.040,1.452)--(11.950,1.452);
\draw[gp path] (1.688,1.608)--(1.778,1.608);
\draw[gp path] (12.040,1.608)--(11.950,1.608);
\draw[gp path] (1.688,1.764)--(1.868,1.764);
\draw[gp path] (12.040,1.764)--(11.860,1.764);
\node[gp node right] at (1.504,1.764) {20};
\draw[gp path] (1.688,1.919)--(1.778,1.919);
\draw[gp path] (12.040,1.919)--(11.950,1.919);
\draw[gp path] (1.688,2.075)--(1.778,2.075);
\draw[gp path] (12.040,2.075)--(11.950,2.075);
\draw[gp path] (1.688,2.231)--(1.778,2.231);
\draw[gp path] (12.040,2.231)--(11.950,2.231);
\draw[gp path] (1.688,2.387)--(1.778,2.387);
\draw[gp path] (12.040,2.387)--(11.950,2.387);
\draw[gp path] (1.688,2.542)--(1.868,2.542);
\draw[gp path] (12.040,2.542)--(11.860,2.542);
\node[gp node right] at (1.504,2.542) {40};
\draw[gp path] (1.688,2.698)--(1.778,2.698);
\draw[gp path] (12.040,2.698)--(11.950,2.698);
\draw[gp path] (1.688,2.854)--(1.778,2.854);
\draw[gp path] (12.040,2.854)--(11.950,2.854);
\draw[gp path] (1.688,3.009)--(1.778,3.009);
\draw[gp path] (12.040,3.009)--(11.950,3.009);
\draw[gp path] (1.688,3.165)--(1.778,3.165);
\draw[gp path] (12.040,3.165)--(11.950,3.165);
\draw[gp path] (1.688,3.321)--(1.868,3.321);
\draw[gp path] (12.040,3.321)--(11.860,3.321);
\node[gp node right] at (1.504,3.321) {60};
\draw[gp path] (1.688,3.477)--(1.778,3.477);
\draw[gp path] (12.040,3.477)--(11.950,3.477);
\draw[gp path] (1.688,3.632)--(1.778,3.632);
\draw[gp path] (12.040,3.632)--(11.950,3.632);
\draw[gp path] (1.688,3.788)--(1.778,3.788);
\draw[gp path] (12.040,3.788)--(11.950,3.788);
\draw[gp path] (1.688,3.944)--(1.778,3.944);
\draw[gp path] (12.040,3.944)--(11.950,3.944);
\draw[gp path] (1.688,4.100)--(1.868,4.100);
\draw[gp path] (12.040,4.100)--(11.860,4.100);
\node[gp node right] at (1.504,4.100) {80};
\draw[gp path] (1.688,4.255)--(1.778,4.255);
\draw[gp path] (12.040,4.255)--(11.950,4.255);
\draw[gp path] (1.688,4.411)--(1.778,4.411);
\draw[gp path] (12.040,4.411)--(11.950,4.411);
\draw[gp path] (1.688,4.567)--(1.778,4.567);
\draw[gp path] (12.040,4.567)--(11.950,4.567);
\draw[gp path] (1.688,4.722)--(1.778,4.722);
\draw[gp path] (12.040,4.722)--(11.950,4.722);
\draw[gp path] (1.688,4.878)--(1.868,4.878);
\draw[gp path] (12.040,4.878)--(11.860,4.878);
\node[gp node right] at (1.504,4.878) {100};
\draw[gp path] (1.688,5.034)--(1.778,5.034);
\draw[gp path] (12.040,5.034)--(11.950,5.034);
\draw[gp path] (1.688,5.190)--(1.778,5.190);
\draw[gp path] (12.040,5.190)--(11.950,5.190);
\draw[gp path] (1.688,5.345)--(1.778,5.345);
\draw[gp path] (12.040,5.345)--(11.950,5.345);
\draw[gp path] (1.688,5.501)--(1.778,5.501);
\draw[gp path] (12.040,5.501)--(11.950,5.501);
\draw[gp path] (1.688,5.657)--(1.868,5.657);
\draw[gp path] (12.040,5.657)--(11.860,5.657);
\node[gp node right] at (1.504,5.657) {120};
\draw[gp path] (1.688,5.813)--(1.778,5.813);
\draw[gp path] (12.040,5.813)--(11.950,5.813);
\draw[gp path] (1.688,5.968)--(1.778,5.968);
\draw[gp path] (12.040,5.968)--(11.950,5.968);
\draw[gp path] (1.688,6.124)--(1.778,6.124);
\draw[gp path] (12.040,6.124)--(11.950,6.124);
\draw[gp path] (1.688,6.280)--(1.778,6.280);
\draw[gp path] (12.040,6.280)--(11.950,6.280);
\draw[gp path] (1.688,6.435)--(1.868,6.435);
\draw[gp path] (12.040,6.435)--(11.860,6.435);
\node[gp node right] at (1.504,6.435) {140};
\draw[gp path] (1.688,6.591)--(1.778,6.591);
\draw[gp path] (12.040,6.591)--(11.950,6.591);
\draw[gp path] (1.688,6.747)--(1.778,6.747);
\draw[gp path] (12.040,6.747)--(11.950,6.747);
\draw[gp path] (1.688,6.903)--(1.778,6.903);
\draw[gp path] (12.040,6.903)--(11.950,6.903);
\draw[gp path] (1.688,7.058)--(1.778,7.058);
\draw[gp path] (12.040,7.058)--(11.950,7.058);
\draw[gp path] (1.688,7.214)--(1.868,7.214);
\draw[gp path] (12.040,7.214)--(11.860,7.214);
\node[gp node right] at (1.504,7.214) {160};
\draw[gp path] (1.688,7.370)--(1.778,7.370);
\draw[gp path] (12.040,7.370)--(11.950,7.370);
\draw[gp path] (1.688,7.526)--(1.778,7.526);
\draw[gp path] (12.040,7.526)--(11.950,7.526);
\draw[gp path] (1.688,7.681)--(1.778,7.681);
\draw[gp path] (12.040,7.681)--(11.950,7.681);
\draw[gp path] (1.688,7.837)--(1.778,7.837);
\draw[gp path] (12.040,7.837)--(11.950,7.837);
\draw[gp path] (1.688,7.993)--(1.868,7.993);
\draw[gp path] (12.040,7.993)--(11.860,7.993);
\node[gp node right] at (1.504,7.993) {180};
\draw[gp path] (1.688,8.148)--(1.778,8.148);
\draw[gp path] (12.040,8.148)--(11.950,8.148);
\draw[gp path] (1.688,8.304)--(1.778,8.304);
\draw[gp path] (12.040,8.304)--(11.950,8.304);
\draw[gp path] (1.895,0.985)--(1.895,1.075);
\draw[gp path] (1.895,8.382)--(1.895,8.292);
\draw[gp path] (2.309,0.985)--(2.309,1.075);
\draw[gp path] (2.309,8.382)--(2.309,8.292);
\draw[gp path] (2.723,0.985)--(2.723,1.165);
\draw[gp path] (2.723,8.382)--(2.723,8.202);
\node[gp node center] at (2.723,0.677) {20000};
\draw[gp path] (3.137,0.985)--(3.137,1.075);
\draw[gp path] (3.137,8.382)--(3.137,8.292);
\draw[gp path] (3.551,0.985)--(3.551,1.075);
\draw[gp path] (3.551,8.382)--(3.551,8.292);
\draw[gp path] (3.965,0.985)--(3.965,1.075);
\draw[gp path] (3.965,8.382)--(3.965,8.292);
\draw[gp path] (4.380,0.985)--(4.380,1.075);
\draw[gp path] (4.380,8.382)--(4.380,8.292);
\draw[gp path] (4.794,0.985)--(4.794,1.165);
\draw[gp path] (4.794,8.382)--(4.794,8.202);
\node[gp node center] at (4.794,0.677) {40000};
\draw[gp path] (5.208,0.985)--(5.208,1.075);
\draw[gp path] (5.208,8.382)--(5.208,8.292);
\draw[gp path] (5.622,0.985)--(5.622,1.075);
\draw[gp path] (5.622,8.382)--(5.622,8.292);
\draw[gp path] (6.036,0.985)--(6.036,1.075);
\draw[gp path] (6.036,8.382)--(6.036,8.292);
\draw[gp path] (6.450,0.985)--(6.450,1.075);
\draw[gp path] (6.450,8.382)--(6.450,8.292);
\draw[gp path] (6.864,0.985)--(6.864,1.165);
\draw[gp path] (6.864,8.382)--(6.864,8.202);
\node[gp node center] at (6.864,0.677) {60000};
\draw[gp path] (7.278,0.985)--(7.278,1.075);
\draw[gp path] (7.278,8.382)--(7.278,8.292);
\draw[gp path] (7.692,0.985)--(7.692,1.075);
\draw[gp path] (7.692,8.382)--(7.692,8.292);
\draw[gp path] (8.106,0.985)--(8.106,1.075);
\draw[gp path] (8.106,8.382)--(8.106,8.292);
\draw[gp path] (8.520,0.985)--(8.520,1.075);
\draw[gp path] (8.520,8.382)--(8.520,8.292);
\draw[gp path] (8.934,0.985)--(8.934,1.165);
\draw[gp path] (8.934,8.382)--(8.934,8.202);
\node[gp node center] at (8.934,0.677) {80000};
\draw[gp path] (9.348,0.985)--(9.348,1.075);
\draw[gp path] (9.348,8.382)--(9.348,8.292);
\draw[gp path] (9.763,0.985)--(9.763,1.075);
\draw[gp path] (9.763,8.382)--(9.763,8.292);
\draw[gp path] (10.177,0.985)--(10.177,1.075);
\draw[gp path] (10.177,8.382)--(10.177,8.292);
\draw[gp path] (10.591,0.985)--(10.591,1.075);
\draw[gp path] (10.591,8.382)--(10.591,8.292);
\draw[gp path] (11.005,0.985)--(11.005,1.165);
\draw[gp path] (11.005,8.382)--(11.005,8.202);
\node[gp node center] at (11.005,0.677) {100000};
\draw[gp path] (11.419,0.985)--(11.419,1.075);
\draw[gp path] (11.419,8.382)--(11.419,8.292);
\draw[gp path] (11.833,0.985)--(11.833,1.075);
\draw[gp path] (11.833,8.382)--(11.833,8.292);
\draw[gp path] (1.688,8.382)--(1.688,0.985)--(12.040,0.985)--(12.040,8.382)--cycle;
\node[gp node center,rotate=90] at (0.614,4.683) {Execution time, s};
\node[gp node center] at (6.864,0.215) {Monte Carlo cycles};
\color{gp_lt_color2}
\gpsetlinetype{gp_lt_plot2}
\gpsetlinewidth{5.00}
\draw[gp path] (1.688,1.672)--(2.206,2.010)--(2.723,2.344)--(3.241,2.695)--(3.758,3.031)%
  --(4.276,3.383)--(4.794,3.716)--(5.311,4.065)--(5.829,4.400)--(6.346,4.746)--(6.864,5.070)%
  --(7.382,5.435)--(7.899,5.803)--(8.417,6.156)--(8.934,6.499)--(9.452,6.876)--(9.970,7.168)%
  --(10.487,7.548)--(11.005,7.861)--(11.522,8.208);
\gpsetpointsize{4.00}
\gppoint{gp_mark1}{(1.688,1.672)}
\gppoint{gp_mark1}{(2.206,2.010)}
\gppoint{gp_mark1}{(2.723,2.344)}
\gppoint{gp_mark1}{(3.241,2.695)}
\gppoint{gp_mark1}{(3.758,3.031)}
\gppoint{gp_mark1}{(4.276,3.383)}
\gppoint{gp_mark1}{(4.794,3.716)}
\gppoint{gp_mark1}{(5.311,4.065)}
\gppoint{gp_mark1}{(5.829,4.400)}
\gppoint{gp_mark1}{(6.346,4.746)}
\gppoint{gp_mark1}{(6.864,5.070)}
\gppoint{gp_mark1}{(7.382,5.435)}
\gppoint{gp_mark1}{(7.899,5.803)}
\gppoint{gp_mark1}{(8.417,6.156)}
\gppoint{gp_mark1}{(8.934,6.499)}
\gppoint{gp_mark1}{(9.452,6.876)}
\gppoint{gp_mark1}{(9.970,7.168)}
\gppoint{gp_mark1}{(10.487,7.548)}
\gppoint{gp_mark1}{(11.005,7.861)}
\gppoint{gp_mark1}{(11.522,8.208)}
\color{gp_lt_color1}
\gpsetlinetype{gp_lt_plot1}
\draw[gp path] (1.688,0.998)--(2.206,0.999)--(2.723,1.003)--(3.241,1.009)--(3.758,1.013)%
  --(4.276,1.017)--(4.794,1.022)--(5.311,1.026)--(5.829,1.031)--(6.346,1.036)--(6.864,1.041)%
  --(7.382,1.045)--(7.899,1.050)--(8.417,1.055)--(8.934,1.059)--(9.452,1.064)--(9.970,1.071)%
  --(10.487,1.074)--(11.005,1.079)--(11.522,1.085);
\gppoint{gp_mark2}{(1.688,0.998)}
\gppoint{gp_mark2}{(2.206,0.999)}
\gppoint{gp_mark2}{(2.723,1.003)}
\gppoint{gp_mark2}{(3.241,1.009)}
\gppoint{gp_mark2}{(3.758,1.013)}
\gppoint{gp_mark2}{(4.276,1.017)}
\gppoint{gp_mark2}{(4.794,1.022)}
\gppoint{gp_mark2}{(5.311,1.026)}
\gppoint{gp_mark2}{(5.829,1.031)}
\gppoint{gp_mark2}{(6.346,1.036)}
\gppoint{gp_mark2}{(6.864,1.041)}
\gppoint{gp_mark2}{(7.382,1.045)}
\gppoint{gp_mark2}{(7.899,1.050)}
\gppoint{gp_mark2}{(8.417,1.055)}
\gppoint{gp_mark2}{(8.934,1.059)}
\gppoint{gp_mark2}{(9.452,1.064)}
\gppoint{gp_mark2}{(9.970,1.071)}
\gppoint{gp_mark2}{(10.487,1.074)}
\gppoint{gp_mark2}{(11.005,1.079)}
\gppoint{gp_mark2}{(11.522,1.085)}
\color{gp_lt_color_b}
\gpsetlinetype{gp_lt_border}
\gpsetlinewidth{1.00}
\draw[gp path] (1.688,8.382)--(1.688,0.985)--(12.040,0.985)--(12.040,8.382)--cycle;
%% coordinates of the plot area
\coordinate (gpbb south west 1) at (1.688,0.985);
\coordinate (gpbb south east 1) at (12.040,0.985);
\coordinate (gpbb north east 1) at (12.040,8.382);
\coordinate (gpbb north west 1) at (1.688,8.382);
\coordinate (gpbb north 1) at (6.864,8.382);
\coordinate (gpbb south 1) at (6.864,0.985);
\coordinate (gpbb west 1) at (1.688,4.684);
\coordinate (gpbb east 1) at (12.040,4.684);
\end{tikzpicture}
%% gnuplot variables
}
        \scalebox{0.42}{\begin{tikzpicture}[gnuplot]
%% generated with GNUPLOT 4.2p5  (Lua 5.1.4; terminal rev. 81, script rev. 88)
%% Sat Sep  5 23:00:35 2009
\color{gp_lt_color_b}
\gpsetlinetype{gp_lt_border}
\gpsetlinewidth{1.00}
\draw[gp path] (1.688,0.985)--(1.868,0.985);
\draw[gp path] (12.040,0.985)--(11.860,0.985);
\node[gp node right] at (1.504,0.985) {0};
\draw[gp path] (1.688,1.270)--(1.778,1.270);
\draw[gp path] (12.040,1.270)--(11.950,1.270);
\draw[gp path] (1.688,1.554)--(1.778,1.554);
\draw[gp path] (12.040,1.554)--(11.950,1.554);
\draw[gp path] (1.688,1.839)--(1.778,1.839);
\draw[gp path] (12.040,1.839)--(11.950,1.839);
\draw[gp path] (1.688,2.123)--(1.778,2.123);
\draw[gp path] (12.040,2.123)--(11.950,2.123);
\draw[gp path] (1.688,2.408)--(1.868,2.408);
\draw[gp path] (12.040,2.408)--(11.860,2.408);
\node[gp node right] at (1.504,2.408) {0.5};
\draw[gp path] (1.688,2.692)--(1.778,2.692);
\draw[gp path] (12.040,2.692)--(11.950,2.692);
\draw[gp path] (1.688,2.976)--(1.778,2.976);
\draw[gp path] (12.040,2.976)--(11.950,2.976);
\draw[gp path] (1.688,3.261)--(1.778,3.261);
\draw[gp path] (12.040,3.261)--(11.950,3.261);
\draw[gp path] (1.688,3.546)--(1.778,3.546);
\draw[gp path] (12.040,3.546)--(11.950,3.546);
\draw[gp path] (1.688,3.830)--(1.868,3.830);
\draw[gp path] (12.040,3.830)--(11.860,3.830);
\node[gp node right] at (1.504,3.830) {1};
\draw[gp path] (1.688,4.115)--(1.778,4.115);
\draw[gp path] (12.040,4.115)--(11.950,4.115);
\draw[gp path] (1.688,4.399)--(1.778,4.399);
\draw[gp path] (12.040,4.399)--(11.950,4.399);
\draw[gp path] (1.688,4.684)--(1.778,4.684);
\draw[gp path] (12.040,4.684)--(11.950,4.684);
\draw[gp path] (1.688,4.968)--(1.778,4.968);
\draw[gp path] (12.040,4.968)--(11.950,4.968);
\draw[gp path] (1.688,5.253)--(1.868,5.253);
\draw[gp path] (12.040,5.253)--(11.860,5.253);
\node[gp node right] at (1.504,5.253) {1.5};
\draw[gp path] (1.688,5.537)--(1.778,5.537);
\draw[gp path] (12.040,5.537)--(11.950,5.537);
\draw[gp path] (1.688,5.821)--(1.778,5.821);
\draw[gp path] (12.040,5.821)--(11.950,5.821);
\draw[gp path] (1.688,6.106)--(1.778,6.106);
\draw[gp path] (12.040,6.106)--(11.950,6.106);
\draw[gp path] (1.688,6.391)--(1.778,6.391);
\draw[gp path] (12.040,6.391)--(11.950,6.391);
\draw[gp path] (1.688,6.675)--(1.868,6.675);
\draw[gp path] (12.040,6.675)--(11.860,6.675);
\node[gp node right] at (1.504,6.675) {2};
\draw[gp path] (1.688,6.960)--(1.778,6.960);
\draw[gp path] (12.040,6.960)--(11.950,6.960);
\draw[gp path] (1.688,7.244)--(1.778,7.244);
\draw[gp path] (12.040,7.244)--(11.950,7.244);
\draw[gp path] (1.688,7.528)--(1.778,7.528);
\draw[gp path] (12.040,7.528)--(11.950,7.528);
\draw[gp path] (1.688,7.813)--(1.778,7.813);
\draw[gp path] (12.040,7.813)--(11.950,7.813);
\draw[gp path] (1.688,8.098)--(1.868,8.098);
\draw[gp path] (12.040,8.098)--(11.860,8.098);
\node[gp node right] at (1.504,8.098) {2.5};
\draw[gp path] (1.688,8.382)--(1.778,8.382);
\draw[gp path] (12.040,8.382)--(11.950,8.382);
\draw[gp path] (1.895,0.985)--(1.895,1.075);
\draw[gp path] (1.895,8.382)--(1.895,8.292);
\draw[gp path] (2.309,0.985)--(2.309,1.075);
\draw[gp path] (2.309,8.382)--(2.309,8.292);
\draw[gp path] (2.723,0.985)--(2.723,1.165);
\draw[gp path] (2.723,8.382)--(2.723,8.202);
\node[gp node center] at (2.723,0.677) {20000};
\draw[gp path] (3.137,0.985)--(3.137,1.075);
\draw[gp path] (3.137,8.382)--(3.137,8.292);
\draw[gp path] (3.551,0.985)--(3.551,1.075);
\draw[gp path] (3.551,8.382)--(3.551,8.292);
\draw[gp path] (3.965,0.985)--(3.965,1.075);
\draw[gp path] (3.965,8.382)--(3.965,8.292);
\draw[gp path] (4.380,0.985)--(4.380,1.075);
\draw[gp path] (4.380,8.382)--(4.380,8.292);
\draw[gp path] (4.794,0.985)--(4.794,1.165);
\draw[gp path] (4.794,8.382)--(4.794,8.202);
\node[gp node center] at (4.794,0.677) {40000};
\draw[gp path] (5.208,0.985)--(5.208,1.075);
\draw[gp path] (5.208,8.382)--(5.208,8.292);
\draw[gp path] (5.622,0.985)--(5.622,1.075);
\draw[gp path] (5.622,8.382)--(5.622,8.292);
\draw[gp path] (6.036,0.985)--(6.036,1.075);
\draw[gp path] (6.036,8.382)--(6.036,8.292);
\draw[gp path] (6.450,0.985)--(6.450,1.075);
\draw[gp path] (6.450,8.382)--(6.450,8.292);
\draw[gp path] (6.864,0.985)--(6.864,1.165);
\draw[gp path] (6.864,8.382)--(6.864,8.202);
\node[gp node center] at (6.864,0.677) {60000};
\draw[gp path] (7.278,0.985)--(7.278,1.075);
\draw[gp path] (7.278,8.382)--(7.278,8.292);
\draw[gp path] (7.692,0.985)--(7.692,1.075);
\draw[gp path] (7.692,8.382)--(7.692,8.292);
\draw[gp path] (8.106,0.985)--(8.106,1.075);
\draw[gp path] (8.106,8.382)--(8.106,8.292);
\draw[gp path] (8.520,0.985)--(8.520,1.075);
\draw[gp path] (8.520,8.382)--(8.520,8.292);
\draw[gp path] (8.934,0.985)--(8.934,1.165);
\draw[gp path] (8.934,8.382)--(8.934,8.202);
\node[gp node center] at (8.934,0.677) {80000};
\draw[gp path] (9.348,0.985)--(9.348,1.075);
\draw[gp path] (9.348,8.382)--(9.348,8.292);
\draw[gp path] (9.763,0.985)--(9.763,1.075);
\draw[gp path] (9.763,8.382)--(9.763,8.292);
\draw[gp path] (10.177,0.985)--(10.177,1.075);
\draw[gp path] (10.177,8.382)--(10.177,8.292);
\draw[gp path] (10.591,0.985)--(10.591,1.075);
\draw[gp path] (10.591,8.382)--(10.591,8.292);
\draw[gp path] (11.005,0.985)--(11.005,1.165);
\draw[gp path] (11.005,8.382)--(11.005,8.202);
\node[gp node center] at (11.005,0.677) {100000};
\draw[gp path] (11.419,0.985)--(11.419,1.075);
\draw[gp path] (11.419,8.382)--(11.419,8.292);
\draw[gp path] (11.833,0.985)--(11.833,1.075);
\draw[gp path] (11.833,8.382)--(11.833,8.292);
\draw[gp path] (1.688,8.382)--(1.688,0.985)--(12.040,0.985)--(12.040,8.382)--cycle;
\node[gp node center,rotate=90] at (0.614,4.683) {Execution time, s};
\node[gp node center] at (6.864,0.215) {Monte Carlo cycles};
\color{gp_lt_color1}
\gpsetlinetype{gp_lt_plot1}
\gpsetlinewidth{5.00}
\draw[gp path] (1.688,1.924)--(2.206,2.009)--(2.723,2.322)--(3.241,2.749)--(3.758,3.033)%
  --(4.276,3.318)--(4.794,3.688)--(5.311,4.001)--(5.829,4.371)--(6.346,4.712)--(6.864,5.053)%
  --(7.382,5.395)--(7.899,5.736)--(8.417,6.106)--(8.934,6.390)--(9.452,6.789)--(9.970,7.272)%
  --(10.487,7.500)--(11.005,7.870)--(11.522,8.297);
\gpsetpointsize{4.00}
\gppoint{gp_mark1}{(1.688,1.924)}
\gppoint{gp_mark1}{(2.206,2.009)}
\gppoint{gp_mark1}{(2.723,2.322)}
\gppoint{gp_mark1}{(3.241,2.749)}
\gppoint{gp_mark1}{(3.758,3.033)}
\gppoint{gp_mark1}{(4.276,3.318)}
\gppoint{gp_mark1}{(4.794,3.688)}
\gppoint{gp_mark1}{(5.311,4.001)}
\gppoint{gp_mark1}{(5.829,4.371)}
\gppoint{gp_mark1}{(6.346,4.712)}
\gppoint{gp_mark1}{(6.864,5.053)}
\gppoint{gp_mark1}{(7.382,5.395)}
\gppoint{gp_mark1}{(7.899,5.736)}
\gppoint{gp_mark1}{(8.417,6.106)}
\gppoint{gp_mark1}{(8.934,6.390)}
\gppoint{gp_mark1}{(9.452,6.789)}
\gppoint{gp_mark1}{(9.970,7.272)}
\gppoint{gp_mark1}{(10.487,7.500)}
\gppoint{gp_mark1}{(11.005,7.870)}
\gppoint{gp_mark1}{(11.522,8.297)}
\color{gp_lt_color_b}
\gpsetlinetype{gp_lt_border}
\gpsetlinewidth{1.00}
\draw[gp path] (1.688,8.382)--(1.688,0.985)--(12.040,0.985)--(12.040,8.382)--cycle;
%% coordinates of the plot area
\coordinate (gpbb south west 1) at (1.688,0.985);
\coordinate (gpbb south east 1) at (12.040,0.985);
\coordinate (gpbb north east 1) at (12.040,8.382);
\coordinate (gpbb north west 1) at (1.688,8.382);
\coordinate (gpbb north 1) at (6.864,8.382);
\coordinate (gpbb south 1) at (6.864,0.985);
\coordinate (gpbb west 1) at (1.688,4.684);
\coordinate (gpbb east 1) at (12.040,4.684);
\end{tikzpicture}
%% gnuplot variables
}
        \end{tabular}
        \caption{\scriptsize Execution time as a function of the number of Monte Carlo cycles for a \textcolor{blue}{Python (blue)} and \textcolor{green}{C++ (green)} simulators implementing the QVMC method with importance sampling for the He atom.}
      \end{figure}
  
 
  \scriptsize{\textcolor{red}{CONCLUSION:}} {\textcolor{blue}{Python}} is {\color{blue}{SLOW}}, except when it is not running!
\end{frame}



\begin{frame}{Detecting bottlenecks in Python}
  \begin{scriptsize}
    \begin{table}
      \centering
      \begin{tabular}{rrrl}
        \toprule[1pt]
        \textbf{\# calls} & \textbf{Total time} & \textbf{Cum. time} & \textbf{Class:function}\\
        \midrule[1pt]
        \scriptsize
              1  &  9.153 &   207.061  & \footnotesize{MonteCarlo.py:(doMCISampling)}\\
              1  &  0.000 &   207.061  & \footnotesize{VMC.py:(doVariationalLoop)}\\
        1910014  & 23.794 &   159.910  & \footnotesize{Psi.py:(getPsiTrial)}\\
          100001 & 12.473 &   117.223  & \footnotesize{Psi.py:(getQuantumForce)}\\
        1910014  & 58.956 &   71.704   & \footnotesize{Psi.py:(getModelWaveFunctionHe)}\\
          50000  &  0.864 &  66.208    & \footnotesize{Energy.py:(getLocalEnergy)}\\
        1910014  & 57.476 &   64.412   & \footnotesize{Psi.py:(getCorrelationFactor)}\\
          50000  &  8.153 &  62.548    & \footnotesize{Energy.py:(getKineticEnergy)}\\
        \textcolor{red}{6180049}  &  \textcolor{red}{21.489} & \textcolor{red}{21.489}    & \textcolor{red}{\footnotesize{:0(sqrt)}}\\
          900002  &  4.968 &  4.968    & \footnotesize{:0(copy)}\\
          300010  &  2.072 &  2.072    & \footnotesize{:0(zeros)}\\
          50000  &  2.272  &  2.796    & \footnotesize{Energy.py:(getPotentialEnergy)}\\
        \bottomrule[1pt]
      \end{tabular}\caption{\scriptsize Profile of a QVMC simulator with importance sampling for the He atom implemented in Python. The run was done with 50000 Monte Carlo cycles.}
      \label{profileHe}
    \end{table}
  \end{scriptsize}
\end{frame}



% % % % % % % % \begin{frame}[fragile]{Detecting bottlenecks in Python}
% % % % % % % %   \begin{scriptsize}
% % % % % % % %     \begin{table}
% % % % % % % %       \centering
% % % % % % % %       \begin{tabular}{rrrl}
% % % % % % % %         \toprule[1pt]
% % % % % % % %         \textbf{\# calls} & \textbf{Total time} & \textbf{Cum. time} & \textbf{Class:function}\\
% % % % % % % %         \midrule[1pt]
% % % % % % % %         \scriptsize
% % % % % % % %               1  & 17.985  & 2417.743 & \footnotesize{MonteCarlo.py:(doMCISampling)}\\
% % % % % % % %               1  &  0.000  & 2417.743 & \footnotesize{VMC.py:(doVariationalLoop)}\\
% % % % % % % %         6220026  & 81.841  & 2305.124 & \footnotesize{Psi.py:(getPsiTrial)}\\
% % % % % % % %           200001 &  41.787 & 1828.758 & \footnotesize{Psi.py:(getQuantumForce)}\\
% % % % % % % %         6220026  & 532.861 & 1171.609 & \footnotesize{Psi.py:(getModelWaveFunctionBe)}\\
% % % % % % % %         6220026  & 921.182 & 1051.674 & \footnotesize{Psi.py:(getCorrelationFactor)}\\
% % % % % % % %           50000  &  0.912  & 477.214  & \footnotesize{Energy.py:(getLocalEnergy)}\\
% % % % % % % %           50000  & 15.313  & 467.341  & \footnotesize{Energy.py:(getKineticEnergy)}\\
% % % % % % % %         24880104 & 295.931 & 295.931  & \footnotesize{Psi.py:(phi2s)}\\
% % % % % % % %         \textcolor{red}{63300273} & \textcolor{red}{220.166} & \textcolor{red}{220.166}  & \textcolor{red}{\footnotesize{:0(sqrt)}}\\
% % % % % % % %         24880104 & 215.998 & 215.998  & \footnotesize{Psi.py:(phi1s)}\\
% % % % % % % %         6820036  & 45.579  & 45.579   & \footnotesize{:0(zeros)}\\
% % % % % % % %         1700002  & 9.369   & 9.369    & \footnotesize{:0(copy)}\\
% % % % % % % %           50000  &  7.108  & 8.961    & \footnotesize{Energy.py:(getPotentialEnergy)}\\
% % % % % % % %         \bottomrule[1pt]
% % % % % % % %       \end{tabular}\caption{\scriptsize Profile of a QVMC simulator with importance sampling for the Be atom implemented in Python. The run was done with 50000 Monte Carlo cycles.}
% % % % % % % %       \label{profileBe}
% % % % % % % %     \end{table}
% % % % % % % %   \end{scriptsize}
% % % % % % % % 
% % % % % % % %   
% % % % % % % % \end{frame}


\subsection{Implementation in mixed Python/C++}
\begin{frame}[fragile]{Can "Python" do better?: Extending Python with C++}
  \begin{scriptsize}
  \begin{Python}
    ...
    sys.path.insert(0, './extensions')  # Set the path to the extensions
    import ext_QVMC                     # Extension module

    class Vmc():
      def __init__(self, _Parameters):
        # Create an object of the 'conversion class' 
        self.convert = ext_QVMC.Convert()
        
        # Get the paramters of the currrent simulation
        simParameters   = _Parameters.getParameters()
        alpha   = simParameters[6]
        self.varpar = array([alpha, beta])

        # Convert a Python array to a MyArray object
        self.v_p = self.convert.py2my_copy(self.varpar)

        # Create objects to be extended in C++
        self.psi    = ext_QVMC.Psi(self.v_p, self.nel, self.nsd)...
  \end{Python}
  \end{scriptsize}
\end{frame}



\begin{frame}[fragile]{Calling code for the mixed Python/C++ simulator}
  \begin{Python}
    import sys

    from SimParameters import * # Class encapsulating the \
                                # parameters of simulation
    from Vmc import *           # Import the simulator box.

    # Create an object containing the 
    # parameters of the current simulation
    simpar = SimParameters('Be.data')

    # Create a Variational Monte Carlo simulation
    vmc = Vmc(simpar)

    vmc.doVariationalLoop()
    vmc.energy.doStatistics("resultsBe.data", 1.0)
  \end{Python}
\end{frame}



\begin{frame}{Comparing Python to C++}
  \begin{figure}
% % %     \begin{tabular}{cc}
% % %       \centering
% % % %       \scalebox{0.45}{\begin{tikzpicture}[gnuplot]
%% generated with GNUPLOT 4.2p5  (Lua 5.1.4; terminal rev. 81, script rev. 88)
%% Wed Sep  9 14:22:33 2009
\color{gp_lt_color_b}
\gpsetlinetype{gp_lt_border}
\gpsetlinewidth{1.00}
\draw[gp path] (1.688,0.985)--(1.868,0.985);
\draw[gp path] (12.040,0.985)--(11.860,0.985);
\node[gp node right] at (1.504,0.985) {0};
\draw[gp path] (1.688,1.281)--(1.778,1.281);
\draw[gp path] (12.040,1.281)--(11.950,1.281);
\draw[gp path] (1.688,1.577)--(1.778,1.577);
\draw[gp path] (12.040,1.577)--(11.950,1.577);
\draw[gp path] (1.688,1.873)--(1.778,1.873);
\draw[gp path] (12.040,1.873)--(11.950,1.873);
\draw[gp path] (1.688,2.169)--(1.778,2.169);
\draw[gp path] (12.040,2.169)--(11.950,2.169);
\draw[gp path] (1.688,2.464)--(1.868,2.464);
\draw[gp path] (12.040,2.464)--(11.860,2.464);
\node[gp node right] at (1.504,2.464) {0.5};
\draw[gp path] (1.688,2.760)--(1.778,2.760);
\draw[gp path] (12.040,2.760)--(11.950,2.760);
\draw[gp path] (1.688,3.056)--(1.778,3.056);
\draw[gp path] (12.040,3.056)--(11.950,3.056);
\draw[gp path] (1.688,3.352)--(1.778,3.352);
\draw[gp path] (12.040,3.352)--(11.950,3.352);
\draw[gp path] (1.688,3.648)--(1.778,3.648);
\draw[gp path] (12.040,3.648)--(11.950,3.648);
\draw[gp path] (1.688,3.944)--(1.868,3.944);
\draw[gp path] (12.040,3.944)--(11.860,3.944);
\node[gp node right] at (1.504,3.944) {1};
\draw[gp path] (1.688,4.240)--(1.778,4.240);
\draw[gp path] (12.040,4.240)--(11.950,4.240);
\draw[gp path] (1.688,4.536)--(1.778,4.536);
\draw[gp path] (12.040,4.536)--(11.950,4.536);
\draw[gp path] (1.688,4.831)--(1.778,4.831);
\draw[gp path] (12.040,4.831)--(11.950,4.831);
\draw[gp path] (1.688,5.127)--(1.778,5.127);
\draw[gp path] (12.040,5.127)--(11.950,5.127);
\draw[gp path] (1.688,5.423)--(1.868,5.423);
\draw[gp path] (12.040,5.423)--(11.860,5.423);
\node[gp node right] at (1.504,5.423) {1.5};
\draw[gp path] (1.688,5.719)--(1.778,5.719);
\draw[gp path] (12.040,5.719)--(11.950,5.719);
\draw[gp path] (1.688,6.015)--(1.778,6.015);
\draw[gp path] (12.040,6.015)--(11.950,6.015);
\draw[gp path] (1.688,6.311)--(1.778,6.311);
\draw[gp path] (12.040,6.311)--(11.950,6.311);
\draw[gp path] (1.688,6.607)--(1.778,6.607);
\draw[gp path] (12.040,6.607)--(11.950,6.607);
\draw[gp path] (1.688,6.903)--(1.868,6.903);
\draw[gp path] (12.040,6.903)--(11.860,6.903);
\node[gp node right] at (1.504,6.903) {2};
\draw[gp path] (1.688,7.198)--(1.778,7.198);
\draw[gp path] (12.040,7.198)--(11.950,7.198);
\draw[gp path] (1.688,7.494)--(1.778,7.494);
\draw[gp path] (12.040,7.494)--(11.950,7.494);
\draw[gp path] (1.688,7.790)--(1.778,7.790);
\draw[gp path] (12.040,7.790)--(11.950,7.790);
\draw[gp path] (1.688,8.086)--(1.778,8.086);
\draw[gp path] (12.040,8.086)--(11.950,8.086);
\draw[gp path] (1.688,8.382)--(1.868,8.382);
\draw[gp path] (12.040,8.382)--(11.860,8.382);
\node[gp node right] at (1.504,8.382) {2.5};
\draw[gp path] (1.902,0.985)--(1.902,1.075);
\draw[gp path] (1.902,8.382)--(1.902,8.292);
\draw[gp path] (2.188,0.985)--(2.188,1.075);
\draw[gp path] (2.188,8.382)--(2.188,8.292);
\draw[gp path] (2.473,0.985)--(2.473,1.075);
\draw[gp path] (2.473,8.382)--(2.473,8.292);
\draw[gp path] (2.759,0.985)--(2.759,1.165);
\draw[gp path] (2.759,8.382)--(2.759,8.202);
\node[gp node center] at (2.759,0.677) {20000};
\draw[gp path] (3.044,0.985)--(3.044,1.075);
\draw[gp path] (3.044,8.382)--(3.044,8.292);
\draw[gp path] (3.330,0.985)--(3.330,1.075);
\draw[gp path] (3.330,8.382)--(3.330,8.292);
\draw[gp path] (3.616,0.985)--(3.616,1.075);
\draw[gp path] (3.616,8.382)--(3.616,8.292);
\draw[gp path] (3.901,0.985)--(3.901,1.075);
\draw[gp path] (3.901,8.382)--(3.901,8.292);
\draw[gp path] (4.187,0.985)--(4.187,1.165);
\draw[gp path] (4.187,8.382)--(4.187,8.202);
\node[gp node center] at (4.187,0.677) {40000};
\draw[gp path] (4.472,0.985)--(4.472,1.075);
\draw[gp path] (4.472,8.382)--(4.472,8.292);
\draw[gp path] (4.758,0.985)--(4.758,1.075);
\draw[gp path] (4.758,8.382)--(4.758,8.292);
\draw[gp path] (5.043,0.985)--(5.043,1.075);
\draw[gp path] (5.043,8.382)--(5.043,8.292);
\draw[gp path] (5.329,0.985)--(5.329,1.075);
\draw[gp path] (5.329,8.382)--(5.329,8.292);
\draw[gp path] (5.615,0.985)--(5.615,1.165);
\draw[gp path] (5.615,8.382)--(5.615,8.202);
\node[gp node center] at (5.615,0.677) {60000};
\draw[gp path] (5.900,0.985)--(5.900,1.075);
\draw[gp path] (5.900,8.382)--(5.900,8.292);
\draw[gp path] (6.186,0.985)--(6.186,1.075);
\draw[gp path] (6.186,8.382)--(6.186,8.292);
\draw[gp path] (6.471,0.985)--(6.471,1.075);
\draw[gp path] (6.471,8.382)--(6.471,8.292);
\draw[gp path] (6.757,0.985)--(6.757,1.075);
\draw[gp path] (6.757,8.382)--(6.757,8.292);
\draw[gp path] (7.042,0.985)--(7.042,1.165);
\draw[gp path] (7.042,8.382)--(7.042,8.202);
\node[gp node center] at (7.042,0.677) {80000};
\draw[gp path] (7.328,0.985)--(7.328,1.075);
\draw[gp path] (7.328,8.382)--(7.328,8.292);
\draw[gp path] (7.614,0.985)--(7.614,1.075);
\draw[gp path] (7.614,8.382)--(7.614,8.292);
\draw[gp path] (7.899,0.985)--(7.899,1.075);
\draw[gp path] (7.899,8.382)--(7.899,8.292);
\draw[gp path] (8.185,0.985)--(8.185,1.075);
\draw[gp path] (8.185,8.382)--(8.185,8.292);
\draw[gp path] (8.470,0.985)--(8.470,1.165);
\draw[gp path] (8.470,8.382)--(8.470,8.202);
\node[gp node center] at (8.470,0.677) {100000};
\draw[gp path] (8.756,0.985)--(8.756,1.075);
\draw[gp path] (8.756,8.382)--(8.756,8.292);
\draw[gp path] (9.041,0.985)--(9.041,1.075);
\draw[gp path] (9.041,8.382)--(9.041,8.292);
\draw[gp path] (9.327,0.985)--(9.327,1.075);
\draw[gp path] (9.327,8.382)--(9.327,8.292);
\draw[gp path] (9.613,0.985)--(9.613,1.075);
\draw[gp path] (9.613,8.382)--(9.613,8.292);
\draw[gp path] (9.898,0.985)--(9.898,1.165);
\draw[gp path] (9.898,8.382)--(9.898,8.202);
\node[gp node center] at (9.898,0.677) {120000};
\draw[gp path] (10.184,0.985)--(10.184,1.075);
\draw[gp path] (10.184,8.382)--(10.184,8.292);
\draw[gp path] (10.469,0.985)--(10.469,1.075);
\draw[gp path] (10.469,8.382)--(10.469,8.292);
\draw[gp path] (10.755,0.985)--(10.755,1.075);
\draw[gp path] (10.755,8.382)--(10.755,8.292);
\draw[gp path] (11.040,0.985)--(11.040,1.075);
\draw[gp path] (11.040,8.382)--(11.040,8.292);
\draw[gp path] (11.326,0.985)--(11.326,1.165);
\draw[gp path] (11.326,8.382)--(11.326,8.202);
\node[gp node center] at (11.326,0.677) {140000};
\draw[gp path] (11.612,0.985)--(11.612,1.075);
\draw[gp path] (11.612,8.382)--(11.612,8.292);
\draw[gp path] (11.897,0.985)--(11.897,1.075);
\draw[gp path] (11.897,8.382)--(11.897,8.292);
\draw[gp path] (1.688,8.382)--(1.688,0.985)--(12.040,0.985)--(12.040,8.382)--cycle;
\node[gp node center,rotate=90] at (0.614,4.683) {Execution time, s};
\node[gp node center] at (6.864,0.215) {Monte Carlo cycles};
\node[gp node right] at (4.045,7.932) {Mixed -O1};
\color{gp_lt_color0}
\gpsetlinetype{gp_lt_plot0}
\gpsetlinewidth{3.00}
\draw[gp path] (4.229,7.932)--(5.145,7.932);
\draw[gp path] (1.688,1.204)--(2.045,1.429)--(2.402,1.642)--(2.759,1.846)--(3.116,2.145)%
  --(3.473,2.278)--(3.830,2.524)--(4.187,2.740)--(4.544,2.964)--(4.901,3.160)--(5.258,3.358)%
  --(5.615,3.621)--(5.972,3.911)--(6.329,4.033)--(6.686,4.243)--(7.042,4.488)--(7.399,4.684)%
  --(7.756,5.065)--(8.113,5.121)--(8.470,5.326)--(8.827,5.533)--(9.184,5.793)--(9.541,5.971)%
  --(9.898,6.169)--(10.255,6.471)--(10.612,6.633)--(10.969,6.829)--(11.326,7.104)--(11.683,7.320)%
  --(12.040,7.530);
\gpsetpointsize{4.00}
\gppoint{gp_mark1}{(1.688,1.204)}
\gppoint{gp_mark1}{(2.045,1.429)}
\gppoint{gp_mark1}{(2.402,1.642)}
\gppoint{gp_mark1}{(2.759,1.846)}
\gppoint{gp_mark1}{(3.116,2.145)}
\gppoint{gp_mark1}{(3.473,2.278)}
\gppoint{gp_mark1}{(3.830,2.524)}
\gppoint{gp_mark1}{(4.187,2.740)}
\gppoint{gp_mark1}{(4.544,2.964)}
\gppoint{gp_mark1}{(4.901,3.160)}
\gppoint{gp_mark1}{(5.258,3.358)}
\gppoint{gp_mark1}{(5.615,3.621)}
\gppoint{gp_mark1}{(5.972,3.911)}
\gppoint{gp_mark1}{(6.329,4.033)}
\gppoint{gp_mark1}{(6.686,4.243)}
\gppoint{gp_mark1}{(7.042,4.488)}
\gppoint{gp_mark1}{(7.399,4.684)}
\gppoint{gp_mark1}{(7.756,5.065)}
\gppoint{gp_mark1}{(8.113,5.121)}
\gppoint{gp_mark1}{(8.470,5.326)}
\gppoint{gp_mark1}{(8.827,5.533)}
\gppoint{gp_mark1}{(9.184,5.793)}
\gppoint{gp_mark1}{(9.541,5.971)}
\gppoint{gp_mark1}{(9.898,6.169)}
\gppoint{gp_mark1}{(10.255,6.471)}
\gppoint{gp_mark1}{(10.612,6.633)}
\gppoint{gp_mark1}{(10.969,6.829)}
\gppoint{gp_mark1}{(11.326,7.104)}
\gppoint{gp_mark1}{(11.683,7.320)}
\gppoint{gp_mark1}{(12.040,7.530)}
\gppoint{gp_mark1}{(4.687,7.932)}
\color{gp_lt_color_b}
\node[gp node right] at (4.045,7.624) {C++ -O1};
\color{gp_lt_color1}
\gpsetlinetype{gp_lt_plot1}
\draw[gp path] (4.229,7.624)--(5.145,7.624);
\draw[gp path] (1.688,1.192)--(2.045,1.429)--(2.402,1.666)--(2.759,1.843)--(3.116,2.109)%
  --(3.473,2.346)--(3.830,2.524)--(4.187,2.731)--(4.544,2.967)--(4.901,3.175)--(5.258,3.441)%
  --(5.615,3.678)--(5.972,3.944)--(6.329,4.062)--(6.686,4.328)--(7.042,4.595)--(7.399,4.802)%
  --(7.756,5.039)--(8.113,5.246)--(8.470,5.423)--(8.827,5.719)--(9.184,5.867)--(9.541,6.045)%
  --(9.898,6.311)--(10.255,6.548)--(10.612,6.695)--(10.969,6.873)--(11.326,7.198)--(11.683,7.346)%
  --(12.040,7.583);
\gppoint{gp_mark2}{(1.688,1.192)}
\gppoint{gp_mark2}{(2.045,1.429)}
\gppoint{gp_mark2}{(2.402,1.666)}
\gppoint{gp_mark2}{(2.759,1.843)}
\gppoint{gp_mark2}{(3.116,2.109)}
\gppoint{gp_mark2}{(3.473,2.346)}
\gppoint{gp_mark2}{(3.830,2.524)}
\gppoint{gp_mark2}{(4.187,2.731)}
\gppoint{gp_mark2}{(4.544,2.967)}
\gppoint{gp_mark2}{(4.901,3.175)}
\gppoint{gp_mark2}{(5.258,3.441)}
\gppoint{gp_mark2}{(5.615,3.678)}
\gppoint{gp_mark2}{(5.972,3.944)}
\gppoint{gp_mark2}{(6.329,4.062)}
\gppoint{gp_mark2}{(6.686,4.328)}
\gppoint{gp_mark2}{(7.042,4.595)}
\gppoint{gp_mark2}{(7.399,4.802)}
\gppoint{gp_mark2}{(7.756,5.039)}
\gppoint{gp_mark2}{(8.113,5.246)}
\gppoint{gp_mark2}{(8.470,5.423)}
\gppoint{gp_mark2}{(8.827,5.719)}
\gppoint{gp_mark2}{(9.184,5.867)}
\gppoint{gp_mark2}{(9.541,6.045)}
\gppoint{gp_mark2}{(9.898,6.311)}
\gppoint{gp_mark2}{(10.255,6.548)}
\gppoint{gp_mark2}{(10.612,6.695)}
\gppoint{gp_mark2}{(10.969,6.873)}
\gppoint{gp_mark2}{(11.326,7.198)}
\gppoint{gp_mark2}{(11.683,7.346)}
\gppoint{gp_mark2}{(12.040,7.583)}
\gppoint{gp_mark2}{(4.687,7.624)}
\color{gp_lt_color_b}
\node[gp node right] at (4.045,7.316) {Mixed -O2};
\color{gp_lt_color2}
\gpsetlinetype{gp_lt_plot2}
\draw[gp path] (4.229,7.316)--(5.145,7.316);
\draw[gp path] (1.688,1.198)--(2.045,1.405)--(2.402,1.621)--(2.759,1.828)--(3.116,2.059)%
  --(3.473,2.251)--(3.830,2.476)--(4.187,2.680)--(4.544,2.905)--(4.901,3.098)--(5.258,3.302)%
  --(5.615,3.509)--(5.972,3.731)--(6.329,3.944)--(6.686,4.195)--(7.042,4.373)--(7.399,4.592)%
  --(7.756,4.775)--(8.113,5.006)--(8.470,5.237)--(8.827,5.417)--(9.184,5.618)--(9.541,5.879)%
  --(9.898,6.059)--(10.255,6.272)--(10.612,6.462)--(10.969,6.755)--(11.326,6.885)--(11.683,7.113)%
  --(12.040,7.349);
\gppoint{gp_mark3}{(1.688,1.198)}
\gppoint{gp_mark3}{(2.045,1.405)}
\gppoint{gp_mark3}{(2.402,1.621)}
\gppoint{gp_mark3}{(2.759,1.828)}
\gppoint{gp_mark3}{(3.116,2.059)}
\gppoint{gp_mark3}{(3.473,2.251)}
\gppoint{gp_mark3}{(3.830,2.476)}
\gppoint{gp_mark3}{(4.187,2.680)}
\gppoint{gp_mark3}{(4.544,2.905)}
\gppoint{gp_mark3}{(4.901,3.098)}
\gppoint{gp_mark3}{(5.258,3.302)}
\gppoint{gp_mark3}{(5.615,3.509)}
\gppoint{gp_mark3}{(5.972,3.731)}
\gppoint{gp_mark3}{(6.329,3.944)}
\gppoint{gp_mark3}{(6.686,4.195)}
\gppoint{gp_mark3}{(7.042,4.373)}
\gppoint{gp_mark3}{(7.399,4.592)}
\gppoint{gp_mark3}{(7.756,4.775)}
\gppoint{gp_mark3}{(8.113,5.006)}
\gppoint{gp_mark3}{(8.470,5.237)}
\gppoint{gp_mark3}{(8.827,5.417)}
\gppoint{gp_mark3}{(9.184,5.618)}
\gppoint{gp_mark3}{(9.541,5.879)}
\gppoint{gp_mark3}{(9.898,6.059)}
\gppoint{gp_mark3}{(10.255,6.272)}
\gppoint{gp_mark3}{(10.612,6.462)}
\gppoint{gp_mark3}{(10.969,6.755)}
\gppoint{gp_mark3}{(11.326,6.885)}
\gppoint{gp_mark3}{(11.683,7.113)}
\gppoint{gp_mark3}{(12.040,7.349)}
\gppoint{gp_mark3}{(4.687,7.316)}
\color{gp_lt_color_b}
\node[gp node right] at (4.045,7.008) {C++ -O2};
\color{gp_lt_color3}
\gpsetlinetype{gp_lt_plot3}
\gpsetlinewidth{5.00}
\draw[gp path] (4.229,7.008)--(5.145,7.008);
\draw[gp path] (1.688,1.458)--(2.045,1.399)--(2.402,1.606)--(2.759,1.813)--(3.116,2.021)%
  --(3.473,2.257)--(3.830,2.435)--(4.187,2.672)--(4.544,2.849)--(4.901,3.086)--(5.258,3.263)%
  --(5.615,3.500)--(5.972,3.737)--(6.329,3.914)--(6.686,4.121)--(7.042,4.299)--(7.399,4.536)%
  --(7.756,4.802)--(8.113,4.979)--(8.470,5.157)--(8.827,5.364)--(9.184,5.601)--(9.541,5.778)%
  --(9.898,6.015)--(10.255,6.192)--(10.612,6.459)--(10.969,6.607)--(11.326,6.873)--(11.683,7.080)%
  --(12.040,7.258);
\gppoint{gp_mark4}{(1.688,1.458)}
\gppoint{gp_mark4}{(2.045,1.399)}
\gppoint{gp_mark4}{(2.402,1.606)}
\gppoint{gp_mark4}{(2.759,1.813)}
\gppoint{gp_mark4}{(3.116,2.021)}
\gppoint{gp_mark4}{(3.473,2.257)}
\gppoint{gp_mark4}{(3.830,2.435)}
\gppoint{gp_mark4}{(4.187,2.672)}
\gppoint{gp_mark4}{(4.544,2.849)}
\gppoint{gp_mark4}{(4.901,3.086)}
\gppoint{gp_mark4}{(5.258,3.263)}
\gppoint{gp_mark4}{(5.615,3.500)}
\gppoint{gp_mark4}{(5.972,3.737)}
\gppoint{gp_mark4}{(6.329,3.914)}
\gppoint{gp_mark4}{(6.686,4.121)}
\gppoint{gp_mark4}{(7.042,4.299)}
\gppoint{gp_mark4}{(7.399,4.536)}
\gppoint{gp_mark4}{(7.756,4.802)}
\gppoint{gp_mark4}{(8.113,4.979)}
\gppoint{gp_mark4}{(8.470,5.157)}
\gppoint{gp_mark4}{(8.827,5.364)}
\gppoint{gp_mark4}{(9.184,5.601)}
\gppoint{gp_mark4}{(9.541,5.778)}
\gppoint{gp_mark4}{(9.898,6.015)}
\gppoint{gp_mark4}{(10.255,6.192)}
\gppoint{gp_mark4}{(10.612,6.459)}
\gppoint{gp_mark4}{(10.969,6.607)}
\gppoint{gp_mark4}{(11.326,6.873)}
\gppoint{gp_mark4}{(11.683,7.080)}
\gppoint{gp_mark4}{(12.040,7.258)}
\gppoint{gp_mark4}{(4.687,7.008)}
\color{gp_lt_color_b}
\node[gp node right] at (4.045,6.700) {Mixed -O3};
\color{gp_lt_color4}
\gpsetlinetype{gp_lt_plot4}
\gpsetlinewidth{3.00}
\draw[gp path] (4.229,6.700)--(5.145,6.700);
\draw[gp path] (1.688,1.189)--(2.045,1.393)--(2.402,1.600)--(2.759,1.805)--(3.116,2.029)%
  --(3.473,2.210)--(3.830,2.417)--(4.187,2.621)--(4.544,2.828)--(4.901,3.062)--(5.258,3.260)%
  --(5.615,3.447)--(5.972,3.686)--(6.329,3.843)--(6.686,4.065)--(7.042,4.311)--(7.399,4.465)%
  --(7.756,4.684)--(8.113,4.876)--(8.470,5.092)--(8.827,5.331)--(9.184,5.465)--(9.541,5.707)%
  --(9.898,5.956)--(10.255,6.124)--(10.612,6.343)--(10.969,6.530)--(11.326,6.802)--(11.683,6.965)%
  --(12.040,7.127);
\gppoint{gp_mark5}{(1.688,1.189)}
\gppoint{gp_mark5}{(2.045,1.393)}
\gppoint{gp_mark5}{(2.402,1.600)}
\gppoint{gp_mark5}{(2.759,1.805)}
\gppoint{gp_mark5}{(3.116,2.029)}
\gppoint{gp_mark5}{(3.473,2.210)}
\gppoint{gp_mark5}{(3.830,2.417)}
\gppoint{gp_mark5}{(4.187,2.621)}
\gppoint{gp_mark5}{(4.544,2.828)}
\gppoint{gp_mark5}{(4.901,3.062)}
\gppoint{gp_mark5}{(5.258,3.260)}
\gppoint{gp_mark5}{(5.615,3.447)}
\gppoint{gp_mark5}{(5.972,3.686)}
\gppoint{gp_mark5}{(6.329,3.843)}
\gppoint{gp_mark5}{(6.686,4.065)}
\gppoint{gp_mark5}{(7.042,4.311)}
\gppoint{gp_mark5}{(7.399,4.465)}
\gppoint{gp_mark5}{(7.756,4.684)}
\gppoint{gp_mark5}{(8.113,4.876)}
\gppoint{gp_mark5}{(8.470,5.092)}
\gppoint{gp_mark5}{(8.827,5.331)}
\gppoint{gp_mark5}{(9.184,5.465)}
\gppoint{gp_mark5}{(9.541,5.707)}
\gppoint{gp_mark5}{(9.898,5.956)}
\gppoint{gp_mark5}{(10.255,6.124)}
\gppoint{gp_mark5}{(10.612,6.343)}
\gppoint{gp_mark5}{(10.969,6.530)}
\gppoint{gp_mark5}{(11.326,6.802)}
\gppoint{gp_mark5}{(11.683,6.965)}
\gppoint{gp_mark5}{(12.040,7.127)}
\gppoint{gp_mark5}{(4.687,6.700)}
\color{gp_lt_color_b}
\node[gp node right] at (4.045,6.392) {C++ -O3};
\color{gp_lt_color5}
\gpsetlinetype{gp_lt_plot5}
\draw[gp path] (4.229,6.392)--(5.145,6.392);
\draw[gp path] (1.688,1.163)--(2.045,1.399)--(2.402,1.606)--(2.759,1.784)--(3.116,2.021)%
  --(3.473,2.198)--(3.830,2.435)--(4.187,2.612)--(4.544,2.819)--(4.901,3.027)--(5.258,3.293)%
  --(5.615,3.441)--(5.972,3.648)--(6.329,3.885)--(6.686,4.062)--(7.042,4.269)--(7.399,4.476)%
  --(7.756,4.684)--(8.113,4.891)--(8.470,5.098)--(8.827,5.305)--(9.184,5.512)--(9.541,5.689)%
  --(9.898,5.926)--(10.255,6.192)--(10.612,6.311)--(10.969,6.518)--(11.326,6.755)--(11.683,6.932)%
  --(12.040,7.139);
\gppoint{gp_mark6}{(1.688,1.163)}
\gppoint{gp_mark6}{(2.045,1.399)}
\gppoint{gp_mark6}{(2.402,1.606)}
\gppoint{gp_mark6}{(2.759,1.784)}
\gppoint{gp_mark6}{(3.116,2.021)}
\gppoint{gp_mark6}{(3.473,2.198)}
\gppoint{gp_mark6}{(3.830,2.435)}
\gppoint{gp_mark6}{(4.187,2.612)}
\gppoint{gp_mark6}{(4.544,2.819)}
\gppoint{gp_mark6}{(4.901,3.027)}
\gppoint{gp_mark6}{(5.258,3.293)}
\gppoint{gp_mark6}{(5.615,3.441)}
\gppoint{gp_mark6}{(5.972,3.648)}
\gppoint{gp_mark6}{(6.329,3.885)}
\gppoint{gp_mark6}{(6.686,4.062)}
\gppoint{gp_mark6}{(7.042,4.269)}
\gppoint{gp_mark6}{(7.399,4.476)}
\gppoint{gp_mark6}{(7.756,4.684)}
\gppoint{gp_mark6}{(8.113,4.891)}
\gppoint{gp_mark6}{(8.470,5.098)}
\gppoint{gp_mark6}{(8.827,5.305)}
\gppoint{gp_mark6}{(9.184,5.512)}
\gppoint{gp_mark6}{(9.541,5.689)}
\gppoint{gp_mark6}{(9.898,5.926)}
\gppoint{gp_mark6}{(10.255,6.192)}
\gppoint{gp_mark6}{(10.612,6.311)}
\gppoint{gp_mark6}{(10.969,6.518)}
\gppoint{gp_mark6}{(11.326,6.755)}
\gppoint{gp_mark6}{(11.683,6.932)}
\gppoint{gp_mark6}{(12.040,7.139)}
\gppoint{gp_mark6}{(4.687,6.392)}
\color{gp_lt_color_b}
\gpsetlinetype{gp_lt_border}
\gpsetlinewidth{1.00}
\draw[gp path] (1.688,8.382)--(1.688,0.985)--(12.040,0.985)--(12.040,8.382)--cycle;
%% coordinates of the plot area
\coordinate (gpbb south west 1) at (1.688,0.985);
\coordinate (gpbb south east 1) at (12.040,0.985);
\coordinate (gpbb north east 1) at (12.040,8.382);
\coordinate (gpbb north west 1) at (1.688,8.382);
\coordinate (gpbb north 1) at (6.864,8.382);
\coordinate (gpbb south 1) at (6.864,0.985);
\coordinate (gpbb west 1) at (1.688,4.684);
\coordinate (gpbb east 1) at (12.040,4.684);
\end{tikzpicture}
%% gnuplot variables
}&
      \scalebox{0.55}{\begin{tikzpicture}[gnuplot]
%% generated with GNUPLOT 4.2p5  (Lua 5.1.4; terminal rev. 81, script rev. 88)
%% Wed Sep  9 14:22:33 2009
\color{gp_lt_color_b}
\gpsetlinetype{gp_lt_border}
\gpsetlinewidth{1.00}
\draw[gp path] (1.504,0.985)--(1.684,0.985);
\draw[gp path] (12.040,0.985)--(11.860,0.985);
\node[gp node right] at (1.320,0.985) {0};
\draw[gp path] (1.504,1.232)--(1.594,1.232);
\draw[gp path] (12.040,1.232)--(11.950,1.232);
\draw[gp path] (1.504,1.478)--(1.594,1.478);
\draw[gp path] (12.040,1.478)--(11.950,1.478);
\draw[gp path] (1.504,1.725)--(1.594,1.725);
\draw[gp path] (12.040,1.725)--(11.950,1.725);
\draw[gp path] (1.504,1.971)--(1.594,1.971);
\draw[gp path] (12.040,1.971)--(11.950,1.971);
\draw[gp path] (1.504,2.218)--(1.684,2.218);
\draw[gp path] (12.040,2.218)--(11.860,2.218);
\node[gp node right] at (1.320,2.218) {5};
\draw[gp path] (1.504,2.464)--(1.594,2.464);
\draw[gp path] (12.040,2.464)--(11.950,2.464);
\draw[gp path] (1.504,2.711)--(1.594,2.711);
\draw[gp path] (12.040,2.711)--(11.950,2.711);
\draw[gp path] (1.504,2.958)--(1.594,2.958);
\draw[gp path] (12.040,2.958)--(11.950,2.958);
\draw[gp path] (1.504,3.204)--(1.594,3.204);
\draw[gp path] (12.040,3.204)--(11.950,3.204);
\draw[gp path] (1.504,3.451)--(1.684,3.451);
\draw[gp path] (12.040,3.451)--(11.860,3.451);
\node[gp node right] at (1.320,3.451) {10};
\draw[gp path] (1.504,3.697)--(1.594,3.697);
\draw[gp path] (12.040,3.697)--(11.950,3.697);
\draw[gp path] (1.504,3.944)--(1.594,3.944);
\draw[gp path] (12.040,3.944)--(11.950,3.944);
\draw[gp path] (1.504,4.190)--(1.594,4.190);
\draw[gp path] (12.040,4.190)--(11.950,4.190);
\draw[gp path] (1.504,4.437)--(1.594,4.437);
\draw[gp path] (12.040,4.437)--(11.950,4.437);
\draw[gp path] (1.504,4.683)--(1.684,4.683);
\draw[gp path] (12.040,4.683)--(11.860,4.683);
\node[gp node right] at (1.320,4.683) {15};
\draw[gp path] (1.504,4.930)--(1.594,4.930);
\draw[gp path] (12.040,4.930)--(11.950,4.930);
\draw[gp path] (1.504,5.177)--(1.594,5.177);
\draw[gp path] (12.040,5.177)--(11.950,5.177);
\draw[gp path] (1.504,5.423)--(1.594,5.423);
\draw[gp path] (12.040,5.423)--(11.950,5.423);
\draw[gp path] (1.504,5.670)--(1.594,5.670);
\draw[gp path] (12.040,5.670)--(11.950,5.670);
\draw[gp path] (1.504,5.916)--(1.684,5.916);
\draw[gp path] (12.040,5.916)--(11.860,5.916);
\node[gp node right] at (1.320,5.916) {20};
\draw[gp path] (1.504,6.163)--(1.594,6.163);
\draw[gp path] (12.040,6.163)--(11.950,6.163);
\draw[gp path] (1.504,6.409)--(1.594,6.409);
\draw[gp path] (12.040,6.409)--(11.950,6.409);
\draw[gp path] (1.504,6.656)--(1.594,6.656);
\draw[gp path] (12.040,6.656)--(11.950,6.656);
\draw[gp path] (1.504,6.903)--(1.594,6.903);
\draw[gp path] (12.040,6.903)--(11.950,6.903);
\draw[gp path] (1.504,7.149)--(1.684,7.149);
\draw[gp path] (12.040,7.149)--(11.860,7.149);
\node[gp node right] at (1.320,7.149) {25};
\draw[gp path] (1.504,7.396)--(1.594,7.396);
\draw[gp path] (12.040,7.396)--(11.950,7.396);
\draw[gp path] (1.504,7.642)--(1.594,7.642);
\draw[gp path] (12.040,7.642)--(11.950,7.642);
\draw[gp path] (1.504,7.889)--(1.594,7.889);
\draw[gp path] (12.040,7.889)--(11.950,7.889);
\draw[gp path] (1.504,8.135)--(1.594,8.135);
\draw[gp path] (12.040,8.135)--(11.950,8.135);
\draw[gp path] (1.504,8.382)--(1.684,8.382);
\draw[gp path] (12.040,8.382)--(11.860,8.382);
\node[gp node right] at (1.320,8.382) {30};
\draw[gp path] (1.819,0.985)--(1.819,1.075);
\draw[gp path] (1.819,8.382)--(1.819,8.292);
\draw[gp path] (2.343,0.985)--(2.343,1.075);
\draw[gp path] (2.343,8.382)--(2.343,8.292);
\draw[gp path] (2.867,0.985)--(2.867,1.075);
\draw[gp path] (2.867,8.382)--(2.867,8.292);
\draw[gp path] (3.391,0.985)--(3.391,1.075);
\draw[gp path] (3.391,8.382)--(3.391,8.292);
\draw[gp path] (3.915,0.985)--(3.915,1.165);
\draw[gp path] (3.915,8.382)--(3.915,8.202);
\node[gp node center] at (3.915,0.677) {500000};
\draw[gp path] (4.439,0.985)--(4.439,1.075);
\draw[gp path] (4.439,8.382)--(4.439,8.292);
\draw[gp path] (4.964,0.985)--(4.964,1.075);
\draw[gp path] (4.964,8.382)--(4.964,8.292);
\draw[gp path] (5.488,0.985)--(5.488,1.075);
\draw[gp path] (5.488,8.382)--(5.488,8.292);
\draw[gp path] (6.012,0.985)--(6.012,1.075);
\draw[gp path] (6.012,8.382)--(6.012,8.292);
\draw[gp path] (6.536,0.985)--(6.536,1.165);
\draw[gp path] (6.536,8.382)--(6.536,8.202);
\node[gp node center] at (6.536,0.677) {1e+06};
\draw[gp path] (7.060,0.985)--(7.060,1.075);
\draw[gp path] (7.060,8.382)--(7.060,8.292);
\draw[gp path] (7.584,0.985)--(7.584,1.075);
\draw[gp path] (7.584,8.382)--(7.584,8.292);
\draw[gp path] (8.109,0.985)--(8.109,1.075);
\draw[gp path] (8.109,8.382)--(8.109,8.292);
\draw[gp path] (8.633,0.985)--(8.633,1.075);
\draw[gp path] (8.633,8.382)--(8.633,8.292);
\draw[gp path] (9.157,0.985)--(9.157,1.165);
\draw[gp path] (9.157,8.382)--(9.157,8.202);
\node[gp node center] at (9.157,0.677) {1.5e+06};
\draw[gp path] (9.681,0.985)--(9.681,1.075);
\draw[gp path] (9.681,8.382)--(9.681,8.292);
\draw[gp path] (10.205,0.985)--(10.205,1.075);
\draw[gp path] (10.205,8.382)--(10.205,8.292);
\draw[gp path] (10.730,0.985)--(10.730,1.075);
\draw[gp path] (10.730,8.382)--(10.730,8.292);
\draw[gp path] (11.254,0.985)--(11.254,1.075);
\draw[gp path] (11.254,8.382)--(11.254,8.292);
\draw[gp path] (11.778,0.985)--(11.778,1.165);
\draw[gp path] (11.778,8.382)--(11.778,8.202);
\node[gp node center] at (11.778,0.677) {2e+06};
\draw[gp path] (1.504,8.382)--(1.504,0.985)--(12.040,0.985)--(12.040,8.382)--cycle;
\node[gp node center,rotate=90] at (0.614,4.683) {Execution time, s};
\node[gp node center] at (6.772,0.215) {Monte Carlo cycles};
\node[gp node right] at (3.680,7.735) {Mixed -O1};
\color{gp_lt_color0}
\gpsetlinetype{gp_lt_plot0}
\gpsetlinewidth{3.00}
\draw[gp path] (3.864,7.735)--(4.780,7.735);
\draw[gp path] (1.504,1.131)--(1.714,1.277)--(1.923,1.426)--(2.133,1.564)--(2.343,1.723)%
  --(2.552,1.852)--(2.762,2.020)--(2.972,2.140)--(3.181,2.306)--(3.391,2.431)--(3.601,2.581)%
  --(3.810,2.736)--(4.020,2.898)--(4.230,3.025)--(4.439,3.198)--(4.649,3.295)--(4.859,3.476)%
  --(5.068,3.610)--(5.278,3.733)--(5.488,3.877)--(5.697,4.015)--(5.907,4.159)--(6.117,4.381)%
  --(6.326,4.468)--(6.536,4.659)--(6.746,4.739)--(6.955,4.982)--(7.165,5.169)--(7.375,5.216)%
  --(7.584,5.344)--(7.794,5.450)--(8.004,5.611)--(8.213,5.750)--(8.423,5.889)--(8.633,6.060)%
  --(8.843,6.195)--(9.052,6.338)--(9.262,6.493)--(9.472,6.636)--(9.681,6.895)--(9.891,6.928)%
  --(10.101,7.035)--(10.310,7.208)--(10.520,7.321)--(10.730,7.478)--(10.939,7.623)--(11.149,7.771)%
  --(11.359,8.007)--(11.568,8.087)--(11.778,8.243)--(11.988,8.340);
\gpsetpointsize{4.00}
\gppoint{gp_mark1}{(1.504,1.131)}
\gppoint{gp_mark1}{(1.714,1.277)}
\gppoint{gp_mark1}{(1.923,1.426)}
\gppoint{gp_mark1}{(2.133,1.564)}
\gppoint{gp_mark1}{(2.343,1.723)}
\gppoint{gp_mark1}{(2.552,1.852)}
\gppoint{gp_mark1}{(2.762,2.020)}
\gppoint{gp_mark1}{(2.972,2.140)}
\gppoint{gp_mark1}{(3.181,2.306)}
\gppoint{gp_mark1}{(3.391,2.431)}
\gppoint{gp_mark1}{(3.601,2.581)}
\gppoint{gp_mark1}{(3.810,2.736)}
\gppoint{gp_mark1}{(4.020,2.898)}
\gppoint{gp_mark1}{(4.230,3.025)}
\gppoint{gp_mark1}{(4.439,3.198)}
\gppoint{gp_mark1}{(4.649,3.295)}
\gppoint{gp_mark1}{(4.859,3.476)}
\gppoint{gp_mark1}{(5.068,3.610)}
\gppoint{gp_mark1}{(5.278,3.733)}
\gppoint{gp_mark1}{(5.488,3.877)}
\gppoint{gp_mark1}{(5.697,4.015)}
\gppoint{gp_mark1}{(5.907,4.159)}
\gppoint{gp_mark1}{(6.117,4.381)}
\gppoint{gp_mark1}{(6.326,4.468)}
\gppoint{gp_mark1}{(6.536,4.659)}
\gppoint{gp_mark1}{(6.746,4.739)}
\gppoint{gp_mark1}{(6.955,4.982)}
\gppoint{gp_mark1}{(7.165,5.169)}
\gppoint{gp_mark1}{(7.375,5.216)}
\gppoint{gp_mark1}{(7.584,5.344)}
\gppoint{gp_mark1}{(7.794,5.450)}
\gppoint{gp_mark1}{(8.004,5.611)}
\gppoint{gp_mark1}{(8.213,5.750)}
\gppoint{gp_mark1}{(8.423,5.889)}
\gppoint{gp_mark1}{(8.633,6.060)}
\gppoint{gp_mark1}{(8.843,6.195)}
\gppoint{gp_mark1}{(9.052,6.338)}
\gppoint{gp_mark1}{(9.262,6.493)}
\gppoint{gp_mark1}{(9.472,6.636)}
\gppoint{gp_mark1}{(9.681,6.895)}
\gppoint{gp_mark1}{(9.891,6.928)}
\gppoint{gp_mark1}{(10.101,7.035)}
\gppoint{gp_mark1}{(10.310,7.208)}
\gppoint{gp_mark1}{(10.520,7.321)}
\gppoint{gp_mark1}{(10.730,7.478)}
\gppoint{gp_mark1}{(10.939,7.623)}
\gppoint{gp_mark1}{(11.149,7.771)}
\gppoint{gp_mark1}{(11.359,8.007)}
\gppoint{gp_mark1}{(11.568,8.087)}
\gppoint{gp_mark1}{(11.778,8.243)}
\gppoint{gp_mark1}{(11.988,8.340)}
\gppoint{gp_mark1}{(4.322,7.735)}
\color{gp_lt_color_b}
\node[gp node right] at (3.680,7.427) {C++ -O1};
\color{gp_lt_color1}
\gpsetlinetype{gp_lt_plot1}
\draw[gp path] (3.864,7.427)--(4.780,7.427);
\draw[gp path] (1.504,1.135)--(1.714,1.286)--(1.923,1.441)--(2.133,1.589)--(2.343,1.732)%
  --(2.552,1.870)--(2.762,2.033)--(2.972,2.183)--(3.181,2.326)--(3.391,2.477)--(3.601,2.627)%
  --(3.810,2.780)--(4.020,2.925)--(4.230,3.071)--(4.439,3.224)--(4.649,3.382)--(4.859,3.493)%
  --(5.068,3.643)--(5.278,3.769)--(5.488,3.924)--(5.697,4.065)--(5.907,4.227)--(6.117,4.343)%
  --(6.326,4.612)--(6.536,4.716)--(6.746,4.863)--(6.955,5.024)--(7.165,5.164)--(7.375,5.312)%
  --(7.584,5.487)--(7.794,5.613)--(8.004,5.783)--(8.213,5.911)--(8.423,6.057)--(8.633,6.323)%
  --(8.843,6.252)--(9.052,6.382)--(9.262,6.543)--(9.472,6.673)--(9.681,6.826)--(9.891,7.021)%
  --(10.101,7.191)--(10.310,7.287)--(10.520,7.448)--(10.730,7.593)--(10.939,7.761)--(11.149,7.849)%
  --(11.359,8.044)--(11.568,8.303)--(11.686,8.382);
\gppoint{gp_mark2}{(1.504,1.135)}
\gppoint{gp_mark2}{(1.714,1.286)}
\gppoint{gp_mark2}{(1.923,1.441)}
\gppoint{gp_mark2}{(2.133,1.589)}
\gppoint{gp_mark2}{(2.343,1.732)}
\gppoint{gp_mark2}{(2.552,1.870)}
\gppoint{gp_mark2}{(2.762,2.033)}
\gppoint{gp_mark2}{(2.972,2.183)}
\gppoint{gp_mark2}{(3.181,2.326)}
\gppoint{gp_mark2}{(3.391,2.477)}
\gppoint{gp_mark2}{(3.601,2.627)}
\gppoint{gp_mark2}{(3.810,2.780)}
\gppoint{gp_mark2}{(4.020,2.925)}
\gppoint{gp_mark2}{(4.230,3.071)}
\gppoint{gp_mark2}{(4.439,3.224)}
\gppoint{gp_mark2}{(4.649,3.382)}
\gppoint{gp_mark2}{(4.859,3.493)}
\gppoint{gp_mark2}{(5.068,3.643)}
\gppoint{gp_mark2}{(5.278,3.769)}
\gppoint{gp_mark2}{(5.488,3.924)}
\gppoint{gp_mark2}{(5.697,4.065)}
\gppoint{gp_mark2}{(5.907,4.227)}
\gppoint{gp_mark2}{(6.117,4.343)}
\gppoint{gp_mark2}{(6.326,4.612)}
\gppoint{gp_mark2}{(6.536,4.716)}
\gppoint{gp_mark2}{(6.746,4.863)}
\gppoint{gp_mark2}{(6.955,5.024)}
\gppoint{gp_mark2}{(7.165,5.164)}
\gppoint{gp_mark2}{(7.375,5.312)}
\gppoint{gp_mark2}{(7.584,5.487)}
\gppoint{gp_mark2}{(7.794,5.613)}
\gppoint{gp_mark2}{(8.004,5.783)}
\gppoint{gp_mark2}{(8.213,5.911)}
\gppoint{gp_mark2}{(8.423,6.057)}
\gppoint{gp_mark2}{(8.633,6.323)}
\gppoint{gp_mark2}{(8.843,6.252)}
\gppoint{gp_mark2}{(9.052,6.382)}
\gppoint{gp_mark2}{(9.262,6.543)}
\gppoint{gp_mark2}{(9.472,6.673)}
\gppoint{gp_mark2}{(9.681,6.826)}
\gppoint{gp_mark2}{(9.891,7.021)}
\gppoint{gp_mark2}{(10.101,7.191)}
\gppoint{gp_mark2}{(10.310,7.287)}
\gppoint{gp_mark2}{(10.520,7.448)}
\gppoint{gp_mark2}{(10.730,7.593)}
\gppoint{gp_mark2}{(10.939,7.761)}
\gppoint{gp_mark2}{(11.149,7.849)}
\gppoint{gp_mark2}{(11.359,8.044)}
\gppoint{gp_mark2}{(11.568,8.303)}
\gppoint{gp_mark2}{(4.322,7.427)}
\color{gp_lt_color_b}
\node[gp node right] at (3.680,7.119) {Mixed -O2};
\color{gp_lt_color2}
\gpsetlinetype{gp_lt_plot2}
\draw[gp path] (3.864,7.119)--(4.780,7.119);
\draw[gp path] (1.504,1.127)--(1.714,1.267)--(1.923,1.407)--(2.133,1.557)--(2.343,1.690)%
  --(2.552,1.835)--(2.762,1.969)--(2.972,2.113)--(3.181,2.253)--(3.391,2.392)--(3.601,2.530)%
  --(3.810,2.673)--(4.020,2.823)--(4.230,2.957)--(4.439,3.092)--(4.649,3.246)--(4.859,3.372)%
  --(5.068,3.525)--(5.278,3.661)--(5.488,3.798)--(5.697,3.955)--(5.907,4.084)--(6.117,4.229)%
  --(6.326,4.372)--(6.536,4.501)--(6.746,4.650)--(6.955,4.786)--(7.165,4.932)--(7.375,5.062)%
  --(7.584,5.221)--(7.794,5.345)--(8.004,5.482)--(8.213,5.632)--(8.423,5.775)--(8.633,5.907)%
  --(8.843,6.048)--(9.052,6.188)--(9.262,6.356)--(9.472,6.477)--(9.681,6.620)--(9.891,6.770)%
  --(10.101,6.902)--(10.310,7.043)--(10.520,7.187)--(10.730,7.353)--(10.939,7.444)--(11.149,7.595)%
  --(11.359,7.742)--(11.568,7.891)--(11.778,8.004)--(11.988,8.156);
\gppoint{gp_mark3}{(1.504,1.127)}
\gppoint{gp_mark3}{(1.714,1.267)}
\gppoint{gp_mark3}{(1.923,1.407)}
\gppoint{gp_mark3}{(2.133,1.557)}
\gppoint{gp_mark3}{(2.343,1.690)}
\gppoint{gp_mark3}{(2.552,1.835)}
\gppoint{gp_mark3}{(2.762,1.969)}
\gppoint{gp_mark3}{(2.972,2.113)}
\gppoint{gp_mark3}{(3.181,2.253)}
\gppoint{gp_mark3}{(3.391,2.392)}
\gppoint{gp_mark3}{(3.601,2.530)}
\gppoint{gp_mark3}{(3.810,2.673)}
\gppoint{gp_mark3}{(4.020,2.823)}
\gppoint{gp_mark3}{(4.230,2.957)}
\gppoint{gp_mark3}{(4.439,3.092)}
\gppoint{gp_mark3}{(4.649,3.246)}
\gppoint{gp_mark3}{(4.859,3.372)}
\gppoint{gp_mark3}{(5.068,3.525)}
\gppoint{gp_mark3}{(5.278,3.661)}
\gppoint{gp_mark3}{(5.488,3.798)}
\gppoint{gp_mark3}{(5.697,3.955)}
\gppoint{gp_mark3}{(5.907,4.084)}
\gppoint{gp_mark3}{(6.117,4.229)}
\gppoint{gp_mark3}{(6.326,4.372)}
\gppoint{gp_mark3}{(6.536,4.501)}
\gppoint{gp_mark3}{(6.746,4.650)}
\gppoint{gp_mark3}{(6.955,4.786)}
\gppoint{gp_mark3}{(7.165,4.932)}
\gppoint{gp_mark3}{(7.375,5.062)}
\gppoint{gp_mark3}{(7.584,5.221)}
\gppoint{gp_mark3}{(7.794,5.345)}
\gppoint{gp_mark3}{(8.004,5.482)}
\gppoint{gp_mark3}{(8.213,5.632)}
\gppoint{gp_mark3}{(8.423,5.775)}
\gppoint{gp_mark3}{(8.633,5.907)}
\gppoint{gp_mark3}{(8.843,6.048)}
\gppoint{gp_mark3}{(9.052,6.188)}
\gppoint{gp_mark3}{(9.262,6.356)}
\gppoint{gp_mark3}{(9.472,6.477)}
\gppoint{gp_mark3}{(9.681,6.620)}
\gppoint{gp_mark3}{(9.891,6.770)}
\gppoint{gp_mark3}{(10.101,6.902)}
\gppoint{gp_mark3}{(10.310,7.043)}
\gppoint{gp_mark3}{(10.520,7.187)}
\gppoint{gp_mark3}{(10.730,7.353)}
\gppoint{gp_mark3}{(10.939,7.444)}
\gppoint{gp_mark3}{(11.149,7.595)}
\gppoint{gp_mark3}{(11.359,7.742)}
\gppoint{gp_mark3}{(11.568,7.891)}
\gppoint{gp_mark3}{(11.778,8.004)}
\gppoint{gp_mark3}{(11.988,8.156)}
\gppoint{gp_mark3}{(4.322,7.119)}
\color{gp_lt_color_b}
\node[gp node right] at (3.680,6.811) {C++ -O2};
\color{gp_lt_color3}
\gpsetlinetype{gp_lt_plot3}
\gpsetlinewidth{5.00}
\draw[gp path] (3.864,6.811)--(4.780,6.811);
\draw[gp path] (1.504,1.123)--(1.714,1.266)--(1.923,1.404)--(2.133,1.547)--(2.343,1.690)%
  --(2.552,1.826)--(2.762,1.964)--(2.972,2.095)--(3.181,2.242)--(3.391,2.381)--(3.601,2.524)%
  --(3.810,2.674)--(4.020,2.800)--(4.230,2.933)--(4.439,3.073)--(4.649,3.214)--(4.859,3.357)%
  --(5.068,3.495)--(5.278,3.641)--(5.488,3.774)--(5.697,3.914)--(5.907,4.052)--(6.117,4.193)%
  --(6.326,4.328)--(6.536,4.469)--(6.746,4.605)--(6.955,4.753)--(7.165,4.881)--(7.375,5.029)%
  --(7.584,5.162)--(7.794,5.302)--(8.004,5.448)--(8.213,5.581)--(8.423,5.719)--(8.633,6.096)%
  --(8.843,5.998)--(9.052,6.133)--(9.262,6.279)--(9.472,6.417)--(9.681,6.560)--(9.891,6.732)%
  --(10.101,6.841)--(10.310,6.989)--(10.520,7.122)--(10.730,7.258)--(10.939,7.398)--(11.149,7.556)%
  --(11.359,7.677)--(11.568,7.810)--(11.778,7.948)--(11.988,8.094);
\gppoint{gp_mark4}{(1.504,1.123)}
\gppoint{gp_mark4}{(1.714,1.266)}
\gppoint{gp_mark4}{(1.923,1.404)}
\gppoint{gp_mark4}{(2.133,1.547)}
\gppoint{gp_mark4}{(2.343,1.690)}
\gppoint{gp_mark4}{(2.552,1.826)}
\gppoint{gp_mark4}{(2.762,1.964)}
\gppoint{gp_mark4}{(2.972,2.095)}
\gppoint{gp_mark4}{(3.181,2.242)}
\gppoint{gp_mark4}{(3.391,2.381)}
\gppoint{gp_mark4}{(3.601,2.524)}
\gppoint{gp_mark4}{(3.810,2.674)}
\gppoint{gp_mark4}{(4.020,2.800)}
\gppoint{gp_mark4}{(4.230,2.933)}
\gppoint{gp_mark4}{(4.439,3.073)}
\gppoint{gp_mark4}{(4.649,3.214)}
\gppoint{gp_mark4}{(4.859,3.357)}
\gppoint{gp_mark4}{(5.068,3.495)}
\gppoint{gp_mark4}{(5.278,3.641)}
\gppoint{gp_mark4}{(5.488,3.774)}
\gppoint{gp_mark4}{(5.697,3.914)}
\gppoint{gp_mark4}{(5.907,4.052)}
\gppoint{gp_mark4}{(6.117,4.193)}
\gppoint{gp_mark4}{(6.326,4.328)}
\gppoint{gp_mark4}{(6.536,4.469)}
\gppoint{gp_mark4}{(6.746,4.605)}
\gppoint{gp_mark4}{(6.955,4.753)}
\gppoint{gp_mark4}{(7.165,4.881)}
\gppoint{gp_mark4}{(7.375,5.029)}
\gppoint{gp_mark4}{(7.584,5.162)}
\gppoint{gp_mark4}{(7.794,5.302)}
\gppoint{gp_mark4}{(8.004,5.448)}
\gppoint{gp_mark4}{(8.213,5.581)}
\gppoint{gp_mark4}{(8.423,5.719)}
\gppoint{gp_mark4}{(8.633,6.096)}
\gppoint{gp_mark4}{(8.843,5.998)}
\gppoint{gp_mark4}{(9.052,6.133)}
\gppoint{gp_mark4}{(9.262,6.279)}
\gppoint{gp_mark4}{(9.472,6.417)}
\gppoint{gp_mark4}{(9.681,6.560)}
\gppoint{gp_mark4}{(9.891,6.732)}
\gppoint{gp_mark4}{(10.101,6.841)}
\gppoint{gp_mark4}{(10.310,6.989)}
\gppoint{gp_mark4}{(10.520,7.122)}
\gppoint{gp_mark4}{(10.730,7.258)}
\gppoint{gp_mark4}{(10.939,7.398)}
\gppoint{gp_mark4}{(11.149,7.556)}
\gppoint{gp_mark4}{(11.359,7.677)}
\gppoint{gp_mark4}{(11.568,7.810)}
\gppoint{gp_mark4}{(11.778,7.948)}
\gppoint{gp_mark4}{(11.988,8.094)}
\gppoint{gp_mark4}{(4.322,6.811)}
\color{gp_lt_color_b}
\node[gp node right] at (3.680,6.503) {Mixed -O3};
\color{gp_lt_color4}
\gpsetlinetype{gp_lt_plot4}
\gpsetlinewidth{3.00}
\draw[gp path] (3.864,6.503)--(4.780,6.503);
\draw[gp path] (1.504,1.122)--(1.714,1.262)--(1.923,1.396)--(2.133,1.548)--(2.343,1.674)%
  --(2.552,1.827)--(2.762,1.948)--(2.972,2.091)--(3.181,2.227)--(3.391,2.350)--(3.601,2.489)%
  --(3.810,2.627)--(4.020,2.762)--(4.230,2.899)--(4.439,3.037)--(4.649,3.176)--(4.859,3.311)%
  --(5.068,3.444)--(5.278,3.578)--(5.488,3.752)--(5.697,3.890)--(5.907,4.003)--(6.117,4.139)%
  --(6.326,4.288)--(6.536,4.406)--(6.746,4.540)--(6.955,4.684)--(7.165,4.813)--(7.375,4.952)%
  --(7.584,5.095)--(7.794,5.228)--(8.004,5.379)--(8.213,5.489)--(8.423,5.649)--(8.633,5.764)%
  --(8.843,5.955)--(9.052,6.031)--(9.262,6.237)--(9.472,6.347)--(9.681,6.460)--(9.891,6.609)%
  --(10.101,6.721)--(10.310,6.880)--(10.520,7.009)--(10.730,7.156)--(10.939,7.269)--(11.149,7.457)%
  --(11.359,7.535)--(11.568,7.686)--(11.778,7.824)--(11.988,7.969);
\gppoint{gp_mark5}{(1.504,1.122)}
\gppoint{gp_mark5}{(1.714,1.262)}
\gppoint{gp_mark5}{(1.923,1.396)}
\gppoint{gp_mark5}{(2.133,1.548)}
\gppoint{gp_mark5}{(2.343,1.674)}
\gppoint{gp_mark5}{(2.552,1.827)}
\gppoint{gp_mark5}{(2.762,1.948)}
\gppoint{gp_mark5}{(2.972,2.091)}
\gppoint{gp_mark5}{(3.181,2.227)}
\gppoint{gp_mark5}{(3.391,2.350)}
\gppoint{gp_mark5}{(3.601,2.489)}
\gppoint{gp_mark5}{(3.810,2.627)}
\gppoint{gp_mark5}{(4.020,2.762)}
\gppoint{gp_mark5}{(4.230,2.899)}
\gppoint{gp_mark5}{(4.439,3.037)}
\gppoint{gp_mark5}{(4.649,3.176)}
\gppoint{gp_mark5}{(4.859,3.311)}
\gppoint{gp_mark5}{(5.068,3.444)}
\gppoint{gp_mark5}{(5.278,3.578)}
\gppoint{gp_mark5}{(5.488,3.752)}
\gppoint{gp_mark5}{(5.697,3.890)}
\gppoint{gp_mark5}{(5.907,4.003)}
\gppoint{gp_mark5}{(6.117,4.139)}
\gppoint{gp_mark5}{(6.326,4.288)}
\gppoint{gp_mark5}{(6.536,4.406)}
\gppoint{gp_mark5}{(6.746,4.540)}
\gppoint{gp_mark5}{(6.955,4.684)}
\gppoint{gp_mark5}{(7.165,4.813)}
\gppoint{gp_mark5}{(7.375,4.952)}
\gppoint{gp_mark5}{(7.584,5.095)}
\gppoint{gp_mark5}{(7.794,5.228)}
\gppoint{gp_mark5}{(8.004,5.379)}
\gppoint{gp_mark5}{(8.213,5.489)}
\gppoint{gp_mark5}{(8.423,5.649)}
\gppoint{gp_mark5}{(8.633,5.764)}
\gppoint{gp_mark5}{(8.843,5.955)}
\gppoint{gp_mark5}{(9.052,6.031)}
\gppoint{gp_mark5}{(9.262,6.237)}
\gppoint{gp_mark5}{(9.472,6.347)}
\gppoint{gp_mark5}{(9.681,6.460)}
\gppoint{gp_mark5}{(9.891,6.609)}
\gppoint{gp_mark5}{(10.101,6.721)}
\gppoint{gp_mark5}{(10.310,6.880)}
\gppoint{gp_mark5}{(10.520,7.009)}
\gppoint{gp_mark5}{(10.730,7.156)}
\gppoint{gp_mark5}{(10.939,7.269)}
\gppoint{gp_mark5}{(11.149,7.457)}
\gppoint{gp_mark5}{(11.359,7.535)}
\gppoint{gp_mark5}{(11.568,7.686)}
\gppoint{gp_mark5}{(11.778,7.824)}
\gppoint{gp_mark5}{(11.988,7.969)}
\gppoint{gp_mark5}{(4.322,6.503)}
\color{gp_lt_color_b}
\node[gp node right] at (3.680,6.195) {C++ -O3};
\color{gp_lt_color5}
\gpsetlinetype{gp_lt_plot5}
\draw[gp path] (3.864,6.195)--(4.780,6.195);
\draw[gp path] (1.504,1.121)--(1.714,1.259)--(1.923,1.394)--(2.133,1.537)--(2.343,1.675)%
  --(2.552,1.813)--(2.762,1.942)--(2.972,2.080)--(3.181,2.215)--(3.391,2.353)--(3.601,2.496)%
  --(3.810,2.627)--(4.020,2.760)--(4.230,2.898)--(4.439,3.039)--(4.649,3.175)--(4.859,3.320)%
  --(5.068,3.448)--(5.278,3.589)--(5.488,3.722)--(5.697,3.860)--(5.907,3.998)--(6.117,4.131)%
  --(6.326,4.272)--(6.536,4.407)--(6.746,4.540)--(6.955,4.686)--(7.165,4.817)--(7.375,4.967)%
  --(7.584,5.085)--(7.794,5.223)--(8.004,5.366)--(8.213,5.512)--(8.423,5.667)--(8.633,6.030)%
  --(8.843,5.914)--(9.052,6.052)--(9.262,6.188)--(9.472,6.328)--(9.681,6.464)--(9.891,6.609)%
  --(10.101,6.740)--(10.310,6.878)--(10.520,7.011)--(10.730,7.147)--(10.939,7.282)--(11.149,7.438)%
  --(11.359,7.561)--(11.568,7.699)--(11.778,7.835)--(11.988,7.980);
\gppoint{gp_mark6}{(1.504,1.121)}
\gppoint{gp_mark6}{(1.714,1.259)}
\gppoint{gp_mark6}{(1.923,1.394)}
\gppoint{gp_mark6}{(2.133,1.537)}
\gppoint{gp_mark6}{(2.343,1.675)}
\gppoint{gp_mark6}{(2.552,1.813)}
\gppoint{gp_mark6}{(2.762,1.942)}
\gppoint{gp_mark6}{(2.972,2.080)}
\gppoint{gp_mark6}{(3.181,2.215)}
\gppoint{gp_mark6}{(3.391,2.353)}
\gppoint{gp_mark6}{(3.601,2.496)}
\gppoint{gp_mark6}{(3.810,2.627)}
\gppoint{gp_mark6}{(4.020,2.760)}
\gppoint{gp_mark6}{(4.230,2.898)}
\gppoint{gp_mark6}{(4.439,3.039)}
\gppoint{gp_mark6}{(4.649,3.175)}
\gppoint{gp_mark6}{(4.859,3.320)}
\gppoint{gp_mark6}{(5.068,3.448)}
\gppoint{gp_mark6}{(5.278,3.589)}
\gppoint{gp_mark6}{(5.488,3.722)}
\gppoint{gp_mark6}{(5.697,3.860)}
\gppoint{gp_mark6}{(5.907,3.998)}
\gppoint{gp_mark6}{(6.117,4.131)}
\gppoint{gp_mark6}{(6.326,4.272)}
\gppoint{gp_mark6}{(6.536,4.407)}
\gppoint{gp_mark6}{(6.746,4.540)}
\gppoint{gp_mark6}{(6.955,4.686)}
\gppoint{gp_mark6}{(7.165,4.817)}
\gppoint{gp_mark6}{(7.375,4.967)}
\gppoint{gp_mark6}{(7.584,5.085)}
\gppoint{gp_mark6}{(7.794,5.223)}
\gppoint{gp_mark6}{(8.004,5.366)}
\gppoint{gp_mark6}{(8.213,5.512)}
\gppoint{gp_mark6}{(8.423,5.667)}
\gppoint{gp_mark6}{(8.633,6.030)}
\gppoint{gp_mark6}{(8.843,5.914)}
\gppoint{gp_mark6}{(9.052,6.052)}
\gppoint{gp_mark6}{(9.262,6.188)}
\gppoint{gp_mark6}{(9.472,6.328)}
\gppoint{gp_mark6}{(9.681,6.464)}
\gppoint{gp_mark6}{(9.891,6.609)}
\gppoint{gp_mark6}{(10.101,6.740)}
\gppoint{gp_mark6}{(10.310,6.878)}
\gppoint{gp_mark6}{(10.520,7.011)}
\gppoint{gp_mark6}{(10.730,7.147)}
\gppoint{gp_mark6}{(10.939,7.282)}
\gppoint{gp_mark6}{(11.149,7.438)}
\gppoint{gp_mark6}{(11.359,7.561)}
\gppoint{gp_mark6}{(11.568,7.699)}
\gppoint{gp_mark6}{(11.778,7.835)}
\gppoint{gp_mark6}{(11.988,7.980)}
\gppoint{gp_mark6}{(4.322,6.195)}
\color{gp_lt_color_b}
\gpsetlinetype{gp_lt_border}
\gpsetlinewidth{1.00}
\draw[gp path] (1.504,8.382)--(1.504,0.985)--(12.040,0.985)--(12.040,8.382)--cycle;
%% coordinates of the plot area
\coordinate (gpbb south west 1) at (1.504,0.985);
\coordinate (gpbb south east 1) at (12.040,0.985);
\coordinate (gpbb north east 1) at (12.040,8.382);
\coordinate (gpbb north west 1) at (1.504,8.382);
\coordinate (gpbb north 1) at (6.772,8.382);
\coordinate (gpbb south 1) at (6.772,0.985);
\coordinate (gpbb west 1) at (1.504,4.684);
\coordinate (gpbb east 1) at (12.040,4.684);
\end{tikzpicture}
%% gnuplot variables
}
% % % %     \end{tabular}
    \caption{\scriptsize{Execution time as a function of the number of Monte Carlo cycles for mixed Python/C++ and pure C++ simulators implementing the QVMC method with importance sampling for He atom.}}
  \end{figure}  
\end{frame}



% % % % 
% % % % \begin{frame}[fragile]{Implementation of a QVMC simulator in C++}
% % % %   
% % % %   \begin{columns}[T,c]
% % % %     \column{5cm}
% % % %     \scriptsize
% % % %     \begin{alertblock}{Requirements}
% % % %       \begin{enumerate}
% % % %         \scriptsize
% % % %         \item Numerical efficiency.
% % % %         \item Flexibility.
% % % %           \begin{itemize}
% % % %             \scriptsize
% % % %             \item Extensibility.
% % % %             \item Independent of the nsd.
% % % %           \end{itemize}
% % % %         \item Readability.
% % % %       \end{enumerate}
% % % %     \end{alertblock}
% % % %     
% % % %     \begin{alertblock}{How?}
% % % %       \begin{enumerate}
% % % %       \scriptsize
% % % %       \item Alternative algorithms.
% % % %       
% % % %       \item Object oriented programming.
% % % %       
% % % %         \begin{itemize}
% % % %           \scriptsize
% % % %           \item Classes, inheritance, templates, etc.
% % % %         \end{itemize}
% % % %       \item Operator overloading, functors, etc.
% % % %       \end{enumerate}
% % % %     \end{alertblock}
% % % %   
% % % %   
% % % %   \column{5cm}
% % % %     \scriptsize
% % % %     \begin{alertblock}{QVMC algorithm with drift diffusion}
% % % %       \begin{algorithmic}%[1]
% % % %         \medskip
% % % %         \REQUIRE \emph{$nel$, $nmc$, $nes$, $\delta t$, $\bfv{R}$ and $\Psi_{\alpha}(\bfv{R})$.}
% % % % 
% % % %         \ENSURE $\langle E_{\alpha} \rangle$.
% % % % 
% % % %         \FOR{$c=1$ to $nmc$}
% % % %           \FOR{$p=1$ to $nel$}
% % % %             \STATE {$\bfv{x}^{new}_p = \bfv{x}^{cur}_p + \chi + D {\color{red}{\bfv{F}(\bfv{x}^{cur}_p)}} \delta t$}\\
% % % %             \medskip
% % % %             \emph{Compute $\color{red}{\bfv{F}(\bfv{x}^{new}) = 2.0\frac{\bfv{\nabla \Psi_T}}{\Psi_T}}$}\\
% % % %             \emph{Accept trial move with probability}\\
% % % %             $min\left[1,\frac{\omega(\bfv{x}^{cur}, \bfv{x}^{new})}{\omega(\bfv{x}^{new}, \bfv{x}^{cur})} {\color{red}{\frac{|\Psi(\bfv{x}^{new})|^2}{|\Psi(\bfv{x}^{cur})|^2}}}\right]$
% % % %           \ENDFOR
% % % %           \\
% % % %           \emph{Compute $\color{red}{E_L = \frac{\Op{H}\Psi_T}{\Psi_T} = -\frac{1}{2}\frac{\nabla^2 \Psi_T}{\Psi_T}} + V.$}
% % % %         \ENDFOR
% % % %         \\
% % % %         \emph{Compute $\langle E \rangle = \frac{1}{nmc} \sum_{c=1}^{nmc} E_L$ and $\sigma^2 = \langle E \rangle^2- \langle E^2 \rangle$.}
% % % %       \end{algorithmic}\label{RSDalgo}
% % % %     \end{alertblock}
% % % %     
% % % %   \end{columns}
% % % % \end{frame}
% % % % 
% % % % 
% % % % % % % \begin{frame}{Algorithm analysis and performance improvement}
% % % % % % % MOstrar los dos determinantes, jastrows y rijs y mostrar que cuando muevo una particula a la vez solo una row cambia.
% % % % % % %   ESCRIBIR AQUI DOS DETERMINANTES GRADES VECES UN JASTRO, LAS DISTANCIAS RELATIVAS, ETC CON LAS ROWS SOMBREADAS.
% % % % % % % \end{frame}
% % % % 
% % % % 
% % % % 
% % % % \begin{frame}{Algorithm analysis and performance improvement}
% % % % 
% % % %   \begin{columns}[T,c]
% % % %     \column{5cm}
% % % %     \scriptsize
% % % %     \begin{alertblock}{No optimization}
% % % %       \begin{itemize}
% % % %         \item $\frac{\Psi_{T}^{new}}{\Psi_{T}^{cur}} = \underbrace{\frac{\Det{D}_{\uparrow}^{new}}{\Det{D}_{\uparrow}^{cur}} \frac{\Det{D}_{\downarrow}^{new}}{\Det{D}_{\downarrow}^{cur}}}_{R_{SD}}\, \underbrace{\frac{\Psi_{C}^{new}}{\Psi_{C}^{cur}}}_{R_{C}}.$
% % % %         
% % % %         \item $\frac{\bfv{\nabla}\Psi_{T}^{new}}{\Psi_{T}^{new}} = \frac{\bfv{\nabla}(\Det{D}_{\uparrow} \Det{D}_{\downarrow} \Psi_{C})}{\Det{D}_{\uparrow} \Det{D}_{\downarrow} \Psi_{C}}$
% % % %         
% % % %         \item $\frac{\nabla^2 \Psi_{T}^{new}}{\Psi_{T}^{new}} = \frac{\nabla^2(\Det{D}_{\uparrow} \Det{D}_{\downarrow} \Psi_{C})}{\Det{D}_{\uparrow} \Det{D}_{\downarrow} \Psi_{C}}$
% % % %       \end{itemize}
% % % %     \end{alertblock}
% % % %   
% % % %   
% % % %   \column{6cm}
% % % %     \scriptsize
% % % %     \begin{alertblock}{Optimization}
% % % %       \begin{itemize}
% % % %         \item $R \equiv R_{SD} R_{C}$\\
% % % %         
% % % %           $R_{SD} = \sum_{j=1}^{N} \phi_j(\bfv{x^{new}_i}) D_{ji}^{-1}(\bfv{x^{cur}}).$\\
% % % %           $R_C = \cdots?$\\
% % % %         
% % % %         \item $\frac{\Grad \Psi}{\Psi} =  \frac{\Grad (\Det{D}_{\uparrow}) }{\Det{D}_{\uparrow}} + \frac{\Grad (\Det{D}_{\downarrow})}{\Det{D}_{\downarrow}} + \frac{\Grad  \Psi_C}{ \Psi_C}.$\\
% % % %         
% % % %         $\frac{\bfv{\nabla_i}|\bfv{D}(\bfv{x})|}{|\bfv{D}(\bfv{x})|} = \sum_{j=1}^{N} \bfv{\nabla_i}\phi_j(\bfv{x_i}) D_{ji}^{-1}(\bfv{x})$\\
% % % %         
% % % %         $\frac{\bfv{\nabla_i}|\bfv{D}(\bfv{y})|}{|\bfv{D}(\bfv{y})|} = \frac{1}{R} \sum_{j=1}^{N} \bfv{\nabla_i}\phi_j(\bfv{y_i}) D_{ji}^{-1}(\bfv{x}).$\\
% % % %         ...
% % % % %         \item 
% % % %       \end{itemize}
% % % %     \end{alertblock}
% % % %   \end{columns}
% % % %   
% % % %   
% % % %     \scriptsize
% % % %     \begin{alertblock}{Inverse updating}
% % % %       $
% % % %  D^{-1}_{kj}(\bfv{x^{new}})=
% % % %   \begin{cases} 
% % % %   D^{-1}_{kj}(\bfv{x^{cur}}) - \frac{D^{-1}_{ki}(\bfv{x^{cur}})}{R} \sum_{l=1}^{N} D_{il}(\bfv{x^{new}})  D^{-1}_{lj}(\bfv{x^{cur}}) & \mbox{if $j \neq i$}\\
% % % %     \frac{D^{-1}_{ki}(\bfv{x^{cur}})}{R} \sum_{l=1}^{N} D_{il}(\bfv{x^{cur}}) D^{-1}_{lj}(\bfv{x^{cur}}) & \mbox{if $j=i$}
% % % % \end{cases}
% % % % $
% % % %   \end{alertblock}
% % % % \end{frame}




%%%%%%%%%%%%%%%%%%% STRUCTURING A SIMULATOR %%%%%%%%%%%%%%%%%%%
%%%%%%%%%%%%%%%%%%%%%%%%%%%%%%%%%%%%%%%%%%%%%%%%%%%%%%%%%%%%%%%
\subsection{Implementation in C++}
\begin{frame}{Structuring a simulator with object-orientation?}
    \begin{alertblock}{Defining classes}
      \begin{enumerate}
        \item From the mathematical model and the algorithms, {\textbf{identify the main components of the problem}} \color{blue}{(classes)}.
          \begin {itemize}
           \item Vmc simulation: {\color{red}{\citecode{VmcSimulation.h}}}
           \item Parameters in simulation: {\color{red}{\citecode{Parameters.h}}}
           \item Monte Carlo method: {\color{red}{\citecode{MonteCarlo.h}}}
           \item Energy: {\color{red}{\citecode{Energy.h}}}
           \item Potential:{\color{red}{\citecode{Potential.h}}}
% % %            \item Relative Distances:{\color{red}{\citecode{RelDistances.h}}}
           \item Trial wave function: {\color{red}{\citecode{PsiTrial}}}
% % %            \item Slater determinant: {\color{red}{\citecode{Slater.h}}}
% % %            \item Correlation function: {\color{red}{\citecode{CorrelationFnc.h}}}
% % %            \item Single particle wave functions: {\color{red}{\citecode{SPWF.h}}}\\
           ...
          \end {itemize}
          
          
                  \item {\textbf{Identify the mathematical and algorithmic parts changing from problem to problem}} \color{blue}{(superclasses)}.
          \begin {itemize}
           \item Potential:{\color{red}{\citecode{Potential.h}}}
           \item Trial wave function: {\color{red}{\citecode{PsiTrial}}}
           \item Monte Carlo method: {\color{red}{\citecode{MonteCarlo.h}}}         ...
         \end {itemize}


      \end{enumerate}
    \end{alertblock}
\end{frame}
 
 
 
\begin{frame}[fragile]
  \begin{alertblock}{Enhance flexibility}
      \begin{enumerate}\setcounter{enumi}{2}
       
       \item  {\textbf{List what each of these classes should do}}, IN GENERAL \color{blue}{(member functions)}, e.g., for {\color{red}{\citecode{PsiTrial.h}}}:
          \begin{itemize}
           \item Compute the acceptance ratio: \citecode{getPsiPsiRatio()}.
           
           \item Compute the quantum force: \citecode{getQuantumForce()}.
% % % %            \item Compute the $\frac{\nabla^2 \Psi_T}{\Psi_T}$ ratio: \citecode{getLapPsiRatio()}\\
           ...
          \end{itemize}

% % %         \begin{scriptsize}
% % %           What we have now are interfaces, i.e. classes with a general appearance without a defined behaviour, i.e, we know what they do, but not specify how. Their functions are declared \emph{virtual or pure virtual functions}.
% % %         \end{scriptsize}
          
      \end{enumerate}
    \end{alertblock}
    
      \begin{scriptsize}
    \begin{C++}
    
      #include "SomeClass.h"
      
      class PsiTrial{
        public:
          virtual double getAcceptanceRatio()=0;
          virtual MyArray<double> getQuantumForce()=0;
          virtual getLapPsiRatio()=0;
      }
    \end{C++}
  \end{scriptsize}

\end{frame}



\begin{frame}[fragile]

  
%   \begin{scriptsize}
    \begin{alertblock}{Inheritance (specializing behaviours):  {\textbf{\emph{is a (kind of)}}-relationship}}
      \begin{enumerate}\setcounter{enumi}{3}
        \item Be specific with the behaviour , i.e., create subclases by finding {\textbf{\emph{is a (kind of)}}~-relationships}.\color{blue}{(subclasses)}
        For example:
      
        \begin{itemize}
          \scriptsize
          \item Slater-Jastrow ({\color{red}{\citecode{SlaterJastrow.h}}}) {\textbf{\emph{is a (kind of)}}} trial wave function ({\color{red}{\citecode{PsiTrial.h}}}).
          
          \item Slater alone ({\color{red}{\citecode{SlaterAlone.h}}}) {\textbf{\emph{is a (kind of)}}} trial wave function ({\color{red}{\citecode{PsiTrial.h}}}).\\
          ...
          
          \item Coulomb one-body ({\color{red}{\citecode{OneBodyCoulomb.h}}}) {\textbf{\emph{is a (kind of)}}} potential ({\color{red}{\citecode{Potential.h}}}).
          % % %           \item Lennard-Johes ({\color{red}{\citecode{LennardJones.h}}}) {\textbf{\emph{is a (kind of)}}} potential ({\color{red}{\citecode{Potential.h}}}).
        \end{itemize}
      \end{enumerate}
    \end{alertblock}
%   \end{scriptsize}
\end{frame}



\begin{frame}[fragile]
 \begin{alertblock}{Composition: {\textbf{\emph{has a}}-relationship}}
  \begin{enumerate}\setcounter{enumi}{4}
   \item Connect the whole structure with relatonships of type  {\textbf{\emph{has a}}} \color{blue}{(composition)}.
   \end{enumerate}
    $$\Psi_T = |\bfv{D}|_{\uparrow} |\bfv{D}|_{\downarrow}| J(rij)$$
   For example: Slater-Jastrow wave function ({\color{red}{\citecode{SlaterJastrow.h}}}) {\textbf{\emph{has~a}}} Slater determinant ({\color{red}{\citecode{SlaterDeterminant.h}}}) and a correlation function  ({\color{red}{\citecode{CorrelationFnc.h}}}).
  

  
 \end{alertblock}
 
\end{frame}


\begin{frame}[fragile]
  \begin{scriptsize}
  \begin{C++}
  
    #include "SlaterDeterminant.h"
    #include "CorrelationFnc.h"
    
    class SlaterJastrow: public PsiTrial{
      private:
        SlaterDeterminant *slater;
        CorrelationFnc *correlation;
      
      public:
        SlaterJastrow(SlaterDeterminant *sd, 
                     CorrelationFnc *cor):
                     slater(sd), correlation(cor){}
        
        double getPsiPsiRatio(){
          return slater->getDetDetRatio(...)
                 *correlation->getCorCorRatio(...);
        }
    };    
  \end{C++}
  \end{scriptsize}
\end{frame}


\frame[fragile]{Creating objects}
  \begin{scriptsize}
    \begin{alertblock}{Creating trial wave functions}
    \begin{C++}
    
      void VmcSimulator::setTrialWaveFnc(){
        ...
        SlaterDeterminant *sd = new SlaterDeterminant(...);
        Correlation *pj = new PadeJastrow(...);
        
        PsiTrial *s = new SlaterJastrow(sd, pj);
      }
    \end{C++}
    
    
    or 
    
    
    \begin{C++}
    
       void VmcSimulator::setTrialWaveFnc(){
        ...
          SlaterDeterminant *sd = new SlaterDeterminant(...);
          PsiTrial *sj = new SlaterAlone(sd);
       }
    \end{C++}
    \end{alertblock}
    
%     Used as:
%     
%     \begin{C++}
%       
%       void someFunction(){
%         sj->getPsiPsiRatio(...);
%       }
%     \end{C++}
  \end{scriptsize}
\end{frame}





% % % % % \begin{frame}{Computational gain}
% % % % % %  \column{6cm}
% % % % % % 		\scriptsize
% % % % % % 		\begin{alertblock}{Performance gain.}
% % % % % % 		
% % % % % % 	
% % % % % % % 		\begin{scriptsize}
% % % % % % 		\begin{table}
% % % % % % 		\centering
% % % % % % 		\resizebox{0.9\textwidth}{!}{
% % % % % % 		\begin{tabular}{l*{6}{l}l}
% % % % % % 		\toprule[1pt]
% % % % % % 		Operation       & Not optimized & Optimized\\
% % % % % % 		\midrule[1pt]
% % % % % % 		Evaluation of $R$    & $\mathcal{O}(N^2)$ & $\mathcal{O}\left(\frac{N^2}{2}\right)$\\
% % % % % % 		Updating inverse& $\mathcal{O}(N^3)$ & $\mathcal{O}\left(\frac{N^3}{4}\right)$  \\
% % % % % % 		Transition of one particle & $\mathcal{O}(N^2) +  \mathcal{O}(N^3)$ & $\mathcal{O}\left(\frac{N^2}{2}\right) + \mathcal{O}\left(\frac{N^3}{4}\right)$\\
% % % % % % 		\bottomrule[1pt]
% % % % % % 		\end{tabular}
% % % % % % 		}
% % % % % % 
% % % % % % 		% \caption{Computational cost of the Slater determinant.}
% % % % % % 		\end{table}
% % % % % % % 		\end{scriptsize}
% % % % % % 	\end{alertblock}
% % % % %  
% % % % % \end{frame}
