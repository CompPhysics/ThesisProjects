\section{Numerical method}
\subsection{Why numerical methods?}

\begin{frame}{Why is it a hard problem to solve?}
  \begin{alertblock}{P.A.M. Dirac:}
    \begin{scriptsize}
    \emph{The fundamental laws necessary for the 
    mathematical treatment of a large part of physics and the 
    whole of chemistry are thus completely known, and the 
    difficulty lies only in the fact that application of these laws 
    leads to equations that are too complex to be solved}. 
    \end{scriptsize}
  \end{alertblock}
  
  \begin{scriptsize}
    
% % % %         \begin{column}{0.35\textwidth}
% %           \begin{itemize}
% %             \item Describes the physics of matter at "normal" time and length scales, i.e., $v<<c$.
% %             \item Gives the energy states of a system, from which physical properties can be derived.
% %             \item Gives a full quantum description of matter.
% %           \end{itemize}
% %         \end{column}


        \begin{table}[hbtp]
          \begin{center} 
            \resizebox{0.45\textwidth}{!}{
            \begin{tabular}{ll}
              \toprule[1pt]
              {\bf Hamiltonian term}    & {\bf Quantum nature}               \\
              \midrule[1pt]
              Kinetic energy            & One-body  \\
% % % % % % %               Nuclei-nuclei             & Zero-body \\
              Nuclei-electron           & One-body  \\       
              Electron-electron         & Two-body (hard)\\
              \bottomrule[1pt]
            \end{tabular}
            }
          \end{center}
          \caption{\footnotesize{Nature of the terms in the Hamiltonian.}}
        \end{table}
    \end{scriptsize}
  \emph{The wave function contains explicit correlations that leads to non-factoring multi-dimensional integrals.}
  
  
  
% % % % % % % % % % % % %   \begin{scriptsize}
% % % % % % % % % % % % %   \begin{columns}[T,c]
% % % % % % % % % % % % %     \column{0.45\textwidth}
% % % % % % % % % % % % %     \begin{block}{One-body terms}
% % % % % % % % % % % % %       \begin{itemize}
% % % % % % % % % % % % %       \item $\Psi(\bfv{r}) = \Psi(\bfv{r}_1) \Psi(\bfv{r}_2)...\Psi(\bfv{r}_N)$.
% % % % % % % % % % % % %       \item $N-$electrons need $N$ functions of three dimensions.
% % % % % % % % % % % % %       \item Discretization of the space into $q$ points along each dimension requires $N q^3$ values to describe electrons. For $N=100, \, q=100, \, 3Nq^3 = 3 \times 10^4$ points.
% % % % % % % % % % % % %       \end{itemize}
% % % % % % % % % % % % %     \end{block}
% % % % % % % % % % % % % 
% % % % % % % % % % % % %     \column{0.55\textwidth}
% % % % % % % % % % % % %     \begin{block}{Two-body terms}
% % % % % % % % % % % % %       \begin{itemize}
% % % % % % % % % % % % %       \item $\Psi(\bfv{r})$: Many-particle wave function cannot be exactly written as a product of spwf.
% % % % % % % % % % % % %       \item $N$ electrons need one function of $3N$ dimension.
% % % % % % % % % % % % %       \item Discretization of the space into $q$ points along each dimension requires $q^{3N}$ values to describe electrons. For $N=100, \, q=100, \, q^{3N} = 100^{300} = ?$ points.
% % % % % % % % % % % % %       \end{itemize}
% % % % % % % % % % % % %     \end{block}
% % % % % % % % % % % % %   \end{columns}
% % % % % % % % % % % % %   \end{scriptsize}
  
  
  
\end{frame}



% % % % % % % % \begin{frame}{Why is it a hard problem to solve?}
% % % % % % % %   \begin{alertblock}{P.A.M. Dirac:}
% % % % % % % %     \begin{scriptsize}
% % % % % % % %     \emph{The fundamental laws necessary for the 
% % % % % % % %     mathematical treatment of a large part of physics and the 
% % % % % % % %     whole of chemistry are thus completely known, and the 
% % % % % % % %     difficulty lies only in the fact that application of these laws 
% % % % % % % %     leads to equations that are too complex to be solved}. 
% % % % % % % %     \end{scriptsize}
% % % % % % % %   \end{alertblock}
% % % % % % % %   
% % % % % % % %   \begin{scriptsize}


\begin{frame}{Quantum Variational Monte Carlo}
	\begin{scriptsize}
		\begin{alertblock}{Variational principle}
			\emph{The expectation value of the energy computed for a Hamiltonian $\Op{H}$ given a (parametrized) trial wave function $\Psi_T$ is an upper bound to the ground state energy $E_0$.}
			$$E_{VMC} = \langle \Op{H} \rangle = \frac{\int \Psi_{T}^* \Op{H} \Psi_{T} \, d\bfv{R}}{\int \Psi_{T}^2 \, d\bfv{R}} = \frac{\langle \Psi_{T} |\Op{H}|\Psi_{T}\rangle}{\langle \Psi_{T}|\Psi_{T}\rangle} \geq E_0$$
		\end{alertblock}

		\begin{alertblock}{How to compute high dimensional integrals efficiently?}
			\textcolor{red}{Solution:} Translate the problem into a Monte Carlo language.
			$$
				E_{vmc} = \frac{\int \Psi_{T}^{*} \textcolor{green}{\Psi_T} \frac{\Op{H} \Psi_{T}}{\textcolor{green}{\Psi_{T}}}\, d\bfv{R}}{\int \Psi_{T}^2 \, d\bfv{R}} = \frac{\int \textcolor{red}{|\Psi_{T}|^2} \textcolor{blue}{\left[\frac{\Op{H} \Psi_{T}}{\Psi_{T}}\right]}\, d\bfv{R}}{\textcolor{red}{\int \Psi_{T}^2 \, d\bfv{R}}} 
					=  \int \textcolor{red}{P(\bfv{R})} \textcolor{blue}{{E}_L}\, d\bfv{R}
			$$
			\begin{itemize}
				\item Sample configurations, $\bfv{R}$, stochastically. 
		\end{itemize}
		\end{alertblock}
	\end{scriptsize}
\end{frame}


\begin{frame}
	\begin{scriptsize}
		\begin{alertblock}{...How to compute high dimensional integrals efficiently? [continued]}
			\begin{itemize}
				\item Average of the local energy $\textcolor{blue}{{E}_L}$ over the probability distribution function $\textcolor{red}{P(\bfv{R})}$.
					$$
					\boxed{\langle E \rangle =  \int \textcolor{red}{P(\bfv{R})} \textcolor{blue}{{E}_L}\, d\bfv{R} \approx \frac{1}{M} \sum_{i=1}^{M} \textcolor{blue}{{E}_L(}\bfv{R}_i\textcolor{blue}{)},}
					$$
					where $M$ is the number of Monte Carlo samples.
					
				\item Statistical variance (assuming uncorrelated data set)
					$$
						\boxed{var(E) =  \langle E^2 \rangle - \langle E \rangle^2}
					$$
					
% % % % % % 				\item The local energy $\textcolor{blue}{E_L}$ is constant for an exact wave function (because it is an eigenfunction of the Hamiltonian). Therefore the variance is zero in this case (\emph{Zero variance principle}).

        \item \emph{Zero variance principle}.
				
				\item The optimization of the trial wave function has as goal to find the optimal set of parameters that gives the minimum energy/variance.
				
				\item \textcolor{red}{BUT: We just need a (parametric) trial wave function...!} 
			\end{itemize}
		\end{alertblock}
	\end{scriptsize}
\end{frame}


\subsection{Trial wave function}
\begin{frame}{Trial wave function: $\Psi_T = \Psi_D \Psi_J$ (Slater-Jastrow)}
	\begin{scriptsize}
		\begin{alertblock}{Slater determinant}
			\begin{columns}
				\begin{column}{0.35\textwidth}
					$$
					\Psi_D \propto %%% \frac{1}{\sqrt{N!}}
					\begin{vmatrix}
					\phi_1(\bfv{r}_1) & \phi_2(\bfv{r}_1) & \cdots & \phi_N(\bfv{r}_1)\\
					\phi_1(\bfv{r}_2) & \phi_2(\bfv{r}_2) & \cdots & \phi_N(\bfv{r}_2)\\
					\vdots  & \vdots & \ddots & \vdots  \\
					\phi_1(\bfv{r}_N) & \phi_2(\bfv{r}_N) & \cdots & \phi_N(\bfv{r}_N)  
					\end{vmatrix}
					$$
				\end{column}
								
				\begin{column}{0.45\textwidth}
					\begin{itemize}
						\item Pauli exclusion principle.
						\item Spin-independent Hamiltonian?\\$\Psi_D = |\bfv{D}|_{\uparrow} |\bfv{D}|_{\downarrow}$.
						\item Electron-nucleus cusp conditions.
				\end{itemize}
				\end{column}
			\end{columns}
		\end{alertblock}	
		
		\begin{alertblock}{Linear Pad\'e-Jastrow correlation function}
			\begin{columns}
				\begin{column}{0.35\textwidth}
					$$
					\Psi_{J} = \exp\left(\sum\limits_{j<i}\frac{a_{ij}r_{ij}}{1 + \beta_{ij}r_{ij}}\right)
					$$
					$$
					r_{ij} = |\bfv{r}_j - \bfv{r}_i| 
					$$
				\end{column}
								
				\begin{column}{0.45\textwidth}
					\begin{itemize}
						\item Linear correlation.
						\item Two-body correlation.
						\item Fullfils electron-electron cusp conditions.
				\end{itemize}
				\end{column}
			\end{columns}
		\end{alertblock}	
	\end{scriptsize}
\end{frame}


% % % % % % % 
% % % % % % % \begin{frame}{The essence of QVMC}
% % % % % % % 	\begin{scriptsize}
% % % % % % % 		\begin{alertblock}{Quantum Variational Monte Carlo algorithm}
% % % % % % % 			QVMC is a method to evaluate the integral $\int \Psi^2(R) E_L dR.$
% % % % % % % 			To accomplish this, we use Markov chain Monte Carlo and proceed in the following manner:
% % % % % % % 			
% % % % % % % 			\begin{enumerate}
% % % % % % % 				\item Choose a trial wave function $\Psi_T$, depending on a set of variational parameters $\bfv{\alpha} = (\alpha_1, \alpha_2, \cdots, \alpha_S)$. More about this later.
% % % % % % % 				
% % % % % % % 				\item  Generate a random set of points $\{R_i\}, \, i=1,2,...,M$ in configuration space that are distributed according to the $P(\bfv{R}) = |\psi_T|^2$.
% % % % % % % 				
% % % % % % % 				\item Evaluate the ratio squared of the wavefunctions for the new and cur trial locations.
% % % % % % % 				
% % % % % % % 				\item If the ratio squared is greater then some random number, then keep the new trial location, otherwise keep the cur trial location. 
% % % % % % % 				
% % % % % % % 			\end{enumerate}
% % % % % % % 			
% % % % % % % 				Typically the integration part of Variational Monte Carlo is coupled with an optimization technique for the parameters. 
% % % % % % % 		\end{alertblock}
% % % % % % % 	\end{scriptsize}
% % % % % % % \end{frame}


% % % % % % % 
% % % % % % % \begin{frame}[fragile]{How to sample the PDE?}
% % % % % % % 	\begin{columns}
% % % % % % % 	\column{6cm}
% % % % % % % 	\begin{scriptsize}
% % % % % % % 		\begin{algorithmic}%[1]
% % % % % % % 			\medskip
% % % % % % % 			\REQUIRE \emph{$nel$, $nmc$, $nes$, $\delta t$, $\bfv{R}$ and $\Psi_{\alpha}(\bfv{R})$.}
% % % % % % % 
% % % % % % % 			\ENSURE $\langle E_{\alpha} \rangle$.
% % % % % % % 
% % % % % % % 			\FOR{$c=1$ to $nmc$}
% % % % % % % 				\FOR{$p=1$ to $nel$}
% % % % % % % 					\STATE {$\color{red}\bfv{x}^{new}_p = \bfv{x}^{cur}_p + \Delta x \zeta, \qquad \zeta \in [-1, 1]$}\\
% % % % % % % 					\medskip
% % % % % % % 					\emph{Define $q(\bfv{x^{new}}, \bfv{x^{cur}}) \equiv \frac{P(\bfv{x^{new}})}{P(\bfv{x^{cur}})} = \frac{|\Psi_{T}(\bfv{x^{new}})|^2}{|\Psi_{T}(\bfv{x^{cur}})|^2}$}\\
% % % % % % % 					\emph{Accept trial move with probability}\\
% % % % % % % 					$min\left[1,q(\bfv{x^{new}}, \bfv{x^{cur}})\right]$
% % % % % % % 				\ENDFOR
% % % % % % % 				\\
% % % % % % % 				\emph{Compute $E_L = \frac{\Op{H}\Psi_T}{\Psi_T} = -\frac{1}{2}\frac{\nabla^2 \Psi_T}{\Psi_T} + V.$}
% % % % % % % 			\ENDFOR
% % % % % % % 			\\
% % % % % % % 			\emph{Compute $\langle E \rangle = \frac{1}{nmc} \sum_{c=1}^{nmc} E_L$ and $\sigma^2 = \langle E \rangle^2- \langle E^2 \rangle$.}
% % % % % % % 		\end{algorithmic}\label{RSDalgo}
% % % % % % % 	\end{scriptsize}
% % % % % % % 	
% % % % % % % 	\column{6cm}
% % % % % % % 	{\color{red}{INCLUIR GRAFICA DE MUESTREO DE METROPOLIS}}
% % % % % % % 		\begin{scriptsize}
% % % % % % % 			\begin{itemize}
% % % % % % % 				\item $\bfv{x}^{new}$ does not depend on $\Psi_T$
% % % % % % % 				\item \emph{Importance sampling:} Bias the moves toward the states of high probability.
% % % % % % % 				\item Now $\bfv{x}^{new}_p = \bfv{x}^{cur}_p + D \bfv{F}(\bfv{x}^{cur}) \delta t + \chi\sqrt{\delta t}$.
% % % % % % % 				\item The acceptance probability is now:
% % % % % % % 				$min(1, \frac{G(\bfv{x^{cur}},\bfv{x^{new}},\Delta t)|\Psi_T(\bfv{x^{new}})|^2}{G(\bfv{x^{new}},\bfv{x^{cur}},\Delta t)|\Psi_T(\bfv{x^{cur}})|^2})$ and
% % % % % % % 				$G(\bfv{x^{new}},\bfv{x^{cur}},\Delta t)$ is a transitition probability matrix.
% % % % % % % 			\end{itemize}
% % % % % % % 		\end{scriptsize}
% % % % % % % 	\end{columns}
% % % % % % % \end{frame}


\subsection{QVMC algorithm}

\begin{frame}[fragile]
	\begin{scriptsize}
	\begin{columns}%[T,l]
		\column{6cm}
		\begin{algorithmic}%[1]
			\medskip
			\REQUIRE \emph{$nel$, $nmc$, $nes$, $\delta t$, $\bfv{R}$ and $\Psi_{\alpha}(\bfv{R})$.}

			\ENSURE $\langle E_{\alpha} \rangle$.

			\FOR{$c=1$ to $nmc$}
				\FOR{$p=1$ to $nel$}
					\STATE {$\bfv{x}^{new}_p = \bfv{x}^{cur}_p + \chi + D \bfv{F}(\bfv{x}^{cur}_p) \delta t$}\\
					\medskip
					\emph{Compute $\bfv{F}(\bfv{x}^{new}) = \frac{\bfv{\nabla \Psi_T}}{\Psi_T}$}\\
					\emph{Accept trial move with probability}\\
					$min\left[1,\frac{\omega(\bfv{x}^{cur}, \bfv{x}^{new})}{\omega(\bfv{x}^{new}, \bfv{x}^{cur})} {\frac{|\Psi(\bfv{x}^{new})|^2}{|\Psi(\bfv{x}^{cur})|^2}}\right]$
				\ENDFOR
				\\
				\emph{Compute $E_L = \frac{\Op{H}\Psi_T}{\Psi_T} = -\frac{1}{2}\frac{\nabla^2 \Psi_T}{\Psi_T} + V.$}
			\ENDFOR
			\\
			\emph{Compute $\langle E \rangle = \frac{1}{nmc} \sum_{c=1}^{nmc} E_L$ and $\sigma^2 = \langle E \rangle^2- \langle E^2 \rangle$.}
		\end{algorithmic}\label{RSDalgo}
		
		\column{4cm}
		\begin{figure}
			\centering
		\scalebox{0.35}{\input{chartFlows/chartFlowQVMC.tex}}
% % % 			\caption{Simulation chart flow}
		\end{figure}
	\end{columns}
	\end{scriptsize}
\end{frame}


\subsection{Optimization of the trial wave function}
\begin{frame}{Wave function optimization, why and how?}
	
	\begin{scriptsize}̈́
		\begin{alertblock}{Goal}
			\emph{Find an optimal set of parameters $\bfv{\alpha}$ in $\Psi_T$ to minimized the estimated energy.}
% 			
% 			\begin{itemize}
% 				\item Important both for VMC and DMC.
% 				\item	Reduce the systematic error.
% 				\item Reduce the statistical uncertainty.		 
% 			\end{itemize}
		\end{alertblock}
	\end{scriptsize}
	
	\begin{scriptsize}̈́
		\begin{alertblock}{Approaches}
			\begin{enumerate}
				\item	Minimization of the variance of the local energy.
				\item Minimization of the energy.
				\item A combination of both.
			\end{enumerate}
		\end{alertblock}

		\begin{alertblock}{Examples optimization algorithms}
			\begin{enumerate}
				\item Stochastic gradient approximation (SGA).
				\item Quasi-Newton method.
			\end{enumerate}
		\end{alertblock}
	\end{scriptsize}
\end{frame}



\begin{frame}{Minimization of energy with quasi-Newton method}
	\begin{scriptsize}
		\begin{alertblock}{Objective function and its derivative}
			\begin{columns}
			\column{4.0cm}
				\begin{itemize}
					\item {\color{red}{Expectation value of the energy:}}\\
					$E_{vmc} = \frac{ \int |\Psi_{T}|^2 \left[\frac{\Op{H} \Psi_{T}}{\Psi_{T}}\right]\, d\bfv{R}}{\int \Psi_{T}^2 \, d\bfv{R}}$
				\end{itemize}
				 
				 \column{6.0cm}
				 \begin{itemize}			 
					\item {\color{red}{Derivative of the expectation value of the energy:}}\\
					$\frac{\partial E}{\partial c_m} = 2\left[\left\langle E_L \frac{\frac{\partial \Psi_{T_{c_m}}}{\partial c_m}}{\Psi_{T_{c_m}}}\right\rangle - E \left\langle \frac{\frac{\partial \Psi_{T_{c_m}}}{\partial c_m}}{\Psi_{T_{c_m}}}\right\rangle \right]$
				 \end{itemize}
				\end{columns}
			\end{alertblock}
		\end{scriptsize}
		
		
		
	\begin{scriptsize}
		\begin{alertblock}{Strategies for computing the derivative}
			
				\begin{itemize}
					\item {\color{red}{Direct analytical differentiation:}}\\
					\vspace{0.2cm}
					$\frac{\partial \Psi_{T_{c_m}}}{\partial c_m} = \frac{\partial}{\partial c_m}\left[|\bfv{D}|_{\uparrow} |\bfv{D}|_{\downarrow} \prod_{i=1}^{N} \prod_{i=j+1}^{N} J(r_{ij})\right] = ...? \qquad \text{Enjoy it!}$
					\vspace{0.2cm}
					\item {\color{red}{Numerical derivative: central differences:}} Just for $\Psi_{SD}$ we have to do\\
					\vspace{0.2cm}
						$\frac{d \Psi_{SD}}{d \alpha_m} = \frac{\Psi_{SD}(\alpha_m + \Delta \alpha_m) - \Psi_{SD}(\alpha_m - \Delta \alpha_m)}{2\Delta \alpha_m} + \mathcal{O}(\Delta \alpha_{m}^2)$ \\
						\vspace{0.2cm}
						(...to be done per parameter).
						\vspace{0.2cm}
				\end{itemize}
				
				%%{\color{green}{TOO EXPENSIVE...}}
			\end{alertblock}
		\end{scriptsize}
\end{frame}


\begin{frame}{Minimization of energy with quasi-Newton method}

	\begin{scriptsize}
		\begin{alertblock}{Trick to compute the derivative of the energy}
			{\color{red}{Goal:}}\\
			$\frac{\partial E}{\partial c_m} = 2\left[\left\langle E_L {\color{red}{\frac{\frac{\partial \Psi_{T_{c_m}}}{\partial c_m}}{\Psi_{T_{c_m}}}}}\right\rangle - E \left\langle {\color{red}{\frac{\frac{\partial \Psi_{T_{c_m}}}{\partial c_m}}{\Psi_{T_{c_m}}}}}\right\rangle \right]$ 
			
			\begin{enumerate}
				\item  Split the trial wave function: $\Psi_{T_{c_m}} = \Psi_{{SD}_{c_m}} \Psi_{{J}_{c_m}} = \Psi_{{SD}_{c_m} \uparrow}  \Psi_{{SD}_{c_m} \downarrow} \Psi_{{J}_{c_m}} $.
				
				\item Rewrite the {\color{red}{derivatives}} as:  $\color{red}{\frac{\partial \ln \Psi_{T_{c_m}}}{\partial c_m} = \frac{\partial \ln(\Psi_{{SD_{c_m}}\uparrow})}{\partial c_m} + \frac{\partial \ln(\Psi_{{SD}_{c_m}\downarrow})}{\partial c_m}  + \frac{\partial \ln(\Psi_{J_{c_m}})}{\partial c_m}}$\label{ww}
				
				\item Divide he standard expression $\frac{\mathrm d}{\mathrm dt}(\det\mathbf A)=(\det\mathbf A)\mathop{\textrm{tr}}\biggl(\mathbf A^{-1}\frac{\mathrm d\mathbf A}{\mathrm dt}\biggr)$ by $\det\mathbf A$ to get: $\frac{d}{dt}\ln\det \mathbf A(t) = \mathrm{tr} \left(\mathbf A^{-1} \frac{d \mathbf{A}}{dt}\right) = \sum_{i=1}^{N} \sum_{j=1}^{N} A^{-1}_{ij} \dot{A}_{ji}$
				
				\item This expression inserted in step (\ref{ww}) gives:
				$\boxed{\frac{\partial \ln \Psi_{T_{c_m}}}{\partial c_m} = (\sum_{i=1}^{N} \sum_{j=1}^{N} D_{ij}^{-1} \dot{D}_{ji})_{\uparrow} + (\sum_{i=1}^{N} \sum_{j=1}^{N} D_{ij}^{-1} \dot{D}_{ji})_{\downarrow} + \frac{\partial \ln(\Psi_{J_{c_m}})}{\partial c_m}}$
			\end{enumerate}
		\end{alertblock}
	\end{scriptsize}

\end{frame}



% % % % % 	\begin{scriptsize}
% % % % % 		\begin{alertblock}{Split Slater-Jastrow}
% % % % % 			$$\Psi_{T}(\bfv{x}) = |\bfv{D}|_{\uparrow} |\bfv{D}|_{\downarrow} \prod_{i=1}^{N} \prod_{i=j+1}^{N} J(r_{ij})$$
% % % % % 		\end{alertblock}
% % % % % 	\end{scriptsize}


% % % % Uncovering Tagged Formulas Piecewise
% % % % % \begin{align}
% % % % %   A &= B \\
% % % % %     \uncover<2->{&= C \\}
% % % % %     \uncover<3->{&= D \\}
% % % % %     \notag
% % % % %   \end{align}
% % % % % \vskip-1.5em
% % % % 
% % % % 
% % % % \end{frame}


% % % \begin{frame}{The Quantum Varitional Monte Carlo algorithm}
% % % \begin{itemize}
% % %  \item \textcolor{red}{Zero temperature method.}
%
% % % \begin{description}
% 
% % %  $\langle \Op{H} \rangle = \int P(\bfv{R}) f(\bfv{R}) d \bfv{R}, \qquad \bfv{R} \in \mathcal{R}^N$
% % %  \item Measure $\Op{H}$ by sampling the \textcolor{blue}{probability distribution} 
% % %  
% % %  
% % %  \item The points \bfv{R}_i are generated using random methods.
% % %  \item We introduce noise in the problem
% % % %  \subitem The results have error bars
% % % %  \subitem BUT it might a good way to proceed.
% % %  \item Reduce considerably the problem's complexity.
% % %  \item It scales better than other methods.
% % % \end{description}
% % % \end{frame}

% % % % \subsection{Monte Carlo Integration}
% % % % \begin{frame}{Monte Calo method}
% % % % \begin{description}
% % % %  \item[Stochastic:] Relies on high quality random number generator and a Markov process. 
% % % %  \item[Variational Monte Carlo:] A trial wave function is optimized using a set of adjustable parameters.
% % % % \end{description}