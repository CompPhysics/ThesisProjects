%\setlength{\parindent}{0pt}   % Set no indentation in the beginning of each paragraph.
%\setlength{\parskip}{2ex}     % Separate lines each paragraph.

\chapter{Software design and implementation}\label{implementation}
The use of compiled low level languages such as Fortran and C is a deeply-rooted tradition among computational scientists and physicists for doing numerical simulations requiring high performance. However, the increased demand for more flexibility has motivated the development of object oriented languages such as C++  and Fortran 95. The use of encapsulation, polymorphism and inheritance allows for setting up source codes into a public interface, and a private implementation of that interface. The style includes overloading of standard operators so that they have an appropriate behaviour according to the context and creating subclasses that are specializations of their parents. Big projects are split into several classes communicating with each other.  This has several advantages, in particular 
for debugging, reuse, maintenance and extension of the code. These ideas will be used in the rest of the chapter for designing and implementing a Quantum Variational Monte Carlo simulator. %Three alternatives are explored.

\section{Structuring a software for QVMC in close shell systems}\label{softwareStructure}
When design software one should take into account the efficient usage of memory, the possibility 
for future extensions (flexibility) and software structure (easy to understand for other programmers and the final users)\cite{Shapira2006,Langtangen2003}. \\
\\
In general, several approaches exist in the design of software, one of them is the use of prototypes. The high level programming language Python\footnote{Python is an interpreted object-oriented language that shares with Matlab many of its characteristics, but which is much more powerful and flexible when equipped with tools for numerical simulations and visualization. Because Python was designed to be extended with legacy code for efficiency, it is easier to interface it with  software written in C++, C and fortran than in other environments. A balance between computational efficienty, to get fast codes, and programming efficiency, is preferred. 

} has been used in this thesis with this in mind. In essence what a prototype does is to help the designer exploring alternatives, making performance tests, and modifying design strategies.

\subsection{Implementing a prototype in Python}\label{PythonPrototype}
We start implementing the Quantum Variational Monte Carlo algorithm from chapter \ref{QMC}. To do so, we implement a \citecode{Vmc} class which serves to administrate the simulation, setting the parameters of the simulation, creating objects related to the algorithm, and running it, as shown below. 
% \begin{lstlisting}[language=Python]
\begin{Python}
...
#Import some packages
...
class VMC():
  def __init__(self, _dim, _np, _charge, ...,_parameters):
    ...
    #Set parameters and create objects for the simulation
    ...
    particle = Particle(_dim, _np, _step) 
    psi      = Psi(_np, _dim, _parameters)
    energy   = Energy(_dim, _np, particle, psi, _nVar, _ncycles, _charge) 
    self.mc  = MonteCarlo(psi, _nVar, _ncycles, _dim, _np, particle, energy)

  
  def doVariationalLoop(self):
    if (self.mcMethod=='BF'):
      for var in xrange(nVar):
		    self.mc.doMonteCarloBruteForce()
		    self.mc.psi.updateVariationalParameters()
    else:
      for var in xrange(nVar):
	      self.mc.doMonteCarloImportanceSampling()
	      self.mc.psi.updateVariationalParameters()
  
  ...
\end{Python}
% \end{lstlisting}
The information regarding the configuration space can be encapsulated in a class called 
\citecode{Particle} with the straightforward implementation\\
%%%\begin{lstlisting}[language=Python]
\begin{Python}
...
class Particle(): 
  def __init__(self, _dim, _np, _step):
    # Initialize matrices for configuration space
    ...

  def acceptMove(self, i):
    for j in xrange(dim):
      self.r_old[i,j] = self.r_new[i,j]

  def resetPosition(self, i):
    for k in xrange(self.np):
      if(k != i):
				for j in xrange(self.dim):
					self.r_new[k,j] = self.r_old[k,j]

	def setTrialPositionsBF(self):
    self.r_old = self.step*random.uniform(-0.5,0.5,size=np*dim).reshape(np,dim)

  def setTrialPositionsIS(self):
    for i in xrange(self.np):
      for j in xrange(self.dim):
				self.r_old[i,j] = random.normal()*sqrt(self.step) # step=dt
\end{Python}

\begin{Python}

  def updatePosition(self, i):
    #Note use of vectorized loop
    self.r_new[i,0:dim]=self.r_old[i,0:dim] + self.step*(random.random()-0.5)
    self.resetPosition(i)
\end{Python}
%%%%\end{lstlisting}
\noindent
In the class \citecode{MonteCarlo} we implement a function \citecode{equilibrate()} meant to deal with the equlibration of the Markov chain, that is to reach the most likely state of the system before we start with the production part of the simulation. The kind of sampling with which to run the simulation can be choosen from the \citecode{Vmc} class.
% % \begin{lstlisting}[language=Python]
\begin{Python}

class MonteCarlo():
  def __init__(self, psi, ..., _ncycles, ..., particle, energy):
    #Set data and member functions
    ...
  
  def doMonteCarloImportanceSampling(self):
    ...  
    #Matrices storing the quantum force terms
    Fold = zeros(np*dim).reshape(np, dim)  
    Fnew = zeros(np*dim).reshape(np, dim)
    
    #Reset energies
    self.energy.resetEnergies()
    deltaEnergy = 0.0		
    ...
    #Set initial trial positions (space configuration)
    self.particle.setTrialPositionsIS()
    
    #Reach the most likely state
    self.equilibrate()

    #Evaluate the trial wave function for the current positions
    wfold = getPsiTrial(r_old)
			    
    #Evaluate the quantum force at the current positions
    Fold = self.psi.getQuantumForce(r_old, wfold)
    
    for c in xrange(1, self.ncycles+1, 1):	# Monte Carlo loop
      for i in xrange(np):									# Loop over particles
				for j in xrange(dim):								# Loop over coordinates
					#Suggest a trial move
					r_new[i,j] = r_old[i,j] + random.normal()*sqrt(dt) + Fold[i,j]*dt*D

				#Reset the position of the particles, but the ones containing i as its first index
				resetPostion(i)

				#Evaluate the trial wave function at the suggested position
				wfnew = self.psi.getPsiTrial(r_new)
						
				#Evaluate the quantum force for particles at the suggested position
				Fnew = self.psi.getQuantumForce(r_new, wfnew)
						
				#Compute the Green's function
				greensFunction = 0.0
						
				for j in xrange(dim):
					greensFunction 	+= 0.5*(Fold[i,j] + Fnew[i,j]) \
														*(D*dt*0.5*(Fold[i,j] - Fnew[i,j])- r_new[i,j] + r_old[i,j])
				
				greensFunction = exp(greensFunction)
\end{Python}

\begin{Python}

				acceptanceRatio = greensFunction*wfnew*wfnew/wfold/wfold

				#Do the metropolis/hasting test
				if(random.random() <= acceptanceRatio):
					for j in xrange(dim):
						r_old[i,j] = r_new[i,j] 
						Fold[i,j] = Fnew[i,j]   
					
					#Update the wave function
					wfold = wfnew

      #Compute local energy and its cumulants
      self.energy.getLocalEnergy(wfold)
    self.energy.computeEnergyExpectationValue()
    self.energy.resetEnergies()
\end{Python}
% % % % \end{lstlisting}
% % % % % % % % % The main difference with the brute force method is that it does not include terms related to the quantum force. Moreover, the new position is suggested according to:
% % % \begin{lstlisting}[language=Python]
% % % % % % % % % \begin{Python}
% % % % % % % % % 
% % % % % % % % % self.particle.r_new[i,j] = self.particle.r_old[i,j] + step*(random.random() - 0.5)
% % % % % % % % % 
% % % % % % % % % \end{Python}
% % % % % % % % % % % % \end{lstlisting}
\noindent
To encapsulate information concerning the trial wave function and energy we create the \citecode{Psi} and \citecode{Energy} classes, respectively.
% % % \begin{lstlisting}[language=Python]
\begin{Python}
...
class Psi:
  def __init__(self, _np, _dim, _parameters):
    ...
    self.cusp = zeros((_np*(_np-1)/2)) 		# Cusp factors
    self.setCusp()
  
  def setCusp(self):
    ...
    if(self.np==2):
      self.cusp[0] = 0.5
    else:
      for i in xrange(size):
				self.cusp[i] = 0.5
	
      self.cusp[0] = 0.5
      self.cusp[5] = 0.5

  #Define orbitals (single particle wave functions)
  def phi1s(self, rij):
    return exp(-self.parameters[0]*rij)
    
  def phi2s(self, rij):
    return (1.0 - self.parameters[0]*rij/2.0)*exp(-self.parameters[0]*rij/2.0)

  def getPsiTrial(self, r):
    return self.getModelWaveFunctionHe(r)*self.getCorrelationFactor(r)
	  
  def getCorrelationFactor(self, r):
    correlation = 1.0
    for i in xrange(self.np-1):
      for j in xrange(i+1, self.np):
				idx = i*(2*self.np - i-1)/2 - i + j - 1
				
		r_ij = 0.0
		for k in xrange(self.dim):
			r_ij += (r[i,k]-r[j,k])*(r[i,k]-r[j,k])
		r_ij=sqrt(r_ij)
		
		correlation *= exp(self.cusp[idx]*r_ij/(1.0 + self.parameters[1]*r_ij))
    return correlation
\end{Python}

\begin{Python}

  #Set the Slater determinant part of Be
  def getModelWaveFunctionBe(self, r):
    argument = zeros((self.np))
    wf = 0.0
    ...
    psi1s = self.phi1s #Shortcut, convenient in for-loops
    ...
    for i in xrange(self.np):
      argument[i] = 0.0
      r_singleParticle = 0.0
				for j in xrange(self.dim):
					r_singleParticle += 	r[i,j]*r[i,j]
	
      argument[i] = sqrt(r_singleParticle)
	
    wf = (psi1s(argument[0])*psi2s(argument[1])	\
				-psi1s(argument[1])*psi2s(argument[0]))	\
				*(psi1s(argument[2])*psi2s(argument[3])	\
				-psi1s(argument[3])*psi2s(argument[2]));
    
    return wf

  #Set the Slater determinant part of He
  def getModelWaveFunctionHe(self, r):
    argument = 0.0
    for i in xrange(self.np):
      r_singleParticle = 0.0
      for j in xrange(self.dim):
				r_singleParticle += r[i,j]*r[i,j]
      argument += sqrt(r_singleParticle)

    return exp(-argument*self.parameters[0])

	#Compute the quantum force numerically 
	def getQuantumForce(self, r, wf): 	 
    ...
    for i in xrange(np):
      for j in xrange(dim):
      				r_plus[i,j] = r[i,j] + h
				r_minus[i,j] = r[i,j] - h
				wfminus = self.getPsiTrial(r_minus) 
				wfplus = self.getPsiTrial(r_plus)
				
				quantumForce[i,j] = (wfplus - wfminus)/wf/h
				
				r_plus[i,j] = r[i,j]
				r_minus[i,j] = r[i,j]
	
    return quantumForce
...
\end{Python}
\noindent
Class \citecode{Energy} is also equipped with some functions to do the statistical analysis of uncorrelated data.

% % % % 	\begin{lstlisting}[language=Python]
\begin{Python}
...
class Energy:	
  def __init__(self, dim, np, particle, psi, maxVar, ncycles, charge):	
    self.cumEnergy = zeros(maxVar)			  #Cumulant for energy
    self.cumEnergy2= zeros(maxVar)			#Cumulant for energy squared
		...
\end{Python}

\begin{Python}

  def getLocalEnergy(self, wfold):
	  EL 	   = self.getKineticEnergy(wfold) + self.getPotentialEnergy()
	  self.E  += EL
	  self.E2 += EL*EL 
  ...

  def getPotentialEnergy(self):
    ...
		PE = 0.0
		
    # Contribution from electron-proton potential
    for i in xrange(np):
      r_single_particle = 0.0
      for j in xrange(dim): 
				r_single_particle += r_old[i,j]*r_old[i,j]
    PE -= charge/sqrt(r_single_particle) 

    # Contribution from electron-electron potential
    for i in xrange(0, np-1):
      for j in xrange(i+1, np):
			r_12 = 0.0
			for k in xrange(0, dim):
				r_12 += (r_old[i,k] - r_old[j,k])*(r_old[i,k] - r_old[j,k])
			PE += 1.0/sqrt(r_12) 
    return PE

  def getKineticEnergy(self, wfold):
		KE = 0.0  		 	 	#Initialize kinetic energy
		h = 0.001		 		 	#Step for the numerical derivative
    h2= 1000000.	 		#hbar squared
    ...
    for i in xrange(np):
      for j in xrange(dim):
				r_plus[i,j] = self.particle.r_old[i,j] + h
				r_minus[i,j]= self.particle.r_old[i,j] - h
				wfplus  = self.psi.getPsiTrial(r_plus)
				wfminus = self.psi.getPsiTrial(r_minus)

				# Get the laplacian_wave_function to wave_function ratio
				KE -= (wfminus + wfplus - 2.0*wfold)
				
				r_plus[i,j]=self.particle.r_old[i,j].copy()
				r_minus[i,j]  = self.particle.r_old[i,j].copy()

  return 0.5*h2*KE/wfold # include electron mass and hbar squared
\end{Python}
% % % % \end{lstlisting}
The calling code is just
% % % % \begin{lstlisting}[language=Python]
\begin{Python}
import sys
from VMC import *

#Set parameters of simulation
dim 	= 3	  						#Number of spatial dimensions
nVar 	= 10							#Number of variations for a quick optimization method
ncycles = 10000					#Number of monte Carlo cycles
np	=2						# Number of electrons
charge	=2.0		 	# Nuclear charge
...
vmc = VMC(dim, np, charge, mcMethod, ncycles, step, nVar, parameters)
vmc.doVariationalLoop()
vmc.mc.energy.printResults()
\end{Python}
% % % % \end{lstlisting}
The source code\footnote{For details, we refer to the CD included with this thesis under \citecode{QVMC/Python}.} above is relatively straightforward to implement, besides being short (600 lines). The syntax used is clear, compact and close to the way we formulate the expressions mathematically. The language is easy to learn and one can develop a program and modify it easily. The dual nature of Python of being both an object oriented and scripting language is convenient in structuring the problem such that we can split it into smaller parts, while testing quickly the results (output) against analytical or experimental data. A limitation, especially in problems managing a high number of degrees of freedom, as happens in quantum mechanics, is its speed as shown in figures \ref{executionTimeHePyCpp} to \ref{delayExecutionTimeHePyCpp}.\\
\\
Computing the ground energy of a He atom (2 electrons) using 10000 Monte Carlo cycles and five parameter variations with this Python simulator takes about  1.5 mi\-nu\-tes on a Pentium-4 laptop running at 2.4 GHz. On the other hand, attempting to do the same for a Be atom (4 electrons) with only $5 \times 10^4$ Monte Carlo cycles (not enough to get statistically acceptable results for Be) takes more than one hour. Several alternatives exists for being able to simulate bigger systems than He. Three of them are exposed here and explored in the following: 
\begin{enumerate}
 \item Keeping the software structure as above, but implement all the classes in C++ (or another low-level programming language).
 \item Integrate Python and C++ in a way that the portions of the software that requires high-performance computing considerations can be implemented in C++. The administrative parts are then written in Python.
 \item Exploting the object-oriented characteristics of C++ for developing an efficient, flexible and easy to use software (optimized algorithm).
\end{enumerate}


\subsection{Moving the prototype to a low level programming language}\label{cppPrototype}
The implementation of the simulator in C++ follows more or less the recipe above. Here, however, we separate the potential energy from the Hamiltonian by introducing a \citecode{Potential} class, which becomes an object member of the class \citecode{Energy}. The use of dynamic memory to manipulate data structures is usually a source of bugs, as C++ does not have automatic garbage collection as Python does. \\
\\
In the present implementation we use a ready made class for manipulating array data structures. {\citecode{MyArray} class stores all the array entries in contiguous blocks of me\-mo\-ry\cite{Langtangen2000}. This design aims at interfacing Python and C++, as the Python arrays are, also, stored in the same way. The class \citecode{MyArray} supports arrays in one, two and three dimensions.}\cite{HPL}. Because it is a template class, it can be reused easily with any type. Moreover, changing its behaviour is easily done in only one place and to free of me\-mo\-ry is made in its destructor\cite{Langtangen2000}.\\


\subsection{Mixing Python and C++}
There are several ways one can integrate Python with C/C++. Two of them are called extending and embedding, respectively. The first one, extending, involves creating a wrapper for C++ that Python imports, builds, and then can execute. We will focus on the wrapper extension. In the second alternative, embedding, the C++ part is given direct access to the Python interpreter. Python is extended for many different reasons. A developer may want to use an existing C++ library, or port work from an old project into a new Python development effort.\\
\\
\begin{lstlisting}[language=c++]
...
template< typename T > class MyArray
{
 public:
  T* A;                   // the data
  int ndim;               // no of dimensions (axis)
  int size[MAXDIM];       // size/length of each dimension
  int length;             // total no of array entries
  T* allocate(int n1);
  T* allocate(int n1, int n2);
  T* allocate(int n1, int n2, int n3);
  void deallocate();
  bool indexOk(int i) const;      
  bool indexOk(int i, int j) const;
  bool indexOk(int i, int j, int k) const;
  
 public:
  MyArray() { A = NULL; length = 0; ndim = 0; }
  MyArray(int n1) { A = allocate(n1); }
  MyArray(int n1, int n2) { A = allocate(n1, n2); }
  MyArray(int n1, int n2, int n3) { A = allocate(n1, n2, n3); }
  MyArray(T* a, int ndim_, int size_[]);
  MyArray(const MyArray<T>& array);
  ~MyArray() { deallocate(); }

  bool redim(int n1);           
  bool redim(int n1, int n2);   
  bool redim(int n1, int n2, int n3);   
  
  // return the size of the arrays dimensions:
  int shape(int dim) const { return size[dim-1]; }
  
  // indexing:
  const T& operator()(int i) const;
  T& operator()(int i);
  const T& operator()(int i, int j) const;
  T& operator()(int i, int j);
  const T& operator()(int i, int j, int k) const;
  T& operator()(int i, int j, int k);
  
  MyArray<T>& operator= (const MyArray<T>& v);
  
  // return pointers to the data:
  const T* getPtr() const { return A;}   
  T* getPtr() { return A; }
  
  ...
};
\end{lstlisting}
\noindent
Before attempting to extend Python, we should identify the parts to be moved to C++. The results of a profile of the  prototype discussed in section \ref{PythonPrototype} are summarized in tables \ref{profileHe} and \ref{profileBe}. It is not surprising that \citecode{Energy.py}, \citecode{MonteCarlo.py} and \citecode{Psi.py} are the most time consuming parts of the algorithm. It is there where the major number of operations and calls to functions in other classes are carried out. Fortunately, most of the work needed to rewrite these classes to their equivalent C++ part 
has been already carried out in the last subsection.\\
\\
The class \citecode{Vmc.py} is responsible of initializing the C++ objects of type \citecode{Psi}, \citecode{Particle}, \citecode{Potential}, \citecode{Energy} and \citecode{MonteCarlo}. Letting Python create the objects has the advantage of introducing automatic garbage collection, reducing the risk implied in managing memory. Memory handling is very often a source of errors in C++. It is also often very difficult to spot such errors.\\
\\
To access C++ via an extension we write a wrapper. The wrapper acts as a glue between the two languages, converting function arguments from Python into C++, and then returning results from C++ back to Python in a way that Python canunderstand and handle. Fortunately, SWIG (Simplified Wrapper and Interface Generator)\cite{Beazley} does much of the work automatically.\\
\\
Before trying to implement a wrapper for the present application, one should note that the \citecode{Psi} class takes a generic object of type \citecode{MyArray<double>} as argument in its constructor. It means that we should find the way of converting the \emph{variational parameters} (a numpy object) into a reference to an object of type \citecode{MyArray<double>}. This kind of conversion is not supported by SWIG and has to be done manually. Doing this for a lot of functions is a hard and error prone job. \\

\begin{lstlisting}[language=c++]
...
#include "MyArray.h"
...
class Psi{
  private:
    ...
    MyArray<double> cusp;			// Cusp factors in the Jastrow form
    MyArray<double> varPar;		// Variational parameters
    MyArray<double> quantumForce;	

  public:
    Psi(MyArray<double>& varParam, int _np, int dim);
    void setParameters(MyArray<double>& varPar);
    double getCorrelationFactor(const MyArray<double>& r);
    double getModelWaveFunctionHe(const MyArray<double>& r);
    MyArray<double>& getQuantumForce(const MyArray<double>& r, double wf);
    ...
}; 
\end{lstlisting}
\noindent
A conversion class \citecode{Convert} specially adapted for \citecode{MyArray} class has already been proposed in \cite{HPL}. The advantage with it is that we do not need to interface the whole \citecode{MyArray} class. Instead, the conversion class is equiped with static functions for converting a \citecode{MyArray} object to a \citecode{Numpy} array and vice versa. The conversion functions in this class can be called both manually or using SWIG to automatically generate wrapper code. The last option is preferible because the conversion functions has only pointers or reference as input and output data.

\begin{lstlisting}[language=c++]
...
#include <Python.h>
#include <pyport.h>
#include <numpy/arrayobject.h>
#include "MyArray.h"
...
class Convert
{
 public:
    Convert();
  ~Convert();

    // borrow data:
    PyObject*        my2py (MyArray<double>& a);
    MyArray<double>* py2my (PyObject* a);

    // copy data:
    PyObject*        my2py_copy (MyArray<double>& a);
    MyArray<double>* py2my_copy (PyObject* a);

    // npy_intp to/from int array for array size:
    npy_intp         npy_size[MAXDIM];
    int              int_size[MAXDIM];
    void             set_npy_size(int*      dims, int nd);
    void             set_int_size(npy_intp* dims, int nd);

    // print array:
    void             dump(MyArray<double>& a);

    ...
    #Code necessary to make callbacks
    ...
 
};
\end{lstlisting}
\noindent
With the information above we create the SWIG interface file \citecode{ext\_QVMC.i} having the same name as the module \cite{Beazley}
\begin{src}
%module ext_QVMC
%{
#include "Convert.h"
#include "Energy.h"
#include "MonteCarlo.h"
#include "Psi.h"
#include "Particle.h"
#include "Potential.h"
%}
%include "Convert.h"
%include "Energy.h"
%include "MonteCarlo.h"
%include "Particle.h"
%include "Potential.h"
%include "Psi.h"
\end{src}
\noindent
To build a Python module with extension to C++ we run SWIG with the options \citecode{-Python} and \citecode{-c++}. Running
\begin{src}
swig -c++ -Python ext_QVMC.i
\end{src}
generates a wrapper code in \citecode{ext\_QVMC\_wrp.cxx} and the Python module \citecode{\_ext\_QMVC.py}. Assuming that all the sources files are in the same directory, this file has to be compiled
\begin{src}
c++ -c -O3 *.cpp *.cxx -I/usr/include/Python2.6/
\end{src}
and linked to a shared library file with name \citecode{\_ext\_QVMC.so}
\begin{src}
c++ -shared -o _ext_QVMC.so *.o
\end{src}
The Python \citecode{VMC.py} class calling the C++ code has to be slightly adjusted such that it can use the new extension module \citecode{ext\_QVMC}.

\begin{Python}

from math import sqrt   # use sqrt for scalar case
from numpy import*
import string

# Set the path to the extensions
import sys
sys.path.insert(0, './extensions')
import ext_QVMC

class Vmc():
  def __init__(self, _Parameters):
    # Create an object of the 'conversion class' for 
    # convert Python-to-C++-to-Python data structure.
    self.convert = ext_QVMC.Convert()
    
    # Get the paramters of the currrent simulation
    simParameters 	= _Parameters.getParameters()
    
    self.nsd 	  	= simParameters[0]
    self.nel 		= simParameters[1]
    self.nVar 		= simParameters[2]
    self.nmcc		= simParameters[3]
    self.charge		= simParameters[4]
    self.step 		= simParameters[5]
    alpha		= simParameters[6]
    beta		= simParameters[7]
    
    # Set the Variational parameters
    self.varpar = array([alpha, beta])
    
    # Convert a Python array to a MyArray object
    self.v_p = self.convert.py2my_copy(self.varpar)
\end{Python}

\begin{Python}
    
    # Create an wave function object
    self.psi 		= ext_QVMC.Psi(self.v_p, self.nel, self.nsd)
    
    self.particle 	= ext_QVMC.Particle(self.nsd, self.nel, self.step)
    self.potential 	= ext_QVMC.Potential(self.nsd, self.nel, self.charge)
    self.energy 	= ext_QVMC.Energy(self.nsd, self.nel, self.potential, self.psi, self.nVar, self.nmcc)
    self.mc		= ext_QVMC.MonteCarlo(self.nmcc, self.nsd, self.nel, 0.001, self.particle, self.psi, self.energy) 
    
  def doVariationalLoop(self):
    nVar = self.nVar
    for var in xrange(nVar):
      self.mc.doMonteCarloLoop()
  
 ...
\end{Python}
\noindent
The \emph{caller}\footnote{The function that calls another (callee).} code in Python resembles the one in the simulator prototype from section \ref{PythonPrototype}, but with an additional class \citecode{SimParameters.py} that is able to read from a conveniently formatted file containing the parameters of the simulation. An example for the Be atom is:
\begin{src}
Number of spatial dimensions 	= 3
Number of electrons 		= 4
Number of variations		= 10
Number of Monte Carlo cycles  	= 100000
Nuclear charge  		= 4.0
Step				= 0.01
Alpha				= 2.765
Beta				= 0.6
\end{src}
\noindent
With this small change, the new caller looks more compact and user friendly.

\begin{lstlisting}[language=Python]
import sys

from SimParameters import * #Class in Python encapsulating the parameters
from Vmc import *

# Create an object containing the parameters of the current simulation
simpar = SimParameters('Be.data')

# Create a Variational Monte Carlo simulation
vmc = Vmc(simpar)

vmc.doVariationalLoop()
vmc.energy.doStatistics("resultsBe.data", 1.0)
\end{lstlisting}

\section{Developing a robust QVMC simulator in pure C++}
The rest of this chapter is dedicated to describe how one can take advantage of the object oriented characteristic of C++ to create a flexible and robust QVMC simulator. In order to reduce the computational cost of evaluating quantities associated with the trial wave function, we implement instead the algorithms discussed in chapter (\ref{TWF}).
This requires that we add some extra functionality. We discuss these additional functionalities 
in the rest of this chapter\footnote{The corresponding source codes can be found in the CD following this thesis in \citecode{QVMC/parallel/}. Also, an extensive documentation has been prepared and can be compiled by running \citecode{doxygen pyVMC} in a terminal under the directory of the sources codes. It will generate the documentation in html (\citecode{QVMC/parallel/doc}).}.\\

\subsubsection{Parameters of the system}
The \citecode{SimParameters} class encapsulates\footnote{This kind of encapsulation reduces the number of parametersthat are sent to the various functions. It reduces also the possibility of making mistakes when swapping them. Furthemore, the localization of bugs in the code can become easier, because the whole program is divided into small parts.} the parameters of the simulation. It creates an object of a class especially designed to read files in ascii format containing comments as well. In this way the user can easyly know what kind of information the software requires. The function \citecode{readSimParameters()} reads the parameters and loads them as data members. Moreover, functions to return each of them are provided.

\begin{src}
sys = 2DQDot2e            # Two dimensional quantum dot with two electrons.
nsd = 2                   # Number of spatial dimensions.
nel = 2                   # Number of electrons.
potpar = 2.0              # Potential parameter (omega).	
lambda = 1                # Confinament parameter in Coulomb potential.
nmc = 1.000000e+07        # Number of Monte Carlo cycles.
nth = 1.000000e+06        # Number of equilibration steps.
dt = 1.0000e-06           # Step size.
nvp = 2                   # Number of Variational parameters.
alpha = 0.75              # Variational parameter.
beta = 0.16               # Variational parameter.
enableBlocking = no       # Blocking option off.
enableCorrelation = yes   # Correlation part included.
\end{src}


\subsubsection{Initializing and setting in operation a simulation}
We declare a class \citecode{VmcSimulation.h} having a pointer to the \citecode{SimParameters} class. The parameters of the simulation are then used in setting a Variational Monte Carlo simulator in operation. A function \citecode{setInitialConditions()} initializes the space configuration and the trial wave function (includes the gradients and Laplacians of the single wave functions and the Jastrow correlation function). By declaring the virtual functions \citecode{setWaveFunction()} and \citecode{setHamiltonian} to be called by \citecode{setPhysicalSystem()}, the software is able to handle atomic systems and quantum dots, that is either non-interacting or interacting electrons in harmomic-oscillator ike traps. The particular implementation of these functions takes place in the subclass \citecode{VmcApp} depending of the physical system treated.
\begin{lstlisting}[language=c++]
void VmcApp::setHamiltonian(){
  //Set the potential form and add it to the kinetic energy term
  potential = new CoulombAtom(simParameters);
  
  energy = new Energy(simParameters, potential, psi, parallelizer);
}
\end{lstlisting}
By calling \citecode{setHamiltonian()}, the application creates a \citecode{Potential} object which is sent as a pointer to the class \citecode{Energy}. In the same function, we call \citecode{setWaveFunction()} to set the \citecode{Jastrow} and the \citecode{SlaterDeterminant} objects, which are used as argument to initialize the \citecode{WaveFnc} class.\\ 
\\
At the time the \citecode{VmcSimulation} is created, we construct a \citecode{RandomNumberGenerator()} object that deals with (pseudo) random numbers and uses them together with other parameters to create a \citecode{MonteCarlo} object implementing the equilibration and a sampling algorithm as the one shown in figure \ref{chartFlowMA}. The Monte Carlo for-loop is initiated by \citecode{VmcSimulation::runSimulation()}. The current code does not provide an interface for the Monte Carlo class, but it should be straightforward to create one  with the following functions declared as virtual: \citecode{doMonteCarloLoop()}, \citecode{suggestMove()} and \citecode{doMetropolisTest()}.\\
\begin{lstlisting}[language=c++]
...
class VmcSimulation{
  protected:
    ...
    Energy *energy;       // Hamiltonian
    MonteCarlo* mc;       // Monte Carlo method
    Optimizer *optimizer; // Optimization algorithm
    Parallelizer *para;   // Encapsulate MPI
		...
\end{lstlisting}
\begin{lstlisting}[language=c++]
    Potential *potential; 
    RandomGenerator *random;
    RelativeDistance *relativeDistance;   // Vectorial and scalar relative distances
    SimParameters *simpar;                // Parameters of simulation
    SingleStateWaveFuncs **singleStates;  // Single state wave functions
    Slater *slater; 
    WaveFnc *psi;                         // Trial wave function

    int numprocs;
    int myRank;
    int rankRoot;
    ...
  public:
    VmcSimulation(SimParameters *_simpa, int argc, char** argv);
    virtual ~VmcSimulation();
    void dumpResults(string resultsFile);
    void runSimulation();
    void setPhysicalSystem(){}
    void setInitialConditions(long& idum);
    void setPhysicalSystem(long& idum);
    
    virtual void optimizeWaveFunction(MyArray<double>& p, ...); 
    virtual void setRandomGenerator();
    virtual void setHamiltonian(){}
    virtual void setWaveFunction(){}
};

\end{lstlisting}
\noindent
To create a trial wave function object, we first instanciate \citecode{singleStateWaveFuncs} and \citecode{JastrowForm} and use them to create the \citecode{SlaterDeterminant} and the \citecode{Jastrow} objects. Function objects are used to encapsulate the forms of the Jastrow function\footnote{In this case in particular we use a function object, an instance of an class that behaves like a function by virtue of its \emph{function call operator ()} are those that can be called as if they were a function.}. For the single particle wave functions we create an array of pointers to objects of type \citecode{SingleSateWaveFnc}. 
%%%%%\begin{figure*}
\begin{lstlisting}[language=c++]
void VmcApp::setWaveFunction(){

  // Set single state wave functions (spin orbitals in atoms)
  singleStates 		= new SingleStateWaveFuncs*[2]; 
	singleStates[0] = new phi1s(nsd, varPar); 
	singleStates[1] = new phi2s(nsd, varPar);

	// set Slater determinant
	slater  = new Slater(simpar, varPar, singleStates); 

  if(enableCorrelation == true){
   // set correlation function
   setJastrow();
	 // set Slater Jastrow trial wave function 
	 psi = new SlaterJastrow(slater, jastrow, simpar, varPar, relativeDistance); 

  }else{
   // set Slater determinant as trial wave function
	 psi = new SlaterNOJastrow(slater, simpar, varPar, relativeDistance);
	}
}
\end{lstlisting}
%%%%\end{figure*}
% % % \end{src}
% % % \end{center}

\subsubsection{Class structure of the trial wave function}

The definition of the trial wave function, single particle wave functions, the Jastrow correlation function, etc, has to be flexible enough in order to easily be provided by the user for different physical systems and computational methods (numerical or analytical). This kind of behaviour is known as \emph{polymorphism} and it is achieved with the use of interfaces, that is classes that define a general behaviour without specifying it. Examples are \citecode{WaveFnc}, \citecode{Potential.h} and the interface for the single particle wave functions \citecode{SingleStateWaveFuncs.h}.
\begin{lstlisting}[language=c++]
...
class SingleStateWaveFuncs{
  ...
  public:
    ...
    virtual double evaluate(int p, const MyArray<double>& r)=0;
    virtual MyArray<double>& getStateGradient(int p, const MyArray<double>& r)=0;
    virtual double getSecondDerivative(int p, const MyArray<double>& r)=0;
    virtual double getParametricDerivative(int p, const MyArray<double>& r)=0;
    ...
  };
  ...
\end{lstlisting}
A convenient class derived from the interface above is \citecode{SlaterNOJastrow}. Because it does not include a Jastrow correlation factor, it can be used both for computing energies of harmonic oscillators and in the validitation of results derived from simulations with trial wave function including only the Slater determinant part. What one should expect in the last case is that experiments carried out in this way reproduce some of the analytical results derived in chapter \ref{cases}. Moreover, because a Slater determinant is an exact wave function when no correlation is included, it is expected that the variance resulting from the experiments is zero. 
\begin{lstlisting}[language=c++]
class WaveFnc{
  public:
    virtual ~WaveFnc(){} 
    virtual double getLapPsiRatio(int i)=0;
    virtual double getPsiPsiRatio(int p, const MyArray<double>& rnew)=0;
    virtual MyArray<double>& getVariationalParameters()=0;
    virtual MyArray<double>& getParametricDerivative(const MyArray<double>& r)=0;
    virtual void getQuantumForce(int p, MyArray<double>& F)=0;
    virtual void resetJastrow(int i)=0;
    virtual void getTrialRow(int p, const MyArray<double>& rnew)=0;
    virtual void getTrialQuantumForce(int p, MyArray<double>& Fnew, ...) = 0;
    virtual void initializePsi(const MyArray<double>& rold)=0;
    virtual void updatePsi(int k, const MyArray<double>& rnew)=0;	
    virtual void setVariationalParameters(MyArray<double>& varPar)=0;	
 };
\end{lstlisting}
On the other hand, the class \citecode{SlaterJastrow} has \citecode{SlaterDeterminant} and \citecode{Jastrow} as data members. Thus, the functionalities of \citecode{SlaterJastrow} can be carried out independently on each of the member classes and sometimes, as in the case of computing the total quantum force, summed up in the calling class, as suggested by table \ref{psiFunctions}. Furthermore, since the the Slater determinant equals the product of two determinants, we create a \citecode{SlaterDeterminant} class containing two pointers, pointing to a \citecode{SlaterMatrix} whose entries are single particle wave functions for electrons with spin up and down, respectively\footnote{The way the spin is set up is by placing half of electrons in each of the matrices.}. Besides the functionality shown in table \ref{psiFunctions}, this class has some arrays for storing $\phi_j(\bfv{x^{cur}_i})$,  $\phi_j(\bfv{x^{trial}_i})$, $\bfv{\nabla}_i \phi_j(\bfv{x^{cur}_i})$, $\bfv{\nabla}_i \phi_j(\bfv{x^{new}_i})$ and $\nabla^{2}_{i}(\bfv{x_i})$.
% % % \begin{src}
\begin{lstlisting}[language=c++]
class SlaterJastrow: public WaveFnc
{
  private:
    SlaterDeterminant *slater;  
    Jastrow           *jastrow;
    MyVector<double>   varPar;    // variational parametes
    ...
  public:
    ...
    SlaterJastrow(Slater *_slt, Jastrow *_jtr,...);
    void getPsiPsiRatio(...);
    double getLapPsiRatio(...);
    MyArray<double>& getGradPsiRatio(...);
    void setVariationalParameters(MyArray<double>& _varPar);
    ...
}
\end{lstlisting}
% % % % \end{src}
% \end{center}

\noindent Because the operations with Slater matrices are done by moving one particle (one row) at the time, the \citecode{SlaterDeterminant} class needs to know which matrix it has to operate on. Therefore, during the initialization process we define an array of pointers to \citecode{SlaterMatrix} objects called \citecode{curSM} (for current Slater matrix) whose size equals the number of electrons in the system. Half of its entries point to a Slater matrix with spin up and the rest to the one with spin down. This can be used to decide which matrix to compute. Doing so is more efficient than using an \citecode{if-test} each time an electron is moved.\\

% % % % The flow of information in the \citecode{SlaterMatrix} is schematized in figure\footnote{In the updating algorithm we need only to invert the Slater determinant matrix once. This can by calling standard LU decomposition methods.} ({\color{red}{CITAR DIAGRAMA QUE ME FALTA HACER}}).\\
\begin{table}\label{psiFunctions}
\centering
\begin{tabular}{llr}
\toprule[1pt]
\textbf{Class} & \textbf{Function} & \textbf{Implemented equation}\\
\midrule[1pt]
\citecode{SlaterMatrix} & \citecode{getDetDetRatio()}  & \ref{RSD}\\
                        & \citecode{getGradDetRatio()} & \ref{gradDetRatioO}-\ref{gradDetRatioN}\\
			& \citecode{getLapDetRatio()}  & \ref{lapDetRatio}\\
			& \citecode{updateInverse()}   & \ref{updatingInverse}\\
\hline
\citecode{Jastrow}	& \citecode{getJasJasRatio()}  & \ref{padepadeRatio}\\
			& \citecode{getGradJasRatio()} & \ref{padeJastrowGradJasRatio}\\
			& \citecode{getLapJasRatio()}  & \ref{lapJasRatio}\\
\hline
\citecode{SlaterJastrow}& \citecode{getPsiPsiRatio()}  & \ref{acceptanceRatio}\\
			& \citecode{getGradPsiRatio()} & \ref{grad_det_ratio_gen}\\
			& \citecode{getLapPsiRatio()}  & \ref{laplacian_psi_psi_ratio}\\
\bottomrule[1pt]
\end{tabular}\caption{Some functions implemented in relation with the trial wave function.}
\end{table}
\noindent
For the \citecode{Jastrow} we create upper-triangular matrices for the scalar quantities $g(r_{ij})$ and $\nabla^2 g(r_{ij})$ (two per term) , and two for the vectorial term $\bfv{\nabla} g(r_{ij})$, all of them evaluated at the current and new positions. A similar storage scheme is used for the matrix $a_{ij}$ containing the electron-electron cusp condition factors given in Eqs. (\ref{cusp2D}) and (\ref{cusp3D}). The access to the scalar and vectorial (interelectronic) distances is carried out by means of a pointer to a \citecode{RelativeDistance} object\footnote{Information on scalar and vectorial interelectronic relative distances are also stored in upper-triangular matrices by the \citecode{RelativeDistance} class.}. Each time a move is suggested, accepted or rejected, the matrices in \citecode{Jastrow} have to be updated following the algorithms described in sections \ref{optimizingCorrelation} to \ref{lapCorOpt}. Similar comments apply to the class \citecode{RelativeDistance}.\\
\\
Each scalar and vectorial matrix has a size $N(N-1)/2$ and $N(N-1)/2 \times nsd$, respectively, with $N$  the number of electrons in the system and $nsd$ being the number of spatial dimensions. In the implementation, the matrices are stored as one-dimensional arrays in contiguous blocks of memory, using the \emph{generic class} \citecode{MyArray}\footnote{Arrays will be read and written a large number of times during the simulation, so the access should be fast as possible. It is reached by storing all the array entries in contiguous blocks\cite{Langtangen2000}.}. \citecode{MyArray} supports arrays in one, two and three dimensions. The mapping between the entry $(i,j)$ in the upper triangular matrix and the index $idx$ in the 1D-array is given by the equation $idx = i(2 N - i-1)/2 - i + j - 1$, with $0\leq i < N-1$ and $i+1 \leq j < N$.

\subsubsection{Computing the energy, parallelizing with MPI and optimizing with a quasi-Newton method}
The collection of samples and evaluation of the energy happens in the \citecode{Energy} class, which has a pointer to a \citecode{Potential} class with some functions declared virtual. 
\begin{lstlisting}[language=c++]
class Potential{
  public:
    virtual ~Potential(){};
    virtual double getPotentialEnergy(const MyArray<double>& r){
      ...
    };
};
\end{lstlisting}
\noindent
In this way, what we need to do for implementing new potential forms is to inherit from it and overriding its functions in subclases. The creation of a new object and its use has the form
\begin{lstlisting}[language=c++]
  // Create a potential form for atoms
  Potential *coulombAt = new CoulombAtoms(...);
  
  // Get the potential energy
  coulombsAt->getPotentialEnergy(...); 
\end{lstlisting}
On the other hand, the \citecode{Energy} class  does not care about the kind of potential gotten in its constructor. Therefore we do not need to modify it when the potential form changes.
% % % % % For the kinetic energy we use the optimized expression 
% % % % % \begin{equation}\label{lap_det_ratio}
% % % % %  \boxed{KE = -\frac{1}{2}\left[\frac{\nabla^2 \Det{D}_{\uparrow}}{\Det{D}_{\uparrow}} + \frac{\nabla^2 \Det{D}_{\downarrow}}{\Det{D}_{\downarrow}} + \frac{\nabla^2 \Psi_C}{\Psi_C}\right] - \left[\frac{\Grad{\Det{D}_{\uparrow}}}{\Det{D}_{\uparrow}} +  \frac{\Grad{\Det{D}_{\downarrow}}}{\Det{D}_{\downarrow}}\right]\bullet \frac{\Grad{\Psi_C}}{\Psi_C}},
% % % % % \end{equation}
% % % % % where each of the components computed by the \citecode{SlaterJastrow} class are obtained by pointers acting on functions.\\
\begin{lstlisting}[language=c++]
....
class Parallelizer{
  private:
    int numprocs;       // Number of procesors
    int myRank;         // Number (label) of this procesor
    int rankRoot;       // Number (label) of the root procesor
    
    #if PARALLEL
      MPI_Status status;
    #endif
\end{lstlisting}

\begin{lstlisting}[language=c++]
  public: 
    Parallelizer(int argc, char** argv);
    ~Parallelizer(){mpi_final();}
    void mpi_final();	// Finalize MPI
    int read_MPI_numprocs() const;
    int read_MPI_myRank() const;
    int read_MPI_rankRoot() const;
};// end of Parallelizer class
\end{lstlisting}



\begin{lstlisting}[language=c++]
...
#include "Parallelizer.h"
...
class Energy{
  private:
    ...// 
    double eMean;	// Mean energy
    double eSum;	// Cumulative energy 
    double eSqSum;	// Cumulative energy squared
    
    #if OPTIMIZER_ON
      MyArray<double> daMean;	  // Mean value of dPhi/dAlpha
      MyArray<double> daSum;	  // Cumulative dPhi/dAlpha
      MyArray<double> ELdaMean;	  // Mean of local energy times dPhi/dAlpha
      MyArray<double> ELdaSum;	  // Cumulative local energy times dPhi/dAlpha
    #endif
                      
    ofstream ofile, blockofile; 
    int totalNMC;
  
    #if PARALLEL
      double totalESum;
      double totalESqSum;
      MyArray<double> allEnergies;	// Save all the local energies
      MPI_Status status;
    #endif
  
    Potential *potential;
    WaveFnc *psi;
    int numprocs, rankRoot, myRank;
    Parallelizer *para;
    
  public:  
    Energy(SimParameters *simpar, Potential *potential, WaveFnc *psi,Parallelizer * para);
    ~Energy();
    void computeExpectationValues(int mcCycles);
    void computeLocalEnergy(int cycle, const MyArray<double>& rnew);
    void dumpResults(const char* resultsFile);
    double getEnergy();
    MyArray<double>& getLocalParametricDerivative(const MyArray<double>& r);
    void getParametricDerivatives(MyArray<double>& dEda); 
    void resetCumulants();
    void sum_dPda(double E_Local, const MyArray<double>& r);
    void updateCumulants(double E_Local, const MyArray<double>& r);
};
...
\end{lstlisting}
Furthemore, the class defined by \citecode{Parallelizer.h}, appearing on the top of \citecode{Energy.h} and \citecode{VmcSimulation.h} files encapsulates information to handle MPI in a straighforward way.\\
\\
When working in parallel (by setting \citecode{PARALLEL=1} during the compilation time) each processor executes the QVMC algorithm separately. The energy data for each processor is accumulated in its own \citecode{eSum} member. To get the expectation value from all the computations we call \citecode{Energy::dumpResults()} with \citecode{PARALLEL=1} directive. Because \citecode{Energy.h} has access to \citecode{MPI} by means of \citecode{Parallelizer.h}, the only execution needed is
\\
\begin{lstlisting}[language=c++]
  // Collect data in total averages
  MPI_Reduce(&eSum, &totalESum, 1, MPI_DOUBLE, MPI_SUM, 0, MPI_COMM_WORLD);
  MPI_Reduce(&eSqSum, &totalESqSum, 1, MPI_DOUBLE, MPI_SUM, 0, MPI_COMM_WORLD);
\end{lstlisting}
inside this function, which can be translated as: "Collect the cumulative energies from all the processors, sum them and divide by the total number of processors to get the mean energy". When the statistical data are assumed to be uncorrelated, the mean energy, variance and error can be computed in
\\
\begin{lstlisting}[language=c++]
  if(myRank==rankRoot){
    ...
    energy    = totalESum/totalNMC;
    variance  = totalESqSum/totalNMC - energy*energy;
    error     = sqrt(variance/(totalNMC - 1)); 
    ...
  }
\end{lstlisting}

The way the energy is computed above gives only an estimate of the real variance, and it serves only as a reference to be compared with the real variance we obtain when using the blocking technique. The data necessary to do it are obtained from the same function by
\\
\begin{lstlisting}[language=c++]
...
// Set file name for this rank
  ostringstream ost;
  ost << "blocks_rank" << myRank << ".dat";
  
  // Open file for writting in binary format
  blockofile.open(ost.str().c_str(), ios::out | ios::binary); 
  
  // Write to file
  blockofile.write((char*)(allEnergies.A), ncycles*sizeof(double));
  
  blockofile.close(); // close file
\end{lstlisting}
\noindent
The \citecode{Energy} class is also equipped with functions for computing and returning the expectation value of the energy and its derivative with respect to the variational parameters. This information is used by the class \citecode{Optimizer} to optimize the parameters of the wave function that hopefully will lead to the computation of the minimum variational energy. The process of optimizing before attempting to run the production stage of a simulation is illustred below:

\begin{lstlisting}[language=c++]
#include "SimParameters.h"
#include "VmcSpecificApplication.h"
#include "VmcSimulation.h"
...
int main(int argc, char *argv[]){
  // Set some trial variational parameters
  ...
  // Declarations for the optimization
  int iter; double minEnergy=0.0;

  // Set the name of the files for the parameters 
  // of this simulation and the output
  ...
  // Encapsulate the parameters of this simulation in an object
  SimParameters *simpa = new SimParameters();
  ...
  // Create a "simulation" 
  VmcSimulation *vmcatoms = new VmcAtoms(simpa,varParam, argc, argv);

  // Set the Hamiltonian and the trial wave function corresponding
  // to atoms. Initialize space configuration
  vmcatoms->setPhysicalSystem(idum);
	
  // Run the optimizer
  vmcatoms->OptimizeWaveFunction(varParam, iter, minEnergy);
  
  // Output the results of the optimization
  ...
  return 0;
} // end main
\end{lstlisting}
And the class implementing the optimizer is:
%%%\begin{figure*}
\begin{lstlisting}[language=c++]
#include "Energy.h"
#include "MonteCarloIS.h"
#include "WaveFnc.h"
...
class Optimizer{
  private:
		int n;  		   				// Number of variational parameters
    double gtol;	   			// Tolerance for convergence
    Energy *pt2energy;
    MonteCarloIS *pt2mc;
    WaveFnc *pt2psi;
    ...
  public:
    Optimizer(int _n, double _gtol, Energy *_pt2energy, 
	      MonteCarloIS *_pt2mc, WaveFnc *_pt2psi):
	      n(_n), gtol(_gtol), pt2energy(_pt2energy), 
	      pt2mc(_pt2mc), pt2psi(_pt2psi){
      ...
    } 
		
		// Function to be called from VmcApp::OptimizeWaveFunction(...).
    void run(MyArray<double>& p, int& _iter, double& _fret){
      dfpmin(p, _iter, _fret); _iter = _iter+1;
    }
\end{lstlisting}

\begin{lstlisting}[language=c++]

    // Gets the expectation value of the energy for a set of variational 
    // parameters in the current Monte Carlo loop. 
    double func(MyArray<double>& p){
      pt2psi->setVariationalParameters(x);
      pt2mc->doMonteCarloLoop();
      return pt2energy->getEnergy();
    }

    // Sets the vector gradient g=df[1...n] of the function returned by 
    // func() evaluated using the input parameters p.
    void dfunc(MyArray<double>& p, MyArray<double>& g){
      pt2energy->getParametricDerivatives(x, g); 
    }
    
    // Given a starting set of parameters p[1...n] performs a Fletcher-
    // Reeves-Polak-Ribiere minimization on func(). 
    void dfpmin(MyArray<double>& p, int& _iter, double& _fret); 	
   ...   
};
\end{lstlisting}

\subsubsection{Running a simulation}
The body of the main method for the productive phase has almost the same form. One needs only to run a simulation with the optimal parameters found in the step above.
\begin{lstlisting}[language=c++]
...
int main(int argc, char** argv){
  ...
  // Set some trial variational parameters
  ...
  // Encapsulate the parameters of this simulation in an object
  SimParameters *simpa = new SimParameters();
  ...
  // Create a "simulation" 
  VmcSimulation *vmcatoms = new VmcAtoms(simpa,varParam, argc, argv);

  // Initialize the simulation
  vmcatoms->setPhysicalSystem(idum);
	
  // Start the simulation
  vmcatoms->runSimulation();

  // Print results to file
  vmcatoms->dumpResults(resultsFile);
  ...
  return 0;
}
\end{lstlisting}
% % % % \end{figure*}
% \end{src}
% \end{center}
\noindent
An overview of the classes and some of the functions to be implemented by the end user of the simulator are summarized in table \ref{endUserData}. Because of limitations of space and time, it is impossible to make a more detailed description of the simulator. The interested reader is, therefore, invited to consult the documentation of the source code following the appended CD of this thesis.


\begin{table}\label{userImplemented}
\centering
\begin{tabular}{llr}
\toprule[1pt]
\textbf{Class} & \textbf{Function}\\
\midrule[1pt]
\citecode{-} & \citecode{main()}  \\
\hline
\citecode{SingleStateWaveFncApp}  & \citecode{evaluate(...)}\\
				  & \citecode{getStateGradient(...)}\\
				  & \citecode{getLaplacian(...)}\\
				  & \citecode{...}\\
\hline
\citecode{PotentialApp}& \citecode{getPotentialEnergy(...)}\\
			& \citecode{getOneBodyPotential(...)}\\
			& \citecode{...}\\
\hline
\citecode{VmcApp}	&\citecode{setHamiltonian(...)}\\
			& \citecode{setWaveFunction(...)}\\
			& \citecode{...}\\
\bottomrule[1pt]
\end{tabular}\caption{Classes and functions to be implemented by the user of the software.}
\label{endUserData}
\end{table}

% % % % % % % \clearemptydoublepage
