
% % % % % % \setlength{\parindent}{0pt}   % Set no indentation in the beginning of each paragraph.
% % % % % %\setlength{\parskip}{2ex}     % Separate lines each paragraph.

\chapter{Physical background}\label{QM}
\section{The quantum mechanical state}\label{quantumState}
In quantum mechanics, \emph{the state of a system is fully defined by its complex wave function}
$$
\Phi(\bfv{x}, t) = \Phi(\bfv{x_1}, \bfv{x_2}, \ldots,\bfv{x_N}, t),
$$
with $\bfv{x}_i$ being the set of coordinates (spatial or spin) that define the attributes of the $i^{th}-$particle.  The interpretation of the wave function is that $|\Phi(\bfv{x},t)|^2$ represents the probability density of a measure of the particles' displacements yielding the value of $\bfv{x}$ at time $t$. Then, the (quantum mechanical) probability of finding a particle $i=1,2,\ldots,N$ between $\bfv{x_i}$ and $\bfv{x_i} + d\bfv{x_i}$ at time $t$ is given by\cite{Fitzpatrick} 
\begin{eqnarray}\label{Max_Born}
P(\bfv{x}, t) d\bfv{x} & = & \Phi^*(\bfv{x},t) \Phi(\bfv{x},t) d\bfv{x} = |\Phi(\bfv{x}, t)|^2 d\bfv{x}\nonumber\\ 
& = & |\Phi(\bfv{x_1}, \bfv{x_2}, \ldots,\bfv{x_N}, t)|^2 d\bfv{x_1} d\bfv{x_2} \ldots d\bfv{x_N},
\end{eqnarray}
which is a real and positive definite quantity. This is Max Born's postulate on how to interpret the wave function resulting from the solution of Schr\"odinger's equation. It is also the commonly accepted and operational interpretation.\\
\\
Since a probability is a real number between 0 and 1, in order for the Born interpretation to make sense, the integral of all the probabilities must satisfy the \emph{normalization condition}
\begin{eqnarray}
\int_{-\infty}^{\infty} P(\bfv{x},t) d\bfv{x} & = & \int_{-\infty}^{\infty} |\Phi(\bfv{x}, t)|^2 d\bfv{x}\nonumber\\
& = &\int_{-\infty}^{\infty} \int_{-\infty}^{\infty} \ldots \int_{-\infty}^{\infty} |\Phi(\bfv{x_1}, \bfv{x_2}, \ldots,\bfv{x_N}, t)|^2 d\bfv{x_1} d\bfv{x_2} \ldots d\bfv{x_N} = 1\nonumber\\
\end{eqnarray}
at all the times. In consequence, $ \int_{-\infty}^{\infty}\Phi(\bfv{x},t)^*\Phi(\bfv{x},t)d\bfv{x} < \infty$, which implies that the wave function and its spatial derivative has to be finite, continuous and single valued. Moreover, since the probabilities are square integrable wave functions, this means that the Born interpretation constrains the wave function to belong to the class of functions in $L^2$, i.e., the wave function is a square-integrable complex-valued function.

\section{Evolution of the quantum mechanical state}

The temporal evolution of the wave function (and therefore of a quantum mechanical system) is given by the time-dependent Schr\"ondinger equation
\begin{equation}\label{timeDependentSE}
i \hbar \frac{\partial \Phi(\bfv{x}, t)}{\partial t} = \Op{H}\Phi(\bfv{x}, t),
\end{equation}
where $\Op{H}$ is the Hamiltonian operator\footnote{An operator $\Op{A}$ is a mathematical entity which transforms one function into another \cite{Bowman}, . In the following we will see, for example, that $\Op{H}$ is a differential operator, then the expressions containing it are differential equations.}. In fact, \emph{any measurable physical quantity (observable) $A(\bfv{x}, \bfv{p})$, depending on position and momentum, has associated its corresponding quantum mechanical operator}. By setting $\bfv{x} \rightarrow \Op{x}$ and $\bfv{p} \rightarrow - i \hbar \bfv{\nabla}$ we get for operator $\Op{A} = A(\Op{x}, - i \hbar \bfv{\nabla})$.\\
\\
The Hamiltonian operator in  Eq.~(\ref{timeDependentSE}) corresponds to the total energy of the system. It may be expressed in terms of a kinetic and potential operator as
\begin{equation}\label{generalHamiltonian}
\boxed{\Op{H} = \Op{K}(\bfv{x}) + \Op{V}(\bfv{x},t),}
\end{equation}
where the kinetic energy becomes 
\begin{equation}\label{kineticEnergy}
\boxed{\Op{K} = \sum_{i=1}^{N} \Op{t} (\bfv{x_i}) = \sum_{i=1}^{N} \frac{\Op{p}_{i}^{2}}{2m} = - \sum_{i=1}^{N}\frac{\nabla_{i}^{2}}{2m}.}
\end{equation}
When the potential energy does not change with xtime, 
\begin{equation}\label{timeIndependentPotential}
 \boxed{\Op{V}(\bfv{x}) = \underbrace{\sum_{i=1}^{N} \Op{u}(\bfv{x_i})}_{\begin{smallmatrix}
  \text{One-particle} \\
  \text{interaction}
\end{smallmatrix}} + \overbrace{\underbrace{\sum_{i>j}^{N} \Op{v}(\bfv{x_i}, \bfv{x_j})}_{\begin{smallmatrix}
  \text{Two-particle} \\
  \text{interaction}
\end{smallmatrix}} + \cdots}^{\begin{smallmatrix}
  \text{Correlations}
\end{smallmatrix}}.}
\end{equation}
and a solution to  Eq.~(\ref{timeDependentSE}) may be found by the technique of separation of variables. Expressing the total $N$-body wave function as $\Phi(\bfv{x}, t) = \Psi(\bfv{x}) T(t)$ and after substitution in  Eq.~(\ref{timeDependentSE}) one obtains two solutions, one of which is the \emph{time-independent Schr\"ondinger equation}
\begin{equation}\label{manyBodyTISE}
\boxed{\Op{H} \Psi(\bfv{x}) = E \Psi(\bfv{x}).}
\end{equation}
\emph{The only possible outcome of  an ideal measurement of the physical quantity $A$ are the ei\-gen\-va\-lues of the corresponding quantum mechanical operator $\Op{A}$}, i.e., the set of energies $E_1, E_2, \cdots, E_n$ are the only outcomes possible of a measurement and the corresponding eigenstates $\Psi_1, \Psi_2, \cdots, \Psi_n$ contain all the relevant information about the system in relation with the operator they are eigenstates of \cite{Bowman}.\\
\\
In this thesis we are concerned with solving the non-relativistic time-independent Schr\"odinger equation  (\ref{manyBodyTISE}) for electronic system described by central symmetric potentials including a ma\-xi\-mum of two-body interations as expressed in  Eq.~(\ref{timeIndependentPotential}). In the following we will omit the time dependence in the equations.


\section{System of non-interacting particles}\label{nonInteractingParticles}
The substitution of Eq.~(\ref{kineticEnergy}) into  Eq.~(\ref{generalHamiltonian}) yields the following 
Hamiltonian for a time-independent and non-relativistic multi-particle system 
\begin{equation}
 \Op{H}(\bfv{x_1}, \ldots, \bfv{x_N}) = \sum_{i=1}^{N}\left[-\frac{1}{2 m}\nabla_{i}^{2} +  \Op{V}(\bfv{x_1}, \ldots, \bfv{x_N})\right],
\end{equation}
where $N$ is the number of electrons in the system. The first term in the bracket re\-pre\-sents the total kinetic energy. The potential term $V(\bfv{x})$ includes the nature of the interactions between the various particles making up the system as well as those with external forces, as described in  Eq.~(\ref{timeIndependentPotential}).\\
\\
Assuming that the particles do not interact is equivalent to say that each particle moves in a common potential, i.e., $V(\bfv{x}) = \sum_{i=1}^{N} V_i(x_i)$ and therefore we can write   
\begin{equation}\label{HamiltonianSum}
\Op{H} = \sum_{i=1}^{N}\left[-\frac{1}{2 m}\nabla_{i}^{2} +  \Op{V}(\bfv{x_i})\right] =  \sum_{i=1}^{N} \Op{h}_i,
\end{equation}
where $\Op{h}_i$ represents the energy of the $i^{th}$ particle. This quantity is 
independent of the other particles. The assumption of a system of non-interaction particles implies also that their positions are independent of each other. Hence, the total wave function can be expressed as a product of $N-$independent single particle wave functions\footnote{This kind of total wave function is known as the \emph{product state} or \emph{Hartree product}.}\cite{Sherrill,Fitzpatrick,Cramer},
\begin{equation}\label{HartreeProduct}
 \Psi(\bfv{x}) = \Psi(\bfv{x_1},\ldots,\bfv{x_N}) = \phi{\alpha_1}(\bfv{x_1})\phi{\alpha_2}(\bfv{x_2})\ldots \phi{\alpha_N}(\bfv{x_N}) = \prod_{i=1}^{N} \phi_{i}(\bfv{x_i}),
\end{equation}
where $\alpha_i$ denotes the set of quantum numbers necessary to specify a single electron state.\\
\\
Comparing with  Eq.~(\ref{manyBodyTISE}) we arrive to 
\begin{equation}\label{totalHUncorrelatedPart}
\Op{H} \Psi(\bfv{x}) = \sum_{i=1}^{N} \Op{h}_i \phi_i(\bfv{x_i}) = E_i \phi_i(\bfv{x_i}) = E \Psi(\bfv{x}). 
\end{equation}
What this shows is that for a system with $N-$non-interating particles, the total energy of the system equals the sum of the total energy of each single particle, i.e., 
\begin{equation}\label{sumEnergies}
\boxed{E = \sum_{i=1}^{N} E_i.}
\end{equation}


\section{Identical particles and the Pauli exclusion principle}\label{PauliPrinciple}
While the functional form  of Eq.~(\ref{HartreeProduct}) is fairly convenient, it violates a very fundamental principle of quantum mechanics: The \emph{principle of indistinguishability.} It states that it is not possible to distinguish identical particles\footnote{Of course one can distinguish, for example, an electron from a proton by applying a magnetic field, but in general there are not experiments nor theory letting us to distinguish between two electrons.} one from another by intrisic properties such as mass, charge and spin. \\
\\
Denoting the $N-$body wave function by $\Psi(\bfv{x_1}, \ldots, \bfv{x_i}, \ldots, \bfv{x_j}, \ldots, \bfv{x_N})$ and using the probabilistic interpretation  given by Eq.~(\ref{Max_Born}), the squared wave function gives 
us the probability of finding a particle $i$ between $\bfv{x_i}$ and $\bfv{x_i} + d\bfv{x_i}$ and another particle $j$  between $\bfv{x_j}$ and $\bfv{x_j} + d\bfv{x_j}$, and so on. Because there is no interaction that can distinguish the particles, a physical observable must be symmetric with respect to the exchange of any pair of two particles. Then, for particles $i$ and $j$ exchanging labels we get
$$ |\Psi(\bfv{x_1}, \ldots, \bfv{x_i}, \ldots, \bfv{x_j}, \ldots, \bfv{x_N})|^2 = |\Psi(\bfv{x_1}, \ldots, \bfv{x_j}, \ldots, \bfv{x_i}, \ldots, \bfv{x_N})|^2,$$
which can be written as
\begin{equation}\label{indistinguisable}
 |\Psi_{ij}|^2 = |\Psi_{ji}|^2,
\end{equation}
implying that $|\Psi_{ij}| = e^{i \varphi}|\Psi_{ji}|$, where $\varphi$ is a real number representing the phase of the wave function. \\
\\
Using the same argument as above, we can swap the index again yielding $|\Psi_{ji}|~=~e^{i \varphi}|\Psi_{ij}|$. Inserting this expression in  Eq.~(\ref{indistinguisable}) leads to $$|\Psi_{ij}|^2 = e^{i 2 \varphi}|\Psi_{ji}|^2.$$
That is, $e^{i 2 \varphi} = 1$, which has two solutions given by $\varphi = 0$ or $\varphi = \pi$. In other words, 
$$|\Psi_{ij}| = \pm|\Psi_{ji}|.$$
Then, for a system consisting of $N$ \emph{identical particles}, its total wave function must be either \emph{symmetric} or \emph{antisymmetric} under the exchange of two particle labels. Particles described by symmetric functions are said to obey Bose-Einstein statistics and are known as \emph{bosons}. Example are photons and gluons. On the other hand, particles having antisymmetric total wave functions are known as \emph{fermions} and they are modeled by \emph{Fermi-Dirac} statistics. In this group we have electrons, protons, neutrons, quarks and other particles with half-integer spin values. Bosons have integer spin while fermions have half-integer spin.\\
\\
In order to illustrate what the particle symmetry principle implies, let $\Psi(x_i, E)$ be a stationary single particle wave function corresponding to a state with energy $E$. Later, assume that it is properly normalized and consider first a system formed by two identical and non-interacting bosons. A total wave function describing the system and fulfilling the symmetry principle under exchange of labels is
$$\Psi_{\text{boson, E}}(\bfv{x_1}, \bfv{x_2}) = \frac{1}{\sqrt{2}}\left[\psi(\bfv{x_1}, E_\alpha)\psi(\bfv{x_2}, E_\beta) + \psi(\bfv{x_2}, E_\alpha)\psi(\bfv{x_1}, E_\beta)\right].$$
Since the particles are identical, we cannot decide which particle has energy $E_\alpha$ or $E_\beta$, only that one has energy $E_\alpha$ and the other $E_\beta$. Note that the particles can be in the same quantum mechanical state, i.e., $E_\alpha = E_\beta$.\\
\\
For the case of two non-interacting identical fermions, the symmetry principle requires the total wave function to be antisymmetric, that is
\begin{equation}\label{twoElectronsWF}
 \Psi_{\text{fermion, E}}(\bfv{x_1}, \bfv{x_2}) = \frac{1}{\sqrt{2}}\left[\psi(\bfv{x_1}, E_\alpha)\psi(\bfv{x_2}, E_\beta) - \psi(\bfv{x_2}, E_\alpha)\psi(\bfv{x_1}, E_\beta)\right].
\end{equation}
Exchanging the labels of the particles fulfills the symmetry principle, but note that if $E_\alpha = E_\beta$ the total wave function becomes zero, which means that it is impossible for any two identical fermions in an $N-$body system to occupy the same single particle (quantum mechanical) state. This important result is known as the \emph{Pauli exclusion principle.}\\
\\
The number appearing in the denominator of the normalization factor equals the number of permutations of $\bfv{x_1}, \bfv{x_2},\ldots,\bfv{x_N}$ in $\phi_{\alpha_1}(\bfv{x_1}), \phi_{\alpha_2}(\bfv{x_2}), \ldots, \phi_{\alpha_N}(\bfv{x_N})$, which is in general $N!$. Then, a total wave function satisfying the Pauli exclusion principle is 
\begin{equation}\label{SlaterDeterminant}
\boxed{
\Psi_D(\bfv{x}) = \frac{1}{\sqrt{N!}}
 \begin{vmatrix}
 \phi_{\alpha_1}(\bfv{x_1}) & \phi_{\alpha_2}(\bfv{x_1}) & \cdots & \phi_{\alpha_N}(\bfv{x_1})\\
\phi_{\alpha_1}(\bfv{x_2}) & \phi_{\alpha_2}(\bfv{x_2}) & \cdots & \phi_{\alpha_N}(\bfv{x_2})\\
\vdots  & \vdots & \ddots & \vdots  \\
\phi_{\alpha_1}(\bfv{x_N}) & \phi_{\alpha_2}(\bfv{x_N}) & \cdots & \phi_{\alpha_N}(\bfv{x_N})  
 \end{vmatrix},}
\end{equation}
which is known as a \emph{Slater determinant}. 
% % % % % % Some authors prefer the notation $\Psi(\bfv{x}) = \frac{1}{\sqrt{N!}} Det\{\phi_{\alpha_1}(\bfv{x_1})\phi_{\alpha_2}(\bfv{x_2})\ldots \phi_{\alpha_N}(\bfv{x_N})\}$
Observe that if two particles $i$ and $j$ are in the same state ($\alpha_i = \alpha_j$), two of the rows will be equal and the determinant will become zero, telling us that two fermions cannot occupy the same quantum mechanical state, in concordance with \emph{Pauli's exclusion principle}. Moreover, exchange of two columns gives as result a change in the sign of the determinant, meaning that the total wave function of a fermionic system is antisymmetric under the exchange of two particles. 

\section{The Born-Openheimer approximation}
The Hamiltonian of a quantum mechanical system is given by Eq.~(\ref{generalHamiltonian}). For atoms, the kinetic energy operator of Eq.~(\ref{kineticEnergy}) takes the form 
$$
  \hat{T}(\mathbf{x}) = \underbrace{-\frac{\hbar^2}{2M}\nabla^{2}_{0}}_{\begin{smallmatrix} \text{Kinetic energy} \\ \text{of the nucleus} \end{smallmatrix}}
  -\underbrace{\sum_{i=1}^{N}\frac{\hbar^2}{2m}\nabla^{2}_{i}}_{\begin{smallmatrix}\text{Kinetic energy} \\ \text{of the electrons} \end{smallmatrix}},
$$
where $M$ is the mass of the nucleus and $m$ is the electron mass.\\
\\
In a system of interacting electrons and a nucleus there will usually be little momentum transfer between the two types of particles due to their differing masses. Moreover, because the forces between the particles are of similar magnitude due to their similar charge, one can assume that the momenta of the particles are also similar. Hence, the nucleus must have a much smaller velocity than the electrons due to its far greater mass. On the time-scale of nuclear motion, one can therefore consider the electrons to relax to a ground-state given by the Hamiltonian with the nucleus at a fixed location\footnote{The nucleus consists of protons and neutrons. The proton-electron mass ratio is about $1/1836$ and the neutron-electron mass ratio is about $1 / 1839$, so regarding the nucleus as stationary is a natural
approximation. References are books on elementary physics or chemistry.}. This separation of the electronic and nuclear degrees of freedom is known as the \emph{Born-Oppenheimer approximation}\cite{jensen,Reine}.\\
\\
\noindent
In the center of mass (CM) reference system the kinetic energy operator reads
$$
  \hat{T}(\mathbf{x}) = -\underbrace{\frac{\hbar^2}{2(M+Nm)}\nabla^2_{CM}}_{\begin{smallmatrix} \text{Kinetic energy of} \\
  \text{the center of mass}\end{smallmatrix}}
  -\underbrace{\frac{\hbar^2}{2\mu}\sum_{i=1}^{N}\nabla^{2}_{i}}_{\begin{smallmatrix} \text{Kinetic energy} \\ \text{of N electrons} \end{smallmatrix}}
  -\underbrace{\frac{\hbar^2}{M}\sum_{i>j}^{N}\nabla_i\cdot\nabla_j}_{\begin{smallmatrix} \text{Mass polarization due} \\ \text{to nuclear motion} \end{smallmatrix}},
$$
where $\mu = mM/(m+M)$ is the reduced mass for each of the $N$ electrons, i.e., its mass $m$ replaced by $\mu$, because of the motion of the nucleus. \\
\\
In the limit $M\to \infty$ the first and third terms of the equation above become zero. Then, the kinetic energy 
operator reduces to
\begin{equation}\label{totalAtomicKE}
 \boxed{\hat{T} = -\sum_{i=1}^{N}\frac{\hbar^2}{2m}\nabla^2_i.}
\end{equation}
% % % % % % % % % % % The Born-Oppenheimer approximation thus disregards both the kinetic energy of the center of mass as well as the mass polarization term. The effects of the Born-Oppenheimer approximation are quite small and
% % % % % % % % % % % they are also well accounted for{\color{red}{REFERENCIAR}}.\\
\noindent
On the other hand, the potential energy operator  Eq.~(\ref{timeIndependentPotential}) is given by 
\begin{equation}\label{totalAtomicPE}
  \boxed{\hat{V}(\mathbf{x}) = 
  - \underbrace{\sum_{i=1}^{N} \frac{Ze^2}{(4\pi \epsilon_0)r_i}}_{\begin{smallmatrix} \text{Nucleus-electron} \\ \text{potential} \end{smallmatrix}}
  + \underbrace{\sum_{i=1,i<j}^{N} \frac{e^2}{(4\pi \epsilon_0)r_{ij}}}_{\begin{smallmatrix} \text{Electron-electron} \\ \text{potential} \end{smallmatrix}},}
\end{equation}
where the $r_i$'s are the electron-nucleus distances and the
$r_{ij}$'s are the inter-electronic distances. The inter-electronic potentials are the main problem in atomic physics. Because of these terms, the Hamiltonian cannot be separated into one-particle parts, and the problem must be solved as a whole. 


\subsection{Hydrogen like-atom}
An hydrogen atom consist of a proton sorrounded by an electron, being it the simplest of all the atoms. In the Born-Openheimer approximation, the non-relativistic Hamiltonian is given by
\begin{equation}
 \Op{H} = -\frac{\hbar^2}{2 m} \frac{1}{2} \nabla^2 - \frac{Z e^2}{4 \pi \epsilon_0}\frac{1}{r}.
\end{equation}
Because of the form of the Coulomb potential, the problem is said to have spherical symmetry. For the above Hamiltonian, it is possible to get an analytical solution of the time-independent Schr\"ondiger equation~(\ref{manyBodyTISE}) by separation of variables. Rewritting the potential in terms of spherical coordinates and setting 
\begin{equation}
\Psi(r, \theta, \phi) = R(r) P(\theta) F(\phi), 
\end{equation}
leads to a set of three ordinary second-order differential equations whose analytical solutions give rise to three quantum numbers associated with the energy levels of the hydrogen atom.\\
\\
The angle-dependent differential equations result in the spherical harmonic
functions as solutions, with quantum numbers $l$ and $m_l$. These functions are given by\cite{Zettili2001}
\be
    Y_{lm_l}(\theta,\phi)=P(\theta)F(\phi)=\sqrt{\frac{(2l+1)(l-m_l)!}{4\pi (l+m_l)!}}
                      P_l^{m_l}(cos(\theta))\exp{(im_l\phi)},
\ee
with $P_l^{m_l}$ being the associated Legendre polynomials.\\
\\
\noindent
The quantum numbers $l$ and $m_l$ represent the orbital momentum and projection of
the orbital momentum, respectively and take the values $l=0,1,2,\dots$ and $-l \leq m_l \leq l$.\\
\\
In general the radial Schr\"odinger equation (in atomic units) reads,
\begin{equation}
\left[ - \frac{1}{2} \left({1 \over r^2}\frac{d}{dr}\left(r^2 \frac{d}{dr}\right) - {l(l+1)\over r^2} \right) + \frac{Z}{r} \right] R(r) = E R(r).
\end{equation}


\section{When two charged particle approach each other: cusp conditions}\label{cuspConditions}
\emph{Cusp conditions} are boundary conditions that must be satisfied by the wave function where the potential diverges. In general, the divergencies appear when the distance between in\-te\-rac\-ting particles approaches zero. Under these circumstances, the forces between these two particles become dominant. Because the local energy is constant everywhere in the space when the wave function is exact, we should take care of constructing wave functions that, at least, make the local energy finite in space. As a  consequence, each of the singularities in the potential should be cancelled by a corresponding term in the kinetic energy. This condition results in a discontinuity (\emph{cusp}) in the first derivative of the wave function $\Psi$ anywhere two charged particles meet. We analyze these two situations in the following.

\subsection{Electron-nucleus cusp of a hydrogenic atom}\label{electronNucleusCusp}
The exact wave function of hydrogenic atoms in spherical polar coordinates can be written as a product between a radial function $R(r)$ and an angular part $\Omega(\theta,\phi)$, yielding
$$\Psi(\bfv{r}, \theta, \phi) = R(r) \Omega(\theta, \phi).$$ Here we are interested in the behaviour of $\Psi$ as $r \rightarrow 0$. Therefore we examine just the radial part, which is the solution to the radial hydrogenic Schr\"odinger equation,
\begin{equation}\label{radialSE}
 \left(\frac{d^2}{dr^2} + \frac{2}{r} \frac{d}{dr} + \frac{2Z}{r} - \frac{l(l+1)}{r^2} + 2E \right) R(r) = 0,
\end{equation}
where $Z$ is the atomic charge, and $l$ is the quantum number for the orbital angular momentum. \\
\\
\noindent
We consider first the case for $l=0$.
For a given electron approaching the nucleus labeled, in order for the equation above 
to be fulfilled $\frac{2}{r} \frac{d}{dr} + \frac{2Z}{r} = 0$, leading to the cusp condition
\begin{equation}\label{enCups}
\boxed{\left(\frac{1}{R(r)}\frac{dR(r)}{dr}\right)\Big|_{r=0} = -Z, \qquad R(0) \neq 0.}
\end{equation}
For $l>0$, we factor out the radial wave function as $R(r) = r^l \rho(r)$, with $\rho(r)$ being a function not going to zero at the origin. Substitution into the radial Schr\"odinger equation leads to 
$$
\frac{2(l+1)}{r}\rho'(r) + \frac{2Z}{r} \rho(r) + \rho''(r) + 2E \rho(r) = 0
$$
Equating the diverging $1/r-$terms we get the general \emph{electron-nucleus cusp condition},
\begin{equation}\label{enCups2}
\boxed{\frac{\rho'}{\rho}\Big|_{r=0} = -\frac{Z}{l+1}.}
\end{equation}
The generalization for the many-electron case follows from considering the asymptotic behaviour of the exact wave function as two electrons approach each other.


\subsection{Electron-electron cusp conditions}
For this case we consider an electron $i$ approaching an electron $j$, constituting a two-body problem. Now, both of them contribute to the kinetic energy. The expansion of the wave function in term of the relative distance only yields
\begin{equation}
 \left(2\frac{d^2}{dr_{ij}^2} + \frac{4}{r_{ij}} \frac{d}{dr_{ij}} + \frac{2}{r_{ij}} - \frac{l(l+1)}{r_{ij}^2} + 2E \right) R_{ij}(r_{ij}) = 0,
\end{equation}
where $r_{ij}$ is the distance between the two electrons. As in the previous section, this equation leads to the electron-electron cusp condition
\begin{equation}
 \boxed{\frac{\rho{'}_{ij}}{\rho_{ij}}\Big|_{r=0} = \frac{1}{2(l+1)}.}
\end{equation}
For a pair of electrons with opposite spins, the wave function with the lowest energy is an $s$-state ($l=0$). Moreover, because of the Pauli exclusion principle, for two electrons with parallel spin, the wave function with lowest energy is a $p$-state  ($l=1$).  With these two conditions the right hand side equals $1/4$ for parallel spin and $1/2$ for antiparallel spin.


\section{Further considerations}
Since the Slater determinant $\Psi_D$ does not depend on the inter-electronic distance $r_{ij}$ where $i$ and $j$ are electrons of opposite spin values, $\partial \Psi_D/\partial r_{ij} = 0$ and $\Psi_D$ cannot satisfy the electron-electron cusp condition. On the other hand, for parallel spin $\Psi_D \rightarrow 0$ as $rij$ because of the Paulis principle. By setting $\Psi_D(\bfv{x_i}, \bfv{x_j}, \ldots) = \Psi_D(\bfv{x_i}, \bfv{x_i} + \bfv{r_{ij}}, \ldots)$, where $\bfv{r_{ij}} = \bfv{x_j} - \bfv{x_i}$ and expanding around $\bfv{x_j} = \bfv{x_i}$ or $r_{ij} = 0$ we get,
$$
\Psi_D(\bfv{x_i}, \bfv{x_j}, \ldots) = \Psi_D(\bfv{x_i}, \bfv{x_i} + \bfv{r_{ij}} = 0, \ldots) + r_{ij}\frac{\partial \Psi_D}{\partial r_{ij}}\Big|_{r_{ij} = 0} + \ldots
$$
The first term is zero, since two electrons cannot occupy the same position having parallel spin. Furthemore,$\frac{\partial \Psi_D}{\partial r_{ij}} \neq 0 $ in most cases\cite{Hammond}. As in section \ref{electronNucleusCusp} we set $\Psi_D = r_{ij}\rho_{ij}$, but because $\rho_{ij}$ is independent on $r_{ij}$, the Slater determinant cannot satisfy the electron-electron cusp conditions. 
%As a consequence, full effect calculations cannot be represented in simplified computations\cite{Hammond,Nesbet2003}, complicating the task of formulating and computing with accuracy many-body wave functions and the physical properties derived from it. 
In order to cancel this singularity, the wave function has to be modified \cite{Hammond,Nesbet2003}.\\
\\
There are, however, ways of taking into account these types of correlations. Typically, a Slater determinant having as entries single particle wave functions is multiplied by a product of correlation functions depending of the inter-electronic coordinates $f(r_{ij}) \equiv f_{ij}$. The independence that they frequently exhibit
as function of  the single particle wave functions makes it possible to write the total wave function as $\Psi = \Psi_{SD} \Psi_{C}$, with $\Psi_{SD}$ and $\Psi_C$ representing the Slater determinant and the correlation part, respectively. The explicit dependence of $f_{ij}$ on $r_{ij}$ is that the total wave function can now satisfy the cusp conditions by constraining $\Psi_C$ so that for three spatial dimensions,
\begin{equation}\label{cusp3D}
\boxed{\frac{1}{\Psi_C}\frac{\partial \Psi_C}{\partial r_{ij}} = \begin{cases} \frac{1}{4}, & \text{like-spin.} \\
\frac{1}{2}, & \text{unlike-spin.}\end{cases}}
\end{equation}
For the two-dimensional case\cite{Albrigtsen}, 
\begin{equation}\label{cusp2D}
\boxed{\frac{1}{\Psi_C}\frac{\partial \Psi_C}{\partial r_{ij}} = \begin{cases} \frac{1}{3}, & \text{like-spin.} \\
1, & \text{unlike-spin.}\end{cases}}
\end{equation}


\section{A final note on units and dimensionless models}
When solving equations numerically, it is often convenient to rewrite the equation in terms of dimensionless variables. Several of the constants may differ largely in value leading to potential losses of numerical precision. Moreover, the equation in dimensionless form is easier to code, sparing one for eventual typographic errors. It is, also, common to scale atomic units (table \ref{atomicUnits}) by setting $m=e=\hbar=4\pi\epsilon_0=1$.\\
\begin{table}[hbtp]
\begin{center} 
\begin{tabular}{llc}
\toprule[1pt]
{\bf Quantity}                 & {\bf SI}               & {\bf Atomic unit}\\
\midrule[1pt]
Electron mass, $m$               & $9.109\cdot 10^{-31}$ kg & 1 \\
Charge, $e$                      & $1.602\cdot 10^{-19}$ C  & 1 \\
Planck's reduced constant, $\hbar$& $1.055\cdot 10^{-34}$ Js& 1 \\       
Permittivity, $4\pi\epsilon_0$   & $1.113\cdot 10^{-10}$ C$^2$ J$^{-1}$ m$^{-1}$&1\\
Energy, $\frac{e^2}{4\pi\epsilon_0 a_0}$ & $27.211$ eV       & 1 \\
Length, $a_0=\frac{4\pi\epsilon_0 \hbar^2}{me^2}$&$0.529\cdot10^{-10}$ m&1\\
\bottomrule[1pt]
\end{tabular} 
\end{center}
\caption{Scaling from SI to atomic units used in atomic and molecular physics calculations.}
\label{atomicUnits}
\end{table}
\\
With the ideas outlined above, the next chapter will be devoted to developing the theory behind the variational Monte carlo method with a strategy to solve Schr\"ondinger equation numerically.


% % % % % % % % % % % % % In order to  do so, we introduce first the dimensionless variable
% % % % % % % % % % % % % $\rho=r/\beta$, where $\beta$ is a constant we can choose.
% % % % % % % % % % % % % Schr\"odinger's equation is then rewritten as 
% % % % % % % % % % % % % \be
% % % % % % % % % % % % % -\frac{1}{2}\frac{\partial^2 u(\rho)}{\partial \rho^2}-
% % % % % % % % % % % % % \frac{mke^2\beta}{\hbar^2\rho}u(\rho)+\frac{l(l+1)}{2\rho^2}u(\rho)=
% % % % % % % % % % % % % \frac{m\beta^2}{\hbar^2}Eu(\rho).
% % % % % % % % % % % % % \ee
% % % % % % % % % % % % % We can determine $\beta$ by simply requiring\footnote{Remember that we are free
% % % % % % % % % % % % % to choose $\beta$.}
% % % % % % % % % % % % % \be
% % % % % % % % % % % % %     \frac{mke^2\beta}{\hbar^2}=1
% % % % % % % % % % % % % \ee
% % % % % % % % % % % % % With this choice, 
% % % % % % % % % % % % % the constant $\beta$ becomes the famous Bohr radius $a_0=0.05$ nm
% % % % % % % % % % % % % $a_0 =\beta ={\hbar^2}/{mke^2}$.

% % % % % % % % % We introduce thereafter the variable $\lambda$ 
% % % % % % % % % \be
% % % % % % % % %  \lambda = \frac{m\beta^2}{\hbar^2}E,
% % % % % % % % % \ee
% % % % % % % % % and inserting $\beta$ and the exact energy  $E=E_0/n^2$, with
% % % % % % % % % $E_0=13.6$ eV, we have that 
% % % % % % % % % \be
% % % % % % % % %  \lambda = -\frac{1}{2n^2},
% % % % % % % % % \ee
% % % % % % % % % $n$ being the principal quantum number.
% % % % % % % % % The equation we are then going to solve numerically is now
% % % % % % % % % \be
% % % % % % % % % -\frac{1}{2}\frac{\partial^2 u(\rho)}{\partial \rho^2}-
% % % % % % % % % \frac{u(\rho)}{\rho}+\frac{l(l+1)}{2\rho^2}u(\rho)-\lambda u(\rho)=0,
% % % % % % % % % \label{eq:hydrodimless1}
% % % % % % % % % \ee
% % % % % % % % % with the hamiltonian
% % % % % % % % % \be
% % % % % % % % % H=-\frac{1}{2}\frac{\partial^2 }{\partial \rho^2}-
% % % % % % % % % \frac{1}{\rho}+\frac{l(l+1)}{2\rho^2}.
% % % % % % % % % \ee
% % % % % % % % % 
% % % % % % % % % The ground state of the hydrogen atom has the energy
% % % % % % % % % $\lambda=-1/2$, or $E=-13.6$ eV. The exact wave function 
% % % % % % % % % obtained from Eq.~(\ref{eq:hydrodimless1}) is
% % % % % % % % % \be
% % % % % % % % %    u(\rho)=\rho e^{-\rho},
% % % % % % % % % \ee
% % % % % % % % % which yields the energy $\lambda = -1/2$. 
% % % % % % % % % Sticking to our variational philosophy, we could now introduce 
% % % % % % % % % a variational parameter $\alpha$ resulting in a trial
% % % % % % % % % wave function 
% % % % % % % % % \be
% % % % % % % % %    u_T^{\alpha}(\rho)=\alpha\rho e^{-\alpha\rho}. 
% % % % % % % % %    \label{eq:trialhydrogen}
% % % % % % % % % \ee
% % % % % % % % % 
% % % % % % % % % Inserting this wave function into the expression for the
% % % % % % % % % local energy $E_L$ of Eq.~(\ref{eq:locale1}) yields (check it!)
% % % % % % % % % \be
% % % % % % % % %    E_L(\rho)=-\frac{1}{\rho}-
% % % % % % % % %               \frac{\alpha}{2}\left(\alpha-\frac{2}{\rho}\right).
% % % % % % % % %       \label{eq:localhydrogen}
% % % % % % % % % \ee

\clearemptydoublepage
