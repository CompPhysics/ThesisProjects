\chapter{Physics of Quantum Dots: the artificial atoms} %%% Criteria (2 pages) a list of criteria to evaluate the quality of the solution
\label{physicsQD}

As described previously, a quantum dot is a semiconductor whose charge-carriers are confined in all three spatial dimensions, so much confined that quantum effects become visible in many ways: fluorescent effect, quantized conductance, quantized energy spectrum, etc.

The physics behind it involves the electronic structure of the material (here mainly semiconductors) and some basic quantum mechanical effects such as size quantization, quantum tunneling or Coulomb blockade.

This chapter reviews the basic quantum mechanical effects that explain the properties of quantum dots. First it explains what size quantization is and how it happens for a confined particle. Then it presents the properties of semiconductors and the specific features of semiconductors quantum dots, and how it enables the size quantization to occur at larger scale. It then explains the consequences of size quantization on the optical and electronic properties of quantum dots. Finally, it presents the quantum dots in a magnetic field.

\section{Size quantization}
\label{sec:sizeQuanta}
Before applying quantum theory to quantum dots, we will explain how quantization arises and why it is not always noticeable in everyday life.

\paragraph{The particle in a box}

We consider the well-known example of a particle in a one-dimensional box of size $a$ trapped by an infinite potential (see figure~\ref{fig:pI01}). 
The potential $V(x)$ is given below:
\begin{align}
V(x) &= 0, \qquad for \; 0<x<a \nonumber   \\ 
V(x) &= \infty, \qquad for \; x\leq 0,x\geq a
\label{eq:potInfinite}
\end{align}

How do these boundary conditions affect the particle? Due to the wave-particle duality we can write the time-independent Schr\"odinger equation for this system:
\begin{equation}
\frac{d^2 \Psi (x)}{d x^2} = \frac{2 m}{\hbar^2} \left[ V(x) - E\right] \Psi(x)
\label{eq:SchoEqInfinite}
\end{equation}


\begin{figure} 
 \centering
 \subfloat[The potential]{\label{fig:pI01}\includegraphics[width=0.4\textwidth]{IMAGES/boxInfinitePot}}

 \subfloat[First eigenfunctions]{\label{fig:pI02}\includegraphics[width=0.45\textwidth]{IMAGES/boxInfiniteAmpli}}
\; 
\subfloat[Probability densities]{\label{fig:pI03}\includegraphics[width=0.46\textwidth]{IMAGES/boxInfiniteProba}}
 \caption{(a) The potential described by equation~(\ref{eq:potInfinite}). As the particle is confined to the range $0\leq x \leq a$, we say that it is confined to a one-dimensional box.\newline (b) The first few eigenfunctions for the particle in a box are shown together with the corresponding energy eigenvalues. The energy scale is shown on the right with the zero for each level indicated by the dashed line.\newline (c) The square of the magnitude of the wavefunction, or probability density, is shown as a function of distance together with the corresponding energy eigenvalues. The energy scale is shown on the left. The square of the wave function amplitude is shown on the right with the zero for each level indicated by the dashed line.}
 \label{fig:InfiniteWell}
\end{figure}


If we write the solutions of the Schr\"odinger equation under the form $\Psi(x) = A sin(kx) + B cos(kx)$, we can find the constants A and B by using the boundary conditions: at the boundaries of the box $\Psi$ is null ie $\Psi(0)=\Psi(a)=0$
\begin{align}
\Psi(0) &= 0 + B = 0 \\
\Psi(a) &= A sin(ka) = 0 \nonumber
\label{eq:conditionsInfinite}
\end{align}

This can only be satisfied by $B=0$ and if either $A=0$ or if $ka=n \pi$. Setting $A=0$ would mean that the wave function is always zero, which is unacceptable, we conclude that:
\begin{equation}
\Psi_n(x) = A sin(\frac{n \pi x}{a}), \qquad for \; n=1,2,3,4,\dots
\label{eq:solutionInfinite}
\end{equation}

The constant $A$ can be determined by normalization, by saying that $\Psi_n(x)^* \Psi_n(x) dx$ is the probability density, ie the probability of finding the particle in the interval of width $dx$ centered on $x$. The probability density at any given point is shown in figure~\ref{fig:pI03}. Because the probability of finding the particle somewhere in the entire interval $[0,a]$ is one,
\begin{equation}
\int_0^a \Psi_n(x)^* \Psi_n(x) = 1 \nonumber
\end{equation}
From that we obtain the normalized eigenfunctions plotted in figure~\ref{fig:pI02}
\begin{equation}
\Psi_n(x) = \sqrt{\frac{2}{a}} sin \left( \frac{n \pi x}{a} \right)
\end{equation}

Now that we have the eigenfunctions, we can re-introduce them into the Schr\"odinger equation to find the eigenvalues (i.e.\ eigen-energies of the system):
\begin{align}
E_n \Psi_n(x) &= - \frac{\hbar^2}{2m}\frac{d^2 \Psi_n(x)}{dx^2} \\
&=\frac{\hbar^2}{2m} \left( \frac{n \pi}{a} \right)^2 \sqrt{\frac{2}{a}} sin \left( \frac{n \pi x}{a} \right)
\end{align}

It leads to the following expression for the eigenvalues, which are also the possible energies of the system:
\begin{equation}
E_n= \frac{\hbar^2}{2m} \left( \frac{n \pi}{a} \right)^2 = \frac{\hbar^2 n^2}{8 m a^2}, \qquad for \; n=1,2,3, \dots
\end{equation}

Compared to a free particle, we see that the energy for a particle in a box is discrete: this is called quantization and the integer $n$ is a quantum number. Another important result of the calculation is that the lowest energy allowed is greater than zero. The particle has a non zero minimum energy compared to a free particle, known as a zero point energy.


Therefore quantization is simply a result of the confinement of the particle and provides new properties to the particle. By making the box size $a$ tend to infinity, the confinement condition is removed and the discrete energy spectrum becomes continous in this limit. More generally, it means that any particle trapped with some boundaries will experience quantization effect, like the particles trapped in the quantum dots.

One way to see quantization effects is to look for observables. The total energy is one example of an observable that can be calculated once the eigenfunctions of the time-independent Schr\"odinger equation are known. Another observable that comes directly from solving this equation is the probability density, which is the quantum mechanical analogue of position.


\paragraph{Observation of size quantization depending on temperature}
What is the limit size of the confinement so that such quantization effects are observable at our scale? We consider for example that the particle is an electron trapped in a box.
The answer will come from the very small constants we have from the Schr\"odinger equation: the reduced Planck constant $\hbar = h /  \pi = 1.05 \e{-34} \; J.s \; ( kg.m^2.s^{-1} )$ and the mass of the electron $m=9.11  \e{-16} \; kg $. To be noticeable, the energy should be much greater than the thermal energy which is in the order of magnitude of $k_B T$, where $k_B=1.38 \e{-34} \; J . K^{-1} $ is the Boltzmann constant and $T$ the temperature, otherwise thermal fluctuations will disturb the motion of electrons and will smear out the quantization effects.

At room temperature (i.e.\ $20 \celsius \simeq 293 K$), $k_B T \simeq 4.045 \e{-21} \; J$. The gap between the first two energy levels should be greater than this value:
\begin{align}
\Delta E &= E_2 -E_1 = \frac{3 \hbar^2}{8 m a^2}= \frac{4.54 \e{-54}}{a^2} \gtrsim k_B T \\ \nonumber
&\Rightarrow a \lesssim 1.06 \e{-11} m = 0.0106 \; nm
\end{align}
At dilution refrigerator temperatures (i.e.\ $\sim 100 \; mK$), $k_B T \simeq 1.380 \e{-35} \; J$. The gap between the first two energy levels doesn't need to be so big this time:
\begin{align}
\Delta E &= E_2 -E_1 = \frac{3 \hbar^2}{8 m a^2}= \frac{4.54 \e{-54}}{a^2} \gtrsim k_B T \\ \nonumber
&\Rightarrow a \lesssim 5.734 \e{-10} m = 0.573 \; nm
\end{align}

We notice here that it is impossible to observe quantum effect with such a ``free electron'' at room temperature, since the box should be roughly the size of an atom, but it could be done at very low temperature.

We could therefore deduce the same for quantum dots, that are made of one or several electrons confined in a semiconductor. From what we saw above it is not possible to create a quantum dot small enough to observe size quantization at room temperature. However we will see in the following section that the mass of the charge carriers, which influences the limit size of the confinement, is not the same when the material is a semiconductor.


\section{Quantum dots made of semiconductors}
\label{semiconductors}
A semiconductor is a material that has a resistivity value between that of a conductor and an insulator. The conductivity of a semiconductor material can be varied under an external electrical field. 

Most semiconductors on the market are made of silicon (Si). Dozens of other materials are used like germanium~(Ge) or gallium arsenide (GaAs).
Semiconductor materials are the basic constituants of modern electronic devices (radio, computers, telephones, and many others). Semiconductor devices include the transistor, solar cells, many kinds of diodes including the light-emitting diode, the silicon controlled rectifier, and digital and analog integrated circuits. Solar photovoltaic panels are large semiconductor devices that directly convert light energy into electrical energy.

\paragraph{Energy bands in semiconductors}
In a metallic conductor, current is carried by the flow of electrons, but in semiconductors, current can be carried either by the flow of electrons or by the flow of positively-charged "holes" in the electron structure of the material.
As shown in the previous section~\ref{sec:sizeQuanta} electrons trapped in matter will experience discretized energies. However compared to the particle in a box, electrons in semiconductors as in other solids will not have narrow discrete energy levels but tickher allowed bands of energy separated by forbidden gaps between them. Therefore electrons trapped in matter can have energies only within certain \textbf{energy bands}; the lowest energy is the \textbf{ground state}, corresponding to electrons tightly bound to the atomic nuclei of the material, and the highest energy is the free electron energy, which is the energy required for an electron to escape entirely from the material.
The energy bands each correspond to a large number of discrete quantum states of the electrons. Most of the states with low energy (closer to the nucleus) are full, up to a particular band called the \textbf{valence band}. Semiconductors and insulators are different from metals because the valence band in the semiconductor materials is very nearly full under usual operating conditions, thus causing more electrons to be available in the \textbf{conduction band}, which is the band immediately above the valence band as shown in figure~\ref{fig:metalInsulator}.
\begin{figure} 
 \centering
 \includegraphics[width=0.6\textwidth]{IMAGES/Isolator-metal}
 \caption{Simplified diagram of the electronic band structure of metals, semiconductors and insulators.
(Image courtesy of P. Kuiper)}
 \label{fig:metalInsulator}
\end{figure}
The ease with which electrons in a semiconductor can be excited from the valence band to the conduction band depends on the band gap between the bands, and it is the size of this energy bandgap that serves as an arbitrary dividing line between semiconductors and insulators.

In the picture of delocalized states, for example in one dimension that is in a wire, for every energy band there is a state with electrons flowing in one direction and one state for the electrons flowing in the other.
For a net current to flow, electrons must occupy more states corresponding to the flow in one direction than they occupy states for the flow in the other direction, and for this they need energy. For a metal this can be a very small energy. In the semiconductor the next higher states lie above the band gap. However, as the temperature of a semiconductor rises above absolute zero, there is more energy in the semiconductor to spend on lattice vibration and on lifting some electrons into an energy states of the conduction band. The current-carrying electrons in the conduction band are known as ``\textbf{free electrons}'', although they are often simply called ``electrons'' if context allows this usage to be clear.

Electrons excited to the conduction band leave behind electron holes, or unoccupied states in the valence band. Both the conduction band electrons and the valence band holes (\textbf{excitons}) contribute to electrical conductivity. The holes themselves don't actually move, but a neighboring electron can move to fill the hole, leaving a hole at the place it has just come from, and in this way the holes appear to move, and the holes behave as if they were actual positively charged particles.

One covalent bond between neighboring atoms in the solid is ten times stronger than the binding of the single electron to the atom, so freeing the electron does not imply destruction of the crystal structure.


%%% effective mass of the ?? quasi particle
In semiconductors, the dielectric constant is generally large, and as a result, screening tends to reduce the Coulomb interaction between electrons and holes. The result is a Mott-Wannier exciton, which has a radius much larger than the lattice spacing. As a result, the effect of the lattice potential can be incorporated into the \textbf{effective masses} of the electron and hole (see table~\ref{table:effectiveMass} for typical values), and because of the lower masses and the screened Coulomb interaction, the binding energy is usually much less than a hydrogen atom, typically the order of $0.1 \; eV$ (Wannier excitons are found in semiconductor crystals with small energy gaps and high dielectric constant).

\begin{table}[ht]
\centering
\begin{tabular}{l|c|c|c} 

\toprule[1pt]  
%heading
\multicolumn{1}{c|}{Semiconductor material} & \multicolumn{1}{c|}{$\epsilon_r$} & \multicolumn{1}{|c|}{$m^*_e$} & \multicolumn{1}{|c}{$m^*_h$} \\
\multicolumn{4}{c}{(Free electron mass $m_e=9.11  \e{-16} \; kg$)} \\
\multicolumn{4}{c}{(Dielectric constant of vaccuum $\epsilon_0 \simeq 8.854 \e{-12} \; A^2 s^4 kg^{-1} m^{-3} $)} \\
\hline                    % inserts single horizontal line

Silicon (Si) $(4.2 \;K)$ & $ 11.7 $ & $1.08 \; m_e$ & $0.56 \; m_e$\\ %[0.7ex]  % inserting body of the table
Germanium (Ge) & $ 16.4 $ & $0.55 \; m_e$ & $0.37 \; m_e$\\ %[0.7ex]  % inserting body of the table
Gallium arsenide (GaAs) & $ 11.1-12.4$ & $0.067 \; m_e$ & $0.45 \; m_e$\\ %[0.7ex]  % inserting body of the table
Indium antimonide (InSb)  & $ 15.9$ (at $77K$) & $0.013 \; m_e$ & $0.6 \; m_e$\\ %[0.7ex]  % inserting body of the table
Zinc oxide (ZnO)  & $ - $ & $0.19 \; m_e$ & $1.21 \; m_e$\\ %[0.7ex]  % inserting body of the table
Zinc selenide (ZnSe)  & $-$ & $0.17 \; m_e$ & $1.44 \; m_e$\\ %[0.7ex]  % inserting body of the table
% \hline   
% \hline  
\bottomrule[1pt]
\end{tabular}
 \caption{Relative dielectric constant ($\epsilon_r$) measured at $290K$ \cite{Willardson1971} and effective mass of charge-carriers for some common semiconductors~\cite{Harrison1989}, $m^*_e$ and $m^*_h$ respectively for the electron and hole effective mass}
\label{table:effectiveMass} 
\end{table} 


%%% difference between bulk and nano-semiconductor

In quantum mechanics, the positions of electrons and holes are described as wavefunctions or probability distributions. The exciton has a certain size, determined by the combined probability distribution functions, and if this size exceeds the particle diameter, quantum confinement occurs.
The size limit for quantum confinement can be approximated from the modified version of the De Broglie wavelength equation considering its effective mass $m^*$:
\begin{equation}
\lambda_B = \frac{h}{p} = \frac{\hbar}{m^* \omega}
\label{eq:deBroglie}
\end{equation}
where $\lambda_B$ is the de Broglie wavelength (the wavelength associated to a particle with momentum $p$), $\hbar$ the reduced Planck constant and $\omega$ the angular frequency of the particle.

Nanocrystals contain much fewer atoms than the bulk, therefore charge screening effects are reduced. The effective mass declines and the de Broglie wavelength can become extremely large, up to several nanometers and make the nanocrystal excited with much less energy than an atom. For cadmium sulfide (CdS) and cadmium telluride (CdTe), these wavelengths are $5.5 \;nm$ and $7.5 \;nm$~\cite{Winter2004}. For particles that are confined within sizes smaller than this wavelength, the excitons ``feel'' restricted. Thus the nanocrystal will display a band gap and associated (optical and electrical) quantum effects inversely proportional to its size. 

\paragraph{Energy spectrum of semiconductor quantum dots}
An additional feature to the transition from bulk crystals to nanocrystals is a radical change in the energy spectrum of the free carriers. It changes when the diameter $d$ of the crystal becomes comparable to the de Broglie wavelength of electrons in the crystal.
Motion in the direction accross the nanocrystal can be assumed bounded, and the energy spectrum in this direction becomes discrete.

Figure~\ref{fig:bulkvsnano} illustrates the different energy spectra of bulk materials, molecules and quantum dots. 
\begin{figure} 
 \centering
 \includegraphics[width=0.8\textwidth]{IMAGES/bulkVSnano_Winter}
 \caption{Possible energy states as a function of the particle size. \newline(\textbf{A}) \textbf{Bulk materials}  have continuous energy bands and absorb energy at a value greater than the band gap. (\textbf{B}) \textbf{Molecular materials} possess discrete energy levels and only absorb energy with certain values. Moreover the band gap is greater than the one of a bulk material as a result of shrinking and splitting of the energy bands. (\textbf{C}) \textbf{Quantum dots} lie between the extremes (A,B). They possess discrete energy bands and absorb energy in discrete intervals. The band gap is between the one of bulk material and the one of a molecular material.
(Image courtesy of J. Winter~\cite{Winter2004})}
 \label{fig:bulkvsnano}
\end{figure}

Bulk semiconductor materials are characterized by bands of allowed potential energy values. For an electron to be excited, it must absorb an energy higher than the band gap. Any value greater than the band gap will produce an excited state.

When we examine a system consisting of only two atoms, the molecular orbitals formed create discrete potential energy states. Electrons will only be excited if the energy absorbed corresponds to specific discrete quantities. Other values are not permitted and will not produce excited states.

Quantum Dots are an intermediate between discrete and continuous energy levels. As the number of atoms in the particle is reduced, the energy bands split and shrink but not to the point of being exactly discrete. Thus electrons in quantum dots may be excited by energies in discrete intervals.

\section{Optical properties}
A lot of applications rely on the optical properties of quantum dots which result from quantum confinement.

The electrical and optical energy of the band gap are equivalent through the following conversion:

\begin{equation}
\Delta E = \hbar \omega = \frac{h c}{\lambda}
\end{equation}
where $\Delta E$ is the band gap difference (figure~\ref{fig:CdS}), $\hbar$ is the reduced Planck's constant, $c$ is the speed of light, $\lambda$ and $\omega$ respectively the wavelength and the angular frequency of the incident light. Thus the energy difference of the band gap is inversely proportional to the wavelength of the incident light. Nanoparticles will only absorb light of wavelengths shorter than the one determined by the band gap value.

For example, CdS (bulk) has a band gap of $2.42 \; eV$, which corresponds to a wavelength of $512 \; nm$. So CdS (bulk) begins to absorb light at $512 \; nm$ and absorbs continuously into the UV (e.g.\ shorter wavelengths/higher energies). As particle size declines, the band gap increases and the absorbance starts at shorter wavelengths (Figure~\ref{fig:CdS}).

\begin{figure} 
 \centering
 \includegraphics[width=0.8\textwidth]{IMAGES/CdS_winter}
 \caption{Band gap energy and optical absorption as a function of the crystal size.\newline The band gap (eV) increases with decreasing nanoparticle size. Band gap is inversely related to the start in absorbance ($\lambda$) through the relationship $E=hc/\lambda$. Therefore smaller particles begin to absorb at shorter wavelengths~\cite{Weller1986} (Image courtesy of J. Winter~\cite{Winter2004}).}
 \label{fig:CdS}
\end{figure}

The influence of particle size on optical properties is not limited to absorbance. Particle fluorescence is also a function of the band gap. After an electron is excited, some of its energy is lost to atomic vibrations, satisfying the second law of thermodynamics. Typically, this energy is converted to heat. When the electron decays into the ground state it will emit light at a longer wavelength because of this energy loss (Figure~\ref{fig:fluo01}). 
\begin{figure} 
 \centering
 \subfloat[Fluorescence and red-shift of $\lambda$ due to energy loss.]{\label{fig:fluo01}\includegraphics[width=0.6\textwidth]{IMAGES/fluo01}}

 \subfloat[Red-shifted emission due to crystal size.]{\label{fig:fluo02}\includegraphics[width=0.5\textwidth]{IMAGES/fluo02}}
 \caption{Fluorescent emission and particle band gap, functions of the crystal size.\newline (a) Photon absorption creates an excited electron. This electron loses some energy to heat; then decays to ground, emitting a photon. The emitted photon has a longer wavelength than the absorbed photon because of the energy lost to heat.\newline (b) As the band gap decreases, the particle will absorb at longer wavelengths. This will produce a red-shift in particle fluorescent emission. (Image courtesy of J. Winter~\cite{Winter2004})}
 \label{fig:fluoresence}
\end{figure}
As the band gap decreases, a smaller amount of energy is dissipated through fluorescent emission to return to the ground state, and the wavelength of emitted light will shift to the red (Figure~\ref{fig:fluo02}). Because the band gap is inversely proportional to nanocrystal size, larger nanocrystals display red-shifted emission. Additionally, the energy lost to heat decreases in a size-dependent manner. Figure~\ref{fig:QDwavelength} presents the emission spectra of quantum dots made from different materials and compare it to the visible wavelengths.

 
\begin{figure}
 \centering
 \includegraphics[width=0.7\textwidth]{IMAGES/QDwavelength}
 \caption{Emission spectra of quantum dots built from different materials.}
 \label{fig:QDwavelength}
\end{figure}

\section{Electronic properties/Manipulation of quantum dots}

\paragraph{Electron-transfer between materials}
Quantum confinement also affects the electrical properties of nanocrystals. Since gap energies are size dependent, electrical properties that depend on this difference will display size dependence as well. One such property is electron transfer. Electrons with no additional energy added prefer to move to lower energy states within a given material. Because there are no energy states in the band gap, the electron will decay until it reaches the lowest state in the conduction band, and then return to the valence band through another mechanism (i.e.\ electron-hole recombination, non-radiative energy loss, etc). However, if the electron encounters a material with lower available
energy states (i.e.\ lower conduction band), it can transfer its electron to that material (Figure~\ref{fig:electronTransfer}). This process is dependent on the band gap. As the band gap increases, excited electrons occupy higher energy levels, and can decay to a greater number of
lower state values. As a result of size-tunable band gaps within the quantum dot, electron transfer can be optimized to many materials.
\begin{SCfigure} 
 \centering
 \includegraphics[width=0.5\textwidth]{IMAGES/ElectronTransfer}
 \caption[short caption for list]{Electron transfer between materials with different band gaps.\newline If an excited electron in one material (A) encounters a second material (B) with a lower band gap energy, it can transfer its electron to that material. (Image courtesy of J. Winter~\cite{Winter2004})}
 \label{fig:electronTransfer}
\end{SCfigure}

\paragraph{Single electron transport in quantum dots: Single-electron tunneling and Coulomb blockade}
Electron transport through a quantum dot is studied by connecting the quantum dot to surrounding reservoirs (Figure~\ref{fig:schemaQD}). The fact that the charge on the electron island is quantized in units of the elementary charge $e$ regulates transport through the quantum dot in the Coulomb blockade regime. Here the transport between the reservoirs and the dot occurs \textbf{via tunnel barriers}, which are thick enough so that the transport is dominated by resonances due to quantum confinement in the dot. This requires a small transmission coefficient through the barriers, and thus the tunnel resistance has to be larger than the quantum resistance $h/e^2$. If the dot is fully decoupled from its environment, it confines a well defined number $N$ of electrons. For weak coupling, deviations due to tunneling through the barriers are small, leading to discrete values in the total electrostatic energy of the dot. This energy can be estimated by $N(N-1)e^2 /(2C)$, where $C$ is the capacitance of the dot. Thus the addition of a single electron requires energy $Ne^2 /C$, which is discretely spaced by the charging energy $e^2 /C$. If this charging energy exceeds the thermal energy $k_B T$, the electrons cannot tunnel on and off the dot by thermal excitations alone, and transport can be blocked, which is referred to as a \textbf{Coulomb blockade}.

The two barriers define the coupling of the channel to its surroundings. The conductance of the double-barrier channel is measured as a function of the gate voltage at different temperatures.


\begin{figure}
\centering
\includegraphics[width=0.4\textwidth]{IMAGES/QDSchematic}
\caption{Schematic of a Single Electron Transistor (SET).\newline Setup for transport measurements on a lateral quantum dot. Because of the small size of the island or quantum dot (QD) in the middle of the two tunnel junctions the capacitance becomes very high and we see coulomb blockade effect.}
\label{fig:schemaQD}
\end{figure}


Following Kouwenhoven and McEuen (1999), figure~\ref{fig:BlockadeA} schematically illustrates an electron island connected to its environment by electrostatic barriers, the so-called source and drain contacts, and a gate to which one can apply a voltage $V_g$ as depicted in figure~\ref{fig:schemaQD}.
\begin{figure}
 \centering
\subfloat[Electron-transport blocked]{\label{fig:BlockadeA}\includegraphics[width=0.5\textwidth]{IMAGES/CoulombBlockade_a}}

 \subfloat[Electron-transport allowed: maximum conductance]{\label{fig:BlockadeB}\includegraphics[width=0.7\textwidth]{IMAGES/CoulombBlockade_b}}

\subfloat[Electron-transport blocked again]{\label{fig:BlockadeC}\includegraphics[width=0.5\textwidth]{IMAGES/CoulombBlockade_c}}

\subfloat[Conductance of the double-barrier channel measured as a function of the gate voltage at different temperatures~\cite{Meir1991}]{\label{fig:conductance}\includegraphics[width=0.5\textwidth]{IMAGES/ConductanceQuantized}}
 \caption{Single-electron transport in a quantum dot. \newline (a)-(c) Schematic picture of the level structures for single-electron transport (courtesy of A; Wacker). The solid lines represent the ionization potentials where the upper equals $\mu_{dot}(N)$, whereas the dashed lines refer to electron affinities, where the lowest one equals $\mu_{dot}(N+1)$. The gate bias increases from (a) to (c)~\cite{reimannManninen}. \newline(d) An example of the first measurements of Coulomb blockade as a function of the gate voltage.}
 \label{fig:Blockade}
\end{figure}

In this example, the level structure of the quantum dot connected to source and drain by tunneling barriers is sketched schematically in Figures.~\ref{fig:Blockade}(a)-(c).
The chemical potential inside the dot, where the discrete quantum states are filled with $N$ electrons [i.e.\, the highest solid line in Figures.~\ref{fig:Blockade}(a)-(c)], equals $\mu_{dot}(N)=E(N)-E(N+1)$, where $E(N)$ is the total groundstate energy (here at zero temperature).

Figure~\ref{fig:conductance} shows the results of the experiment. We can see how the Coulomb blockade affects transport: clear peaks, equidistantly spaced, are separated by regions of zero conductance.

When a bias voltage is applied to the source $s$ and the drain $d$, the electrochemical potentials $\mu_s$ and $\mu_d$ are different, and a transport window $\mu_s -\mu_d=-eV_{ds}$ opens up, where $e$ is the electron charge.
In the linear regime the transport window $-eV_{ds}$ is smaller than the spacing of the quantum states, and only the ground state of the dot can contribute to the conductance.
By changing the voltage on the back gate, $\mu_{dot(N+1)}$ can be aligned with the transport window [Fig.~\ref{fig:BlockadeB}], and electrons can subsequently tunnel on and off the island at this particular gate voltage. This situation corresponds to a conductance maximum, as marked by the label $(b)$ in Fig.~\ref{fig:conductance}.
Otherwise transport is blocked, as a finite energy is needed to overcome the charging energy. This scenario corresponds to zero conductance as marked by the labels (a) and (c) in Fig.~\ref{fig:Blockade}. The mechanism of discrete charging and discharging of the dot leads to Coulomb blockade oscillations in the conductance as a function of gate voltage (as observed, for example, in Fig.~\ref{fig:conductance}): at zero conductance, the number of electrons on the dot is fixed, whereas it is increased by one each time a conductance maximum is crossed.~\cite{reimannManninen}

Spectroscopic information about the charge state and energy levels of the interacting quantum dot electrons can be obtained by analyzing the precise shape of the Coulomb oscillations and the Coulomb staircase. In this way, single electron transport can be used as a spectroscopic tool~\cite{Mizuta2001}.

\section{Quantum dot in a magnetic field}


In many experimental situations with quantum dots, the electrons in quantum dots are manipulated using an external magnetic field. This field is usually created so that the magnetic field vector $\overrightarrow{B}$ is normal to the dot surface. For a typical quantum dot this results in a complicated spectrum of energy levels shown in figure~\ref{fig:additionSpectrum}.

\begin{figure}
\centering
\includegraphics[width=0.6\textwidth]{IMAGES/additionSpectrum}
\caption{Additional energy spectrum as a function of a magnetic field. The magnetic field induces level crossings of single particle eigenstates which appear as cusps on the figure. (Image courtesy of M. Ciorga~\cite{Ciorga2000})}
\label{fig:additionSpectrum}
\end{figure}

The theoretical approach to the quantum dot given in the following chapter will serve as a basis to understand the spectrum of figure~\ref{fig:additionSpectrum}, even if the model used includes many approximations.

% % % 
% % % % \section{Structure, production and manipulation}
% % % % 
% % % % %% PRODUCTION
% % % % %% $http://www.ringsurf.com/online/2017-quantum_dots.html$
% % % % Quantum dots can be fabricated with either a top-down technique or bottom-up technique.  Top down techniques are great for generating a uniform distribution of diameters.  This is crucial if researchers wish to create large arrays of dots that will emit the same wavelength of light.  Unfortunately, top down approaches like lithography are limited by the diffraction limit (that we previous discussed) and cannot create dense networks of quantum dots.  Furthermore, a top down approach inherently implies material damage and many quantum dots produced with these techniques have defects that reduce their effectiveness.
% % % % The most common way to produce a quantum dot is through a bottom up approach.  This can be done either with chemical vapor deposition or molecular beam epitaxy on a highly mismatched substrate.  By layering a desired material that doesn't fit properly with the lattice of the substrate, high strain occurs at the interface and that layer will start nucleating into small quantum dots.  Bottom up approaches are a proven way to create quantum dots in dense arrays that will self-assemble in an orderly manner.  Unfortunately, the uniformity of their size distribution isn't as tight as with a top down approach mainly because it's impossible to control their formation as strictly.
% % % % 
% % % % 
% % % % [Structure and magic numbers] \cite{fewElectronQDExperiment},\cite{reimannManninen} 
% % % % 
% % % % Confinement of motion can be created by:
% % % % \begin{itemize}
% % % %  \item Electrostatic potential: doping, strain, impurities, external electrodes
% % % % \item the presence of the semiconductor surface between different semiconductor materials: i.e.\ in the case of self-assembled QDs
% % % % \item the presence of the semiconductor surface: i.e.\ in the case of a semiconductor nanocrystal
% % % % \item or by a combination of these
% % % % \end{itemize}
% % % % 
% % % % \section{QD properties and phenomenon}
% % % % [define effective mass, which is an effect of the periodic lattice (cf.Rontani 2.1.1), otherwise the motion of electrons in the bulk is that of a free electron gas... to appear later in QDot model]
% % % % 
% % % % [charge quantization]
% % % % 
% % % % energy quantization (discrete change of energy) by tunneling and Coulomb Staircase (continuous change of energy) in~\cite{RontaniThesis} 
% % % % 
% % % % [Coulomb Blockade]
% % % % 
% % % % [Quantum Hall effect]
% % % % 
% % % % [Exp. curve of energy in a magnetic field]: what we know, what we would like to know
