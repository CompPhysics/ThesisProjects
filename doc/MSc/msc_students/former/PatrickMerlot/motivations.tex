\chapter{History and Motivations} %%% Problem statement (5-10 pages), Describes and defines the problem taht the thesis tries to solve
\label{motivations}

In our current understanding of nanotechnology, quantum dots are the most functional and reproducible nanostructures available to researchers. Common shapes include pyramids, cylinders, lens shapes, and spheres.  Different synthesis routes create different kinds of quantum dots. They are very small by nature, the smallest objects that we can synthesize on the nanoscale.  From this fact, they are assimilated to dots, though one quantum dot can be made out of roughly thousands of atoms. All the atoms pool their electrons to "sing with one voice", that is, the electrons are shared and coordinated as if there was only one atomic nucleus setting up an  attraction at the centre. That property enables numerous revolutionary schemes for electronic devices and quantum dots are often referred to as artificial atoms.

%The reason why quantum dots are so important is because they confine electrons in three dimensions, and unlike atoms, we can tune this confinement. 
The total diameter of a quantum dot varies between 2-10 nm depending on its application, corresponding to 10-50 atoms to sizes of hundreds of nanometers that can contain a total of $100-100,000$ atoms within the quantum dot volume~\cite{shubeurRahman2008}, with an equivalent number of electrons. Almost all electrons are tightly bound to the nuclei of these atoms, however the number of ``free electrons'' in the dot can be very small: between one and a few hundreds~\cite{Kouwenhoven1997}. The reason why 'quantum' prefixes the name is because the dots exhibit quantum confinement properties in all three dimensions. This means that electrons within the dot cannot move freely around in any direction leading to quantization as we will show in~\ref{physicsQD}. The only thing that behaves like this in nature is the atom. Commpared to an atom, a quantum dot is at least ten times bigger and above all tunable. This has a lot of important consequences for researchers.  For example they exhibit quantized energy levels like an atom. For a given energy of excitation, for instance, a quantum dot will only emit specific spectra of light.  Quantum theory predicts that if their diameter is decreased there will be a corresponding increase in frequency (e.g.\ in energy) of the emitted light as depicted in figure~\ref{fig:QDwavelength}, and this property is now used in many applications.

\begin{figure}
 \centering
 \includegraphics[width=1\textwidth]{IMAGES/QDwavelength}
 \caption{Emission spectra of quantum dots built from different materials. (Image courtesy of ?????)}
 \label{fig:QDwavelength}
\end{figure}

This element of control over quantum dots' emission properties has huge implications for both electronic devices and medical applications.
Due to their excellent confinement properties not seen in nanowires or quantum wells, quantum dots are extremely efficient at emitting light. They have been the source of some of the world's most powerful lasers produced to date, though the practicality of a quantum dot laser is still being improved.  In medical studies, quantum dots are already in practice as tags that can be inserted into patients. These tags can be seen under most medical scanning technologies and can help pinpoint biological processes as they occur.




\section{History of Quantum Dots}
\label{sec:history}

In the late 50's began the first studies of artificial quantum systems, mostly theoretical due to the lack of funds.
In the 60's, the technique of epitaxial depositions was developed and with it, the possibility to build ultra-clean composite layers of semiconductor material sandwiched between two other layers of another semiconductor. The first optical properties were discovered and the two-dimensional character of the sample has been observed.
At the beginning of the 80's, rapid progress in technology was made with accurate lithography technics (the first quasi one-dimensional quantum wire was done based on these advances)~\cite{Heinzel2003}. 

Collo\"idal quantum dots were discovered in 1981, during the development of materials for the photo-cleavage of water.
Bulk cadmium sulfide (CdS) is known to be an ideal electrode material; however it experiences photocorrosion upon irradiation. It was believed that colloidal particles of cadmium sulfide, coated with a protective agent (i.e.\ $RuO_2$), would be more resistant to corrosion. Therefore, a synthesis method was developed to produce colloidal CdS through aqueous precipitation. The resulting particles displayed unique properties not found in the bulk, including fluorescent emission. These properties were determined to be the result of quantum size effects~\cite{Kalyanasundaram1981}, and were found to be tunable by altering the size of the particle~\cite{Rossetti1983}. This provided a method for selecting excitation and emission wavelengths and particle band gaps. 


In the middle of the 80's, the first quantum dot was developed based on etching techniques [Reed et al.;1986]. As a consequence, a complete quantization of the electron free motion was possible.

At the end of the 80's and in the 90's, the methods evolved: lithography and etching are still in use, but electron or ion-lithography have replaced light-lithography resulting in an  increased precision~\cite{Heinzel2003}.


Lent predicted in 1993 the need for building quantum cellular automata (QCA) cells of 2 nm in order to work at room temperature, where quantum cellular automata refers to any models of quantum computation. %lent2003
 
``Ultimately, temperature effects are the principal problem to be overcome in physically realizing the QCA computing paradigm. The critical energy is the energy difference between the ground state and the first excited state of the array. If this is sufficiently large compared with $k_B T$, the system will be reliably in the ground state after a characteristic relaxation time. Fortunately, this energy difference increases quadratically as the cell dimensions shrink. If the cell size could be made a few \AA{}ngstroms, the energy differences would be comparable to atomic energy levels (i.e.\ several electron-Volts!)''.

''As technology advances to smaller and smaller dimensions on the few-nanometer scale, the temperature of operation will be allowed to increase. Perhaps our envisioned QCA will find its first room temperature implementation in molecular electronics.''\cite{Lent1993}

Interest in the use of quantum dots in biomedicine began in 1998. Coupling the quantum dots directly to biorecognition molecules (e.g.\ antibodies, proteins), the particles could be targeted to particular parts of the cell, producing a fluorescent indicator\cite{Bruchez1998,Chan1998}.


Early this year (Jan. 2009), the four-quantum dot cell dreamed by Lent to build his ``quantum cellular automata'' as a replacement for classical computation using \textit{CMOS} technology has now been achieved with the fabrication and control of a 1 nm-scale assembly of a four coupled silicon dangling bond (In condensed matter physics, a dangling bond occurs when an atom is missing a neighbor to which it would be able to bind. Such dangling bonds are defects that disrupt the flow of electrons and that are able to collect the electrons).
Indeed single-atom quantum dots make possible a new level of control over individual electrons, a development that suddenly brings quantum dot-based devices within reach~\cite{smallestQD}. \textbf{Composed of a single atom of silicon and measuring less than one nanometre in diameter, these are the smallest quantum dots ever created}.
Until now, quantum dots have been useable only at impractically low temperatures, but the new atom-sized quantum dots perform at room temperature. And because they operate at room temperature and exist on the familiar silicon crystals used in today's computers, researchers expect these single atom quantum dots to transform theoretical plans into real devices. 
Figure~(\ref{fig:smallestCoupledQD}) shows how atom-sized quantum dots can be manipulated at room temperature.
The single-atom quantum dots have also demonstrated another advantage: significant control over individual electrons by using very little energy. This low energy control is seen as the key to quantum dot application in entirely new forms of silicon-based electronic devices, such as ultra low power computers.

\begin{SCfigure}
 \caption{Two coupled atomic quantum dots are shown in this room temperature scanning tunneling microscopy image.
In the top frame the dots share one electron. The electron moves freely between the dots just like an electron in a chemical bond within a molecule. The lower frame demonstrates control over that single electron and the potential to do computations in a new way. The electric field from the control charge pushes the electron to prefer staying on only one of the quantum dots.
(Image courtesy of University of Alberta/Prof. Robert A. Wolkow)}
 \label{fig:smallestCoupledQD}

\centering
 \includegraphics[width=0.41\textwidth]{IMAGES/wolkow_coupledQD_090127170710-large}
\end{SCfigure}




\section{Some applications of QD}
\label{sec:applications}

%%% QD for ELECTRONICS: transistors, quantum point contact, quantum computers
Exceptional electrical and optical properties make quantum dots \textbf{attractive components for integration into electronic devices}. One significant asset of quantum dots over traditional optoelectronic materials is that they exist in the solid state. Solids tend to be more compact, easly cooled, and allow for direct charge injection. Additionally, quantum dots can interconvert light and electricity in a tunable manner dependant on crystal size, allowing for easy wavelength selection. This is a significant improvement over silicon-based materials, which require modification of their chemical composition (i.e.\ doping) to alter optical properties~\cite{Weisbuch2000}. Thus researchers have experimented with quantum dots in lasers, LEDS, photovoltaics and also for new generations of transistors, prototypes of spin devices, logic gates with quantum computers as a final aim. Most of these applications are still in early development; however the benefits of quantum dot components are evident, soon leading to a complete revolution of the way of building electronic components at atomic scale.

% Use coupled-QD to build futur quantum computer or quatum cellular automata (QCA)
% 
% [ATTENTION, Qcomputer not to be confused with QCA~\cite{Cole2001}]

%%% QD for nanobiotechnology
Also one of the fastest moving and most exciting interfaces of nanotechnology is the use of (collo\"idal) \textbf{quantum dots in biology}. Again their unique optical properties make them appealing as \textit{in vitro} and \textit{in vivo} fluorophores in a variety of biological investigations, in which traditional fluorescent labels based on organic molecules fall short of providing long-term stability and simultaneous detection of multiple signals~\cite{Medintz2005}. The ability to make quantum dots water soluble and target them to specific biomolecules has led to promising applications in cellular labelling, thus improving diagnostic methods (ex. tracking cancer cells \textit{in vivo} during metastasis\cite{Gao2004,Hoshino2004,Voura12004}) and in developing better drug delivery systems to improve disease therapy~\cite{Kralj2003}. It is even currently studied as neuroelectronic interface for converting optical energy into electrical signal responding to the need for prosthetic devices that can repair or replace nerve function~\cite{Winter2004}.
However there are still many open questions about the toxicity of inorganic QD. The size and charge of most nanoparticles preclude their efficient clearance from the body as intact nanoparticles. Without such clearance or their biodegradation into biologically benign components, toxicity is potentially amplified and radiological imaging is hindered. Some neutral organic coatings prevents adsoption of serum proteins (which otherwise increased the total diameter by $>15nm$ and prevent renal clearance). A final hydrodynamic diameter $<5.5nm$ resulted in rapid and efficient urinary excretion and elimination of quantum dots from the body~\cite{Soo2007}.


%%% QD for Theoretical studies
These achievements, even in their premises, have laid the \textbf{foundations for theoretical investigations} to enable advances in the understanding of the fundamental structure, stability and aqueous assembly of nanoparticle architectures. A physical systems consisting of between $10s-1000s$ of atoms is already too complex to be studied otherwise than using numerical methods for a reliable description. The concept of artificial atom can even be generalized to artificial molecules.

Moreover QDs appears as good tools for studying atomic spectra of many-body systems on a theoretical point of view using computational techniques. Thanks to the possibility to build our own artificial atoms without considering the complexity of the nucleus, it becomes simpler to confront numerical and experimental results. 


\section{Clarifications about computational studies}
\label{whyCompPhys}

The usefulness of numerical simulation is more and more recognized and today it is used in many domains of research and development: mechanics, fluid mechanics, solid state physics, astrophysics, nuclear physics, climatology, quantum mechanics, biology, chemistry... More than being limited to scientific subjects numerical simulation is also used in human sciences (demography, sociology) as well as in finance or economy.

In physics, beside the importance for our basic understanding of quantal systems, the capability to develop and study stable numerical quantum mechanical systems with many degrees of freedom is of great importance, as analytic solutions are rare or impossible to obtain.

Some definitions:

A numerical simulation reproduces the fundamental behaviour of a complex 
system in order to study its properties and predict its evolution. It is based on the implementation of theoretical models, i.e.\ 
it is an adaptation of mathematical models to numerical tools.
Data mining and virtual reality are different from numerical simulation and should not be mistaken with it.

A numerical simulation is performed in several steps:
\paragraph{The model}
 describes the system analysed by listing its essential parameters and by writing the physical laws that rule its behaviour (and link the parameters) as mathematical equations.

\paragraph{The simulation}
 itself is the translation of the equations into computer language, associated with the discretization of the physical domain to make it finite (select a time step, a finite number of points, an acceptable level of accuracy, etc)

\paragraph{Computational techniques}
 The resolution of the equations leads to the determination of the numerical values of all the parameters of the 
system in every points, i.e.\ the state of the system is known.
Various computational techniques can be used to solve the equations, they can be grouped into two main approaches: the deterministic and the statistical (or probabilistic) methods.

In the first approach, an algorithm will solve predictably the equations. For example the object (or the domain) is discretized and the parameters of each element are linked to its neighbours through algebraic equations. It is up to the computer to solve the system that links all the equations. A deterministic method will always produce the same output when given the same input, and the underlying machine will always go through the same sequence of states (which is why it is called "deterministic"). The Hartree-Fock method used in this thesis belongs to this category as well as the Finite element method or the large scale diagonalisation.

The second approach, which groups the “Monte-Carlo” methods, is particularly suited to phenomena characterized by a sequence of steps in which each element of the object can be affected by different “a priori” possible events. From step to step, the evolution of the sample will be determined through a random draw (the name of the method comes from this idea)


\paragraph{Validation of the results:}
 the theoretical model and its translation into computer programming must be validated by comparing with experimental data, or by testing a very simple case for which an analytical solution can be found.

\paragraph{ Numerical simulations} once validated, can explore more cases or unused configurations that were not tested by experiments, sometimes predicting unexpected behaviours, leading to a greater knowledge of the physical behaviour of the system.
Therefore numerical simulation is the third form of study of phenomena, after theory and experiment. 


% 
% \section{Why computational studies?}
% 
% The inestimable value of modelling is more and more important recognized in many disciplines and not only in scientific ones.% "email 6/05/2008"
% In physics, beside the importance for our basic understanding of quantal systems, the capability to handle numerical quantum mechanical systems with many degrees of freedom is of great importance. Analytic solutions are rare or impossible to obtain. Thus to develop and study stable numerical schemes is of utter importance.%[CMA Annual Report 2007]
% 
% \paragraph{What is numerical simulation?}
% 
% Today numerical simulation is used in many domains of research and development: mechanics, fluid mechanics, solid state physics, astrophysics, nuclear physics, climatology, quantum mechanics, biology, chemistry,$\dots$ More than limited to scientific subjects numerical simulation is also used in human sciences: demography, sociology as well as in some areas like finance or economy.
% 
% When thinking of going faster, better and cheaper, numerical simulation offers all the advantages.
% 
% But one should avoid some analogies. While biology uses numerical simulation in order to represent, for example, the structure of proteins in space, genomics uses a lot of computational resources for data mining and not for numerical simulation. Another wrong analogy would be the virtual reality that, even if using numerical simulation, has for first goal to reproduce an appearance, i.e.\ to produce an image which looks real. This is quite far away from the numerical simulation which attempts to reproduce the fondamental behaviour of a complex system.
% 
% So the numerical simulation refers to the process of running a (several) program(s) on machines in order to reproduce a physical phenomenon.
% The scientific numerical simulations are based on the implementation of theoretical models. Therefore they are an adaptation of mathematical models to numerical tools. They are used to study the behaviour and the properties of a system and to predict its evolution.
% 
% \paragraph{Modelling first using equations$\dots$}
% 
% A model is the translation of a phenomenon into mathematical equations. For a long time it has been obtained from careful observations. Since the $17^{th}$ century, models were more and more inspired from lab experiments. It happens also that the model is deduced from theory itself built without experiments, so thanks to a pure intellectual approach (ex.: The theory of relativity from A. Einstein in 1915 couldn't be experimented before 1919).
% 
% Modelling of a phenomenon consists in taking into account the fundamental principles, such as for example the energy conservation, and to assess the parameters essential to describe it as simply and as faithfully as possible 
% Most of the time, those quantities are not independant but linked to each other through equations, which are the mathematical translation of the physical laws that rule the behaviour of the object.
% 
% \paragraph{$\dots$then simulating using computer programs}
% 
% The step is short from equations to the simulation code (i.e.\ discretization of the equations). This operation translate the equations into program language, the only one understood by the machines. Simulating the state of an object is done by computing the numerical values of its parameters in every points. Quite often, there is an infinite number of points to compute and this goal is impossible to achieve. So for feasability reasons, it is admitted to limit the computation to a finite number of points and the number of necessary operations becomes reasonable for a machine. The effective number of points computed will depend of the power of this machine. The discretization of the physical domain consists precisely in this restriction from infinity to a finite domain.
% 
% \paragraph{Computational techniques: from equations to computer programs}
% 
% It exists two main approaches for solving numerically the mathematical equations of a model: the deterministic and the statistical (or probabilistic) methods .
% In the first one, an algorithm will solves predictably the equations. For example the object (or the domain) can be discretized and the parameters of each element are linked to its neighbours through algebraic equations. It is up to the computer to solve the system that links all the equations. In any deterministic method, given a particular input, it will always produce the same output, and the underlying machine will always pass through the same sequence of states.
% 
% Several deterministic methods exist and Finite element method, large scale diagonalisation or the Hartree-Fock method belong to this category, as well as any mathematical formula.
% 
% The second method, also called ``Monte-Carlo'', is particularly suited to phenomena characterized by a sequence of steps in which each element of the object can be affected by different ``a priori'' possible events. Fron step to step, the evolution of the sample will be determined through a random draw (the name of the method comes from this idea).
% 
% Therefore the tools of the numerical simulation are computer programs run on machines: those software (or codes) are the translation, through numerical algorithms, of mathematical formula that rule the physical model under study.
% 
% The numerical simulation materializes when one visualizes the initial phenomenon on the screen of a computer when it is possible. A critical analysis of the results, checking the validity of the theoretical model chosen, checking with experimental data must be part of the approach.
% This analysis opens onto improvements of the physical models, of their parameters and of computer programs used in the simulations.
% 
% \paragraph{Simulation may lead to discoveries}
% 
% Of course experiment feeds the simulation. Inversely, the exploration of a multitude of solutions now possible with numerical simulations allows to observe or to predict some unexpected behaviour, which sometimes suggests new experiments and consequently leads to a greater knowledge of the physics.
% Therefore numerical simulation is the third form of study of phenomenon, after the theory and the experiment.
% 
% The model and the simulation that goes with it allows to anticipate the futur of a system or the behaviour it may have in a configuration in which it has never been before. To predict completely new situations is one essential interest of the numerical simulation.
% 
% \paragraph{Limits of the simulation}
% 
% There is three main limitations to numerical simulations.
% \begin{itemize}
%  \item
% First some phenomena are still not completely understood, then hardly translatable into equations, which is the only way to ``converse'' with the machine.
% \item
% Also some models requires computational resources still unavailable at the present time.
% \item 
% Finally it exist unfortunally a theoretical limit. As the number of operations required for solving a model is increasing exponentially as a function of the degree of precision, some mathematical models can not be solved by a computer in a reaonnable time.
% \end{itemize}
% 
% 
% As an example, a team of researcher at Lehigh and Princeton University modelled semiconductor quantum dots in order to establish their high temperature interdiffusion effects in three-dimensions. With such a model, researchers are able to predict (see figure~\ref{fig:thermalInterdiffusion}) the process temperature that can be tolerated by quantum dots during epitaxial growth and post growth processing to increase the manufacturing productivity~\cite{ooi2005}.
% \begin{figure}
%  \centering
%  \includegraphics[width=0.7\textwidth]{IMAGES/ThermalInterdiffusion}
%  % ThermalInterdiffusion.jpg: 623x168 pixel, 500dpi, 3.16x0.85 cm, bb=0 0 90 24
%  \caption{Model of semiconductor quantum-dot structures displaying quantum-dot interdiffusion (or quantum-dot intermixing) at different temperatures. (Image courtesy of University of Lehigh/Prof. Boon S. Ooi)}
%  \label{fig:thermalInterdiffusion}
% \end{figure}
