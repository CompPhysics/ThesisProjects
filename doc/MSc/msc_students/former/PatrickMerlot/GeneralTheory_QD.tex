\chapter{Modelling of Quantum Dots} %%% Generic model (5 pages) a description of how the solution is related to the big model
\label{model}

Quantum systems are governed by the Schr\"odinger equation ($4^{th}$ postulate of quantum mechanics):
\begin{equation}
H |\psi(t) \rangle = i \hbar \frac{d}{dt} |\psi(t) \rangle
\end{equation}
where $H$ is the quantum Hamilton operator (or Hamiltonian) and $|\psi(t) \rangle$ the state vector of the system.

The solutions to the stationary form of this equation determine many physical properties of the system at hand, such as the ground state energy of the system which is the ultimate result of our simulations. Indeed as expressed in section~\ref{whyCompPhys} we need a way to validate our model and we choose the ground state energy as physical quantity that can be compared with experiments and other numerical simulations.

Solving the Schr\"odinger equation in the Hamiltonian formalism obviously requires a definition of this Hamiltonian which translates as well as possible our knowledge of the system into equations. We must identify the different forces/fields applied to the system in order to include their respective ``potentials'' into the Hamiltonian.

In the case of a quantum dot, the Hamiltonian is basically characterized by the different forces applied to its constituants. This could be done by summing over all the interactions between electrons and nuclei that constitute the quantum dot. However, if the final objective is to perform fast predictions about the system, the complexity of such a model may quickly reach the limits of the computational resources.

Therefore we prefer a simpler model in which the system is limited to the free charge-carriers by modelling a \textbf{confining potential} that traps them into the dot as well as an \textbf{interaction potential} that characterizes the repulsion between those electrons.

This chapter discusses possible models for those potentials. The derivation of a Hamiltonian is then given for electrons that repeal each other by a Coulomb interaction and that are trapped in a parabolic potential, and even more confined by applying or not an external magnetic field.
We show finally that rescaling the problem with proper length and energy units leads to a simple form of the Hamiltonian even when applying an external magnetic field to the quantum dot.


\section{Theoretical approximation of the quantum dot Hamiltonian}

%%% introduce Zeeman effect (detailled in the document "Symmetry breaking and...``)

\label{sec:potential}
\paragraph{Two-body interaction potential}
The interaction potential between two electrons is usually approximated proportional to the Coulomb repulsion in free space $V(\vec{r_i},\vec{r_j})=1/r_{ij}$. Other studies have investigated different forms of potential. For example Johnson and Payne \cite{johnsonPayne} assumed the interaction potential $V(\vec{r_i},\vec{r_j})$ between particles $i$ and $j$ moving in the confining potential to saturate at small particle separation and to decrease quadratically with increasing separation.
More recently, in order to investigate spin relaxation in quantum dots, Chaney and Maksym in~\cite{Chaney2007} built a model where the electron-electron interactions were designed to follow experimental data.

For sake of simplicity and since it is still in use in most studies of quantum dots, we will stick to the approximation of the interaction potential proportional to the Coulomb repulsion.

%model of QD in this book~\cite{Joyce2005} (including Auger effect, but complicated form of the interaction potential).

\paragraph{The confining potential} Defining the second potential that confines these electrons is a more difficult issue when modelling a quantum dot. Some numerical~\cite{KumarA1990,Macucci1993,Macucci1997,Stopa1996} and experimental~\cite{Kohn1961,Heitmann1993,Heitmann1995} studies have shown that for a small number of trapped electrons, the harmonic oscillator potential is a good approximation, at least to first approximation.
In~\cite{johnsonPayne}, the bare (i.e.\ unscreened) confining potential $V(\vec{r_i})$ for the $i^{th}$ particle is also modelled to be parabolic (i.e.\ the harmonic oscillator potential). It has been shown theoretically that for electrons contained in a parabolic potential there is a strong absorption of far-infrared light at the frequency corresponding to the bare parabola~\cite{Brey1989,Peeters1990,Yip1991,Li1991}. This theoretical prediction is consistent with some experimental measurements on quantum dots~\cite{Sikorski1989}. Further evidence that the bare potential in many quantum-dot samples is close to parabolic is provided by simple electrostatic models~\cite{Dempsey1990}.

%[\textit{\textcolor{red}{Waltersson discuss the limit  of this choice: ok only for small Nb of electrons $Ne<20$,etc}}]

\paragraph{Relations between confining potential and electron interactions} Other studies tested different spherically symmetric confining potentials with different profiles (``soft'' and ``hard'') on electrons in coupled QDs~\cite{Kwasniowski2008}, and observed the resulting electron interactions. It shows very different behaviours of the electron interactions between the soft (Gaussian) and the hard (rectangular-like) confining potential. This means that the model of the confining potential has a strong influence on the electron-electron interactions, and that it depends itself on the type and shape of the QD under study.

\paragraph{Motivation for the model chosen} Since we are focusing on the limits of the Hartree-Fock method %%%%applied to an idealized dot model
with respect to other techniques rather than an exhaustive study of different types of quantum dots, we choose in the rest of the thesis to model the single quantum dot by a definite number of electron $N_e$, trapped by a pure isotropic harmonic oscillator potential and repealing each other with a two-body Coulomb interaction. Only closed shell systems are studied, meaning that the number of electrons present in the quantum dot are filling all single particle states until the Fermi level. This simplifies greatly the problem since all combinations of single particle states are reduced to one Slater determinant as detailled in~\ref{SlaterDet}.


\section{General form of $\hat{H}$ with explicit physical interactions}
\label{sec:HamiltonianScaling}
In this section we derive the Hamiltonian of the quantum dot model with and without external magnetic field in order to show that the interaction with the external magnetic field will basically result in a modified harmonic oscillator frequency and a shift of the energy proportional to the strength of the field.

\subsection{Electrons trapped in an harmonic oscillator potential}
We consider a system of electrons confined in a pure isotropic harmonic oscillator potential $V(\vec{r})=m^* \omega_0^2 r^2/2$, where $m^*$ is the effective mass of the electrons in the host semiconductor (as defined in section~\ref{semiconductors}), $\omega_0$ is the oscillator frequency of the confining potential, and $\vec{r}=(x,y,z)$ denotes the position of the particle.

The Hamiltonian of a single particle trapped in this harmonic oscillator potential simply reads
\begin{equation}
  \hat{H}= \frac{\textbf{p}^2}{2m^*}  + \frac{1}{2} m^* \omega_0^2 \norm{\textbf{r}}^2
\end{equation}
where $\textbf{p}$ is the canonical momentum of the particle.

When considering several particles trapped in the same quantum dot, the Coulomb repulsion between those electrons has to be added to the single particle Hamiltonian which gives
\begin{equation}
\hat{H}=\sum_{i=1}^{N_e} \left( \frac{\mathbf{p_i}^2}{2m^*}+ \frac{1}{2} m^* \omega_0^2 \norm{\mathbf{r_i}}^2 \right) + \frac{e^2}{4 \pi \epsilon_0 \epsilon_r} \sum_{i<j} \frac{1}{\norm{\mathbf{r_i}-\mathbf{r_j}}},
\end{equation}
where $N_e$ is the number of electrons, $-e \;  (e>0)$ is the charge of the electron, $\epsilon_0$ and $\epsilon_r$ are respectively the free space permitivity and the relative permitivity of the host material (also called dielectric constant), and the index $i$ labels the electrons.

\subsection{Electrons trapped in an harmonic oscillator potential in the presence of an external magnetic field} 
We assume that the magnetic field $\overrightarrow{B}$ is static and along the $z$ axis.
At first we ignore the spin-dependent terms. The Hamiltonian of these electrons in a magnetic field now reads ~\cite{Bransden2003}
\begin{align}
  \hat{H}&=&\sum_{i=1}^{N_e}\left(  \frac{(\mathbf{p_i}+e\mathbf{A})^2}{2m^*}  + \frac{1}{2} m^* \omega_0^2 \norm{\mathbf{r_i}}^2  \right) + \frac{e^2}{4 \pi \epsilon_0 \epsilon_r} \sum_{i<j}\frac{1}{\norm{\mathbf{r_i}-\mathbf{r_j}}}, \\
&=&\sum_{i=1}^{N_e}\left(  \frac{\mathbf{p_i}^2}{2m^*} + \frac{e}{2m^*}(\mathbf{A}\cdot \mathbf{p_i}+\mathbf{p_i}\cdot \mathbf{A}) + \frac{e^2}{2m^*}\mathbf{A}^2  + \frac{1}{2} m^* \omega_0^2 \norm{\mathbf{r_i}}^2  \right) \\
& &+ \frac{e^2}{4 \pi \epsilon_0 \epsilon_r} \sum_{i<j}\frac{1}{\norm{\mathbf{r_i}-\mathbf{r_j}}},
\end{align}
where $\mathbf{A}$ is the vector potential defined by $\mathbf{B}=\nabla \times \mathbf{A}$.

In coordinate space, $\mathbf{p_i}$ is the operator $-i \hbar \nabla_i$ and by applying the Hamiltonian on the total wave function $\Psi(\mathbf{r})$ in the Schr\"odinger equation, we obtain the following operator acting on $\Psi(\mathbf{r})$
\begin{align}
\mathbf{A}\cdot \mathbf{p_i}+\mathbf{p_i}\cdot \mathbf{A} &= - i \hbar \left( \mathbf{A}\cdot \nabla_i+\nabla_i\cdot \mathbf{A} \right) \Psi \\
&=  - i \hbar \left( \mathbf{A}\cdot  ( \nabla_i \Psi)+\nabla_i\cdot (\mathbf{A} \Psi) \right) 
\end{align}

We note that if we use the product rule and the Coulomb gauge $\nabla \cdot \mathbf{A} = 0$ (by choosing the vector potential as $\mathbf{A} = \frac{1}{2} \mathbf{B} \times \mathbf{r}$), $\mathbf{p_i}$ and $\nabla_i$ commute and we obtain
\begin{equation}
 \nabla_i \cdot (\mathbf{A}\Psi) = \mathbf{A} \cdot (\nabla_i\Psi) + (\underbrace{\nabla_i \cdot \mathbf{A})}_0 \Psi=\mathbf{A} \cdot (\nabla_i\Psi) 
\end{equation}

This leads us to the following Hamiltonian:
\begin{align}
\label{eq:Hamiltonian2}
  \hat{H}&=&\sum_{i=1}^{N_e}\left(  -\frac{ \hbar^2}{2m^*} \nabla_i^2- i \hbar \frac{e}{m^*} \mathbf{A}\cdot \nabla_i + \frac{e^2}{2m^*}\mathbf{A}^2  + \frac{1}{2} m^* \omega_0^2 \norm{\mathbf{r_i}}^2  \right) \\
& &+ \frac{e^2}{4 \pi \epsilon_0 \epsilon_r} \sum_{i<j}\frac{1}{\norm{\mathbf{r_i}-\mathbf{r_j}}},
\end{align}

The linear term in $\mathbf{A}$ becomes, in terms of $\mathbf{B}$:
\begin{align}
\label{eq:linearTermA}
\frac{-i \hbar e}{m^*} \mathbf{A} \cdot \nabla_i &= -\frac{i \hbar e}{2m^*} (\mathbf{B} \times \mathbf{r_i}) \cdot \nabla_i \\
&= \frac{-i \hbar e}{2m^*} \mathbf{B} \cdot( \mathbf{r_i} \times \nabla_i)  \\
&= \frac{ e}{2m^*} \mathbf{B} \cdot \mathbf{L} 
\end{align}
where $\mathbf{L}=-i \hbar (\mathbf{r_i} \times \nabla_i)$ is the orbital angular momentum operator of the electron $i$.

If we assume that the electrons are confined in the $xy$-plane, the quadratic term in $\mathbf{A}$ appearing in \ref{eq:Hamiltonian2} can be written under the form
\begin{align}
\frac{e^2}{2m^*} \mathbf{A}^2 &= \frac{e^2}{8m^*} (\mathbf{B} \times \mathbf{r})^2 \\
&= \frac{e^2}{8m^*} B^2 r_i^2
\end{align}

Until this point we have neglected the intrinsic magnetic moment of the electrons which is due to the electron spin in the host material. We will now add its effect to the Hamiltonian. This intrinsic magnetic moment is given by $\mathcal{M}_s=-g^*_s (e \mathbf{S})/(2m^*)$, where $\mathbf{S}$ is the spin operator of the electron and $g^*_s$ its effective spin gyromagnetic ratio (or effective \textit{g-factor} in the host material).% Dirac's relativistic theory predicts for $g_s$, the value $g_s=2$ which is in very good agreement with experiment~\cite{Bransden2003}.
We see that the spin magnetic moment $\mathcal{M}_s$ gives rise to an additional interaction energy~\cite{Bransden2003}, linear in the magnetic field,
\begin{equation}
\label{eq:spinHs}
  \hat{H_s}= - \mathcal{M}_s \cdot \mathbf{B} = g^*_s \frac{e }{2 m^*} B \hat{S_z}= g^*_s \frac{\omega_c}{2} \hat{S_z}
\end{equation}
where $\omega_c=e B/m^*$ is known as the cyclotron frequency.

The final Hamiltonian reads
\begin{align}
  \hat{H}&=\sum_{i=1}^{N_e} \bigg(  \frac{- \hbar^2}{2m^*} \nabla_i^2 + \overbrace{\frac{1}{2} m^* \omega_0^2 \norm{\mathbf{r_i}}^2}^{\begin{smallmatrix}
  \text{Harmonic ocscillator} \\
  \text{potential}
\end{smallmatrix}} \bigg) + \overbrace{\frac{e^2}{4 \pi \epsilon_0 \epsilon_r} \sum_{i<j}\frac{1}{\abs{\mathbf{r_i}-\mathbf{r_j}}}}^{\begin{smallmatrix}
  \text{Coulomb} \\
  \text{interactions}
\end{smallmatrix}}  \nonumber \\
&+  \underbrace{\sum_{i=1}^{N_e} \left( \frac{1}{2} m^* \left( \frac{\omega_c}{2} \right)^2 \norm{\mathbf{r_i}}^2 + \frac{1}{2}  \omega_c \hat{L}_z^{(i)}+ \frac{1  }{2} g_s^*  \omega_c \hat{S}_z^{(i)}\right)}_{\begin{smallmatrix}
  \text{single particle interactions} \\
  \text{with the magnetic field}
\end{smallmatrix}},
\end{align}

\subsection{Scaling the problem: Dimensionless form of $\hat{H}$ }
\label{sec:scaling}
%also look at what Simen did:  \url{http://folk.uio.no/simenkva/openfci/html/qdot_8cc.html}, he includes B field and this change lambda to lambda*.
In order to simplify the computation, the Hamiltonian can be rewritten on dimensionless form.
For this purpose, we introduce the following constants:
\begin{itemize}
 \item the oscillator frequency $\omega = \omega_0\sqrt{1+\omega_c^2/ (4\omega_0^2)}$,
 \item a new energy unit $\hbar \omega$,
\item 	a new length unit, the oscillator length defined by $l=\sqrt{\hbar /(m^* \omega)}$, also called the characteristic length unit.
\end{itemize}

We rewrite the Hamiltonian in dimensionless units using:\\
$$\mathbf{r} \longrightarrow \frac{\mathbf{r}}{l}, \quad \nabla \longrightarrow l \;\nabla \quad \text{and} \quad \hat{L}_z \longrightarrow \hat{L}_z$$ 


It leads to the following Hamiltonian:
\begin{align}
\hat{H}&=\sum_{i=1}^{N_e} \left(  -\frac{1}{2} \nabla_i^2 + \frac{1}{2} r_i^2 \right)  + \overbrace{\frac{e^2}{4 \pi \epsilon_0 \epsilon_r} \frac{1}{\hbar \omega l}}^{\begin{smallmatrix}
  \text{Dimensionless} \\
 \text{confinement } \\
  \text{strength ($\lambda$)}
\end{smallmatrix}}
\sum_{i<j}\frac{1}{r_{ij}}  \nonumber \\
&+  \sum_{i=1}^{N_e} \left(  \frac{1}{2}  \frac{\omega_c}{\hbar \omega} \hat{L}_z^{(i)}+ \frac{1  }{2} g_s^* \frac{\omega_c}{\hbar \omega} \hat{S}_z^{(i)}\right),
\end{align}
Lengths are now measured in units of $l=\sqrt{\hbar/(m^*\omega)}$, and energies in units of $\hbar \omega$.
 
A new dimensionless parameter $\lambda=l / a_0^*$ (where $a_0^*= 4 \pi \epsilon_0 \epsilon_r \hbar^2 / (e^2 m^*)$ is the effective Bohr radius) describes the strength of the electron-electron interaction.
Large $\lambda$ implies strong interaction and/or large quantum dot~\cite{Tavernier2003}. Since both $\hat{L_z}$ and $\hat{S_z}$ commute with the Hamiltonian we can perform the calculations separately in subspaces of given quantum numbers $L_z$ and $S_z$.
Figure \ref{fig:valuesLambda} displays values of the different parameters as a function of the magnetic field strength for a particular type of semiconductor: Gallium arsenide (GaAs) with know characteristics given in table~\ref{table:effectiveMass}.

\figtikzb{IMAGES/omegaB.tex}{IMAGES/lB.tex}{IMAGES/homegaB.tex}{IMAGES/lambdaB.tex}{Typical values for the oscillator frequency $\omega$, the oscillator length $l$, the energy unit $\hbar\omega$ and the dimensionless confinement strength $\lambda$ as a function of the magnetic field strength in GaAs semiconductors assuming: $\hbar \omega_0=5\e{-3}eV$~\cite{fewElectronQDExperiment}, $\epsilon_r\simeq 12$ and $m^*=0.067\;m_e$}{fig:valuesLambda}


The simplified dimensionless Hamiltonian becomes
\begin{equation}
\label{eq:dimlessH}
  \hat{H}=\sum_{i=1}^{N_e} \left[  -\frac{1}{2} \nabla_i^2 + \frac{1}{2} r_i^2  \right]+ \lambda \sum_{i<j}\frac{1}{r_{ij}} +  \sum_{i=1}^{N_e} \left(  \frac{1}{2}  \frac{\omega_c}{\hbar \omega} L_z^{(i)}+ \frac{1  }{2} g_s^* \frac{\omega_c}{\hbar \omega} S_z^{(i)}\right),
\end{equation}

The last sum which is proportional to the magnetic field involves only the quantum numbers $L_z$ and $S_z$ and not the operators themselves~\cite{Tavernier2003}. Therefore these terms can be put aside during the resolution, the squizzing effect of the magnetic field being included simply in the parameter $\lambda$. The contribution of these terms will be added when the other part has been solved. This brings us to the simple and general form of the Hamiltonian:
\begin{equation}
\label{eq:lambdaSimp}
\hat{H}=\sum_{i=1}^{N_e} \left(  -\frac{1}{2} \nabla_i^2 + \frac{1}{2} r_i^2  \right)+ \lambda \sum_{i<j}\frac{1}{r_{ij}}.
\end{equation}

In the next chapters, we will look for approximations of the ground state of the quantum dot by solving the Schr\"odinger equation using this definition $\hat{H}$.


% 
% \section{More about modelling QDs}
% What about dynamics of a quantum dot? What would be the model of a time-dependent QD?
% 
% What about coupled QD? How to model the tunneling effect btw the dots?
% 
% doc from V. Popsueva (structure of lateral 2-electron QD molecules in EM fields)- paper from Morten (10/08/08)
% 

