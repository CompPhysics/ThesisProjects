\documentclass[../main.tex]{subfiles}
 
\begin{document}
%\thispagestyle{empty}
I first discovered physics during high school, thanks to my brother Alexander Fleischer. He knew how much I liked mathematics, so he recommended a course in physics to me. I ended up liking it even more than mathematics, and as a result, after high school I applied to the physics, meteorology and astronomy Bachelor program at the University of Oslo.

The very first semester I was introduced to a field I liked just as much as physics, namely programming. Unfortunately, my Bachelor did not involve much programming after the first year. During the third year I had the option of taking an introductory course in computational physics, but I had several other physics courses I wanted to take, and I prioritized these over computational physics.

When it was time to apply for a Masters program, I was unsure of what to pick. I then remembered how fun all the programming had been during the first year, and decided to give the combination of physics and programming a try, so I applied to the computational physics Masters program. The first semester I took the introductory computational physics course, and knew immediately that I had picked the right Masters program for me.

I would like to thank my supervisor Morten Hjorth-Jensen for all the help and motivation he gave me during the past two years. I would also like to thank my brother Alexander for introducing me to physics in the first place, and for the helpful discussions we had about this thesis. 

Additionally, I would like to thank Håkon V. Treider for the fun times we have had sharing an office for the better part of two years, and for the help he has given me with general programming related issues. Finally, I would like to thank the various other people I shared an office with over the past two years, and the rest of the people at the computational physics research group, for making it a great place to be.

{\raggedleft\vfill\itshape\Longstack[l]{%
  Christian Fleischer\\
  Oslo, May 2017
}\par
}

\end{document}