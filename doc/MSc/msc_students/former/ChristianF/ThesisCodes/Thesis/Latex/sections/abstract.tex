\documentclass[../main.tex]{subfiles}
 
\begin{document}
\noindent
We approximate single particle wave functions for some given potential by expanding the solutions from diagonalizing the single particle problem in a harmonic oscillator basis. These single particle wave function approximations allow us to perform variational Monte Carlo (VMC) simulations without having explicit expressions for the single particle wave functions. We use the VMC simulations to analyze the ground states of quantum dots trapped in various confinement potentials, namely single and double harmonic oscillator potentials and single finite square well potentials. A point of interest is to see how many harmonic oscillator basis functions are needed for high quality approximations to the single particle wave functions for the different potentials. This is of interest because the number of basis functions needed is an important constraint of the method when it comes to computational efficiency. To verify the method used, we compare our ground state energy results with other \emph{ab initio} many-body methods, such as the Similarity Renormalization Group method, the Coupled Cluster method, and the Full Configuration Interaction method, as well as with other VMC and DMC (Diffusion Monte Carlo) results.

\end{document}