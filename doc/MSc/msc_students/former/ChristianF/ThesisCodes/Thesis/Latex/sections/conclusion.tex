\documentclass[../main.tex]{subfiles}
 
\begin{document}

The focus of this thesis has been to develop a general quantum variational Monte Carlo (VMC) solver which can do simulations of quantum dots trapped in various types of confinement potentials. The goal was to be able to use harmonic oscillator single particle wave functions to approximate the single particle wave functions of more complicated potentials. This was done by diagonalizing the single particle problem in the potential of interest, and then expanding the solutions in terms of harmonic oscillator basis functions. The diagonalization solutions and the harmonic oscillator basis functions were used to find the overlap coefficients between them. These overlap coefficient were then used in the Monte Carlo simulation together with the basis functions to approximate the single particle wave functions of particles with specific quantum numbers and with positions that could change from one Monte Carlo cycle to the next. 

The approximations to the single particle wave functions were calculated as the inner product of the overlap coefficients and the basis functions, so a loop over basis functions was needed. The quality of the approximations depended on the amount of basis functions we used, and consequently the computational cost of calculating accurate approximations could be significantly greater than using explicit expressions for the single particle wave functions. As a result, it was important that the program was optimized well, so that computational efficiency was maintained when approximations needed many basis functions. Since the variational Monte Carlo solver was similar to the solver developed in Ref. \cite{Jorgen}, we implemented in our own solver, most of the optimizations done to that solver. For the reference system we used, the optimizations yielded a speed up factor of $31.246$, which corresponds to the total run-time being reduced to about $3.2\%$ of the pre-optimization run-time.

For this thesis we have implemented three different types of potentials. The first is the single harmonic oscillator well. Our method is not useful for studying systems with this confinement potential, since the basis functions we use are also for a single harmonic oscillator, making the approximations to the single particle wave functions redundant. However, for exactly that reason, this potential was necessary for testing our implementation since we for this potential should be able to exactly reproduce the results we get when using explicit expressions for the single particle wave functions instead. The other two potentials we have implemented are the double harmonic oscillator well and the single finite square well. The somewhat similar, but still significantly different shape of these potentials compared to the single harmonic oscillator well, allowed us to see how good approximations we could make, and how many basis functions we needed to do it.

For the single harmonic oscillator well we found that our method was not able to exactly reproduce the results we get when using explicit expressions for the single particle wave functions, but it was still close. The reason for the discrepancy seems to be that there is supposed to be a dependence on the variational parameter $\alpha$ in the overlap coefficients, which is not included when we make the coefficients. This dependence should be included by running a Hartree-Fock calculation on the coefficients.

When it came to the double harmonic oscillator well we were able to make good approximations with a reasonable number of basis functions for systems of few particles. However, as the number of particles rises, so does the amount of basis functions needed, so the computational cost associated with simulating a high number of particles is an important constraint of the method. We also saw that for small harmonic oscillator frequencies $\omega$, the double harmonic oscillator more resembled the single harmonic oscillator, and consequently fewer basis functions were needed. For few particles we also saw that the one-body density was similar to what we would get from two independent single harmonic oscillator wells, but when we had enough particles to fill two energy levels, the shape of the one-body density became more interesting.

With the two harmonic oscillator potentials we had explicit expressions for the single particle wave functions, which allowed us to benchmark our results. However, it also allowed us to properly optimize the variational parameters. As mentioned above, our overlap coefficients were missing an $\alpha$ dependency. This prevented the approximations to the single particle wave functions from being used to optimize $\alpha$. This was not a problem for the harmonic oscillator potentials, since we could use the explicit expressions instead. For the finite square well however, we did not have explicit expressions for the single particle wave functions. Instead we kept $\alpha$ constant at $\alpha = 1$ and only varied $\beta$. As a result, the upper bound estimates to the ground state energy should be less accurate for this potential. We also did not have anything to benchmark our results with, so it is hard to say how close the estimates are to the ground state. However, the energy estimates did converge as we increased the number of basis functions, allowing us to get an idea of how many basis functions are needed for the finite square well. The one-body density also revealed that for a system with $6$ particles, the particles are most likely to be in the corners of the well, which makes sense since the strength of the potential is equal everywhere inside the well, and the Coulomb repulsion pushes the particles away from each other. We also saw that whether or not the harmonic oscillator basis was a good fit for the finite square well potential, depended on the harmonic oscillator frequency $\omega$ of the basis functions.

\section*{Further Work}
As of now, the variational Monte Carlo solver we have developed works well, and can easily be extended to include more types of potentials. However, it has one big flaw which is the missing $\alpha$ dependence in the overlap coefficients. Therefore the next step would be to implement a Hartree-Fock calculation which can modify the coefficients to include the $\alpha$ dependency. 

Another interesting extension would be to make it possible to do diffusion Monte Carlo simulations as well. Diffusion Monte Carlo can be used to make more precise estimates to the ground state energy, while still keeping the flexibility of the solver when it comes to different types of potentials.

There is also the possibility of adding other types of basis functions which may be more suited for certain types of potentials than the harmonic oscillator basis functions are. For example, hydrogen-like basis functions could be used when simulating finite square well systems where only some of the particles are bound while the rest are unbound.

Finally, more optimizations can be added to the solver, which would allow the solver to stay computationally efficient while using more basis functions. This would make it easier and faster to simulate systems with higher numbers of particles.

\end{document}