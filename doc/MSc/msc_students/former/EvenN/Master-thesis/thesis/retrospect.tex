\chapter{Retrospect and Future Work} \label{sec:conclusion}
In this chapter, we will make an attempt to compress the relatively comprehensive discussion in the previous chapter down to a more tangible conclusion. Thereafter, we address some possible extensions of our work, how our contributions can be used to solve the many-body quantum puzzle and why it is important. 

We have seen that the plain restricted Boltzmann machine (RBM) is capable of producing reasonable ground state energy estimates, and when we add more intuition in the form of Jastrow factors of different complexities the energy drops further towards the diffusion Monte Carlo (DMC) energy. Most notably, an RBM with the Padé-Jastrow  factor (RBM+PJ) provides ground state energies and statistical errors lower than the VMC energy for the smallest dots. This indicates that the method can provide a wave function closer to the exact one than standard VMC. However, for larger quantum dots, the RBM+PJ gives a slightly higher energy than the VMC, but we suspect this is a consequence of a large number of variational parameters as we consequently set the number of hidden units, $H$, equal to the number of electrons in the dot, $N$. In machine learning terms, we use a too complex model for our problem. We decided to do this because \citet{nordhagen_computational_2018} found $H=N$ to be optimal for small quantum dots, but it could be different for larger dots. \citet{carleo_solving_2017} operates with a hidden variable density $\alpha=H/F$ with $F$ as the degrees of freedom (number of visible units), which they set to an integer number and thus end up with more variational parameters than we do. A conclusion is that the number of hidden nodes might not be optimal, and thus some more investigation is needed. We also observe that all the methods more or less give the same ratio between kinetic and potential energy for all system sizes and all frequencies, which means the electron configuration is fundamentally different for the different methods.

Throughout the results, we had a thorough discussion of the electron density provided by the various methods, which revealed some significant differences between the methods that cannot be seen just from the ground state energy. The most notable difference is found for the one-body density produced using VMC and RBM, where RBM tends to exaggerate the fluctuations compared to VMC. As discussed, this difference is probably caused by how the two methods model the electron correlations. The same effect was found in the two-body density plots, where the difference between the various correlation models is even more significant. In general, the RBM+PJ and VMC give a more significant electron-electron repulsion than the fellow methods RBM and RBM with a simple Jastrow factor (RBM+SJ).

This announces that energy estimates are not necessarily the best way to compare an RBM to standard VMC, other observable are potentially more crucial. The RBM+SJ is an excellent example of this, as it provides energy estimates similar to the VMC, but the two-body density plots exploit that the correlations were somewhat weaker. In general, we believe that the Padé-Jastrow factor works better than the simple Jastrow factor as it provides a lower energy and is constructed to model the electron-electron cusp correctly. As the simple Jastrow factor is more or less as computationally expensive as the Padé-Jastrow factor, we see no reason to select RBM+SJ instead of RBM+PJ.

Based on the discussions above, the RBM provides exciting results, but at its current version it is not able to compete with the existing many-body methods neither when it comes to performance, nor computational cost. However, we see the outcome of this work as a step in the right direction, and with some more investigation we believe that the RBM can be an alternative to traditional methods. More precisely, the plain RBM has some properties that makes it able to estimate the ground state energy at a lower cost than the VMC, and other RBM structures, for instance based on spherical coordinates, might enhance the performance at the same cost. We also see a bright future for the RBM+PJ, which for some systems give a lower energy than VMC, and our thought is that it can outperform the VMC if the cost is reduced.

If we now recall the goals presented in the introduction, we see that the first goal was to develop a VMC framework for the studies of large fermionic systems. This framework has been validated in the results, and give consistent results with references. The next goal was to extend the code to include RBMs. As far as we know, no one has done anything comparable, and it is therefore hard to validate this specific implementation. However, when comparing to the results obtained using other methods, we are confident that also this implementation is correct. Thus, it implies that our 7000 lines of code is implemented correctly. The third goal was to use our framework to study ground state properties of atoms and quantum dots. The quantum dots were studied thoroughly, both using VMC and RBM with various correlation factors. Further, we studied the atoms using VMC, but did not have time or manage to study them using the RBMs. We put some effort in trying to model the atoms using the same RBMs as for the quantum dots, but even with a large number of hidden nodes, those Gaussian-binary unit RBMs were not flexible enough to capture the properties of the atoms. Other possible attempts include expanding a Hartree-Fock basis in a set of RBMs, or simply choose another RBM structure which is not based on the Gaussian mapping. This is something that can and should be tried, and is one among many things that could be done in the future work. 

\section*{Future work}
As the use of machine learning for solving the many-body problem is just in the starting block, there are millions of things one could try. With the same approach as we did, there are plenty of structures, hyper-parameter setting and initial conditions that we did not have time to check out. An example is to investigate RBMs with a smaller number of hidden nodes than we did. Also, writing an RBM-code in spherical coordinates, instead of Cartesian coordinates, could be interesting as it might be easier to model the correlations in that coordinate system. Additionally, implementing a VMC code in spherical coordinates is less hassle because of the Jastrow factors. 

Moreover, investigating more complicated systems with an unknown wave function would be interesting, as we believe those systems are the primary applications of the RBMs. In the first place, multi-quantum dots (multiple quantum dots with intern connections) is a good candidate as there exist comparable experiments \cite{marzin_photoluminescence_1994,brunner_sharp-line_1994}. To simulate quantum-dots in-medium, one can use a screening of Coulomb interactions. As the RBM is able to model the two-body correlations, it is also imaginable that it can model the three-body correlations and therefore be used to simulate nuclear systems.

In section \ref{sec:dmc}, we discussed the sign-problem plaguing DMC, and that shadow wave functions can be used to bypass this problem. There are many similarities between the shadow wave functions and wave functions based on restricted Boltzmann machines, and it would be interesting to investigate this link further. 

As the quantum simulations are costly, one should always try to find the bottleneck and optimize that part of the code. For RBMs, the bottleneck might be the neural networks, which can be evaluated extremely fast on a GPU. However, the remaining framework is probably faster to evaluate on CPUs, so a combination of GPU and CPU would might be optimal.