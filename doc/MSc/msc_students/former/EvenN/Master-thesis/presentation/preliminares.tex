\usepackage{graphicx} % Allows including images
\usepackage{booktabs} % Allows the use of \toprule, \midrule and \bottomrule in tables
\usepackage{tikz,pgfplots}
\usepackage[absolute,overlay]{textpos}
\usetikzlibrary{shapes, snakes, arrows}

 \setlength{\TPHorizModule}{10mm}
 \setlength{\TPVertModule}{\TPHorizModule}
 \textblockorigin{1mm}{1mm} % start everything near the top-left corner
 \setlength{\parindent}{0pt}

\beamertemplatetransparentcovereddynamic
\beamertemplatesolidbuttons
\beamertemplateroundedblocks
\definecolor{azul}{RGB}{1,80,159}
\definecolor{azulclaro}{RGB}{230,230,255}	
\setbeamercolor{titlelike}{fg=azul}
\setbeamercolor{block title}{fg=white}
\setbeamercolor{block title}{bg=azul}
\setbeamercolor{block body}{bg=azulclaro}

%%%%%%%%%%%%%%%%%%%%%%%%%%%%%%%%%%%%%%%%%%%%%%%%%%%%%%%%%%%%%%%%%%%%%%%%%%%%%%%%%%%%%%%%%%%%%%%%%%%%%%%%%%%%%%%%%%%%%
\newcommand{\incertarcontador}{\begin{textblock}{3}(0,8.5)
\begin{tikzpicture}[x=1mm,y=1mm]
	 \foreach \x in {1,2,...,\inserttotalframenumber} 
	 					\draw [gray!30,very thick] (\x/\inserttotalframenumber*360:3) -- (\x/\inserttotalframenumber*360:4);
   \foreach \x in {1,2,...,\insertframenumber} 
   					\draw[azul,very thick] (\x/\inserttotalframenumber*360:3) -- (\x/\inserttotalframenumber*360:4);
   \node at (0,0) [] {\tiny{\textbf{\insertframenumber}}};
	\end{tikzpicture}
\end{textblock}
}


\newcommand{\incertarlogo}{\begin{textblock}{3}(10.5,0)		
\centering
\definecolor{colora}{rgb}{0.66,0.69,0.2}
\definecolor{colorb}{RGB}{103,99,74}
\definecolor{colorc}{RGB}{7,78,64}
\definecolor{colord}{RGB}{2,151,93}
\definecolor{colore}{RGB}{0,59,88}
\definecolor{colorf}{RGB}{0,128,189}
\definecolor{colorg}{RGB}{212,22,47}
\definecolor{colorh}{RGB}{226,112,42}
\definecolor{colori}{RGB}{237,173,24}
\definecolor{colorj}{RGB}{103,20,46}
\begin{tikzpicture}[x=0.1mm,y=0.1mm]     
    \draw[colorc, fill] (30,-30) -- +(0,30) -- +(10,30) -- +(10,0) -- cycle;
    \draw[colorb, fill] (10,-30) -- +(30,0) -- +(20,-10) -- +(0,-10) -- cycle;    
    \draw[colora, fill] (0,0) -- +(0,-30) -- +(10,-40) -- +(10,10) -- cycle;        
    \draw[colord, fill] (30,0) -- +(10,10) -- +(30,10) -- +(20,0) -- cycle;
    \draw[colorf, fill] (53,0) -- +(0,-40) -- +(10,-30) -- +(10,0) -- cycle;
    \draw[colore, fill] (53,0) -- +(10,10) -- +(30,10) -- +(20,0) -- cycle;
    \draw[colori, fill] (107,-12) -- +(-20,0) -- +(-20,-10) -- +(0,-10) -- cycle ;
    \draw[colorh, fill] (107,0) -- +(10,0) -- +(10,-12) -- +(0,-22) -- cycle;
    \draw[colorg, fill] (77,0) -- +(10,10) -- +(30,10) -- +(40,0) -- cycle;
    \draw[colorj, fill] (87,-12) -- +(-10,-10) -- +(-10,-28) -- +(0,-18);        
\end{tikzpicture}
\end{textblock}
}

%%%%%%%%%%%%%%%%%%%%%%%%%%%%%%%%%%%%%%%%%%%%%%%%%%%%%%%%%%%%%%%%%%%%%%%%%%%%%%%%%%%%%%%%%%%%%%%%%%%%%%%%%%%%%%%%%%%%%
\newcommand{\mframe}[1]{
\frame{
\begin{textblock}{3}(0,8.5)
\begin{tikzpicture}[x=1mm,y=1mm]
	 \foreach \x in {1,2,...,\inserttotalframenumber} 
	 					\draw [gray!30,very thick] (\x/\inserttotalframenumber*360:3) -- (\x/\inserttotalframenumber*360:4);
   \foreach \x in {1,2,...,\insertframenumber} 
   					\draw[azul,very thick] (\x/\inserttotalframenumber*360:3) -- (\x/\inserttotalframenumber*360:4);
   \node at (0,0) [] {\tiny{\textbf{\insertframenumber}}};
	\end{tikzpicture}
\end{textblock}

\begin{textblock}{3}(10.5,0)		
\centering



\definecolor{colora}{rgb}{0.66,0.69,0.2}
\definecolor{colorb}{RGB}{103,99,74}
\definecolor{colorc}{RGB}{7,78,64}
\definecolor{colord}{RGB}{2,151,93}
\definecolor{colore}{RGB}{0,59,88}
\definecolor{colorf}{RGB}{0,128,189}
\definecolor{colorg}{RGB}{212,22,47}
\definecolor{colorh}{RGB}{226,112,42}
\definecolor{colori}{RGB}{237,173,24}
\definecolor{colorj}{RGB}{103,20,46}
\begin{tikzpicture}[x=0.1mm,y=0.1mm]     
    \draw[colorc, fill] (30,-30) -- +(0,30) -- +(10,30) -- +(10,0) -- cycle;
    \draw[colorb, fill] (10,-30) -- +(30,0) -- +(20,-10) -- +(0,-10) -- cycle;    
    \draw[colora, fill] (0,0) -- +(0,-30) -- +(10,-40) -- +(10,10) -- cycle;        
    \draw[colord, fill] (30,0) -- +(10,10) -- +(30,10) -- +(20,0) -- cycle;
    \draw[colorf, fill] (53,0) -- +(0,-40) -- +(10,-30) -- +(10,0) -- cycle;
    \draw[colore, fill] (53,0) -- +(10,10) -- +(30,10) -- +(20,0) -- cycle;
    \draw[colori, fill] (107,-12) -- +(-20,0) -- +(-20,-10) -- +(0,-10) -- cycle ;
    \draw[colorh, fill] (107,0) -- +(10,0) -- +(10,-12) -- +(0,-22) -- cycle;
    \draw[colorg, fill] (77,0) -- +(10,10) -- +(30,10) -- +(40,0) -- cycle;
    \draw[colorj, fill] (87,-12) -- +(-10,-10) -- +(-10,-28) -- +(0,-18);        
\end{tikzpicture}
\end{textblock}

#1}
}
%%%%%%%%%%%%%%%%%%%%%%%%%%%%%%%%%%%%%%%%%%%%%%%%%%%%%%%%%%%%%%%%%%%%%%%%%%%%%%%%%%%%%%%%%%%%%%%%%%%%%%%%%%%%%%%%%%%%%%%
\usepackage{listings}
\lstset{frameround=fttt,language=C++, keywordstyle=\color{blue!30!green}, tabsize=4, commentstyle=\color{orange},
basicstyle=\ttfamily, numbers=left,numberstyle=\tiny}

\usefonttheme{serif}
\usepackage{mathtools}  
\mathtoolsset{showonlyrefs}