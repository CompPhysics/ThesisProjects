\chapter{Conclusion}

The goal of this thesis was to see if we could predict the $E2_1^+$ transition
strength for  $^{16}$C that was measured by the authors of Ref.~\citep{16CLifetime}. 
This experimental result would have indicated that the physics of the low-lying states
would have been dominated by collective proton-neutron excitations. New experiments corrected however this value
by almost one order of magnitude. 
Our theoretical results are in line with the revised experimental results and do not predict any new
physics involving the nuclei $^{14}$C, $^{15}$C, $^{15}$B and $^{16}$C.

We could however not predict the $E2_1^+$ transition strength of $^{14}$C. This
may suggest that this transition is more complex than we have taken into account
for.

An important part in the calculations has been to 
not adjust the effective interaction. 
We wanted to study the role played by different model spaces and computed therefore effective interactions
taylored to each specific model space.   
We have however used single particle
(sp) energies that have been adjusted to $^{16}$O, but have not done any
further adjustments on these sp energies. 

We have seen that we reproduce  rather well most of the low-lying states,
though the energy values sometimes are too high. For all the nuclei we reproduce 
the first excited. For $^{14}$C and $^{15}$C we need to work more on
the effective interaction and the size of the model space. 
The nuclei $^{15}$B and $^{16}$C, on the otherhand, come out
well.

We have seen that the $0p\frac32-0f\frac72$ model space gives a better reproduction of the 
experimental values when compared to the $0p\frac32-0d\frac32$ model space. In some
cases (see $^{16}$C, figure \ref{fig:16C}) we obtain a very good agreement with  the
experimental states. It should also be noted that we have only
calculated a few of the states for each nucleus, and it is therefore difficult
to draw a conclusion on how much of an improvement the $0p\frac32-0f\frac72$
model space is.  To further test the effect of higher-lying single-particle states,
we would need to include degrees of freedom from the $1p_{3/2}$ orbit and eventually from
the  $1p_{1/2}$ and $0f_{5/2}$ single particle orbitals. 
With the new parallel version of the shell-model code this can be tested
soon.

%Regarding the energy spectra, we have only/primarily? looked at low-lying
%states. The $0p\frac32-0d\frac32$ model space predicts most of these states,
%but the energys are generally considerably higher than the energys of the
%experimental states. $0p\frac32-0f\frac72$ model space brings the simulated?
%states much closer to the experimental states. In some cases (see $^{16}$C,
%figure \ref{fig:16C}) we have almost exactly the experimental states. However,
%we need to calculate higher-lying states in the $0p\frac32-0f\frac72$ model
%space to see if those states also comes out good. To do this, it is possible
%that we need to increase the number of particles we allow in the $0f\frac72$
%orbital.

Though there are several states, mostly in $^{14}$C, that we have not
reproduced and though the excitation energy sometimes is significantly higher
than the experimental value, there are several  important
aspects of studied the nuclei that we have been able to describe.

We have seen that $^{14}$C is mostly dominated by proton excitations, while
$^{15}$C, $^{15}$B and $^{16}$C are mostly dominated by neutron excitations. We
have also seen that for $^{15}$C the $0d\frac52$ orbital has higher energy than
the $1s\frac12$ orbital. All of these are in agreement with experimental
findings.

There are still several things that can be improved in our effective
interaction. I will list these here. The first three requires additional code
to be written for the program we have used, while the fourth implies a larger
Hamiltonian matrix to diagonalize.

\begin{itemize}

\item Resonances. They are important for $^{14}$C, as previously discussed, and
would most likely improve our results for this nucleus.

\item Removal of spurious center of mass corrections. This is another correction that probably
could improve our results, in particular for negative parity
would give.

\item Three-particle interactions arising from effective three-nucleon interactions and eventually from
three-body forces. These are important corrections, although  they tend to  effect mainly the ground state energies.   

\item Size of model space. Increasing our model-space is an obvious improvement.
However, this is restricted by computational power and would require better
algorithms or more powerful computers with more storage capasity. Still, it
would be a good improvement, as we could then see if we have a convergence in
our states. There are two ways to improve our model space. One is to increase
the number of particles we allow in the high-lying orbitals, such as the
$0f\frac72$ orbital, which would be an improvement for high-lying states. For the
low-lying states we have looked at, it is doubtful that it would bring any
significant change to our states. However, it is worth studying 
if our restrictions on the $0p\frac32-0f\frac72$ model-space are
good, especially the exclusion of protons from the $0f\frac72$ orbital, which I have only looked at qualitatively. Another,
more important improvement, would be to increase our model-space by including
the $1p\frac32$, $1p\frac12$ and $0f_{5/2}$ single-particle orbitals.

\end{itemize}
