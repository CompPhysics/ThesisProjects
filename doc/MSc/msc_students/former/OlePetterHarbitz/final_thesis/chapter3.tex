\newcommand{\br}{\mathbf r}
\newcommand{\bA}{\mathbf A}
\newcommand{\bj}{\mathbf j}
\renewcommand{\t}[1]{\textrm{#1}}
\chapter{Shell-model calculations}

We have now seen how we have reduced the full two-particle Hamiltonian in the
Hilbert space to a corresponding one in our model space. We now need to solve
the real, symmetric $m\times m$ matrix eigenvalue equation
\begin{align}
	\he\ket{\phi_k} = E_k\ket{\phi_k},
	\label{eq:Hm}
\end{align}
where $k = 1,\dots,m$.

In most situations one is only interested in the states with the lowest energy,
and their corresponding eigenvectors. Since the dimension of the configuration
space often is quite large, we do not want to diagonalize the whole matrix.
Most diagonalization methods, however, requires one to diagonalize the complete
Hamiltonian.

A diagonalization method that does not require one to diagonalize the whole
matrix is the Lanczos method. We discuss this method in section 2, and in
section 3 we will look at how we can find the $E2$ transition strength from the
diagonalized matrix. First, however, we will look at the so-called m-scheme
representation of the basis states we use.

\section{m-scheme representation}

The m-scheme representation of the basis states makes numerical calculations of
the eigenvalue problem relatively easy. It involves grouping the basis states
after the total spin projection $M$, instead of grouping them after the total
spin $J$ and isospin projection $T_z$. I have taken much of the information
here from \citep{HBmaster}.

Using the m-scheme representation will make our Hamiltonian much larger. The Hamiltonian will,
however, still have rotational symmetry, and commute with $J^2$ and $T_z$. This
ensures that the eigenvalues will have good $J$ and $T_z$ values.

We want to write the eigenstates of equation \ref{eq:Hm} as a linear
combination of Slater determinants, which are given by
\begin{align}
	\ket{\textrm{SD}_\nu} = \prod_{jm\in\nu}a_{jm}^\dagger\ket c,
	\label{SD}
\end{align}
where $\ket c$ is the reference vacuum state given by equation \ref{eq:core}.

We get the full basis-set by distributing the nucleons we have in every
possible way over our model space, with the restriction that the total angular
momentum projection $M$ must be $0$ for even nuclei and $\frac12$ for odd nuclei. The
number of states we have becomes very large for large model spaces. See table \ref{tab:dim} 
for the dimensions of the model spaces for the different nuclei.

\begin{table}
\begin{center}
\begin{tabular}{|c|c|c|c|c|}
	\hline
	& $^{14}$C & $^{15}$C & $^{15}$B & $^{16}$C \\
	\hline
	$0p\frac32-0d\frac32$ & $3.2\cdot10^6$ & & $2.0\cdot10^6$ & $7.4\cdot10^6$ \\
	\hline
	$0p\frac32-0f\frac72$ & $9.7\cdot10^6$ & $52.8\cdot10^6$ & $76.7\cdot10^6$ & $95.7\cdot10^6$ \\
	\hline
\end{tabular}
\caption{Table over the dimensions of the model spaces for the different nuclei. The first coloumn lists the model spaces, while the first row lists the nuclei.}
\label{tab:dim}
\end{center}
\end{table}


The Slater determinants in the m-scheme can be expressed very efficiently
numerically, requiring one integer? variable for each Slater determinant.
Numerically a Slater determinant can be coded as
\begin{align}
	\ket{\textrm{SD}} \rightarrow (010011101000\dots),
\end{align}
where there is one digit (bit) for every available single particle state in our
Slater determinant.

The bits represent the occupancy of a state. Bit 1 indicates that a state is
occupied, and bit 0 indicates that a state is unoccupied. The states are sorted
first after spin projection $m$, then angular momentum $j$, then orbital spin $l$
and finally the number of nodes $n$of the single-particle wave function.

I will illustrate this with an example. Table \ref{tab:m} shows the states we
have in the $1s\frac12$ and $0d\frac32$ orbitals. The number $i$ represents
which state we talk about, and is unique for each state. For $i = 2$ we look at
the state with $j = \frac32$ and $m = -\frac12$. Equation \ref{eq:SD3} shows a
configuration with three particles in the $1s\frac12$ and $0d\frac32$ orbitals.

\begin{table}
\begin{center}
\begin{tabular}{|c|c|c|c|c|c|c|}
	\hline
	i = & 1 & 2 & 3 & 4 & 5 & 6 \\
	\hline
	m = & $-\frac32$ & $-\frac12$ & $-\frac12$ & $\frac12$ & $\frac12$ & $\frac32$ \\
	\hline
	j = & $\frac32$ & $\frac32$ & $\frac12$ & $\frac12$ & $\frac32$ & $\frac32$ \\
	\hline
\end{tabular}
\caption{Table showing the available states in the $1s\frac12$ and $0d\frac32$ orbitals. The number $i$ says which state we have.}
\label{tab:m}
\end{center}
\end{table}

\begin{align}
	\ket{1s_{\frac12},0d_{\frac32}^2} &= a_{j=\frac32,m=\frac32}^\dagger a_{j=\frac32,m=-\frac12}^\dagger a_{j=\frac12,m=-\frac12}^\dagger\ket c\notag\\
	&= a_2^\dagger a_3^\dagger a_6^\dagger\ket c
	\label{eq:SD3}
\end{align}

The calculations used when we let the Hamiltonian operate on a state of this
type are simple. Let us give an example of this. We write our Hamiltonian as
\begin{align}
	\he = \sum_{i=1}^d\epsilon_ia_i^\dagger a_i+\frac14\sum_{ijkl}^d\bra{ij}\ve\ket{kl}a_i^\dagger a_j^\dagger a_la_k.
\end{align}
where $d$ is the dimension of the Hamiltonian matrix.
\\!Tror at det er $d$ istedenfor $A$. $d$ er det samme som $m$ i forklaringen
til ligning \ref{eq:Hm}.!

We look at the configuration $i=1$, $j=4$, $k=6$, $l=2$. We then get
\begin{align}
	a_1^\dagger a_4^\dagger a_2a_6(011001) &= - a_1^\dagger a_4^\dagger(001000)\notag \\
	&=-(101100).
\end{align}
This illustrates the simplicity of calculations in the m-scheme representation.

We have seen that the m-scheme representation will make computations with the
Slater determinants relatively easy, the drawback being that our Hamiltonian
matrix gets much larger. We will now go on to discuss how we will actually
diagonalize our Hamiltonian with the Lanczos method.

%In the m-scheme we represent the basis states by their total? angular momentum
%quantum numbers $n$, $l$ and $j$, and their projection of the spin j on the
%z-axis $m$. The quantum numbers $n$, $l$ and $j$ can be denoted by $k \equiv
%(n,l,j)$. The single particle wavefunctions will then be labeled by $\alpha =
%(km)$. Each value of $k$ stands for a unique set of $(n,l,j)$ and can, for a
%given value of $n$, $l$ and $j$, be computed by: !Det her er naermest avskrift
%av kap. 16 fra Alex Brown's ark. Boer skrives om.!
%\begin{equation}
%k = \frac12[(2n+l)(2n+l+3)-2j+3]
%\end{equation}
%
%The advantage of the m-scheme is that calculations are easy and
%straightforward. The disadvantage is that, since the m-scheme does not take
%into account rotational symmetry, the dimension of the Hamiltonian matrix gets
%very large.
%
%An alternative to the m-scheme is the J-scheme, where the basis states are
%given by coupled angular spins $J$. In this scheme the rotational symmetry is
%taken into account, and the dimension of the Hamiltonian matrix is heavily
%reduced. However, The calculations are more complicated, involving
%Clebsch-Gordan coefficients.

\section{Lanczos' method}

We will use the Lanczos method to diagonalize the effective two-particle
Hamiltonian in the m-scheme basis. One of the main advantages of the Lanczos
method is that we do not have to diagonalize the whole matrix. I will below
outline this method. For more details see \citep{lanczo1}, and \citep{lanczo2}

\begin{enumerate}

\item One chooses an initial Lanczos' vector $\ket{lanc_0}$ as a 0th order
approximation to the eigenvalue equation $H\ket{\psi_k} = E_k \ket{\psi_k}$,
with $k = 1,\ldots, K$. $K$ is the size of of the $H_{eff}$ matrix. The initial
Lanczos' vector should not have good angular momentum, as the iteration would
then terminate too early.

\item A new vector is generated by letting the Hamiltonian work on the Lanczos
vector: $|new_{p+1}> = H |lanc_p>$, where $p$ goes from $0$ to $K-1$. The
diagonal matrix elements of $H$ can now be found by

\be
\bra{lanc_p} H \ket{lanc_p} = \bra{lanc_p} \left . new_{p+1}\right\rangle.
\label{lanc1}
\ee

\item We orthogonalize the Lanczos vectors by

\be
\ket{new_{p+1}^{'}} = \ket{new_{p+1}} - \ket{lanc_p} \cdot
\bra{lanc_p} \left . new_{p+1} \right \rangle
	- \sum_{q = 0}^{p-1} \ket{lanc_q} \cdot
	\bra{lanc_q} \left . new_{p+1} \right \rangle,
\ee
and normalize it,
\be
\ket{lanc_{p+1}} = \frac{1}{\sqrt{\bra{new_{p+1}^{'}}
					\left . new_{p+1}^{'} \right \rangle}}
					\ket{new_{p+1}^{'}}.
\ee
This produces a new Lanczos' vector.

\item The off-diagonal matrix elements of $H$ can now be calculated by

\be
\bra{lanc_{p+1}} H \ket{lanc_p} = \bra{new_{p+1}^{'}}
								\left . new _{p+1}^{'}\right \rangle.
\label{off1}
\ee

The other matrix elements are zero.

\item After n iterations we will have a tri-diagonal matrix of the form

\be
H_n = 
\left \{
\begin{array}{ccccc}
H_{0,0} & H_{0,1} & 0       & \cdots   & 0  \\
H_{0,1} & H_{1,1} & H_{1,2} & \cdots   & 0  \\
0       & H_{2,1} & H_{2,2} & \cdots   & 0  \\
\vdots  & \vdots  & \vdots  & \vdots   & H_{p-1,p}  \\
0       & 0       & 0       & H_{p,p-1}   & H_{p,p}\\
\end{array}
\right \}.
\label{matr1}
\ee

\item This process is repeated until we have a convergence, that is $H_n
\approx H_{n-1}$.

\end{enumerate}

A problem with the Lanczos method is that it uses a lot of data storage. When
$m$ gets large, the size of each Lanczos' vector also increases. The storage
capacity needed gets very large when the number of Lanczos' vectors increases
beyond 100. However, to get a good convergence, we usually need a big number of
Lanczos' vectors, as the Lanczos method has a slow rate of convergence.

In our program we stop when we have reached a suitable convergence criteria for
the angular momenta, or when we reach the maximum number of iterations
specified at the start of the calculation. If we reach the iteration limit
before we have a convergence, we then need to look at the angular momentum $J$
from the program output. If the state we look at has a converged value for the
angular momentum, then that state has converged. In the energy spectrum
figures in chapter 4 I have only included those energy states that have a
converged value for the angular momentum.

The Lanzos method gives us, in addition to the energy for the excited states,
also their eigenvectors. These eigenvectors can be used to find the
electromagnetic transitions between the states.

\section{Electromagnetic transitions}

We will here show how to calculate the electromagnetic transitions. We will
concentrate on the E2 transition for even nuclei.

The electromagnetic transitions can be divided into electric and magnetic
transitions, and are defined by the spherical harmonics $Y_\mu^L$. For $L = 2$
both the E2 and the M2 transitions are possible, but they have differing
parity. The electric transitions are much more probable than the magnetic
transitions for equal values of $L$.

The theory for electromagnetic transitions is based on Maxwell's equations for
a propagating electromagnetic field.

The interaction Hamiltonian is given as (see \citet{brussaard} for details)
\begin{align}
	H_{e.m.} = \frac1c\int\bj(\br,t)\bA(\br,t)d\br + \int\rho(\br,t)\phi(\br,t)d\br,
\end{align}
where $\bj(\br,t)$ represents the current density, and $\rho(\br,t)$ represents
the charge density, of the nucleus. $\bA(\br,t)$ is a vector potential and
$\phi(\br,t)$ is a scalar potential.

The transverse gauge condition is
\begin{align}
	\nabla \bA(\br,t) = 0.
	\label{Eq:tg}
\end{align}
%we can derive the vector Helmholtz equation
%%Morse and Feshbach (1953)
%\begin{align}
%	\nabla^2\bA(q\br) + q^2\bA(q\br) = 0.
%\end{align}
%!Er usikker paa hva denne sier, og om den er viktig. Derfor boer jeg nok ikke
%ta den med.!
The transverse gauge condition reflects the fact that the photon only has two
independent polarization states.

The transition rate $T$, that is the transition probability per time, for a
transition between an initial state i and final state f, is given as
\citep{brussaard},
\begin{align}
	T(L) = \frac{8\pi(L+1)}{L[(2L+1)!!]^2}\frac{k^{2L+1}}{\hbar}B(L),
\end{align}
where B(L) is the "reduced transition probability" defined as
\begin{align}
	B(L) = \frac{|\bra{J_f}|\hat O(L)|\ket{J_i}|^2}{2J_i+1}.
\end{align}
Here the Wigner-Eckarts theorem has been used to define the reduced matrix
element
\begin{align}
	\bra{njm}T_\mu^\lambda\ket{n'j'm'} = \bra{nj}|T^\lambda|\ket{n'j'}C_{\lambda\mu j'm'}^{jm},
\end{align}
where $C_{\lambda\mu j'm'}^{jm}$ is a Clebsch-Gordan coefficient.

The initial and final states for systems where one photon is emitted, as is the
case in our calculations, can be written as
\begin{align}
	\ket I = \ket{J_iM_i}
\ket F = \ket{J_fM_f,1_{k,\alpha}},
\end{align}
where $1_{k,\alpha}$ is a one-photon state.

The electromagnetic transition depends on the reduced matrix elements,
\begin{align}
	\bra F \hat O\ket I
	\label{Eq:Hem}
\end{align}
where $\hat O$ is the sum of the electric and magnetic multipole operators
\begin{align}
	\hat O = \sum_{L,\mu}\left(\hat O(EL)_\mu + \hat O(ML)_\mu\right).
\end{align}

The electric transition operator \citep{brussaard} is given by
\begin{align}
	O(EL)_\mu = \sum_{k=1}^Ae(k)r^L(k)Y_\mu^L(\br),
	\label{Eq:EO}
\end{align}
where $Y_\mu^L$ are the spherical harmonics and $e(k)$ is the electric charge
for nucleon $k$. In our calculations the electric charge will be replaced by
the effective electric charge, as the neutron, while neutral as a whole, feels
an effective charge from the neutrons and protons in the closed core. The same
is true for the proton. The value of the effective charges are adjusted to fit
the experimental data.

The reduced transition probability for E2 transitions can be written as
\begin{align}
	B(E2,L) = \frac{1}{2J_i+1}|\bra{J_f}|\sum_ke_kr_k^2Y^2(\theta_i\phi_i)|\ket{J_i}|^2.
	\label{Eq:BE2}
\end{align}

The program finds the reduced matrix elements of \ref{Eq:BE2} by using the
eigenvectors that we found with the Lanczos method. It then uses these
reduced matrix elements to find the transition probability for a given L.

With this we have now briefly discussed the physics behind the program. We will
in the next chapter go on to look at the results from our computations.
