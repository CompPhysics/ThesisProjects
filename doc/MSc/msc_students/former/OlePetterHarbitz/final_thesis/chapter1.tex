\chapter{Introduction}
%

The purpose of this thesis was to see if nuclear many-body theory could predict
the unexpectedly low experimental value of the $E(2_1^+)$ excitation strength
for $^{16}$C. This value was reported by \citet{16CLifetime}. They measured an
unexpectedly low E2 transition strength between the first $2_1^+$ and $0^+$
states of $^{16}$C. The measured value was $0.63 e^2$fm$^4$.
%http://link.aps.org/abstract/PRL/v100/e152501  annen artikkel.

This result came as a surprise because the reduced electric quadrupole (E2)
transition probability B(E2) between the first $2_1^+$ and $0^+$ state is
proportional to the inverse of the excitation energy of the $2_1^+$ state
$E(2_1^+)$. This implies that the B(E2) should be low at the beginning or end
of a closed shell, and high in the middle of a shell. For $^{14}$C the
excitation energy is $7.01$ MeV and the transition strength is $3.7 e^2$fm$^4$.
Since the excitation energy for $^{16}$C is $1.77$ MeV, one would expect the
transition strength to increase. However, the measured transition strength of
$0.63 e^2$fm$^4$ was much lower than the $E(2_1^+)$ transition for $^{14}$C.

This experiment led to speculations that some new physics might be
involved in $^{16}$C. To understand this experiment and  the low-lying structure of 
$^{16}$C, possible proton-neutron  correlations were discussed, in contrast to 
the accepted picture  which said that  these low-lying states
were ruled mainly by neutron excitations.

%\footnote{The detectors in the \citep{16CLifetime} experiment measured in a
%too low scattering angle. The new experiments had the detectors measure in a
%much wider scattering angle.}
The experimental results of \citep{16CLifetime} lead different experimental
groups to remeasure this transition. The transition strength of $^{16}$C has
recently been measured by several groups, such as \citet{16CE2}. Their new experimental value is
$4.15$fm$^4$, in accordance with theoretical predictions. 

The aim of this thesis is to calculate this transition strength of $^{16}$C
and study the structure of nuclei near $^{16}$C in order to understand the degrees
of freedom involved in the low-lying states of these nuclei.The main emphasis, in addition to the structure of
 $^{16}$C and the above transition, is to study the energy spectra for $^{14}$C, $^{16}$C,
$^{15}$B and, if time permits, $^{15}$C.
I will do shell-model calculations using existing codes for shell-model and effective
interaction calculations developed in Oslo. All codes are available at http://www.fys.uio.no/compphys.



Finding the energy spectra is interesting because the nucleon-nucleon (NN)
interaction is not known in closed form like the Coulomb interaction. To
calculate it accurately one would have to go to Quantum Chromo Dynamics and
look at the interaction between the quarks making up the nucleons, but this is
at present impossible to calculate accurately. Furthermore the nucleons form a
many-particle problem that can only be solved approximately. It is therefore
important to get more data on how our theoretical models correspond with the
experimental data, so we can achieve a better understanding of the NN
interaction in the medium.

The program I will use finds the energy for the ground state and a given 
number of  excited states, for a given nucleus in a given model space.
The program works in three steps. First it renormalizes the NN interaction.
Then it finds an effective interaction using many-body perturbation theoy. These
calculations depend on the chosen model space. This model space is chosen
according to the relevant degrees of freedom of a specific nucleus. Finally it
finds the energy eigenvalues of the Hamiltonian by calculating two-body
interactions in the shell-model.

One of the larger problems in nuclear physics is finding good single particle
(sp) energies that accurately predict the energy spectrum. These are usually
free parameters that are adjusted to fit the experimental data. I will try out
two different sets of single particle energies. The program I use calculates
the single particle energies when it finds the effective interaction. I will
use both these single-particle energies and the single particle energies that E. K. Warburton and B. A.
Brown give for $^{16}$O in Ref.~\citep{brown}. The single-particle energies derived from  $^{16}$O
are expected to be close to the single-particle energies for $^{16}$C.

There have been few calculations in large model spaces such as the ones I will
look at. It will be interesting to see if using such large model spaces can
predict the experimental values without having to adjust the effective
interaction to each nucleus. There have also been few calculations that mix the
$0p$ and $1s0d$ shells, as these calculations have been difficult to do. Such
large calculations are, however, now possible due to better algorithms and the
advancement of computation power.

The reason I look at the nuclei $^{14}$C, $^{15}$B and $^{15}$C in addition to
$^{16}$C, is that their structure is close to $^{16}$C (see figures
\ref{modellrom1}, \ref{modellrom2}, \ref{modellrom3}, \ref{modellrom4}). The
particles in a nucleus have a higher probability to excite to another orbital
within the same shell (f.ex. $0d_{\frac{5}{2}}$ to $1s_{\frac{1}{2}}$) than
exciting to an orbital in another shell (f.ex. $0p_{\frac{1}{2}}$,
$0d_{\frac{5}{2}}$). The protons in $^{16}$C fills the $0p_{\frac{3}{2}}$
orbital, and the nucleus has two neutrons outside the filled $0p$ shell. We can
then expect that the number of neutrons in the $0p$ shell will not change much
when we go from the ground state to higher-lying excited states. The neutrons
in the $1s0d$ shell however, will have a high excitation probability. The
protons in $^{16}$C will mainly excite to the $0p_{\frac{1}{2}}$ orbital, and
should also be lower than the neutron excitation, as they fill an orbital.
$^{15}$B and $^{16}$C have the same number of neutrons. The neutron excitations
of $^{15}$B should therefore be similar to the neutron excitations of $^{16}$C.
$^{15}$C and $^{16}$C should have similar proton excitations. 


Ideally one should use the full Hilbert space when calculating the energy
spectrum. Since the Hilbert space is infinitely large, this is impossible. I
will therefore use a reduced model space, such as the one given in figures
\ref{modellrom1}, \ref{modellrom2}, \ref{modellrom3} and \ref{modellrom4}. The
higher-lying shells should give less and less contribution to the energy of the
nucleus, as the particles need more energy to excite up to them. I will,
because of this and because the size of the Hamiltonian matrix quickly explodes
for a large number of orbitals, use only the 0s0p and $1s0d$ shells as my model
space. More specifically, I will use the orbitals $0p_{\frac{3}{2}}$,
$0p_{\frac{1}{2}}$, $0d_{\frac{5}{2}}$, $1s_{\frac{1}{2}}$ and
$0p_{\frac{3}{2}}$, $0p_{\frac{1}{2}}$, $0d_{\frac{5}{2}}$, $1s_{\frac{1}{2}}$,
$0d_{\frac{3}{2}}$ as my model spaces. The $0p_{\frac{3}{2}}$ - $1s_\frac12$
and $0p_{\frac{3}{2}}$ - $0d_{\frac{3}{2}}$ model spaces will hopefully show a
convergence of energy. If not I should include the next orbital, namely the
$0f_{\frac{7}{2}}$ orbital. However, this model space will probably be too big,
as the number of possible sp states for $^{16}$C increases by a factor of 35.
Also I expect the contribution from the $0f_{\frac{7}{2}}$ orbital to be very
low, as it is in a new shell, with a high energy gap between it and the
$0d_\frac32$ orbital.

With convergence, I mean that the energy values we get from the program will
not change when we increase our model space. This will imply that increasing
our model space further will not significantly change the energy values we get.

The energy spectra for $^{14}$C, $^{16}$C, $^{15}$B and $^{15}$C will be
calculated using four different approaches to the effective interaction, based
on many-body perturbation theory to third order \citep{g261}. These four
approaches employ as starting points a $G$-matrix with a harmonic-oscillator
basis, a $G$-matrix with a Hartree-Fock basis, a renormalized NN interaction
based on the Vlowk method \citep{G-matrix} and a harmonic oscillator basis, and
finally a Vlowk interaction with a Hartree-Fock basis. I will in the final
chapter discuss which of these methods are best, and choose that method to
represent our findings.

%The number of (states/proton neutron configurations)? is calculated by first
%looking at the number of ways one can excite two nucleons in the model-space,
%which is given by the binomial coefficients?

In the next two chapters we will discuss how the shell model program works.

In chapter 2 we discuss the renormalization of the Hamiltonian by approximating
the many-body interaction to a two-particle interaction working in our
model space. We also discuss the Hartree-Fock method.

In chapter 3 we show how we numerically solve our Hamiltonian in the m-scheme
through an iterative numerical method known as the Lanczos algorithm. We also
discuss some of the theory for electromagnetic transitions in the nuclei.

In the chapter 4 we discuss the results we have gotten from our calculations.

Our conclusions are presented in the last chapter.

%The transition Strength is defined as the sum over all the final states (for a given transition) of the final and initial states working on the electric or magnetic transition operator, squared.
%\begin{equation}
%	\sum_{f,\mu}|\langle f|\hat{O}_{\mu}^{\lambda}|i\rangle |^2
%	\label{eq:1.1}
%\end{equation}
%The electric transition operator $\hat{O}_{\mu}^{\lambda}$ is given as
%\begin{equation}
%	\hat{O}_{\mu}^{\lambda} = r^{\lambda}Y_{\mu}^{\lambda}(\hat{r})e_{t_z}e
%	\label{eq:1.2}
%\end{equation}
%where $Y_{\mu}^{\lambda}$ are the spherical harmonics, and $e_{t_z}$ are the electric charges for the proton and neutron in units of e.
%!(referanse til B. Alex Brown sine forelesningsnotater.)!

%?Magnetisk overgang er når det totale spinnet forandres (eks. s-skall til
%annet p-skall (s-skall har 0 i totalt spinn))? ?Elektrisk overgang når
%ladningsfordelingen forandres, men ikke spinnet (eks. s-skall til annet
%s-skall)?

%Radiative lifetime = transition strength? Jo kortere tid en overgang tar, jo
%større er overgangs styrken?

%$w_{if}$ er vinkelfrekvensen (wpi*frekvens) (E \hbar w).

\fig{C14_0p3-0d3.eps}{Model-space for $^{14}$C. Filled circles represent
nucleons that are free to be excited. The circles with a cross over them
represent nucleons that are held still in their respective orbitals, and are not
allowed to be excited. The empty circles represent available states that the
nucleons can be excited to. The gray box indicates the excluded states that form
the effective interaction.}{modellrom1}
\fig{C15_0p3-0d3.eps}{Model-space for $^{15}$C. Filled circles represent
nucleons that are free to be excited. The circles with a cross over them
represent nucleons that are held still in their respective orbitals, and are not
allowed to be excited. The empty circles represent available states that the
nucleons can be excited to. The gray box indicates the excluded states that form
the effective interaction.}{modellrom2}
\fig{B15_0p3-0d3.eps}{Model-space for $^{15}$B. Filled circles represent
nucleons that are free to be excited. The circles with a cross over them
represent nucleons that are held still in their respective orbitals, and are not
allowed to be excited. The empty circles represent available states that the
nucleons can be excited to. The gray box indicates the excluded states that form
the effective interaction.}{modellrom3}
\fig{C16_0p3-0d3.eps}{Model-space for $^{16}$C. Filled circles represent
nucleons that are free to be excited. The circles with a cross over them
represent nucleons that are held still in their respective orbitals, and are not
allowed to be excited. The empty circles represent available states that the
nucleons can be excited to. The gray box indicates the excluded states that form
the effective interaction.}{modellrom4}
