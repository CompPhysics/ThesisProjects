\chapter{Results}

We will here look at the energy spectra for $^{14}$C, $^{15}$C, $^{15}$B and
$^{16}$C, and selected  excitation strengths for $B(E2)$ of $^{14}$C and $^{16}$C. We will begin
by discussing the energy spectra for each nucleus, starting with $^{14}$C and
ending with $^{16}$C. We will then discuss the excitation strengths we got for
$^{14}$C and $^{16}$C.

As mentioned earlier, I used two sets of single particle (sp) energies: Those
that the program calculated, and those given by E. K. Warburton
and B. A. Brown in Ref.~\citep{brown} for $^{16}$O. During the calculations it became
apparent that the sp energies from Brown's article gave excitation energies,
for all four nuclei, that were in much better agreement with the experimental
values. Also, when I used the sp energies that the program calculated, the
states had a slow rate of convergence in the Lanczos method. See figures
\ref{fig:14C_own_3pert_0d3_4part} and \ref{fig:14C} for a comparison between
the program's sp energies and Brown's sp energies.

Figures \ref{fig:14C_own_3pert_0d3_4part} and \ref{fig:16C_brown_3pert_1s1}
each show four different calculations done in the same model space. The
calculations with a Hartree-Fock basis gave worse excitation energies than the
calculations without Hartree-Fock. We see this behaviour for the other
nuclei too. The Hartree-Fock single-particle energies result in a single-particle gap which is much larger than the experimental values. This results in large enery denominators when we do many-body perturbation theory and thereby an effective interaction which is smaller on average than the one obtained using hamronic oscillator energies.
The final result is an excitation spectrum which is more compressed.
Because of this I will not include calculations with
Hartree-Fock in our discussions of the nuclei.

In figures \ref{fig:14C_own_3pert_0d3_4part} and \ref{fig:16C_brown_3pert_1s1}
we see that the $G$-matrix and Vlowk methods give almost identical energy
spectra. This is also the case for the other calculations in the
$0p\frac32-0d\frac32$ model space. I cannot tell from the figures which is
best, but the Vlowk method depends strongly on the chosen cutoff in momentum space. 
A larger  cutoff produces smaller matrix elements and thereby a more compressed spectrum.
On the other hand, a smaller cutoff than that chosen here, leads to larger effective matrix elements.  
Unless the two-body interactionis accompanied by a three-body interaction or higher-body interaction computed with the 
same cutoff value,  many-body correlations will be large when doing shell-model calculations. 
I will therefore use the $G$-matrix method in our
discussions of the nuclei.

\begin{figure}[htbp]
\setlength{\unitlength}{3.0cm}
\begin{center}
\begin{picture}(6.9,6.5)(0,-1)
\psset{xunit=0.80cm, yunit=6.0cm}
\newcommand{\drawlevel}[5]{\psline[origin={#1,#2}, linewidth=0.3pt](0,0)(1.5,0.0)\rput(#3,#4){\scriptsize \makebox(0,0){$#5$}}}
\newcommand{\connect}[4]{\psline[linewidth=0.3pt, linestyle=dotted, dotsep=1.2pt](#1,#2)(#3,#4)}
\thicklines
\rput(16.9,-0.2){\makebox(0,0){{\large exp. data}}}
\drawlevel{-16.1}{-0}{18.8}{0}{\ 0^+ \ (0)}
\drawlevel{-16.1}{-1.737}{18.8}{1.737}{\ 1^- \ (6.0938)}
\drawlevel{-16.1}{-1.879}{18.8}{1.879}{\ 0^+ \ (6.5894)}
\connect{17.6}{1.879}{17.9}{1.879}
\drawlevel{-16.1}{-1.919}{18.8}{1.949}{\ 3^- \ (6.7282)}
\connect{17.6}{1.919}{17.9}{1.949}
\drawlevel{-16.1}{-1.9688}{18.8}{2.019}{\ 0^- \ (6.9026)}
\connect{17.6}{1.9688}{17.9}{2.019}
\drawlevel{-16.1}{-2}{18.8}{2.159}{\ 2^+ \ (7.012)}
\connect{17.6}{2}{17.9}{2.159}

\rput(0.1,-0.2){\makebox(0,0){{\large G-matrix; no HF}}}
\drawlevel{0.7}{-0}{2}{0}{\ 0.00^+ \ (0)}
\drawlevel{0.7}{-0.7037}{2}{0.7037}{\ 1^- \ (2.4668)}
\drawlevel{0.7}{-1.699}{2}{1.699}{\ 2^+ \ (5.9581)}

\rput(4.3,-0.2){\makebox(0,0){{\large G-matrix; with HF}}}
\drawlevel{-3.5}{-0}{6.2}{0}{\ 0^+ \ (0)}
\drawlevel{-3.5}{-1.383}{6.2}{1.383}{\ 1^- \ (4.8493)}
\connect{5}{1.383}{5.3}{1.383}
\drawlevel{-3.5}{-1.49}{6.2}{1.49}{\ 2^+ \ (5.2247)}
\connect{5}{0.857922}{5.3}{0.866279}
\drawlevel{-3.5}{-1.14017}{6.2}{1.14017}{\ 1^+ \ (6.9436)}

\rput(8.5,-0.2){\makebox(0,0){{\large Vlowk; no HF}}}
\drawlevel{-7.7}{-0}{10.4}{0}{\ 0^+ \ (0)}
\drawlevel{-7.7}{-0.426999}{10.4}{0.426999}{\ 1^- \ (2.6004)}
\drawlevel{-7.7}{-1.00075}{10.4}{1.00075}{\ 2^+ \ (6.0945)}

\rput(12.7,-0.2){\makebox(0,0){{\large Vlowk; with HF}}}
\drawlevel{-11.9}{-0}{14.6}{0}{\ 0^+ \ (0)}
\drawlevel{-11.9}{-0.6317}{14.6}{0.6317}{\ 1^- \ (2.215)}
\drawlevel{-11.9}{-0.8721}{14.6}{0.8721}{\ 2^+ \ (3.0579)}
\drawlevel{-11.9}{-1.3512}{14.6}{1.3512}{\ 1^+ \ (4.7373)}

\end{picture}
\end{center}
\caption{Energy spectra for $^{14}$C, in the 0p3-0d3 model space, using 3. order perturbation theory. The spectra have been found using the orbital-energies from the program. 'HF' here means Hartree-Fock. 'G-matrix' and 'Vlowk' are different renormalization methods, see chapter 2 for details. The energies are given in MeV.}
\label{fig:14C_own_3pert_0d3_4part}
\end{figure}




\begin{figure}[htbp]
\setlength{\unitlength}{3.0cm}
\begin{center}
\begin{picture}(6.9,6.5)(0,-1)
\psset{xunit=0.80cm, yunit=6.0cm}
\newcommand{\drawlevel}[5]{\psline[origin={#1,#2}, linewidth=0.3pt](0,0)(1.5,0.0)\rput(#3,#4){\scriptsize \makebox(0,0){$#5$}}}
\newcommand{\connect}[4]{\psline[linewidth=0.3pt, linestyle=dotted, dotsep=1.2pt](#1,#2)(#3,#4)}
\thicklines
\rput(16.9,-0.2){\makebox(0,0){{\large exp. data}}}
\drawlevel{-16.1}{-0}{18.8}{0}{\ 0^+ \ (0)}
\drawlevel{-16.1}{-0.627476}{18.8}{0.627476}{\ 2^+ \ (1.77)}
\drawlevel{-16.1}{-1.07552}{18.8}{1.07552}{\ (0^+) \ (3.03)}
\drawlevel{-16.1}{-1.41626}{18.8}{1.41626}{\ 2 \ (3.99)}
\connect{17.6}{1.41626}{17.9}{1.41626}
\drawlevel{-16.1}{-1.4525}{18.8}{1.48626}{\ 3(^+) \ (4.09)}
\connect{17.6}{1.4525}{17.9}{1.48626}
\drawlevel{-16.1}{-1.47169}{18.8}{1.55626}{\ 4^+ \ (4.14)}
\connect{17.6}{1.47169}{17.9}{1.55626}
\drawlevel{-16.1}{-2.17058}{19.1}{2.17058}{\ (2^+,3^-,4^+) \ (6.11)}

\rput(0.1,-0.2){\makebox(0,0){{\large G-matrix; no HF}}}
\drawlevel{0.7}{-0}{2}{0}{\ 0^+ \ (0)}
\drawlevel{0.7}{-0.753433}{2}{0.753433}{\ 2^+ \ (2.12)}
\drawlevel{0.7}{-1.03992}{2}{1.03992}{\ 0^+ \ (2.93)}
\drawlevel{0.7}{-1.6136}{2}{1.6136}{\ 2^+ \ (4.54)}
\connect{0.8}{1.6136}{1.1}{1.6136}
\drawlevel{0.7}{-1.66267}{2}{1.6836}{\ 3^+ \ (4.68)}
\connect{0.8}{1.66267}{1.1}{1.6836}
\drawlevel{0.7}{-1.96163}{2}{1.96163}{\ 4^+ \ (5.52)}

\rput(4.3,-0.2){\makebox(0,0){{\large G-matrix; with HF}}}
\drawlevel{-3.5}{-0}{6.2}{0}{\ 0^+ \ (0)}
\drawlevel{-3.5}{-0.477926}{6.2}{0.477926}{\ 2^+ \ (1.35)}
\drawlevel{-3.5}{-0.876974}{6.2}{0.876974}{\ 4^+ \ (2.47)}
\drawlevel{-3.5}{-1.13607}{6.2}{1.13607}{\ 2^+ \ (3.20)}
\drawlevel{-3.5}{-1.37714}{6.2}{1.37714}{\ 3^+ \ (3.88)}
\drawlevel{-3.5}{-1.55927}{6.2}{1.55927}{\ 0^+ \ (4.39)}

\rput(8.5,-0.2){\makebox(0,0){{\large Vlowk; no HF}}}
\drawlevel{-7.7}{-0}{10.4}{0}{\ 0^+ \ (0)}
\drawlevel{-7.7}{-0.785944}{10.4}{0.785944}{\ 2^+ \ (2.21)}
\drawlevel{-7.7}{-1.02748}{10.4}{1.02748}{\ 0^+ \ (2.89)}
\drawlevel{-7.7}{-1.62661}{10.4}{1.62661}{\ 2^+ \ (4.58)}
\drawlevel{-7.7}{-1.73007}{10.4}{1.73007}{\ 3^+ \ (4.87)}
\drawlevel{-7.7}{-2}{10.4}{2}{\ 4^+ \ (5.63)}

\rput(12.7,-0.2){\makebox(0,0){{\large Vlowk; with HF}}}
\drawlevel{-11.9}{-0}{14.6}{0}{\ 0^+ \ (0)}
\drawlevel{-11.9}{-0.484606}{14.6}{0.484606}{\ 2^+ \ (1.36)}
\drawlevel{-11.9}{-0.919753}{14.6}{0.919753}{\ 4^+ \ (2.59)}
\drawlevel{-11.9}{-1.06205}{14.6}{1.06205}{\ 2^+ \ (2.99)}
\drawlevel{-11.9}{-1.23196}{14.6}{1.23196}{\ 3^+ \ (3.47)}
\connect{13.4}{1.23196}{13.7}{1.23196}
\drawlevel{-11.9}{-1.27645}{14.6}{1.30196}{\ 0^+ \ (3.59)}
\connect{13.4}{1.27645}{13.7}{1.30196}

\end{picture}
\end{center}
\caption{Energy spectra for 16C, in the 0p3-1s1 model space. The spectra have been found using the orbital-energies for $^{16}$O given by B.A. Brown and E.K. Warburton \citep{16CLifetime}. 'HF' here means Hartree-Fock. 'G-matrix' and 'Vlowk' are different renormalization methods, see chapter 2 for details. The energies are given in MeV.}
\label{fig:16C_brown_3pert_1s1}
\end{figure}





The energy spectra for the final results are given in figures \ref{fig:14C},
\ref{fig:15C}, \ref{fig:15B} and \ref{fig:16C}. Unless otherwise notified in
the figure labels, each figure has been calculated using: the orbital-energies
for $^{16}$O given by B.A. Brown and E.K. Warburton \citep{brown}, the
$G$-matrix method without Hartree-Fock, full $0p\frac32-0d\frac32$ model space,
a $0p\frac32-0f\frac72$ model space reduced by only allowing 4 neutrons and 4
protons in the $0d\frac52$orbital, and 2 neutrons and 2 protons in each of the
$0d\frac32$ and $0f\frac72$ orbitals. For $^{15}$C and $^{16}$C the
$0p\frac32-0f\frac72$ model space has been reduced further by allowing 0
protons to ecxite to the $0f\frac72$ orbital.

The energy spectra on the right side of the figures show the experimental
excitation energies. For $^{16}$C and especially $^{15}$B only the lowest
energy states have been found experimentally. For $^{14}$C (see figure
\ref{fig:14C}) I have included only those excitation states with energies below
12 MeV. This is because above 12 MeV there are no states with angular momenta
$1^-$, $0^+$ and $2^+$ in the experimental data, and no states with angular
momentum $3^-$ above 16 MeV \citep{nndc}.

The first number in each energy spectrum is the angular momentum of that
excited state, with the parity also given. The next number, given in
parenthesis, is the energy difference (in MeV) between the ground state and
that excited state. The excited states are characterized by the angular
momentum, parity and energy. The first line in each energy spectrum is the
ground state. The experimental angular momenta and parity sometimes have
parenthesis around them. This means that there is some uncertainty regarding
these values. GE means greater than or equal to.

Figures
\ref{fig:14C_g_0hf_3pert_0d3_4part_brown_0}-\ref{fig:14C_g_0hf_3pert_0f7_2part_brown_1},
\ref{fig:15C_g_0hf_3pert_0f7_2part_brown_0}-\ref{fig:15C_g_0hf_3pert_0f7_2part_brown_1},
\ref{fig:16C_g_0hf_3pert_0d3_4part_brown_0}-\ref{fig:16C_g_0hf_3pert_0f7_2part_brown_1}
and
\ref{fig:15B_g_0hf_3pert_0d3_4part_brown_0}-\ref{fig:15B_g_0hf_3pert_0f7_2part_brown_1}.
show the sp orbital occupancy for $^{14}$C, $^{15}$C, $^{16}$C and $^{15}$B. I
have here also only looked at results calculated with the $G$-matrix method,
without a Hartree-Fock single-particle basis. The figures for the ground state show how many protons
and neutrons are in each orbital as a bar diagram. The figures for the excited
states show the occupation difference between that excited state and the ground
state. I have only included figures for the ground state and the first excited
state, as it is the first excited $2^+$ state that we are interested in.

Tables \ref{C14_0d3_p}-\ref{C16_0f7_n} show the orbital occupancy for the
ground state and excited  states and the orbital occupancy difference form the
ground state for the first three excites states of each nucleus. Though we are
mainly interested in the first excited $2^+$ state, I do discuss the other
excited states. While the bar diagrams give us a better overall picture of the
orbital occupancies, the added tables list the  numerical values as well.

We notice in the orbital occupancy figures that the occupancy of the ground
state differs from our model's occupancy (see for example figure
\ref{modellrom3}). 
The simple model presumes one single Slater determinant.
Our final
wavefunction, on the other hand, includes the effect correlations as well, mixing many 
Slater determinants.  This means in turn that single-particle occupancy can be higly fragmented.


%\emph{EXPLAIN THIS BETTER} A point of interest regarding the orbital occupancy
%figures, is that the occupancy of the ground state differs from our model's
%occupancy (see figure \ref{modellrom1}). We also note that, although the total
%number of particles is a whole number, we do not have a whole number of
%particles in each orbital. This is because the orbitals we have are sp
%orbitals, that is, orbitals that are calculated for non-interacting particles.
%Our slater determinant, on the other hand, has nucleon nucleon interactions.
%Therefore the real orbitals for the interacting particles will be given by the
%full slater determinant, and will not be the same as the single particle
%orbitals. This means that when we have, f.ex. 0.3 neutrons in the $0d\frac32$
%orbital, what we really have is 0.3 particles that overlaps between the real
%interacting particle orbitals and the $0d\frac32$ single particle orbital. The
%interaction might also cause the excitation energies for the real orbitals to
%change, thus switching the order of orbitals that are close to each other in
%excitation energy. !Ikke sikker paa dette.!

The $0p\frac32-1s\frac12$ model space quickly proved to be too small (see
figure \ref{fig:16C_brown_3pert_1s1}), as the results did not correspond well
with the experimental data. We therefore had to include the $0d\frac32$ orbital
in our model space. The results from the calculations in the
$0p\frac32-0d\frac32$ model space are given, with the results from the
calculations in the $0p\frac32-0f\frac72$ model space, in figures
\ref{fig:14C}, \ref{fig:15B} and \ref{fig:16C}. We see from the
energy spectra that the excitation energies had not yet converged in the
$0p\frac32-01s\frac12$ model space. The results are, however, still rather
different from the experimental values, and we need to include the next orbital
(the $0f\frac72$ orbital) to see if the excitation energies have converged as functions
of the model spaces.

To test if the excitation energies had stabilized, I ran the program in the
$0p\frac32-0f\frac72$ model space. Because this model space is over $50$ times
larger than the $0p\frac32-0d\frac32$ model space, using the full model space
would be too large for the computers I used. Therefore I needed to reduce the
model space. If one looks at figures
\ref{fig:15B_g_0hf_3pert_0d3_4part_brown_0} and
\ref{fig:15B_g_0hf_3pert_0d3_4part_brown_1} of the occupancy of single particle
orbitals for the ground state and the first excited state of $^{15}$B, one sees
that the $0d\frac32$ orbital has an average of under $0.3$ particles in it, and
the $0d\frac52$ orbital has under $1.8$ particles. The $0f\frac72$ orbital
should have even less particles in it than the $0d\frac32$ orbital. I should
therefore get a good result if I make an approximation with only 4 neutrons and
protons allowed in the $0d\frac52$ orbital and 2 neutrons and protons in the
$0d\frac32$ and $0f\frac72$ orbitals. To test this, I ran a simulation with
reduced $0p\frac32-0d\frac32$ model space, 4 neutrons and protons in the
$0d\frac52$ orbital and 2 neutrons and protons in the $0d\frac32$ orbital. For
$^{15}$C and $^{16}$C I reduced the $0p\frac32-0f\frac72$ model space further
by allowing 0 protons to excite to the $0f\frac72$ orbital. The result is shown
in figure \ref{fig:15B}. We see that the energy spectrum for the reduced model
space is very similar to the energy spectrum for the full model space. The
reduced $0p\frac32-0f\frac72$ model space mentioned above should therefore be a
good approximation. I will also only calculate the first four states for
odd and even parity, as this will also reduce the computing time.

I expected that the $0f\frac72$ orbital would have small impact on the
interaction, as there would be a very low number of particles in it, however,
it proved to have a significant impact on the energy spectra. See figures
\ref{fig:14C}, \ref{fig:15C}, \ref{fig:15B} and \ref{fig:16C}. The figures
shows that the excitation energies had not converged yet for the
$0p\frac32-0d\frac32$ model space. The $1p\frac32$ orbital should be included
in our model space to see if we have convergence. However the model space will
then be too big to calculate with the present version of the shell-model code and within the 
time limts of a Master of Science thesis.

We shall now look closer at our results, starting with $^{14}$C.

\section{$^{14}$C}

There have been done many calculations on $^{14}$C. Most have used a
model space based solely on the $0p$ shell, looking at a two-proton hole
situation. With careful adjustments of the interaction, one can obtain a 
good correspondence with the experimental states with positive parity. (The
excitations cannot have negative parity as they only excite within the $0p$
shell and thus do not change $l$-value for a two-hole state.) Calculations mixing the $0p$ and
$1s0d$ shells have been difficult to do. Our results will hopefully show if one
needs to include the $1s0d$ shell, and possibly the $0f1p$ shell to get states
corresponding to the experimental states, or if it is enough to only calculate
in the $0p$ shell, without adjusting the interaction to fit the experimental
values.

We see in figure \ref{fig:14C} that we predict several of the experimental
states, most notably the first $2^+$ state. The energy correspondence of these
states to the experimental states is not good for the $0p\frac32-0d\frac32$
model space. For the $0p\frac32-0f\frac72$ model space it gets considerably
better, though we have too few states to say anything conclusive about this
model space.

Another thing to note is that the orbital occupancy in the
$0p\frac32-0f\frac72$ model space is almost identical to the orbital occupancy
in the $0p\frac32-0d\frac32$ model space, see figures
\ref{fig:14C_g_0hf_3pert_0d3_4part_brown_0}-\ref{fig:14C_g_0hf_3pert_0f7_2part_brown_1}))
and tables \ref{C14_0d3_p}-\ref{C14_0f7_n}). The orbital occupancies also shows
that $^{14}$C is ruled mostly by proton excitations, except for the second
$2^+$ state.

\subsection{$0p\frac32-0d\frac32$ model space}

Figure \ref{fig:14C} shows the energy spectra for $^{14}$C. We will look at the
energy spectrum in the $0p\frac32-0d\frac32$ model space here. Two things
come to our notice. One being that there are some experimental
states that we have not predicted, such as the first $0^+$ and $4^+$ states.
The other is that our states have too high energy, especially for the
high-lying states. It is difficult to say what this high energy difference
comes from, as there are several things that might cause this. The nucleus $^{14}$C has a
high amount of resonance dependent states. We cannot hope to predict these
states, as our effective interaction is derived using a harmonic oscillator basis. We also do not take into
account Center of Mass Corrections. The amount of spurious center of mass contamination has not been
evaluated in this thesis and needs to be done prior to an eventual publication of these results.

In our ground state (see figure \ref{fig:14C_g_0hf_3pert_0d3_4part_brown_0}) we
have roughly 3.5 protons in the $0p\frac32$ orbital and 0.5 protons in the
$0d\frac52$ orbital. For neutrons we have roughly 3.5 neutrons in the $0p\frac32$
orbital, 2 neutrons in the $0p\frac12$ orbital and 0.5 protons in the $0d\frac52$
orbital. We see that this is very close to our single particle picture of
$^{14}$C's ground state \ref{modellrom1}. 

%Our first excited state, the $1^-$ state, corresponds to the experimental first
%excited state state, though its energy is almost 1 MeV lower than experimental
%state. This state is not ruled by proton excitations, but instead have an
%equal number of neutrons and protons exciting mainly from the $0p\frac32$ orbital
%to the $0d\frac52$ orbital. (See figure
%\ref{fig:14C_g_0hf_3pert_0d3_4part_brown_1}) This state is unique among the 6
%excited states with negative parity I predict, in that it is the only one that
%is that is not ruled by neutron excitations.

The first $2^+$ state corresponds to the experimental first $2^+$ state, though
we do not get a good energy correspondence, with the energy being about 0.5 MeV
above the experimental value. The excitation is mainly a one-neutron
excitation from the $0p\frac32$ orbital to the $0p\frac12$ orbital. (See figure
\ref{fig:14C_g_0hf_3pert_0d3_4part_brown_1}.)

For the $1^+$ state we see that we have two experimental states. 
We have one state where they believe the angular momentum to be
greater than or equal to 1 (they do not know its parity). Its energy is about
0.7 MeV (6.6\%) lower than our predicted state. There is another state with
angular momentum $1^+$ that lies roughly 0.2 MeV (1.5\%) higher than our
predicted state. This is quite near our state.


We do not have the first $0^+$ state nor the $4^+$ and higher-lying states. Our
predicted second $2^+$ state, $0^+$ state and third $2^+$ state seems to
correspond with the experimental second $2^+$ state, second $0^+$ state and
third $2^+$ state, though our states lie roughly 6-7 MeV (over 50\%) above the
experimental states. Our three states are close to each other though, with an
equal energy gap between them, which the experimental states also have.
Because of this I believe our second $2^+$, first $0^+$ and third $2^+$ states
to correspond to the experimental second $2^+$, second $0^+$ and third $2^+$
states. 

%The first $3^-$, first $2^-$ and second $2^-$ states seems to correspond with
%the experimental first $3^-$, first $2^-$ and second $2^-$ states, though there
%is a huge gap in energy between our states and the experimental states. For the
%second and third $1^-$ states it is difficult to say which of the tree possible
%experimental states they correspond to, as the experimental states lie very
%close to each other in energy.
%
%Figures \ref{fig:14C_g_0hf_3pert_0d3_4part_brown_0} -
%\ref{fig:14C_g_0hf_3pert_0d3_4part_brown_3} shows the sp orbitals occupancy of
%$^{14}$C in the $0p\frac32-0d\frac32$ model space, calculated using the
%$G$-matrix method, without Hartree-Fock. One thing we note from the figures is
%that, except for the first excited state, with angular momentum $1^-$, the
%excited states are dominated by proton excitations, that go mostly from the
%$0p\frac32$ orbital to the $0p\frac12$ orbital, in accordance with our
%predictions.
%
%We see that the occupancy of the ground state is roughly as follows: 3.3
%protons in the $0p\frac32$ orbital and 0.7 protons divided mostly among the
%$0p\frac12$ and $0d\frac52$ orbitals. For neutrons we have 3.5 neutrons in the
%$0p\frac32$ orbital, 1.8 in the $0p\frac12$ orbital, 0.5 in the $0d\frac52$ orbital
%and 0.3 in the $0d\frac32$ orbital.\\ For the first excited state of $^{14}$C,
%with angular momentum $1^-$, we have roughly 0.2 protons that excite from the
%$0p\frac32$ orbital, mainly to the $0d\frac52$ orbital. For neutrons we have
%roughly a 0.4 particle 0.4 hole excitation from the $0p$ shell to the $0d$
%shell. The second excited state of $^{14}$C, with angular momentum $2^+$ has
%an excitation of roughly 0.7 protons from the $0p\frac32$ orbital and 0.1
%protons from the $0d\frac52$ orbital, mainly to the $0p\frac12$ orbital. For the
%third excited state of $^{14}$C, with angular momentum $1^+$, we have a very
%similar orbital occupancy as for the second excited state of $^{14}$C, with
%angular momentum $2^+$, but with slightly less particles being excited.\\ We
%see that the excited states of $^{14}$C mainly consists of neutron excitations
%from the $0p\frac32$ orbital to the $0p\frac12$ orbital.

\subsection{$0p\frac32-0f\frac72$ model space}

We will now look at the energy spectrum in the $0p\frac32-0f\frac72$
model space shown in figure \ref{fig:14C}.

We see that the first $2^+$ state is now very close to the experimental state,
with an excitation energy difference of roughly 0.03 (0.14\%).

The $1^+$ state is almost unchanged, which means that we most likely have a
convergence for this state. Again I cannot say for sure which experimental
$1^+$ state it corresponds to, though the experimental $1^+$ state at 11.306
MeV seems likely, as it is much closer to our $1^+$ state.

The second $2^+$ state has had its energy lowered by roughly 3.6 MeV, and is now
considerably closer to both the second and third experimental $2^+$ states.
From our discussion about the $0p\frac32-0d\frac32$ model space this state most
likely corresponds to the second experimental $2^+$ state. We see that the
orbital occupation for this state is actually not dominated by proton
excitations but have an almost equal number of proton and neutron excitations,
going mostly from the $0p\frac32$ orbital to the $0d\frac52$ orbital. This state is
the only one of the five excited states with positive parity I have predicted that
is not ruled by proton excitations

In conclusion I want to say that the $0p\frac32-0d\frac32$ model space is
enough to predict several of the experimental states, but we do not get a good
energy correlation. The $0p\frac32-07\frac72$ model space seems to considerably
improve the reproduction of the experimental results, but we have too few states to be able to draw
a conclusion about how much of an improvement the $0p\frac32-07\frac72$ model
space is.

It should be enough with the $0p$ shell to predict most of the states. To get
the second $2^+$ state, however, one needs to include the $1s0d$ shell in one's
calculations.

\begin{figure}[htbp]
\setlength{\unitlength}{3.0cm}
\begin{center}
\begin{picture}(6.9,6.5)(0,-1)
\psset{xunit=1.00cm, yunit=6.0cm}
\newcommand{\drawlevel}[5]{\psline[origin={#1,#2}, linewidth=0.3pt](0,0)(1.5,0.0)\rput(#3,#4){\scriptsize \makebox(0,0){$#5$}}}
\newcommand{\connect}[4]{\psline[linewidth=0.3pt, linestyle=dotted, dotsep=1.2pt](#1,#2)(#3,#4)}
\thicklines
\rput(0.1,-0.2){\makebox(0,0){{\large 0p3-0d3}}}
\drawlevel{0.7}{-0}{2}{0}{\ 0^+ \ (0)}
\drawlevel{0.7}{-0.844025}{2}{0.844025}{\ 2^+ \ (7.54^)}
\drawlevel{0.7}{-1.24609}{2}{1.24609}{\ 1^+ \ (11.14^)}
\drawlevel{0.7}{-1.65866}{2}{1.65866}{\ 2^+ \ (14.83^)}
\drawlevel{0.7}{-1.79772}{2}{1.79772}{\ 0^+ \ (16.07^)}
\drawlevel{0.7}{-2}{2}{2}{\ 2^+ \ (17.88^)}
\connect{0.8}{2}{1.1}{2}

\rput(4.3,-0.2){\makebox(0,0){{\large 0p3-0f7}}}
\drawlevel{-3.5}{-0}{6.2}{0}{\ 0^+ \ (0)}
\drawlevel{-3.5}{-0.780148}{6.2}{0.780148}{\ 2^+ \ (6.97^)}
\drawlevel{-3.5}{-1.2524}{6.2}{1.2524}{\ 1^+ \ (11.20^)}
\connect{5}{1.2524}{5.3}{1.2524}
\drawlevel{-3.5}{-1.25674}{6.2}{1.3224}{\ 2^+ \ (11.23^)}
\connect{5}{1.25674}{5.3}{1.3224}

\rput(8.5,-0.2){\makebox(0,0){{\large exp. data}}}
\drawlevel{-7.7}{-0}{10.4}{0}{\ 0^+ \ (0)}
\drawlevel{-7.7}{-0.737156}{10.4}{0.737156}{\ 0^+ \ (6.59^)}
\connect{9.2}{0.737156}{9.5}{0.737156}
\drawlevel{-7.7}{-0.784432}{10.4}{0.807156}{\ 2^+ \ (7.01^)}
\connect{9.2}{0.784432}{9.5}{0.807156}
\drawlevel{-7.7}{-0.930523}{10.4}{0.930523}{\ 2^+ \ (8.32^)}
\drawlevel{-7.7}{-1.09028}{10.4}{1.09028}{\ 0^+ \ (9.75^)}
\drawlevel{-7.7}{-1.16624}{10.4}{1.16624}{\ 2^+ \ (10.43^)}
\connect{9.2}{1.16624}{9.5}{1.16624}
\drawlevel{-7.7}{-1.16893}{10.4}{1.23624}{\ GE1 \ (10.45^)}
\connect{9.2}{1.16893}{9.5}{1.23624}
\drawlevel{-7.7}{-1.20104}{10.4}{1.30624}{\ 4^+ \ (10.74^)}
\connect{9.2}{1.20104}{9.5}{1.30624}
\drawlevel{-7.7}{-1.2648}{10.4}{1.37624}{\ 1^+ \ (11.31^)}
\connect{9.2}{1.2648}{9.5}{1.37624}

\end{picture}
\end{center}
\caption{Energy spectra for 14C, in the full 0p3-0d3 and reduced 0p3-0f7 model spaces. By reduced 0p3-0f7 model space we mean that we have reduced our 0p3-0f7 model space so that we have a maximum of 4 particles in the 0d5 orbital, a maximum of 2 particles in the 0d3 orbital and a maximum of 2 particles in the 0f7 orbital. The energies are given in MeV.}
\label{fig:14C}
\end{figure}




\begin{figure}[htbp]
\setlength{\unitlength}{1.0cm}
\begin{center}
\begin{picture}(0,5)(0,-1)
\put(0,4.0){\makebox(0,0){\large nr of particles}}
\thicklines
\put(0,0){\line(0,1){3.8}}
\multiput(0,.0)(0,1){4}{\line(1,0){.1}}
\multiput(0,.5)(0,1){4}{\line(1,0){.05}}
\put(0.2,3){\makebox(0,0){3}}
\put(0.2,2){\makebox(0,0){2}}
\put(0.2,1){\makebox(0,0){1}}
\put(-4,0){\line(1,0){8}}
\put(-3.6,0){\line(0,1){3.337}}
\put(-3.6,3.337){\line(1,0){0.2}}
\put(-3.4,0){\line(0,1){3.337}}
\put(0.4,3.474){\line(1,0){0.2}}
\put(0.4,0){\line(0,1){3.474}}
\put(0.6,0){\line(0,1){3.474}}
\put(-3.5,-0.2){\makebox(0,0){{ 0p$\frac{3}{2}$}}}
\put(0.5,-0.2){\makebox(0,0){{ 0p$\frac{3}{2}$}}}
\put(-2.9,0){\line(0,1){0.166}}
\put(-2.9,0.166){\line(1,0){0.2}}
\put(-2.7,0){\line(0,1){0.166}}
\put(1.1,1.784){\line(1,0){0.2}}
\put(1.1,0){\line(0,1){1.784}}
\put(1.3,0){\line(0,1){1.784}}
\put(-2.8,-0.2){\makebox(0,0){{ 0p$\frac{1}{2}$}}}
\put(1.2,-0.2){\makebox(0,0){{ 0p$\frac{1}{2}$}}}
\put(-2.2,0){\line(0,1){0.403}}
\put(-2.2,0.403){\line(1,0){0.2}}
\put(-2,0){\line(0,1){0.403}}
\put(1.8,0.448){\line(1,0){0.2}}
\put(1.8,0){\line(0,1){0.448}}
\put(2,0){\line(0,1){0.448}}
\put(-2.1,-0.2){\makebox(0,0){{ 0d$\frac{5}{2}$}}}
\put(1.9,-0.2){\makebox(0,0){{ 0d$\frac{5}{2}$}}}
\put(-1.5,0){\line(0,1){0.018}}
\put(-1.5,0.018){\line(1,0){0.2}}
\put(-1.3,0){\line(0,1){0.018}}
\put(2.5,0.024){\line(1,0){0.2}}
\put(2.5,0){\line(0,1){0.024}}
\put(2.7,0){\line(0,1){0.024}}
\put(-1.4,-0.2){\makebox(0,0){{ 1s$\frac{1}{2}$}}}
\put(2.6,-0.2){\makebox(0,0){{ 1s$\frac{1}{2}$}}}
\put(-0.8,0){\line(0,1){0.077}}
\put(-0.8,0.077){\line(1,0){0.2}}
\put(-0.6,0){\line(0,1){0.077}}
\put(3.2,0.27){\line(1,0){0.2}}
\put(3.2,0){\line(0,1){0.270}}
\put(3.4,0){\line(0,1){0.270}}
\put(-0.7,-0.2){\makebox(0,0){{ 0d$\frac{3}{2}$}}}
\put(3.3,-0.2){\makebox(0,0){{ 0d$\frac{3}{2}$}}}
\put(-1.75,-.7){\makebox(0,0){\large Protons}}
\put(1.75,-.7){\makebox(0,0){\large Neutrons}}
\end{picture}
\end{center}
\caption{Orbital occupation for the groundstate state of $^{14}C$ with orbital momentum $J = 0^+$, in the 0p3-0d3 model space.}
\label{fig:14C_g_0hf_3pert_0d3_4part_brown_0}
\end{figure}

%\begin{figure}[htbp]
%\setlength{\unitlength}{1.0cm}
%\begin{center}
%\begin{picture}(0,5)(0,-3)
%\put(0,2.0){\makebox(0,0){\large nr of particles}}
%\thicklines
%\put(0,-2){\line(0,1){3.8}}
%\multiput(0,-2.0)(0,1){4}{\line(1,0){.1}}
%\multiput(0,-1.5)(0,1){4}{\line(1,0){.05}}
%\put(0.2,1){\makebox(0,0){1}}
%\put(0.2,-1){\makebox(0,0){-1}}
%\put(0.2,-2){\makebox(0,0){-2}}
%\put(-4,0){\line(1,0){8}}
%\put(-3.6,0){\line(0,-1){0.221}}
%\put(-3.6,-0.221){\line(1,0){0.2}}
%\put(-3.4,0){\line(0,-1){0.221}}
%\put(0.4,-0.259){\line(1,0){0.2}}
%\put(0.4,0){\line(0,-1){0.259}}
%\put(0.6,0){\line(0,-1){0.259}}
%\put(-3.5,-0.2){\makebox(0,0){{ 0p$\frac{3}{2}$}}}
%\put(0.5,-0.2){\makebox(0,0){{ 0p$\frac{3}{2}$}}}
%\put(-2.9,0){\line(0,1){0.01}}
%\put(-2.9,0.01){\line(1,0){0.2}}
%\put(-2.7,0){\line(0,1){0.01}} 
%\put(1.1,-0.111){\line(1,0){0.2}}
%\put(1.1,0){\line(0,-1){0.111}}
%\put(1.3,0){\line(0,-1){0.111}}
%\put(-2.8,-0.2){\makebox(0,0){{ 0p$\frac{1}{2}$}}}
%\put(1.2,-0.2){\makebox(0,0){{ 0p$\frac{1}{2}$}}}
%\put(-2.2,0){\line(0,1){0.147}}
%\put(-2.2,0.147){\line(1,0){0.2}}
%\put(-2,0){\line(0,1){0.147}}
%\put(1.8,0.211){\line(1,0){0.2}} 
%\put(1.8,0){\line(0,1){0.211}}
%\put(2,0){\line(0,1){0.211}}
%\put(-2.1,-0.2){\makebox(0,0){{ 0d$\frac{5}{2}$}}}
%\put(1.9,-0.2){\makebox(0,0){{ 0d$\frac{5}{2}$}}}
%\put(-1.5,0){\line(0,1){0.025}}
%\put(-1.5,0.025){\line(1,0){0.2}}
%\put(-1.3,0){\line(0,1){0.025}}
%\put(2.5,0.026){\line(1,0){0.2}}
%\put(2.5,0){\line(0,1){0.026}}
%\put(2.7,0){\line(0,1){0.026}}
%\put(-1.4,-0.2){\makebox(0,0){{ 1s$\frac{1}{2}$}}}
%\put(2.6,-0.2){\makebox(0,0){{ 1s$\frac{1}{2}$}}}
%\put(-0.8,0){\line(0,1){0.038}}
%\put(-0.8,0.038){\line(1,0){0.2}}
%\put(-0.6,0){\line(0,1){0.038}}
%\put(3.2,0.132){\line(1,0){0.2}}
%\put(3.2,0){\line(0,1){0.132}}
%\put(3.4,0){\line(0,1){0.132}}
%\put(-0.7,-0.2){\makebox(0,0){{ 0d$\frac{3}{2}$}}}
%\put(3.3,-0.2){\makebox(0,0){{ 0d$\frac{3}{2}$}}}
%\put(-1.75,-2.7){\makebox(0,0){\large Protons}}
%\put(1.75,-2.7){\makebox(0,0){\large Neutrons}}
%\end{picture}
%\end{center}
%\caption{Orbital occupation difference from the ground state for the 1. excited state of $^{14}C$ with orbital momentum $J = 1^-$, in the 0p3-0d3 model\-space and 3. order perturbation. Calculated using the G-matrix method, no Hartree-Fock, using the orbital-energys for $^{16}$O given by B.A. Brown and E.K. Warburton in their article in Physical rewiev C,V46,Nr3.}
%\label{fig:14C_g_0hf_3pert_0d3_4part_brown_1}
%\end{figure}

\begin{figure}[htbp]
\setlength{\unitlength}{1.0cm}
\begin{center}
\begin{picture}(0,5)(0,-3)
\put(0,2.0){\makebox(0,0){\large nr of particles}}
\thicklines
\put(0,-2){\line(0,1){3.8}}
\multiput(0,-2.0)(0,1){4}{\line(1,0){.1}}
\multiput(0,-1.5)(0,1){4}{\line(1,0){.05}}
\put(0.2,1){\makebox(0,0){1}}
\put(0.2,-1){\makebox(0,0){-1}}
\put(0.2,-2){\makebox(0,0){-2}}
\put(-4,0){\line(1,0){8}}
\put(-3.6,0){\line(0,-1){0.739}}
\put(-3.6,-0.739){\line(1,0){0.2}}
\put(-3.4,0){\line(0,-1){0.739}}
\put(0.4,0.016){\line(1,0){0.2}}
\put(0.4,0){\line(0,1){0.016}}
\put(0.6,0){\line(0,1){0.016}}
\put(-3.5,-0.2){\makebox(0,0){{ 0p$\frac{3}{2}$}}}
\put(0.5,-0.2){\makebox(0,0){{ 0p$\frac{3}{2}$}}}
\put(-2.9,0){\line(0,1){0.782}}
\put(-2.9,0.782){\line(1,0){0.2}}
\put(-2.7,0){\line(0,1){0.782}}
\put(1.1,0.018){\line(1,0){0.2}}
\put(1.1,0){\line(0,1){0.018}}
\put(1.3,0){\line(0,1){0.018}}
\put(-2.8,-0.2){\makebox(0,0){{ 0p$\frac{1}{2}$}}}
\put(1.2,-0.2){\makebox(0,0){{ 0p$\frac{1}{2}$}}}
\put(-2.2,0){\line(0,-1){0.099}}
\put(-2.2,-0.099){\line(1,0){0.2}}
\put(-2,0){\line(0,-1){0.099}}
\put(1.8,-0.009){\line(1,0){0.2}}
\put(1.8,0){\line(0,-1){0.009}}
\put(2,0){\line(0,-1){0.009}}
\put(-2.1,-0.2){\makebox(0,0){{ 0d$\frac{5}{2}$}}}
\put(1.9,-0.2){\makebox(0,0){{ 0d$\frac{5}{2}$}}}
\put(-1.5,0){\line(0,-1){0.001}}
\put(-1.5,-0.001){\line(1,0){0.2}}
\put(-1.3,0){\line(0,-1){0.001}}
\put(2.5,0.001){\line(1,0){0.2}}
\put(2.5,0){\line(0,1){0.001}}
\put(2.7,0){\line(0,1){0.001}}
\put(-1.4,-0.2){\makebox(0,0){{ 1s$\frac{1}{2}$}}}
\put(2.6,-0.2){\makebox(0,0){{ 1s$\frac{1}{2}$}}}
\put(-0.8,0){\line(0,1){0.055}}
\put(-0.8,0.055){\line(1,0){0.2}}
\put(-0.6,0){\line(0,1){0.055}}
\put(3.2,-0.026){\line(1,0){0.2}}
\put(3.2,0){\line(0,-1){0.026}}
\put(3.4,0){\line(0,-1){0.026}}
\put(-0.7,-0.2){\makebox(0,0){{ 0d$\frac{3}{2}$}}}
\put(3.3,-0.2){\makebox(0,0){{ 0d$\frac{3}{2}$}}}
\put(-1.75,-2.7){\makebox(0,0){\large Protons}}
\put(1.75,-2.7){\makebox(0,0){\large Neutrons}}
\end{picture}
\end{center}
\caption{Orbital occupation difference from the ground state for the first $2^+$ state of $^{14}$C, in the 0p3-0d3 model space.}
\label{fig:14C_g_0hf_3pert_0d3_4part_brown_1}
\end{figure}

%\begin{figure}[htbp]
%\setlength{\unitlength}{1.0cm} 
%\begin{center}
%\begin{picture}(0,5)(0,-3)
%\put(0,2.0){\makebox(0,0){\large nr of particles}}
%\thicklines
%\put(0,-2){\line(0,1){3.8}}
%\multiput(0,-2.0)(0,1){4}{\line(1,0){.1}}
%\multiput(0,-1.5)(0,1){4}{\line(1,0){.05}}
%\put(0.2,1){\makebox(0,0){1}}
%\put(0.2,-1){\makebox(0,0){-1}}
%\put(0.2,-2){\makebox(0,0){-2}}
%\put(-4,0){\line(1,0){8}}
%\put(-3.6,0){\line(0,-1){0.703}}
%\put(-3.6,-0.703){\line(1,0){0.2}}
%\put(-3.4,0){\line(0,-1){0.703}}
%\put(0.4,0.026){\line(1,0){0.2}}
%\put(0.4,0){\line(0,1){0.026}}
%\put(0.6,0){\line(0,1){0.026}}
%\put(-3.5,-0.2){\makebox(0,0){{ 0p$\frac{3}{2}$}}}
%\put(0.5,-0.2){\makebox(0,0){{ 0p$\frac{3}{2}$}}}
%\put(-2.9,0){\line(0,1){0.743}}
%\put(-2.9,0.743){\line(1,0){0.2}}
%\put(-2.7,0){\line(0,1){0.743}}
%\put(1.1,0.02){\line(1,0){0.2}}
%\put(1.1,0){\line(0,1){0.02}}
%\put(1.3,0){\line(0,1){0.02}}
%\put(-2.8,-0.2){\makebox(0,0){{ 0p$\frac{1}{2}$}}}
%\put(1.2,-0.2){\makebox(0,0){{ 0p$\frac{1}{2}$}}}
%\put(-2.2,0){\line(0,-1){0.09}}
%\put(-2.2,-0.09){\line(1,0){0.2}}
%\put(-2,0){\line(0,-1){0.09}}
%\put(1.8,-0.017){\line(1,0){0.2}}
%\put(1.8,0){\line(0,-1){0.017}}
%\put(2,0){\line(0,-1){0.017}}
%\put(-2.1,-0.2){\makebox(0,0){{ 0d$\frac{5}{2}$}}}
%\put(1.9,-0.2){\makebox(0,0){{ 0d$\frac{5}{2}$}}}
%\put(-1.5,0){\line(0,1){0}}
%\put(-1.5,0){\line(1,0){0.2}}
%\put(-1.3,0){\line(0,1){0}}
%\put(2.5,-0.001){\line(1,0){0.2}}
%\put(2.5,0){\line(0,-1){0.001}}
%\put(2.7,0){\line(0,-1){0.001}}
%\put(-1.4,-0.2){\makebox(0,0){{ 1s$\frac{1}{2}$}}}
%\put(2.6,-0.2){\makebox(0,0){{ 1s$\frac{1}{2}$}}}
%\put(-0.8,0){\line(0,1){0.049}}
%\put(-0.8,0.049){\line(1,0){0.2}}
%\put(-0.6,0){\line(0,1){0.049}}
%\put(3.2,-0.029){\line(1,0){0.2}}
%\put(3.2,0){\line(0,-1){0.029}}
%\put(3.4,0){\line(0,-1){0.029}}
%\put(-0.7,-0.2){\makebox(0,0){{ 0d$\frac{3}{2}$}}}
%\put(3.3,-0.2){\makebox(0,0){{ 0d$\frac{3}{2}$}}}
%\put(-1.75,-2.7){\makebox(0,0){\large Protons}}
%\put(1.75,-2.7){\makebox(0,0){\large Neutrons}}
%\end{picture}
%\end{center}
%\caption{Orbital occupation difference from the ground state for the 3. excited state of $^{14}C$ with orbital momentum $J = 1$, in the 0p3-0d3 model\-space and 3. order perturbation. Calculated using the G-matrix method, no Hartree-Fock, using the orbital-energys for $^{16}$O given by B.A. Brown and E.K. Warburton in their article in Physical rewiev C,V46,Nr3.}
%\label{fig:14C_g_0hf_3pert_0d3_4part_brown_3}
%\end{figure}

\clearpage

\begin{figure}[htbp]
\setlength{\unitlength}{1.0cm}
\begin{center}
\begin{picture}(0,5)(0,-1)
\put(0,4.0){\makebox(0,0){\large nr of particles}}
\thicklines
\put(0,0){\line(0,1){3.8}}
\multiput(0,.0)(0,1){4}{\line(1,0){.1}}
\multiput(0,.5)(0,1){4}{\line(1,0){.05}}
\put(0.2,3){\makebox(0,0){3}}
\put(0.2,2){\makebox(0,0){2}}
\put(0.2,1){\makebox(0,0){1}}
\put(-4.7,0){\line(1,0){9.4}}
\put(-4.3,0){\line(0,1){3.176}}
\put(-4.3,3.176){\line(1,0){0.2}}
\put(-4.1,0){\line(0,1){3.176}}
\put(0.4,3.309){\line(1,0){0.2}}
\put(0.4,0){\line(0,1){3.309}}
\put(0.6,0){\line(0,1){3.309}}
\put(-4.2,-0.2){\makebox(0,0){{ 0p$\frac{3}{2}$}}}
\put(0.5,-0.2){\makebox(0,0){{ 0p$\frac{3}{2}$}}}
\put(-3.6,0){\line(0,1){0.210}}
\put(-3.6,0.21){\line(1,0){0.2}}
\put(-3.4,0){\line(0,1){0.210}}
\put(1.1,1.761){\line(1,0){0.2}}
\put(1.1,0){\line(0,1){1.761}}
\put(1.3,0){\line(0,1){1.761}}
\put(-3.5,-0.2){\makebox(0,0){{ 0p$\frac{1}{2}$}}}
\put(1.2,-0.2){\makebox(0,0){{ 0p$\frac{1}{2}$}}}
\put(-2.9,0){\line(0,1){0.473}}
\put(-2.9,0.473){\line(1,0){0.2}}
\put(-2.7,0){\line(0,1){0.473}}
\put(1.8,0.557){\line(1,0){0.2}}
\put(1.8,0){\line(0,1){0.557}}
\put(2,0){\line(0,1){0.557}}
\put(-2.8,-0.2){\makebox(0,0){{ 0d$\frac{5}{2}$}}}
\put(1.9,-0.2){\makebox(0,0){{ 0d$\frac{5}{2}$}}}
\put(-2.2,0){\line(0,1){0.018}}
\put(-2.2,0.018){\line(1,0){0.2}}
\put(-2,0){\line(0,1){0.018}}
\put(2.5,0.03){\line(1,0){0.2}}
\put(2.5,0){\line(0,1){0.030}}
\put(2.7,0){\line(0,1){0.030}}
\put(-2.1,-0.2){\makebox(0,0){{ 1s$\frac{1}{2}$}}}
\put(2.6,-0.2){\makebox(0,0){{ 1s$\frac{1}{2}$}}}
\put(-1.5,0){\line(0,1){0.093}}
\put(-1.5,0.093){\line(1,0){0.2}}
\put(-1.3,0){\line(0,1){0.093}}
\put(3.2,0.284){\line(1,0){0.2}}
\put(3.2,0){\line(0,1){0.284}}
\put(3.4,0){\line(0,1){0.284}}
\put(-1.4,-0.2){\makebox(0,0){{ 0d$\frac{3}{2}$}}}
\put(3.3,-0.2){\makebox(0,0){{ 0d$\frac{3}{2}$}}}
\put(-0.8,0){\line(0,1){0.029}}
\put(-0.8,0.029){\line(1,0){0.2}}
\put(-0.6,0){\line(0,1){0.029}}
\put(3.9,0.058){\line(1,0){0.2}}
\put(3.9,0){\line(0,1){0.058}}
\put(4.1,0){\line(0,1){0.058}}
\put(-0.7,-0.2){\makebox(0,0){{ 0f$\frac{7}{2}$}}}
\put(4,-0.2){\makebox(0,0){{ 0f$\frac{7}{2}$}}}
\put(-2.1,-.7){\makebox(0,0){\large Protons}}
\put(2.1,-.7){\makebox(0,0){\large Neutrons}}
\end{picture}
\end{center}
\caption{Orbital occupation for the ground state of $^{14}C$ with orbital momentum $J = 0^+$, in the 0p3-0f7 model space.}
\label{fig:14C_g_0hf_3pert_0f7_2part_brown_0}
\end{figure}

\begin{figure}[htbp]
\setlength{\unitlength}{1.0cm}
\begin{center}
\begin{picture}(0,5)(0,-3)
\put(0,2.0){\makebox(0,0){\large nr of particles}}
\thicklines
\put(0,-2){\line(0,1){3.8}}
\multiput(0,-2.0)(0,1){4}{\line(1,0){.1}}
\multiput(0,-1.5)(0,1){4}{\line(1,0){.05}}
\put(0.2,1){\makebox(0,0){1}}
\put(0.2,-1){\makebox(0,0){-1}}
\put(0.2,-2){\makebox(0,0){-2}}
\put(-4.7,0){\line(1,0){9.4}}
\put(-4.3,0){\line(0,-1){0.642}}
\put(-4.3,-0.642){\line(1,0){0.2}}
\put(-4.1,0){\line(0,-1){0.642}}
\put(0.4,-0.043){\line(1,0){0.2}}
\put(0.4,0){\line(0,-1){0.043}}
\put(0.6,0){\line(0,-1){0.043}}
\put(-4.2,-0.2){\makebox(0,0){{ 0p$\frac{3}{2}$}}}
\put(0.5,-0.2){\makebox(0,0){{ 0p$\frac{3}{2}$}}}
\put(-3.6,0){\line(0,1){0.637}}
\put(-3.6,0.637){\line(1,0){0.2}}
\put(-3.4,0){\line(0,1){0.637}}
\put(1.1,-0.003){\line(1,0){0.2}}
\put(1.1,0){\line(0,-1){0.003}}
\put(1.3,0){\line(0,-1){0.003}}
\put(-3.5,-0.2){\makebox(0,0){{ 0p$\frac{1}{2}$}}}
\put(1.2,-0.2){\makebox(0,0){{ 0p$\frac{1}{2}$}}}
\put(-2.9,0){\line(0,-1){0.064}}
\put(-2.9,-0.064){\line(1,0){0.2}}
\put(-2.7,0){\line(0,-1){0.064}}
\put(1.8,0.028){\line(1,0){0.2}}
\put(1.8,0){\line(0,1){0.028}}
\put(2,0){\line(0,1){0.028}}
\put(-2.8,-0.2){\makebox(0,0){{ 0d$\frac{5}{2}$}}}
\put(1.9,-0.2){\makebox(0,0){{ 0d$\frac{5}{2}$}}}
\put(-2.2,0){\line(0,1){0.002}}
\put(-2.2,0.002){\line(1,0){0.2}}
\put(-2,0){\line(0,1){0.002}}
\put(2.5,0.007){\line(1,0){0.2}}
\put(2.5,0){\line(0,1){0.007}}
\put(2.7,0){\line(0,1){0.007}}
\put(-2.1,-0.2){\makebox(0,0){{ 1s$\frac{1}{2}$}}}
\put(2.6,-0.2){\makebox(0,0){{ 1s$\frac{1}{2}$}}}
\put(-1.5,0){\line(0,1){0.061}}
\put(-1.5,0.061){\line(1,0){0.2}}
\put(-1.3,0){\line(0,1){0.061}}
\put(3.2,-0.002){\line(1,0){0.2}}
\put(3.2,0){\line(0,-1){0.002}}
\put(3.4,0){\line(0,-1){0.002}}
\put(-1.4,-0.2){\makebox(0,0){{ 0d$\frac{3}{2}$}}}
\put(3.3,-0.2){\makebox(0,0){{ 0d$\frac{3}{2}$}}}
\put(-0.8,0){\line(0,1){0.007}}
\put(-0.8,0.007){\line(1,0){0.2}}
\put(-0.6,0){\line(0,1){0.007}}
\put(3.9,0.013){\line(1,0){0.2}}
\put(3.9,0){\line(0,1){0.013}}
\put(4.1,0){\line(0,1){0.013}}
\put(-0.7,-0.2){\makebox(0,0){{ 0f$\frac{7}{2}$}}}
\put(4,-0.2){\makebox(0,0){{ 0f$\frac{7}{2}$}}}
\put(-2.1,-2.7){\makebox(0,0){\large Protons}}
\put(2.1,-2.7){\makebox(0,0){\large Neutrons}}
\end{picture}
\end{center}
\caption{Orbital occupation difference from the ground state for the first$ 2^+$ state in $^{14}$C, in the 0p3-0f7 model space.}
\label{fig:14C_g_0hf_3pert_0f7_2part_brown_1}
\end{figure}

%\begin{figure}[htbp]
%\setlength{\unitlength}{1.0cm}
%\begin{center}
%\begin{picture}(0,5)(0,-3)
%\put(0,2.0){\makebox(0,0){\large nr of particles}}
%\thicklines
%\put(0,-2){\line(0,1){3.8}}
%\multiput(0,-2.0)(0,1){4}{\line(1,0){.1}}
%\multiput(0,-1.5)(0,1){4}{\line(1,0){.05}}
%\put(0.2,1){\makebox(0,0){1}}
%\put(0.2,-1){\makebox(0,0){-1}}
%\put(0.2,-2){\makebox(0,0){-2}}
%\put(-4.7,0){\line(1,0){9.4}}
%\put(-4.3,0){\line(0,-1){0.635}}
%\put(-4.3,-0.635){\line(1,0){0.2}}
%\put(-4.1,0){\line(0,-1){0.635}}
%\put(0.4,0.057){\line(1,0){0.2}}
%\put(0.4,0){\line(0,1){0.057}}
%\put(0.6,0){\line(0,1){0.057}}
%\put(-4.2,-0.2){\makebox(0,0){{ 0p$\frac{3}{2}$}}}
%\put(0.5,-0.2){\makebox(0,0){{ 0p$\frac{3}{2}$}}}
%\put(-3.6,0){\line(0,1){0.693}}
%\put(-3.6,0.693){\line(1,0){0.2}}
%\put(-3.4,0){\line(0,1){0.693}}
%\put(1.1,0.021){\line(1,0){0.2}}
%\put(1.1,0){\line(0,1){0.021}}
%\put(1.3,0){\line(0,1){0.021}}
%\put(-3.5,-0.2){\makebox(0,0){{ 0p$\frac{1}{2}$}}}
%\put(1.2,-0.2){\makebox(0,0){{ 0p$\frac{1}{2}$}}}
%\put(-2.9,0){\line(0,-1){0.103}}
%\put(-2.9,-0.103){\line(1,0){0.2}}
%\put(-2.7,0){\line(0,-1){0.103}}
%\put(1.8,-0.032){\line(1,0){0.2}}
%\put(1.8,0){\line(0,-1){0.032}}
%\put(2,0){\line(0,-1){0.032}}
%\put(-2.8,-0.2){\makebox(0,0){{ 0d$\frac{5}{2}$}}}
%\put(1.9,-0.2){\makebox(0,0){{ 0d$\frac{5}{2}$}}}
%\put(-2.2,0){\line(0,1){0.004}}
%\put(-2.2,0.004){\line(1,0){0.2}}
%\put(-2,0){\line(0,1){0.004}}
%\put(2.5,-0.001){\line(1,0){0.2}}
%\put(2.5,0){\line(0,-1){0.001}}
%\put(2.7,0){\line(0,-1){0.001}}
%\put(-2.1,-0.2){\makebox(0,0){{ 1s$\frac{1}{2}$}}}
%\put(2.6,-0.2){\makebox(0,0){{ 1s$\frac{1}{2}$}}}
%\put(-1.5,0){\line(0,1){0.048}}
%\put(-1.5,0.048){\line(1,0){0.2}}
%\put(-1.3,0){\line(0,1){0.048}}
%\put(3.2,-0.029){\line(1,0){0.2}}
%\put(3.2,0){\line(0,-1){0.029}}
%\put(3.4,0){\line(0,-1){0.029}}
%\put(-1.4,-0.2){\makebox(0,0){{ 0d$\frac{3}{2}$}}}
%\put(3.3,-0.2){\makebox(0,0){{ 0d$\frac{3}{2}$}}}
%\put(-0.8,0){\line(0,-1){0.005}}
%\put(-0.8,-0.005){\line(1,0){0.2}}
%\put(-0.6,0){\line(0,-1){0.005}}
%\put(3.9,-0.014){\line(1,0){0.2}}
%\put(3.9,0){\line(0,-1){0.014}}
%\put(4.1,0){\line(0,-1){0.014}}
%\put(-0.7,-0.2){\makebox(0,0){{ 0f$\frac{7}{2}$}}}
%\put(4,-0.2){\makebox(0,0){{ 0f$\frac{7}{2}$}}}
%\put(-2.1,-2.7){\makebox(0,0){\large Protons}}
%\put(2.1,-2.7){\makebox(0,0){\large Neutrons}}
%\end{picture}
%\end{center}
%\caption{Orbital occupation difference from the ground state for the 2. excited state of $^{14}C$ with orbital momentum $J = 1$, in the 0p3-0f7 model\-space and 3. order perturbation. Calculated using the G-matrix method, no Hartree-Fock, using the orbital-energys for $^{16}$O given by B.A. Brown and E.K. Warburton in their article in Physical rewiev C,V46,Nr3.}
%\label{fig:14C_g_0hf_3pert_0f7_2part_brown_2}
%\end{figure}
%
%\begin{figure}[htbp]
%\setlength{\unitlength}{1.0cm}
%\begin{center}
%\begin{picture}(0,5)(0,-3)
%\put(0,2.0){\makebox(0,0){\large nr of particles}}
%\thicklines
%\put(0,-2){\line(0,1){3.8}}
%\multiput(0,-2.0)(0,1){4}{\line(1,0){.1}}
%\multiput(0,-1.5)(0,1){4}{\line(1,0){.05}}
%\put(0.2,1){\makebox(0,0){1}}
%\put(0.2,-1){\makebox(0,0){-1}}
%\put(0.2,-2){\makebox(0,0){-2}}
%\put(-4.7,0){\line(1,0){9.4}}
%\put(-4.3,0){\line(0,-1){0.502}}
%\put(-4.3,-0.502){\line(1,0){0.2}}
%\put(-4.1,0){\line(0,-1){0.502}}
%\put(0.4,-0.502){\line(1,0){0.2}}
%\put(0.4,0){\line(0,-1){0.502}}
%\put(0.6,0){\line(0,-1){0.502}}
%\put(-4.2,-0.2){\makebox(0,0){{ 0p$\frac{3}{2}$}}}
%\put(0.5,-0.2){\makebox(0,0){{ 0p$\frac{3}{2}$}}}
%\put(-3.6,0){\line(0,1){0.117}}
%\put(-3.6,0.117){\line(1,0){0.2}}
%\put(-3.4,0){\line(0,1){0.117}}
%\put(1.1,-0.164){\line(1,0){0.2}}
%\put(1.1,0){\line(0,-1){0.164}}
%\put(1.3,0){\line(0,-1){0.164}}
%\put(-3.5,-0.2){\makebox(0,0){{ 0p$\frac{1}{2}$}}}
%\put(1.2,-0.2){\makebox(0,0){{ 0p$\frac{1}{2}$}}}
%\put(-2.9,0){\line(0,1){0.231}}
%\put(-2.9,0.231){\line(1,0){0.2}}
%\put(-2.7,0){\line(0,1){0.231}}
%\put(1.8,0.37){\line(1,0){0.2}}
%\put(1.8,0){\line(0,1){0.37}}
%\put(2,0){\line(0,1){0.37}}
%\put(-2.8,-0.2){\makebox(0,0){{ 0d$\frac{5}{2}$}}}
%\put(1.9,-0.2){\makebox(0,0){{ 0d$\frac{5}{2}$}}}
%\put(-2.2,0){\line(0,1){0.051}}
%\put(-2.2,0.051){\line(1,0){0.2}}
%\put(-2,0){\line(0,1){0.051}}
%\put(2.5,0.043){\line(1,0){0.2}}
%\put(2.5,0){\line(0,1){0.043}}
%\put(2.7,0){\line(0,1){0.043}}
%\put(-2.1,-0.2){\makebox(0,0){{ 1s$\frac{1}{2}$}}}
%\put(2.6,-0.2){\makebox(0,0){{ 1s$\frac{1}{2}$}}}
%\put(-1.5,0){\line(0,1){0.056}}
%\put(-1.5,0.056){\line(1,0){0.2}}
%\put(-1.3,0){\line(0,1){0.056}}
%\put(3.2,0.175){\line(1,0){0.2}}
%\put(3.2,0){\line(0,1){0.175}}
%\put(3.4,0){\line(0,1){0.175}}
%\put(-1.4,-0.2){\makebox(0,0){{ 0d$\frac{3}{2}$}}}
%\put(3.3,-0.2){\makebox(0,0){{ 0d$\frac{3}{2}$}}}
%\put(-0.8,0){\line(0,1){0.048}}
%\put(-0.8,0.048){\line(1,0){0.2}}
%\put(-0.6,0){\line(0,1){0.048}}
%\put(3.9,0.08){\line(1,0){0.2}}
%\put(3.9,0){\line(0,1){0.08}}
%\put(4.1,0){\line(0,1){0.08}}
%\put(-0.7,-0.2){\makebox(0,0){{ 0f$\frac{7}{2}$}}}
%\put(4,-0.2){\makebox(0,0){{ 0f$\frac{7}{2}$}}}
%\put(-2.1,-2.7){\makebox(0,0){\large Protons}}
%\put(2.1,-2.7){\makebox(0,0){\large Neutrons}}
%\end{picture}
%\end{center}
%\caption{Orbital occupation difference from the ground state for the 3. excited state of $^{14}C$ with orbital momentum $J = 2$, in the 0p3-0f7 model\-space and 3. order perturbation. Calculated using the G-matrix method, no Hartree-Fock, using the orbital-energys for $^{16}$O given by B.A. Brown and E.K. Warburton in their article in Physical rewiev C,V46,Nr3.}
%\label{fig:14C_g_0hf_3pert_0f7_2part_brown_3}
%\end{figure}

\clearpage

\begin{table}
\begin{center}
\begin{tabular}{|c|c|c|c|c|c|}
	\hline
	Protons & $0p\frac32$ & $0p\frac12$ & $0d\frac52$ & $1s\frac12$ & $0d\frac32$ \\
	\hline
	Ground state $J=0^+$ & 3.34 & 0.17 & 0.40 & 0.02 & 0.08 \\
	\hline
	First $2^+$ state & -0.74 & +0.79 & -0.10 & 0.00 & +0.05 \\
	\hline
	First $1^+$ state & -0.71 & +0.75 & -0.09 & 0.00 & +0.05 \\
	\hline
	Second $2^+$ state & -0.48 & +0.03 & +0.31 & +0.07 & +0.07 \\
	\hline
\end{tabular}
\caption{Proton orbital occupancy for $^{14}$C in the $0p\frac32-0d\frac32$ model space. The horizontal line gives the orbitals, and the vertical line gives the excited states. For the excited state the diagram shows the orbital occupancy difference from the ground state.}
\label{C14_0d3_p}
\end{center} 
\end{table}

\begin{table}
\begin{center}
\begin{tabular}{|c|c|c|c|c|c|}
	\hline
	Neutrons & $0p\frac32$ & $0p\frac12$ & $0d\frac52$ & $1s\frac12$ & $0d\frac32$  \\
	\hline
	Ground state $J=0^+$ & 3.47 & 1.79 & 0.45 & 0.02 & 0.27 \\
	\hline
	First $2^+$ state & +0.02 & +0.01 & -0.01 & +0.01 & -0.03 \\
	\hline
	First $1^+$ state & +0.03 & +0.02 & -0.02 & 0.00 & -0.03 \\
	\hline
	Second $2^+$ state & -0.56 & -0.26 & +0.46 & +0.07 & +0.29 \\
	\hline
\end{tabular}
\caption{Neutron orbital occupancy for $^{14}$C in the $0p\frac32-0d\frac32$ model space. The horizontal line gives the orbitals, and the vertical line gives the excited states. For the excited state the diagram shows the orbital occupancy difference from the ground state.}
\label{C14_0d3_n}
\end{center}
\end{table}

\begin{table}
\begin{center}
\begin{tabular}{|c|c|c|c|c|c|c|}
	\hline
	Protons & $0p\frac32$ & $0p\frac12$ & $0d\frac52$ & $1s\frac12$ & $0d\frac32$ & $0f\frac72$  \\
	\hline
	Ground state $J=0^+$ & 3.18 & 0.21 & 0.47 & 0.02 & 0.09 & 0.03 \\
	\hline
	First $2^+$ state & -0.64 & +0.64 & -0.06 & 0.00 & +0.06 & +0.01 \\
	\hline
	First $1^+$ state & -0.64 & +0.69 & -0.10 & 0.00 & +0.05 & -0.01 \\
	\hline
	Second $2^+$ state & -0.51 & +0.12 & +0.23 & +0.05 & +0.06 & +0.05 \\
	\hline
\end{tabular}
\caption{Proton orbital occupancy for $^{14}$C in the $0p\frac32-0f\frac72$ model space. The horizontal line gives the orbitals, and the vertical line gives the excited states. For the excited state the diagram shows the orbital occupancy difference from the ground state.}
\label{C14_0f7_p}
\end{center}
\end{table}

\begin{table}
\begin{center}
\begin{tabular}{|c|c|c|c|c|c|c|}
	\hline
	Neutrons & $0p\frac32$ & $0p\frac12$ & $0d\frac52$ & $1s\frac12$ & $0d\frac32$ & $0f\frac72$  \\
	\hline
	Ground state $J=0^+$ & 3.31 & 1.77 & 0.55 & 0.03 & 0.28 & 0.06 \\
	\hline
	First $2^+$ state & -0.04 & -0.01 & +0.03 & +0.01 & 0.00 & +0.01 \\
	\hline
	First $1^+$ state & +0.06 & +0.01 & -0.03 & 0.00 & -0.02 & -0.02 \\
	\hline
	Second $2^+$ state & -0.51 & -0.17 & +0.38 & +0.03 & +0.18 & +0.08 \\
	\hline
\end{tabular}
\caption{Neutron orbital occupancy for $^{14}$C in the $0p\frac32-0f\frac72$ model space. The horizontal line gives the orbitals, and the vertical line gives the excited states. For the excited state the diagram shows the orbital occupancy difference from the ground state.}
\label{C14_0f7_n}
\end{center}
\end{table}

\clearpage

%
%Figure \ref{fig:14C} shows the sp orbitals occupancy of
%$^{14}$C in the $0p\frac32-0f\frac72$ model space. The energy spectrum given
%here is in much better agreement with the experimental data. I have here only
%calculated the states with positive parity. The reasons for this is that I am
%mainly interested in the states with positive parities, as I will later look at
%the $E2^+$ transition for this nuclei. It is also because of the large amount
%of time it takes to do one calculation in the $0p\frac32-0f\frac72$
%model space. The energy difference between the experimental and our first $2^+$
%state is about 0.04 MeV. For the first $1^+$ state the difference is about
%0.46 MeV. This difference is still rather large. Possible causes for this
%energy difference is ??. !Hvilke andre muligheter enn modell-rommet har jeg
%her?! The second $2^+$ state is predicted to lie above the first $1^+$ state.
%According to the experimental data, we only have two more $2^+$ states, and
%both of them lie between the first $2^+$ and $1^+$ states. I do not know which
%of these states our $2^+$ state refers to. There is also a large energy
%difference between them, that can be explained by ??.
%
%Figures \ref{fig:14C_g_0hf_3pert_0f7_2part_brown_0} -
%\ref{fig:14C_g_0hf_3pert_0f7_2part_brown_3} shows the sp orbitals occupancy of
%$^{14}$C in the $0p\frac32-0f\frac72$ model space, calculated using the
%$G$-matrix method, without HF. We see that the occupancy of the ground state
%and the first $2^+$ excited state is very similar to the corresponding states
%of $^{14}$C in the $0p\frac32-0d\frac32$ model space. For protons we have: 3.2
%protons in the $0p\frac32$ orbital, 0.2 in the $0p\frac12$ orbital, 0.5 protons in
%the $0d\frac52$ orbitals and 0.1 in the $0f\frac72$ orbital. For neutrons we have:
%3.3 neutrons in the $0p\frac32$ orbital, 1.8 in the $0p\frac12$ orbital, 0.6
%neutrons in the $0d\frac52$ orbital, 0.3 neutrons in the $0d\frac32$ orbital and
%0.1 neutrons spread out between the $1s\frac12$ and $0f\frac72$ orbitals.\\
%For the first $2^+$ excited state of $^{14}$C, we have roughly a 0.6 protons
%that excite from the $0p\frac32$ orbital to the $0p\frac12$ orbital. There is very
%little change in the neutron occupancy.

\section{$^{15}$C}

For $^{15}$C we see from the orbital occupancy figures
\ref{fig:15C_g_0hf_3pert_0f7_2part_brown_0} and
\ref{fig:15C_g_0hf_3pert_0f7_2part_brown_1}, and the orbital occupancy tables
\ref{C15_0f7-p} and \ref{C15_0f7_n} that this nucleus is governed mostly by the
excitation of one neutron from the $1s\frac12$ orbital to the $0d\frac52$
orbital. This means that the physics we have for $^{15}$C is different from
that of $^{14}$C. It is interesting that the same Hamiltonian can predict
these different behaviors. The neutron dominance is also in agreement with our
expectations. Due to time limits we have only done calculations in the
$0p\frac32-0f\frac72$ model space for $^{15}$C.


\subsection{$0p\frac32-0f\frac72$ model space}

Figure \ref{fig:15C} shows the energy spectra for $^{15}$C in the
$0p\frac32-0f\frac72$ model space. From the figure we see that our energy
spectrum corresponds somewhat with the experimental data. We have a
corresponding experimental state for most of our excited states, the exception
being our $\frac72^-$ state.

For the ground state (see figure \ref{fig:15C_g_0hf_3pert_0f7_2part_brown_0})
we see that the proton configuration is almost identical to that for $^{14}$C
and $^{16}$C. For the neutrons we have almost one neutron more in the
$1s\frac12$ orbital compared to the ground state of $^{14}$C. $^{16}$C has
nearly 1 neutron more in the $0d\frac52$ orbital, and a slight amount more in the
$0f\frac72$ orbital. This shows that a Slater determinant based on an occupancy of the same  
orbitals as $^{14}$C plus one neutron is a good approximation,
as when we add (subtract) a nucleon to (from) the nucleus, most of it will enter
(leave) one orbital, instead of being widely scattered among several orbitals.

One point of interest here is that the neutron added to $^{14}$C goes to the
$1s\frac12$ orbital instead of going to the $0d\frac52$ orbital, which means
that the $0d\frac52$ orbital lies above the $1s\frac12$ orbital in energy. This
reflects experimental findings for $^{15}$C, where the first experimental state
has angular momentum $\frac12^+$, meaning that the valence neutron outside the
$0p$ shell lies in the $1s\frac12$ orbital. Also, the experimental first
excited state has angular momentum $\frac52^+$. The ordering of both the
experimental and our $0d\frac52$ and $1s\frac12$ orbitals are the opposite of
the ordering of our sp orbitals. It is good to see that our program manages to
get the same angular momenta as the experimental ground state, even though our
sp orbital does not have the same order.

For the excited states I have mostly neutron excitations from the $1s\frac12$
orbital to the $0d\frac52$ orbital (see figure
\ref{fig:15C_g_0hf_3pert_0f7_2part_brown_1}). This is in  accordance with
our observations of $^{16}$C, where the added neutron goes to the $0d\frac52$
orbital.

For the first $\frac52^+$ state, we have almost a 1p1h neutron excitation from
the $1s\frac12$ to the $0d\frac52$ orbital.
This excitation has, according to
theory, an angular momentum of either $\frac52-\frac12 = 2$ or $\frac52+\frac12
= 3$, and a parity shift of $\pi = (-1)^2 = 1$. This is in accordance with our
predicted angular momentum and parity of $\frac52^+$.
This excited state corresponds with
the first experimental $\frac52^+$ state, and has and energy difference of
0.225 MeV, or 30.3\% less than the experimental value.

Our $\frac72^-$ state might be the experimental $\frac72$ state with energy
6.449 MeV, but the energy difference is quite large, being approximately 3.76
MeV. We also see that it has a large excitation to the $0f\frac72$ orbital,
having almost a 1p1h neutron excitation from the $1s\frac12$ orbital that
spreads evenly among the $0d\frac52$ and $0f\frac72$ orbitals. Something that
might explain part of this value is that, since this state has an excitation of
half a neutron to the $0f\frac72$ orbital, increasing the number of possible
neutron excitations to 3 or more could increase the number of excited neutrons
by a significant amount. However, I do not think that this will be enough to
bring the $\frac72^-$ state up to the energy level of the experimental
$\frac72$ state at 6.417 MeV. It may also be that the energy of this state is
correct, and it has simply not been measured yet.

Another way to study the properties of these excited states is to perform a seniority analysis and study 
the quasi-particle content  of these excited states. 


The $\frac32^-$ and $\frac52^-$ states appear in the opposite order of the
corresponding experimental states, and is not too far from the experimental
states in energy. For the $\frac32^-$ state we have about 0.5 neutrons that excite from the $0p\frac32$ and $1s\frac12$ orbitals to the $0d\frac52$ orbital.

Our $\frac12^-$ state also appears in an unexpected place. When we look at the
ordering of the states, it seems to correspond better with the second
experimental $\frac12^-$ state. However, the ordering of the states here does
not seem to be good, so it might also be the first experimental $\frac12^-$
state.

Our second $\frac52^+$ state corresponds to the $\frac52$ experimental state at
6.358 MeV, having an energy difference of 7.42\% of the experimental value.

Our$\frac72^+$ state most likely corresponds to the first experimental
$\frac72^+$ state, having an energy difference of 2.47\% of the experimental
value.

Our calculations of $^{15}$C does not correspond to the experimental results to
a satisfying degree. It gives an acceptable picture of the experimental states, but
the energy of our theoretical states is in general too low, and we have a
different ordering of the states. This can be caused by the lack of Center of
Mass Corrections, too small model space and too much limitations on the number
of particles allowed to excite up to the higher-lying states (as seems to be
the case with the $\frac72^-$ state). Eventual three-body contribuitions, either effective ones of from
three-nucleon interactions could also influence the description of some of these states.

\begin{figure}[htbp]
\setlength{\unitlength}{3.0cm}
\begin{center}
\begin{picture}(6.9,6.5)(0,-1)
\psset{xunit=1.00cm, yunit=6.0cm}
\newcommand{\drawlevel}[5]{\psline[origin={#1,#2}, linewidth=0.3pt](0,0)(1.5,0.0)\rput(#3,#4){\scriptsize \makebox(0,0){$#5$}}}
\newcommand{\connect}[4]{\psline[linewidth=0.3pt, linestyle=dotted, dotsep=1.2pt](#1,#2)(#3,#4)}
\thicklines
\rput(0.1,-0.2){\makebox(0,0){{\large 0p3-0f7}}}
\drawlevel{0.7}{-0}{2}{0}{\ \frac{1}{2}^+ \ (0)}
\drawlevel{0.7}{-0.166266}{2}{0.166266}{\ \frac{5}{2}^+ \ (0.52^)}
\drawlevel{0.7}{-0.866326}{2}{0.866326}{\ \frac{7}{2}^- \ (2.69^)}
\drawlevel{0.7}{-1.30649}{2}{1.30649}{\ \frac{3}{2}^- \ (4.05^)}
\drawlevel{0.7}{-1.46217}{2}{1.46217}{\ \frac{5}{2}^- \ (4.53^)}
\drawlevel{0.7}{-1.66028}{2}{1.66028}{\ \frac{1}{2}^- \ (5.15^)}
\drawlevel{0.7}{-1.89834}{2}{1.89834}{\ \frac{5}{2}^+ \ (5.89^)}
\drawlevel{0.7}{-2}{2}{2}{\ \frac{7}{2}^+ \ (6.20^)}
\connect{0.8}{2}{1.1}{2}

\rput(4.3,-0.2){\makebox(0,0){{\large exp. data}}}
\drawlevel{-3.5}{-0}{6.2}{0}{\ \frac12^+ \ (0)}
\drawlevel{-3.5}{-0.238675}{6.2}{0.238675}{\ \frac52^+ \ (0.74^)}
\drawlevel{-3.5}{-1.00082}{6.2}{1.00082}{\ \frac12^- \ (3.10^)}
\drawlevel{-3.5}{-1.36109}{6.2}{1.36109}{\ \frac52^- \ (4.22^)}
\drawlevel{-3.5}{-1.50204}{6.2}{1.50204}{\ \frac32^- \ (4.66^)}
\connect{5}{1.50204}{5.3}{1.50204}
\drawlevel{-3.5}{-1.54171}{6.2}{1.57204}{\ \frac32^+ \ (4.78^)}
\connect{5}{1.54171}{5.3}{1.57204}
\drawlevel{-3.5}{-1.88134}{6.2}{1.88134}{\ (\frac32^+) \ (5.83^)}
\connect{5}{1.88134}{5.3}{1.88134}
\drawlevel{-3.5}{-1.89198}{6.2}{1.95134}{\ \frac12^- \ (5.87^)}
\connect{5}{1.89198}{5.3}{1.95134}
\drawlevel{-3.5}{-2.05067}{6.4}{2.05067}{\ (\frac52,\frac72^+,\frac92^+) \ (6.36^)}
\connect{5}{2.05067}{5.3}{2.05067}
\drawlevel{-3.5}{-2.0697}{6.3}{2.12067}{\ (\frac32$TO$\frac72) \ (6.42^)}
\connect{5}{2.0697}{5.3}{2.12067}
\drawlevel{-3.5}{-2.08002}{6.3}{2.19067}{\ (\frac92^-,\frac{11}{2}) \ (6.45^)}
\connect{5}{2.08002}{5.3}{2.19067}

\end{picture}
\end{center}
\caption{Energy spectra for 15C, in the reduced 0p3-0f7 model space. By reduced 0p3-0f7 model space we mean that we have reduced our 0p3-0f7 model space so that we have a maximum of 4 particles in the 0d5 orbital, a maximum of 2 particles in the 0d3 orbital and a maximum of 2 neutrons and 0 protons in the 0f7 orbital. The energies are given in MeV.}
\label{fig:15C}
\end{figure}




\begin{figure}[htbp]
\setlength{\unitlength}{1.0cm}
\begin{center}
\begin{picture}(0,5)(0,-1)
\put(0,4.0){\makebox(0,0){\large nr of particles}}
\thicklines
\put(0,0){\line(0,1){3.8}}
\multiput(0,.0)(0,1){4}{\line(1,0){.1}}
\multiput(0,.5)(0,1){4}{\line(1,0){.05}}
\put(0.2,3){\makebox(0,0){3}}
\put(0.2,2){\makebox(0,0){2}}
\put(0.2,1){\makebox(0,0){1}}
\put(-4.7,0){\line(1,0){9.4}}
\put(-4.3,0){\line(0,1){3.263}}
\put(-4.3,3.263){\line(1,0){0.2}}
\put(-4.1,0){\line(0,1){3.263}}
\put(0.4,3.448){\line(1,0){0.2}}
\put(0.4,0){\line(0,1){3.448}}
\put(0.6,0){\line(0,1){3.448}}
\put(-4.2,-0.2){\makebox(0,0){{ 0p$\frac{3}{2}$}}}
\put(0.5,-0.2){\makebox(0,0){{ 0p$\frac{3}{2}$}}}
\put(-3.6,0){\line(0,1){0.268}}
\put(-3.6,0.268){\line(1,0){0.2}}
\put(-3.4,0){\line(0,1){0.268}}
\put(1.1,1.802){\line(1,0){0.2}}
\put(1.1,0){\line(0,1){1.802}}
\put(1.3,0){\line(0,1){1.802}}
\put(-3.5,-0.2){\makebox(0,0){{ 0p$\frac{1}{2}$}}}
\put(1.2,-0.2){\makebox(0,0){{ 0p$\frac{1}{2}$}}}
\put(-2.9,0){\line(0,1){0.366}}
\put(-2.9,0.366){\line(1,0){0.2}}
\put(-2.7,0){\line(0,1){0.366}}
\put(1.8,0.519){\line(1,0){0.2}}
\put(1.8,0){\line(0,1){0.519}}
\put(2,0){\line(0,1){0.519}}
\put(-2.8,-0.2){\makebox(0,0){{ 0d$\frac{5}{2}$}}}
\put(1.9,-0.2){\makebox(0,0){{ 0d$\frac{5}{2}$}}}
\put(-2.2,0){\line(0,1){0.016}}
\put(-2.2,0.016){\line(1,0){0.2}}
\put(-2,0){\line(0,1){0.016}}
\put(2.5,0.941){\line(1,0){0.2}}
\put(2.5,0){\line(0,1){0.941}}
\put(2.7,0){\line(0,1){0.941}}
\put(-2.1,-0.2){\makebox(0,0){{ 1s$\frac{1}{2}$}}}
\put(2.6,-0.2){\makebox(0,0){{ 1s$\frac{1}{2}$}}}
\put(-1.5,0){\line(0,1){0.086}}
\put(-1.5,0.086){\line(1,0){0.2}}
\put(-1.3,0){\line(0,1){0.086}}
\put(3.2,0.24){\line(1,0){0.2}}
\put(3.2,0){\line(0,1){0.240}}
\put(3.4,0){\line(0,1){0.240}}
\put(-1.4,-0.2){\makebox(0,0){{ 0d$\frac{3}{2}$}}}
\put(3.3,-0.2){\makebox(0,0){{ 0d$\frac{3}{2}$}}}
\put(-0.8,0){\line(0,1){0}}
\put(-0.8,0){\line(1,0){0.2}}
\put(-0.6,0){\line(0,1){0}}
\put(3.9,0.051){\line(1,0){0.2}}
\put(3.9,0){\line(0,1){0.051}}
\put(4.1,0){\line(0,1){0.051}}
\put(-0.7,-0.2){\makebox(0,0){{ 0f$\frac{7}{2}$}}}
\put(4,-0.2){\makebox(0,0){{ 0f$\frac{7}{2}$}}}
\put(-2.1,-.7){\makebox(0,0){\large Protons}}
\put(2.1,-.7){\makebox(0,0){\large Neutrons}}
\end{picture}
\end{center}
\caption{Orbital occupation for the ground state of $^{15}C$ with orbital momentum $J = \frac12$, in the 0p3-0f7 model space.}
\label{fig:15C_g_0hf_3pert_0f7_2part_brown_0}
\end{figure}

\begin{figure}[htbp]
\setlength{\unitlength}{1.0cm}
\begin{center}
\begin{picture}(0,5)(0,-3)
\put(0,2.0){\makebox(0,0){\large nr of particles}}
\thicklines
\put(0,-2){\line(0,1){3.8}}
\multiput(0,-2.0)(0,1){4}{\line(1,0){.1}}
\multiput(0,-1.5)(0,1){4}{\line(1,0){.05}}
\put(0.2,1){\makebox(0,0){1}}
\put(0.2,-1){\makebox(0,0){-1}}
\put(0.2,-2){\makebox(0,0){-2}}
\put(-4.7,0){\line(1,0){9.4}}
\put(-4.3,0){\line(0,-1){0.149}}
\put(-4.3,-0.149){\line(1,0){0.2}}
\put(-4.1,0){\line(0,-1){0.149}}
\put(0.4,-0.023){\line(1,0){0.2}}
\put(0.4,0){\line(0,-1){0.023}}
\put(0.6,0){\line(0,-1){0.023}}
\put(-4.2,-0.2){\makebox(0,0){{ 0p$\frac{3}{2}$}}}
\put(0.5,-0.2){\makebox(0,0){{ 0p$\frac{3}{2}$}}}
\put(-3.6,0){\line(0,1){0.078}}
\put(-3.6,0.078){\line(1,0){0.2}}
\put(-3.4,0){\line(0,1){0.078}}
\put(1.1,-0.033){\line(1,0){0.2}}
\put(1.1,0){\line(0,-1){0.033}}
\put(1.3,0){\line(0,-1){0.033}}
\put(-3.5,-0.2){\makebox(0,0){{ 0p$\frac{1}{2}$}}}
\put(1.2,-0.2){\makebox(0,0){{ 0p$\frac{1}{2}$}}}
\put(-2.9,0){\line(0,1){0.049}}
\put(-2.9,0.049){\line(1,0){0.2}}
\put(-2.7,0){\line(0,1){0.049}}
\put(1.8,0.737){\line(1,0){0.2}}
\put(1.8,0){\line(0,1){0.737}}
\put(2,0){\line(0,1){0.737}}
\put(-2.8,-0.2){\makebox(0,0){{ 0d$\frac{5}{2}$}}}
\put(1.9,-0.2){\makebox(0,0){{ 0d$\frac{5}{2}$}}}
\put(-2.2,0){\line(0,1){0.004}}
\put(-2.2,0.004){\line(1,0){0.2}}
\put(-2,0){\line(0,1){0.004}}
\put(2.5,-0.899){\line(1,0){0.2}}
\put(2.5,0){\line(0,-1){0.899}}
\put(2.7,0){\line(0,-1){0.899}}
\put(-2.1,-0.2){\makebox(0,0){{ 1s$\frac{1}{2}$}}}
\put(2.6,-0.2){\makebox(0,0){{ 1s$\frac{1}{2}$}}}
\put(-1.5,0){\line(0,1){0.019}}
\put(-1.5,0.019){\line(1,0){0.2}}
\put(-1.3,0){\line(0,1){0.019}}
\put(3.2,0.047){\line(1,0){0.2}}
\put(3.2,0){\line(0,1){0.047}}
\put(3.4,0){\line(0,1){0.047}}
\put(-1.4,-0.2){\makebox(0,0){{ 0d$\frac{3}{2}$}}}
\put(3.3,-0.2){\makebox(0,0){{ 0d$\frac{3}{2}$}}}
\put(-0.8,0){\line(0,1){0}}
\put(-0.8,0){\line(1,0){0.2}}
\put(-0.6,0){\line(0,1){0}}
\put(3.9,0.17){\line(1,0){0.2}}
\put(3.9,0){\line(0,1){0.17}}
\put(4.1,0){\line(0,1){0.17}}
\put(-0.7,-0.2){\makebox(0,0){{ 0f$\frac{7}{2}$}}}
\put(4,-0.2){\makebox(0,0){{ 0f$\frac{7}{2}$}}}
\put(-2.1,-2.7){\makebox(0,0){\large Protons}}
\put(2.1,-2.7){\makebox(0,0){\large Neutrons}}
\end{picture}
\end{center}
\caption{Orbital occupation difference from the ground state for the first $\frac52^+$ state of $^{15}C$, in the 0p3-0f7 model space.}
\label{fig:15C_g_0hf_3pert_0f7_2part_brown_1}
\end{figure}

%\begin{figure}[htbp]
%\setlength{\unitlength}{1.0cm}
%\begin{center}
%\begin{picture}(0,5)(0,-3)
%\put(0,2.0){\makebox(0,0){\large nr of particles}}
%\thicklines
%\put(0,-2){\line(0,1){3.8}}
%\multiput(0,-2.0)(0,1){4}{\line(1,0){.1}}
%\multiput(0,-1.5)(0,1){4}{\line(1,0){.05}}
%\put(0.2,1){\makebox(0,0){1}}
%\put(0.2,-1){\makebox(0,0){-1}}
%\put(0.2,-2){\makebox(0,0){-2}}
%\put(-4.7,0){\line(1,0){9.4}}
%\put(-4.3,0){\line(0,-1){0.275}}
%\put(-4.3,-0.275){\line(1,0){0.2}}
%\put(-4.1,0){\line(0,-1){0.275}}
%\put(0.4,-0.052){\line(1,0){0.2}}
%\put(0.4,0){\line(0,-1){0.052}}
%\put(0.6,0){\line(0,-1){0.052}}
%\put(-4.2,-0.2){\makebox(0,0){{ 0p$\frac{3}{2}$}}}
%\put(0.5,-0.2){\makebox(0,0){{ 0p$\frac{3}{2}$}}}
%\put(-3.6,0){\line(0,1){0.117}}
%\put(-3.6,0.117){\line(1,0){0.2}}
%\put(-3.4,0){\line(0,1){0.117}}
%\put(1.1,-0.073){\line(1,0){0.2}}
%\put(1.1,0){\line(0,-1){0.073}}
%\put(1.3,0){\line(0,-1){0.073}}
%\put(-3.5,-0.2){\makebox(0,0){{ 0p$\frac{1}{2}$}}}
%\put(1.2,-0.2){\makebox(0,0){{ 0p$\frac{1}{2}$}}}
%\put(-2.9,0){\line(0,1){0.098}}
%\put(-2.9,0.098){\line(1,0){0.2}}
%\put(-2.7,0){\line(0,1){0.098}}
%\put(1.8,0.453){\line(1,0){0.2}}
%\put(1.8,0){\line(0,1){0.453}}
%\put(2,0){\line(0,1){0.453}}
%\put(-2.8,-0.2){\makebox(0,0){{ 0d$\frac{5}{2}$}}}
%\put(1.9,-0.2){\makebox(0,0){{ 0d$\frac{5}{2}$}}}
%\put(-2.2,0){\line(0,1){0.018}}
%\put(-2.2,0.018){\line(1,0){0.2}}
%\put(-2,0){\line(0,1){0.018}}
%\put(2.5,-0.897){\line(1,0){0.2}}
%\put(2.5,0){\line(0,-1){0.897}}
%\put(2.7,0){\line(0,-1){0.897}}
%\put(-2.1,-0.2){\makebox(0,0){{ 1s$\frac{1}{2}$}}}
%\put(2.6,-0.2){\makebox(0,0){{ 1s$\frac{1}{2}$}}}
%\put(-1.5,0){\line(0,1){0.042}}
%\put(-1.5,0.042){\line(1,0){0.2}}
%\put(-1.3,0){\line(0,1){0.042}}
%\put(3.2,0.091){\line(1,0){0.2}}
%\put(3.2,0){\line(0,1){0.091}}
%\put(3.4,0){\line(0,1){0.091}}
%\put(-1.4,-0.2){\makebox(0,0){{ 0d$\frac{3}{2}$}}}
%\put(3.3,-0.2){\makebox(0,0){{ 0d$\frac{3}{2}$}}}
%\put(-0.8,0){\line(0,1){0}}
%\put(-0.8,0){\line(1,0){0.2}}
%\put(-0.6,0){\line(0,1){0}}
%\put(3.9,0.477){\line(1,0){0.2}}
%\put(3.9,0){\line(0,1){0.477}}
%\put(4.1,0){\line(0,1){0.477}}
%\put(-0.7,-0.2){\makebox(0,0){{ 0f$\frac{7}{2}$}}}
%\put(4,-0.2){\makebox(0,0){{ 0f$\frac{7}{2}$}}}
%\put(-2.1,-2.7){\makebox(0,0){\large Protons}}
%\put(2.1,-2.7){\makebox(0,0){\large Neutrons}}
%\end{picture}
%\end{center}
%\caption{Orbital occupation difference from the ground state for the 2. excited state of $^{15}C$ with orbital momentum $J = 3.5^-$, in the 0p3-0f7 model\-space and 3. order perturbation. Calculated using the G-matrix method, no Hartree-Fock, using the orbital-energys for $^{16}$O given by B.A. Brown and E.K. Warburton in their article in Physical rewiev C,V46,Nr3.}
%\label{fig:15C_g_0hf_3pert_0f7_2part_brown_2}
%\end{figure}
%
%\begin{figure}[htbp]
%\setlength{\unitlength}{1.0cm}
%\begin{center}
%\begin{picture}(0,5)(0,-3)
%\put(0,2.0){\makebox(0,0){\large nr of particles}}
%\thicklines
%\put(0,-2){\line(0,1){3.8}}
%\multiput(0,-2.0)(0,1){4}{\line(1,0){.1}}
%\multiput(0,-1.5)(0,1){4}{\line(1,0){.05}}
%\put(0.2,1){\makebox(0,0){1}}
%\put(0.2,-1){\makebox(0,0){-1}}
%\put(0.2,-2){\makebox(0,0){-2}}
%\put(-4.7,0){\line(1,0){9.4}}
%\put(-4.3,0){\line(0,-1){0.292}}
%\put(-4.3,-0.292){\line(1,0){0.2}}
%\put(-4.1,0){\line(0,-1){0.292}}
%\put(0.4,-0.354){\line(1,0){0.2}}
%\put(0.4,0){\line(0,-1){0.354}}
%\put(0.6,0){\line(0,-1){0.354}}
%\put(-4.2,-0.2){\makebox(0,0){{ 0p$\frac{3}{2}$}}}
%\put(0.5,-0.2){\makebox(0,0){{ 0p$\frac{3}{2}$}}}
%\put(-3.6,0){\line(0,1){0.064}}
%\put(-3.6,0.064){\line(1,0){0.2}}
%\put(-3.4,0){\line(0,1){0.064}}
%\put(1.1,-0.115){\line(1,0){0.2}}
%\put(1.1,0){\line(0,-1){0.115}}
%\put(1.3,0){\line(0,-1){0.115}}
%\put(-3.5,-0.2){\makebox(0,0){{ 0p$\frac{1}{2}$}}}
%\put(1.2,-0.2){\makebox(0,0){{ 0p$\frac{1}{2}$}}}
%\put(-2.9,0){\line(0,1){0.154}}
%\put(-2.9,0.154){\line(1,0){0.2}}
%\put(-2.7,0){\line(0,1){0.154}}
%\put(1.8,0.513){\line(1,0){0.2}}
%\put(1.8,0){\line(0,1){0.513}}
%\put(2,0){\line(0,1){0.513}}
%\put(-2.8,-0.2){\makebox(0,0){{ 0d$\frac{5}{2}$}}}
%\put(1.9,-0.2){\makebox(0,0){{ 0d$\frac{5}{2}$}}}
%\put(-2.2,0){\line(0,1){0.026}}
%\put(-2.2,0.026){\line(1,0){0.2}}
%\put(-2,0){\line(0,1){0.026}}
%\put(2.5,-0.261){\line(1,0){0.2}}
%\put(2.5,0){\line(0,-1){0.261}}
%\put(2.7,0){\line(0,-1){0.261}}
%\put(-2.1,-0.2){\makebox(0,0){{ 1s$\frac{1}{2}$}}}
%\put(2.6,-0.2){\makebox(0,0){{ 1s$\frac{1}{2}$}}}
%\put(-1.5,0){\line(0,1){0.049}}
%\put(-1.5,0.049){\line(1,0){0.2}}
%\put(-1.3,0){\line(0,1){0.049}}
%\put(3.2,0.142){\line(1,0){0.2}}
%\put(3.2,0){\line(0,1){0.142}}
%\put(3.4,0){\line(0,1){0.142}}
%\put(-1.4,-0.2){\makebox(0,0){{ 0d$\frac{3}{2}$}}}
%\put(3.3,-0.2){\makebox(0,0){{ 0d$\frac{3}{2}$}}}
%\put(-0.8,0){\line(0,1){0}}
%\put(-0.8,0){\line(1,0){0.2}}
%\put(-0.6,0){\line(0,1){0}}
%\put(3.9,0.074){\line(1,0){0.2}}
%\put(3.9,0){\line(0,1){0.074}}
%\put(4.1,0){\line(0,1){0.074}}
%\put(-0.7,-0.2){\makebox(0,0){{ 0f$\frac{7}{2}$}}}
%\put(4,-0.2){\makebox(0,0){{ 0f$\frac{7}{2}$}}}
%\put(-2.1,-2.7){\makebox(0,0){\large Protons}}
%\put(2.1,-2.7){\makebox(0,0){\large Neutrons}}
%\end{picture}
%\end{center}
%\caption{Orbital occupation difference from the ground state for the 3. excited state of $^{15}C$ with orbital momentum $J = 1.5^-$, in the 0p3-0f7 model\-space and 3. order perturbation. Calculated using the G-matrix method, no Hartree-Fock, using the orbital-energys for $^{16}$O given by B.A. Brown and E.K. Warburton in their article in Physical rewiev C,V46,Nr3.}
%\label{fig:15C_g_0hf_3pert_0f7_2part_brown_3}
%\end{figure}

\clearpage

\begin{table}
\begin{center}
\begin{tabular}{|c|c|c|c|c|c|c|}
	\hline
	Protons & $0p\frac32$ & $0p\frac12$ & $0d\frac52$ & $1s\frac12$ & $0d\frac32$ & $0f\frac72$ \\
	\hline
	Ground state $J=\frac12^+$ & 3.26 & 0.27 & 0.37 & 0.02 & 0.09 & 0 \\
	\hline
	First $\frac52^+$ state & -0.15 & +0.08 & +0.05 & 0.00 & +0.02 & 0 \\
	\hline
	First $\frac72^-$ state & -0.27 & +0.12 & +0.09 & +0.01 & +0.04 & 0 \\
	\hline
	Second $\frac32^-$ state & -0.29 & +0.06 & +0.15 & +0.02 & +0.05 & 0 \\
	\hline
\end{tabular}
\caption{Proton orbital occupancy for $^{15}$C in the $0p\frac32-0f\frac72$ model space. The horizontal line gives the orbitals, and the vertical line gives the excited states. For the excited state the diagram shows the orbital occupancy difference from the ground state.}
\label{C15_0f7_p}
\end{center}
\end{table}

\begin{table}
\begin{center}
\begin{tabular}{|c|c|c|c|c|c|c|}
	\hline
	Neutrons & $0p\frac32$ & $0p\frac12$ & $0d\frac52$ & $1s\frac12$ & $0d\frac32$ & $0f\frac72$ \\
	\hline
	Ground state $J=\frac12^+$ & 3.45 & 1.80 & 0.52 & 0.94 & 0.24 & 0.05 \\
	\hline
	First $\frac52^+$ state & -0.02 & -0.03 & +0.74 & -0.90 & +0.05 & +0.17 \\
	\hline
	First $\frac72^-$ state & -0.05 & -0.07 & +0.45 & -0.90 & +0.09 & +0.48 \\
	\hline
	Second $\frac32^-$ state & -0.36 & -0.11 & +0.51 & -0.26 & +0.14 & +0.08 \\
	\hline
\end{tabular}
\caption{Neutron orbital occupancy for $^{15}$C in the $0p\frac32-0f\frac72$ model space. The horizontal line gives the orbitals, and the vertical line gives the excited states. For the excited state the diagram shows the orbital occupancy difference from the ground state.}
\label{C15_0f7_n}
\end{center}
\end{table}

\clearpage


\section{$^{15}$B}

For $^{15}$B we have much the same situation as for $^{15}$C, with a nucleus
dominated by neutron excitations from the $1s\frac12$ orbital to the
$0d\frac52$ orbital (see figure \ref{fig:15B_g_0hf_3pert_0f7_2part_brown_1} and
tables \ref{B15_0d3-p}-\ref{B15_0f7_n}). We also get a good correspondence
between our calculated states in the $0p\frac32-0f\frac72$ model space and the
experimental states.

It will be interesting to see if $^{16}$C also follows the expectations
previously held about its excitations, or if it is ruled by other types of
excitations that supports the experimental data on $^{16}$C from the
\citet{16CLifetime} experiment.

\subsection{$0p\frac32-0d\frac32$ model space}

Figure \ref{fig:15B} shows the energy spectra for $^{15}$B. Looking at the
spectrum for $0p\frac32-0d\frac32$ model space we see that we predict both of
the experimental values, but we have a much higher energy on our states, the
first excited state being over 50\% larger than the corresponding experimental
state. We also predict a $\frac32^-$ state above the $\frac72^-$ state.

Figure \ref{fig:15B_g_0hf_3pert_0d3_4part_brown_0} shows the ground state
occupancy of $^{15}$B in the $0p\frac32-0d\frac32$ model space. The ground
state is very similar to the ground state of $^{15}$C, but with only 3 protons
in the $0p\frac32$ orbital and one neutron more that spreads out among all the
available orbitals.

Figure \ref{fig:15B_g_0hf_3pert_0d3_4part_brown_1} shows the sp orbitals
occupancy of the excited states of $^{15}$B in the $0p\frac32-0d\frac32$ model
space. We see that we have almost exclusively neutron excitations from the
$1s\frac12$ orbital to the $0d\frac52$ orbital.

\subsection{$0p\frac32-0f\frac72$ model space}

We now take for us the $0p\frac32-0f\frac72$ model space energy spectrum of
$^{15}$B in figure \ref{fig:15B}. The excited states are now much closer to
their corresponding experimental values.

The ground state (see figure \ref{fig:15B_g_0hf_3pert_0f7_2part_brown_0}) has
now an equal amount of neutrons (roughly 1 neutron) in the $0d\frac52$ and
$1s\frac12$ orbitals. Otherwise it is the same as for the $0p\frac32-0d\frac32$
model space.

The $\frac52^-$ state now lies 0.13 MeV (10.1\%) above the corresponding
experimental state. The $\frac72^-$ state lies 0.47 MeV (17.0\%) above the
corresponding experimental state. These states now have a considerably lower
amount of neutrons being excited from the $1s\frac12$ orbital to the $0d\frac52$
orbital.

The third excited state we predict has changed from being a $\frac32^-$ state
to being a $\frac12^-$ state. This state has an excitation energy of 4.32 MeV.
The state has, in addition to a one neutron excitation from the $1s\frac12$
orbital to the $0d\frac32$ orbital, small proton excitations from the $0p\frac32$
orbital to the $0p\frac12$ orbital and small neutron excitations from the
$1s\frac12$ orbital to the $0f\frac72$ orbital

We predict the two experimental states which have been obtained, though we get too high energy
for our states. The excited states are ruled by neutron excitations from the
$1s\frac12$ orbital to the $0d\frac52$ orbital, in correspondence with our
expectations.

\begin{figure}[htbp]
\setlength{\unitlength}{3.0cm}
\begin{center}
\begin{picture}(6.9,6.5)(0,-1)
\psset{xunit=1.00cm, yunit=6.0cm}
\newcommand{\drawlevel}[5]{\psline[origin={#1,#2}, linewidth=0.3pt](0,0)(1.5,0.0)\rput(#3,#4){\scriptsize \makebox(0,0){$#5$}}}
\newcommand{\connect}[4]{\psline[linewidth=0.3pt, linestyle=dotted, dotsep=1.2pt](#1,#2)(#3,#4)}
\thicklines
\rput(0.1,-0.2){\makebox(0,0){{\large Full 0p3-0d3}}}
\drawlevel{0.7}{-0}{2}{0}{\ \frac{3}{2}^- \ (0)}
\drawlevel{0.7}{-0.986992}{2}{0.986992}{\ \frac{5}{2}^- \ (2.22^)}
\drawlevel{0.7}{-1.76563}{2}{1.76563}{\ \frac{7}{2}^- \ (3.97^)}
\drawlevel{0.7}{-2}{2}{2}{\ \frac{3}{2}^- \ (4.50^)}

\rput(4.3,-0.2){\makebox(0,0){{\large Reduced 0p3-0d3}}}
\drawlevel{-3.5}{-0}{6.2}{0}{\ \frac{3}{2}^- \ (0)}
\drawlevel{-3.5}{-0.991617}{6.2}{0.991617}{\ \frac{5}{2}^- \ (2.23^)}
\drawlevel{-3.5}{-1.77248}{6.2}{1.77248}{\ \frac{7}{2}^- \ (3.99^)}
\drawlevel{-3.5}{-1.9928}{6.2}{1.9928}{\ \frac{3}{2}^- \ (4.48^)}

\rput(8.5,-0.2){\makebox(0,0){{\large 0p3-0f7}}}
\drawlevel{-7.7}{-0}{10.4}{0}{\ \frac{3}{2}^- \ (0)}
\drawlevel{-7.7}{-0.654422}{10.4}{0.654422}{\ \frac{5}{2}^- \ (1.47^)}
\drawlevel{-7.7}{-1.42696}{10.4}{1.42696}{\ \frac{7}{2}^- \ (3.21^)}
\drawlevel{-7.7}{-1.92084}{10.4}{1.92084}{\ \frac{1}{2}^- \ (4.32^)}

\rput(12.7,-0.2){\makebox(0,0){{\large exp. data}}}
\drawlevel{-11.9}{-0}{14.6}{0}{\ \frac{3}{2}^- \ (0)}
\drawlevel{-11.9}{-0.594161}{14.6}{0.594161}{\ \frac{5}{2}^- \ (1.34^)}
\drawlevel{-11.9}{-1.2199}{14.6}{1.2199}{\ \frac{7}{2}^- \ (2.74^)}
\connect{13.4}{1.2199}{13.7}{1.2199}

\end{picture}
\end{center}
\caption{Energy spectra for 15B, in the full 0p3-0d3 and reduced 0p3-0f7 model spaces. Reduced 0p3-0d3 means that we have reduced our 0p3-0d3 model space so that we have a maximum of 4 particles in the 0d5 orbital and a maximum of 2 particles in the 0d3 orbital. Reduced 0p3-0f7 model space we mean that we have reduced our 0p3-0f7 model space so that we have a maximum of 4 particles in the 0d5 orbital, a maximum of 2 particles in the 0d3 orbital and a maximum of 2 particles in the 0f7 orbital. The energies are given in MeV.}
\label{fig:15B}
\end{figure}




\begin{figure}[htbp]
\setlength{\unitlength}{1.0cm}
\begin{center}
\begin{picture}(0,5)(0,-1)
\put(0,4.0){\makebox(0,0){\large nr of particles}}
\thicklines
\put(0,0){\line(0,1){3.8}}
\multiput(0,.0)(0,1){4}{\line(1,0){.1}}
\multiput(0,.5)(0,1){4}{\line(1,0){.05}}
\put(0.2,3){\makebox(0,0){3}}
\put(0.2,2){\makebox(0,0){2}}
\put(0.2,1){\makebox(0,0){1}}
\put(-4,0){\line(1,0){8}}
\put(-3.6,0){\line(0,1){2.569}}
\put(-3.6,2.569){\line(1,0){0.2}}
\put(-3.4,0){\line(0,1){2.569}}
\put(0.4,3.665){\line(1,0){0.2}}
\put(0.4,0){\line(0,1){3.665}}
\put(0.6,0){\line(0,1){3.665}}
\put(-3.5,-0.2){\makebox(0,0){{ 0p$\frac{3}{2}$}}}
\put(0.5,-0.2){\makebox(0,0){{ 0p$\frac{3}{2}$}}}
\put(-2.9,0){\line(0,1){0.199}}
\put(-2.9,0.199){\line(1,0){0.2}}
\put(-2.7,0){\line(0,1){0.199}}
\put(1.1,1.859){\line(1,0){0.2}}
\put(1.1,0){\line(0,1){1.859}}
\put(1.3,0){\line(0,1){1.859}}
\put(-2.8,-0.2){\makebox(0,0){{ 0p$\frac{1}{2}$}}}
\put(1.2,-0.2){\makebox(0,0){{ 0p$\frac{1}{2}$}}}
\put(-2.2,0){\line(0,1){0.187}}
\put(-2.2,0.187){\line(1,0){0.2}}
\put(-2,0){\line(0,1){0.187}}
\put(1.8,0.831){\line(1,0){0.2}}
\put(1.8,0){\line(0,1){0.831}}
\put(2,0){\line(0,1){0.831}}
\put(-2.1,-0.2){\makebox(0,0){{ 0d$\frac{5}{2}$}}}
\put(1.9,-0.2){\makebox(0,0){{ 0d$\frac{5}{2}$}}}
\put(-1.5,0){\line(0,1){0.009}}
\put(-1.5,0.009){\line(1,0){0.2}}
\put(-1.3,0){\line(0,1){0.009}}
\put(2.5,1.427){\line(1,0){0.2}}
\put(2.5,0){\line(0,1){1.427}}
\put(2.7,0){\line(0,1){1.427}}
\put(-1.4,-0.2){\makebox(0,0){{ 1s$\frac{1}{2}$}}}
\put(2.6,-0.2){\makebox(0,0){{ 1s$\frac{1}{2}$}}}
\put(-0.8,0){\line(0,1){0.035}}
\put(-0.8,0.035){\line(1,0){0.2}}
\put(-0.6,0){\line(0,1){0.035}}
\put(3.2,0.219){\line(1,0){0.2}}
\put(3.2,0){\line(0,1){0.219}}
\put(3.4,0){\line(0,1){0.219}}
\put(-0.7,-0.2){\makebox(0,0){{ 0d$\frac{3}{2}$}}}
\put(3.3,-0.2){\makebox(0,0){{ 0d$\frac{3}{2}$}}}
\put(-1.75,-.7){\makebox(0,0){\large Protons}}
\put(1.75,-.7){\makebox(0,0){\large Neutrons}}
\end{picture}
\end{center}
\caption{Orbital occupation for the ground state of $^{15}B$ with orbital momentum $J = \frac32^-$, in the 0p3-0d3 model space.}
\label{fig:15B_g_0hf_3pert_0d3_4part_brown_0}
\end{figure}

\begin{figure}[htbp]
\setlength{\unitlength}{1.0cm}
\begin{center}
\begin{picture}(0,5)(0,-3)
\put(0,2.0){\makebox(0,0){\large nr of particles}}
\thicklines
\put(0,-2){\line(0,1){3.8}}
\multiput(0,-2.0)(0,1){4}{\line(1,0){.1}}
\multiput(0,-1.5)(0,1){4}{\line(1,0){.05}}
\put(0.2,1){\makebox(0,0){1}}
\put(0.2,-1){\makebox(0,0){-1}}
\put(0.2,-2){\makebox(0,0){-2}}
\put(-4,0){\line(1,0){8}}
\put(-3.6,0){\line(0,-1){0.018}}
\put(-3.6,-0.018){\line(1,0){0.2}}
\put(-3.4,0){\line(0,-1){0.018}}
\put(0.4,0.05){\line(1,0){0.2}}
\put(0.4,0){\line(0,1){0.05}}
\put(0.6,0){\line(0,1){0.05}}
\put(-3.5,-0.2){\makebox(0,0){{ 0p$\frac{3}{2}$}}}
\put(0.5,-0.2){\makebox(0,0){{ 0p$\frac{3}{2}$}}}
\put(-2.9,0){\line(0,1){0.043}}
\put(-2.9,0.043){\line(1,0){0.2}}
\put(-2.7,0){\line(0,1){0.043}}
\put(1.1,0.016){\line(1,0){0.2}}
\put(1.1,0){\line(0,1){0.016}}
\put(1.3,0){\line(0,1){0.016}}
\put(-2.8,-0.2){\makebox(0,0){{ 0p$\frac{1}{2}$}}}
\put(1.2,-0.2){\makebox(0,0){{ 0p$\frac{1}{2}$}}}
\put(-2.2,0){\line(0,-1){0.021}}
\put(-2.2,-0.021){\line(1,0){0.2}}
\put(-2,0){\line(0,-1){0.021}}
\put(1.8,0.549){\line(1,0){0.2}}
\put(1.8,0){\line(0,1){0.549}}
\put(2,0){\line(0,1){0.549}}
\put(-2.1,-0.2){\makebox(0,0){{ 0d$\frac{5}{2}$}}}
\put(1.9,-0.2){\makebox(0,0){{ 0d$\frac{5}{2}$}}}
\put(-1.5,0){\line(0,-1){0.002}}
\put(-1.5,-0.002){\line(1,0){0.2}}
\put(-1.3,0){\line(0,-1){0.002}}
\put(2.5,-0.635){\line(1,0){0.2}}
\put(2.5,0){\line(0,-1){0.635}}
\put(2.7,0){\line(0,-1){0.635}}
\put(-1.4,-0.2){\makebox(0,0){{ 1s$\frac{1}{2}$}}}
\put(2.6,-0.2){\makebox(0,0){{ 1s$\frac{1}{2}$}}}
\put(-0.8,0){\line(0,-1){0.001}}
\put(-0.8,-0.001){\line(1,0){0.2}}
\put(-0.6,0){\line(0,-1){0.001}}
\put(3.2,0.019){\line(1,0){0.2}}
\put(3.2,0){\line(0,1){0.019}}
\put(3.4,0){\line(0,1){0.019}}
\put(-0.7,-0.2){\makebox(0,0){{ 0d$\frac{3}{2}$}}}
\put(3.3,-0.2){\makebox(0,0){{ 0d$\frac{3}{2}$}}}
\put(-1.75,-2.7){\makebox(0,0){\large Protons}}
\put(1.75,-2.7){\makebox(0,0){\large Neutrons}}
\end{picture}
\end{center}
\caption{Orbital occupation difference from the ground state for the first $\frac52^-$ state of $^{15}B$, in the 0p3-0d3 model space.}
\label{fig:15B_g_0hf_3pert_0d3_4part_brown_1}
\end{figure}

%\begin{figure}[htbp]
%\setlength{\unitlength}{1.0cm}
%\begin{center}
%\begin{picture}(0,5)(0,-3)
%\put(0,2.0){\makebox(0,0){\large nr of particles}}
%\thicklines
%\put(0,-2){\line(0,1){3.8}}
%\multiput(0,-2.0)(0,1){4}{\line(1,0){.1}}
%\multiput(0,-1.5)(0,1){4}{\line(1,0){.05}}
%\put(0.2,1){\makebox(0,0){1}}
%\put(0.2,-1){\makebox(0,0){-1}}
%\put(0.2,-2){\makebox(0,0){-2}}
%\put(-4,0){\line(1,0){8}}
%\put(-3.6,0){\line(0,-1){0.006}}
%\put(-3.6,-0.006){\line(1,0){0.2}}
%\put(-3.4,0){\line(0,-1){0.006}}
%\put(0.4,0.026){\line(1,0){0.2}}
%\put(0.4,0){\line(0,1){0.026}}
%\put(0.6,0){\line(0,1){0.026}}
%\put(-3.5,-0.2){\makebox(0,0){{ 0p$\frac{3}{2}$}}}
%\put(0.5,-0.2){\makebox(0,0){{ 0p$\frac{3}{2}$}}}
%\put(-2.9,0){\line(0,1){0.01}}
%\put(-2.9,0.01){\line(1,0){0.2}}
%\put(-2.7,0){\line(0,1){0.01}}
%\put(1.1,0.007){\line(1,0){0.2}}
%\put(1.1,0){\line(0,1){0.007}}
%\put(1.3,0){\line(0,1){0.007}}
%\put(-2.8,-0.2){\makebox(0,0){{ 0p$\frac{1}{2}$}}}
%\put(1.2,-0.2){\makebox(0,0){{ 0p$\frac{1}{2}$}}}
%\put(-2.2,0){\line(0,-1){0.002}}
%\put(-2.2,-0.002){\line(1,0){0.2}}
%\put(-2,0){\line(0,-1){0.002}}
%\put(1.8,0.595){\line(1,0){0.2}}
%\put(1.8,0){\line(0,1){0.595}}
%\put(2,0){\line(0,1){0.595}}
%\put(-2.1,-0.2){\makebox(0,0){{ 0d$\frac{5}{2}$}}}
%\put(1.9,-0.2){\makebox(0,0){{ 0d$\frac{5}{2}$}}}
%\put(-1.5,0){\line(0,-1){0.001}}
%\put(-1.5,-0.001){\line(1,0){0.2}}
%\put(-1.3,0){\line(0,-1){0.001}}
%\put(2.5,-0.666){\line(1,0){0.2}}
%\put(2.5,0){\line(0,-1){0.666}}
%\put(2.7,0){\line(0,-1){0.666}}
%\put(-1.4,-0.2){\makebox(0,0){{ 1s$\frac{1}{2}$}}}
%\put(2.6,-0.2){\makebox(0,0){{ 1s$\frac{1}{2}$}}}
%\put(-0.8,0){\line(0,1){0}}
%\put(-0.8,0){\line(1,0){0.2}}
%\put(-0.6,0){\line(0,1){0}}
%\put(3.2,0.037){\line(1,0){0.2}}
%\put(3.2,0){\line(0,1){0.037}}
%\put(3.4,0){\line(0,1){0.037}}
%\put(-0.7,-0.2){\makebox(0,0){{ 0d$\frac{3}{2}$}}}
%\put(3.3,-0.2){\makebox(0,0){{ 0d$\frac{3}{2}$}}}
%\put(-1.75,-2.7){\makebox(0,0){\large Protons}}
%\put(1.75,-2.7){\makebox(0,0){\large Neutrons}}
%\end{picture}
%\end{center}
%\caption{Orbital occupation difference from the ground state for the 2. excited state of $^{15}B$ with orbital momentum $J = 3.5^-$, in the 0p3-0d3 model\-space and 3. order perturbation. Calculated using the G-matrix method, no Hartree-Fock, using the orbital-energys for $^{16}$O given by B.A. Brown and E.K. Warburton in their article in Physical rewiev C,V46,Nr3.}
%\label{fig:15B_g_0hf_3pert_0d3_4part_brown_2}
%\end{figure}
%
%\begin{figure}[htbp]
%\setlength{\unitlength}{1.0cm}
%\begin{center}
%\begin{picture}(0,5)(0,-3)
%\put(0,2.0){\makebox(0,0){\large nr of particles}}
%\thicklines
%\put(0,-2){\line(0,1){3.8}}
%\multiput(0,-2.0)(0,1){4}{\line(1,0){.1}}
%\multiput(0,-1.5)(0,1){4}{\line(1,0){.05}}
%\put(0.2,1){\makebox(0,0){1}}
%\put(0.2,-1){\makebox(0,0){-1}}
%\put(0.2,-2){\makebox(0,0){-2}}
%\put(-4,0){\line(1,0){8}}
%\put(-3.6,0){\line(0,-1){0.017}}
%\put(-3.6,-0.017){\line(1,0){0.2}}
%\put(-3.4,0){\line(0,-1){0.017}}
%\put(0.4,-0.014){\line(1,0){0.2}}
%\put(0.4,0){\line(0,-1){0.014}}
%\put(0.6,0){\line(0,-1){0.014}}
%\put(-3.5,-0.2){\makebox(0,0){{ 0p$\frac{3}{2}$}}}
%\put(0.5,-0.2){\makebox(0,0){{ 0p$\frac{3}{2}$}}}
%\put(-2.9,0){\line(0,1){0.008}}
%\put(-2.9,0.008){\line(1,0){0.2}}
%\put(-2.7,0){\line(0,1){0.008}}
%\put(1.1,-0.006){\line(1,0){0.2}}
%\put(1.1,0){\line(0,-1){0.006}}
%\put(1.3,0){\line(0,-1){0.006}}
%\put(-2.8,-0.2){\makebox(0,0){{ 0p$\frac{1}{2}$}}}
%\put(1.2,-0.2){\makebox(0,0){{ 0p$\frac{1}{2}$}}}
%\put(-2.2,0){\line(0,1){0.008}}
%\put(-2.2,0.008){\line(1,0){0.2}}
%\put(-2,0){\line(0,1){0.008}}
%\put(1.8,0.303){\line(1,0){0.2}}
%\put(1.8,0){\line(0,1){0.303}}
%\put(2,0){\line(0,1){0.303}}
%\put(-2.1,-0.2){\makebox(0,0){{ 0d$\frac{5}{2}$}}}
%\put(1.9,-0.2){\makebox(0,0){{ 0d$\frac{5}{2}$}}}
%\put(-1.5,0){\line(0,1){0.002}}
%\put(-1.5,0.002){\line(1,0){0.2}}
%\put(-1.3,0){\line(0,1){0.002}}
%\put(2.5,-0.282){\line(1,0){0.2}}
%\put(2.5,0){\line(0,-1){0.282}}
%\put(2.7,0){\line(0,-1){0.282}}
%\put(-1.4,-0.2){\makebox(0,0){{ 1s$\frac{1}{2}$}}}
%\put(2.6,-0.2){\makebox(0,0){{ 1s$\frac{1}{2}$}}}
%\put(-0.8,0){\line(0,1){0}}
%\put(-0.8,0){\line(1,0){0.2}}
%\put(-0.6,0){\line(0,1){0}}
%\put(3.2,-0.002){\line(1,0){0.2}}
%\put(3.2,0){\line(0,-1){0.002}}
%\put(3.4,0){\line(0,-1){0.002}}
%\put(-0.7,-0.2){\makebox(0,0){{ 0d$\frac{3}{2}$}}}
%\put(3.3,-0.2){\makebox(0,0){{ 0d$\frac{3}{2}$}}}
%\put(-1.75,-2.7){\makebox(0,0){\large Protons}}
%\put(1.75,-2.7){\makebox(0,0){\large Neutrons}}
%\end{picture}
%\end{center}
%\caption{Orbital occupation difference from the ground state for the 3. excited state of $^{15}B$ with orbital momentum $J = 1.5^-$, in the 0p3-0d3 model\-space and 3. order perturbation. Calculated using the G-matrix method, no Hartree-Fock, using the orbital-energys for $^{16}$O given by B.A. Brown and E.K. Warburton in their article in Physical rewiev C,V46,Nr3.}
%\label{fig:15B_g_0hf_3pert_0d3_4part_brown_3}
%\end{figure}

\clearpage

\begin{figure}[htbp]
\setlength{\unitlength}{1.0cm}
\begin{center}
\begin{picture}(0,5)(0,-1)
\put(0,4.0){\makebox(0,0){\large nr of particles}}
\thicklines
\put(0,0){\line(0,1){3.8}}
\multiput(0,.0)(0,1){4}{\line(1,0){.1}}
\multiput(0,.5)(0,1){4}{\line(1,0){.05}}
\put(0.2,3){\makebox(0,0){3}}
\put(0.2,2){\makebox(0,0){2}}
\put(0.2,1){\makebox(0,0){1}}
\put(-4.7,0){\line(1,0){9.4}}
\put(-4.3,0){\line(0,1){2.429}}
\put(-4.3,2.429){\line(1,0){0.2}}
\put(-4.1,0){\line(0,1){2.429}}
\put(0.4,3.619){\line(1,0){0.2}}
\put(0.4,0){\line(0,1){3.619}}
\put(0.6,0){\line(0,1){3.619}}
\put(-4.2,-0.2){\makebox(0,0){{ 0p$\frac{3}{2}$}}}
\put(0.5,-0.2){\makebox(0,0){{ 0p$\frac{3}{2}$}}}
\put(-3.6,0){\line(0,1){0.264}}
\put(-3.6,0.264){\line(1,0){0.2}}
\put(-3.4,0){\line(0,1){0.264}}
\put(1.1,1.837){\line(1,0){0.2}}
\put(1.1,0){\line(0,1){1.837}}
\put(1.3,0){\line(0,1){1.837}}
\put(-3.5,-0.2){\makebox(0,0){{ 0p$\frac{1}{2}$}}}
\put(1.2,-0.2){\makebox(0,0){{ 0p$\frac{1}{2}$}}}
\put(-2.9,0){\line(0,1){0.223}}
\put(-2.9,0.223){\line(1,0){0.2}}
\put(-2.7,0){\line(0,1){0.223}}
\put(1.8,1.089){\line(1,0){0.2}}
\put(1.8,0){\line(0,1){1.089}}
\put(2,0){\line(0,1){1.089}}
\put(-2.8,-0.2){\makebox(0,0){{ 0d$\frac{5}{2}$}}}
\put(1.9,-0.2){\makebox(0,0){{ 0d$\frac{5}{2}$}}}
\put(-2.2,0){\line(0,1){0.010}}
\put(-2.2,0.01){\line(1,0){0.2}}
\put(-2,0){\line(0,1){0.010}}
\put(2.5,1.05){\line(1,0){0.2}}
\put(2.5,0){\line(0,1){1.050}}
\put(2.7,0){\line(0,1){1.050}}
\put(-2.1,-0.2){\makebox(0,0){{ 1s$\frac{1}{2}$}}}
\put(2.6,-0.2){\makebox(0,0){{ 1s$\frac{1}{2}$}}}
\put(-1.5,0){\line(0,1){0.057}}
\put(-1.5,0.057){\line(1,0){0.2}}
\put(-1.3,0){\line(0,1){0.057}}
\put(3.2,0.231){\line(1,0){0.2}}
\put(3.2,0){\line(0,1){0.231}}
\put(3.4,0){\line(0,1){0.231}}
\put(-1.4,-0.2){\makebox(0,0){{ 0d$\frac{3}{2}$}}}
\put(3.3,-0.2){\makebox(0,0){{ 0d$\frac{3}{2}$}}}
\put(-0.8,0){\line(0,1){0.017}}
\put(-0.8,0.017){\line(1,0){0.2}}
\put(-0.6,0){\line(0,1){0.017}}
\put(3.9,0.174){\line(1,0){0.2}}
\put(3.9,0){\line(0,1){0.174}}
\put(4.1,0){\line(0,1){0.174}}
\put(-0.7,-0.2){\makebox(0,0){{ 0f$\frac{7}{2}$}}}
\put(4,-0.2){\makebox(0,0){{ 0f$\frac{7}{2}$}}}
\put(-2.1,-.7){\makebox(0,0){\large Protons}}
\put(2.1,-.7){\makebox(0,0){\large Neutrons}}
\end{picture}
\end{center}
\caption{Orbital occupation for the ground state of $^{15}B$ with orbital momentum $J = \frac32^-$, in the 0p3-0f7 model space.}
\label{fig:15B_g_0hf_3pert_0f7_2part_brown_0}
\end{figure}

\begin{figure}[htbp]
\setlength{\unitlength}{1.0cm}
\begin{center}
\begin{picture}(0,5)(0,-3)
\put(0,2.0){\makebox(0,0){\large nr of particles}}
\thicklines
\put(0,-2){\line(0,1){3.8}}
\multiput(0,-2.0)(0,1){4}{\line(1,0){.1}}
\multiput(0,-1.5)(0,1){4}{\line(1,0){.05}}
\put(0.2,1){\makebox(0,0){1}}
\put(0.2,-1){\makebox(0,0){-1}}
\put(0.2,-2){\makebox(0,0){-2}}
\put(-4.7,0){\line(1,0){9.4}}
\put(-4.3,0){\line(0,-1){0.049}}
\put(-4.3,-0.049){\line(1,0){0.2}}
\put(-4.1,0){\line(0,-1){0.049}}
\put(0.4,-0.018){\line(1,0){0.2}}
\put(0.4,0){\line(0,-1){0.018}}
\put(0.6,0){\line(0,-1){0.018}}
\put(-4.2,-0.2){\makebox(0,0){{ 0p$\frac{3}{2}$}}}
\put(0.5,-0.2){\makebox(0,0){{ 0p$\frac{3}{2}$}}}
\put(-3.6,0){\line(0,1){0.024}}
\put(-3.6,0.024){\line(1,0){0.2}}
\put(-3.4,0){\line(0,1){0.024}}
\put(1.1,-0.013){\line(1,0){0.2}}
\put(1.1,0){\line(0,-1){0.013}}
\put(1.3,0){\line(0,-1){0.013}}
\put(-3.5,-0.2){\makebox(0,0){{ 0p$\frac{1}{2}$}}}
\put(1.2,-0.2){\makebox(0,0){{ 0p$\frac{1}{2}$}}}
\put(-2.9,0){\line(0,1){0.018}}
\put(-2.9,0.018){\line(1,0){0.2}}
\put(-2.7,0){\line(0,1){0.018}}
\put(1.8,0.335){\line(1,0){0.2}}
\put(1.8,0){\line(0,1){0.335}}
\put(2,0){\line(0,1){0.335}}
\put(-2.8,-0.2){\makebox(0,0){{ 0d$\frac{5}{2}$}}}
\put(1.9,-0.2){\makebox(0,0){{ 0d$\frac{5}{2}$}}}
\put(-2.2,0){\line(0,1){0.002}}
\put(-2.2,0.002){\line(1,0){0.2}}
\put(-2,0){\line(0,1){0.002}}
\put(2.5,-0.381){\line(1,0){0.2}}
\put(2.5,0){\line(0,-1){0.381}}
\put(2.7,0){\line(0,-1){0.381}}
\put(-2.1,-0.2){\makebox(0,0){{ 1s$\frac{1}{2}$}}}
\put(2.6,-0.2){\makebox(0,0){{ 1s$\frac{1}{2}$}}}
\put(-1.5,0){\line(0,1){0.005}}
\put(-1.5,0.005){\line(1,0){0.2}}
\put(-1.3,0){\line(0,1){0.005}}
\put(3.2,0.029){\line(1,0){0.2}}
\put(3.2,0){\line(0,1){0.029}}
\put(3.4,0){\line(0,1){0.029}}
\put(-1.4,-0.2){\makebox(0,0){{ 0d$\frac{3}{2}$}}}
\put(3.3,-0.2){\makebox(0,0){{ 0d$\frac{3}{2}$}}}
\put(-0.8,0){\line(0,1){0}}
\put(-0.8,0){\line(1,0){0.2}}
\put(-0.6,0){\line(0,1){0}}
\put(3.9,0.049){\line(1,0){0.2}}
\put(3.9,0){\line(0,1){0.049}}
\put(4.1,0){\line(0,1){0.049}}
\put(-0.7,-0.2){\makebox(0,0){{ 0f$\frac{7}{2}$}}}
\put(4,-0.2){\makebox(0,0){{ 0f$\frac{7}{2}$}}}
\put(-2.1,-2.7){\makebox(0,0){\large Protons}}
\put(2.1,-2.7){\makebox(0,0){\large Neutrons}}
\end{picture}
\end{center}
\caption{Orbital occupation difference from the ground state for the first $\frac52^-$ state of $^{15}B$, in the 0p3-0f7 model space.}
\label{fig:15B_g_0hf_3pert_0f7_2part_brown_1}
\end{figure}

%\begin{figure}[htbp]
%\setlength{\unitlength}{1.0cm}
%\begin{center}
%\begin{picture}(0,5)(0,-3)
%\put(0,2.0){\makebox(0,0){\large nr of particles}}
%\thicklines
%\put(0,-2){\line(0,1){3.8}}
%\multiput(0,-2.0)(0,1){4}{\line(1,0){.1}}
%\multiput(0,-1.5)(0,1){4}{\line(1,0){.05}}
%\put(0.2,1){\makebox(0,0){1}}
%\put(0.2,-1){\makebox(0,0){-1}}
%\put(0.2,-2){\makebox(0,0){-2}}
%\put(-4.7,0){\line(1,0){9.4}}
%\put(-4.3,0){\line(0,-1){0.05}}
%\put(-4.3,-0.05){\line(1,0){0.2}}
%\put(-4.1,0){\line(0,-1){0.05}}
%\put(0.4,-0.059){\line(1,0){0.2}}
%\put(0.4,0){\line(0,-1){0.059}}
%\put(0.6,0){\line(0,-1){0.059}}
%\put(-4.2,-0.2){\makebox(0,0){{ 0p$\frac{3}{2}$}}}
%\put(0.5,-0.2){\makebox(0,0){{ 0p$\frac{3}{2}$}}}
%\put(-3.6,0){\line(0,1){0}}
%\put(-3.6,0){\line(1,0){0.2}}
%\put(-3.4,0){\line(0,1){0}}
%\put(1.1,-0.028){\line(1,0){0.2}}
%\put(1.1,0){\line(0,-1){0.028}}
%\put(1.3,0){\line(0,-1){0.028}}
%\put(-3.5,-0.2){\makebox(0,0){{ 0p$\frac{1}{2}$}}}
%\put(1.2,-0.2){\makebox(0,0){{ 0p$\frac{1}{2}$}}}
%\put(-2.9,0){\line(0,1){0.04}}
%\put(-2.9,0.04){\line(1,0){0.2}}
%\put(-2.7,0){\line(0,1){0.04}}
%\put(1.8,0.455){\line(1,0){0.2}}
%\put(1.8,0){\line(0,1){0.455}}
%\put(2,0){\line(0,1){0.455}}
%\put(-2.8,-0.2){\makebox(0,0){{ 0d$\frac{5}{2}$}}}
%\put(1.9,-0.2){\makebox(0,0){{ 0d$\frac{5}{2}$}}}
%\put(-2.2,0){\line(0,1){0.004}}
%\put(-2.2,0.004){\line(1,0){0.2}}
%\put(-2,0){\line(0,1){0.004}}
%\put(2.5,-0.49){\line(1,0){0.2}}
%\put(2.5,0){\line(0,-1){0.49}}
%\put(2.7,0){\line(0,-1){0.49}}
%\put(-2.1,-0.2){\makebox(0,0){{ 1s$\frac{1}{2}$}}}
%\put(2.6,-0.2){\makebox(0,0){{ 1s$\frac{1}{2}$}}}
%\put(-1.5,0){\line(0,1){0.007}}
%\put(-1.5,0.007){\line(1,0){0.2}}
%\put(-1.3,0){\line(0,1){0.007}}
%\put(3.2,0.054){\line(1,0){0.2}}
%\put(3.2,0){\line(0,1){0.054}}
%\put(3.4,0){\line(0,1){0.054}}
%\put(-1.4,-0.2){\makebox(0,0){{ 0d$\frac{3}{2}$}}}
%\put(3.3,-0.2){\makebox(0,0){{ 0d$\frac{3}{2}$}}}
%\put(-0.8,0){\line(0,1){0}}
%\put(-0.8,0){\line(1,0){0.2}}
%\put(-0.6,0){\line(0,1){0}}
%\put(3.9,0.07){\line(1,0){0.2}}
%\put(3.9,0){\line(0,1){0.07}}
%\put(4.1,0){\line(0,1){0.07}}
%\put(-0.7,-0.2){\makebox(0,0){{ 0f$\frac{7}{2}$}}}
%\put(4,-0.2){\makebox(0,0){{ 0f$\frac{7}{2}$}}}
%\put(-2.1,-2.7){\makebox(0,0){\large Protons}}
%\put(2.1,-2.7){\makebox(0,0){\large Neutrons}}
%\end{picture}
%\end{center}
%\caption{Orbital occupation difference from the ground state for the 2. excited state of $^{15}B$ with orbital momentum $J = 3.5^-$, in the 0p3-0f7 model\-space and 3. order perturbation. Calculated using the G-matrix method, no Hartree-Fock, using the orbital-energys for $^{16}$O given by B.A. Brown and E.K. Warburton in their article in Physical rewiev C,V46,Nr3.}
%\label{fig:15B_g_0hf_3pert_0f7_2part_brown_2}
%\end{figure}
%
%\begin{figure}[htbp]
%\setlength{\unitlength}{1.0cm}
%\begin{center}
%\begin{picture}(0,5)(0,-3)
%\put(0,2.0){\makebox(0,0){\large nr of particles}}
%\thicklines
%\put(0,-2){\line(0,1){3.8}}
%\multiput(0,-2.0)(0,1){4}{\line(1,0){.1}}
%\multiput(0,-1.5)(0,1){4}{\line(1,0){.05}}
%\put(0.2,1){\makebox(0,0){1}}
%\put(0.2,-1){\makebox(0,0){-1}}
%\put(0.2,-2){\makebox(0,0){-2}}
%\put(-4.7,0){\line(1,0){9.4}}
%\put(-4.3,0){\line(0,-1){0.242}}
%\put(-4.3,-0.242){\line(1,0){0.2}}
%\put(-4.1,0){\line(0,-1){0.242}}
%\put(0.4,-0.056){\line(1,0){0.2}}
%\put(0.4,0){\line(0,-1){0.056}}
%\put(0.6,0){\line(0,-1){0.056}}
%\put(-4.2,-0.2){\makebox(0,0){{ 0p$\frac{3}{2}$}}}
%\put(0.5,-0.2){\makebox(0,0){{ 0p$\frac{3}{2}$}}}
%\put(-3.6,0){\line(0,1){0.16}}
%\put(-3.6,0.16){\line(1,0){0.2}}
%\put(-3.4,0){\line(0,1){0.16}}
%\put(1.1,-0.054){\line(1,0){0.2}}
%\put(1.1,0){\line(0,-1){0.054}}
%\put(1.3,0){\line(0,-1){0.054}}
%\put(-3.5,-0.2){\makebox(0,0){{ 0p$\frac{1}{2}$}}}
%\put(1.2,-0.2){\makebox(0,0){{ 0p$\frac{1}{2}$}}}
%\put(-2.9,0){\line(0,1){0.054}}
%\put(-2.9,0.054){\line(1,0){0.2}}
%\put(-2.7,0){\line(0,1){0.054}}
%\put(1.8,0.686){\line(1,0){0.2}}
%\put(1.8,0){\line(0,1){0.686}}
%\put(2,0){\line(0,1){0.686}}
%\put(-2.8,-0.2){\makebox(0,0){{ 0d$\frac{5}{2}$}}}
%\put(1.9,-0.2){\makebox(0,0){{ 0d$\frac{5}{2}$}}}
%\put(-2.2,0){\line(0,1){0.009}}
%\put(-2.2,0.009){\line(1,0){0.2}}
%\put(-2,0){\line(0,1){0.009}}
%\put(2.5,-0.87){\line(1,0){0.2}}
%\put(2.5,0){\line(0,-1){0.87}}
%\put(2.7,0){\line(0,-1){0.87}}
%\put(-2.1,-0.2){\makebox(0,0){{ 1s$\frac{1}{2}$}}}
%\put(2.6,-0.2){\makebox(0,0){{ 1s$\frac{1}{2}$}}}
%\put(-1.5,0){\line(0,1){0.019}}
%\put(-1.5,0.019){\line(1,0){0.2}}
%\put(-1.3,0){\line(0,1){0.019}}
%\put(3.2,0.06){\line(1,0){0.2}}
%\put(3.2,0){\line(0,1){0.06}}
%\put(3.4,0){\line(0,1){0.06}}
%\put(-1.4,-0.2){\makebox(0,0){{ 0d$\frac{3}{2}$}}}
%\put(3.3,-0.2){\makebox(0,0){{ 0d$\frac{3}{2}$}}}
%\put(-0.8,0){\line(0,-1){0.002}}
%\put(-0.8,-0.002){\line(1,0){0.2}}
%\put(-0.6,0){\line(0,-1){0.002}}
%\put(3.9,0.235){\line(1,0){0.2}}
%\put(3.9,0){\line(0,1){0.235}}
%\put(4.1,0){\line(0,1){0.235}}
%\put(-0.7,-0.2){\makebox(0,0){{ 0f$\frac{7}{2}$}}}
%\put(4,-0.2){\makebox(0,0){{ 0f$\frac{7}{2}$}}}
%\put(-2.1,-2.7){\makebox(0,0){\large Protons}}
%\put(2.1,-2.7){\makebox(0,0){\large Neutrons}}
%\end{picture}
%\end{center}
%\caption{Orbital occupation difference from the ground state for the 3. excited state of $^{15}B$ with orbital momentum $J = 0.50075^-$, in the 0p3-0f7 model\-space and 3. order perturbation. Calculated using the G-matrix method, no Hartree-Fock, using the orbital-energys for $^{16}$O given by B.A. Brown and E.K. Warburton in their article in Physical rewiev C,V46,Nr3.}
%\label{fig:15B_g_0hf_3pert_0f7_2part_brown_3}
%\end{figure}

\clearpage

\begin{table}
\begin{center}
\begin{tabular}{|c|c|c|c|c|c|}
	\hline
	Protons & $0p\frac32$ & $0p\frac12$ & $0d\frac52$ & $1s\frac12$ & $0d\frac32$ \\
	\hline
	Ground state $J=\frac32^-$ & 2.57 & 0.20 & 0.19 & 0.01 & 0.03 \\
	\hline
	First $\frac52^-$ state & -0.02 & +0.04 & -0.02 & 0.00 & 0.00 \\
	\hline
	First $\frac72^-$ state & -0.01 & +0.01 & 0.00 & 0.00 & 0.00 \\
	\hline
	Second $\frac32^-$ state & -0.02 & +0.01 & +0.01 & 0.00 & 0.00 \\
	\hline
\end{tabular}
\caption{Proton orbital occupancy for $^{15}$B in the $0p\frac32-0d\frac32$ model space. The horizontal line gives the orbitals, and the vertical line gives the excited states. For the excited state the diagram shows the orbital occupancy difference from the ground state.}
\label{B15_0d3_p}
\end{center}
\end{table}

\begin{table}
\begin{center}
\begin{tabular}{|c|c|c|c|c|c|}
	\hline
	Neutrons & $0p\frac32$ & $0p\frac12$ & $0d\frac52$ & $1s\frac12$ & $0d\frac32$  \\
	\hline
	Ground state $J=\frac32^-$ & 3.66 & 1.86 & 0.83 & 1.43 & 0.22 \\
	\hline
	First $\frac52^-$ state & +0.06 & +0.01 & +0.55 & -0.64 & +0.02 \\
	\hline
	First $\frac72^-$ state & +0.03 & +0.01 & +0.60 & -0.67 & +0.03 \\
	\hline
	Second $\frac32^-$ state & +0.01 & -0.01 & +0.30 & -0.28 & 0.00 \\
	\hline
\end{tabular}
\caption{Neutron orbital occupancy for $^{15}$B in the $0p\frac32-0d\frac32$ model space. The horizontal line gives the orbitals, and the vertical line gives the excited states. For the excited state the diagram shows the orbital occupancy difference from the ground state.}
\label{B15_0d3_n}
\end{center}
\end{table}

\begin{table}
\begin{center}
\begin{tabular}{|c|c|c|c|c|c|c|}
	\hline
	Protons & $0p\frac32$ & $0p\frac12$ & $0d\frac52$ & $1s\frac12$ & $0d\frac32$ & $0f\frac72$ \\
	\hline
	Ground state $J=\frac32^-$ & 2.43 & 0.26 & 0.22 & 0.01 & 0.06 & 0.02 \\
	\hline
	First $\frac52^-$ state & -0.05 & +0.03 & +0.02 & 0.00 & 0.00 & 0.00 \\
	\hline
	First $\frac72^-$ state & -0.05 & 0.00 & +0.04 & 0.00 & 0.00 & 0.00 \\
	\hline
	Second $\frac32^-$ state & -0.24 & +0.16 & +0.06 & +0.01 & +0.02 & 0.00 \\
	\hline
\end{tabular}
\caption{Proton orbital occupancy for $^{15}$B in the $0p\frac32-0f\frac72$ model space. The horizontal line gives the orbitals, and the vertical line gives the excited states. For the excited state the diagram shows the orbital occupancy difference from the ground state.}
\label{B15_0f7_p}
\end{center}
\end{table}

\begin{table}
\begin{center}
\begin{tabular}{|c|c|c|c|c|c|c|}
	\hline
	Neutrons & $0p\frac32$ & $0p\frac12$ & $0d\frac52$ & $1s\frac12$ & $0d\frac32$ & $0f\frac72$ \\
	\hline
	Ground state $J=\frac32^-$ & 3.62 & 1.84 & 1.09 & 1.05 & 0.23 & 0.17 \\
	\hline
	First $\frac52^-$ state & -0.02 & -0.02 & +0.33 & -0.38 & +0.03 & +0.05 \\
	\hline
	First $\frac72^-$ state & -0.06 & -0.03 & +0.45 & -0.51 & +0.06 & +0.07 \\
	\hline
	Second $\frac32^-$ state & -0.06 & -0.06 & +0.69 & -0.87 & +0.06 & +0.24 \\
	\hline
\end{tabular}
\caption{Neutron orbital occupancy for $^{15}$B in the $0p\frac32-0f\frac72$ model space. The horizontal line gives the orbitals, and the vertical line gives the excited states. For the excited state the diagram shows the orbital occupancy difference from the ground state.}
\label{B15_0f7_n}
\end{center}
\end{table}

\clearpage


\section{$^{16}$C}

In $^{16}$C we have again mostly one-neutron excitations from the $1s\frac12$
orbital to the $0d\frac52$ orbital, similar to what we have seen in $^{15}$C and $^{15}$B, see
figure \ref{fig:16C_g_0hf_3pert_0f7_2part_brown_1} and tables
\ref{C16_0d3-p}-\ref{C16_0f7_n}. We also have a very good correspondence
between our predicted states in the $0p\frac32-0f\frac72$ model space and the
experimental states.

\subsection{$0p\frac32-0d\frac32$ model space}

Figure \ref{fig:16C} shows the energy spectra for $^{16}$C in the
$0p\frac32-0d\frac32$ model space. We notice that the excited states have a
much higher energy than the experimental states, our first excited state having
almost twice the energy of the corresponding experimental state.

Figures \ref{fig:16C_g_0hf_3pert_0d3_4part_brown_0} -
\ref{fig:16C_g_0hf_3pert_0d3_4part_brown_1} show the sp orbital occupancy of
$^{16}$C in the $0p\frac32-0d\frac32$ model space. The ground state has
roughly two neutrons more in the $1s\frac12$ orbital than $^{14}$C, and has
otherwise almost the same orbital occupancy. The excited states of $^{16}$C
are essentially dominated by neutron excitations from the $1s\frac12$ orbital to the
$0d\frac52$ orbital. This shows that for $^{16}$C, the $0d\frac52$ orbital has
a higher excitation energy than the $1s\frac12$ orbital.
%We also notice that the first $2^+$ and first $3^+$ excitations are almost
%identical. !Hvorfor? Hva mer kan jeg si om dette?!

%There is very little difference in the proton orbital occupancy. For the first
%excited state of $^{16}$C, with angular momentum $2^+$, we have roughly a 0.7
%particle 0.7 hole excitation from the $1s\frac12$ orbital to the $0d\frac52$
%orbital. The second excited state of $^{16}$C, with angular momentum $0^+$ has
%roughly a 1.4 particle 1.4 hole excitation from the $1s\frac12$ orbital to the
%$0d\frac52$ orbital. For the third excited state of $^{16}$C, with angular
%momentum $3^+$, we have roughly a 0.7 particle 0.7 hole excitation from the
%$1s\frac12$ orbital to the $0d\frac52$ orbital, almost the same orbital occupancy
%as the first excited state of $^{16}$C.\\
%We see that the excited states of $^{16}$C mainly consists of neutron
%excitations from the $1s\frac12$ orbital to the $0d\frac52$ orbital, in accordance
%with our predictions.

\subsection{$0p\frac32-0f\frac72$ model space}

Figure \ref{fig:16C} shows the sp orbital occupancy of
$^{16}$C in the $0p\frac32-0f\frac72$ model space. The energy spectrum given
here is in much better agreement with the experimental data. The three excited
states calculated appear in the correct order. The first $2^+$ state
corresponds well with the experimental value, and the second $2^+$ state and
first $0^+$ state corresponds very well with the experimental data
%The energy difference between the experimental and our first $2^+$ state is
%about 0.10 MeV. For the second $2^+$ state the difference is about 0.060 MeV.
%For the first $0^+$ state the difference is 0.0018 MeV.

Figures \ref{fig:16C_g_0hf_3pert_0f7_2part_brown_0} -
\ref{fig:16C_g_0hf_3pert_0f7_2part_brown_1} show the sp orbital occupancy of
$^{16}$C in the $0p\frac32-0f\frac72$ model space. The ground state has
roughly one neutron more in the $0d\frac52$ orbital and one neutron less in the
$1s\frac12$ orbital when compared with the ground state for $^{16}$C in the
$0p\frac32-0d\frac32$ model space. When compared to the ground state for
$^{15}$C in the $0p\frac32-0f\frac72$ model space, we have roughly one neutron
more in the $0d\frac52$ orbital. This follows our predictions. It is interesting,
however, that the inclusion of the $0f7$ orbital moves one particle from the
$1s\frac12$ orbital to the $0d\frac52$ orbital.

Our first $2^+$ state has an energy difference with the corresponding
experimental state of 6.2\%. It consists mainly of neutron excitations from the
$1s\frac12$ orbital to the $0d\frac52$ orbital, but with roughly half as many
neutrons as the same state in the $0p\frac32-0d\frac32$ model space.

The first $0^+$ state has an energy difference with the corresponding
experimental state of only 0.06\%.

The second $2^+$ state has an energy difference with the corresponding
experimental state of 1.5\%.  The protons play a larger role here, with a low
amount of protons being excited from the $0p\frac32$ orbital to mostly the
$0p\frac12$ orbital. We again have a relatively large amount of neutrons being
excited from the $1s\frac12$ orbital, to the $0d\frac52$ and $0f\frac72$
orbitals, with most of the neutrons going to the $0d\frac52$ orbital, in
accordance with theory.

\begin{figure}[htbp]
\setlength{\unitlength}{3.0cm}
\begin{center}
\begin{picture}(6.9,6.5)(0,-1)
\psset{xunit=1.00cm, yunit=6.0cm}
\newcommand{\drawlevel}[5]{\psline[origin={#1,#2}, linewidth=0.3pt](0,0)(1.5,0.0)\rput(#3,#4){\scriptsize \makebox(0,0){$#5$}}}
\newcommand{\connect}[4]{\psline[linewidth=0.3pt, linestyle=dotted, dotsep=1.2pt](#1,#2)(#3,#4)}
\thicklines
\rput(0.1,-0.2){\makebox(0,0){{\large 0p3-0d3}}}
\drawlevel{0.7}{-0}{2}{0}{\ 0^+ \ (0)}
\drawlevel{0.7}{-1.03973}{2}{1.03973}{\ 2^+ \ (3.36^)}
\drawlevel{0.7}{-1.70674}{2}{1.70674}{\ 0^+ \ (5.51^)}
\drawlevel{0.7}{-2}{2}{2}{\ 3^+ \ (6.46^)}
\connect{0.8}{2}{1.1}{2}

\rput(4.3,-0.2){\makebox(0,0){{\large 0p3-0f7}}}
\drawlevel{-3.5}{-0}{6.2}{0}{\ 0^+ \ (0)}
\drawlevel{-3.5}{-0.515253}{6.2}{0.515253}{\ 2^+ \ (1.66^)}
\drawlevel{-3.5}{-0.938028}{6.2}{0.938028}{\ 0^+ \ (3.03^)}
\drawlevel{-3.5}{-1.2158}{6.2}{1.2158}{\ 2^+ \ (3.93^)}
\connect{5}{1.2158}{5.3}{1.2158}

\rput(8.5,-0.2){\makebox(0,0){{\large exp. data}}}
\drawlevel{-7.7}{-0}{10.4}{0}{\ 0^+ \ (0)}
\drawlevel{-7.7}{-0.546935}{10.4}{0.546935}{\ 2^+ \ (1.77^)}
\drawlevel{-7.7}{-0.937471}{10.4}{0.937471}{\ (0^+) \ (3.03^)}
\drawlevel{-7.7}{-1.23448}{10.4}{1.23448}{\ 2 \ (3.99^)}
\connect{9.2}{1.23448}{9.5}{1.23448}
\drawlevel{-7.7}{-1.26607}{10.4}{1.30448}{\ 3(^+) \ (4.09^)}
\connect{9.2}{1.26607}{9.5}{1.30448}
\drawlevel{-7.7}{-1.28279}{10.4}{1.37448}{\ 4^+ \ (4.14^)}
\connect{9.2}{1.28279}{9.5}{1.37448}
\drawlevel{-7.7}{-1.89198}{10.4}{1.89198}{\ (2^+,3^-,4^+) \ (6.11^)}

\end{picture}
\end{center}
\caption{Energy spectra for 16C, in the full 0p3-0d3 and reduced 0p3-0f7 model spaces. By reduced 0p3-0f7 model space we mean that we have reduced our 0p3-0f7 model space so that we have a maximum of 4 particles in the 0d5 orbital, a maximum of 2 particles in the 0d3 orbital and maximum of 2 neutrons and 0 protons in the 0f7 orbital. The energies are given in MeV.}
\label{fig:16C}
\end{figure}




\begin{figure}[htbp]
\setlength{\unitlength}{1.0cm}
\begin{center}
\begin{picture}(0,5)(0,-1)
\put(0,4.0){\makebox(0,0){\large nr of particles}}
\thicklines
\put(0,0){\line(0,1){3.8}}
\multiput(0,.0)(0,1){4}{\line(1,0){.1}}
\multiput(0,.5)(0,1){4}{\line(1,0){.05}}
\put(0.2,3){\makebox(0,0){3}}
\put(0.2,2){\makebox(0,0){2}}
\put(0.2,1){\makebox(0,0){1}}
\put(-4,0){\line(1,0){8}}
\put(-3.6,0){\line(0,1){3.301}}
\put(-3.6,3.301){\line(1,0){0.2}}
\put(-3.4,0){\line(0,1){3.301}}
\put(0.4,3.49){\line(1,0){0.2}}
\put(0.4,0){\line(0,1){3.490}}
\put(0.6,0){\line(0,1){3.490}}
\put(-3.5,-0.2){\makebox(0,0){{ 0p$\frac{3}{2}$}}}
\put(0.5,-0.2){\makebox(0,0){{ 0p$\frac{3}{2}$}}}
\put(-2.9,0){\line(0,1){0.236}}
\put(-2.9,0.236){\line(1,0){0.2}}
\put(-2.7,0){\line(0,1){0.236}}
\put(1.1,1.787){\line(1,0){0.2}}
\put(1.1,0){\line(0,1){1.787}}
\put(1.3,0){\line(0,1){1.787}}
\put(-2.8,-0.2){\makebox(0,0){{ 0p$\frac{1}{2}$}}}
\put(1.2,-0.2){\makebox(0,0){{ 0p$\frac{1}{2}$}}}
\put(-2.2,0){\line(0,1){0.370}}
\put(-2.2,0.37){\line(1,0){0.2}}
\put(-2,0){\line(0,1){0.370}}
\put(1.8,0.703){\line(1,0){0.2}}
\put(1.8,0){\line(0,1){0.703}}
\put(2,0){\line(0,1){0.703}}
\put(-2.1,-0.2){\makebox(0,0){{ 0d$\frac{5}{2}$}}}
\put(1.9,-0.2){\makebox(0,0){{ 0d$\frac{5}{2}$}}}
\put(-1.5,0){\line(0,1){0.018}}
\put(-1.5,0.018){\line(1,0){0.2}}
\put(-1.3,0){\line(0,1){0.018}}
\put(2.5,1.729){\line(1,0){0.2}}
\put(2.5,0){\line(0,1){1.729}}
\put(2.7,0){\line(0,1){1.729}}
\put(-1.4,-0.2){\makebox(0,0){{ 1s$\frac{1}{2}$}}}
\put(2.6,-0.2){\makebox(0,0){{ 1s$\frac{1}{2}$}}}
\put(-0.8,0){\line(0,1){0.076}}
\put(-0.8,0.076){\line(1,0){0.2}}
\put(-0.6,0){\line(0,1){0.076}}
\put(3.2,0.291){\line(1,0){0.2}}
\put(3.2,0){\line(0,1){0.291}}
\put(3.4,0){\line(0,1){0.291}}
\put(-0.7,-0.2){\makebox(0,0){{ 0d$\frac{3}{2}$}}}
\put(3.3,-0.2){\makebox(0,0){{ 0d$\frac{3}{2}$}}}
\put(-1.75,-.7){\makebox(0,0){\large Protons}}
\put(1.75,-.7){\makebox(0,0){\large Neutrons}}
\end{picture}
\end{center}
\caption{Orbital occupation for the ground state of $^{16}C$ with orbital momentum $J = 0^+$, in the 0p3-0d3 model space.}
\label{fig:16C_g_0hf_3pert_0d3_4part_brown_0}
\end{figure}

\begin{figure}[htbp]
\setlength{\unitlength}{1.0cm}
\begin{center}
\begin{picture}(0,5)(0,-3)
\put(0,2.0){\makebox(0,0){\large nr of particles}}
\thicklines
\put(0,-2){\line(0,1){3.8}}
\multiput(0,-2.0)(0,1){4}{\line(1,0){.1}}
\multiput(0,-1.5)(0,1){4}{\line(1,0){.05}}
\put(0.2,1){\makebox(0,0){1}}
\put(0.2,-1){\makebox(0,0){-1}}
\put(0.2,-2){\makebox(0,0){-2}}
\put(-4,0){\line(1,0){8}}
\put(-3.6,0){\line(0,-1){0.034}}
\put(-3.6,-0.034){\line(1,0){0.2}}
\put(-3.4,0){\line(0,-1){0.034}}
\put(0.4,0.06){\line(1,0){0.2}}
\put(0.4,0){\line(0,1){0.06}}
\put(0.6,0){\line(0,1){0.06}}
\put(-3.5,-0.2){\makebox(0,0){{ 0p$\frac{3}{2}$}}}
\put(0.5,-0.2){\makebox(0,0){{ 0p$\frac{3}{2}$}}}
\put(-2.9,0){\line(0,1){0.071}}
\put(-2.9,0.071){\line(1,0){0.2}}
\put(-2.7,0){\line(0,1){0.071}}
\put(1.1,0.018){\line(1,0){0.2}}
\put(1.1,0){\line(0,1){0.018}}
\put(1.3,0){\line(0,1){0.018}}
\put(-2.8,-0.2){\makebox(0,0){{ 0p$\frac{1}{2}$}}}
\put(1.2,-0.2){\makebox(0,0){{ 0p$\frac{1}{2}$}}}
\put(-2.2,0){\line(0,-1){0.032}}
\put(-2.2,-0.032){\line(1,0){0.2}}
\put(-2,0){\line(0,-1){0.032}}
\put(1.8,0.732){\line(1,0){0.2}}
\put(1.8,0){\line(0,1){0.732}}
\put(2,0){\line(0,1){0.732}}
\put(-2.1,-0.2){\makebox(0,0){{ 0d$\frac{5}{2}$}}}
\put(1.9,-0.2){\makebox(0,0){{ 0d$\frac{5}{2}$}}}
\put(-1.5,0){\line(0,-1){0.003}}
\put(-1.5,-0.003){\line(1,0){0.2}}
\put(-1.3,0){\line(0,-1){0.003}}
\put(2.5,-0.838){\line(1,0){0.2}}
\put(2.5,0){\line(0,-1){0.838}}
\put(2.7,0){\line(0,-1){0.838}}
\put(-1.4,-0.2){\makebox(0,0){{ 1s$\frac{1}{2}$}}}
\put(2.6,-0.2){\makebox(0,0){{ 1s$\frac{1}{2}$}}}
\put(-0.8,0){\line(0,-1){0.004}}
\put(-0.8,-0.004){\line(1,0){0.2}}
\put(-0.6,0){\line(0,-1){0.004}}
\put(3.2,0.029){\line(1,0){0.2}}
\put(3.2,0){\line(0,1){0.029}}
\put(3.4,0){\line(0,1){0.029}}
\put(-0.7,-0.2){\makebox(0,0){{ 0d$\frac{3}{2}$}}}
\put(3.3,-0.2){\makebox(0,0){{ 0d$\frac{3}{2}$}}}
\put(-1.75,-2.7){\makebox(0,0){\large Protons}}
\put(1.75,-2.7){\makebox(0,0){\large Neutrons}}
\end{picture}
\end{center}
\caption{Orbital occupation difference from the ground state for the first $2^+$ state of $^{16}C$, in the 0p3-0d3 model space.}
\label{fig:16C_g_0hf_3pert_0d3_4part_brown_1}
\end{figure}

%\begin{figure}[htbp]
%\setlength{\unitlength}{1.0cm}
%\begin{center}
%\begin{picture}(0,5)(0,-3)
%\put(0,2.0){\makebox(0,0){\large nr of particles}}
%\thicklines
%\put(0,-2){\line(0,1){3.8}}
%\multiput(0,-2.0)(0,1){4}{\line(1,0){.1}}
%\multiput(0,-1.5)(0,1){4}{\line(1,0){.05}}
%\put(0.2,1){\makebox(0,0){1}}
%\put(0.2,-1){\makebox(0,0){-1}}
%\put(0.2,-2){\makebox(0,0){-2}}
%\put(-4,0){\line(1,0){8}}
%\put(-3.6,0){\line(0,-1){0.061}}
%\put(-3.6,-0.061){\line(1,0){0.2}}
%\put(-3.4,0){\line(0,-1){0.061}}
%\put(0.4,0.032){\line(1,0){0.2}}
%\put(0.4,0){\line(0,1){0.032}}
%\put(0.6,0){\line(0,1){0.032}}
%\put(-3.5,-0.2){\makebox(0,0){{ 0p$\frac{3}{2}$}}}
%\put(0.5,-0.2){\makebox(0,0){{ 0p$\frac{3}{2}$}}}
%\put(-2.9,0){\line(0,1){0.111}}
%\put(-2.9,0.111){\line(1,0){0.2}}
%\put(-2.7,0){\line(0,1){0.111}}
%\put(1.1,0.02){\line(1,0){0.2}}
%\put(1.1,0){\line(0,1){0.02}}
%\put(1.3,0){\line(0,1){0.02}}
%\put(-2.8,-0.2){\makebox(0,0){{ 0p$\frac{1}{2}$}}}
%\put(1.2,-0.2){\makebox(0,0){{ 0p$\frac{1}{2}$}}}
%\put(-2.2,0){\line(0,-1){0.04}}
%\put(-2.2,-0.04){\line(1,0){0.2}}
%\put(-2,0){\line(0,-1){0.04}}
%\put(1.8,1.42){\line(1,0){0.2}}
%\put(1.8,0){\line(0,1){1.42}}
%\put(2,0){\line(0,1){1.42}}
%\put(-2.1,-0.2){\makebox(0,0){{ 0d$\frac{5}{2}$}}}
%\put(1.9,-0.2){\makebox(0,0){{ 0d$\frac{5}{2}$}}}
%\put(-1.5,0){\line(0,-1){0.003}}
%\put(-1.5,-0.003){\line(1,0){0.2}}
%\put(-1.3,0){\line(0,-1){0.003}}
%\put(2.5,-1.501){\line(1,0){0.2}}
%\put(2.5,0){\line(0,-1){1.501}}
%\put(2.7,0){\line(0,-1){1.501}}
%\put(-1.4,-0.2){\makebox(0,0){{ 1s$\frac{1}{2}$}}}
%\put(2.6,-0.2){\makebox(0,0){{ 1s$\frac{1}{2}$}}}
%\put(-0.8,0){\line(0,-1){0.008}}
%\put(-0.8,-0.008){\line(1,0){0.2}}
%\put(-0.6,0){\line(0,-1){0.008}}
%\put(3.2,0.028){\line(1,0){0.2}}
%\put(3.2,0){\line(0,1){0.028}}
%\put(3.4,0){\line(0,1){0.028}}
%\put(-0.7,-0.2){\makebox(0,0){{ 0d$\frac{3}{2}$}}}
%\put(3.3,-0.2){\makebox(0,0){{ 0d$\frac{3}{2}$}}}
%\put(-1.75,-2.7){\makebox(0,0){\large Protons}}
%\put(1.75,-2.7){\makebox(0,0){\large Neutrons}}
%\end{picture}
%\end{center}
%\caption{Orbital occupation difference from the ground state for the 2. excited state of $^{16}C$ with orbital momentum $J = 0$, in the 0p3-0d3 model\-space and 3. order perturbation. Calculated using the G-matrix method, no Hartree-Fock, using the orbital-energys for $^{16}$O given by B.A. Brown and E.K. Warburton in their article in Physical rewiev C,V46,Nr3.}
%\label{fig:16C_g_0hf_3pert_0d3_4part_brown_2}
%\end{figure}
%
%\begin{figure}[htbp]
%\setlength{\unitlength}{1.0cm}
%\begin{center}
%\begin{picture}(0,5)(0,-3)
%\put(0,2.0){\makebox(0,0){\large nr of particles}}
%\thicklines
%\put(0,-2){\line(0,1){3.8}}
%\multiput(0,-2.0)(0,1){4}{\line(1,0){.1}}
%\multiput(0,-1.5)(0,1){4}{\line(1,0){.05}}
%\put(0.2,1){\makebox(0,0){1}}
%\put(0.2,-1){\makebox(0,0){-1}}
%\put(0.2,-2){\makebox(0,0){-2}}
%\put(-4,0){\line(1,0){8}}
%\put(-3.6,0){\line(0,1){0.018}}
%\put(-3.6,0.018){\line(1,0){0.2}}
%\put(-3.4,0){\line(0,1){0.018}}
%\put(0.4,0.084){\line(1,0){0.2}}
%\put(0.4,0){\line(0,1){0.084}}
%\put(0.6,0){\line(0,1){0.084}}
%\put(-3.5,-0.2){\makebox(0,0){{ 0p$\frac{3}{2}$}}}
%\put(0.5,-0.2){\makebox(0,0){{ 0p$\frac{3}{2}$}}}
%\put(-2.9,0){\line(0,1){0.03}}
%\put(-2.9,0.03){\line(1,0){0.2}}
%\put(-2.7,0){\line(0,1){0.03}}
%\put(1.1,0.022){\line(1,0){0.2}}
%\put(1.1,0){\line(0,1){0.022}}
%\put(1.3,0){\line(0,1){0.022}}
%\put(-2.8,-0.2){\makebox(0,0){{ 0p$\frac{1}{2}$}}}
%\put(1.2,-0.2){\makebox(0,0){{ 0p$\frac{1}{2}$}}}
%\put(-2.2,0){\line(0,-1){0.035}}
%\put(-2.2,-0.035){\line(1,0){0.2}}
%\put(-2,0){\line(0,-1){0.035}}
%\put(1.8,0.691){\line(1,0){0.2}}
%\put(1.8,0){\line(0,1){0.691}}
%\put(2,0){\line(0,1){0.691}}
%\put(-2.1,-0.2){\makebox(0,0){{ 0d$\frac{5}{2}$}}}
%\put(1.9,-0.2){\makebox(0,0){{ 0d$\frac{5}{2}$}}}
%\put(-1.5,0){\line(0,-1){0.003}}
%\put(-1.5,-0.003){\line(1,0){0.2}}
%\put(-1.3,0){\line(0,-1){0.003}}
%\put(2.5,-0.76){\line(1,0){0.2}}
%\put(2.5,0){\line(0,-1){0.76}}
%\put(2.7,0){\line(0,-1){0.76}}
%\put(-1.4,-0.2){\makebox(0,0){{ 1s$\frac{1}{2}$}}}
%\put(2.6,-0.2){\makebox(0,0){{ 1s$\frac{1}{2}$}}}
%\put(-0.8,0){\line(0,-1){0.011}}
%\put(-0.8,-0.011){\line(1,0){0.2}}
%\put(-0.6,0){\line(0,-1){0.011}}
%\put(3.2,-0.037){\line(1,0){0.2}}
%\put(3.2,0){\line(0,-1){0.037}}
%\put(3.4,0){\line(0,-1){0.037}}
%\put(-0.7,-0.2){\makebox(0,0){{ 0d$\frac{3}{2}$}}}
%\put(3.3,-0.2){\makebox(0,0){{ 0d$\frac{3}{2}$}}}
%\put(-1.75,-2.7){\makebox(0,0){\large Protons}}
%\put(1.75,-2.7){\makebox(0,0){\large Neutrons}}
%\end{picture}
%\end{center}
%\caption{Orbital occupation difference from the ground state for the 3. excited state of $^{16}C$ with orbital momentum $J = 3$, in the 0p3-0d3 model\-space and 3. order perturbation. Calculated using the G-matrix method, no Hartree-Fock, using the orbital-energys for $^{16}$O given by B.A. Brown and E.K. Warburton in their article in Physical rewiev C,V46,Nr3.}
%\label{fig:16C_g_0hf_3pert_0d3_4part_brown_3}
%\end{figure}

\clearpage

\begin{figure}[htbp]
\setlength{\unitlength}{1.0cm}
\begin{center}
\begin{picture}(0,5)(0,-1)
\put(0,4.0){\makebox(0,0){\large nr of particles}}
\thicklines
\put(0,0){\line(0,1){3.8}}
\multiput(0,.0)(0,1){4}{\line(1,0){.1}}
\multiput(0,.5)(0,1){4}{\line(1,0){.05}}
\put(0.2,3){\makebox(0,0){3}}
\put(0.2,2){\makebox(0,0){2}}
\put(0.2,1){\makebox(0,0){1}}
\put(-4.7,0){\line(1,0){9.4}}
\put(-4.3,0){\line(0,1){3.079}}
\put(-4.3,3.079){\line(1,0){0.2}}
\put(-4.1,0){\line(0,1){3.079}}
\put(0.4,3.461){\line(1,0){0.2}}
\put(0.4,0){\line(0,1){3.461}}
\put(0.6,0){\line(0,1){3.461}}
\put(-4.2,-0.2){\makebox(0,0){{ 0p$\frac{3}{2}$}}}
\put(0.5,-0.2){\makebox(0,0){{ 0p$\frac{3}{2}$}}}
\put(-3.6,0){\line(0,1){0.402}}
\put(-3.6,0.402){\line(1,0){0.2}}
\put(-3.4,0){\line(0,1){0.402}}
\put(1.1,1.775){\line(1,0){0.2}}
\put(1.1,0){\line(0,1){1.775}}
\put(1.3,0){\line(0,1){1.775}}
\put(-3.5,-0.2){\makebox(0,0){{ 0p$\frac{1}{2}$}}}
\put(1.2,-0.2){\makebox(0,0){{ 0p$\frac{1}{2}$}}}
\put(-2.9,0){\line(0,1){0.388}}
\put(-2.9,0.388){\line(1,0){0.2}}
\put(-2.7,0){\line(0,1){0.388}}
\put(1.8,1.232){\line(1,0){0.2}}
\put(1.8,0){\line(0,1){1.232}}
\put(2,0){\line(0,1){1.232}}
\put(-2.8,-0.2){\makebox(0,0){{ 0d$\frac{5}{2}$}}}
\put(1.9,-0.2){\makebox(0,0){{ 0d$\frac{5}{2}$}}}
\put(-2.2,0){\line(0,1){0.021}}
\put(-2.2,0.021){\line(1,0){0.2}}
\put(-2,0){\line(0,1){0.021}}
\put(2.5,0.962){\line(1,0){0.2}}
\put(2.5,0){\line(0,1){0.962}}
\put(2.7,0){\line(0,1){0.962}}
\put(-2.1,-0.2){\makebox(0,0){{ 1s$\frac{1}{2}$}}}
\put(2.6,-0.2){\makebox(0,0){{ 1s$\frac{1}{2}$}}}
\put(-1.5,0){\line(0,1){0.110}}
\put(-1.5,0.11){\line(1,0){0.2}}
\put(-1.3,0){\line(0,1){0.110}}
\put(3.2,0.298){\line(1,0){0.2}}
\put(3.2,0){\line(0,1){0.298}}
\put(3.4,0){\line(0,1){0.298}}
\put(-1.4,-0.2){\makebox(0,0){{ 0d$\frac{3}{2}$}}}
\put(3.3,-0.2){\makebox(0,0){{ 0d$\frac{3}{2}$}}}
\put(-0.8,0){\line(0,1){0.000}}
\put(-0.8,0){\line(1,0){0.2}}
\put(-0.6,0){\line(0,1){0.000}}
\put(3.9,0.273){\line(1,0){0.2}}
\put(3.9,0){\line(0,1){0.273}}
\put(4.1,0){\line(0,1){0.273}}
\put(-0.7,-0.2){\makebox(0,0){{ 0f$\frac{7}{2}$}}}
\put(4,-0.2){\makebox(0,0){{ 0f$\frac{7}{2}$}}}
\put(-2.1,-.7){\makebox(0,0){\large Protons}}
\put(2.1,-.7){\makebox(0,0){\large Neutrons}}
\end{picture}
\end{center}
\caption{Orbital occupation for the ground state of $^{16}C$ with orbital momentum $J = 0^+$, in the 0p3-0f7 model space.}
\label{fig:16C_g_0hf_3pert_0f7_2part_brown_0}
\end{figure}

\begin{figure}[htbp]
\setlength{\unitlength}{1.0cm}
\begin{center}
\begin{picture}(0,5)(0,-3)
\put(0,2.0){\makebox(0,0){\large nr of particles}}
\thicklines
\put(0,-2){\line(0,1){3.8}}
\multiput(0,-2.0)(0,1){4}{\line(1,0){.1}}
\multiput(0,-1.5)(0,1){4}{\line(1,0){.05}}
\put(0.2,1){\makebox(0,0){1}}
\put(0.2,-1){\makebox(0,0){-1}}
\put(0.2,-2){\makebox(0,0){-2}}
\put(-4.7,0){\line(1,0){9.4}}
\put(-4.3,0){\line(0,-1){0.042}}
\put(-4.3,-0.042){\line(1,0){0.2}}
\put(-4.1,0){\line(0,-1){0.042}}
\put(0.4,-0.032){\line(1,0){0.2}}
\put(0.4,0){\line(0,-1){0.032}}
\put(0.6,0){\line(0,-1){0.032}}
\put(-4.2,-0.2){\makebox(0,0){{ 0p$\frac{3}{2}$}}}
\put(0.5,-0.2){\makebox(0,0){{ 0p$\frac{3}{2}$}}}
\put(-3.6,0){\line(0,1){0.009}}
\put(-3.6,0.009){\line(1,0){0.2}}
\put(-3.4,0){\line(0,1){0.009}}
\put(1.1,-0.013){\line(1,0){0.2}}
\put(1.1,0){\line(0,-1){0.013}}
\put(1.3,0){\line(0,-1){0.013}}
\put(-3.5,-0.2){\makebox(0,0){{ 0p$\frac{1}{2}$}}}
\put(1.2,-0.2){\makebox(0,0){{ 0p$\frac{1}{2}$}}}
\put(-2.9,0){\line(0,1){0.025}}
\put(-2.9,0.025){\line(1,0){0.2}}
\put(-2.7,0){\line(0,1){0.025}}
\put(1.8,0.301){\line(1,0){0.2}}
\put(1.8,0){\line(0,1){0.301}}
\put(2,0){\line(0,1){0.301}}
\put(-2.8,-0.2){\makebox(0,0){{ 0d$\frac{5}{2}$}}}
\put(1.9,-0.2){\makebox(0,0){{ 0d$\frac{5}{2}$}}}
\put(-2.2,0){\line(0,1){0.002}}
\put(-2.2,0.002){\line(1,0){0.2}}
\put(-2,0){\line(0,1){0.002}}
\put(2.5,-0.322){\line(1,0){0.2}}
\put(2.5,0){\line(0,-1){0.322}}
\put(2.7,0){\line(0,-1){0.322}}
\put(-2.1,-0.2){\makebox(0,0){{ 1s$\frac{1}{2}$}}}
\put(2.6,-0.2){\makebox(0,0){{ 1s$\frac{1}{2}$}}}
\put(-1.5,0){\line(0,1){0.006}}
\put(-1.5,0.006){\line(1,0){0.2}}
\put(-1.3,0){\line(0,1){0.006}}
\put(3.2,0.028){\line(1,0){0.2}}
\put(3.2,0){\line(0,1){0.028}}
\put(3.4,0){\line(0,1){0.028}}
\put(-1.4,-0.2){\makebox(0,0){{ 0d$\frac{3}{2}$}}}
\put(3.3,-0.2){\makebox(0,0){{ 0d$\frac{3}{2}$}}}
\put(-0.8,0){\line(0,1){0}}
\put(-0.8,0){\line(1,0){0.2}}
\put(-0.6,0){\line(0,1){0}}
\put(3.9,0.036){\line(1,0){0.2}}
\put(3.9,0){\line(0,1){0.036}}
\put(4.1,0){\line(0,1){0.036}}
\put(-0.7,-0.2){\makebox(0,0){{ 0f$\frac{7}{2}$}}}
\put(4,-0.2){\makebox(0,0){{ 0f$\frac{7}{2}$}}}
\put(-2.1,-2.7){\makebox(0,0){\large Protons}}
\put(2.1,-2.7){\makebox(0,0){\large Neutrons}}
\end{picture}
\end{center}
\caption{Orbital occupation difference from the ground state for the first $2^+$ state of $^{16}C$, in the 0p3-0f7 model space.}
\label{fig:16C_g_0hf_3pert_0f7_2part_brown_1}
\end{figure}

%\begin{figure}[htbp]
%\setlength{\unitlength}{1.0cm}
%\begin{center}
%\begin{picture}(0,5)(0,-3)
%\put(0,2.0){\makebox(0,0){\large nr of particles}}
%\thicklines
%\put(0,-2){\line(0,1){3.8}}
%\multiput(0,-2.0)(0,1){4}{\line(1,0){.1}}
%\multiput(0,-1.5)(0,1){4}{\line(1,0){.05}}
%\put(0.2,1){\makebox(0,0){1}}
%\put(0.2,-1){\makebox(0,0){-1}}
%\put(0.2,-2){\makebox(0,0){-2}}
%\put(-4.7,0){\line(1,0){9.4}}
%\put(-4.3,0){\line(0,-1){0.034}}
%\put(-4.3,-0.034){\line(1,0){0.2}}
%\put(-4.1,0){\line(0,-1){0.034}}
%\put(0.4,0.04){\line(1,0){0.2}}
%\put(0.4,0){\line(0,1){0.04}}
%\put(0.6,0){\line(0,1){0.04}}
%\put(-4.2,-0.2){\makebox(0,0){{ 0p$\frac{3}{2}$}}}
%\put(0.5,-0.2){\makebox(0,0){{ 0p$\frac{3}{2}$}}}
%\put(-3.6,0){\line(0,1){0.037}}
%\put(-3.6,0.037){\line(1,0){0.2}}
%\put(-3.4,0){\line(0,1){0.037}}
%\put(1.1,0.003){\line(1,0){0.2}}
%\put(1.1,0){\line(0,1){0.003}}
%\put(1.3,0){\line(0,1){0.003}}
%\put(-3.5,-0.2){\makebox(0,0){{ 0p$\frac{1}{2}$}}}
%\put(1.2,-0.2){\makebox(0,0){{ 0p$\frac{1}{2}$}}}
%\put(-2.9,0){\line(0,-1){0.007}}
%\put(-2.9,-0.007){\line(1,0){0.2}}
%\put(-2.7,0){\line(0,-1){0.007}}
%\put(1.8,-0.033){\line(1,0){0.2}}
%\put(1.8,0){\line(0,-1){0.033}}
%\put(2,0){\line(0,-1){0.033}}
%\put(-2.8,-0.2){\makebox(0,0){{ 0d$\frac{5}{2}$}}}
%\put(1.9,-0.2){\makebox(0,0){{ 0d$\frac{5}{2}$}}}
%\put(-2.2,0){\line(0,-1){0.001}}
%\put(-2.2,-0.001){\line(1,0){0.2}}
%\put(-2,0){\line(0,-1){0.001}}
%\put(2.5,-0.028){\line(1,0){0.2}}
%\put(2.5,0){\line(0,-1){0.028}}
%\put(2.7,0){\line(0,-1){0.028}}
%\put(-2.1,-0.2){\makebox(0,0){{ 1s$\frac{1}{2}$}}}
%\put(2.6,-0.2){\makebox(0,0){{ 1s$\frac{1}{2}$}}}
%\put(-1.5,0){\line(0,1){0.005}}
%\put(-1.5,0.005){\line(1,0){0.2}}
%\put(-1.3,0){\line(0,1){0.005}}
%\put(3.2,-0.012){\line(1,0){0.2}}
%\put(3.2,0){\line(0,-1){0.012}}
%\put(3.4,0){\line(0,-1){0.012}}
%\put(-1.4,-0.2){\makebox(0,0){{ 0d$\frac{3}{2}$}}}
%\put(3.3,-0.2){\makebox(0,0){{ 0d$\frac{3}{2}$}}}
%\put(-0.8,0){\line(0,1){0}}
%\put(-0.8,0){\line(1,0){0.2}}
%\put(-0.6,0){\line(0,1){0}}
%\put(3.9,0.029){\line(1,0){0.2}}
%\put(3.9,0){\line(0,1){0.029}}
%\put(4.1,0){\line(0,1){0.029}}
%\put(-0.7,-0.2){\makebox(0,0){{ 0f$\frac{7}{2}$}}}
%\put(4,-0.2){\makebox(0,0){{ 0f$\frac{7}{2}$}}}
%\put(-2.1,-2.7){\makebox(0,0){\large Protons}}
%\put(2.1,-2.7){\makebox(0,0){\large Neutrons}}
%\end{picture}
%\end{center}
%\caption{Orbital occupation difference from the ground state for the 2. excited state of $^{16}C$ with orbital momentum $J = 0$, in the 0p3-0f7 model\-space and 3. order perturbation. Calculated using the G-matrix method, no Hartree-Fock, using the orbital-energys for $^{16}$O given by B.A. Brown and E.K. Warburton in their article in Physical rewiev C,V46,Nr3.}
%\label{fig:16C_g_0hf_3pert_0f7_2part_brown_2}
%\end{figure}
%
%\begin{figure}[htbp]
%\setlength{\unitlength}{1.0cm}
%\begin{center}
%\begin{picture}(0,5)(0,-3)
%\put(0,2.0){\makebox(0,0){\large nr of particles}}
%\thicklines
%\put(0,-2){\line(0,1){3.8}}
%\multiput(0,-2.0)(0,1){4}{\line(1,0){.1}}
%\multiput(0,-1.5)(0,1){4}{\line(1,0){.05}}
%\put(0.2,1){\makebox(0,0){1}}
%\put(0.2,-1){\makebox(0,0){-1}}
%\put(0.2,-2){\makebox(0,0){-2}}
%\put(-4.7,0){\line(1,0){9.4}}
%\put(-4.3,0){\line(0,-1){0.197}}
%\put(-4.3,-0.197){\line(1,0){0.2}}
%\put(-4.1,0){\line(0,-1){0.197}}
%\put(0.4,0.001){\line(1,0){0.2}}
%\put(0.4,0){\line(0,1){0.001}}
%\put(0.6,0){\line(0,1){0.001}}
%\put(-4.2,-0.2){\makebox(0,0){{ 0p$\frac{3}{2}$}}}
%\put(0.5,-0.2){\makebox(0,0){{ 0p$\frac{3}{2}$}}}
%\put(-3.6,0){\line(0,1){0.135}}
%\put(-3.6,0.135){\line(1,0){0.2}}
%\put(-3.4,0){\line(0,1){0.135}}
%\put(1.1,-0.03){\line(1,0){0.2}}
%\put(1.1,0){\line(0,-1){0.03}}
%\put(1.3,0){\line(0,-1){0.03}}
%\put(-3.5,-0.2){\makebox(0,0){{ 0p$\frac{1}{2}$}}}
%\put(1.2,-0.2){\makebox(0,0){{ 0p$\frac{1}{2}$}}}
%\put(-2.9,0){\line(0,1){0.031}}
%\put(-2.9,0.031){\line(1,0){0.2}}
%\put(-2.7,0){\line(0,1){0.031}}
%\put(1.8,0.423){\line(1,0){0.2}}
%\put(1.8,0){\line(0,1){0.423}}
%\put(2,0){\line(0,1){0.423}}
%\put(-2.8,-0.2){\makebox(0,0){{ 0d$\frac{5}{2}$}}}
%\put(1.9,-0.2){\makebox(0,0){{ 0d$\frac{5}{2}$}}}
%\put(-2.2,0){\line(0,1){0.004}}
%\put(-2.2,0.004){\line(1,0){0.2}}
%\put(-2,0){\line(0,1){0.004}}
%\put(2.5,-0.622){\line(1,0){0.2}}
%\put(2.5,0){\line(0,-1){0.622}}
%\put(2.7,0){\line(0,-1){0.622}}
%\put(-2.1,-0.2){\makebox(0,0){{ 1s$\frac{1}{2}$}}}
%\put(2.6,-0.2){\makebox(0,0){{ 1s$\frac{1}{2}$}}}
%\put(-1.5,0){\line(0,1){0.027}}
%\put(-1.5,0.027){\line(1,0){0.2}}
%\put(-1.3,0){\line(0,1){0.027}}
%\put(3.2,0.033){\line(1,0){0.2}}
%\put(3.2,0){\line(0,1){0.033}}
%\put(3.4,0){\line(0,1){0.033}}
%\put(-1.4,-0.2){\makebox(0,0){{ 0d$\frac{3}{2}$}}}
%\put(3.3,-0.2){\makebox(0,0){{ 0d$\frac{3}{2}$}}}
%\put(-0.8,0){\line(0,1){0}}
%\put(-0.8,0){\line(1,0){0.2}}
%\put(-0.6,0){\line(0,1){0}}
%\put(3.9,0.193){\line(1,0){0.2}}
%\put(3.9,0){\line(0,1){0.193}}
%\put(4.1,0){\line(0,1){0.193}}
%\put(-0.7,-0.2){\makebox(0,0){{ 0f$\frac{7}{2}$}}}
%\put(4,-0.2){\makebox(0,0){{ 0f$\frac{7}{2}$}}}
%\put(-2.1,-2.7){\makebox(0,0){\large Protons}}
%\put(2.1,-2.7){\makebox(0,0){\large Neutrons}}
%\end{picture}
%\end{center}
%\caption{Orbital occupation difference from the ground state for the 3. excited state of $^{16}C$ with orbital momentum $J = 2$, in the 0p3-0f7 model\-space and 3. order perturbation. Calculated using the G-matrix method, no Hartree-Fock, using the orbital-energys for $^{16}$O given by B.A. Brown and E.K. Warburton in their article in Physical rewiev C,V46,Nr3.}
%\label{fig:16C_g_0hf_3pert_0f7_2part_brown_3}
%\end{figure}

\clearpage

\begin{table}
\begin{center}
\begin{tabular}{|c|c|c|c|c|c|}
	\hline
	Protons & $0p\frac32$ & $0p\frac12$ & $0d\frac52$ & $1s\frac12$ & $0d\frac32$ \\
	\hline
	Ground state $J = 0^+$ & 3.30 & 0.24 & 0.37 & 0.02 & 0.08 \\
	\hline
	First $2^+$ state & -0.03 & +0.07 & -0.03 & 0.00 & 0.00 \\
	\hline
	First $0^+$ state & -0.06 & +0.11 & -0.04 & 0.00 & -0.01 \\
	\hline
	Second $3^+$ state & +0.02 & +0.03 & -0.03 & 0.00 & -0.01 \\
	\hline
\end{tabular}
\caption{Proton orbital occupancy for $^{16}$C in the $0p\frac32-0d\frac32$ model space. The horizontal line gives the orbitals, and the vertical line gives the excited states. For the excited state the diagram shows the orbital occupancy difference from the ground state.}
\label{C16_0d3_p}
\end{center}
\end{table}

\begin{table}
\begin{center}
\begin{tabular}{|c|c|c|c|c|c|}
	\hline
	Neutrons & $0p\frac32$ & $0p\frac12$ & $0d\frac52$ & $1s\frac12$ & $0d\frac32$ \\
	\hline
	Ground state $J=0^+$ & 3.49 & 1.79 & 0.70 & 1.73 & 0.29 \\
	\hline
	First $2^+$ state & +0.06 & +0.02 & +0.74 & -0.84 & +0.03 \\
	\hline
	First $0^+$ state & +0.03 & +0.02 & +1.42 & -1.50 & +0.03 \\
	\hline
	Second $3^+$ state & +0.08 & +0.02 & +0.69 & -0.76 & -0.04 \\
	\hline
\end{tabular}
\caption{Neutron orbital occupancy for $^{16}$C in the $0p\frac32-0d\frac32$ model space. The horizontal line gives the orbitals, and the vertical line gives the excited states. For the excited state the diagram shows the orbital occupancy difference from the ground state.}
\label{C16_0d3_n}
\end{center}
\end{table}

\begin{table}
\begin{center}
\begin{tabular}{|c|c|c|c|c|c|c|}
	\hline
	Protons & $0p\frac32$ & $0p\frac12$ & $0d\frac52$ & $1s\frac12$ & $0d\frac32$ & $0f\frac72$ \\
	\hline
	Ground state $J=0^+$ & 3.08 & 0.40 & 0.39 & 0.02 & 0.11 & 0 \\
	\hline
	First $2^+$ state & -0.04 & +0.01 & +0.02 & 0.00 & +0.01 & 0 \\
	\hline
	First $0^+$ state & -0.03 & +0.04 & -0.01 & 0.00 & +0.01 & 0 \\
	\hline
	Second $2^+$ state & -0.20 & +0.14 & +0.03 & +0.01 & +0.03 & 0 \\
	\hline
\end{tabular}
\caption{Proton orbital occupancy for $^{16}$C in the $0p\frac32-0f\frac72$ model space. The horizontal line gives the orbitals, and the vertical line gives the excited states. For the excited state the diagram shows the orbital occupancy difference from the ground state.}
\label{C16_0f7_p}
\end{center}
\end{table}

\begin{table}
\begin{center}
\begin{tabular}{|c|c|c|c|c|c|c|}
	\hline
	Neutrons & $0p\frac32$ & $0p\frac12$ & $0d\frac52$ & $1s\frac12$ & $0d\frac32$ & $0f\frac72$ \\
	\hline
	Ground state $J=0^+$ & 3.46 & 1.78 & 1.23 & 0.96 & 0.30 & 0.27 \\
	\hline
	First $2^+$ state & -0.03 & -0.02 & +0.30 & -0.32 & +0.03 & +0.04 \\
	\hline
	First $0^+$ state & +0.04 & +0.01 & -0.03 & -0.03 & -0.01 & +0.03 \\
	\hline
	Second $2^+$ state & 0.00 & -0.03 & +0.43 & -0.62 & +0.03 & +0.20 \\
	\hline
\end{tabular}
\caption{Neutron orbital occupancy for $^{16}$C in the $0p\frac32-0f\frac72$ model space. The horizontal line gives the orbitals, and the vertical line gives the excited states. For the excited state the diagram shows the orbital occupancy difference from the ground state.}
\label{C16_0f7}
\end{center}
\end{table}

\clearpage


%we have roughly a 0.3 particle 0.3 hole excitation from the $1s\frac12$ orbital
%to the $0d\frac52$ orbital. There is very little change in the proton occupancy.
%For the second excited state of $^{16}$C, with angular momentum $0^+$, there is
%almost no change in the orbital occupancies. For the third excited state of
%$^{16}$C, with angular momentum $2^+$, we have 0.6 neutrons less in the
%$1s\frac12$ orbital, 0.4 more neutrons in the $0d\frac52$ orbital and 0.2 neutrons
%more in the $0f\frac72$ orbital. For protons we have 0.2 protons that excite from
%the $0p\frac32$ orbital to mostly the $0p\frac12$ orbital.

\section{Final notes about the energy spectra}

Concerning the orbital occupancies, we see that going to the
$0p\frac32-0f\frac72$ model space we have, in general, excitations from and to
the same orbitals that we had in the $0p\frac32-0d\frac32$ model space, but often
with a lower amount of particles being excited. The energy values of our
excited states in the $0p\frac32-0d\frac32$ model space are, for the most part,
too high, lying typically 1-2 MeV above the experimental states. Including the
$0p\frac32-0f\frac72$ model space brings this energy down to being 0.01-0.8 MeV
away from from the experimental states.


The orbital occupations of the ground state of $^{15}$B and $^{16}$C are very
similar. The orbital occupations for the excited states are also similar to
each other. This is as expected, as the two nuclei have the same number of
neutrons, and the only difference between the them is one proton in the
$0p\frac12$ orbital.

%The orbital occupations of the protons are also rather similar for $^{15}$B and
%$^{16}$C, which is more surprising. This shows that the binding energy of the
%filled $0p\frac32$ orbital is almost the same as the binding energy for 3
%protons in the $0p\frac32$ orbital. This similarity, coupled with how similar
%the neutron orbital occupations are, suggests the energy spectra will be
%similar to each other. Looking at figures \ref{fig:14C_brown_3pert_0d3_4part}
%and \ref{fig:16C_brown_3pert_0d3_4part} we see that they follow the same
%pattern. This would suggest that the third, fourth, fifth excited states of
%$^{15}$B will be very close to each other in energy. !Er litt usikker paa
%dette!

Notice that when we have six free neutrons ($^{14}$C and $^{16}$O) the ground
state will have between 0.5 and 1 neutrons in the $0d$ shell instead of the
$0p$ shell. And when we have eight free neutrons ($^{16}$C and $^{15}$B) we
generally have more neutrons in the $1s\frac12$ orbital than in the $0d\frac52$
orbital. When the nucleus is excited, particles in the $1s\frac12$ orbital are
excited to the $0d\frac52$ orbital. This shows that, as there begins to be more
than one particle in the in the $0d\frac52$ orbital, the $1s\frac12$ orbital
becomes more energetically favorable to excite to than the $0d\frac52$ orbital.


\section{E2 Transitions}


The transition strength of $^{14}$C is, as was mentioned in the introduction,
$3.7 e^2$fm$^4$ \citep{16CLifetime}. For $^{16}$C the transition strength is
$4.15 e^2$fm$^4$ \citep{16CE2}.  


To find the transition strength I need to specify the effective neutron and
proton charges. These charges need to reflect the constraints we have on our
model space, and are typically chosen as $1.0-1.5$. However, since we have such
a large model space, then our effective charges should be much closer to the
real charges of $0$ and $1$ for the neutron and proton, respectively.

Figure \ref{fig:C14E2} shows a 3-D plot of the transition strength for $^{14}$C
as a function of the neutron and proton charges, marked in red. The
experimental value is plotted in green. Figure \ref{fig:C16E2}
shows the neutron and proton charges for $^{16}$C that gives us the
experimental value of the transition strength at $0.474e^2$fm$^4$. The neutron
charge goes from $0.1-0.5$, and the proton charge goes from $1.0-1.4$. The
values are discussed below.

\fig{C14E2.eps}{Transition strength for $^{14}$C as a function of proton and neutron effective charge. The red field shows the calculated transition strength, while the green field shows the experimental transition strength. The important thing to notice here is that the red field never crosses the green field, and the red field decreases with the proton and neutron charge.}{fig:C14E2}

\fig{C16E2.eps}{The effective neutron and proton charges for $^{16}$C that correspond to the experimental transition strength at $0.474 e^2$fm$^4$.}{fig:C16E2}

\subsection{$^{14}$C}

For $^{14}$C we see that our data does not correspond so well with the
experimental value of $0.423 e^2$fm$^4$. We see that the transition strength
increases when we increase the proton or neutron charge, and it is lowest when
we have an effective charge of $0.1$ for the neutron and $1.0$ for the proton.
The transition strength is also more dependent on the proton charge than on the
neutron charge. For an effective charge of $0$ for the neutron and $1$ for the
proton we get a transition strength of $0.667 e^2$fm$^4$. This is $1.5$ times
the experimental value. This high value indicates that the E2 transition for
$^{14}$C is complicated, and cannot be accurately described by our model. There
are mainly three effects, in addition to the model space factor, that we have
not taken into account that might be strongly involved in the transition. One
is the resonances in $^{14}$C. The other is center-of-mass corrections, the
third is three-body interactions. As $^{14}$C has many resonant states, I
expect this to be the main contributor to the transition strength of these
three effect.
\\ !Er ikke sikker paa den siste setningen her, eller paa hvor stor innvirkning disse effektene kan ha for overgangsstyrken.! \\
Although we could
not predict the transition strength, the high proton charge dependency
%can be explained by looking at the ground state configuration in figure
%\ref{modellrom1}. Since the neutrons fill the $0p$ shell, they are not easily
%excited. The protons, filling the $0p\frac32$ orbital, are easier to excite.
is in accordance with our previous observations of the orbital occupancy of
$^{14}$C.

\subsection{$^{16}$C}

For $^{16}$C our data corresponds well with the experimental value of $0.474
e^2$fm$^4$. A good choice for the effective charges would be to set the proton
charge to $1.0$, as this value is closest to the real charge of the proton, and
because the E2 transition for $^{16}$C is mostly dependent upon the neutron
charge. For a proton charge of $1.0$, the neutron charge $1.4$ gives us the
transition strength $.470 e^2$fm$^4$, which corresponds very well with the
experimental value. That the transition is ruled by the neutron charge is in
accordance with our previous observations of the orbital occupancy of $^{16}$C.

