\kap{One Boson Exchange (OBE)}\label{chap:OBE}

The name of this model is misleading to the way it is being used today. Where it is 
a platform from where we can explane and simulate the strong force. The strong force is a product
of only meson exchanges. The weak force and EM force are created from other boson exchanges. So the name of the theory 
really says that it is a theory not only dealing with the strong force but also the weak force, EM force and even
gravitation (if the gravitation is a boson). This is however not the way this theory is build up, where all 
bosons except mesons are neglected. i.e. Leaving only the strong force.

The OBE model is an approximation of the real process where we have neglected higher order exchange diagrams.
It is normal to include the five lightest mesons in the model, and construct one fictive $\sigma$-meson.
The $\sigma$-meson then is supposed to include all the heavier mesons and also all the higher order meson exchange diagrams.
The potential vi can create from the OBE model is illustrated in  fig~\ref{figOBE}.
%\nl

\begin{figure}[!hbp]
\begin{center}
\begin{picture}(350,95)(0,0)
\Text(5,50)[l]{$\quad V_{{\text{OBE}}} =$}
\Text(55,50)[l]{$\quad\sum^{}_{\alpha=\pi,\eta,\rho,\omega,\delta,\sigma}\quad V^{{\text{OBE}}}_\alpha$}
\Text(190,50)[l]{$\equiv$} 
\Line(250,5)(250,95)
\Vertex(250,50){2}
\DashLine(250,50)(350,50){5}
\Line(350,5)(350,95)
\Vertex(350,50){2}
\Text(300,40)[c]{$\pi,\eta,\rho,\omega,\delta,\sigma$}
%\Text(160,50)[l]{$+$}
%\Line(180,5)(180,95)
%\Boxc(182,50)(4,40)
%\Text(190,50)[l]{$\Delta$}
%\DashLine(182,70)(240,70){5}
%\DashLine(182,30)(240,30){5}
%\Line(240,5)(240,95)
%\Text(210,80)[c]{$\pi,\rho$}
%\Text(210,20)[c]{$\pi,\rho$}
%\Text(250,50)[l]{$+$}
%\Line(270,5)(270,95)
%\Boxc(272,50)(4,40)
%\Text(280,50)[l]{$\Delta$}
%\DashLine(272,70)(330,70){5}
%\DashLine(272,30)(330,30){5}
%\Text(313,50)[l]{$\Delta$}
%\Line(330,5)(330,95)
%\Boxc(328,50)(4,40)
%\Text(300,80)[c]{$\pi,\rho$}
%\Text(300,20)[c]{$\pi,\rho$}
\end{picture}
\caption{\label{figOBE} \sl The standard OBE-model, where the five lightest mesons are included.
The $\sigma$-meson is artificial constructed to include an average of all the other possible exchanges.
Solid lines denotes nucleons.
}
\end{center}
\end{figure}

To calculate this potential, we need field theory, since particles with different spin will have different
properties. A correct model would involve QCD, since the mesons and nucleons are made up of quarks.
The fields we find from the mesons are therefor only an effective field, which only works for low energy NN interaction.
The mesons are grouped into different categories:
\begin{itemize}
\item Pseudo-scalar field ($ps$). Used for mesons with spin-0 like $\pi$ and $\eta$.
\item Scalar field ($s$). Mesons with spin-0 like $\delta$ and the fictive $\sigma$.  
\item Pseudo-vector field ($pv$). From a gradient coupling of the $ps$ field. 
This term is arising from
the chiral symmetry and is included as an effective $pv$ term.
In OBE models it is normal only to
include this effect in the $\rho$-meson and neglect the property in the $\omega$-meson. i.e. $f^\omega_{ps}=0$.
\item Vector field ($v$). Mesons with spin-1 like $\rho$ and $\omega$.
\end{itemize}
%The $\pi$ and $\eta$ are pseudo-scalar, $\sigma$ and $\delta$ are scalar, and $\rho$ and $\omega$ are vector
%mesons. For the $ps$ field, we also get a gradient coupling to the nucleon. This term is arising from
%the chiral symmetry and is included as an effective $pv$ term.

The Lagrangians that couples these fields can be found to be
\footnote{
Where
$\gamma^0=\left(
\begin{array}{rr}
1&0\\
0&-1
\end{array}\right)
$,
$\gamma^k=\left(
\begin{array}{rr}
0&\sigma^k\\
-\sigma^k &0
\end{array}\right)
$. The Pauli matrices($\sigma^k$) are:
%
$\sigma^1=\left(
\begin{array}{rr}
0&1\\
1&0
\end{array}\right)
$,
$\sigma^2=\left(
\begin{array}{rr}
0&-i\\
i&0
\end{array}\right)
$ and
$\sigma^3=\left(
\begin{array}{rr}
1&0\\
0&-1
\end{array}\right)
$,
$\gamma^5=\gamma_5=\gamma^1\gamma^2\gamma^3=\left(
\begin{array}{rr}
0&1\\
1&0
\end{array}\right)$
and
$\sigma^{\mu\nu}=\frac{i}{2}[\gamma^\mu,\gamma^\nu]$
}
:
\begin{eqnarray}\label{eq:Lagrang-ps}
{\mathcal L}_{ps}&=&-g_{ps}\bar{\psi}i\gamma_5\psi\phi^{(ps)}\\
%\end{equation}
%\begin{equation}\label{eq:Lagrang-s}
{\mathcal L}_{s}&=&g_{s}\bar{\psi}\psi\phi^{(s)}\\
%\end{equation}
%\begin{equation}
\label{eq:Lagrang-pv}
{\mathcal L}_{pv}&=&-\frac{f_{ps}}{m_{ps}}\bar{\psi}\gamma_5\gamma^\mu\psi\partial_\mu\phi^{(ps)}\\
%\end{equation}
%\begin{equation}
\label{eq:Lagrang-v}
{\mathcal L}_{v}&=&-g_{v}\bar{\psi}\gamma^\mu\psi\phi^{(v)}_\mu-\frac{f_v}{4M}\bar{\psi}\sigma^{\mu\nu}\psi(
\partial_\mu\phi^{(v)}_\nu-\partial_\nu\phi^{(v)}_{\mu})
\end{eqnarray}
M is the nucleon mass, $\psi$ denotes the Dirac spinor field of the nucleon and $\phi$ are the respective meson fields.
All the Lagrangian include a coupling constant $f$ and $g$. These coupling factors are relative large compared with other coupling 
constants in physics since we here are dealing with coupling constants arising from
the strong force.
The fist term on the right hand side of eq(\ref{eq:Lagrang-v}) is called vector coupling($v$), the second
for tensor coupling($t$).

From  these Lagrangian we are able to calculate the amplitude. The easiest way to do find this amplitude is
to use Feynman diagrams in a helisety basis.
\begin{figure}[!hbp]
\begin{center}
\begin{picture}(200,110)(0,0)
%\Text(5,50)[l]{$\quad V_{{\text{OBE}}} =$}
%\Text(55,50)[l]{$\quad\sum^{}_{\alpha=\pi,\eta,\rho,\omega,\delta,\sigma}\quad V^{{\text{OBE}}}_\alpha$}
%\Text(190,50)[l]{$\equiv$}
\ArrowLine(0,10)(50,50)
\ArrowLine(50,50)(0,90)
\Vertex(50,50){2}
\DashLine(50,50)(150,50){5}
\ArrowLine(200,10)(150,50)
\ArrowLine(150,50)(200,90)
\Vertex(150,50){2}
\Text(100,60)[c]{$m_\alpha$}
\Text(100,40)[c]{$q'-q$} 
\Text(40,50)[r]{${}_{\Gamma_1}$}
\Text(160,50)[l]{${}_{\Gamma_2}$}
\Text(0,0)[c]{$E,{{\bf q}}$}
\Text(0,100)[c]{$E',{{\bf q}}'$}
\Text(200,0)[c]{$E,-{{\bf q}}$}
\Text(200,100)[c]{$E',-{{\bf q}}'$}
\end{picture}
\caption{\label{figOBEFeynman} \sl 
Feynman diagram of a one meson exchange between two nucleons in the center of mass system.
}
\end{center}
\end{figure}
%
\begin{equation}\label{eq:amplitudefein} 
\braketm{{{\bf q}}'\lambda'_1\lambda'_2}{V}{{{\bf q}}\lambda_1\lambda_2}= 
\frac{
\bar{u}_{\lambda'_1}(-{{\bf q}}')\Gamma_1{u}_{\lambda'_1}(-{{\bf q}})
\bar{u}_{\lambda'_2}(-{{\bf q}}')\Gamma_2{u}_{\lambda'_2}(-{{\bf q}}) 
}{(q'-q)^2-m_{\alpha}^2}
\end{equation}
Using the relativistic eigen states also known as Dirac spinors 
\begin{equation}
{u}_{\lambda'_s}({{\bf q}})=\bigg(\frac{E+M}{2M}\bigg)\left(
\begin{array}{rr}
1\\
\frac{{{\bf \sigma}}\cdot{{\bf q}}}{E+M}
\end{array}\right)\chi_s
\end{equation}
So that
\begin{equation}
{u}_{\lambda}({{\bf q}})=\bigg(\frac{E+M}{2M}\bigg)\left(
\begin{array}{rr}
1\\
\frac{2\lambda q}{E+M}
\end{array}\right)\ket{\lambda}
\end{equation}
where $\lambda=\lambda_1 $or$ \lambda_2$, where $\lambda_1=\frac{1}{2}$ and $ \lambda_2=-\frac{1}{2}$. We also have $\bar{u}=u^\dagger\gamma^0$

The operator $\Gamma$ from the vertex in the Feynman diagrams will there for be (when using the Lagrangian above)
\begin{eqnarray}\label{eq:Gamma-ps}
\Gamma_{ps}&=&-\frac{g_{ps}}{\sqrt{4\pi}}\gamma_5\\
%\end{equation}
%\begin{equation}\label{eq:Lagrang-s}
\Gamma_{s}&=&-\frac{g_{s}}{\sqrt{4\pi}}\op{1}\\
%\end{equation}
%\begin{equation}
\label{eq:Gamma-pv}
\Gamma_{pv}&=&-\frac{if_{pv}}{\sqrt{4\pi}}\gamma^5\gamma^\mu\\
%\end{equation}
%\begin{equation}
\label{eq:Gamma-v}
\Gamma_{ps}&=&-\frac{g_{v}}{\sqrt{4\pi}}\gamma^\mu+\frac{if_{v}}{\sqrt{4\pi}2M}\sigma^{\mu\nu}(p'-p)
\end{eqnarray}
It is also normal to use
\begin{equation}
\frac{g_{ps}}{f_{ps}}=\frac{2M}{m_{ps}}
\end{equation}
whitch is on-mass-shell predictions where the $ps$ and the $pv$ coupling are identical.

Since the meson model doesn't include the break down at high energies where we have to deal with quarks instead of
mesons, it is necessary to introduce a form factor ${\cal F}[({{\bf q}}'-{{\bf q}})^2]$. The form factor has to be applied at each meson-nucleon vertex,
which is the same as multiplying ${\cal F}^2[({{\bf q}}'-{{\bf q}})^2]$ to the amplitude from eq(\ref{eq:amplitudefein}). This will
give us an amplitudes that converges.
\begin{equation}\label{eq:formfactor}
{\cal F}_\alpha[({{\bf q}}'-{{\bf q}})^2]=\bigg(\frac{\Lambda^2_\alpha-m^2_\alpha}{\Lambda^2_\alpha+
({{\bf q}}'-{{\bf q}})^2}\bigg)^{n_\alpha}
\end{equation}
Where $\Lambda$ is the cutoff parameter/mass. It is normal to use different $\Lambda$ for the mesons
in the OBE-model.

INKLUDER V(p)!!!!!!!!!!!!!!!!!!!!!!!!!!!!!!!!








A convenient way to  estimate the different parameters like the cutoffs($\Lambda$) and the $\sigma$-meson, 
is to partial wave compose the potential. In this frame one can fit very accurately the different partial waves 
with the experimental ones.

When the Bonn B parameters are fitted we find the parameters to be something like in table(\ref{tabBonnBparametere}),
which also include the rest of the Bonn B parameters like mass and spin.
\begin{table}[!hbp]
\begin{tabular}{|l|c|r|c|c|c|c|c|}
\hline 
    & $m_\alpha$ [MeV]& $g_\alpha^2\quad$ &$\frac{f_\alpha}{g_\alpha}$&$\Lambda$ [MeV]&$n_\alpha$&$J^P$&I\\
\hline
$\pi$ & 138.03&14.4000&0.00&1700&1&$0^-$&1\\
\hline
$\rho$& 769.00&0.9000&6.10&1850&2&$1^-$&1\\
\hline
$\eta$& 548.80&3.0000&0.00&1500&1&$0^-$&0\\
\hline
$\omega$& 782.60&24.5000&0.00&1850&2&$1^-$&0\\
\hline
$\delta$& 983.00&2.4880&0.00&2000&1&$0^+$&1\\
\hline
$\sigma$& 720.00&18.3773&0.00&2000&1&$0^+$&0\\
\hline
\end{tabular}
\caption{
\label{tabBonnBparametere}
\sl Bonn B parameters for the minimal relativity equation. Mass (${{\text m}}$), 
width ($\Lambda$), coupling constants (f and g), spin (${{\text J}}$), parity (${{\text P}}$), isospin (${{\text I}}$) 
}
\end{table}
