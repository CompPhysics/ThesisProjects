\documentstyle[a4wide]{article}
\newcommand{\OP}[1]{{\bf\widehat{#1}}}

\newcommand{\be}{\begin{equation}}

\newcommand{\ee}{\end{equation}}
\newcommand{\bra}[1]{\left\langle #1 \right|}
\newcommand{\ket}[1]{\left| #1 \right\rangle}
\newcommand{\braket}[2]{\left\langle #1 \right| #2 \right\rangle}


\begin{document}

\pagestyle{plain}

\section*{Thesis project for Espen Flage-Larsen:  Simulations of quantum dots}

{\bf The aim of this thesis is to study numerically systems consisting of several
interacting electrons in two dimensions}, confined to small regions
between layers of semiconductors. 
These electron systems
are dubbed quantum dots in the literature. 

In this thesis,
the study of such a system of two-dimensional electrons 
entails more specifically the use of many-body techniques through the 
development of large scale shell-model diagonalization techniques. These results will be compared 
with those from  
diffusion Monte Carlo  (DMC) and variational Monte Carlo (VMC) programs 
to solve Schr\"odinger's equation, in order to obtain various expectation values of interest, 
such as the energy
of the ground state, magnetization etc. 
The shell-model and VMC/DMC approaches allow, in principle, for a numerically
exact solution of Schr\"odinger's equation. 


\subsection*{Introduction to quantum dots}

Quantum computing has attracted much interest 
recently as it opens up the possibility of outperforming 
classical computation through new and more powerful quantum algorithms
such as the ones discovered by Shor and by Grover. 
There is now a growing list of quantum tasks 
such as cryptography, error correcting schemes, quantum
teleportation, etc. that have indicated even more the desirability 
of experimental implementations of quantum computing. 
In a quantum computer each quantum bit (qubit) is allowed
to be in any state of a quantum two-level system. 
All quantum algorithms can be implemented by 
concatenating one- and two-qubit gates. There is a growing number of proposed
physical implementations of qubits and quantum gates. 
A few examples are: Trapped ions, cavity QED, nuclear spins, 
superconducting devices, and
our qubit proposal 
based on the spin of the electron in quantum-confined nanostructures. 

Coupled quantum dots provide a powerful source of 
deterministic entanglement between qubits of localized 
but also of delocalized electrons. E.g., with such quantum gates it is
possible to create a singlet state out of two electrons 
and subsequently separate (by electronic transport) 
the two electrons spatially with the spins of the two electrons still being
entangled--the prototype of an EPR pair. 
This opens up the possibility to study a new class 
of quantum phenomena in electronic nanostructures 
such as the entanglement and
non-locality of electronic EPR pairs, tests of Bell inequalities, 
quantum teleportation, and quantum cryptography 
which promises secure information transmission. 


Semiconductor quantum dots are structures where
charge carriers are confined in all three spatial dimensions, 
the dot size being of the order of the Fermi wavelength 
in the host material, typically between  10 nm and  1 $\mu$m.
The confinement is usually achieved by electrical gating of a 
two-dimensional electron gas (2DEG), 
possibly combined with etching techniques. Precise control of the
number of electrons in the conduction band of a quantum dot 
(starting from zero) has been achieved in GaAs heterostructures. 
The electronic spectrum of typical quantum dots
can vary strongly when an external magnetic field is applied, 
since the magnetic length corresponding to typical 
laboratory fields  is comparable to typical dot sizes.
In coupled quantum dots Coulomb blockade effects, 
tunneling between neighboring dots, and magnetization 
have been observed as well as the formation of a
delocalized single-particle state. 


Quantum mechanical studies of such many-body systems are also
interesting per se. 
This thesis deals with a numerical {\em ab initio}  solution of  
Schr\"odinger's equation for quantum dots, from few electrons to many.
A critical understanding of our ability 
to solve such many-body systems through Monte Carlo methods is one 
of the aims
of this thesis project. Presently, large scale shell-model diagonalization and
Monte Carlo methods are
the only ones which allow us to solve, in principle exactly, 
systems with many interacting particles. 
The main focus is on large scale shell-model techniques. If possible these results will be compared 
with existing Variational Monte Carlo results. A diffusion Monte Carlo program is under 
development and can also be used to compare the results from shell-model diagonalizations.

The ability to study different methods is crucial to the reliability of each method.
This thesis project will thus be able to shed light on differences between VMC and DMC calculations
using the shell-model calculations as the exact frame. Presently, see Ref. [1] below, discrepancies exist between DMC and VMC calculations regarding the spin of the ground  and excited states. 

The shell-model program is  briefly sketched 
in the next section.


\subsection*{Shell-model studies}
Here we describe the philosophy behind the shell-model code developed here in Oslo by Torgeir Engeland. This code will be used in order to deal with 
the diagonalization of the hamiltonian for several electrons confined
in a quantum dot. 

The shell model problem requires the solution of a real symmetric
$n \times n$ matrix eigenvalue equation
\be
H\ket{vec_k} = E_k \ket{vec_k},
\label{e1}
\ee
with $k = 1,\ldots, K$. The eigenvalues $E_k$ are understood to be
numbered in increasing order. In a typical shell model problem
we are interested in only the lowest eigenstates of Eq.~(\ref{e1}),
so $K$ may be of the order of 10 to 50.
The total dimension $n$ of the eigenvalue matrix $H$ can be large,
up to $n \approx 2 \times 10^{8}$ or more.
Different computational approaches to solve Eq.(\ref{e1}) can
be distinguished
based on the size of $n$.
For $n$ small, i.e. $10^2 < n < 10^3$ and with  the number of
matrix elements of $H$
less than $10^6$ such  problems can be accomodated within the direct
access memory of a modern work station and can be diagonalized by
standard matrix routines.
In a second domain  with  $ n > 10^3$ but small enough that $H$
has no more than $10^8$ elements. This will require $\approx 1.5$~Gbyte
of storage.Then all matrix elements may be stored in memory
or alternatively on a standard disk.
In these cases the complete diagonalization of
$H$ will not be of physical interest and efficient iteration
procedures have been developed to find the lowest energy eigenvalues
and eigenvectors.

Based on the present computer methods we have developed a code
which is under continuous improvement
to solve the eigenvalue problem given in Eq.~(\ref{e1}).
The basic requirement
is to be able to handle problems with $n > 10^6$. In the following
we discuss some of the important elements which enter the algorithm.

We separate the discussion into two parts:
%
\begin{itemize}
%
\item The m--scheme representation of the basic states.
%
\item The Lanczos iteration algorithm.
%
%
\end{itemize}
%
\subsection*{\it The m--scheme representation.}
%
We write the eigenstates in Eq.~(\ref{e1})  as  linear combinations
of Slater determinants. In a second quantization representation
a Slater determinant (SD) is given by
%
\be
\ket{SD_{\nu}(N)} = \prod_{(jm)\in {\nu}} a_{jm}^{\dagger}\ket{0},
\ee
%
and the complete set is generated by distributing the $N$ particles
in all possible ways throughout the basic one--particle states constituting
the P--space. This is a very efficient
representation. A single $\ket{SD}$ requires only one  computer word
(32 or 64 bits) and  in memory a $\ket{SD}$ with $N$ particles
is given by
%
\be
\ket{SD} \longrightarrow (\underbrace{00111101010 \cdots}_{N 1's}),
\ee
%
where each 0 and 1 corresponds to an m--orbit in the valence
P--space. Occupied orbits  have a 1 and empty orbits a 0.
 Furthermore, all important calculations  can
be handled in Boolean algebra which is very efficient on modern computers.
The action  of operators of the form $a_{\alpha}^{\dagger} a_{\beta}$ or
$a_{\alpha}^{\dagger} a_{\beta}^{\dagger} a_{\gamma} a_{\delta}$
acting on an $\ket{SD}$ is easy to perform.

The $m$-scheme allows also for a straightforward definition of many-body operators 
such as one--, two-- and three--particle operators
     \begin{equation}
     a_{\alpha}^{\dagger} a_{\beta},
     \end{equation}
\begin{equation}
     a_{\alpha_1}^{\dagger}a_{\alpha_2}^{\dagger} a_{\beta_1} a_{\beta_2} ,
\end{equation}
\begin{equation}
     a_{\alpha_1}^{\dagger}a_{\alpha_2}^{\dagger}a_{\alpha_3}^{\dagger}
            a_{\beta_1} a_{\beta_2} a_{\beta_3},
\end{equation}
respectively.

\subsection*{\it The Lanczos iteration process.}
%
At present our basic approach to finding solutions to Eq.~(\ref{e1})
is the Lanczos algorithm. 
For the present discussion we outline the basic elements
of the method.
\begin{enumerate}
%
\item We choose  an initial Lanczos vector $\ket{lanc_0}$ as the zeroth order
approximation to the lowest eigenvector in Eq.~(\ref{e1}). Our experience
is that any  reasonable choice  is acceptable as long as the
vector does not have special properties such as good angular momentum.
That would usually terminate the iteration process at too early a
stage.
%
\item The next step involves generating a new  vector
through the process $|new_{p+1}> = H |lanc_p>$.
Throughout this process we construct the energy matrix elements
of $H$ in this Lanczos basis. First, the diagonal matrix elements of $H$
are then obtained by

%
\be
\bra{lanc_p} H \ket{lanc_p} = \bra{lanc_p} \left . new_{p+1} \right \rangle,
\label{lanc1}
\ee
%

\item The new vector $\ket{new_{p+1}}$ is then orthogonalized to all
previously calculated Lanczos vectors
%
\be
\ket{new_{p+1}^{'}} = \ket{new_{p+1}} - \ket{lanc_p} \cdot
	                \bra{lanc_p} \left . new_{p+1} \right \rangle		 - \sum_{q = 0}^{p-1} \ket{lanc_q} \cdot
	          \bra{lanc_q} \left . new_{p+1} \right \rangle,
\ee
%
and finally normalized

%
\be
\ket{lanc_{p+1}} = \frac{1}{\sqrt{\bra{new_{p+1}^{'}}
                      \left . new_{p+1}^{'} \right \rangle}}
						\ket{new_{p+1}^{'}},
\ee
%
to produce a new Lanczos vector.
%
\item The off--diagonal matrix elements of $H$ are calculated by
%
\be
\bra{lanc_{p+1}} H \ket{lanc_p} = \bra{new_{p+1}^{'}}
                                \left . new _{p+1}^{'}\right \rangle,
\label{off1}
\ee

%
and all others are zero.
%
\item After n iterations we have an energy matrix of the form
%
\be
\left \{
\begin{array}{ccccc}
H_{0,0} & H_{0,1} & 0       & \cdots   & 0  \\
H_{0,1} & H_{1,1} & H_{1,2} & \cdots   & 0  \\
0       & H_{2,1} & H_{2,2} & \cdots   & 0  \\
\vdots  & \vdots  & \vdots  & \vdots   & H_{p-1,p}  \\
0       & 0       & 0       & H_{p,p-1}   & H_{p,p}\\
\end{array}
\right \}
\label{matr1}
\ee
as the p'th approximation to the eigenvalue problem in Eq.~(\ref{e1}).
The number p is a reasonably small number and we can diagonalize
the matrix by standard methods to obtain eigenvalues and eigenvectors
which are linear combinations of the Lanczos vectors.
%
\item This process is repeated until a suitable convergence
criterium has been reached.
%
\end{enumerate}
%
In this method each Lanczos vector is a linear combination
of the basic $\ket{SD}$ with dimension $n$.




\section*{Progress plan}
A program for solving the diffusion Monte Carlo problem for fermions is under development as part of two other Master of Science thesis projects and will be used in this thesis in order to compare
with the shell-model results.
The task here is to use two-body effective interactions for a given set of single-particle orbits
and perform a shell-model diagonalization. The effective interaction will be provided by Morten
Hjorth-Jensen but will also be developed by Espen. 
A variational Monte Carlo program has already been developed and is documented in Victoria Popsueva's thesis. She will also work closely with Espen. She is presently a PhD student at UiB working on
quantum dot related problems. 



The thesis is expected to be finished towards the end  of the spring
semester of 2006.
\begin{itemize}
\item Fall 2004: 
      Exams in FYS4520 (Subatomic Many-Body Physics I), FYS4170 (Relativistic Quantum Field Theory) and  INF-MAT4350	Numerical Linear Algebra.
Thesis work: Set up a given model space using harmonic oscillator energies with and without a finite $B$-field and obtain the energies for the ground state and
excited states with no interaction. Use also the shell-model code to chech that the results are the same.
Develop a numerical code which computes the two-body interaction matrix elements using a two-dimensional harmonic oscillator basis.
\item Spring 2005:  Exams in FYS4530	Subatomic Many-Body Physics II, FYS5120
Advanced Quantum Field theory and INF5620	Numerical methods for partial differential equations.

Thesis work: Read in the two-body interaction and study the spectra of 
two electrons confined in a quantum dot. Compare the results with the analytic ones of Ref. [2].
\item Fall 2005: Thesis work: extensive shell-model calculations for systems with many electrons in several shells. Compare the results with the Variational
Monte Carlo code. 
\item Spring 2006: Finalize thesis and final exam. Eventually also compare the results with the diffusion Monte Carlo code.
\end{itemize}

\section*{Literature}
\begin{enumerate}
\item S.M. Reimann and M. Manninen, Reviews of Modern Physics {\bf 74}, 1283 (2002).  
\item M. Dyneykhan and R.G. Naszmitdinov, Physical Review {\bf B 55}, 13707 (1997). Exact solution for two electrons.
\item A. Harju, V.A. Sverdlov, R.M. Nieminen and V. Halonen, Physical Review {\bf B 59}, 5622 (1999). Variational Monte Carlo paper
\item F. Bolton, Physical Review {\bf B 54}, 4780 (1996). Diffusion Monte Carlo paper.
\item W. D. Heiss and R. G. Nazmitdinov, Physical Review {\bf B 55}, 16310 (1997). Shell-model paper.
\item A useful text on Matrix computations is Golub and Van Loan.
\end{enumerate}

\end{document}
