\documentclass{article}
\usepackage{graphicx} % Required for inserting images
\usepackage{hyperref}

\title{Machine learning and artificial intelligence for quantum-mechanical systems}
\author{Daniel Haas Beccatini Lima}
\date{February 2024}

\begin{document}

\maketitle  

\section{Scientific aims}


The aim of this PhD project is to develop many-body theories for
studies of strongly interacting quantum mechanical many-particle
systems using novel methods from deep learning theories, in particular
advanced neural networks and other generative models.

For interacting many-particle systems
where the degrees of freedom increase exponentially, quantum
mechanical many-body methods like quantum Monte Carlo methods, Coupled Cluster theory, Green's
function theories, density functional theories and other, play a
central role in understanding experiments in a wide range of fields,
spanning from atomic and molecular physics and thereby quantum chemistry to
condensed matter physics, materials science, nano-technologies and
quantum technologies and finally processes like fusion and fission in
nuclear physics. This list is definitely not exhaustive as the are many other
areas of applications for quantum mechanical many-body theories.


Quantum Monte Carlo techniques are widely applicable and have been
used in studies of a large range of systems.  The main difficulty with QMC
calculations of fermionic systems is ensuring the fermionic
antisymmetry is respected. In Diffusion Monte Carlo calculations,
which in principle yield the exact solutions in the limit of lon
simulation times, the prescription to ensure this is fixing the nodes
of the system to prevent the large errors that would come with the
summation of alternating signs. Unfortunately this prescription is not
variational in nature and can in some cases result in convergence to
energies lower than the true ground state of the system. This is one
of the advantages of VMC calculations since they are known to be
variational the resulting wavefunction will always have an energy
larger than or equal to the true ground state wavefunction.

The problem with variational Monte Carlo calculations has been the
choice of trial wave function.  Recently, several research groups have
introduced, with great success, neural networks as a way to represent
the trial wave function. Recent works on infinite nuclear
matter\cite{us2023a}, the unitary Fermi gas \cite{us2023b}, the
electron gas in three dimensions \cite{us2024}, have shown that one
can obtain results of equal accuratness as the theoretical benchmark
calculations provided by diffusion Monte Carlo results.  This has
opened up a new area of research and the present thesis project aims
at developing further deep learning approaches to the studies of
strongly interacting many-body systems. 

The plans here are to extend the areas of studies to low-dimensional
systems such as quantum dots and the infinite electron gas in two
dimensions. These are systems of great interest for materials science
studies, nano-technologies and quantum technologies. A proper
understanding of the properties of such systems will play a crucial
role in designing for example quantum gates and circuits.
These systems are studied experimentally
at the university of Oslo at the Center for Materials Science and
Nanotechnologies (SMN). The theoretical activity at the Center for
Computing in Science Education and the Computational Physics research
group have through the last years developed a strong collaboration
with several researchers at the SMN. 


Furthermore, following up the studies of infinite systems, we plan to
explore further properties of dense nuclear matter and neutron matter
as seen in neutron stars. The studies of \cite{us2023a} will be
expanded to a detailed with results of other many-body methods such as
Coupled Cluster theory and many-body perturbation, with the aim to
understand which correlations are introduced additionally by deep
learning methods.  These studies can be extended to studies of finite nuclei as well, with strong research overlaps with the nuclear physics group at the department of Physics.

\begin{thebibliography}{99}

\bibitem{us2023a} Dilute neutron star matter from neural-network quantum states by Fore et al, Physical Review Research 5, 033062 (2023)
\bibitem{us2023b} Neural-network quantum states for ultra-cold Fermi gases, Jane Kim et al, Nature Physics Communcication, in press and arXiv.2305.08831
\bibitem{us2024} Message-Passing Neural Quantum States for the Homogeneous Electron Gas, Gabriel Pescia, Jane Kim et al. arXiv.2305.07240 and Physical Review Letters, in press
\end{thebibliography}  


\end{document}




