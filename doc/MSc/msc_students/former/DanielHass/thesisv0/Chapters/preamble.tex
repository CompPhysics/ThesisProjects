\section*{Abstract}

The theoretical understanding behind the principles of quantum mechanics is just the first step towards solving quantum problems in real-world scenarios. Recently, a promising new field has emerged at the intersection of quantum variational methods and machine learning. This approach, named Neural Quantum States, takes advantage of neural networks as universal approximators to efficiently parametrise quantum systems.

This thesis explores the intersection of quantum many-body problems and machine learning, focussing on the application of neural network architectures to solve challenging quantum systems. We specifically investigate three methods: a standard variational Monte Carlo (VMC) parametrisation, a restricted Boltzmann machine (RBM), and a Deep Set feed-forward network (DSFFN). We applied these ansätze to two bound fermionic systems: a one-dimensional polarised fermionic system with Gaussian finite-range interaction, up to six particles and a two-dimensional quantum dots system with Coulomb interaction, up to 20 particles. The study employs Slater-Jastrow variants for the ansätze to impose fermionic antisymmetry, correlations, and cusp conditions. We experimented with various machine learning optimisation techniques, including an extensive Bayesian hyperparameter search, several well-known machine learning optimisers, and a stochastic reconfiguration, a quantum analogue of the natural gradient optimiser, known for its advantage in capturing the geometry of the quantum landscape.

Our implementation is based on Python, aiming at building a modern, modular framework with reasonable efficiency provided by using JAX as back-end. We compare the performance of these methods, discussing their strengths and limitations in representing quantum states efficiently. Our computational efficiency evaluation revealed reasonable scalability with JAX as back-end, though not matching C++ implementations as expected.

We successfully obtained ground-state energies, energy components, density profiles, and correlation energies for quantum systems replicating the results of other research studies with occasional better accuracy. Our results for 1D systems consistently surpassed Hartree-Fock and matched small-basis CI calculations. For 2D quantum dots, the addition of correlation factors repeatedly yielded lower energies than Hartree-Fock, approaching DMC calculations and even surpassing them in one instance using Stochastic Reconfiguration. We addressed the challenges of training neural networks in this unsupervised manner, highlighting the importance of reference energy values and correlation factors. Although more expressive models such as DSFFN showed excellent results, they were harder to train, requiring a balance between model complexity and user intuition. 



\newpage
\section*{Acknowledgements}

This thesis is what brought me to Oslo, but my friends are the reason I stay. I dedicate this work to all my friends and family. To those who started to love me because I came into their life and those who did not stop despite me leaving. None of this would be possible without the support of my family. None of this would be pleasant without the presence of my friends.

I would especially like to thank my supervisor, Morten Hjorth-Jensen, for all the guidance and support. Your enthusiasm and trust in whatever path I wanted to explore kept me motivated from the first day of this project onwards.

A special thank you to Nigar Abbasova, Adam Jakobsen, and Leah Hansen, for being the best second family I could have asked for. To Anna Aasen and Jonny Aarstad Igeh for taking me into the group and making me feel like I belonged somewhere, and to Hkon Kvernmoen, who, on top of that, endured my non-sensical questions about quantum mechanics, convincing me they were in fact not trivial. Finally, to everyone at CCSE: office mates, lunch companies, and table tennis rivals, thank you.