\chapter{Conclusion}\label{sec:conclusion}
\subsubsection{Concluding Points}

In conclusion, this thesis has explored the application of various machine learning techniques to solve quantum many-body problems, specifically focussing on one-dimensional trapped spinless fermions and two-dimensional quantum dots. By making use of neural networks such as Deep Set feed-forward networks (DSFFN) and restricted Boltzmann machines (RBM), we have demonstrated the potential of these networks to be used as variational Monte Carlo (VMC) functions, following ideas of reinforcement learning. From our findings, this approach enabled us to approximate ground-state energies and wavefunctions with good accuracy.

In particular, the effectiveness of neural network ansätze was evident as the DSFFN in combination to stochastic reconfiguration yielded in an isolated case, energy values below DMC calculations, and overall very good results. If an average analysis of the energies is made and if we disregard the stochastic reconfiguration method due to its difficult convergence behaviour, the RBM implementation was slightly favoured in the two-dimensional fermionic trap with Coulomb interaction. In contrast, for the one-dimensional case, where SR convergence was easy and used for all ansätze, the VMC without neural networks was optimal.

In general, the inclusion of correlation factors, such as the Jastrow and Padé-Jastrow factors, significantly improved the energy estimates, bringing them closer to the reference values of other research. This improvement from the addition of correlation factors was present in both one-dimensional and two-dimensional systems, and the Padé-Jastrow factor, in particular, was shown to be the best correlation factor for the Coulomb interaction potential.

We additionally were able to study correlations from one- and two-body density profiles, observing typical fermionic behaviour. In the spinless fermion system, we were able to observe crystallisation and bosonisation under the repulsive and interaction attractive regimes, respectively. This was also observed for the two-dimensional system as a function of the frequency of the trap. Similarly, we were able to study the distribution of energy components for both cases qualitatively, further showing that neural networks can be used to extract and understand the physics of simulated systems. For large quantum dots systems and lower frequency values, it was significantly hard for all ansätze to obtain equally good results. Nonetheless, the values obtained were still acceptable and below the Hartree-Fock energy.

The optimisation of hyperparameters through a Bayesian approach proved to be useful, yet misleading, in some cases. While it guided us to generally good choices, it prevented us from properly experimenting with the optimiser that eventually yielded the best results. For the two-dimensional quantum dot system, systematically lower energies were found with the SR reconfiguration method. However, this method was avoided by Bayesian optimisation search due to a rare convergence. Different optimisers, learning rates, and network architectures were explored, with RMSProp and Adam being overall the most robust for the tested configurations. In fact, SR yielded excellent results, but Adam and RMSProp were consistently good, rarely displaying divergence. 

The time scaling analysis indicated that the neural network models could handle an increasing number of particles with reasonable computational resources. Although Python showed slightly poorer scaling compared to C++ implementations, overall performance was not a constraint in our case.

In summary, we can confidently say that the work here developed demonstrated the promising capabilities of machine learning techniques in tackling quantum many-body problems. Our investigations contributed to an understanding of the newly conceptualised field of neural quantum states, while also providing meaningful results, comparable to other studies.


\subsubsection{Paths For Future Work}

This thesis concludes with several possible paths for future research. Given the reasonable quality of the results obtained, a natural progression would be to extend our methodologies to larger quantum systems. This expansion would allow us to investigate the limits of our methods in more complex scenarios, exploring different interaction potentials and, at the same time, higher-dimensional systems.

An orthogonal but equally important direction is to improve the accuracy and reliability of our quantum state optimisation. As already mentioned, our results obtained with the stochastic reconfiguration method demonstrated very positive results, but with unreliable behaviour. Then perhaps the most immediate path to try and improve this method is to investigate recent advancements in optimisation techniques, such as the Decision Geometry approach proposed by \cite{drissi2024second}. Implementing and evaluating these novel optimisation schemes could enhance the convergence and stability of our models.

Our current incorporation of physical constraints to the ansatz is also somewhat primitive. A natural path to improve this would be to extend our Deep Set implementations to utilise equivariant layers, constructing more general Slater determinants with the networks parametrising the single-particle states while embedding backflow correlations \cite{Luo_2019}. Similarly, exploring the integration of a Pfaffian ansatz, as demonstrated in the work of \cite{jane}, could be a fruitful direction.
 
Lastly, to enable more extensive parameter experimentation and improve ground-state minimisation for larger systems, it would be necessary to conduct a thorough profiling of our JAX implementation to identify and address performance bottlenecks. 











