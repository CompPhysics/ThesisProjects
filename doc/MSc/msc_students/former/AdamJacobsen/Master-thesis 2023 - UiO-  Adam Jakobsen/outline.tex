\documentclass{article}
\usepackage[utf8]{inputenc}
\usepackage{enumitem}

\title{Thesis outline}
\date{}



\begin{document}

\maketitle


\section{Introduction}
\begin{itemize}
    \item Introduce the topic of simulating electric waves in the heart and its importance in understanding cardiac function and disease.
    \item Explain motivation for using PINNs (Discuss the limitations of traditional numerical methods for solving these equations e.g. finite difference, finite element methods.
    \item Review existing work on PINNs(State of the Art), especially those relevant to cardiac electrophysiology simulations.
    \item Identify any gaps in the literature that our work will address (unstructured domains and MRI-based geometries).
    \item State research goals and hypotheses.
\end{itemize}

\section{Theory}
\begin{itemize}
    \item Briefly describe the fundamental concepts of deep learning and neural networks.
    \item Introduce the basics of cardiac electrophysiology and the relevant governing equations (physiology of the heart, cell models, ionic current models, and monodomain model).
    \item Explain the concept of PINNs and how they incorporate physical laws, with a focus on cardiac electrophysiology equations.
\end{itemize}

\section{Method}
[keep it simple]
\begin{itemize}
    \item Describe the architecture of our chosen PINN model (e.g., fully connected, GNN).
    \item Discuss the choice of activation functions, loss functions, and optimization techniques.
    \item Explain how the governing equations of cardiac electrophysiology are incorporated into the model.
    \item Explain the process of data preparation.
    \item Discuss the choice of hyperparameters and optimization procedures.
    \item Describe the training, validation, and testing splits for our dataset, as well as any ground truth data sources (results from FEM).
\end{itemize}

\section{Results}
\begin{itemize}
    \item Present the results of the PINN model in simulating electric waves in the heart, including comparisons with ground truth data (established numerical methods).
    \item Evaluate the model's performance in terms of accuracy, computational efficiency, and robustness.
    %\item Discuss the impact of hyperparameter choices and model architecture on the performance.
    \item Provide visualizations of the electric wave propagation to illustrate the model's capabilities.
\end{itemize}

\section{Discussion and Future Work}
\begin{itemize}
    \item Briefly restate the motivation, objectives, and main contributions of the thesis.
    \item Discuss the limitations of our approaches and compare them.
    \item Sggest possible improvements to the model or areas for further exploration.
    \item Describe potential applications of our work, such as early detection of cardiac diseases.
    \item Emphasise the significance of our findings in the context of cardiac electrophysiology modeling and the potential for future developments.
\end{itemize}

\section{Conclusion}
\begin{itemize}
    \item Open
\end{itemize}

\section{References}


\end{document}

