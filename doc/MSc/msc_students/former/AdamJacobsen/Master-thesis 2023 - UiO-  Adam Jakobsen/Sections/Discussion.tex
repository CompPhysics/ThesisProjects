\chapter{Discussion}
This study demonstrates the potential of PINNs for efficient simulations of cardiac electrophysiology. The training of PINNs was performed using various geometrical configurations, including 1D and 2D structured geometries, as well as  MRI-based geometries. The main, novel, contributions of this work are as follows:
\begin{itemize}
    \item An MRI-derived Cardiac Electrophysiology PINN model which includes image-informed scar regions with reduced conductivity.
    \item Extension of the PINN model to variable conductivity values, which  removes the need for the PINN to be retrained when simulating scenarios with differing conductivities, as occurs in real-world physiological and pathophysiological scenarios. 
\end{itemize}

\section{1D cable and 2D Square geometry}

The PINNs successfully reproduced the electrical wave propagation in 1D cable and 2D square geometries. The models showed high accuracy in capturing the dynamics.  Specifically, the RMSE values were consistently low across different training sample sizes and noise levels. For both the 1D and 2D scenarios, our results exhibit a strong concordance with the RMSE values reported in \cite{EP-PINNs}, which specified $\mathrm{RMSE}\leq 2.5 \times 10^{-2}$ for the 1D case when using $100$ training and $\mathrm{RMSE}\leq 3.0 \times 10^{-2}$ for the 2D structured grid scenario with $1.4\times 10^5$ training points, without added noise in both cases. If we exclude the outliers seen in Figure \ref{fig:RMSE_1D}, which were models getting stuck early on in training, we get $\mathrm{RMSE\leq 3.6\times10^{-2}}$ in the 1D scenario. In the 2D scenario (Figure \ref{fig:RMSE_2D}), we have $2.0\times10^{-3}\geq\mathrm{RMSE}\leq 1.5 \times 10^{-2}$ when using $10^4$ training points, even with normally distributed noise with a standard deviation of approximately $10\%$ of the maximum $V$ value.
Nevertheless, our models demonstrated significant variability when trained on smaller and noisier datasets (Figures \ref{fig:RMSE_1D} and \ref{fig:RMSE_2D}). Additionally, We conducted a more extensive investigation into the variability of the models compared to \cite{EP-PINNs}, which trained only 5 models, whereas we trained 10 models.
 Furthermore, the authors utilized DeepXDE~\cite{lu2021deepxde}, a library designed for Physics-Informed machine learning and employed a different optimization process consisting of first training on the data only with Adam for a few epochs, followed by training using the complete loss and finally finishing training with the L-BFGS optimization algorithm~\cite{L-BFGS}. Therefore, variations in both implementation and training procedures could have contributed to these observed differences.
The presence of extreme outlier in Figures \ref{fig:RMSE_1D} and \ref{fig:RMSE_2D},  can likely be attributed to the inherent stochastic nature of neural network training. Specifically, from the random initialization of the neural network parameters.
%In the  models also converged much faster, possibly due to the absence of input scaling in their approach, which may have led to the saturation of activation functions, resulting in very small gradients and consequently slow learning. However, to confirm this hypothesis, an examination of the gradients during training would be required.
%-Different implementation DeepXDE vs PyTorch
%-Our models did converge much faster: I think it's because they didn't scale the inputs, which means the activation function is getting saturated->gradients very small-> very slow learning.
%EP-PINNs 1D RMSE ≤ 2.5 × 10−2 . 2D RMSE < 3.0 × 10−2
%For both 1D and 2D scenarios, our findings align closely with the RMSE values documented in \\cite{EP-PINNs}, which indicated $\mathrm{RMSE}\leq 2.5 \times 10^{-2}$ for the 1D case and $\mathrm{RMSE}\leq 3.0 \times 10^{-2}$ for the 2D case.  Our models showed high variability when trained with smaller and noisy sample sizes(Figures \ref{fig:RMSE_1D} and \ref{fig:RMSE_2D}). We prune the variability of the models a bit more than EP-PINNs than only trained 5 models insted of 10 like we did.
%The outlier cases in Figure\ref{fig:box_plot_MRI} failed to converge which shows PINNs sensitivity to initialization.


\section{MRI-based 2D geometry with isotropic conductivities}
Extending the analysis to more anatomically accurate models, the novel isotropic MRI-based 2D geometry experiments further highlight the strengths of PINNs. The comparison between FEM and PINN results, particularly the relative $ L_1$ error values of 0.02 at both the 100 ms and 530 ms time points demonstrates the effectiveness of PINNs in handling complex geometrical structures derived from MRI data. In particular, this experiment shows that PINNs can reproduce the dynamics of electrophysiological wave propagation from sparse measurements ($100$ spatial points) and noisy measurements ($\sigma=8.0$), which is crucial in a clinical setting where data are often limited. Although this is a proof-of-concept and further refinement is needed, it provides a promising foundation for future clinical applications, supporting Hypothesis 2 by showing improved computational efficiency and maintaining accuracy in simulating these complex dynamics. 
In contrast to the work by Xie and Yao \cite{PDL}, which utilized a 3D geometry with a relatively sparse $1094$ vertices, our 2D model employed $72434$ vertices and achieved an \(RL^2\) error of $0.05$ compared to their \(RL^2\) error of $0.08$. Although these results are not directly comparable due to the dimensional difference, our initial experiments (structured geometries) suggest that there isn't significant loss in accuracy when transitioning from 1D to 2D with our approach. Therefore, similar accuracy could be maintained when extending our work from 2D to 3D.
This comparison is particularly relevant as prior work in this field is scarce, and Xie and Yao \cite{PDL} provide an example of an isotropic model in a 3D geometry, demonstrating that accurate PINN solutions are possible in such complex structures. However, it is important to note that the 3D study did not cover the modeling of scars nor MRI-derived geometries, which are unique contributions of our work. 
%number of collocation points is a limiting factor(memory usage).

%Scaling inputs (-2,2) was better than both standardization (mean 0 and unit variance) and scaling(-1,1). 


\section{MRI-based 2D geometry with anisotropic conductivities}
The anisotropic MRI-based 2D geometry experiment demonstrates the generalization capabilities of PINNs.
Figures \ref{fig:box_plot_MRI_aniso} and \ref{fig:RL1_anisotropic} reveal several noteworthy patterns. Contrary to initial expectations, there is no marked distinction in the relative $L^1$ error between training, interpolation, and extrapolation datasets. This could be attributed to the majority of the training dataset being unseen by the model, resulting in similar performance across different subsets. Additionally, there appears to be a slight increase in error with higher conductivity values. This trend could be explained by the fact that larger conductivity corresponds to faster wave speeds, which makes the model more susceptible to errors due to slight phase shifts. 
 %The relative \( L_1 \) error comparison for interpolating and extrapolating conductivities (Figure \ref{fig:RL1_anisotropic}) reveals a slight increase in error during extrapolation, yet it remains comparable to the training data error during interpolation. This indicates that while the model performs exceptionally well within the range of the training data, its accuracy slightly diminishes when predicting beyond this range. However, the error increase is not substantial, demonstrating the robustness of PINNs in handling varying conductivity scenarios.
 
This novel approach, unprecedented in the context of cardiac electrophysiology, is particularly noteworthy as it illustrates the possibility to create a neural network based solver capable of running different simulations in a fraction of a second by simply inputting different conductivities. In contrast, the FEM method necessitates a full repetition of the process for every new simulation. Additionally, the results show that the wave propagation slows down in regions of low conductivity, indicating that the PINN has successfully learned how conductivity affects wave propagation. These findings support Hypotheses 1 and 3, and underscore the significant advantages of using PINNs over traditional methods in terms of both speed and adaptability. 
The PINN training process can be slow and resource-intensive, presenting a significant investment in computational resources and time. However, our study demonstrates that this investment can be highly beneficial. Once trained, our PINN model can efficiently handle varying conductivities as inputs, offering significant flexibility, adaptability and fast inference times.
%Discuss inference times and training time: PINNs are expensive(GPUs and need a lot of memory) and slow to train, but we show that this investement can be worth it because our model can handle varying conductivities as inputs and inference time is very short and inexpensive.

%This novel approach, unprecedented in the context of cardiac electrophysiology provides a promising avenue 