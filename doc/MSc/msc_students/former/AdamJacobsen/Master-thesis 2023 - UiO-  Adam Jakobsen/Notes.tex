\documentclass{article}
\usepackage[svgnames]{xcolor}
\usepackage{amsmath}
\usepackage[most]{tcolorbox}
\usepackage[margin=1in]{geometry} 

\begin{document}
\newtcolorbox{goalbox}[1][]{
    colback=ForestGreen!5!white,
    colframe=ForestGreen!75!black,
    title=Goals,
    fonttitle=\bfseries\large\centering
    #1
}

\newtcolorbox{hypothesisbox}[1][]{
    colback=blue!5!white,
    colframe=blue!75!black,
    title=Hypotheses,
    fonttitle=\bfseries\large\centering
    #1
}

\newtcolorbox{methodbox}[1][]{
    colback=violet!5!white,
    colframe=violet!75!black,
    title=Methods for testing hypotheses,
    fonttitle=\bfseries\large\centering
    #1
}
\begin{goalbox}


\begin{enumerate}
\item[\textbf{G1:}] To evaluate the efficacy of PINNs in simulating electric waves in the heart compared to classical methods.
\item[\textbf{G2:}] To investigate the application of PINNs in simulating electric waves in MRI-based heart geometries.
\item[\textbf{G3:}] To investigate the ability of PINNs to extrapolate PDE parameters accurately.
\end{enumerate}
\end{goalbox}


\begin{hypothesisbox}

\begin{enumerate}
\item[\textbf{H1:}] The use of PINNs will result in faster simulations of electric waves in the heart compared to traditional methods.
\item[\textbf{H2:}] The application of Physics-informed Neural Networks will improve the computational efficiency of simulating electric wave propagation in MRI-based heart geometries.
\item[\textbf{H3:}] A trained PINN can effectively extrapolate PDE parameters beyond the range of training data, providing accurate and reliable predictions.
\end{enumerate}
\end{hypothesisbox}

\begin{methodbox}

\begin{enumerate}
\item[\textbf{M1:}] 
    \begin{enumerate}
        \item Develop a PINN model for simulating electric waves in the heart.
        \item Compare the performance of the PINN model against classical methods computational time, and convergence.
        \item Test the PINN's ability to make predictions using arbitrary spatial and temporal resolutions. 
        \item Perform a quantitative analysis using appropriate metrics.
    \end{enumerate}

\item[\textbf{M2:}] 
    \begin{enumerate}
        \item Train the PINN model on MRI-based heart geometries.
        \item Test the generalizability of the trained PINN model on unseen data.
        \item Evaluate the performance using relevant metrics and compare the results with the performance of classical methods.
    \end{enumerate}

\item[\textbf{M3:}] 
    \begin{enumerate}
        \item Train the PINN using a dataset with varying parameter ranges.
        \item Evaluate the accuracy and reliability of the predictions on a separate test set with PDE parameters outside the training range.
        \item Evaluate the PINN model's performance using appropriate evaluation metrics.
    \end{enumerate}
\end{enumerate}
\end{methodbox}

\newpage
\section{Notes}
Q:
\begin{itemize}
    \item level of detail 
    \item Discuss results for powerpoint
    
     
\end{itemize}

\begin{itemize}
    \item RMSE matrix: Might have a better way to sample experiments: longitudonal conductivities need more representations.
\end{itemize}



\subsection{TODO}

\begin{itemize}
    \item Upload videos github
    \item Scaled vs unscaled PINNs
    \item Get statistics 
    \item Matrix of conductivity-results
    
     
\end{itemize}

\textbf{Tentative deadlines}\\
\begin{table}[h]
  \centering
  \begin{tabular}{|c|c|c|}
    \hline
    \textbf{Date} & \textbf{Section} & \textbf{Notes} \\ \hline
    9.3 & Theory &  \\ \hline
    8.3 & Results w. stat & 2D square, iso, aniso, scaled vs unscaled \\ \hline
    29.3& Method section  & This could take longer \\ \hline
    - & - & - \\ \hline
  \end{tabular}
\end{table}




\subsection{DONE}
\begin{itemize}
    \item 19.09.23: Map out the outline
    \item Reorganized the code. Made it more modular + experiment folders with configs.
    \item Added utility: Picking GPU with the most free memory.
    \item Changed the PDE + ODE loss function: Using rescaled units, but the network still outputs Vm.
    \item Implemented residual-based adaptive refinement (sampling geomtime points and adding points where the residual is high until some threshold for the mean residual error is reached).
    \item  Using fewer points from the point cloud and resampling avoids memory issues.
    \item Hyperparameter tuning for learning rate, sampled points from domain and boundary, and resampling period.(Learning rate had the most importance)

    \item Conductivities as input code runs. Waiting for results.
    \item Implement fibers.(Fix memory issue) 
    

\end{itemize}


\subsection{RA work}
\begin{itemize}
\item Software review (what tools are out there? pros/cons, etc)
\item Bring lit- review up to date.
\item Missing references.
\item List what we have done.
\item Work on results for powerpoint.
\end{itemize}

\end{document}