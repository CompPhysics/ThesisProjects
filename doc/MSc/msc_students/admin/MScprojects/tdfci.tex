\documentstyle[a4wide]{article}
\newcommand{\OP}[1]{{\bf\widehat{#1}}}

\newcommand{\be}{\begin{equation}}

\newcommand{\ee}{\end{equation}}

\begin{document}

\pagestyle{plain}

\section*{Thesis title: Quantum-mechanical systems in traps and density functional theory}
The aim of this thesis is to study the structure of fermionic systems using
variational and diffusion function Monte Carlo techniques. The thesis will explore various Monte Carlo
optimalization strategies and use these to define the best possible density functional using the so-called adiabatic connection method.
Furthermore, if possible, the developed density functional will be applied in so-called time-dependent density functional approaches
to quantum mechanical systems with external time-dependen fields.
\subsection*{General introduction to possible physical systems}




Strongly confined electrons
offer a wide variety of complex and subtle phenomena which pose severe 
challenges to existing many-body methods.
Quantum dots in particular, that is, electrons confined in semiconducting heterostructures,
exhibit, due to their small size, discrete quantum levels. 
The ground states of, for example, circular dots
show similar shell structures and magic numbers 
as seen for atoms and nuclei. These structures are particularly evident in
measurements of the change in electrochemical potential due to the addition of
one extra electron, 
$\Delta_N=\mu(N+1)-\mu(N)$. Here $N$ is the number of electrons in the quantum dot, and
$\mu(N)=E(N)-E(N-1)$ is the electrochemical potential of the system.
Theoretical predictions of $\Delta_N$ and the excitation energy spectrum require
accurate calculations of ground-state and of excited-state energies.
Small confined systems, such as quantum dots (QD), have become very popular for experimental 
study \cite{MesoTran97,HeissQdotBook}. Beyond their possible relevance for nanotechnology, they are highly tunable 
in experiments and introduce level quantization and quantum interference in a controlled way. In a finite system, 
there can not, of course, be a true phase transition, but a cross-over between weakly and strongly correlated regimes is 
still expected. There are several other fundamental differences between quantum dots and bulk systems: (a)\,Broken translational 
symmetry in a QD reduces the ability of the electrons to delocalize. As a result, a Wigner-type cross-over 
is expected for a smaller value of $r_s$. (b)\,Mesoscopic fluctuations, inherent in any confined system \cite{MesoTran97,MesoHouches}, 
lead to a rich interplay with the correlation effects. These two added features make strong correlation physics particularly 
interesting in a QD. As clean 2D bulk samples with large $r_s$ are regularly fabricated these days in semiconductor 
heterostructures \cite{lowdens2DEG}, it seems to be just a matter of time before these systems are patterned into a QD, 
thus providing an excellent probe of correlation effects.



The above-mentioned quantum mechanical levels can, in turn, be tuned by means
of, for example, the application of various external fields.  
The spins of the electrons in quantum dots
provide a natural basis for representing so-called qubits \cite{divincenzo1996}. The capability to manipulate
and study such states is evidenced by several recent experiments (see, for example, Refs.~\onlinecite{exp1,exp2}).
Coupled quantum dots are particularly interesting since so-called  
two-qubit quantum gates can be realized by manipulating the 
exchange coupling which originates from the repulsive Coulomb interaction 
and the underlying Pauli principle.  For such states, the exchange coupling splits singlet and triplet states, 
and depending on the shape of the confining potential and the applied magnetic field, one can allow
for electrical or magnetic control of the exchange coupling. In particular, several recent experiments and 
theoretical investigations have analyzed the role of effective spin-orbit interactions 
in quantum dots \cite{exp5,exp6,pederiva2010,spinorbit} and their influence on the exchange coupling.

A proper theoretical understanding of the exchange coupling, correlation energies, 
ground state energies of quantum dots, the role of spin-orbit interactions
and other properties of quantum dots as well, requires the development of appropriate and reliable  
theoretical  few- and many-body methods. 
Furthermore, for quantum dots with more than two electrons and/or specific values of the 
external fields, this implies the development of few- and many-body methods where   
uncertainty
quantifications are provided.  
For most methods, this means providing an estimate of the error due 
to the truncation made in the single-particle basis and the truncation made in 
limiting the number of possible excitations.
For systems with more than three or four electrons,  {\em ab initio} methods that have 
been employed in studies of quantum dots are
 variational and diffusion Monte Carlo \cite{harju2005,pederiva2001, pederiva2003}, path integral approaches \cite{pi1999}, 
large-scale diagonalization (full configuration 
interaction) \cite{Eto97,Maksym90,simen2008,modena2000}, and to a very limited extent 
coupled-cluster theory \cite{shavittbartlett2009,bartlett2007,bartlett2003,indians,us2011}. 
Exact diagonalization studies are accurate for a very small number
of electrons, but the number of basis functions needed to obtain a given
accuracy and the computational cost grow very rapidly with electron number.
In practice they have been used for up to eight electrons\cite{Eto97,Maksym90,modena2000}, but the accuracy is
very limited for all except $N\le 3$ (see, for example, Refs.~\onlinecite{simen2008,kvaal2009}).  
Monte Carlo methods have been applied up to $N=24$ electrons 
\cite{pederiva2001,pederiva2003}. Diffusion Monte Carlo, with statistical and systematic errors, provide, in principle,
exact benchmark solutions to various properties of quantum dots. However, 
the computations start becoming rather time-consuming for larger systems.   
Hartree\cite{Kum90}, restricted Hartree-Fock, spin- and/or space-unrestricted
Hartree-Fock\cite{Fuj96,Mul96,Yan99} and
local spin-density, and current density functional methods\cite{Kos97,Hir99,finns1,finns2}
give results that are satisfactory for a qualitative understanding of some
systematic properties. However, comparisons with exact results show
discrepancies in the energies that are substantial
on the scale of energy differences. 



\subsection*{Specific tasks}




These calculations will in turn provide the basis for determining a as good as possible
ground state wave function. This wave function will in turn be used to define the quantum
mechanical density.  The density will be used to construct a density functional for quantum dots
using the adiabatic-connection method as described by Teale {\em et al} in J.~Chem.~Phys.~{\bf 130},
104111 (2009).  The results will be compared with existing density functionals for various quantum dots.

\section*{Progress plan and milestones}
The aims and progress plan of this thesis are as follows
\begin{itemize}
\item Fall 2012: Develop first a Hartree-Fock code for electrons trapped in a single harmonic oscillator  
in two dimensions.   This part entails developing a code for computing the Coulomb interaction
in two dimensions in the laboratory system.
\item The Hartree-Fock interaction is then used as input to the variational Monte Carlo code.
The results will be compared with large scale diagonalization and coupled-cluster
techniques for 2, 6, 12 and 20 
electrons in a single harmonic oscillator well.
\item Fall 2009: Write a code which solves the variational Monte-Carlo (VMC) problem
      for quantum dots. Both closed-core and open shell quantum dots will be studied. 
Construct  a Green's function Monte Carlo code
      based on the Variational Monte Carlo code. 
      The GFMC code receives as input the optimal 
      variational energy and wave function from the VMC calculation and solves
      in principle the Schr\"odinger equation exactly.
      The Slater determinant used in the VMC calculation includes also
      single-particle wave functions from Hartree-Fock calculations.
 \item Fall 2010: The obtained ground states will in turn be used to define a as exact as possible 
density functional for quantum dots
using the adiabatic-connection method. The density functional can in turn be used to model
systems with a large number of elctrons in quantum dots. Comparisons with a density functional derived from coupled-cluster methods will also be made in order to
test the validity of the Monte Carlo approach.
Possible applications are to mechanical studies of solar cells.
\end{itemize}
 


The thesis is expected to be handed in June 1 2013.


\begin{thebibliography}{200}

\bibitem{divincenzo1996} D.~Loss and D.~P.~DiVincenzo, Phys.~Rev.~A {\bf 57}, 120 (1998).
\bibitem{exp1} R.~Hanson, L.~H.~Willems van Beveren, I.~T.~Vink, J.~M.~Elzerman, W.~J.~M. Naber, F.~H.~L.~Koppens, L.~P.~Kouwenhoven, and L.~M.~K.~Vandersypen, \prl {\bf 94}, 196802 (2005).
\bibitem{exp2} H.-A.~Engel, V.~N.~Golovach, D.~Loss, L.~M.~K.~Vandersypen, J.~M.~Elzerman, R.~Hanson, and L.~P.~ Kouwenhoven, \prl {\bf 93}, 106804 (2004). 
\bibitem{exp5} C.~Fasth, A.~Fuhrer, L.~Samuelson, V.~N.~Golovach, and D.~Loss, \prl {\bf 98}, 266801 (2007).
\bibitem{exp6} S.~Roddaro {\em et al.}, \prl {\bf 101}, 18682 (2008).
\bibitem{pederiva2010}A.~Ambrosetti, J.~M.~Escartin, E.~Lipparini, and F.~Pederiva, arXiv:1003.2433.
\bibitem{spinorbit} F.~Baruffa, P.~Stano, and J.~Fabian, \prl {\bf 104}, 126401 (2010).
\bibitem{harju2005} A.~Harju, J.~Low Temp.~Phys.~{\bf 140}, 181 (2005).
\bibitem{bolton} F.~Bolton, Phys.~Rev.~B {\bf 54}, 4780 (1996).
\bibitem{pederiva2001}F.~Pederiva, C.~J.~Umrigar, and E.~Lipparini, \prb {\bf 62}, 8120 (2000).
\bibitem{pederiva2003}L.~Colletti, F.~Pederiva, and  E.~Lipparini,  Eur.~Phys.~J.~B {\bf 27}, 
385 (2002).
\bibitem{pi1999} M.~Harowitz, D.~Shin, and J.~Shumway, J.~Low.~Temp.~Phys.~{\bf 140}, 211 (2005).
\bibitem{Eto97} M. Eto, Jpn. J. Appl. Phys. {\bf 36}, 3924 (1997).
\bibitem{Maksym90} P.~A.~Maksym and T.~Chakraborty, Phys.~Rev.~Lett.~{\bf 65}, 108 (1990);
D.~Pfannkuche, V.~Gudmundsson, and P.~A.~Maksym, Phys. Rev. B {\bf 47}, 2244 (1993);
P.~Hawrylak and D.~Pfannkuche, Phys.~Rev.~Lett.~{\bf 70}, 485 (1993); J.J. Palacios,
L.~Moreno, G.~Chiappe, E.~Louis, and C.~Tejedor, Phys.~Rev.~B {\bf 50}, 5760 (1994);
T.~Ezaki, N.~Mori, and C.~Hamaguchi, Phys.~Rev.~B {\bf 56}, 6428 (1997).
\bibitem{simen2008} S.~Kvaal, Phys.~Rev.~C {\bf 78}, 044330 (2008).
\bibitem{simen2008b} S.~Kvaal, arXiv:0810.2644, unpublished.
\bibitem{modena2000} M.~Rontani, C.~Cavazzoni, 
D.~Belucci, and G.~Goldoni, J.~Chem.~Phys.~{\bf 124}, 124102 (2006).
\bibitem{shavittbartlett2009} I.~Shavitt and R.\ J.\ Bartlett, {\em Many-body Methods in Chemistry and Physics},  
(Cambridge University Press, Cambridge UK, 2009). 
\bibitem{bartlett2007} R.\ J.\ Bartlett and M.\ Musia{\l}, \rmp {\bf 79}, 291 (2007).
\bibitem{bartlett2003} T.~M.~Henderson, K.~Runge, and R.~J.~Bartlett, Chem.~Phys.~Lett.~{\bf 337}, 138 (2001); \prb {\bf 67}, 045320 (2003).
\bibitem{indians} I.~Heidari, S.~Pal, B.~S.~Pujari, and D.~G.~Kanhere, J.~Chem.~Phys.~{\bf 127}, 
114708 (2007).
\bibitem{kvaal2009} S.~Kvaal, Phys.~Rev.~B {\bf 80}, 045321 (2009).
\bibitem{Tar96} S. Tarucha, D.G. Austing, T. Honda, R.J. van der Hage, and L.P. Kouwenhoven,
Phys. Rev. Lett. {\bf 77}, 3613 (1996); Jpn. J. Appl. Phys. {\bf 36}, 3917 (1997);
S.~Sasaki, D.~G.~Austing, and S.~Tarucha, Physica B {\bf 256}, 157 (1998).
\bibitem{Note_DMC}Diffusion Monte Carlo  
calculations for $N=6$ and $N=12$, with $\omega=0.28$,
have been published in Ref.~\onlinecite{pederiva2001}. All the other 
results have been computed for this paper.
\bibitem{Kum90} A.~Kumar, S.~E.~Laux, and F.~Stern, Phys.~Rev.~B {\bf 42}, 5166 (1990).
\bibitem{Fuj96} M.~Fujito, A.~Natori, and H.~Yasunaga, Phys.~Rev.~B {\bf 53}, 9952 (1996).
\bibitem{Mul96} H.~M.~Muller and S.~Koonin, Phys.~Rev.~B {\bf 54}, 14532 (1996).
\bibitem{Yan99} C.~Yannouleas and U.~Landman, Phys. Rev. Lett. {\bf 82}, 5325 (1999).
\bibitem{Kos97} M. Koskinen, M. Manninen, and S.M. Reimann, Phys. Rev. Lett. {\bf 79}, 1389 (1997).
\bibitem{Hir99} K. Hirose and N. S. Wingreen, Phys. Rev. B {\bf 59}, 4604 (1999).
\bibitem{finns1} P.~Gori-Giorgi, M.~Seidl, and G.~Vignale, \prl {\bf 103}, 166402 (2009).
\bibitem{finns2} E.~R\"as\"anen, S.~Pittalis, J.~G.~Vilhena, M.~A.~L.~Marques, Int.~J.~Quantum Chem.~{\bf 110}, 2308 (2010). 
\bibitem{helgaker2003} T.~U.~Helgaker, P.~J{\o}rgensen, and J.~Olsen,
  \emph{Molecular Electronic Structure Theory. Energy and Wave Functions}, (Wiley, New York, USA, 2000).
\bibitem{ccsdt-n} Y.~S.~Lee, S.~A.~Kucharski, and R.~J.~Bartlett, 
J.~Chem.~Phys. {\bf 81}, 5906 (1984); {\it ibid} {\bf 82}, 761 (E) (1982);
J.~Noga, R.~J.~Bartlett, and M. Urban, Chem.~Phys.~Lett.~{\bf 134}, 126 (1987).
\bibitem{Deegan94} M.~J.~O.~Deegan and P.~J.~Knowles, Chem.~Phys.~Lett. \textbf{227}, 321 (1994).
\bibitem{Kucharski98} S.~A.~Kucharski and R.~J.~Bartlett, J.~Chem.~Phys. \textbf{108}, 5243 (1998).
\bibitem{crawford1998} T.~D.~Crawford and J.~F.~Stanton, Int.~J.~Quantum Chem.~{\bf 70}, 601 (1998).
\bibitem{Taube08} A.~D.~Taube and R.~J.~Bartlett, J.~Chem.~Phys. \textbf{128}, 044110 (2008).
\bibitem{taube2} A.~D.~Taube and R.~J.~Bartlett, J.~Chem.~Phys. {\bf 128}, 044111 (2008).

\bibitem{hagen2008} G.~Hagen, T.~Papenbrock, D.~J. Dean, and M.~Hjorth-Jensen, 
Phys.~Rev.~Lett.~{\bf 101}, 092502 (2008).
\bibitem{hagen2009} G.~Hagen, T.~Papenbrock, D.~J.~Dean, M.~Hjorth-Jensen, and B.~Velamur Asokan, Phys.~Rev.~C {\bf 80}, 021306 (2009).
\bibitem{hagen2010a}  G.~Hagen, T.~Papenbrock, and M.~Hjorth-Jensen, \prl {\bf 104}, 182501 (2010). 
\bibitem{hagen2010b}  G. Hagen, T.~Papenbrock, D.~J.~Dean, M.~Hjorth-Jensen, Phys.~Rev.~C {\bf 82}, 034330 (2010).
\bibitem{schneider2008}  R.~Schneider, Numer.~Math.~{\bf 113}, 433 (2009). 
\bibitem{navratil1} P.~Navr\'atil and B.~R.~Barrett, Phys.~Rev.~C \textbf{57}, 562 (1998).
\bibitem{navratil2} P.~Navr\'atil, J.P.~Vary, and B.~R.~Barrett, 
Phys.~Rev.~Lett. \textbf{84}, 5728 (2000).
\bibitem{Ash96} R.C. Ashoori, Nature {\bf 379}, 413 (1996);
L.~P.~Kouwenhoven, T.~H.~Oosterkamp, M.~W.~Danoesastro, M.~Eto, D.~G.~Austing,
T.~Honda and S.~Tarucha, Science {\bf 278}, 1788 (1997).
\bibitem{taut1994} M.~Taut, J.~Phys.~A: Math.~Gen.~{\bf 27} 1045 (1994).
\bibitem{mhj1995} M.~Hjorth-Jensen, T.~T.~S. Kuo, and E.~Osnes, Phys.~Rep.~{\bf 261}, 125 (1995).
\bibitem{navratil2009}   P.~Navratil, S.~Quaglioni, I.~Stetcu, and B.~R.~Barrett, J.~Phys.~G {\bf 36}, 08310 (2009).
\bibitem{navratildots} K.~Varga, P.~Navr\'atil, J.~Usukura, and Y.~Suzuki, \prb {\bf 63}, 205308 (2001).
\bibitem{kvaal2007} S.~Kvaal, M.~Hjorth-Jensen, and H.~M\o ll Nilsen, \prb {\bf 76}, 085421 (2007).
\bibitem{UWW88} C.~J.~Umrigar, K.~G.~Wilson and J.~W.~Wilkins, in {\it Computer
Simulation Studies in Condensed Matter Physics: Recent Developments},
edited by D.P. Landau and H.~B.~Sch\"uttler (Springer-Verlag, Berlin, 1988);
Phys.~Rev.~Lett.~{\bf 60}, 1719 (1988).
\bibitem{Umr93} C.~J.~Umrigar, Nightingale, and K.~J.~Runge, J.~Chem.~Phys.
{\bf 99}, 2865 (1993).
\bibitem{hdhk2010} M.~Hjorth-Jensen, D.~J.~Dean, G.~Hagen, and S.~Kvaal, J.~Phys.~G {\bf 37}, 064035 (2010).
\bibitem{Chi98} C.~J.~Huang, C.~Filippi, and C.~J.~Umrigar, J.~Chem.~Phys.~{\bf 108},
8838 (1998).
\bibitem{kutzelnigg1991} W.~Kutzelnigg, Theor.~Chim.~Acta {\bf 80}, 349--386 (1991)
\bibitem{klein1974} D.~J.~Klein, J.~Chem.~Phys. {\bf  61}, 786--98 (1974)
\bibitem{hagen2011} G.~Hagen, M.~Hjorth-Jensen, S.~Kvaal and F.~Pederiva, unpublished.
\bibitem{ghosal2007} A.~Ghosal, A.~D.~G\"{u}\c{c}l\"{u}, C.~J.~Umrigar, D.~Ullmo, and H.~U.~Baranger, Phys.~Rev.~B
{\bf 76}, 085341 (2007).
\bibitem{gustav2010} G.~R.~Jansen, M.~Hjorth-Jensen, G.~Hagen, and T.~Papenbrock, 
Phys.~Rev.~C {\bf 83},  054306 (2011)


\bibitem {ref1}  A.~M.~Teale, S.~Coriani, and T.~Helgaker, J.~Chem.~Phys.~{\bf 130},
104111 (2009).


\end{thebibliography}



\end{document}



