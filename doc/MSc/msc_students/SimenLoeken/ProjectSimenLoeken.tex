\documentclass{article}
\usepackage[utf8]{inputenc}
\usepackage[margin=1.2in]{geometry}
% Math
\usepackage{amsmath}
\usepackage{physics}
\usepackage{isotope}

% Citetations
%\usepackage[ backend=bibtex, sorting=none, autocite=superscript]{biblatex}
%\addbibresource{refs/references}
\usepackage{xcolor}
\usepackage{hyperref}
\hypersetup{
    colorlinks,
    linkcolor={red!50!black},
    citecolor={blue!50!black},
    urlcolor={blue!80!black}}


% Formatting
\usepackage{float}
\usepackage{graphicx}
\graphicspath{ {./figs/} } 

% Misc
\usepackage{appendix}

% Commands
\newcommand{\comment}[1]{\textcolor{red}{#1}}
\newcommand{\core}[3]{^{#2}_{#3}\text{#1}}

\title{Time Evolution of Quantum Mechanical systems \\ \vspace{5px} \large Implementation of Time-Dependent Coupled Cluster methods \\ \vspace{20px} \large Master of Science thesis project}
% \author{Håkon Kvernmoen}
\date{November 2022}

\begin{document}
\maketitle

\subsection*{Introduction and overview}

The aim of this project is the study of time evolution in fermionic
systems through the use of quantum many-body theory. Quantum many-body
theory has provided methods to solve problems in such diverse areas as
atomic, molecular, solid-state and nuclear physics, chemistry and
materials science. In the past decades, static properties such as
binding energies and various expectation values have been
calculated. The introduction of time in these calculations yields an
insight into the dynamics of quantum mechanical systems, such as the
electron behavior under an external potential in quantum dots and the
consolidation of composite systems undergoing nuclear
fusion. Specifically, the goal is to implement the time dependent
version of the Coupled Cluster approach
\cite{Kvaal2012,Schoyen2019,Pigg2012,Hagen2014,Lietz2017}, a method containing a plethora
of desired properties such as size consistency and extensivity. The
results can be compared with exisiting codes using time-dependent full
configuration interaction theory \cite{Skattum2013,Hochstuhl2014}.

\subsection*{General introduction to possible physical systems.}

The code developed during this thesis will be written in such a way
that different fermionic system can be handled. Examples of potential
applications are ions confined in various traps
\cite{Reimann2002}. Strongly confined electrons offer a wide
variety of complex and subtle phenomena which pose severe challenges
to existing many-body methods. Quantum dots in particular, that is,
electrons confined in semiconducting heterostructures, exhibit, due to
their small size, discrete quantum levels. The ground states of, for
example, circular dots show similar shell structures and magic numbers
as seen for atoms and nuclei. These structures are particularly
evident in measurements of the change in electrochemical potential due
to the addition of one extra electron.

Application in nuclear physics might also be of
interest\cite{Pigg2012}, where the Coupled Cluster method was
originally introduced. In particular, the process of two $\alpha$-particles
fusing into Beryllium $(\isotope[4][2]{He} + \isotope[4][2]{He}
\xrightarrow{} \isotope[8][4]{Be})$ could serve as the ultimate goal
of the thesis, being part of the tripe-alpha process, the backbone of
stellar nucleosynthesis.

\subsection*{Specific tasks and milestones}

The specific task here is to study the time evolution of quantum
mechanical systems using the Coupled Cluster (CC) method, in order to
be able to study the time evolution of an interacting quantum
mechanical system. In order to achieve this, the following milestones outline the thesis:

\begin{enumerate}
    \item Spring 2023: Start writing a time-independent Coupled Cluster code with double excitations first following \cite{Lietz2017} and applied to a small confined fermionic system in one dimension. Here we have in mind electrons confined in harmonic oscillator traps \cite{Zanghellini2004} and/or the Calogero-Sutherland \cite{Calogero1971} type of Hamiltonians which provide analytical answers to many-body problems. Singles excitations can be added later. Finalize remaining courses. 
    \item Fall 2023: Extend the program from spring 2023 to include singles excitations and time-dependence without time-dependence for single-particle orbitals, see Ref.~\cite{Pigg2012}. Compare with existing results obtained with full configuration theory as done in \cite{Skattum2013,Hochstuhl2014}.
    \item Spring 2024: Extend the program to include time-dependence of the single-particle orbitals and compare with corresponding full configuration theory calculations. Apply to systems of quantum dots and/or the fusion of two $\alpha$-particles. Finalize thesis and present final results.
\end{enumerate}
The thesis is expected to be handed in May/June 2024


\begin{thebibliography}{99}

\bibitem{Kvaal2012} Simen Kvaal, Journal of Chemical Physics {\bf 136}, 194109 (2012), \url{ https://doi.org/10.1063/1.4718427}
\bibitem{Schoyen2019} Øyvind Schøyen Sigmundson, Master of Science Thesis, University of Oslo (2019), \url{https://www.duo.uio.no/handle/10852/72881}
\bibitem{Hagen2014} Gaute Hagen, Thomas Papenbrock, Morten Hjorth-Jensen, and David J. Dean, Reports on Progress in Physics  {\bf 77}, 096302 (2014), \url{10.1088/0034-4885/77/9/096302}
\bibitem{Lietz2017} Justin Lietz, Samuel Novario, Gustav R. Jansen, Gaute Hagen, and Morten Hjorth-Jensen, Lecture Notes in Physics {\bf 936}, 293 (2017), \url{https://link.springer.com/chapter/10.1007/978-3-319-53336-0_8}
\bibitem{Pigg2012} David Pigg, Gaute Hagen, H. Nam, and Thomas Papenbrock, Physical Review C {\bf 86}, 014308 (2012), \url{https://doi.org/10.1103/PhysRevC.86.014308}
\bibitem{Skattum2013} Sigve Skattum, Master of Science Thesis, University of Oslo (2013), \url{https://www.duo.uio.no/handle/10852/37170}  
\bibitem{Hochstuhl2014} D. Hochstuhl, C.M. Hinz, and M. Bonitz, The European Physical Journal Special Topics {\bf 223}, 177 (2014), \url{https://link.springer.com/article/10.1140/epjst/e2014-02092-3}
\bibitem{Reimann2002} Stephanie M. Reimann and Matti Manninen, Reviews of  Modern Physics {\bf 74}, 1283 (2002), \url{https://doi.org/10.1103/RevModPhys.74.1283}
\bibitem{Zanghellini2004} J. Zanghellini, M. Kitzler, T. Brabec, and A. Scrinzi, Journal of Physics  B {\bf 37}, 763 (2004), \url{https://doi.org/10.1088/0953-4075/37/4/004}
\bibitem{Calogero1971} F. Calogero, Journal of Mathematical Physics {\bf 12}, 419,  (1971), \url{https://doi.org/10.1063/1.1665604}

\end{thebibliography}  
%\printbibliography

\end{document}
