\documentclass{article}
\usepackage[utf8]{inputenc}
\usepackage[margin=1.2in]{geometry}
% Math
\usepackage{amsmath}
\usepackage{physics}
\usepackage{isotope}

% Citetations
%\usepackage[ backend=bibtex, sorting=none, autocite=superscript]{biblatex}
%\addbibresource{refs/references}
\usepackage{xcolor}
\usepackage{hyperref}
\hypersetup{
    colorlinks,
    linkcolor={red!50!black},
    citecolor={blue!50!black},
    urlcolor={blue!80!black}}


% Formatting
\usepackage{float}
\usepackage{graphicx}
\graphicspath{ {./figs/} } 

% Misc
\usepackage{appendix}

% Commands
\newcommand{\comment}[1]{\textcolor{red}{#1}}
\newcommand{\core}[3]{^{#2}_{#3}\text{#1}}

\title{Deep Learning and Quantum Many-body problems \\ \large Master of Science thesis project}
% \author{Simen Løken}
\date{November 2022}

\begin{document}
\maketitle

\subsection*{Introduction and overview}

Predicting the structure of quantum many-body systems from first
principles in quantum mechanics is a central challenge in physics,
chemistry, and material science. Deep learning techniques have
recently proven to be powerful tools for solving interacting quantum
mechanical problems in condensed matter physics, atomic and molecular
physics and quantum chemistry, nuclear physics and materials science.


The above implies solving the Schrödinger equation for systems of many interacting
bosons or fermions. This  is classified as an NP-hard problem due to the
complexity of the required many-dimensional wave function, resulting
in an exponential growth of degrees of freedom. Reducing the
dimensionalities of quantum mechanical many-body systems is an
important aspect of modern physics, ranging from the development of
efficient algorithms for studying many-body systems to exploiting the
increase in computing power. To write software that can fully utilize
the available resources has long been known to be an important aspect
of these endeavors. Despite tremendous progress has been made in this
direction, traditional many-particle methods, either quantum
mechanical or classical ones, face huge dimensionality problems when
applied to studies of systems with many interacting particles.

Over the last two decades, quantum computing and machine learning have
emerged as some of the most promising approaches for studying complex
physical systems where several length and energy scales are involved.
Machine learning techniques and in particular neural-network quantum
states~\cite{Goodfellow2016} have recently been applied to studies of
many-body systems, see for example
Refs.~\cite{carleo_solving_2017,carra2021,pfau2019abinitio,calcavecchia_sign_2014,carleo2019,amber2022,Lovato2021,lovato2022},
in various fields of physics and quantum chemistry, with very
promising results. In many of these studies, one has obtained results
that align well with exact analytical solutions or are in close
agreement with state-of-the-art quantum Monte Carlo calculations.

The variational and diffusion Monte Carlo algorithms are among the
most popular and successful methods available for ground-state studies
of quantum mechanical systems. They both rely on a suitable ansatz for
the ground-state of the system, often dubbed {\it trial wave
  function}, which is defined in terms of a set of variational
parameters whose optimal values are found by minimizing the total
energy of the system. Devising flexible and accurate functional forms
for the trial wave functions requires prior knowledge and physical
intuition about the system under investigation. However, for many
systems we do not have this intuition, and as a result it is often
difficult to define a good ansatz for the state function.

According to the universal approximation theorem, a deep neural
network can represent any continuous function within a certain error
\cite{hornik_multilayer_1989} --- see also
Refs.~\cite{Murphy2012,Hastie2009,Bishop2006,Goodfellow2016} for
further discussions of deep leaning methods. Since the variational
state wave function in principle can take any functional form, it is
natural to replace the trial wave function with a neural network and
treat it as a machine learning problem. This approach has been
successfully implemented in recent works, see for example
Refs.~\cite{pfau2019abinitio,carleo_solving_2017,casella2022,Lovato2021,lovato2022},
and forms the motivation for the present study. Here, the neural
network of choice was derived from so-called restricted Boltzmann
machines, much inspired by the recent contributions by Carleo {\em et
  al.}, see for example Refs.~\cite{carleo_solving_2017,carleo2019}.
Note that neural-networks representations of variational states are
more general, as they do not in principle require prior knowledge on
the ground-state wave function, thereby opening the door to systems
that have yet to be solved. Particular attention however has to be
devoted to the symmetries of the problem, whose inclusion is critical
to achieve accurate results..




\subsection*{Thesis Project}

The aim here is to study systems of confined fermions and bosons in one and two dimensions.
Strongly confined electrons offer a wide variety of complex and subtle
phenomena which pose severe challenges to existing many-body methods.
Quantum dots in particular, that is, electrons confined in
semiconducting heterostructures, exhibit, due to their small size,
discrete quantum levels.  The ground states of, for example, circular
dots show similar shell structures and magic numbers as seen for atoms
and nuclei. These structures are particularly evident in measurements
of the change in electrochemical potential due to the addition of one
extra electron.

Small confined systems, such as quantum dots (QD), have become very
popular for experimental study.  Beyond their possible relevance for
nanotechnology, they are highly tunable in experiments and introduce
level quantization and quantum interference in a controlled way.

Similarly, other fermionic systems like atoms, molecules, nuclei, the
infinite homogeneous electron gas and infinite nuclear matter, are all
systems which can be studied with the same many-body methods.
Thus, a proper theoretical understanding of such systems requires the
development of appropriate and reliable theoretical few- and many-body
methods.  Furthermore, for say quantum dots with more than two electrons
and/or specific values of the external fields, this implies the
development of few- and many-body methods where uncertainty
quantifications are provided.  For most methods, this means providing
an estimate of the error due to the truncation made in the
single-particle basis and the truncation made in limiting the number
of possible excitations.  For systems with more than three or four
electrons, \textbf{ab initio} methods that have been employed in studies of
quantum dots are variational and diffusion Monte Carlo, path integral
approaches, large-scale diagonalization (full configuration
interaction and to a more limited extent coupled-cluster theory.
Exact diagonalization studies are accurate for a very small number of
electrons, but the number of basis functions needed to obtain a given
accuracy and the computational cost grow very rapidly with electron
number.  In practice they have been used for up to eight electrons,
but the accuracy is very limited for all except $N\le 3$ .  Monte
Carlo methods have been applied up to $N\sim 100$ electrons. Diffusion
Monte Carlo, with statistical and systematic errors, provide, in
principle, exact benchmark solutions to various properties of quantum
dots. However, the computations start becoming rather time-consuming
for larger systems.  Mean field methods like various Hartree-Fock
approaches and/or current density functional methods give results that
are satisfactory for a qualitative understanding of some systematic
properties. However, comparisons with exact results show discrepancies
in the energies that are substantial on the scale of energy
differences. The above-mentioned many-body methods all experience what
is the loosely called the \emph{curse of dimensionality}. This means that
the increased number of degrees freedom hinders the application of
most first principle methods. As an example, for direct
diagonalization methods, Hamiltonian matrices of dimensionalities
larger than ten billion basis states, are simply computationally
intractable. Such a dimensionality translates into few interacting
particles only. For larger systems one is limited to much more
approximative methods.  Reecent approaches in Machine Learning as well
as in quantum computing, hold promise however to circumvent partly the
above problems with increasing degrees of freedom.

The specific aim of this
thesis topic is to, based on quantum Monte Carlo methods, to
explore deep learning approaches to many-body systems based on neural networks, see the recent work of for example
\href{{https://arxiv.org/abs/2007.14282}}{Adams et al.}. 

\paragraph{Specific tasks and milestones.}

The specific task here is to implement and study deep learning methods 
for solving quantum mechanical many-particle
problems of fermions. The results can  be easily compared with exisiting standard
many-particle codes developed by former students at the Computational
Physics group. These codes will serve as useful comparisons in order
to gauge the appropriateness of recent Machine Learning approaches to
quantum mechanical problems.


The basic framework to be developed contains:
\begin{enumerate}

\item The development of a Variational Monte Carlo code for fermions.

\item The development of a trial wave function for the correlated part based on deep neural networks.

\item Be able to study quantum mechanical systems with interactions for bosons and fermions in one and two dimensions.

\item Compare results with those obtained with recent libraries like FermiNet, SchNet and other.

\end{enumerate}

\noindent
The milestones are

\begin{enumerate}
\item Spring semester 2023: Develop a Variational Monte Carlo framework for fermions without a self-consistent single-particle basis. Parts of this will be done in the course FYS4411, Computational Quantum Mechanics.


\item Fall semester 2023: Introduce neural networks for the correlated part of the trial wave function used in Monte Carlo studies. Study simpler Hamiltonians in one and two dimensions like those described in \cite{Zanghellini2004,Calogero1971}. 

\item Spring semester 2024: Perform numerical studies of quantum systems involving quantum dots in two dimensions and compare with results from other libaries.

\item Spring semester 2024: Final thesis
\end{enumerate}

\noindent
The thesis is expected to be handed in May/June  2024.


%merlin.mbs apsrev4-1.bst 2010-07-25 4.21a (PWD, AO, DPC) hacked
%Control: key (0)
%Control: author (0) dotless jnrlst
%Control: editor formatted (1) identically to author
%Control: production of article title (0) allowed
%Control: page (1) range
%Control: year (0) verbatim
%Control: production of eprint (0) enabled
\begin{thebibliography}{31}%
\makeatletter
\providecommand \@ifxundefined [1]{%
 \@ifx{#1\undefined}
}%
\providecommand \@ifnum [1]{%
 \ifnum #1\expandafter \@firstoftwo
 \else \expandafter \@secondoftwo
 \fi
}%
\providecommand \@ifx [1]{%
 \ifx #1\expandafter \@firstoftwo
 \else \expandafter \@secondoftwo
 \fi
}%
\providecommand \natexlab [1]{#1}%
\providecommand \enquote  [1]{``#1''}%
\providecommand \bibnamefont  [1]{#1}%
\providecommand \bibfnamefont [1]{#1}%
\providecommand \citenamefont [1]{#1}%
\providecommand \href@noop [0]{\@secondoftwo}%
\providecommand \href [0]{\begingroup \@sanitize@url \@href}%
\providecommand \@href[1]{\@@startlink{#1}\@@href}%
\providecommand \@@href[1]{\endgroup#1\@@endlink}%
\providecommand \@sanitize@url [0]{\catcode `\\12\catcode `\$12\catcode
  `\&12\catcode `\#12\catcode `\^12\catcode `\_12\catcode `\%12\relax}%
\providecommand \@@startlink[1]{}%
\providecommand \@@endlink[0]{}%
\providecommand \url  [0]{\begingroup\@sanitize@url \@url }%
\providecommand \@url [1]{\endgroup\@href {#1}{\urlprefix }}%
\providecommand \urlprefix  [0]{URL }%
\providecommand \Eprint [0]{\href }%
\providecommand \doibase [0]{http://dx.doi.org/}%
\providecommand \selectlanguage [0]{\@gobble}%
\providecommand \bibinfo  [0]{\@secondoftwo}%
\providecommand \bibfield  [0]{\@secondoftwo}%
\providecommand \translation [1]{[#1]}%
\providecommand \BibitemOpen [0]{}%
\providecommand \bibitemStop [0]{}%
\providecommand \bibitemNoStop [0]{.\EOS\space}%
\providecommand \EOS [0]{\spacefactor3000\relax}%
\providecommand \BibitemShut  [1]{\csname bibitem#1\endcsname}%
\let\auto@bib@innerbib\@empty
%</preamble>
\bibitem [{\citenamefont {Goodfellow}\ \emph {et~al.}(2016)\citenamefont
  {Goodfellow}, \citenamefont {Bengio},\ and\ \citenamefont
  {Courville}}]{Goodfellow2016}%
  \BibitemOpen
  \bibfield  {author} {\bibinfo {author} {\bibfnamefont {I.}~\bibnamefont
  {Goodfellow}}, \bibinfo {author} {\bibfnamefont {Y.}~\bibnamefont {Bengio}},
  \ and\ \bibinfo {author} {\bibfnamefont {A.}~\bibnamefont {Courville}},\
  }\href@noop {} {\emph {\bibinfo {title} {Deep Learning}}}\ (\bibinfo
  {publisher} {The MIT Press, Cambridge, Massachusetts},\ \bibinfo {year}
  {2016})\BibitemShut {NoStop}%
\bibitem [{\citenamefont {Carleo}\ and\ \citenamefont
  {Troyer}(2017)}]{carleo_solving_2017}%
  \BibitemOpen
  \bibfield  {author} {\bibinfo {author} {\bibfnamefont {G.}~\bibnamefont
  {Carleo}}\ and\ \bibinfo {author} {\bibfnamefont {M.}~\bibnamefont
  {Troyer}},\ }\href {\doibase 10.1126/science.aag2302} {\bibfield  {journal}
  {\bibinfo  {journal} {Science}\ }\textbf {\bibinfo {volume} {355}},\ \bibinfo
  {pages} {602} (\bibinfo {year} {2017})}\BibitemShut {NoStop}%
\bibitem [{\citenamefont {Carrasquilla}\ and\ \citenamefont
  {Torlai}(2021)}]{carra2021}%
  \BibitemOpen
  \bibfield  {author} {\bibinfo {author} {\bibfnamefont {J.}~\bibnamefont
  {Carrasquilla}}\ and\ \bibinfo {author} {\bibfnamefont {G.}~\bibnamefont
  {Torlai}},\ }\href {\doibase 10.48550/arxiv.2101.11099} {\  (\bibinfo {year}
  {2021}),\ 10.48550/arxiv.2101.11099}\BibitemShut {NoStop}%
\bibitem [{\citenamefont {Pfau}\ \emph {et~al.}(2020)\citenamefont {Pfau},
  \citenamefont {Spencer}, \citenamefont {Matthews},\ and\ \citenamefont
  {Foulkes}}]{pfau2019abinitio}%
  \BibitemOpen
  \bibfield  {author} {\bibinfo {author} {\bibfnamefont {D.}~\bibnamefont
  {Pfau}}, \bibinfo {author} {\bibfnamefont {J.~S.}\ \bibnamefont {Spencer}},
  \bibinfo {author} {\bibfnamefont {Alexander G. D.~G.}\ \bibnamefont
  {Matthews}}, \ and\ \bibinfo {author} {\bibfnamefont {W.~M.~C.}\ \bibnamefont
  {Foulkes}},\ }\href {\doibase 10.1103/PhysRevResearch.2.033429} {\bibfield
  {journal} {\bibinfo  {journal} {Physical Review Research}\ }\textbf {\bibinfo
  {volume} {2}},\ \bibinfo {pages} {033429} (\bibinfo {year}
  {2020})}\BibitemShut {NoStop}%
\bibitem [{\citenamefont {Calcavecchia}\ \emph {et~al.}(2014)\citenamefont
  {Calcavecchia}, \citenamefont {Pederiva}, \citenamefont {Kalos},\ and\
  \citenamefont {K{\"u}hne}}]{calcavecchia_sign_2014}%
  \BibitemOpen
  \bibfield  {author} {\bibinfo {author} {\bibfnamefont {F.}~\bibnamefont
  {Calcavecchia}}, \bibinfo {author} {\bibfnamefont {F.}~\bibnamefont
  {Pederiva}}, \bibinfo {author} {\bibfnamefont {M.~H.}\ \bibnamefont {Kalos}},
  \ and\ \bibinfo {author} {\bibfnamefont {T.~D.}\ \bibnamefont {K{\"u}hne}},\
  }\href {\doibase 10.1103/PhysRevE.90.053304} {\bibfield  {journal} {\bibinfo
  {journal} {Physical Review E}\ }\textbf {\bibinfo {volume} {90}},\ \bibinfo
  {pages} {053304} (\bibinfo {year} {2014})}\BibitemShut {NoStop}%
\bibitem [{\citenamefont {Carleo}\ \emph {et~al.}(2019)\citenamefont {Carleo},
  \citenamefont {Cirac}, \citenamefont {Cranmer}, \citenamefont {Daudet},
  \citenamefont {Schuld}, \citenamefont {Tishby}, \citenamefont
  {Vogt-Maranto},\ and\ \citenamefont {Zdeborov\'a}}]{carleo2019}%
  \BibitemOpen
  \bibfield  {author} {\bibinfo {author} {\bibfnamefont {G.}~\bibnamefont
  {Carleo}}, \bibinfo {author} {\bibfnamefont {I.}~\bibnamefont {Cirac}},
  \bibinfo {author} {\bibfnamefont {K.}~\bibnamefont {Cranmer}}, \bibinfo
  {author} {\bibfnamefont {L.}~\bibnamefont {Daudet}}, \bibinfo {author}
  {\bibfnamefont {M.}~\bibnamefont {Schuld}}, \bibinfo {author} {\bibfnamefont
  {N.}~\bibnamefont {Tishby}}, \bibinfo {author} {\bibfnamefont
  {L.}~\bibnamefont {Vogt-Maranto}}, \ and\ \bibinfo {author} {\bibfnamefont
  {L.}~\bibnamefont {Zdeborov\'a}},\ }\href {\doibase
  10.1103/RevModPhys.91.045002} {\bibfield  {journal} {\bibinfo  {journal}
  {Reviews of Modern Physics}\ }\textbf {\bibinfo {volume} {91}},\ \bibinfo
  {pages} {045002} (\bibinfo {year} {2019})}\BibitemShut {NoStop}%
\bibitem [{\citenamefont {Boehnlein}\ \emph {et~al.}(2022)\citenamefont
  {Boehnlein}, \citenamefont {Diefenthaler}, \citenamefont {Sato},
  \citenamefont {Schram}, \citenamefont {Ziegler}, \citenamefont {Fanelli},
  \citenamefont {Hjorth-Jensen}, \citenamefont {Horn}, \citenamefont {Kuchera},
  \citenamefont {Lee}, \citenamefont {Nazarewicz}, \citenamefont {Ostroumov},
  \citenamefont {Orginos}, \citenamefont {Poon}, \citenamefont {Wang},
  \citenamefont {Scheinker}, \citenamefont {Smith},\ and\ \citenamefont
  {Pang}}]{amber2022}%
  \BibitemOpen
  \bibfield  {author} {\bibinfo {author} {\bibfnamefont {A.}~\bibnamefont
  {Boehnlein}}, \bibinfo {author} {\bibfnamefont {M.}~\bibnamefont
  {Diefenthaler}}, \bibinfo {author} {\bibfnamefont {N.}~\bibnamefont {Sato}},
  \bibinfo {author} {\bibfnamefont {M.}~\bibnamefont {Schram}}, \bibinfo
  {author} {\bibfnamefont {V.}~\bibnamefont {Ziegler}}, \bibinfo {author}
  {\bibfnamefont {C.}~\bibnamefont {Fanelli}}, \bibinfo {author} {\bibfnamefont
  {M.}~\bibnamefont {Hjorth-Jensen}}, \bibinfo {author} {\bibfnamefont
  {T.}~\bibnamefont {Horn}}, \bibinfo {author} {\bibfnamefont {M.~P.}\
  \bibnamefont {Kuchera}}, \bibinfo {author} {\bibfnamefont {D.}~\bibnamefont
  {Lee}}, \bibinfo {author} {\bibfnamefont {W.}~\bibnamefont {Nazarewicz}},
  \bibinfo {author} {\bibfnamefont {P.}~\bibnamefont {Ostroumov}}, \bibinfo
  {author} {\bibfnamefont {K.}~\bibnamefont {Orginos}}, \bibinfo {author}
  {\bibfnamefont {A.}~\bibnamefont {Poon}}, \bibinfo {author} {\bibfnamefont
  {X.-N.}\ \bibnamefont {Wang}}, \bibinfo {author} {\bibfnamefont
  {A.}~\bibnamefont {Scheinker}}, \bibinfo {author} {\bibfnamefont {M.~S.}\
  \bibnamefont {Smith}}, \ and\ \bibinfo {author} {\bibfnamefont {L.-G.}\
  \bibnamefont {Pang}},\ }\href {\doibase 10.1103/RevModPhys.94.031003}
  {\bibfield  {journal} {\bibinfo  {journal} {Reviews of Moddern Physics}\
  }\textbf {\bibinfo {volume} {94}},\ \bibinfo {pages} {031003} (\bibinfo
  {year} {2022})}\BibitemShut {NoStop}%
\bibitem [{\citenamefont {Adams}\ \emph {et~al.}(2021)\citenamefont {Adams},
  \citenamefont {Carleo}, \citenamefont {Lovato},\ and\ \citenamefont
  {Rocco}}]{Lovato2021}%
  \BibitemOpen
  \bibfield  {author} {\bibinfo {author} {\bibfnamefont {C.}~\bibnamefont
  {Adams}}, \bibinfo {author} {\bibfnamefont {G.}~\bibnamefont {Carleo}},
  \bibinfo {author} {\bibfnamefont {A.}~\bibnamefont {Lovato}}, \ and\ \bibinfo
  {author} {\bibfnamefont {N.}~\bibnamefont {Rocco}},\ }\href {\doibase
  10.1103/PhysRevLett.127.022502} {\bibfield  {journal} {\bibinfo  {journal}
  {Physical Review Letters}\ }\textbf {\bibinfo {volume} {127}},\ \bibinfo
  {pages} {022502} (\bibinfo {year} {2021})}\BibitemShut {NoStop}%
\bibitem [{\citenamefont {Lovato}\ \emph {et~al.}(2022)\citenamefont {Lovato},
  \citenamefont {Adams}, \citenamefont {Carleo},\ and\ \citenamefont
  {Rocco}}]{lovato2022}%
  \BibitemOpen
  \bibfield  {author} {\bibinfo {author} {\bibfnamefont {A.}~\bibnamefont
  {Lovato}}, \bibinfo {author} {\bibfnamefont {C.}~\bibnamefont {Adams}},
  \bibinfo {author} {\bibfnamefont {G.}~\bibnamefont {Carleo}}, \ and\ \bibinfo
  {author} {\bibfnamefont {N.}~\bibnamefont {Rocco}},\ }\href {\doibase
  10.48550/arxiv.2206.10021} {\  (\bibinfo {year} {2022}),\
  10.48550/arxiv.2206.10021}\BibitemShut {NoStop}%
\bibitem [{\citenamefont {Hornik}\ \emph {et~al.}(1989)\citenamefont {Hornik},
  \citenamefont {Stinchcombe},\ and\ \citenamefont
  {White}}]{hornik_multilayer_1989}%
  \BibitemOpen
  \bibfield  {author} {\bibinfo {author} {\bibfnamefont {K.}~\bibnamefont
  {Hornik}}, \bibinfo {author} {\bibfnamefont {M.}~\bibnamefont {Stinchcombe}},
  \ and\ \bibinfo {author} {\bibfnamefont {H.}~\bibnamefont {White}},\ }\href
  {\doibase 10.1016/0893-6080(89)90020-8} {\bibfield  {journal} {\bibinfo
  {journal} {Neural Networks}\ }\textbf {\bibinfo {volume} {2}},\ \bibinfo
  {pages} {359} (\bibinfo {year} {1989})}\BibitemShut {NoStop}%
\bibitem [{\citenamefont {Murphy}(2012)}]{Murphy2012}%
  \BibitemOpen
  \bibfield  {author} {\bibinfo {author} {\bibfnamefont {K.~P.}\ \bibnamefont
  {Murphy}},\ }\href@noop {} {\emph {\bibinfo {title} {Machine Learning: A
  Probabilistic Perspective}}}\ (\bibinfo  {publisher} {The MIT Press,
  Cambdridge, Massachusetts},\ \bibinfo {year} {2012})\BibitemShut {NoStop}%
\bibitem [{\citenamefont {Hastie}\ \emph {et~al.}(2009)\citenamefont {Hastie},
  \citenamefont {Tibshirani},\ and\ \citenamefont {Friedman}}]{Hastie2009}%
  \BibitemOpen
  \bibfield  {author} {\bibinfo {author} {\bibfnamefont {T.}~\bibnamefont
  {Hastie}}, \bibinfo {author} {\bibfnamefont {R.}~\bibnamefont {Tibshirani}},
  \ and\ \bibinfo {author} {\bibfnamefont {J.}~\bibnamefont {Friedman}},\
  }\href@noop {} {\emph {\bibinfo {title} {The Elements of Statistical
  Learning: Data Mining, Inference and Prediction}}}\ (\bibinfo  {publisher}
  {Springer Verlag, Berlin},\ \bibinfo {year} {2009})\BibitemShut {NoStop}%
\bibitem [{\citenamefont {Bishop}(2006)}]{Bishop2006}%
  \BibitemOpen
  \bibfield  {author} {\bibinfo {author} {\bibfnamefont {C.~M.}\ \bibnamefont
  {Bishop}},\ }\href@noop {} {\emph {\bibinfo {title} {Pattern Recognition and
  Machine Learning}}}\ (\bibinfo  {publisher} {Springer Verlag, Berlin},\
  \bibinfo {year} {2006})\BibitemShut {NoStop}%
\bibitem [{\citenamefont {Cassella}\ \emph {et~al.}(2022)\citenamefont
  {Cassella}, \citenamefont {Sutterud}, \citenamefont {Azadi}, \citenamefont
  {Drummond}, \citenamefont {Pfau}, \citenamefont {Spencer},\ and\
  \citenamefont {Foulkes}}]{casella2022}%
  \BibitemOpen
  \bibfield  {author} {\bibinfo {author} {\bibfnamefont {G.}~\bibnamefont
  {Cassella}}, \bibinfo {author} {\bibfnamefont {H.}~\bibnamefont {Sutterud}},
  \bibinfo {author} {\bibfnamefont {S.}~\bibnamefont {Azadi}}, \bibinfo
  {author} {\bibfnamefont {N.~D.}\ \bibnamefont {Drummond}}, \bibinfo {author}
  {\bibfnamefont {D.}~\bibnamefont {Pfau}}, \bibinfo {author} {\bibfnamefont
  {J.~S.}\ \bibnamefont {Spencer}}, \ and\ \bibinfo {author} {\bibfnamefont
  {W.~M.~C.}\ \bibnamefont {Foulkes}},\ }\href {\doibase
  10.48550/arxiv.2202.05183} {\  (\bibinfo {year} {2022}),\
  10.48550/arxiv.2202.05183}\BibitemShut {NoStop}%
\bibitem [{Note1()}]{Note1}%
  \BibitemOpen
  \bibinfo {note} {Natural units are used with energy given in units of $\hbar
  $ and length given in units of $\protect \sqrt {\hbar /m}$}\BibitemShut
  {NoStop}%
\bibitem [{\citenamefont {Drummond}\ \emph {et~al.}(2004)\citenamefont
  {Drummond}, \citenamefont {Towler},\ and\ \citenamefont
  {Needs}}]{drummond_jastrow_2004}%
  \BibitemOpen
  \bibfield  {author} {\bibinfo {author} {\bibfnamefont {N.~D.}\ \bibnamefont
  {Drummond}}, \bibinfo {author} {\bibfnamefont {M.~D.}\ \bibnamefont
  {Towler}}, \ and\ \bibinfo {author} {\bibfnamefont {R.~J.}\ \bibnamefont
  {Needs}},\ }\href@noop {} {\bibfield  {journal} {\bibinfo  {journal}
  {Physical Review B}\ }\textbf {\bibinfo {volume} {70}},\ \bibinfo {pages}
  {235119} (\bibinfo {year} {2004})}\BibitemShut {NoStop}%
\bibitem [{\citenamefont {Huang}\ \emph {et~al.}(1998)\citenamefont {Huang},
  \citenamefont {Filippi},\ and\ \citenamefont {Umrigar}}]{huang_spin_1998}%
  \BibitemOpen
  \bibfield  {author} {\bibinfo {author} {\bibfnamefont {C.-J.}\ \bibnamefont
  {Huang}}, \bibinfo {author} {\bibfnamefont {C.}~\bibnamefont {Filippi}}, \
  and\ \bibinfo {author} {\bibfnamefont {C.~J.}\ \bibnamefont {Umrigar}},\
  }\href@noop {} {\bibfield  {journal} {\bibinfo  {journal} {The Journal of
  Chemical Physics}\ }\textbf {\bibinfo {volume} {108}},\ \bibinfo {pages}
  {8838} (\bibinfo {year} {1998})}\BibitemShut {NoStop}%
\bibitem [{\citenamefont {Neidinger}(2010)}]{autodiff2010}%
  \BibitemOpen
  \bibfield  {author} {\bibinfo {author} {\bibfnamefont {R.~D.}\ \bibnamefont
  {Neidinger}},\ }\href {\doibase 10.1137/080743627} {\bibfield  {journal}
  {\bibinfo  {journal} {SIAM Review}\ }\textbf {\bibinfo {volume} {52}},\
  \bibinfo {pages} {545} (\bibinfo {year} {2010})}\BibitemShut {NoStop}%
\bibitem [{\citenamefont {Baydin}\ \emph {et~al.}(2018)\citenamefont {Baydin},
  \citenamefont {Pearlmutter}, \citenamefont {Andreyevich~Radul},\ and\
  \citenamefont {Siskind}}]{autodiff2018}%
  \BibitemOpen
  \bibfield  {author} {\bibinfo {author} {\bibfnamefont {A.~G.}\ \bibnamefont
  {Baydin}}, \bibinfo {author} {\bibfnamefont {B.~A.}\ \bibnamefont
  {Pearlmutter}}, \bibinfo {author} {\bibfnamefont {A.}~\bibnamefont
  {Andreyevich~Radul}}, \ and\ \bibinfo {author} {\bibfnamefont {J.~M.}\
  \bibnamefont {Siskind}},\ }\href {http://jmlr.org/papers/v18/17-468.html}
  {\bibfield  {journal} {\bibinfo  {journal} {Journal of Machine Learning
  Research}\ }\textbf {\bibinfo {volume} {18}},\ \bibinfo {pages} {1} (\bibinfo
  {year} {2018})}\BibitemShut {NoStop}%
\bibitem [{\citenamefont {Hammond}\ \emph {et~al.}(1994)\citenamefont
  {Hammond}, \citenamefont {Lester},\ and\ \citenamefont
  {Reynolds}}]{hammond_monte_1994}%
  \BibitemOpen
  \bibfield  {author} {\bibinfo {author} {\bibfnamefont {B~L}\ \bibnamefont
  {Hammond}}, \bibinfo {author} {\bibfnamefont {W~A}\ \bibnamefont {Lester}}, \
  and\ \bibinfo {author} {\bibfnamefont {P~J}\ \bibnamefont {Reynolds}},\
  }\href@noop {} {\emph {\bibinfo {title} {Monte Carlo Methods in Ab Initio
  Quantum Chemistry}}}\ (\bibinfo  {publisher} {World Scientific, Singapore},\
  \bibinfo {year} {1994})\BibitemShut {NoStop}%
\bibitem [{\citenamefont {Umrigar}\ and\ \citenamefont
  {Filippi}(2005)}]{umrigar_energy_2005}%
  \BibitemOpen
  \bibfield  {author} {\bibinfo {author} {\bibfnamefont {C.~J.}\ \bibnamefont
  {Umrigar}}\ and\ \bibinfo {author} {\bibfnamefont {C.}~\bibnamefont
  {Filippi}},\ }\href@noop {} {\bibfield  {journal} {\bibinfo  {journal}
  {Physical Review Letters}\ }\textbf {\bibinfo {volume} {94}},\ \bibinfo
  {pages} {150201} (\bibinfo {year} {2005})}\BibitemShut {NoStop}%
\bibitem [{\citenamefont {H{\o}gberget}(2013)}]{hogberget_quantum_2013}%
  \BibitemOpen
  \bibfield  {author} {\bibinfo {author} {\bibfnamefont {J.}~\bibnamefont
  {H{\o}gberget}},\ }\emph {\bibinfo {title} {Quantum Monte Carlo Studies of
  Generalized Many-body Systems}},\ \href
  {https://www.duo.uio.no/handle/10852/37167} {Master's thesis},\ \bibinfo
  {school} {University of Oslo} (\bibinfo {year} {2013})\BibitemShut {NoStop}%
\bibitem [{\citenamefont {Taut}(1994)}]{taut_two_1994}%
  \BibitemOpen
  \bibfield  {author} {\bibinfo {author} {\bibfnamefont {M.}~\bibnamefont
  {Taut}},\ }\href {\doibase 10.1088/0305-4470/27/3/040} {\bibfield  {journal}
  {\bibinfo  {journal} {Journal of Physics A}\ ,\ \bibinfo {pages} {1045}}
  (\bibinfo {year} {1994})}\BibitemShut {NoStop}%
\bibitem [{\citenamefont {Mariadason}(2018)}]{mariadason_quantum_2018}%
  \BibitemOpen
  \bibfield  {author} {\bibinfo {author} {\bibfnamefont {A.~A.}\ \bibnamefont
  {Mariadason}},\ }\emph {\bibinfo {title} {Quantum {Many}-{Body} {Simulations}
  of {Double} {Dot} {System}}},\ \href
  {https://github.com/Oo1Insane1oO/HartreeFock} {Master's thesis},\ \bibinfo
  {school} {University of Oslo} (\bibinfo {year} {2018})\BibitemShut {NoStop}%
\bibitem [{\citenamefont {Saito}(2018)}]{saito_method_2018}%
  \BibitemOpen
  \bibfield  {author} {\bibinfo {author} {\bibfnamefont {H.}~\bibnamefont
  {Saito}},\ }\href@noop {} {\bibfield  {journal} {\bibinfo  {journal} {Journal
  of the Physical Society of Japan}\ }\textbf {\bibinfo {volume} {87}},\
  \bibinfo {pages} {074002} (\bibinfo {year} {2018})}\BibitemShut {NoStop}%
\bibitem [{\citenamefont {Kim}\ \emph {et~al.}()\citenamefont {Kim},
  \citenamefont {Fore}, \citenamefont {Nordhagen}, \citenamefont {Lovato},\
  and\ \citenamefont {Hjorth-Jensen}}]{kim2022}%
  \BibitemOpen
  \bibfield  {author} {\bibinfo {author} {\bibfnamefont {J.M.}\ \bibnamefont
  {Kim}}, \bibinfo {author} {\bibfnamefont {B.}~\bibnamefont {Fore}}, \bibinfo
  {author} {\bibfnamefont {E.~M.}\ \bibnamefont {Nordhagen}}, \bibinfo {author}
  {\bibfnamefont {A.}~\bibnamefont {Lovato}}, \ and\ \bibinfo {author}
  {\bibfnamefont {M.}~\bibnamefont {Hjorth-Jensen}},\ }\href@noop {} {\enquote
  {\bibinfo {title} {Deep learning and confined electrons in two dimensions},}\
  }\BibitemShut {NoStop}%
\bibitem [{\citenamefont {Ghosal}\ \emph {et~al.}(2007)\citenamefont {Ghosal},
  \citenamefont {G{\"u}{\c c}l{\"u}}, \citenamefont {Umrigar}, \citenamefont
  {Ullmo},\ and\ \citenamefont {Baranger}}]{ghosal_incipient_2007}%
  \BibitemOpen
  \bibfield  {author} {\bibinfo {author} {\bibfnamefont {A.}~\bibnamefont
  {Ghosal}}, \bibinfo {author} {\bibfnamefont {A.~D.}\ \bibnamefont {G{\"u}{\c
  c}l{\"u}}}, \bibinfo {author} {\bibfnamefont {C.~J.}\ \bibnamefont
  {Umrigar}}, \bibinfo {author} {\bibfnamefont {D.}~\bibnamefont {Ullmo}}, \
  and\ \bibinfo {author} {\bibfnamefont {H.~U.}\ \bibnamefont {Baranger}},\
  }\href@noop {} {\bibfield  {journal} {\bibinfo  {journal} {Physical Review
  B}\ }\textbf {\bibinfo {volume} {76}},\ \bibinfo {pages} {085341} (\bibinfo
  {year} {2007})}\BibitemShut {NoStop}%
\bibitem [{\citenamefont {Fock}(1930)}]{fock_bemerkung_1930}%
  \BibitemOpen
  \bibfield  {author} {\bibinfo {author} {\bibfnamefont {V.}~\bibnamefont
  {Fock}},\ }\href {\doibase 10.1007/BF01339281} {\bibfield  {journal}
  {\bibinfo  {journal} {Zeitschrift f{\"u}r Physik}\ }\textbf {\bibinfo
  {volume} {63}},\ \bibinfo {pages} {855} (\bibinfo {year} {1930})}\BibitemShut
  {NoStop}%
\bibitem [{\citenamefont {Metropolis}\ \emph {et~al.}(1953)\citenamefont
  {Metropolis}, \citenamefont {Rosenbluth}, \citenamefont {Rosenbluth},
  \citenamefont {Teller},\ and\ \citenamefont
  {Teller}}]{metropolis_equation_1953}%
  \BibitemOpen
  \bibfield  {author} {\bibinfo {author} {\bibfnamefont {N.}~\bibnamefont
  {Metropolis}}, \bibinfo {author} {\bibfnamefont {A.~W.}\ \bibnamefont
  {Rosenbluth}}, \bibinfo {author} {\bibfnamefont {M.~N.}\ \bibnamefont
  {Rosenbluth}}, \bibinfo {author} {\bibfnamefont {A.~H.}\ \bibnamefont
  {Teller}}, \ and\ \bibinfo {author} {\bibfnamefont {E.}~\bibnamefont
  {Teller}},\ }\href {\doibase 10.1063/1.1699114} {\bibfield  {journal}
  {\bibinfo  {journal} {The Journal of Chemical Physics}\ }\textbf {\bibinfo
  {volume} {21}},\ \bibinfo {pages} {1087} (\bibinfo {year}
  {1953})}\BibitemShut {NoStop}%
\bibitem [{\citenamefont {Kingma}\ and\ \citenamefont
  {Ba}(2014)}]{kingma_adam:_2014}%
  \BibitemOpen
  \bibfield  {author} {\bibinfo {author} {\bibfnamefont {D.~P.}\ \bibnamefont
  {Kingma}}\ and\ \bibinfo {author} {\bibfnamefont {J.}~\bibnamefont {Ba}},\
  }\href {\doibase 10.48550/arxiv.1412.6980} {\  (\bibinfo {year} {2014}),\
  10.48550/arxiv.1412.6980}\BibitemShut {NoStop}%
\bibitem [{\citenamefont {Glorot}\ and\ \citenamefont
  {Bengio}(2010)}]{glorot_understanding_2010}%
  \BibitemOpen
  \bibfield  {author} {\bibinfo {author} {\bibfnamefont {X.}~\bibnamefont
  {Glorot}}\ and\ \bibinfo {author} {\bibfnamefont {Y.}~\bibnamefont
  {Bengio}},\ }\bibfield  {title} {\enquote {\bibinfo {title} {Understanding
  the difficulty of training deep feedforward neural networks},}\ }in\
  \href@noop {} {\emph {\bibinfo {booktitle} {Proceedings of the Thirteenth
  International Conference on Artificial Intelligence and Statistics}}},\
  \bibinfo {series} {Proceedings of Machine Learning Research}, Vol.~\bibinfo
  {volume} {9},\ \bibinfo {editor} {edited by\ \bibinfo {editor} {\bibfnamefont
  {Y.~W.}\ \bibnamefont {Teh}}\ and\ \bibinfo {editor} {\bibfnamefont
  {M.}~\bibnamefont {Titterington}}}\ (\bibinfo  {publisher} {PMLR},\ \bibinfo
  {year} {2010})\ p.\ \bibinfo {pages} {249}\BibitemShut {NoStop}%




\bibitem{Zanghellini2004} J. Zanghellini, M. Kitzler, T. Brabec, and A. Scrinzi, Journal of Physics  B {\bf 37}, 763 (2004), \url{https://doi.org/10.1088/0953-4075/37/4/004}
\bibitem{Calogero1971} F. Calogero, Journal of Mathematical Physics {\bf 12}, 419,  (1971), \url{https://doi.org/10.1063/1.1665604}

\end{thebibliography}%


\end{document}



