\documentclass{article}
\usepackage[utf8]{inputenc}
\usepackage[margin=1.2in]{geometry}
% Math
\usepackage{amsmath}
\usepackage{physics}
\usepackage{isotope}

\usepackage{xcolor}
\usepackage{hyperref}
\hypersetup{
    colorlinks,
    linkcolor={red!50!black},
    citecolor={blue!50!black},
    urlcolor={blue!80!black}}


% Formatting
\usepackage{float}
\usepackage{graphicx}


\title{Quantum Computing \\ \vspace{5px} \large Studies of Confined electron systems\\ \vspace{20px} \large Master of Science thesis project}
% \author{Keran Chen}
\date{December 1,  2022}

\begin{document}
\maketitle

\section*{Introduction and overview}

The aim of this project is the study of time evolution in fermionic
systems through the use of quantum many-body theory applied to
candidate systems for realizing quantum circuits and gates. In
particular we will analyze and study properties like the time evolution of
entanglement and how to realize quantum gates for systems of electrons
confined in one and two dimensions, so-called quantum dot systems \cite{Reimann2002}.

Quantum many-body
theory has provided methods to solve problems in such diverse areas as
atomic, molecular, solid-state and nuclear physics, chemistry and
materials science. In the past decades, static properties such as
binding energies and various expectation values have been
calculated. The introduction of time in these calculations yields an
insight into the dynamics of quantum mechanical systems, such as the
electron behavior under an external potential in quantum dots \cite{Reimann2002}.
Specifically, the goal here is to develop a computational framework (codes and theory) for studies of
interacting systems of electrons with time-dependence using time-dependent full
configuration interaction theory \cite{Skattum2013,Hochstuhl2014}.

Based on the theoretical solutions and design of specific quantum
gates, the final aim is to study the simulation of systems of specific quantum
circuits as function of time.  The aim is to simulate systems of two
to at most ten electrons confined in various harmonic oscillator
traps.

Strongly confined electrons offer a wide variety of complex and subtle
phenomena which pose severe challenges to existing many-body
methods. Quantum dots in particular, that is, electrons confined in
semiconducting heterostructures, exhibit, due to their small size,
discrete quantum levels.

Recently, several experimental groups
have started to study how one can use confined eletrons to make
quantum gates and circuits in order to implement different quantum
computing algorithms. To study their feasibility using quantum
mechanical simulation tools like time-dependent many-body methods can
hopefully shed light on such candidate systems.


\section*{Specific tasks and milestones}

The specific task here 

\begin{enumerate}
    \item Spring 2023: Start writing a time-dependent Hartree-Fock code applied to a system of two electrons in one dimension in one and thereafter two harmonic oscillator traps. Here we have in mind electrons confined in harmonic oscillator traps as done by the authors of Ref.~\cite{Zanghellini2004}. Finalize remaining courses. 
    \item Fall 2023: Extend the program from spring 2023 to study time-dependent full configuration interaction \cite{Skattum2013,Hochstuhl2014} and study systems of two and up to four interacting electrons in one or more harmonic oscillator traps. Study the time-evolution of entanglement for these systems.
    \item Spring 2024: Extend the program to include the simulation of specific quantum circuits and the stability of such systems. Here one can think of devising various quantum gates like CNOT, iSWAP gates and other and study the feasibility of such quantum circuits.
\end{enumerate}
The thesis is expected to be handed in May/June 2024


\begin{thebibliography}{99}

\bibitem{Reimann2002} Stephanie M. Reimann and Matti Manninen, Reviews of  Modern Physics {\bf 74}, 1283 (2002), \url{https://doi.org/10.1103/RevModPhys.74.1283}
\bibitem{Skattum2013} Sigve Skattum, Master of Science Thesis, University of Oslo (2013), \url{https://www.duo.uio.no/handle/10852/37170}  
\bibitem{Hochstuhl2014} D. Hochstuhl, C.M. Hinz, and M. Bonitz, The European Physical Journal Special Topics {\bf 223}, 177 (2014), \url{https://link.springer.com/article/10.1140/epjst/e2014-02092-3}
\bibitem{Zanghellini2004} J. Zanghellini, M. Kitzler, T. Brabec, and A. Scrinzi, Journal of Physics  B {\bf 37}, 763 (2004), \url{https://doi.org/10.1088/0953-4075/37/4/004}

\end{thebibliography}  
%\printbibliography

\end{document}




