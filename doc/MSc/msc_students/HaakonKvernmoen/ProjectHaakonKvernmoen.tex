\documentclass{article}
\usepackage[utf8]{inputenc}
\usepackage[margin=1.2in]{geometry}
% Math
\usepackage{amsmath}
\usepackage{physics}
\usepackage{isotope}

% Citetations
\usepackage[ backend=bibtex, sorting=none, autocite=superscript]{biblatex}
\addbibresource{refs/references}
\usepackage{xcolor}
\usepackage{hyperref}
\hypersetup{
    colorlinks,
    linkcolor={red!50!black},
    citecolor={blue!50!black},
    urlcolor={blue!80!black}}


% Formatting
\usepackage{float}
\usepackage{graphicx}
\graphicspath{ {./figs/} } 

% Misc
\usepackage{appendix}

% Commands
\newcommand{\comment}[1]{\textcolor{red}{#1}}
\newcommand{\core}[3]{^{#2}_{#3}\text{#1}}

\title{Time Evolution of Quantum Mechanical systems \\ \vspace{5px} \large Implementation of Time-Dependent Coupled Cluster methods \\ \vspace{20px} \large Master of Science thesis project}
% \author{Håkon Kvernmoen}
\date{November 2022}

\begin{document}
\maketitle

\subsection*{Introduction and overview}

The aim of this project is the study of time evolution in fermionic
systems through the use of quantum many-body theory. Quantum many-body
theory has provided methods to solve problems in such diverse areas as
atomic, molecular, solid-state and nuclear physics, chemistry and
materials science. In the past decades, static properties such as
binding energies and various expectation values has been
calculated. The introduction of time in these calculations yields an
insight into the dynamics of quantum mechanical systems, such as the
electron behavior under an external potential in quantum dots and the
consolidation of composite systems undergoing nuclear
fusion. Specifically, the goal is to implement the time dependent
version of the Coupled Cluster approach
\autocite{doi:10.1063/1.4718427}, a method containing a plethora of
desired properties such as size consistency and extensivity.

\subsection*{General introduction to possible physical systems.}

The code developed during this thesis will be written in such a way
that different fermionic system can be handled. Examples of potential
applications are ions confined in various traps
\autocite{RevModPhys.72.895}. Strongly confined electrons offer a wide
variety of complex and subtle phenomena which pose severe challenges
to existing many-body methods. Quantum dots in particular, that is,
electrons confined in semiconducting heterostructures, exhibit, due to
their small size, discrete quantum levels. The ground states of, for
example, circular dots show similar shell structures and magic numbers
as seen for atoms and nuclei. These structures are particularly
evident in measurements of the change in electrochemical potential due
to the addition of one extra electron.

Application in nuclear physics might also be of
interest\autocite{Pigg_2012}, where the Coupled Cluster method was
originally introduced. In particular, the process of two Helium cores
transforming into Beryllium $(\isotope[4][2]{He} + \isotope[4][2]{He}
\xrightarrow{} \isotope[8][4]{Be})$ could serve as the ultimate goal
of the thesis, being part of the tripe-alpha process, the backbone of
stellar nucleosynthesis.

\subsection*{Specific tasks and milestones}

The specific task here is to study the time evolution of quantum
mechanical systems using the Coupled Cluster (CC) method, in order to
be able to study the time evolution of an interacting quantum
mechanical system.

\begin{enumerate}
    \item Spring 2023: Start writing a time-independent Coupled Cluster code with single and double excitations \comment{only singles or doubles?}, applied to a small confined fermionic system. Finalize remaining courses. 
    \item Fall 2023:
    \item Spring 2024:
\end{enumerate}
The thesis is expected to be handed in May/June 2024



\printbibliography

\end{document}
