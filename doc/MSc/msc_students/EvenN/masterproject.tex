\documentclass[10pt]{article}
\usepackage{graphicx,amsmath,amssymb,bm}
\usepackage{amsfonts}

\begin{document}

\section*{Master thesis project:  Solving Quantum Mechanical Problems with Machine Learning}

\section*{Overview}

Quantum Computing and Machine Learning are two of the most promising
approaches for studying complex physical systems where several length
and energy scales are involved.  Traditional many-particle methods,
either quantum mechanical or classical ones, face huge dimensionality
problems when applied to studies of systems with many interacting
particles. To be able to define properly effective potentials for
realistic Molecular Dynamics simulations of billions or more particles,
requires both precise quantum mechanical studies as well as algorithms
that allow for parametrizations and simplifications of quantum
mechanical results. Quantum Computing offers now an interesting
avenue, together with traditional algorithms, for studying complex
quantum mechanical systems. Machine Learning on the other hand allows us to parameterize
these results in terms of classical interactions. These interactions
are in turn suitable for large scale Molecular Dynamics simulations of
complicated systems spanning from subatomic physics to materials
science and life science.

In addition, Machine Learning plays nowadays a central role
in the analysis of large data sets in order to extract information
about complicated correlations. This information is often difficult to
obtain with traditional methods. For example, there are about one
trillion web pages; more than one hour of video is uploaded to YouTube
every second, amounting to 10 years of content every day; the genomes
of 1000s of people, each of which has a length of $3.8\times 10^9$
base pairs, have been sequenced by various labs and so on. This deluge
of data calls for automated methods of data analysis, which is exactly
what machine learning provides.  Developing activities in these
frontier computational technologies is thus of strategic importance
for our capability to address future science problems.


Enabling simulations of large-scale many-body systems is a
long-standing problem in scientific computing.  Quantum many-body
interactions define the structure of the universe, from nucleons and
nuclei, to atoms, molecules, and even stars. Since the discovery of
quantum mechanics, a lot of progress has been made in understanding
the dynamics of certain many-body systems. While some of our insight
comes from a small set of analytically solvable models, numerical
simulations have become a mainstay in our understanding of many-body
dynamics. The progress in numerical simulations has accelerated in the
last few decades with the advent of modern high performance computing
 and clever developments in classical simulation algorithms such
as, quantum Monte Carlo,large-scale diagonalization approaches,
Coupled-Cluster theory and other renormalization schemes.  Despite the
monumental advances, classical simulation techniques are reaching
fundamental limits in terms of the size of the quantum systems that
can be processed. Fortunately, new developments in the fields of quantum
simulations and machine learning have emerged, promising to enable simulations far beyond
those which are classically tractable. 


The approaches to machine learning are many, but are often split into
two main categories.  In {\em supervised learning} we know the answer
to a problem, and let the computer deduce the logic behind it. On the
other hand, {\em unsupervised learning} is a method for finding
patterns and relationship in data sets without any prior knowledge of
the system.  Some authours also operate with a third category, namely
{\em reinforcement learning}. This is a paradigm of learning inspired
by behavioural psychology, where learning is achieved by
trial-and-error, solely from rewards and punishment.  In this thesis,
the aim is to explore new developments in the field of machine 
learning, with an emphasis on unsupervided learning. Much of the work
here and its implementations is motivated by the recent article of
Carleo and Troyer in Science, 2017. In particular we have extended
their work, which focused on spin-like quantum mechanical systems, to
systems of interacting bosons and fermions confined to move in
trapping potentials, with the harmonic oscillator as one of the
foremost examples.  
In this work, we will start with quantum Monte Carlo methods,
with an emphasis on Variational Monte Carlo methods.  This approach to
studies of complicated interacting many-particle systems has been
widely used in almost all fields of physics where first principle calculations are 
employed. It provides an important starting  point
for almost exact solutions to Schr\"odinger's equation for many
interacting particles using so-called Green's function Monte Carlo methods.

A variational Monte Carlo (VMC) calculation is based on an ansatz for
say the ground state wave function.  For fermionic systems this ansatz
is often composed of a single-particle part (via a so-called Slater
determinant which acoounts for the anti-symmetry) and a correlated
part, normally called the Jastrow factor.  For bosonic systems there
may also be a product function of single-particle functions and a
Jastrow factor that aims at incorporating correlations beyond a mean
field.  These trial wave functions are thereafter used in an
optimization procedure where various variational parameters are
optmized in order to find a minimum for expectation values like the
energy and the variance.

Constructing both the single-particle part and the Jastrow part can
often be complicated and tedious.  In systems like interacting
nucleons, the correlation part of the wave function contains often
complicated two- and three-body operators that require dedicated code
developments. Similalry, for systems of atoms and molecules (and
nucleons as well), the single-particle part is often constructed using
mean-field methods like Hartree-Fock theory.

The aim here is to see whether methods inspired from Machine Learning
can do an equally good job as the standard approach to VMC
calculations, this time however with trial wave functions determined
by neural networks.  These trial wave functions 
are based on what in the literature is called Boltzmann
machines. These functions contain several parameters which are used to
find an energy minimum and thereby the optimal solution for the
energy.

A typical machine learning algorithm consists of three basic
ingredients, a dataset $\mathbf{x}$ (could be some observable quantity
of the system we are studying), a model which is a function of a set
of parameters $\mathbf{\alpha}$ that relates to the dataset, say a
likelihood function $p(\mathbf{x}\vert \mathbf{\alpha})$ or just a
simple model $f(\mathbf{\alpha})$, and finally a so-called {\em cost}
function $\mathcal{C} (\mathbf{x}, f(\mathbf{\alpha}))$ which allows
us to decide how well our model represents the dataset.

We seek to minimize the function $\mathcal{C} (\mathbf{x},
f(\mathbf{\alpha}))$ by finding the parameter values which minimize
$\mathcal{C}$. Thus, VMC calculations serve both as input to exact
solutions via Green's functions methods and employ similar
optimization approaches as employed in machine learning.  

\section*{Progress plan and milestones}
The aims and progress plan of this thesis are as follows
\begin{itemize}
\item Set up (based upon Alocias Master thesis) the codes for doing VMC with a Hartree-Fock basis.
\item Use the Hartree-Fock basis as a way to define the Slater determinant and the anti-symmetry for the Boltzmann machine.
\item  Start by studying a system of bosons and fermions trapped in two and three dimensions by a Harmonic Oscillator potential. Use a standard Coulomb interaction and a screened Coulomb interaction. Use Boltzmann machines as an ansatz for the Jastrow factor. Try various types with $N=2$ particles first. Then move to several interacting particles. For quantum dots, use magic numbers only for the two- and three-dimensional systems. 
\item Make a critical evaluation of the Boltzmann machines compared with standard Jastrow factors used in VMC studies. Study the variance in particular, and quantities like one-body densities and two-body densities. 
\item Try to implement a variance optimization as done in VMC studies by Filippi and Umrigar (2005), see discussion in slides. 
\item Compare the results obtained with Boltzmann machines with so-called Shadow wave function approaches, see \url{https://arxiv.org/abs/1404.6944} and references therein.
\end{itemize}
 
The thesis is expected to be handed in May/June 2019.





\end{document}



\title{
























