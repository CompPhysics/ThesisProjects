\documentstyle[a4wide]{article}
\newcommand{\OP}[1]{{\bf\widehat{#1}}}

\newcommand{\be}{\begin{equation}}

\newcommand{\ee}{\end{equation}}

\begin{document}

\pagestyle{plain}

\section*{Master Thesis project for Alexander Fleischer: Time dependent analysis of quantum mechanical systems}
Supervisor: Morten Hjorth-Jensen, Department of Physics, University of Oslo, Norway\newline\newline
The aim of this thesis is to study the time evolution  of quantum dots using
time-dependent methods like the multi-configuration interaction time-dependent Hartree-Fock approach \cite{Skattum2013}. 

Semiconductor quantum dots are structures where
charge carriers are confined in all three spatial dimensions, 
the dot size being of the order of the Fermi wavelength 
in the host material, typically between  10 nm and  1 $\mu$m.
The confinement is usually achieved by electrical gating of a 
two-dimensional electron gas (2DEG), 
possibly combined with etching techniques. Precise control of the
number of electrons in the conduction band of a quantum dot 
(starting from zero) has been achieved in GaAs heterostructures. 
The electronic spectrum of typical quantum dots
can vary strongly when an external magnetic field is applied, 
since the magnetic length corresponding to typical 
laboratory fields  is comparable to typical dot sizes.
In coupled quantum dots Coulomb blockade effects, 
tunneling between neighboring dots, and magnetization 
have been observed as well as the formation of a
delocalized single-particle state. 


 Experimental realization of conditional
quantum dynamics of two isolated solid-state quantum systems has
become a holy grail of mesoscopic physics research, due to its
potential implications for scalable quantum information
processing.  The majority of theoretical proposals aimed at this
goal is based on simplified and not too realistic models like the nearest-neighbor interactions such as the
Heisenberg exchange coupling between quantum dot (QD) spins
\cite{Loss1999}. However, to achieve lower accuracy thresholds
for quantum error correction, the
implementation of coherent long-range interactions between two
qubits is highly desirable \cite{Imamoglu99}. 
Optical dipole-dipole interactions \cite{Calarco2003},
capacitive coupling \cite{Taylor2005}
and optical cavity-mediated interactions \cite{Imamoglu99} 
between spins could be used to realize
controlled quantum gate operations on length-scales comparable to
optical wavelengths: these mechanisms may then enable coherent
interactions between a limited number ($\le 10$) of QD spins. A simple implementation of these ideas was done in Ref.~\cite{Burkard2006}.

A proper calculation of such systems implies an introduction of more 
realistic Hamiltonians, that is Hamiltonians which handle properly for example the Coulomb interaction between electrons, the symmetries of the potentials involved, time-dependent interactions etc. 
In order to study coherent states and the dynamics of quantum mechanical states, it is essential to develop reliable tools that include properly the time evolution and the spatial structure of the various wave functions. 


This thesis 
entails thus the development of a quantum mechanical program that is based on the so-called multi-configuration interaction time-dependent Hartree-Fock (MCITDHF) method applied in order to study systems
of quantum dots with few electrons. These electrons can be located in one single potential well or two coupled potential wells. Preliminary studies of
MCITDHF have been performed by a former Master of Science student at the Computational Physics group, see Ref.~\cite{Skattum2013}.
Rather promising results based on the MCITDHF method have been accomplished recently in studies of photoionization processes of many-electron atoms, see for example Ref.~\cite{Hochstuhl2014}.

The first step is thus to extend the work of Ref~\cite{Skattum2013} in order to include the effect of a time-dependent electromagnetic field applied to the study of two electrons confined to either one single harmonic oscillator potential well or to two coupled harmonic oscillator wells. These studies will in turn be extended to more electrons and provide thereby a true benchmark for the time evolution of quantum mechanical systems. The formalism and software to be developed can easily be extended to other quantum mechanical systems like the abovementioned  photoionization processes of many-electron atoms~\cite{Hochstuhl2014}.


The thesis is expected to be handed in May/June 2016.

\begin{thebibliography}{999}

\bibitem{Skattum2013} S.~B\o e Skattum, Master of Science thesis, University of Oslo (2013) \url{https://www.duo.uio.no/handle/10852/37170}


\bibitem{Loss1999}
D.~Loss and D.~P.~DiVincenzo, Phys.~Rev.~{\bf A 57}, 120 (1998).

\bibitem{Imamoglu99}
A.~Imamoglu, D.~D.~Awschalom, G.~Burkard, D.~P.~DiVincenzo,
D.~Loss, M.~Sherwin, and A.~Small,
Phys.~Rev.~Lett.~{\bf 83}, 4204 (1999).

\bibitem{Calarco2003}
T.~Calarco, A.~Datta, P.~Fedichev, E.~Pazy, and P.~Zoller, Phys.~Rev.~{\bf A 68}, 012310 (2003).

\bibitem{Taylor2005}
J.~M.~Taylor, H.-A. Engel, W.~Dur, A.~Yacoby, C.~M.~Marcus,
P.~Zoller, and M.~D.~Lukin, 
Nature Physics {\bf 1}, 177 (2005).

\bibitem{Burkard2006}
G.~Burkard and A.~Imamoglu, Phys. Rev. {\bf B 74}, 041307(R) (2006).

\bibitem{Hochstuhl2014} D.~Hochstuhl, C.~M.~Hinz, M.~Bonitz, 
European Phys.~J.~ Special Topics {\bf 223}, 177 (2014). \url{http://link.springer.com/article/10.1140%2Fepjst%2Fe2014-02092-3}




\end{thebibliography}



\end{document}



