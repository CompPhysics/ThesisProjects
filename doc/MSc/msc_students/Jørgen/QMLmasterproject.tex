\documentclass[oneside,                 % oneside: electronic viewing, twoside: printing
final,                   % draft: marks overfull hboxes, figures with paths
10pt]{article}

\listfiles               %  print all files needed to compile this document

\usepackage{relsize,makeidx,color,setspace,amsmath,amsfonts,amssymb}
\usepackage[table]{xcolor}
\usepackage{bm,ltablex,microtype}

\usepackage[pdftex]{graphicx}

\usepackage[T1]{fontenc}
%\usepackage[latin1]{inputenc}
\usepackage{ucs}
\usepackage[utf8x]{inputenc}

\usepackage{lmodern}         % Latin Modern fonts derived from Computer Modern

% Hyperlinks in PDF:
\definecolor{linkcolor}{rgb}{0,0,0.4}
\usepackage{hyperref}
\hypersetup{
    breaklinks=true,
    colorlinks=true,
    linkcolor=linkcolor,
    urlcolor=linkcolor,
    citecolor=black,
    filecolor=black,
    %filecolor=blue,
    pdfmenubar=true,
    pdftoolbar=true,
    bookmarksdepth=3   % Uncomment (and tweak) for PDF bookmarks with more levels than the TOC
    }
\setcounter{tocdepth}{2}  % levels in table of contents

\clubpenalty = 10000
\widowpenalty = 10000
\raggedbottom
\makeindex
\usepackage[totoc]{idxlayout}   % for index in the toc
\usepackage[nottoc]{tocbibind}  % for references/bibliography in the toc
\begin{document}

\thispagestyle{empty}

\begin{center}
{\LARGE\bf
\begin{spacing}{1.25}
Quantum Machine Learning for Finance
\end{spacing}
}
\end{center}

\begin{center}
{\bf Master of Science thesis project${}^{}$} \\ [0mm]
\end{center}

\begin{center}
% List of all institutions:
\end{center}

\begin{center}
November 25, 2025
\end{center}

\vspace{1cm}


\subsection*{Quantum Computing and Machine Learning}

\textbf{Quantum Computing and Machine Learning} are two of the most promising
approaches for studying complex systems with many degrees of freedom.

Quantum computing is an emerging area of computer science that
leverages the principles of quantum mechanics to perform computations
beyond the capabilities of classical computers. Unlike classical
computers, which use bits to represent data as bits $0$ or $1$,
quantum computers use quantum bits, or qubits. Qubits can exist in
multiple states simultaneously (superposition) and can be entangled
with one another, allowing quantum computers to process vast amounts
of information in parallel.

These unique properties enable quantum computers to tackle problems
that are currently intractable for classical systems, such as complex
simulations in chemistry and physics, optimization problems, and
large-scale data analysis.

Quantum machine learning (QML) is an interdisciplinary field that
combines quantum computing with machine learning techniques. The goal
is to enhance the performance of machine learning algorithms by
utilizing quantum computing’s capabilities.

In QML, quantum algorithms are developed to process and analyze data
more efficiently than classical algorithms. This includes tasks like
classification, regression, clustering, and dimensionality
reduction. By exploiting quantum phenomena, QML has the potential to
accelerate machine learning processes and handle larger datasets more
effectively.

Quantum computing and QML hold promise for many different types of  applications, including:

\begin{enumerate}
\item Drug Discovery: Simulating molecular structures to expedite the development of new medications.

\item Financial Modeling: Optimizing portfolios and detecting fraudulent activities through complex data analysis.

\item Artificial Intelligence: Enhancing machine learning algorithms for faster and more accurate predictions. 
\end{enumerate}

As quantum hardware continues to advance, the integration of quantum
computing into practical applications is becoming increasingly
feasible, opening up for a new era of computational possibilities.
\end{document}
This thesis project deals with the study and implementation of quantum
machine learning methods applied to classical machine learning data
for supervised learning.  The methods we will focus on are

\begin{enumerate}
\item Support vector machines and quantum support vector machines

\item Neural networks and quantum neural networks and possibly (if time allows)

\item Classical and quantum Boltzmann machines
\end{enumerate}


The data sets will span both regression and classification problems,
with an emphasis on simulating time series, in particular of relevance
for financial problems.



\subsection*{Support vector machines}

A central model in classical supervised learning is the support vector
machine (SVM), which is a maximal-margin classifier.  SVMs are widely
used for binary classification and have extensions to regression
problems as well.  They build on statistical learning theory and are
known for finding decision boundaries with maximal margin.  In
particular, SVMs can perform non-linear classification by employing
the kernel trick, which implicitly maps data into a high-dimensional
feature space via a kernel function.

A Quantum Support Vector Machine (QSVM) replaces the classical kernel
or feature map with a quantum procedure.  In QSVM, classical data
points $\bm{x}$ are encoded into quantum states $|\phi(\bm{x})\rangle$
via a quantum feature map (a parameterized quantum circuit).  The
inner product (overlap) between two such states serves as a quantum
kernel, measuring data similarity in a high-dimensional Hilbert space.

\subsection*{Quantum Neural Networks and Variational Circuits}

The Variational Quantum Algorithm (VQA) is a Variational Quantum
Circuit (VQC), that is a quantum circuit with tunable parameters and
which is trained using a classical optimizer.  In practice, a VQC
(also called a Parameterized Quantum Circuit (PQC)) is used as a
Quantum Neural Network (QNN): data are encoded into quantum states, a
parameterized circuit is applied, and measurements yield outputs.  For
example, it has been shown recently that certain QNNs can exhibit
higher effective dimension (and thus capacity to generalize) than
comparable classical networks, suggesting a potential quantum
advantage.

\subsection*{Boltzmann machines}

Boltzmann Machines (BMs) offer a powerful framework for modeling
probability distributions.  These types of neural networks use an
undirected graph-structure to encode relevant information.  More
precisely, the respective information is stored in bias coefficients
and connection weights of network nodes, which are typically related
to binary spin-systems and grouped into those that determine the
output, the visible nodes, and those that act as latent variables, the
hidden nodes.  The aim of BM training is to learn a set of weights
such that the resulting model approximates a target probability
distribution which is implicitly given by training data.  This setting
can be formulated as discriminative as well as generative learning
task.  Applications have been studied in a large variety of domains
such as the analysis of quantum many-body systems, statistics,
biochemistry, social networks, signal processing and finance

Quantum Boltzmann Machines (QBMs) are a natural adaption of BMs to the
quantum computing framework. Instead of an energy function with nodes
being represented by binary spin values, QBMs define the underlying
network using a Hermitian operator, normally a parameterized
Hamiltonian.

\paragraph{Specific tasks and milestones.}

The aim of this thesis is to study the implementation and development
of codes for several quantum machine learning methods, including
quantum support vector machines, quantum neural networks and possibly
Boltzmann machines, if time allows. The results will be compared with
those from their classical counterparts.  The final aim is to study
data from finance with both classical and quantum Machine Learning
algorithms in order to assess and test quantum machine learning
algorithms and their potential for the analysis of data from finance.
In setting up the algorithms, existing software libraries like
Scikit-Learn, PennyLane, Qiskit and other will be used. This will
allow for an efficient development and study of both classical and
quantum machine learning algorithms.

The thesis consists of three basic steps:

\begin{enumerate}
\item Develop a classical machine framework for studies of supervised classification and regression problems, with an emphasis on data from finance. The main emphasis rests on deep learning methods (neural networks, Boltzmann machines and recurrent neural networks) and support vector machines.

\item Compare and evaluate the results from the classical machine learning methods and assess their relevance for financial data.

\item Develop and implement  codes for quantum machine learning algorithms (quantum support vector machines, quantum neural networks and possibly quantum Boltzmann machines) to be run on existing quantum computers and classical computers. Compare the performance of the quantum machine learning  with the abovementioned classical methods with an emphasis on  financial data.
\end{enumerate}

\noindent
The milestones are:
\begin{enumerate}
\item Spring 2026: Study basic quantum machine learning algorithms (quantum support vector machines, quantum neural networks) for simpler supervised problems from finance and/or other fields.

\item Spring 2026: Compare the results of the simpler data sets with classical machine learning methods

\item Fall 2026: Set up final data from finance to be analyzed with classical and quantum machine learning algorithms

\item Fall 2026: Develop a software framework which includes quantum support vector machines and quantum neural networks.

\item Spring 2027: The final part is to include Quantum Boltzmann machines, if time allows,  and analyze the results from the diffirent methods. Finalize thesis. 
\end{enumerate}

\noindent
The thesis is expected to be handed in May/June 2027.

\paragraph{Literature.}
\begin{enumerate}


\item Maria Schuld and Francesco Petruccione, \textbf{Supervised Learning with Quantum Computers}, Springer, 2018.

\item Claudio Conti, Quantum Machine Learning (Springer), see \href{{https://link.springer.com/book/10.1007/978-3-031-44226-1}}{\nolinkurl{https://link.springer.com/book/10.1007/978-3-031-44226-1}}.

\item M. Zhao et al., \textbf{A tutorial on quantum machine learning and quantum neural networks}, arXiv:2504.16131 (2025) 
\item Amin et al., \textbf{Quantum Boltzmann Machines}, Physical Review X \textbf{8}, 021050 (2018).

\item Morten Hjorth-Jensen, Quantum Computing and Quantum Machine Learning, lecture notes with extensive codes at \href{{https://github.com/CompPhysics/QuantumComputingMachineLearning}}{\nolinkurl{https://github.com/CompPhysics/QuantumComputingMachineLearning}}, in particular the last five sets of lectures.
\end{enumerate}

\end{document}

