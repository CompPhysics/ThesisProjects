\documentclass{article}
\usepackage{graphicx} % Required for inserting images
\usepackage{hyperref}

\title{Master Thesis}
\author{Yevhenii Volkov}
\date{January 2024}

\begin{document}

\maketitle  

Master Thesis - Project Description
Tentative title: Efficient optimization and parallelization of deep learning quantum algorithms

\section{Background}

The aim of this thesis is to develop a Variational Monte Carlo (VMC)
program with deep learning based approaches and various optimization approaches
in order to test the potential of graphical processing units (GPUs).

The thesis will start with existing VMC codes for bosons and fermions,
tailored to run on large supercomputing facilities with thousands of cores.
These codes will then be rewritten in Python and tested on different GPUs using libraries like JAX \cite{jax},
in order to see if a considerable speedup and further optimization can be gained.

The physics cases deal with Bose-Einstein condensation of trapped ions and
studies of the structure of fermionic systems like quantum dots, of
great relevance for nanotechnologies and quantum technologies.



\section{Main objectives}

The main objective is to develop an efficient programming environment
for quantum mechanical many-body studies of both bosonic and fermionic
systems based on Variational Monte Carlo (VMC) methods \cite{vmc,nilsen}.

The trial wave functions to be used in the Variational Monte Carlo
approaches will start with a standard Jastrow-Pade correlation factor
in order to benchmark the codes to be developed against existing
codes that include parallelization with CPUs and GPUs.
This will allow for an efficient study of codes written in Python that utilize the recently JAX \cite{jax}
library with just-in time complilation.

The next step is to replace the trail wave functions with neural
networks \cite{fore,kim} and generative models like Boltzmann machines
\cite{nordhagen}.  In the optimization, one needs also to implement
stochastic reconfiguration \cite{stochreconfig} and symmetries
correctly using the deepnets library
\url{https://pypi.org/project/deepNets/}.

The final codes with simulation results will be compared with existing codes with an emphasis on computationa performance.



\section{Project Tasks and Tentative Timeline}

\begin{enumerate}
\item Spring 2024: Develop a VMC  code for bosons and fermions that includes parallelization with GPUs and CPUs using c++ and Python
\item Fall 2024: Implement efficiently the usage of JAX and compare with the codes developed during spring 2024. Add stochastic reconfiguration and neural networks as trial wave functions to the VMC code.
\item Fall 2024: Include also Boltzmann machines in the studies of bosonic and fermionic systems
\item Spring 2025: Perform extensive optimizations and analyze the performance of the codes.
\item Spring 2025: Finalize thesis with deadline medio May/June 2025.
\end{enumerate}

\section{References}
\begin{enumerate}
\bibitem{jax} James Bradbury and Roy Frostig and Peter Hawkins and Matthew James Johnson and Chris Leary and Dougal Maclaurin and George Necula and Adam Paszke and Jake Vander{P}las and Skye Wanderman-{M}ilne and Qiao Zhang, JAX: composable transformations of {P}ython+{N}um{P}y programs}, \url{http://github.com/google/jax}.
\bibitem{vmc} Federico Becca and Sandro Sorella, Quantum Monte Carlo Approaches for Correlated Systems, Cambridge University Press, .(2017)
\bibitem{nilsen} J.K.~Nilsen, {\em Comp.~Phys.~Comm.} {\bf 177}, 799 (2007).
\bibitem{nordhagen} Even M. Nordhagen, Jane M. Kim, Bryce Fore, Alessandro Lovato, Morten Hjorth-Jensen, Efficient Solutions of Fermionic Systems using Artificial Neural Networks, \url{https://arxiv.org/abs/2210.00365}
\bibitem{fore} Bryce Force, Jane M. Kim, Giuseppe Carleo, Morten Hjorth-Jensen, Alessandro Lovato, and Maria Piarulli,  Dilute neutron star matter from neural-network quantum states, Physical Review Research 5, 033062 (2023).
\bibitem{kim} Jane Kim,  Gabriel Pescia, Bryce Fore, Jannes Nys, Giuseppe Carleo, Stefano Gandolfi, Morten Hjorth-Jensen, Alessandro Lovato, Neural-network quantum states for ultra-cold Fermi gases, Nature Physics, in press and \url{https://arxiv.org/abs/2305.08831}
\bibitem{stochreconfig} Chae-Yeun Park and Michael J. Kastoryano, Geometry of learning neural quantum states, Phys. Rev. Research 2, 023232 (2020)

\end{enumerate}

\end{document}




