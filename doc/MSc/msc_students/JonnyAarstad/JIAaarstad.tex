\documentclass{article}
\usepackage{graphicx} % Required for inserting images
\usepackage{hyperref}

\title{Master Thesis project}
\author{Jonny Aarstad Igeh}
\date{January 2024}

\begin{document}

\maketitle  

Master Thesis - Project Description
Tentative title: {\bf Time-dependent many-body methods for quantum technologies}

\section{Background}

The aim of this project is the study of time evolution in fermionic
systems through the use of quantum many-body theory applied to
candidate systems for realizing quantum circuits and gates. In
particular we will analyze and study properties like the time evolution of
entanglement and how to realize quantum gates for systems of electrons
confined in one, two and three dimensions, so-called quantum dot systems \cite{Reimann2002}.

Quantum many-body
theory has provided methods to solve problems in such diverse areas as
atomic, molecular, solid-state and nuclear physics, chemistry and
materials science. In the past decades, static properties such as
binding energies and various expectation values have been
calculated. The introduction of time in these calculations yields an
insight into the dynamics of quantum mechanical systems, such as the
electron behavior under an external potential in quantum dots \cite{Reimann2002}.
Specifically, the goal here is to develop a computational framework (codes and theory) for studies of
interacting systems of electrons with time-dependence using time-dependent full
configuration interaction theory \cite{Skattum2013,Hochstuhl2014}.

Based on the theoretical solutions and design of specific quantum
gates, the final aim is to study the simulation of systems of specific quantum
circuits as function of time.  The aim is to simulate systems of two
to several  electrons confined in various potential
traps \cite{us2024}. 

Strongly confined electrons offer a wide variety of complex and subtle
phenomena which pose severe challenges to existing many-body
methods. Quantum dots in particular, that is, electrons confined in
semiconducting heterostructures, exhibit, due to their small size,
discrete quantum levels.

Recently, several experimental groups have started to study how one
can use confined eletrons to make quantum gates and circuits in order
to implement different quantum computing algorithms. To study their
feasibility using quantum mechanical simulation tools like
time-dependent many-body methods can hopefully shed light on such
candidate systems. Furthermore, the design of circuits and systems to
simulate will include algorithms from quantum computing like the
Variational Quantum Eigensolver \cite{vqe}.

This thesis will also include the utilization of efficient
single-particle computational basis sets from density functional
theories. These basis sets allow for the inclusion of specific
medium-dependent properties and can thus be closely linked with
experimental work conducted at the center for Materials Science and
Nanotechnology (SMN) at the University of Oslo. The Computational
Physics group at the University of Oslo works closely with
experimentalists at the SMN and at Michigan State University, with the
aim to develop a theoretical quantum engineering platform that
describes in the best possible way experimental realizations of
various quantum gates and circuits.


\section{Project Tasks and Tentative Timeline}


\begin{enumerate}
\item Spring 2024: Start writing a time-dependent Hartree-Fock code applied to a system of two electrons in one dimension in one and thereafter two harmonic oscillator traps. Here we have in mind electrons confined in harmonic oscillator traps as done by the authors of Ref.~\cite{Zanghellini2004}. Finalize remaining courses.
\item Spring 2024: Extend these studies to include single-particle basis functions from density functional theory.   
    \item Fall 2024: Extend the program from spring 2024 to study time-dependent full configuration interaction \cite{Skattum2013,Hochstuhl2014} and study systems of two and up to four interacting electrons in one or more harmonic oscillator traps. Study the time-evolution of entanglement for these systems based on the recent work of the Computational Physics group \cite{us2024}.
    \item Spring 2025: Extend the program to include the simulation of specific quantum circuits and the stability of such systemsbased on the Variational Quantum Eigensolver algorithm \cite{vqe}. Here one can think of devising various quantum gates like CNOT, iSWAP gates and other and study the feasibility of such quantum circuits.
\end{enumerate}
The thesis is expected to be handed in May/June 2025


\begin{thebibliography}{99}

\bibitem{Reimann2002} Stephanie M. Reimann and Matti Manninen, Reviews of  Modern Physics {\bf 74}, 1283 (2002), \url{https://doi.org/10.1103/RevModPhys.74.1283}
\bibitem{Skattum2013} Sigve Skattum, Master of Science Thesis, University of Oslo (2013), \url{https://www.duo.uio.no/handle/10852/37170}  
\bibitem{Hochstuhl2014} D. Hochstuhl, C.M. Hinz, and M. Bonitz, The European Physical Journal Special Topics {\bf 223}, 177 (2014), \url{https://link.springer.com/article/10.1140/epjst/e2014-02092-3}
\bibitem{Zanghellini2004} J. Zanghellini, M. Kitzler, T. Brabec, and A. Scrinzi, Journal of Physics  B {\bf 37}, 763 (2004), \url{https://doi.org/10.1088/0953-4075/37/4/004}
\bibitem{us2024} Niyaz R. Beysengulov, Johannes Pollanen, Øyvind S. Schøyen, Stian D. Bilek, Jonas B. Flaten, Oskar Leinonen, Håkon Emil Kristiansen, Zachary J. Stewart, Jared D. Weidman, Angela K. Wilson, Morten Hjorth-Jensen, Coulomb interaction-driven entanglement of electrons on helium, PRX Quantum, in press and \url{https://arxiv.org/abs/2310.04927}
\bibitem{vqe} Jules Tilly, Hongxiang Chen, Shuxiang Cao, Dario Picozzi, Kanav Setia, Ying Li, Edward Grant, Leonard Wossnig, Ivan Rungger, George H. Booth, Jonathan Tennyson, The Variational Quantum Eigensolver: a review of methods and best practices, Physics Reports {\bf 986}, 1 (2022), \url{https://doi.org/10.1016/j.physrep.2022.08.003}
\end{thebibliography}  


\end{document}




