\documentclass[10pt]{article}
\usepackage{graphicx,amsmath,amssymb,bm}
\usepackage{amsfonts}

\begin{document}

\section*{Master thesis project: Using Machine Learning for analyzing Nuclear Physics Experiements}

\section*{Overview}


Machine Learning plays nowadays a central role
in the analysis of large data sets in order to extract information
about complicated correlations. This information is often difficult to
obtain with traditional methods. For example, there are about one
trillion web pages; more than one hour of video is uploaded to YouTube
every second, amounting to 10 years of content every day; the genomes
of 1000s of people, each of which has a length of $3.8\times 10^9$
base pairs, have been sequenced by various labs and so on. This deluge
of data calls for automated methods of data analysis, which is exactly
what machine learning provides.  Developing activities in these
frontier computational technologies is thus of strategic importance
for our capability to address future science problems.

In this thesis, the aim is to evaluate machine learning methods for
event classification in the Active-Target Time Projection Chamber
(AT-TPC) detector at the National Superconducting Cyclotron Laboratory
(NSCL) at Michigan State University. The AT-TPC detects products from
nuclear physics reactions to study the nuclear structure of rare
isotopes. The detector records many different types of events, but
experimentalists are typically only interested in one reaction
product. We will develop an automated method to single out the desired reaction
product, which may  result in more accurate physics results as well as a
faster analysis process. Single-class, binary, and multi-class
classification methods based on deep neural networks will be developed and tested against earlier and recent experiments at the NSCL

\section*{Progress plan and milestones}
The aims and progress plan of this thesis are as follows:
\begin{itemize}
\item 
\item 
\item 
\item 
\item 
\item 
\end{itemize}
 
The thesis is expected to be handed in May/June 2019 or later depending on extensions.





\end{document}



\title{
























