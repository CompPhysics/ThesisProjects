\documentstyle[a4wide]{article}
\newcommand{\OP}[1]{{\bf\widehat{#1}}}

\newcommand{\be}{\begin{equation}}

\newcommand{\ee}{\end{equation}}

\begin{document}

\pagestyle{plain}

\section*{Thesis title: Monte Carlo studies of atoms}

The aim of this thesis is to study the structure of several light atoms using
variational Monte Carlo techniques, combining results
from Hartree-Fock calculations in order to achieve a as good as possible
variational wave function. The thesis will explore various Monte Carlo
optimalization strategies and computational issues like efficient programming 
and parallelization.

This thesis 
entails the development of variational Monte Carlo (VMC)
program to solve Schr\"odingers equation
and obtain various expectation values of interest, such as the energy
of the ground state and excited states. 
The Slater determinant for the variational wave function is set up using
single-particle wave functions from a Hartree-Fock calculation.
The method is briefly described in the following section.

A progress plan for this thesis project is given at the end.

\subsection*{Variational Monte Carlo}
The variational quantum Monte Carlo (VMC) has been widely applied 
to studies of quantal systems. 
The recipe consists in choosing 
a trial wave function
$\psi_T({\bf R})$ which we assume to be as realistic as possible. 
The variable ${\bf R}$ stands for the spatial coordinates, in total 
$3N$ if we have $N$ particles present. 
The trial wave function serves then as
a mean to define the quantal probability distribution 
\be
   P({\bf R})= \frac{\left|\psi_T({\bf R})\right|^2}{\int \left|\psi_T({\bf R})\right|^2d{\bf R}}.
\ee
The expectation value of the energy $E$
is given by
\be
   \langle E \rangle =
   \frac{\int d{\bf R}\Psi^{\ast}({\bf R})H({\bf R})\Psi({\bf R})}
        {\int d{\bf R}\Psi^{\ast}({\bf R})\Psi({\bf R})},
\ee
where $\Psi$ is the exact eigenfunction. Using our trial
wave function we define a new operator, 
the so-called  
local energy, 
\be
   E_L({\bf R})=\frac{1}{\psi_T({\bf R})}H\psi_T({\bf R}),
   \label{eq:locale1}
\ee
which, together with our trial PDF allows us to rewrite the 
expression for the energy as
\be
  \langle H \rangle =\int P({\bf R})E_L({\bf R}) d{\bf R}.
  \label{eq:vmc1}
\ee
This equation expresses the variational Monte Carlo approach.

The first part of this thesis deals thus with a VMC calculation of light atoms
from He to Ca, where the main goals are to study various approximations
to the trial wave function  and testing optimalization techniques based 
on energy and variance optimalization.

The trial wave function is a combination of a Slater determinant 
and a correlation part. The Slater determinant will be constructed
using both Hydrogen-like single-particle wave functions and single-particle
wave functions based on Hartree-Fock theory.

\section*{Progress plan and milestones}
The aims and progress plan of this thesis are as follows
\begin{itemize}
\item Fall 2013: Develop first a Hartree-Fock code for electrons in atoms.  
This part entails developing a code for computing the Coulomb interaction.
\item Write a code which solves the variational Monte-Carlo (VMC) problem
      for light atoms, typically with up to 20 electrons.  Both closed-core and open shell
atoms will be studied. 
      The Slater determinant used in the VMC calculation includes also
      single-particle wave functions from Hartree-Fock calculations.
 \item Spring 2014 and Fall 2014: The spring semester includes a continuation of the last point, final calculations and thesis finalization.

\end{itemize}
 
\end{document}


