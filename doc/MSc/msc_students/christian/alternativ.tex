\documentclass[twocolumn]{revtex4}
\usepackage{graphicx,amsmath,amssymb,bm}

\newcommand{\OP}[1]{{\bf\widehat{#1}}}

\newcommand{\be}{\begin{equation}}

\newcommand{\ee}{\end{equation}}

\begin{document}
\title{Thesis title: Diffusion Monte Carlo studies of systems of confined fermions}
\author{Christian Fleischer}
\maketitle
\section*{Aims}
The aim of this thesis is to develop a diffusion Monte Carlo (DMC)
code that can be used to study properties of confined fermions. The
code should be flexible enough to deal with both closed-shell and
open-shell systems.  If time allows, the implementation of the
algorithm of Booth {\em et al}, see Ref.~\cite{booth2009}, which
combines DMC calculations and full-scale diagonalization
implementations  would allievate considerably the Monte Carlo sign
problem in DMC studies. 

The applications will be directed towards studies of electron-electron
correlations in quantum dots as function of the density of the
systems. Properties like single-particle and pair densities will be
studied as the density is lowered towards the eventual Wigner
crystallization.  Quantum dot molecules can also be studied and the
first step would be to repeat calculations done with large-scale
diagonalization techniques for two-electron quantum dot molecules.
The project would be based on an earlier Master of Science of thesis developed at the Computational Physics group, see Karl Leikanger \url{https://www.duo.uio.no/handle/10852/37172}. 

\subsection*{General introduction to possible physical systems}


What follows here is a general introduction to systems of confined
electrons in two or three dimensions.  However, although the thesis
will focus on such systems, the codes will be written so that other
systems of trapped fermions or eventually bosons can be
handled. Examples could be neutrons in a harmonic oscillator trap, see
for example Ref.~\cite{bogner2011}, or ions in various traps
\cite{yoram2008}.

Strongly confined electrons offer a wide variety of complex and subtle
phenomena which pose severe challenges to existing many-body methods.
Quantum dots in particular, that is, electrons confined in
semiconducting heterostructures, exhibit, due to their small size,
discrete quantum levels.  The ground states of, for example, circular
dots show similar shell structures and magic numbers as seen for atoms
and nuclei. These structures are particularly evident in measurements
of the change in electrochemical potential due to the addition of one
extra electron, $\Delta_N=\mu(N+1)-\mu(N)$. Here $N$ is the number of
electrons in the quantum dot, and $\mu(N)=E(N)-E(N-1)$ is the
electrochemical potential of the system.  Theoretical predictions of
$\Delta_N$ and the excitation energy spectrum require accurate
calculations of ground-state and of excited-state energies.  Small
confined systems, such as quantum dots (QD), have become very popular
for experimental study. Beyond their possible relevance for
nanotechnology, they are highly tunable in experiments and introduce
level quantization and quantum interference in a controlled way. In a
finite system, there cannot, of course, be a true phase transition,
but a cross-over between weakly and strongly correlated regimes is
still expected. There are several other fundamental differences
between quantum dots and bulk systems: (a)\,Broken translational
symmetry in a QD reduces the ability of the electrons to
delocalize. As a result, a Wigner-type cross-over is expected for a
smaller value of $r_s$ (this is the so-called gas parameter
$r_s=(c_d/a_B)(1/n)^d$, where $n$ is the electron density, $d$ is the
spatial dimension, $a_B$ the effective Bohr radius and $c_d$ a
dimension dependent constant). (b)\,Mesoscopic fluctuations, inherent
in any confined system, lead to a rich interplay with the correlation
effects. These two added features make strong correlation physics
particularly interesting in a QD. As clean 2D bulk samples with large
$r_s$ are regularly fabricated these days in semiconductor
heterostructures, it seems to be just a matter of time before these
systems are patterned into a QD, thus providing an excellent probe of
correlation effects.



The above-mentioned quantum mechanical levels can, in turn, be tuned
by means of, for example, the application of various external fields.
The spins of the electrons in quantum dots provide a natural basis for
representing so-called qubits \cite{divincenzo1996}. The capability to
manipulate and study such states is evidenced by several recent
experiments \cite{exp1,exp2}.  Coupled quantum dots are particularly
interesting since so-called two-qubit quantum gates can be realized by
manipulating the exchange coupling which originates from the repulsive
Coulomb interaction and the underlying Pauli principle.  For such
states, the exchange coupling splits singlet and triplet states, and
depending on the shape of the confining potential and the applied
magnetic field, one can allow for electrical or magnetic control of
the exchange coupling. In particular, several recent experiments and
theoretical investigations have analyzed the role of effective
spin-orbit interactions in quantum dots
\cite{exp5,exp6,pederiva2010,spinorbit} and their influence on the
exchange coupling.

A proper theoretical understanding of the exchange coupling,
correlation energies, ground state energies of quantum dots, the role
of spin-orbit interactions and other properties of quantum dots as
well, requires the development of appropriate and reliable theoretical
few- and many-body methods.  Furthermore, for quantum dots with more
than two electrons and/or specific values of the external fields, this
implies the development of few- and many-body methods where
uncertainty quantifications are provided.  For most methods, this
means providing an estimate of the error due to the truncation made in
the single-particle basis and the truncation made in limiting the
number of possible excitations.  For systems with more than three or
four electrons, {\em ab initio} methods that have been employed in
studies of quantum dots are variational and diffusion Monte Carlo
\cite{harju2005,pederiva2001, pederiva2003}, path integral approaches
\cite{pi1999}, large-scale diagonalization (full configuration
interaction) \cite{Eto97,Maksym90,simen2008,modena2000}, and to a very
limited extent coupled-cluster theory
\cite{shavittbartlett2009,bartlett2007,bartlett2003,indians}.  Exact
diagonalization studies are accurate for a very small number of
electrons, but the number of basis functions needed to obtain a given
accuracy and the computational cost grow very rapidly with electron
number.  In practice they have been used for up to eight
electrons\cite{Eto97,Maksym90,modena2000}, but the accuracy is very
limited for all except $N\le 3$ .  Monte Carlo methods have been
applied up to $N=24$ electrons
\cite{pederiva2001,pederiva2003}. Diffusion Monte Carlo, with
statistical and systematic errors, provide, in principle, exact
benchmark solutions to various properties of quantum dots. However,
the computations start becoming rather time-consuming for larger
systems.  Hartree\cite{Kum90}, restricted Hartree-Fock, spin- and/or
space-unrestricted Hartree-Fock\cite{Fuj96,Mul96,Yan99} and local
spin-density, and current density functional
methods\cite{Kos97,Hir99,finns1,finns2} give results that are
satisfactory for a qualitative understanding of some systematic
properties. However, comparisons with exact results show discrepancies
in the energies that are substantial on the scale of energy
differences.


\subsection*{Specific tasks}

The first task of this thesis is develop a variational (VMC) and diffusion Monte Carlo (DMC)
code that are fully object oriented and parallelized and can tackle
both closed-shell systems and open shell systems. If time allows, the
implementation of an algorithm which combines DMC studies with
full-scale configuration interaction (FCI) methods will allievate the
DMC sign problem. The latter is normally treated using the fixed node
approach. The algorithm of Booth {\em et al} \cite{booth2009} holds
great promise for circumventing the problems inherent to the fixed
node approach. To combine DMC with FCI requires the usage and adaption
of an existing and efficient FCI code written by a fellow Master of
Science student, Frank Olsen.

The second part includes applications of the codes and studies of
quantum dot systems at different densities, computing both
single-electron densities and pairing densities in order to understand
better electron-electron correlations towards the Wigner
crystallization limit, see Goshal {\em et al} for an excellent
reference \cite{ghosal2007}.

The final application part would be to change the confining potential
to that of a double well. The first step is to study a system of two
electrons in a double oscillator like well.  This means that we would
the two-electron 'quantum dot molecule' (QDM) with the two-dimensional
Hamiltonian
\begin{equation}
H = \sum _{i=1}^2\left ( \frac{ ( {- i {\hbar} \nabla_i}
-\frac ec \mathbf{A})^2 }{2 m^{*}} + V_\mathrm{c}({\bf
r}_{i}) \right ) +  \frac {e^{2}}{ \epsilon   r_{12} } \ ,
\label{ham}
\end{equation}
where $V_\mathrm{c}$ is the external confinement potential
taken to be
\begin{equation}
 V_\mathrm{c}({\bf r}) = \frac 12 m^* \omega_0^2 \min \left[
 \sum_j^M ({\bf r} - {\bf L}_j)^2 \right] \ ,
\label{Vc}
\end{equation}
where the coordinates are in two dimensions ${\bf r} = (x,y)$ and the ${\bf
L}_j$'s (${\bf L}_j = (\pm L_x, \pm L_y)$) give the positions of the
minima of the QDM potential, and $M$ is the number of minima. When
${\bf L}_1=(0,0)$ (and $M=1$) we have a single parabolic QD. With
$M=2$ and ${\bf L}_{1,2} = (\pm L_x,0)$ we get a double-dot
potential. One can also study four-minima QDM ($M=4$) with minima at four
possibilities of $(\pm L_x,\pm L_y)$ (see Fig.~\ref{pot}).
%
\begin{figure}
\includegraphics*[width=1.0\columnwidth]{doublewell.pdf}
\caption{Confinement potential of square-symmetric ($L_x=L_y=5$ nm)
four-minima quantum dot molecule.}
\label{pot}
\end{figure}
%
 We study both square-symmetric ($L_x=L_y$) and rectangular-symmetric
($L_x \neq L_y$) four-minima QDMs. The confinement
potential can also be written using the absolute values of $x$ and $y$
coordinates as
\begin{eqnarray}
V_\mathrm{c}(x,y) &=& \frac 12 m^* \omega_0^2 \times \nonumber\\ &&
\left[ r^2 - 2 L_x |x| - 2 L_y |y| + L_x^2 + L_y^2 \right] \ .
\label{Vc_auki}
\end{eqnarray}
For non-zero $L_x$ and $L_y$, 
the perturbation to the parabolic potential comes from the linear
terms of $L_x$ or $L_y$ containing also the absolute value of the
associated coordinate.

\section*{Progress plan and milestones}
The aims and progress plan of this thesis are as follows
\begin{itemize}
\item Spring 2016 and partly fall 2016:  Develop a Hartree-Fock for electrons confined to one single oscillator well.
\item Fall 2016: Use the Hartree-Fock in the construction of a Slater determinant for quantum dots
and write a variational Monte Carlo code which can handle closed shell systems up to approximately
$6-8$ oscillator shells.
\item Fall 2016:
Develop and finalize a fully object-oriented and parallelized diffusion   Monte Carlo (DMC) code
that can tackle closed-shell quantum dot systems.
\item Spring 2017: The final step is to write a code which employs the Full Configuration Interaction Quantum Monte Carlo method of Alavi, Booth {\em et al} in order to study systems of quantum dots. 
\item Spring 2017: The last part deals with a proper write-up of the thesis. 
\end{itemize}
 
The thesis is expected to be handed in May/June 2017.


\begin{thebibliography}{200}
\bibitem{booth2009}G.~H.~Booth, A.~J.~W.~Thom, and A.~Alavi, J.~Chem.~Phys.~{\bf 131}, 054106 (2009). 
\bibitem{bogner2011} S. K. Bogner, R. J. Furnstahl, H. Hergert, M. Kortelainen, P. Maris, M. Stoitsov, and J. P. Vary, Phys.~Rev.~C {\bf 84}, 044306 (2011).
\bibitem{yoram2008} Y. Alhassid, G. F. Bertsch, and L. Fang, Phys.~Rev.~Lett.~{\bf 100}, 230401 (2008).
\bibitem{divincenzo1996} D.~Loss and D.~P.~DiVincenzo, Phys.~Rev.~A {\bf 57}, 120 (1998).
\bibitem{exp1} R.~Hanson, L.~H.~Willems van Beveren, I.~T.~Vink, J.~M.~Elzerman, W.~J.~M. Naber, F.~H.~L.~Koppens, L.~P.~Kouwenhoven, and L.~M.~K.~Vandersypen, \prl {\bf 94}, 196802 (2005).
\bibitem{exp2} H.-A.~Engel, V.~N.~Golovach, D.~Loss, L.~M.~K.~Vandersypen, J.~M.~Elzerman, R.~Hanson, and L.~P.~ Kouwenhoven, \prl {\bf 93}, 106804 (2004). 
\bibitem{exp5} C.~Fasth, A.~Fuhrer, L.~Samuelson, V.~N.~Golovach, and D.~Loss, \prl {\bf 98}, 266801 (2007).
\bibitem{exp6} S.~Roddaro {\em et al.}, \prl {\bf 101}, 18682 (2008).
\bibitem{pederiva2010}A.~Ambrosetti, J.~M.~Escartin, E.~Lipparini, and F.~Pederiva, arXiv:1003.2433.
\bibitem{spinorbit} F.~Baruffa, P.~Stano, and J.~Fabian, \prl {\bf 104}, 126401 (2010).
\bibitem{harju2005} A.~Harju, J.~Low Temp.~Phys.~{\bf 140}, 181 (2005).
\bibitem{bolton} F.~Bolton, Phys.~Rev.~B {\bf 54}, 4780 (1996).
\bibitem{pederiva2001}F.~Pederiva, C.~J.~Umrigar, and E.~Lipparini, \prb {\bf 62}, 8120 (2000).
\bibitem{pederiva2003}L.~Colletti, F.~Pederiva, and  E.~Lipparini,  Eur.~Phys.~J.~B {\bf 27}, 
385 (2002).
\bibitem{pi1999} M.~Harowitz, D.~Shin, and J.~Shumway, J.~Low.~Temp.~Phys.~{\bf 140}, 211 (2005).
\bibitem{Eto97} M. Eto, Jpn. J. Appl. Phys. {\bf 36}, 3924 (1997).
\bibitem{Maksym90} P.~A.~Maksym and T.~Chakraborty, Phys.~Rev.~Lett.~{\bf 65}, 108 (1990);
D.~Pfannkuche, V.~Gudmundsson, and P.~A.~Maksym, Phys. Rev. B {\bf 47}, 2244 (1993);
P.~Hawrylak and D.~Pfannkuche, Phys.~Rev.~Lett.~{\bf 70}, 485 (1993); J.J. Palacios,
L.~Moreno, G.~Chiappe, E.~Louis, and C.~Tejedor, Phys.~Rev.~B {\bf 50}, 5760 (1994);
T.~Ezaki, N.~Mori, and C.~Hamaguchi, Phys.~Rev.~B {\bf 56}, 6428 (1997).
\bibitem{simen2008} S.~Kvaal, Phys.~Rev.~C {\bf 78}, 044330 (2008).
\bibitem{simen2008b} S.~Kvaal, arXiv:0810.2644, unpublished.
\bibitem{modena2000} M.~Rontani, C.~Cavazzoni, 
D.~Belucci, and G.~Goldoni, J.~Chem.~Phys.~{\bf 124}, 124102 (2006).
\bibitem{shavittbartlett2009} I.~Shavitt and R.\ J.\ Bartlett, {\em Many-body Methods in Chemistry and Physics},  
(Cambridge University Press, Cambridge UK, 2009). 
\bibitem{bartlett2007} R.\ J.\ Bartlett and M.\ Musia{\l}, \rmp {\bf 79}, 291 (2007).
\bibitem{bartlett2003} T.~M.~Henderson, K.~Runge, and R.~J.~Bartlett, Chem.~Phys.~Lett.~{\bf 337}, 138 (2001); \prb {\bf 67}, 045320 (2003).
\bibitem{indians} I.~Heidari, S.~Pal, B.~S.~Pujari, and D.~G.~Kanhere, J.~Chem.~Phys.~{\bf 127}, 
114708 (2007).
\bibitem{Kum90} A.~Kumar, S.~E.~Laux, and F.~Stern, Phys.~Rev.~B {\bf 42}, 5166 (1990).
\bibitem{Fuj96} M.~Fujito, A.~Natori, and H.~Yasunaga, Phys.~Rev.~B {\bf 53}, 9952 (1996).
\bibitem{Mul96} H.~M.~Muller and S.~Koonin, Phys.~Rev.~B {\bf 54}, 14532 (1996).
\bibitem{Yan99} C.~Yannouleas and U.~Landman, Phys. Rev. Lett. {\bf 82}, 5325 (1999).
\bibitem{Kos97} M. Koskinen, M. Manninen, and S.M. Reimann, Phys. Rev. Lett. {\bf 79}, 1389 (1997).
\bibitem{Hir99} K. Hirose and N. S. Wingreen, Phys. Rev. B {\bf 59}, 4604 (1999).
\bibitem{finns1} P.~Gori-Giorgi, M.~Seidl, and G.~Vignale, \prl {\bf 103}, 166402 (2009).
\bibitem{finns2} E.~R\"as\"anen, S.~Pittalis, J.~G.~Vilhena, M.~A.~L.~Marques, Int.~J.~Quantum Chem.~{\bf 110}, 2308 (2010). 
\bibitem{ghosal2007} A.~Ghosal, A.~D.~G\"{u}\c{c}l\"{u}, C.~J.~Umrigar, D.~Ullmo, and H.~U.~Baranger, Phys.~Rev.~B {\bf 76}, 085341 (2007).

\end{thebibliography}



\end{document}


